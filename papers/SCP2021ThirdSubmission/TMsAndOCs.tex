\section{Background on Reference and Object Capabilities}
\label{s:TMsAndOCs}
Reasoning about imperative OO programs is a non-trivial task,
made particularly difficult by mutation, aliasing, dynamic dispatch, I/O, and exceptions. There are many ways to perform such reasoning;
instead of using automated theorem proving, 
it is becoming more popular to verify aliasing and immutability properties using a type system.
For example, three languages: L42~\cite{ServettoZucca15,ServettoEtAl13a,JOT:issue_2011_01/article1,GianniniEtAl16}, Pony~\cite{clebsch2015deny,clebsch2017orca}, and the language of Gordon \etal~\cite{GordonEtAl12} use \emph{reference capabilities}\footnote{reference capabilities are called \emph{Type Modifiers} in former works on L42.} and \emph{object capabilities} to statically ensure  deterministic parallelism and the absence of data races.
While studying those languages, we discovered an elegant way to enforce invariants: we use capabilities to restrict how/when the result of invariant methods changes; this is done by restricting I/O, and how mutation through aliases can affect the state seen by invariants.

That is, our work shows that reference and object capabilities are very usueful also outside of the context of safe parallelism.

% We use these restrictions to reason as to when an object’s invariant could have been violated, and when such object can next be used, we then inject a runtime check between these two points. See \autoref{s:protocol} for the exact details of our invariant protocol.
%We aim to leverage these existing TM guarantees with minimal modification and additional concepts; in particular we do not want to add new syntax, 
%when the same expressive power can be expressed using (verbose) programming patterns;
%see \autoref{s:patterns}.
%This approach allows our sound invariant protocol to only rely on a few simple and easy to understand rules.

% here we use the type system to restrict, but not prevent such behaviour in order to be able to soundly enforce invariants with runtime verification (RV).
% [dynamic class loading],

\subheading{Reference Capabilities}
Reference capabilities, as used in this paper, are a type system feature that allows reasoning about aliasing and mutation. A more recent design for them has emerged that radically improves their usability;
three different research languages are being independently developed relying on this new design: the language of Gordon \etal, Pony, and L42.
These projects are quite large: several million lines of code are written in Gordon \etal's language and are used by a large private Microsoft project; Pony and L42 have large libraries and are active open source projects. In particular the reference capabilities of these languages are used to provide automatic and correct parallelism~\cite{GordonEtAl12,clebsch2015deny,clebsch2017orca,ServettoEtAl13a}.

Reference capabilities
 are a well known mechanism~\cite{TschantzErnst05,BirkaErnst04,OstlundEtAl08,clebsch2015deny,GianniniEtAl16,GordonEtAl12} that 
 allow statically reasoning about the mutability and aliasing properties of objects. Here we refer to the interpretation of~\cite{GordonEtAl12}, that introduced the concept of recovery/promotion. This concept is the basis for L42, Pony, and Gordon \etal's type systems~\cite{GordonEtAl12,ServettoEtAl13a,ServettoZucca15,clebsch2015deny,clebsch2017orca}. With slightly different names and semantics, those languages all support the following reference capabilities for object references: %(i.e. expressions and variables):
\begin{itemize}

%\noindent\rule{1.2ex}{1.2ex}
\item Mutable (\Q@mut@): the referenced object can be mutated and shared/aliased without restriction; as in most imperative languages without reference capabilities.

%\noindent\rule{1.2ex}{1.2ex}
\item Immutable (\Q@imm@): the referenced object cannot mutate, not even through other aliases. An object with any \Q!imm! aliases is an \emph{immutable object}.
Any other object is a \emph{mutable object}.
All objects are born mutable and may later become immutable.
Thus, an object can be classified as \emph{mutable} even if it has no fields that can be updated or mutated.


%\noindent\rule{1.2ex}{1.2ex}
\item Readonly (\Q@read@): the referenced object cannot be mutated by such references, but there may also be mutable aliases to the same object, thus mutation can be observed. Readonly references can refer to both mutable and immutable objects, as \Q!read! types are supertypes of both their \Q!imm! and \Q!mut! variants.
%\noindent\rule{1.2ex}{1.2ex}
\item Encapsulated (\Q@capsule@):
every mutable object in the reachable object graph of a capsule reference (including itself) is only reachable through that reference. 
Immutable objects in the reachable object graph of a capsule reference are not constrained, and can be freely referred to without passing through that reference. 
\end{itemize}
%This means that if a capsule reference $r$ is usable in the same expression as a reference $r'$, then either $r'$ does not refer to an object reachable from $r$, or $r'$ refers to an immutable object. Note that it is safe to use a capsule reference as either mutable or immutable, since there could have been no other aliases to it.
%\end{itemize}
\noindent That is, there are only two kinds of objects: mutable and immutable, but there are more kinds of reference capabilities.
In L42 only \Q@mut@ and \Q@imm@ references can be saved on the heap; that is, \Q@capsule@ and \Q@read@ references only exists on the stack.
%the 'read' field in 42 can indeed be understood as a 'mut' field of an object that is born 'read'

Reference capabilities are different to field or variable qualifiers like Java's \Q@final@: reference capabilities apply to references, whereas \Q@final@ applies to fields themselves. Unlike a variable/field of a \Q@read@ type, a \Q@final@ variable/field cannot be reassigned, it always refers to the same object, however the variable/field can still be used to mutate the referenced object.
On the other hand, an object cannot be mutated through a \Q@read@ reference, however a \Q@read@ variable can still be reassigned.\footnote{In C, this is similar to the difference between \Q@A* const@ (like \Q@final@) and \Q@const A*@ (like \Q@read@), where \Q@const A* const@ is like \Q@final read@.}

Reference capabilities are applied to all types. This includes types in the receiver and parameters of methods.
A \Q!mut! method is a method where \Q@this@ is typed \Q!mut!;
An \Q!imm! method is a method where \Q@this@ is typed \Q!imm!, and so on for all the other reference capabilities.

%\end{itemize}

Consider the following  example usage of \Q@mut@, \Q@imm@, and \Q@read@, where we can observe a change in \Q@rp@ caused by a mutation inside \Q@mp@.
\begin{lstlisting}
mut Point mp = new Point(1, 2);
mp.x = 3; // ok
imm Point ip = new Point(1, 2);
$\Comment{}$ip.x = 3; // type error
read Point rp = mp;
$\Comment{}$rp.x = 3; // type error
mp.x = 5; // ok, now we can observe rp.x == 5
ip = new Point(3, 5); // ok, ip is not final
\end{lstlisting} 

Reference capabilities influence the access to the whole reachable object graph; not just the referenced object itself, as in the full/deep interpretation of type modifiers~\cite{ZibinEtAl10,Potanin2013}:
\begin{itemize}
%  \item No casting or promotion from \Q@read@ to \Q@mut@ is allowed.

  \item 
A \Q!mut! field accessed from a \Q@read@ reference produces a \Q@read@ reference; thus a \Q!read! reference cannot be used to mutate the reachable object graph of the referenced object.

  \item 
Any field accessed from an \Q!imm! reference produces an \Q!imm! reference; thus all the objects in the reachable object graph of an immutable object are also immutable.
%  \item no \emph{down}-casting is allowed between different type modifiers.
%  \item promotion, is a type-system feature allowing implicit and safe casting from \Q@read@ and \Q@mut@ to \Q@imm@.
\end{itemize}
A common misconception of this line of work is that a \Q!mut! field will always refer to a mutable object.
Classes declare reference capabilities for their methods and field types, but  what kinds of object is stored in a field also depends on the kind of the object: a \Q!mut! field of a mutable object will contain a mutable object; but a \Q!mut! field of an immutable object will contain an immutable object.
This is different with respect to work prior to Gordon \etal's~\cite{GordonEtAl12}, where the
declaration fully determines what values can be stored. In those other approaches, any contextual information must be explicitly passed through the type system, for example, with a generic reference capability parameter.


Another common misconception is the belief that \Q!capsule! fields and \Q!capsule! local variables always hold \Q!capsule! references, i.e. the referenced object cannot be reached except via that field/variable.
How \Q!capsule! local variables are handled differs widely in the literature:

%\noindent
In L42, a \Q!capsule! local variable always holds a \Q!capsule! reference: this is ensured by allowing them to be read only once (similar to linear and affine types~\cite{boyland2001alias}). 
Pony and Gordon \etal follow a more complicated approach: \Q!capsule! variables can be accessed multiple times, however in those cases the result will not be a \Q!capsule! reference but another kind of reference, that can be promoted to \Q!capsule!, but only under certain conditions. Pony and Gordon also provide destructive reads, where the variable's old value is returned as \Q!capsule!.
%Later on, we discuss \Q!capsule! fields, which behave differently to \Q@capsule@ local variables.

Like \Q!capsule! variables, how \Q!capsule! fields are handled differs widely in the literature, however they must always be initialised and updated with \Q!capsule! references. In order for access to a \Q!capsule! field to safely produce a \Q!capsule! reference, Gordon \etal only allows them to be read destructively (i.e. by replacing the field's old value with a new one, such as \Q!null!). 
%In contrast \Q!capsule! field behave differently in Pony, where 
%In contrast, \Q!capsule! fields in Pony need not be a unique reference to an object's reachable object graph. In pon
In contrast, 
Pony does not guarantee that \Q@capsule@ fields contain a \Q!capsule! reference at all times, as it also provides non-destructive reads.

%Pony does not guarantee that \Q@capsule@ fields %contain a \Q!capsule! reference, as Pony provides non-%destructive reads.
%Pony's \Q!capsule! fields are still useful for safe %parallelism, as destructive reads of a \Q!capsule! %field return a \Q!capsule! reference (which can then %be sent to other actors), however the reachable object graph of a \Q!%capsule! field can be mutated by the same actor, even %within methods of unrelated objects.
%L42 supports another variation of \Q!capsule! fields %similar to Pony's, but does not support destructive %reads~\cite{ServettoEtAl13a,GIANNINI2019145}.
%Pony's \Q!capsule! fields
% are useful for safe parallelism, but not invariant checking; while Gordon \etal's cannot be read non-destructively.

The formal model of L42~\cite{GIANNINI2019145} does not
contains \Q@capsule@ fields.
The L42 concrete language interprets the syntax for capsule fields as private \Q@mut@ fields with some extra restrictions, including being initialised and updated only with \Q@capsule@ references.
Those \emph{encapsulated} fields allows to model various forms of ownership and parallelism.
\footnote{It may seem surprising that 
those weaker forms of encapsulation are still sufficient to ensure safe parallelism.
The detailed way L42 parallelism works is unrelated to the presented work.
Please see the tutorial on \url{Forty2.is} (specifically, section 5 and 6) for more information on parallelism in L42.
}
In \autoref{s:protocol} we present a novel kind of ``\Q!rep!'' field.
These, like \Q!capsule! fields, can only be initialised/updated with \Q!capsule! references,
however alias to it can be created in restricted ways.
Unlike \Q!capsule! fields, which are usually designed for safe parallelism,
these \Q!rep! fields are specifically useful for invariant checking;
 we added support for them to L42, and believe they could be easily added to Pony and Gordon \etal's language.
%We repeat here for more clarity: 
%a capsule field is not the same concept of a capsule reference. 
%However, different languages have different behaviour when a capsule field is accessed, and a capsule reference is not always produced.

%None of those forms of \Q!capsule! fields are suitable for invariant checking, since they are designed to prevent  data-races. In \autoref{s:protocol} we present a novel kind of \Q!capsule! field, designed to prevent representation exposure. We added this new kind of \Q!capsule! field to L42, and it could be easily added to Gordon \etal and Pony.

\subheading{Promotion and Recovery}
\noindent Many different techniques and type systems handle the reference capabilities above~\cite{ZibinEtAl10,ClarkeWrigstad03,HallerOdersky10,GordonEtAl12,ServettoZucca15}.
The main progress in the last few years is with the flexibility of such type systems:
 where the programmer should use \Q@imm@ when  representing immutable data
and \Q@mut@ nearly everywhere else. The system will be able to transparently promote/recover~\cite{GordonEtAl12,clebsch2015deny,ServettoZucca15} the reference capability, adapting them to their use context.
To see a glimpse of this flexibility, consider the following:
%//the same expression can create mut, imm or capsule
%\saveSpace
\begin{lstlisting}
    mut Circle mc = new Circle(new Point(0, 0), 7);
capsule Circle cc = new Circle(new Point(0, 0), 7);
    imm Circle ic = new Circle(new Point(0, 0), 7);
\end{lstlisting}
Here \Q@mc@, \Q@cc@, and \Q@ic@ are all syntactically initialised with the same exact expression. All
\Q@new@ expressions return a \Q@mut@~\cite{clebsch2015deny,GIANNINI2019145}, so \Q@mc@ is well typed. The declarations of \Q@cc@ and \Q@ic@ are also well typed, since 
%\footnote{Capsules must encapsulate their entire reachable object graph, thus a \Q@new@ expression
%can not directly return \Q@capsule@ in the case of objects with \Q@mut@ fields.}
any expression (not just \Q@new@ expressions) 
of a \Q@mut@ type that has no \Q@mut@ or \Q@read@ free 
variables can be implicitly promoted to \Q@capsule@ or \Q@imm@.
This requires the absence of \Q!read! and \Q!mut! \emph{global/static} variables, as in L42, Pony, and Gordon \etal's language.
L42 also allows such expression to use \Q!read! free variables as well as \Q!mut! variables as if they were \Q!read!. For this to be sound, L42 does not allow \Q!read! fields.
%Additionally, a \Q@capsule@ can be implicitly converted to \Q@imm@, thus \Q@ic@ is also ok.
%\footnote{
%This requires some restrictions on \Q@read@ fields not discussed in detail for lack of space.
%}

This is the main improvement on the flexibility of reference capabilities in recent literature~\cite{ServettoEtAl13a,ServettoZucca15,GordonEtAl12,clebsch2015deny,clebsch2017orca}.
From a usability perspective, this improvement means that
these reference capabilities are opt-in: a programmer can write many classes
simply using \Q@mut@ types and be free to have rampant aliasing. 
Then, at a later stage, another programmer may still 
be able to encapsulate instances of those data structures into an \Q@imm@ or \Q@capsule@ reference.

For example, immagine a program where most objects belong to classes designed in a simple minded imperative style: without worrying about ownership, aliasing and encapsulation and with most methods requiring mutation.
Thanks to the flexibility discussed above, those objects can still take advantage of our invariant protocol; we just need to apply our Box pattern around those.

\subheading{Exceptions}\label{s:exceptions}
In most languages exceptions may be thrown at any point. Combined with mutation this complicates reasoning about the state of programs after exceptions are caught: if an exception was thrown while mutating an object, what state is that object in? Does its invariant hold?
The concept of \emph{strong exception safety}~\cite{Abrahams2000,JOT:issue_2011_01/article1} simplifies reasoning:
if a \Q@try@--\Q@catch@ block caught an exception, the state visible before execution of the \Q@try@ block is unchanged, and the exception object does not expose any object that was being mutated; this prevents exposing objects whose invariant was left broken in the middle of mutations.
%\LINE
%\noindent{\textit{Exceptions:}}
% M\#, L42 and Pony rely on SES for all unchecked exceptions to ensure safe and transparent parallelism,
% They wish to ensure the code behave as if the execution was fully sequential.
% Exceptions create additional difficulties in such context: if two operations are running in parallel in
% a fork-join, and the first one produces an exception, it should be safe to cancel the other operation and
% to propagate the exception outwards. The system need to guarantee
% the progress the second operation accumulated is not observable.
% Pony avoids this problem simply by not supporting exceptions;
% while
%M\# and L42 will parallelize only expressions that do not leak checked exceptions,
%and they enforce Strong Exception Safety(SES)~\cite{Abrahams2000} for unchecked exceptions.
%Other authors have identified the concept of SES as
% a general issue when reasoning about objects state after catching an exception.
% while we need it to soundly capture invariant failures.

L42 enforces strong exception safety for unchecked exceptions using reference capabilities\footnote{%
This is needed to support safe parallelism. Pony takes a drastic approach and not support exceptions. 
We are not aware of how Gordon \etal handles exceptions, however to have sound unobservable parallelism it must have some restrictions.%
%We do not know how M\# conciliate deterministic parallelism and unchecked exceptions, we suspect some variation of SES must be in place.
}
in the following way:\footnote{%
Formal proof that these restriction are sufficient
is in the work of Lagorio and Servetto~\cite{JOT:issue_2011_01/article1}.
}
\begin{itemize}
\item Only \Q@imm@ objects may be thrown as unchecked exceptions.
\item Code inside a \Q@try@ block that captures unchecked exceptions is typed as 
if all variables declared outside of the block are \Q@final@ and all those of a \Q@mut@ type were \Q@read@.
With such restrictions those \Q@try-catch@es can not rely on side effects
to produce a result. In L42 \Q@try-catch@ is an expression, so the \Q@try@ can produce a result without the need of updating local variables.
In a language where the \Q@try-catch@ is a statement, the \Q@try@ can still produce a result; for example using the \Q@return@ keyword.

\end{itemize} 
%Of course this has the effect that even if no-exception is thrown, no mutation could have occured, which is an even stronger property than SES, other work is more flexible~\cite{?}, at the cost of more complicated typing rules.
%With SES we can soundly capture invariant-failures as an exception, since any mutation that caused the invariant failure cannot be observed. However, we also need to prevent a broke-object from being reachable from the exception object; since the only way a broken-object can be seen is within the \Q@read@ \Q@invariant()@  method, it follows that if the exception-object contains no \Q@read@ references in its reachable object graph it cannot leak a broken object. Preventing this in the-typsystem is non-trivial, so instead we simply require that:
This strategy does not restrict when exceptions can be \emph{thrown}, but only restricts when unchecked exceptions can be \emph{caught}.
Strong exception safety allows us to throw invariant failures as unchecked exceptions: if an object's reachable object graph was mutated into a broken state within a \Q@try@, when the invariant failure is caught, the mutated object will be unreachable/garbage-collectable. This works since strong exception safety guarantees that no object mutated within a \Q@try@ is visible when it catches an unchecked exception.%
\footnote{Transactions are another way of enforcing strong exception safety, but they require specialised and costly run time support.}

Similarly to Java, L42 distinguishes between checked and unchecked exceptions,
and \Q@try-catch@es over checked exceptions impose no limits on object mutation during the \Q@try@.
That is, strong exception safety is only enforced for unchecked exceptions.
%Similarly to what happens in Java, the L42 type system simply restricts where checked exceptions can be thrown, but does not enforce strong exception safety for checked exceptions. 

%For the purposes of soundly catching invariant failures, it would be sufficient to enforce SES only when capturing exceptions caused by such failures.

%The ability to catch and recover from such failures is extremely useful as it allows the program to take corrective action.(DUPLICATED)

% We think this restriction is acceptable for run time verification, other works are much more restrictive,



%The above rules need only be enforced for catch blocks that could catch invariant-failures (including exceptions thrown within execution of \Q@invariant()@) itself;
%, and since \Q@invariant()@ declares no checked exceptions, this includes all exceptions throw-able by it.

% TMs are very useful in restricting the scope of mutation. 
% Any expression that does not use any \Q@mut@ 
% variable declared outside of such expression does not modify objects visible outside.
% With this observation in mind, we can use Reference capabilities to enforce SES in the following way:\footnote{
% 

% \begin{itemize}
% \item all thrown exceptions are immutable objects,
% \item 
% \end{itemize}

% For the aim of enforcing invariants, we could relax SES to hold only when capturing exceptions caused by invariant failures; but we are building on approaches that enforce SES on all unchecked exceptions .


% Intro to object capabilities

\subheading{Object Capabilities}\label{s:OCs}
Object capabilities, which L42, Pony, and Gordon \etal's work have, are a widely used~\cite{miller2003capability,noble2016abstract,karger1988improving,RobustComposition} programming technique where access rights to resources are encoded as references to objects. When this style
is respected, 
code 
unable to reach a reference
%that does not possess an alias
 to such an object cannot use its associated resource.
%Object capabilities are programming style used to control and restrict use of operations such as access to external resources
Here, as in Gordon \etal's work, we enforce the object capabilities pattern with reference capabilities in order to reason about determinism and I/O. To properly enforce this, the object capabilities style needs to be respected while implementing the primitives of the standard library, and when performing foreign function calls that could be non-deterministic, such as operations that read from files or generate random numbers. Such operations would not be provided by static methods, but instead by instance methods of classes whose instantiation is kept under control
by carefully designing their implementation. 
% \noindent\REVComm{\textit{Object Capabilities:}}{2}{Citations here?}
% While type modifiers are statically verified properties of references, object capabilities are run time characteristics of specific objects.

% Conceptually, an object capability is a communicable, unforgeable token of authority, a key to access special functionality: only certain objects with `special' powers can do `special' actions, and those objects are obtained in a controlled way. We call such objects `capability objects'.


% Their main use case is to allow for fine grained control over what sections of code are allowed to do. 

\lstset{language=Java}
 For example, in Java, \Q@System.in@%
 \lstset{language=FortyThree}
 is a \emph{capability object} that provides access to the standard input resource. However, since it is globally accessible it completely prevents reasoning about determinism. 
 % In contrast, in the object capability style, one would not have-global variables but have the main por
% a capability object (it has the capability to read input); however it is globally accessible: thus any code could use it, preventing reasoning about determinism.
In contrast, if Java were to respect the object capability style, the \Q@main@ method could take a \Q@System@ parameter, as in
\lstset{language=Java}
\begin{lstlisting}
public static void main(System s){... s.in().read() ...}
\end{lstlisting}
\lstset{language=FortyThree}
Calling methods on that \Q@System@ instance would be the only way to perform I/O;
moreover, the only \Q@System@ instance would be the one created by the runtime system before calling \Q@main(s)@. % would have no usable constructor, and all its I/O methods would require a mutable (\Q@mut@) receiver.
% Other non deterministic operations would also work this.
%may just take a \Q@mut System@ object as a parameter.
% could also work this way.
This design has been explored by Joe-E~\cite{finifter2008verifiable}.


Object capabilities are typically not part of the type system nor do they require runtime checks or special support beyond that provided by a memory safe language. 

However, L42 has no predefined standard library, but many can be defined by the community.
Thus, the only way to perform I/O operations is via foreign function calls.
Since enforcing the object capabilities pattern can not be done via a unique standard library, the type system of L42 directly enforces the object capabilities pattern as follows:


%To reason about determinism, L42 connects Reference capabilities with the object capabilities style as follows: % style by requiring:
\begin{itemize}
\item Foreign methods (which have not been whitelisted as deterministic) and methods whose names start with \texttt{\#\$} are \emph{capability operations}.
\item Classes containing capability operations are \emph{capability classes}.
\item Constructors of capability classes are also \emph{capability operations}.
\item Capability operations can only be called by other capability operations or \Q@mut@/\Q@capsule@ methods of capability classes.
\item In L42 there is no \Q@main@ method, rather it has several \emph{main expressions}; such expressions can also call capability operations, thus they can instantiate object capabilities and pass them around to the rest of the program.
% \item Any method that uses non deterministic primitive operations (such as native calls or access to global variables\footnote{ Even just allowing unrestricted access to \Q@imm@ global variables would prevent reasoning over determinism due to the possibility of global variable updates; however constant/final globals of an \Q@imm@ type would not cause such problems.
% }) must be an instance method requiring a \Q@mut@ receiver.
% Classes having such methods are \emph{capability classes}, and their instances are \emph{capability objects}.
% \item A capability object can only be created inside a \Q@mut@ method of a capability class; or
% by the runtime system, and passed to the main method.

% \item If the language has global variables, they should only be 
%\item There are no global variables.\footnote{}
\end{itemize}

%\noindent L42 capability operations are mostly used internally by capability classes, whereas user code would call normal methods on already existing object capabilities.

%For the purposes of invariant checking, we only care about the effects that methods could have on the running program and heap. As such, \emph{output} methods (such as a \Q@print@ method) could be white listed as `deterministic', provided they do not affect execution, such as by non-deterministically throwing I/O errors.

