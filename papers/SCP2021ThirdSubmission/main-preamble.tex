\usepackage{verbatim}
%\addbibresource{main.bib}
\usepackage{wrapfig}
\usepackage{etoolbox}
\usepackage{xparse}
%\usepackage[stable]{footmisc}
%
\makeatletter
% Make code font lighter?
\DeclareFontSeriesDefault[tt]{md}{l}
%\DeclareFontSeriesDefault[tt]{bf}{b}
\renewcommand{\ttdefault}{lmtt}

\DeclareOldFontCommand{\rm}{\normalfont\rmfamily}{\mathrm}
\DeclareOldFontCommand{\sf}{\normalfont\sffamily}{\mathsf}
\DeclareOldFontCommand{\tt}{\normalfont\ttfamily}{\mathtt}
\DeclareOldFontCommand{\bf}{\normalfont\bfseries}{\mathbf}
\DeclareOldFontCommand{\it}{\normalfont\itshape}{\mathit}
\DeclareOldFontCommand{\sl}{\normalfont\slshape}{\@nomath\sl}
\DeclareOldFontCommand{\sc}{\normalfont\scshape}{\@nomath\sc}
\makeatother

\usepackage{mathpartir}
\usepackage{amsmath}
\usepackage{amssymb}

\ifdefined\proof%
\else%
\usepackage{amsthm}
\usepackage{thm-restate}
%\theoremstyle{plain}
\makeatletter
\newtheoremstyle{romansans} % Based on plain, with upright body and sans serif note
	{\topsep}{\topsep}  % Space above/below (plain)
	{\upshape}{}        % Body font & indent (changed to upright)
	{\bfseries}{.}      % Theorem head font  (plain) and punctuation (plain)
	{.5em}{\thmhead@plain{#1}{#2}{\textsf{#3}}} % Space after theorem head (plain) and head formatting (custom note shape)
\theoremstyle{romansans}
\newcommand{\providecounter}[1]{%
  \ifcsname c@#1\endcsname % do nothing, counter allready defined
  \else
    \newcounter{#1}%
  \fi
}
\makeatother


\newcounter{definition}
\newtheorem{Definition}[definition]{Definition}
\fi%
\usepackage{xspace}
\newcounter{requirement}
\newtheorem{Requirement}[requirement]{Requirement}
\newcounter{llemma} % For some reason 'lemma' dosn't work, so I've called it llemma
\newtheorem{Lemma}[llemma]{Lemma}
\newcounter{corollary}
\newtheorem{Corollary}[corollary]{Corollary}

\let\origMapsTo=\mapsto % save it as Def redefines it
\usepackage{listings}
\usepackage{xcolor}
\usepackage{letltxmacro}
\usepackage{mathtools}
\usepackage{mathpartir}
%\usepackage{stix}

\definecolor{darkRed}{RGB}{100,0,10}
\definecolor{darkBlue}{RGB}{10,0,100}
\newcommand*{\ttfamilywithbold}{\fontfamily{pcr}\selectfont}
%\newcommand*{\ttfamilywithbold}{\ttfamily}

%found on http://tex.stackexchange.com/questions/4198/emphasize-word-beginning-with-uppercase-letters-in-code-with-lstlisting-package
%\lstset{language=FortyTwo,identifierstyle=\idstyle}
%
\makeatletter
\newcommand*\idstyle{%
        \expandafter\id@style\the\lst@token\relax
}
\def\id@style#1#2\relax{%
        \ifcat#1\relax\else
                \ifnum`#1=\uccode`#1%
                        \ttfamilywithbold\bfseries
                \fi
        \fi
}
\makeatother

\lstset{language=Java,
  basicstyle=\upshape\ttfamily\footnotesize,%\small,%\scriptsize,
  keywordstyle=\upshape\bfseries\color{darkRed},
  showstringspaces=false,
  mathescape=true,
  xleftmargin=0pt,
  xrightmargin=0pt,
  breaklines=false,
  breakatwhitespace=false,
  breakautoindent=false,
 identifierstyle=\idstyle,
 morekeywords={method,Use,This,constructor,as,into,rename},
 deletekeywords={double},
 literate=
  {\%}{{\mbox{\textbf{\%}}}}1
  {~} {$\sim$}1
%  {<}{$\langle$}1
%  {>}{$\rangle$}1
}

\newcommand*{\SavedLstInline}{}
\LetLtxMacro\SavedLstInline\lstinline
\DeclareRobustCommand*{\lstinline}{%
	\ifmmode
	\let\SavedBGroup\bgroup
	\def\bgroup{%
		\let\bgroup\SavedBGroup
		\hbox\bgroup
	}%
	\fi
	\SavedLstInline
}

\newcommand\saveSpace{\vspace{-2pt}}

\newcommand\Rotated[1]{\begin{turn}{90}\begin{minipage}{12em}#1\end{minipage}\end{turn}}

\newcommand{\Q}{\lstinline}
\newenvironment{bnf}{$\begin{aligned}}{\end{aligned}$}
\newcommand{\production}[3]{\textit{#1}&\Coloneqq\textit{#2}&\text{#3}}
\newcommand{\prodNextLine}[2]{&\quad\quad\textit{#1}&\text{#2}}
\newenvironment{defye}{\\\indent$\begin{aligned}}{\end{aligned}$\\}
\newcommand{\defy}[2]{\!\!\!\!\!\!&&#1&\coloneqq#2\\}
%\newcommand{\defyc}[1]{&\phantom{\coloneqq}\ \ #1\\}
\newcommand{\defyc}[1]{\!\!\!\!\!\!\rlap{\quad \quad #1}&&\\}
\newcommand{\defya}[2]{#1&\!\!\!\!\!\!&\coloneqq#2\\}

%\newcommand{\prodFull}[3]{#1&::=&\mbox{#2}&\mbox{#3}}
\newcommand{\prodInline}[2]{#1\Coloneqq#2}
\newcommand{\terminal}[1]{\ensuremath{$\texttt{#1}$}}
%\newcommand{\metavariable}[1]{\ensuremath{\mathit{#1}}}

\newcommand{\Rulename}[1]{{\textsc{#1}}}
\newcommand{\ctx}[1]{\ensuremath{\mathcal{E}_#1}\!}
\newcommand{\libi}[2]{\Q@\{@\Q!interface!\ #1\Q{;} #2\Q@\}@}
\newcommand{\lib}[3]{\Q!interface!\ensuremath{?}\ \libc{#1}{#2}{#3}}
\newcommand{\libc}[3]{\,\Q@\{@\!#1\Q{;}\ #2 \Q{;}\ #3\Q@\}@\!\!}

\newcommand{\rp}[1]{\Q{(}\!#1\Q{)}}
\newcommand{\eq}[1]{\,\Q{=}#1}
\newcommand{\red}[3]{#1\,\Q{<}#2\eq#3\,\Q{>}}
\newcommand{\summ}[2]{#1\ \Q{<+}\ #2}
\newcommand{\from}[2]{#1\ensuremath{[}#2\ensuremath{]}}
\newcommand{\mmid}{{\ensuremath{{\mid}}}\!}
\newcommand{\hole}{\ensuremath{\square}}
\newcommand{\s}[1]{\ensuremath{\mathit{#1s}}}
\makeatletter
\newcommand{\This}[1]{\Q!This!#1\nextpath}
\newcommand{\Cs}[1]{#1\nextpath}
\newcommand{\nextpath}{\@ifnextchar\bgroup{\gobblenextpath}{}}
\newcommand{\gobblenextpath}[1]{\Q!.!#1\@ifnextchar\bgroup{\gobblenextpath}{}}
\makeatother



%--------------------------
\newcommand{\mynotes}[3]{{\color{#2} {\sc #1}: #3}}
\newcommand\isaac[1]{\mynotes{Isaac}{blue}{#1}}

\newcommand\IO[1]{\color{blue}{#1}}
\newcommand\marco[1]{\mynotes{Marco}{green}{#1}}


\let\mapsto=\origMapsTo % undo changes
%\renewcommand{\mapsto}{\mathrel{\origMapsTo\!}}
\renewcommand\scaleKw{1} % Turn off the horrible keyword squashing in the formalism

% The following code defines a macro, \markLine that puts it’s argument in the margins on either side of the next line
% it can be called multiple times for the same line, causing the arguments to placed next to eachother
\begin{comment} 
\usepackage{lineno}
\linenumbers
\renewcommand\makeLineNumber{}
\newbox\LeftMarkers \newbox\RightMarkers
\newcommand{\markLine}[2]{%
\setbox\LeftMarkers\hbox{#1\unhbox\LeftMarkers}%
\setbox\RightMarkers\hbox{\unhbox\RightMarkers#1}}

\newcommand{\markLineD}[1]{\markLineD{#1}{#1}} % just a shortcut
\renewcommand{\makeLineNumber}{%
\ifvoid\LeftMarkers%
\else \hss\unhcopy\LeftMarkers\ \rlap{\hskip\textwidth\ \unhbox\RightMarkers}%
\fi}
\end{comment}

%\newcommand{\saveSpace}{\vspace{-3px}}
%\newcommand{\loseSpace}{\vspace{1ex}}
\newcommand{\etal}{\emph{et~al.}\xspace}
\newcommand{\REV}[3]{%
	\NoteColour{red}{#1\NoteText{\footnote{%
				\textcolor{red}{\textbf{REV#2{:} #3}}}}}}
			
\newcommand{\REVm}[1]{\NoteColour{red}{#1\NoteText{\footnotemark}}}
\newcommand{\REVt}[2]{\footnotetext{\textcolor{red}{\textbf{REV#1{:} #2}}}}
			
\newcommand{\subheading}[1]{%
	\vspace{1ex}%
	\noindent\textsf{\textbf{#1\\\noindent}}
}



%Make the "." in the syntax bigger so it stands out more, but give it less spacing
\newcommand{\D}{\scalebox{1.2}{\kern-0.1em\singleDot\kern-0.1em}}

% Reduce spacing arround the '.' so it looks nicer in math
\newcommand{\MD}{\kern-0.3pt\text{.}\kern-0.9pt}
% Hack (see https://tex.stackexchange.com/questions/299798/make-characters-active-via-macro-in-math-mode)
% this reduces the spacing around '.' in math mode
\newcommand{\defActiveMathChar}[2]{%
	\begingroup\lccode`~=`#1\relax%
	\lowercase{\endgroup\def~}{#2}%
	\AtBeginDocument{\mathcode`#1="8000}%
}
\defActiveMathChar{.}{\MD} 
%This way 'l.f' looks good as does 'l\D f', but the latter is clearly different; not that it really matters, but it's nice to have "code symbols" look like code and math symbols look like math

% Just to ensure consistency amongst the various syntactic forms when we use them in maths
\newcommand{\M}[3]{\ensuremath{\Kw{M}\oR{}#1\semiColon\,#2\semiColon\,#3\cR}}
\newcommand{\as}[2]{\ensuremath{#1\,\Kw{as}\,#2}} %#1\ \Kw{promote}\ \oC#2\cC}}
\newcommand{\cas}[1]{\as{#1}{\Kw{capsule}}}
\newcommand{\call}[3]{#1\D #2\oR#3\cR}
\newcommand{\new}[2]{\Kw{new}\ #1\oR#2\cR}
\newcommand{\field}[3]{#1\,#2\,#3}
\newcommand{\try}[2]{\Kw{try}\ \oC#1\cC\ \Kw{catch}\ \oC#2\cC}
\newcommand{\clazz}[4]{\Kw{class}\ #1\ \Kw{implements}\ #2\ \oC#3\semiColon#4\cC}
\newcommand{\iclazz}[3]{\Kw{interface}\ #1\ \Kw{implements}\ #2\ \oC#3\cC}
\newcommand{\methods}[4]{#1\,\Kw{method}\,#2\,#3\oR#4\cR}
\newcommand{\method}[5]{\methods{#1}{#2}{#3}{#4}\,#5}
\newcommand{\trys}[3]{\Kw{try}^{#1}\oC#2\cC\ \Kw{catch}\ \oC#3\cC}

\newcommand{\fmdf}{\V\kappa} % field modifier metavariable, I'm open to different letters

\newcommand{\invariant}[1]{\call{\oR\Kw{read}\,#1\cR}{\Kw{invariant}}{}}
\renewcommand\ttfamilywithbold\ttfamily
\lstset{
%  basicstyle=\ttfamilywithbold,
%  keywordstyle=\small\ttfamily\bfseries\color{darkRed},
  morekeywords={assert, expose, iso, isolated, baloon, rep, tag}
  }

\lstdefinelanguage{FortyThree}[]{FortyTwo}{literate={@}{@}{1},morekeywords={rep,as,promote}} % 42 but with rep keyword, also stops @ from being treated as part of an identifier
\lstdefinelanguage{FortyFour}[]{FortyThree}{basicstyle=\small\ttfamilywithbold\bfseries} % 43 but all code in bold, for use in math sections
\lstset{language=FortyThree}

\newcommand{\thm}[1]{\textsf{#1}}

%\HideNotes

% Spacing hacks!
\renewcommand{\SS}[1][0.5]{\vspace{-#1\baselineskip}}
\newcommand{\LS}[1][0.5]{\vspace{#1\baselineskip}}
% for paragraph breaks in itemize
\newcommand{\LSitem}{\LS[0.2]}
% for paragraph breaks in an itemize (inside an itemize or enumerate)
\newcommand{\LSiitem}{\LS[0.35]}
% for paragraph breaks in an enumerate
\newcommand{\LSenum}{\LS[0.5]}
% Paragraph breaks for an intemize nested within at leastt 2 itemize/enumerates
\newcommand{\LSiiitem}{\LS[0.5]}

\lstset{aboveskip=0.25\baselineskip,belowskip=0.25\baselineskip} % spacing arround lstlistings
\newcommand{\SSI}{\SS} % spacing before top level itemizes/enumerations

\newcommand{\newmathword}[2]{\def#1{\ensuremath{\mathit{#2}}\xspace}	}
\newmathword{\fr}{fresh}
\newmathword{\HNO}{headNotObservable}
\newmathword{\dom}{dom}

\newmathword{\CR}{circular}
\newmathword{\RCR}{repCircular}

\newmathword{\RM}{repMutating}
%\newcommand{\NRM}{\text{not }\RM}

\newmathword{\CN}{confined}
\newmathword{\RCN}{repConfined}
\newmathword{\rog}{ROG}
\newmathword{\mrog}{MROG}
\newmathword{\rf}{repFields}
\newmathword{\muty}{mutatable}
\newmathword{\mony}{monitored}
\newmathword{\valid}{valid}
\newmathword{\trusted}{trusted}
\newmathword{\reach}{reachable}
\newmathword{\error}{error}
\newmathword{\encap}{encapsulated}
\newmathword{\immut}{immutable}
\newmathword{\VS}{validState}

\let\f=\undefined
\let\m=\undefined
\let\T=\undefined
\let\Many=\undefined

\newcommand{\CD}{\mathit{CD}}
\renewcommand{\vs}{\mathit{vs}}
\newcommand{\ls}{\mathit{ls}}
\renewcommand{\es}{\mathit{es}}
\newcommand{\ws}{\mathit{ws}}
\newcommand{\Cs}{\mathit{Cs}}
\newcommand{\Ms}{\mathit{Ms}}
\newcommand{\Fs}{\mathit{Fs}}
\newcommand{\Ps}{\mathit{Ps}}
\newcommand{\Ss}{\mathit{Ss}}


% Change the spacing of enumerates slightly to be consistenct with itemize
\renewcommand{\labelenumi}{\theenumi.\hskip-0.25em}

% Too much space arround itemizes used in definitions, so fix that
\newenvironment{iitemize}{\SS\begin{itemize}}{\end{itemize}\SS}
\newenvironment{ienumerate}{\SS\begin{enumerate}}{\end{enumerate}\SS}
% Like the above, but when nested one level deep within an outer itemize/enumerate
\newenvironment{nitemize}{\SS[0.15]\begin{itemize}}{\end{itemize}\SS[0.15]}
\newenvironment{nenumerate}{\SS[0.15]\begin{enumerate}}{\end{enumerate}\SS[0.15]}

\NewDocumentCommand\ty{O{\s} O{\G} O{\vdash} m m}{\ensuremath{#1; #2 #3 #4 : #5}\xspace}
\NewDocumentCommand\nty{O{\s} O{\G} m m}{\ty[#1][#2][\nvdash]{#3}{#4}}
\NewDocumentCommand\tyr{O{\s} m m}{\ty[#1][\emptyset]{#2}{#3}}
\NewDocumentCommand\ntyr{O{\s} m m}{\nty[#1][\emptyset]{#2}{#3}}


\NewDocumentCommand\tye{O{\s} O{\G} O{\vdash} m m}{\ensuremath{#1; #2 #3 #4 :: #5}\xspace}
\NewDocumentCommand\tyer{O{\s} m m}{\tye		[#1][\emptyset]{#2}{#3}}
%\renewcommand{\h}{\ensuremath{\square}\xspace}
%\ensuremath{\scalebox{0.5}[1]{\ensuremath{\square}}}\xspace}
\providecommand\G{}
\renewcommand{\G}{\ensuremath{\Gamma}\xspace}
\newcommand{\EV}{\ensuremath{\ctx_v}\xspace}
\newcommand{\ER}{\ensuremath{\ctx_r}\xspace}
\newcommand{\E}{\ensuremath{\ctx}\xspace}
\newcommand{\s}{\ensuremath{\sigma}\xspace}
%\renewcommand{\e}{\ensuremath{\sigma}\xspace}
%\renewcommand{\l}{\ensuremath{\sigma}\xspace}
%\renewcommand{\v}{\ensuremath{\sigma}\xspace}

\let\@=\undefined % So I don't accidently type it!
%\HideNotes
%\NoNotes
%\newcommand{\square}{[]} % use an actual square as it's more standard
%\newcommand{\nvdash}{\not\vdash}

\newcommand{\C}[2][\s]{\ensuremath{\mathrm{C}^{#1}_{#2}}}
\newcommand{\rmdf}[2]{\ensuremath{#1{::}#2}}
\newcommand{\derep}[1]{\widetilde{#1}}
\newcommand{\demut}[1]{\widehat{#1}}
\renewcommand{\equals}{\,\Kw{=}\,} % Because there are usually modifies on both the left and right, it looks better with spaces
\renewcommand{\Terminale}[1]{\text{\lstinline|#1|}}
%\newcommand{\WT}[2][\s]{#1 \vdash #2}


\NewDocumentCommand\rangeX{O{,} m m}{\ensuremath{#2 #1 \ldots #1 #3}}

% For dot dot dot expantions \range[<sep>][<start>][<end>]{<body>} == <body>_<start> <sep>...<sep> <body>_<end>
% Using the default parameter vlaues, this is "<body>_1 , ..., <body>_n"
\NewDocumentCommand\range{O{,} O{1} O{n} m}{\rangeX[#1]{#4_{#2}}{#4_{#3}}}
% Similarly, but with two bodiies:   <body1>_1 <body2>_1 ,..., <body2>_n <body2>_n
% This automatically add a space beetween the <body1>_i and <body2>_i, so start <body2> with a \! to supress this (e.g. if <body2> starts with an operator, and so comes with it's own spacing)
\NewDocumentCommand\drange{O{,} O{1} O{n} m m}{\rangeX[#1]{#4_{#2}\,#5_{#2}}{#4_{#3}\,#5_{#3}}}
% Ditto, but with 3 bodies
\NewDocumentCommand\trange{O{,} O{1} O{n} m m m}{\rangeX[#1]{#4_{#2}\,#5_{#2}\,#6_{#2}}{#4_{#3}\,#5_{#3}\,#6_{#3}}}
% 4 bodies
\NewDocumentCommand\qrange{O{,} O{1} O{n} m m m m}{\rangeX[#1]{#4_{#2}\,#5_{#2}\,#6_{#2}\,#7_{#2}}{#4_{#3}\,#5_{#3}\,#6_{#3}\,#7_{#3}}}

% Like \drange, but the second body does not get a subscript:  <body1>_1 <body2> ... <body2>_n <body2>
% Usefull for symbols or constants
\NewDocumentCommand\rangex{O{,} m}{\rangeX[#1]{#2}{#2}}
\NewDocumentCommand\drangex{O{,} O{1} O{n} m m}{\rangeX[#1]{#4_{#2}\,#5}{#4_{#3}\,#5}}
% Like \trange, but the 3rd argument doesn't get a subscript
\NewDocumentCommand\trangex{O{,} O{1} O{n} m m m}{\rangeX[#1]{#4_{#2}\,#5_{#2}\,#6}{#4_{#3}\,#5_{#3}\,#6}}
% Dittot, but for \qrange
\NewDocumentCommand\qrangex{O{,} O{1} O{n} m m m m}{\rangeX[#1]{#4_{#2}\,#5_{#2}\,#6_{#2}\,#7}{#4_{#3}\,#5_{#3}\,#6_{#3}\,#7}}

% Reduction rule: \crule{<name>}{<s1>}{<e1>}{<s2>}{<e2>}
% Just put side conditions after it in \text{...}
\newcommand{\crule}[5]{\textsc{(#1)}\ #2|#3 \rightarrow #4|#5}
\newcommand{\rrule}[5]{\crule{#1}{#2}{\EV[#3]}{#4}{\EV[#5]}}
% Typing rule: \irule{<name>}{<premises>\...}{<conclusion>}
\newcommand{\irule}[3]{\inferrule*[vcenter,left={(\textsc{#1})}]{#2}{#3}}
% Typing rule: \irules{<name>}{<premises>\...}{<conclusion>}{<side-conditions...>}
\newcommand{\irules}[4]{\inferrule*[vcenter,left={(\textsc{#1})},right={%
	\!\!\!\ensuremath{\begin{array}{l}#4\end{array}}%
}]{#2}{#3}}
%Like \irule but for use in typing derivations
\newcommand{\iruled}[3]{\inferrule*[left={(\textsc{#1})}]{#2}{#3}}
\newcommand{\irulesd}[4]{\inferrule*[vcenter,left={(\textsc{#1})},right={%
		\!\!\!\ensuremath{\begin{array}{l}#4\end{array}}%
	}]{#2}{#3}}
% So a line containing \vdots has the same height as a normal line, and the \vdots are properly centered
\let\ovdots=\vdots
\renewcommand{\vdots}{\smash{\raisebox{-0.5ex}{\ovdots}}}

\renewcommand{\seguitoProduzione}[2]{&\!\!\!\hphantom{\produzioneDef}\llap{$\mid$}\!\!&#1&\!\!\mbox{{\small{#2}}}}

% Right justify the grammar names
\renewenvironment{grammatica}{$\begin{array}[t]{lclr}}{\end{array}$}

\renewcommand{\sectionautorefname}{Section} % Capatalise references to sections (whereas figures and appendices capatalise by default)
\renewcommand{\subsectionautorefname}{Section} % So 'Section 1.2' instead of 'subsection 1.2'
\makeatletter
\renewcommand{\appendixname}{\@gobble} % Surpresses the extra "Appendix " that gets added when you use \autoref

\renewcommand{\IO}[2][\@nil]{\NoteColour{Blue}{%
	\def\tmp{#1}\ifx\tmp\@nnil\else{\textsuperscript{\^{}1}}\fi%
	#2}}

\makeatother