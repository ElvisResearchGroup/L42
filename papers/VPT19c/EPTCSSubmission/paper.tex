\PassOptionsToPackage{svgnames}{xcolor}
\documentclass[submission,copyright,creativecommons]{eptcs}
\providecommand{\event}{VPT 2019} % Name of the event you are submitting to
%\usepackage{breakurl}             % Not needed if you use pdflatex only.
\usepackage{underscore}           % Only needed if you use pdflatex.
%\usepackage{listings}
\usepackage{xcolor}
\usepackage{letltxmacro}
\usepackage{mathtools}
\usepackage{mathpartir}
%\usepackage{stix}

\definecolor{darkRed}{RGB}{100,0,10}
\definecolor{darkBlue}{RGB}{10,0,100}
\newcommand*{\ttfamilywithbold}{\fontfamily{pcr}\selectfont}
%\newcommand*{\ttfamilywithbold}{\ttfamily}

%found on http://tex.stackexchange.com/questions/4198/emphasize-word-beginning-with-uppercase-letters-in-code-with-lstlisting-package
%\lstset{language=FortyTwo,identifierstyle=\idstyle}
%
\makeatletter
\newcommand*\idstyle{%
        \expandafter\id@style\the\lst@token\relax
}
\def\id@style#1#2\relax{%
        \ifcat#1\relax\else
                \ifnum`#1=\uccode`#1%
                        \ttfamilywithbold\bfseries
                \fi
        \fi
}
\makeatother

\lstset{language=Java,
  basicstyle=\upshape\ttfamily\footnotesize,%\small,%\scriptsize,
  keywordstyle=\upshape\bfseries\color{darkRed},
  showstringspaces=false,
  mathescape=true,
  xleftmargin=0pt,
  xrightmargin=0pt,
  breaklines=false,
  breakatwhitespace=false,
  breakautoindent=false,
 identifierstyle=\idstyle,
 morekeywords={method,Use,This,constructor,as,into,rename},
 deletekeywords={double},
 literate=
  {\%}{{\mbox{\textbf{\%}}}}1
  {~} {$\sim$}1
%  {<}{$\langle$}1
%  {>}{$\rangle$}1
}

\newcommand*{\SavedLstInline}{}
\LetLtxMacro\SavedLstInline\lstinline
\DeclareRobustCommand*{\lstinline}{%
	\ifmmode
	\let\SavedBGroup\bgroup
	\def\bgroup{%
		\let\bgroup\SavedBGroup
		\hbox\bgroup
	}%
	\fi
	\SavedLstInline
}

\newcommand\saveSpace{\vspace{-2pt}}

\newcommand\Rotated[1]{\begin{turn}{90}\begin{minipage}{12em}#1\end{minipage}\end{turn}}

\newcommand{\Q}{\lstinline}
\newenvironment{bnf}{$\begin{aligned}}{\end{aligned}$}
\newcommand{\production}[3]{\textit{#1}&\Coloneqq\textit{#2}&\text{#3}}
\newcommand{\prodNextLine}[2]{&\quad\quad\textit{#1}&\text{#2}}
\newenvironment{defye}{\\\indent$\begin{aligned}}{\end{aligned}$\\}
\newcommand{\defy}[2]{\!\!\!\!\!\!&&#1&\coloneqq#2\\}
%\newcommand{\defyc}[1]{&\phantom{\coloneqq}\ \ #1\\}
\newcommand{\defyc}[1]{\!\!\!\!\!\!\rlap{\quad \quad #1}&&\\}
\newcommand{\defya}[2]{#1&\!\!\!\!\!\!&\coloneqq#2\\}

%\newcommand{\prodFull}[3]{#1&::=&\mbox{#2}&\mbox{#3}}
\newcommand{\prodInline}[2]{#1\Coloneqq#2}
\newcommand{\terminal}[1]{\ensuremath{$\texttt{#1}$}}
%\newcommand{\metavariable}[1]{\ensuremath{\mathit{#1}}}

\newcommand{\Rulename}[1]{{\textsc{#1}}}
\newcommand{\ctx}[1]{\ensuremath{\mathcal{E}_#1}\!}
\newcommand{\libi}[2]{\Q@\{@\Q!interface!\ #1\Q{;} #2\Q@\}@}
\newcommand{\lib}[3]{\Q!interface!\ensuremath{?}\ \libc{#1}{#2}{#3}}
\newcommand{\libc}[3]{\,\Q@\{@\!#1\Q{;}\ #2 \Q{;}\ #3\Q@\}@\!\!}

\newcommand{\rp}[1]{\Q{(}\!#1\Q{)}}
\newcommand{\eq}[1]{\,\Q{=}#1}
\newcommand{\red}[3]{#1\,\Q{<}#2\eq#3\,\Q{>}}
\newcommand{\summ}[2]{#1\ \Q{<+}\ #2}
\newcommand{\from}[2]{#1\ensuremath{[}#2\ensuremath{]}}
\newcommand{\mmid}{{\ensuremath{{\mid}}}\!}
\newcommand{\hole}{\ensuremath{\square}}
\newcommand{\s}[1]{\ensuremath{\mathit{#1s}}}
\makeatletter
\newcommand{\This}[1]{\Q!This!#1\nextpath}
\newcommand{\Cs}[1]{#1\nextpath}
\newcommand{\nextpath}{\@ifnextchar\bgroup{\gobblenextpath}{}}
\newcommand{\gobblenextpath}[1]{\Q!.!#1\@ifnextchar\bgroup{\gobblenextpath}{}}
\makeatother



%--------------------------
\newcommand{\mynotes}[3]{{\color{#2} {\sc #1}: #3}}
\newcommand\isaac[1]{\mynotes{Isaac}{blue}{#1}}

\newcommand\IO[1]{\color{blue}{#1}}
\newcommand\marco[1]{\mynotes{Marco}{green}{#1}}


\usepackage{listings}
\usepackage{tikz}
\makeatletter
\newcommand*\idstyle{%
        \expandafter\id@style\the\lst@token\relax
}
\def\id@style#1#2\relax{%
        \ifcat#1\relax\else
                \ifnum`#1=\uccode`#1%
                        \ttfamilywithbold\bfseries
                \fi
        \fi
}
\makeatother

\definecolor{darkRed}{RGB}{100,0,10}
\definecolor{darkBlue}{RGB}{10,0,100}
%\newcommand*{\ttfamilywithbold}{\fontfamily{pcr}\selectfont}
\newcommand*{\ttfamilywithbold}{\ttfamily}

\lstdefinelanguage{FortyTwo}[]{Java}{morekeywords={%
  M,
  exception,error,mut,imm,
  read,capsule,lent,assert
  with,in,immutable,trait,using,
  on,var,loop,reuse,method,is
  },
   basicstyle=\ttfamily,
   keywordstyle=\ttfamilywithbold\bfseries\color{darkRed},
   identifierstyle=\idstyle,
   showstringspaces=false,
   mathescape=true,
%   texcl=true,
   xleftmargin=0pt,
   xrightmargin=0pt,
   breaklines=false,
   breakatwhitespace=false,
   breakautoindent=false,
   tabsize=2,
   commentstyle=\color{darkBlue}\ttfamily,
   stringstyle=\color{darkRed}\ttfamily,
   literate=
                 {\%}{{\mbox{\textbf{\%}}}}1
%                 {[}{{\ttfamilywithbold\textbf{[}}}1
%                 {]}{{\ttfamilywithbold\textbf{]}}}1
%                 {(}{{\ttfamilywithbold\textbf{(}}}1
%                 {)}{{\ttfamilywithbold\textbf{)}}}1
%                 {\{}{{\fontfamily{cmr}\selectfont\textbf{\{}}}1
%                 {\}}{{\fontfamily{cmr}\selectfont\textbf{\}}}}1
%                 {;}{{\ttfamilywithbold\textbf{;}}}1
                 {~} {$\sim$}1
 }

\lstset{language=FortyTwo}


\usepackage{xcolor}
\newcommand{\Q}{\lstinline}
\lstset{
    numbers=left,xleftmargin=1.8em,
    stepnumber=1,
    showstringspaces=false,
    firstnumber=last
}

\providecommand*{\code}[1]{\Q`#1`}
\usepackage{verbatim}

\title{Iteratively Composing Statically Verified Traits}
\def\titlerunning{Iteratively Composing Statically Verified Traits}

%magic code from https://tex.stackexchange.com/questions/344794/centering-issues-with-multiple-authors-with-the-same-affiliation-eptcs-format
\RequirePackage{array}
\newenvironment{authors}[1]%
  {\begingroup
   \gdef\estyle{}%
   \renewcommand\institute[1]%
     {\\\multicolumn{#1}{@{}c@{}}{\scriptsize\begin{tabular}[t]{@{}>{\footnotesize}c@{}}##1\end{tabular}}}%
   \renewcommand\email[1]%
     {\gdef\estyle{\footnotesize\ttfamily}\\##1\gdef\estyle{}}
   \begin{tabular}[t]{@{}*{#1}{>{\estyle}c}@{}}
  }%
  {\end{tabular}%
   \endgroup
  }

\def\anauthor#1#2{%
	#1%
	\institute{}
	\email{#2}%
}
\def\vuw{\institute{School of Engineering and Computer Science\\%
			Victoria University of Wellington\\%
			Wellington, New Zealand}}
\author{
	\begin{authors}{4}
	Isaac Oscar Gariano & Marco Servetto & Alex Potanin & Hrshikesh Arora
	\vuw
%	\email{\{isaac,& marco.servetto,& alex,& arorahrsh\}@myvuw.ac.nz}
	\email{~&\hspace{-6pt}\{isaac, marco.servetto,&  \hspace{-13pt} alex, arorahrsh\}@myvuw.ac.nz &~}
	\end{authors}
}

\def\authorrunning{I.\,O. Gariano, M. ,Servetto, A. Potanin \& H. Arora}

% Allows one to write /hello/ instead of \Q@hello@.
% Use // to get a normal text slash
\chardef\Slash=`\/
\catcode\Slash=\active
\chardef\other=12 % char code for other characters
\def/#1/{%
	\ifx/#1/% #1 is empty
		\Slash% just print a slash
	\else%
		\lstinline/#1/%
	\fi%
}
\let\oldinput=\input
\def\input#1{\oldinput{\detokenize{#1}}} % Don't expand commands in input (needed since / is a command)
\newcommand{\sref}[1]{Section~\ref{s:#1}}


%\definecolor{blue}{HTML}{0000F0} %
%\definecolor{purple}{HTML}{700090}
%\definecolor{orange}{HTML}{F07000}
%\definecolor{teal}{HTML}{0090B0}
%\definecolor{brown}{HTML}{A00000}
%\definecolor{green}{HTML}{008000}
%\definecolor{pink}{HTML}{F000F0}

\makeatletter
\renewcommand*\idstyle{%
	\expandafter\id@style\the\lst@token\relax}
\def\id@style#1#2\relax{%
        \ifcat#1\relax\else
                \ifnum`#1=\uccode`#1%
                        \color{DarkGreen}%
                \fi%
        \fi%
}
\makeatother


\lstset{%
	language=Java, morekeywords={exists, forall, @requires, @ensures, result, rename, hide, with},
	tabsize=2,
	identifierstyle=\idstyle,
%	aboveskip=0pt
%	keywordstyle=\color{blue},
%	commentstyle=\color{green},
%	stringstyle=\color{brown}
%	literate=
%		{||}{{$\vee$}}1
%		{&&}{{$\wedge$}}1
%		{<=}{{$\leq$}}1
%		{>=}{{$\geq$}}1
%		{**}{{$^{**}$}}1
}
%citations
\newcommand{\REV}[3]{%
	\NoteColour{red}{#1\NoteText{\footnote{%
		\textcolor{red}{\textbf{REV#2{:} #3}}}}}}
	
\begin{document}
\maketitle
\begin{abstract}
Static verification relying on an automated theorem prover can be very slow and brittle: since static verification is undecidable, correct code may not pass a particular static verifier.
In this work we use metaprogramming to generate code that is correct by construction.
A theorem prover is used only to verify initial ``traits'': units of code that can be used to compose bigger programs.

In our work, meta-programming is done by trait composition, which starting from correct code, is guaranteed to produce correct code.
We do this by extending conventional traits 
with pre- and post-conditions for the methods; we also extend the  traditional trait composition (/+/) operator to check the compatibility of contracts. In this way, there is no need to re-verify the produced code.

We show how our approach can be applied to the standard ``power'' function example, where metaprogramming generates optimised, and correct, versions when the exponent is known in advance.
\end{abstract}

\noindent Object oriented languages supporting static verification (SV) usually extend the syntax for method declarations
to support \emph{contracts} in the form of pre and post-conditions~\cite{Meyer:1988:OSC:534929}.
Correctness is defined only for code annotated with such contracts.

We say that a method is \emph{correct}, if whenever its precondition holds on entry, the precondition of every directly invoked method holds, and the postcondition of the method holds when the method returns. Automated SV typically works by asking an automated theorem prover to verify that each method is correct individually, by assuming the correctness of every other method~\cite{barnett2004spec}. This process can be very slow and can produce unexpected results: since SV is undecidable correct code may not pass SV.
Many SV approaches are not resilient to
\MSDel{some} standard refactoring techniques like 
method inlining. Sometimes SV even \IO{times out, making it harder to use such refactoring techniques.} \MSDel{terminate for a time-out, exacerbating the impact of transformations like method inlining.}

Metaprogramming is often used to programmatically generate faster specialised code when some parameters are known in advance, this is particularly useful where the specialisation mechanism is too complicated for a generic compiler to automatically derive~\cite{Ofenbeck:2017:SGP:3136040.3136060}
We could use metaprogramming to generate code together with contracts, and then once the metaprogramming has been run,
 \MSDel{ensure the correctness of} \IO{SV} the resulting code \MSDel{by applying SV.}. \IOComm{You don't `apply static verification', rather you `statically verify', you could also `use a static verifier'.}
However, the resulting code could be much larger than the input to the metaprogramming, and so it could take a long time to SV.
Moreover, one of the \MSDel{main} \IO{many} goal\IO{s} of metaprogramming is \IO{to} make it \IO{easier} \MSDel{easy} to generate \MSDel{many specialized} \IO{specialised} versions of the same \MSDel{functionality} \IO{code}. \IOComm{A `version of functionality' doesn't really make sense, a `version of code' does, we could also use `function//method'.}
%, Even if the generated code was produced by using straightforward transformations and compositions over the input code, a SV might not verify it's correctness.
The aim of our work is to \MSDel{apply} SV only \MSDel{to the code manually wrote by the programmer} \IO{code written directly, and not code produced by metaprogramming;}
\IO{instead, we}
\MSDel{and to} ensure that the result of metaprogramming is \MSDel{instead} correct by construction.

Here we use the disciplined form of metaprogramming introduced by Servetto \& Zucca \cite{servetto2014meta}, which is based on trait composition and adaptation~\cite{scharli2003traits}.
Here a /Trait/ is a unit of code: a set of method declarations.
\IO{Such methods can be abstract and be}
\MSDel{Those methods can be abstract, and they can}
be mutually recursive by using the implicit parameter /this/.

As in~\cite{servetto2014meta}
we require that all the traits are well-typed
before they are used.
\MSDel{Moreover, in our proposed approach we}
\IO{we extend this by allowing}
\MSDel{annotating} methods \IO{to be annotated} with pre//post-conditions, and 
\IO{ensuring that traits are correct in terms of such contracts}
\MSDel{we require that all traits are also correct}.
/Trait/s directly written in the source code are \IO{SVed} \MSDel{proven correct by SV}, while traits resulting from metaprogramming are \IO{ensured} correct by \IO{only providing trait operations that preserve correctness} \MSDel{construction}. \IOComm{SV proves correctness (that's the whole point), so no need to say 'proven correct by SV'. It was not clear how we `ensure correctness by construction', so I fixed that.}
\MSDel{Crucially}
\IO{In particular}, we extend the checking performed by \IO{the traditional trait composition (/+/)  operator, to also check the compatibility of contracts} \MSDel{composition and adaptation of /Trait/s to also check that contracts are composed correctly; thus ensuring the correctness of the result}. \IOComm{we only extend the /+/ operator here, so I've made that explicit.}
\IOComm{The above paragraph of changes are trying to make it more clear what our contribution is, but even then it's not good enough}.

%
%The result of composing and adapting /Trait/s is also correct and well-typed.
%
Our metaprogramming approach does not \IO{allow generating} \IODel{generate} code from scratch \IO{(such as by generating ASTs), rather the language provides a specific set of primitive composition and adapation operators which preserve correctness}.
\MSDel{; rather code is only generated by composing and adapting traits.
Each composition//adaptation step is guaranteed
to produce well typed and correct code; thus also the result of metaprogramming is well typed and correct.} \IOComm{I reworded it to make it sound like you are forced to use our safe operators, and can't subvert the system, since this makes our `guarantees` meaningful.}
Note that generated code may not be able to pass \IO{a particular} SVer, since theorem provers are not complete. \IOComm{In principle, since our code is correct, it should be able to pass some form of `static verification', however that dosn't mean it will base every `static verifier'.}


SV handles /extends/ and /implements/ by verifying that every 
time a method is implemented//overriden, 
the Liskov substitution principle~\cite{Liskov:1994:BNS:197320.197383} is satisfied
by checking that the \MSDel{new contract} \IO{contract of the override//implementation} implies the \IO{contract of the overriden//implemented method} \MSDel{overridden one}. 
\IOComm{In your version, it was not clear what contracts you were refering to.}
 In this way, there is no need to re-verify
inherited code in the context of the derived class.
This concept is easily adapted
to handle trait composition, which simply provides another way to implement an /abstract/ method.
When traits are composed,
it is sufficient
to match the contracts of the few composed methods
to ensure the whole result is correct.

In our examples we will use the notation /@requires($predicate$)/ 
to specify a precondition, and /@ensures($predicate$)/ 
to specify a postcondition; where $predicate$ is a boolean expression
in terms of the parameters of the method (including /this/), and for the /@ensures/ case, the /result/ of the method.
Suppose we want to implement an efficient exponentiation function, we could use recursion and the common technique of `repeated squaring':
\vspace{-1ex}
\begin{lstlisting}
@requires(exp > 0)
@ensures(result == x**exp) // Here x**y means x to the power of y
Int pow(Int x, Int exp) {
	if (exp == 1) return x;
	if (exp %2 == 0) return pow(x*x, exp/2); // exp is even
	return x*pow(x, exp-1); }  // exp is odd
\end{lstlisting}
If the exponent is known at compile time,
unfolding the recursion produces even more efficient code:
\vspace{-1ex}
\begin{lstlisting}
@ensures(result == x**7) Int pow7(Int x) { 
  Int x2 = x*x; // x**2
  Int x4 = x2*x2; // x**4
  return x*x2*x4; } // Since 7 = 1 + 2 + 4
\end{lstlisting}
\vspace{-1ex}


Now we show how the technique of \emph{Iterative Composition} (introduced in~\cite{servetto2014meta} and
enriched by \MSDel{our contract composition check} \IO{the contract compatibility check we propose performing in trait composition}) \IOComm{We never call `a contract composition check`, so it's not clear what you are talking about, since I already mentioned a `contract compatibility check above', I'm referencing it here}
can be used to write a metaprogram that given an exponent, produces code like the above.
Iterative Composition is a metacircular metaprogramming technique relying on \emph{compile-time execution} (as \IO{defined by}~\cite{sheard2002template}), \IOComm{I'm not sure what `as [10]' was meant to mean, correct me if my guess was wrong.} 
\MSDel{thus a metaprogram is just a function or a method wrote in the target programming language that is executed during compilation.}
\IO{, in our context this means that arbitrary expressions can be used as the RHS of a class declaration, during compilation such expressions will be evaluated to produce a /Trait/, which provides the body of the class. In this way metaprograms can be represented as otherwise normal functions//methods that return a /Trait/, without requiring the use of any additional `meta language'.} \IOComm{Major rewording, as your version didn't explain anything, in particular it was not clear what you meant by `during compilation'}.
 

\vspace{-1ex}
\begin{lstlisting}
Trait base=class {//induction base case: pow(x) == x**1
  @ensures(result>0) Int exp(){return 1;}  
  @ensures(result==x**exp()) Int pow(Int x){return x;}
  }
Trait even=class {//if _pow(x)== x**_exp(), pow(x) == x**(2*_exp())
  @ensures(result>0) Int $\_$exp();
  @ensures(result==2*$\_$exp()) Int exp(){return 2*$\_$exp();}
  @ensures(result==x**$\_$exp()) Int $\_$pow(Int x);
  @ensures(result==x**exp()) Int pow(Int x){return $\_$pow(x*x);}
}
Trait odd=class {//if _pow(x)== x**_exp(), pow(x) == x**(1+_exp())
  @ensures(result>0) Int $\_$exp();
  @ensures(result==1+$\_$exp()) Int exp(){return 1+$\_$exp();}
  @ensures(result==x**$\_$exp()) Int $\_$pow(Int x);
  @ensures(result==x**exp()) Int pow(Int x){return x*$\_$pow(x);}
}
//`compose' performs a step of iterative composition
Trait compose(Trait current, Trait next){
  current = current[rename exp->$\_$exp, pow->$\_$pow];
  return (current+next)[hide $\_$exp, $\_$pow];}
@requires(exp>0)//the entry point for our metaprogramming
Trait generate(Int exp) {
  if (exp==1) return base;
  if (exp%2==0) return compose(generate(exp/2),even);
  return compose(generate(exp-1),odd);
};
class Pow7: generate(7) //generate(7) is executed at compile time
//the body of class Pow7 is the result of generate(7)
/*example usage:*/new Pow7().pow(3)==2187//Compute 3**7
\end{lstlisting}
\vspace{-1ex}


%The /+/ operator is the main way to compose traits%
%~\cite{scharli2003traits,LagorioSZ09}.
%The result of /+/ will contain all the methods from both operands. 

%Crucially, it is possible to sum traits where a method is declared in both operands; in this case at least one of the two competing methods needs to be abstract, and the signatures of the two competing methods need to be \emph{compatible}.
Then, the operator /+/ is used to compose the code of the parameters.
Here we show how we ensure that the traditional /+/ operator also handles contracts: we require that the contract annotations of the two competing methods are \emph{compatible}.
In this paper, we just require them to be syntactically identical. Relaxing this constraint is an important future work.
Thanks to this constraint \textbf{the sum operator also preserves code correctness}. %\IO{There are many variations of the /+/ operator, in particular, we could easily extend our contract matching to work with an nary operator}.

The sum is executed when the method /compose/
%\IO{\footnote{\IO{a generic implementation of this method that renames and hides conflicting methods has been implemented L42~\cite{l42}}}}
runs: if the matched contracts are not identical an exception will be raised. A leaked exception during compile-time metaprogramming would become a compile-time error. 
Our approach is very similar to~\cite{servetto2014meta} and does not guarantee the success of the code generation process, rather it guarantees that if it succeeds, correct code is generated.

Executing /compose(base,even)/ or /compose(base,odd)/ will pass this test: since the contract of /base.pow()/
is the same of /even.$\_$pow()/ and /even.$\_$pow()/, and the same for /exp()/.

Finally the /$\_$pow(x)/ and /$\_$exp()/ method are hidden, so that the structural shape of the result is
the same as /base/'s.
Note that this structural equality includes the contracts of methods.

Note that /Trait/s are first class values and can be manipulated with a set of primitive operators that preserve code correctness and well-typedness.
In this way, by inductive reasoning, we can start from the /base/ case and then recursively compose /even/ and /odd/ until we get the desired code.
Note how the code of /generate(exp)/ follows the same scheme of the code of /pow(x,exp)/ in line 1.

To understand our example better, imagine executing the code of /generate(7)/ while keeping /compose/ in symbolic form. We would get the following (where /c/ is short for /compose/):
\vspace{-1ex}
\begin{lstlisting}[numbers=none]
generate(7) == c(generate(6),odd) == ...
 == c(c(c(c(base,even),odd),even),odd)
\end{lstlisting}
\vspace{-1ex}
As /base/ represents /pow1(x)/; /c(base,even)/ represents /pow2(x)/. Then \Q@c(/*pow2(x)*/,odd)@ represents \Q@pow3(x)@, \Q@c(/*pow3(x)*/,even)@ represents \Q@pow6(x)@, and finally,
\Q@c(/*pow6(x)*/,odd)@ represents \Q@pow7(x)@.
The code of each /$\_$pow(x)/ method is only executed once for each top-level /pow(x)/ call, so the /hide/ operator can inline them.
Thus, the result could be identical to the manually optimized code in line 7.
We can use our /generate(7)/ as follows:
\begin{lstlisting}
class Pow7: generate(7)//generate is executed at compile time
//the body of class Pow7 is the result of generate(7)
/*example usage:*/
new Pow7().pow(3)==2187//Compute 3**7
\end{lstlisting}

%\IO{We are investigation how an additional check can be performed to ensure the resulting code has specific contracts. However, our approach does guarantee that the result will be correct according to whatever contracts it contains.} 
\section{Future Work}
Our approach, as presented in this short paper, only guarantees that code resulting from metaprogramming follows its own contracts, it does
not statically ensure what those contracts may be. As future work, we are investigating how the resulting contracts can be ensured to have a particular meaning or form.
To do so, we need to allow assertions on the contracts of /Trait/s to be used within pre//post conditions.
For example we could allow post conditions like\\*
%\begin{lstlisting}[numbers=none]
/@ensures(result.$\mathit{methName}$.ensures ==\ $\mathit{predicate}$)/ \\*
%\end{lstlisting}
to mean that the resulting /Trait/ has
a method
called $\mathit{methName}$, whose /@ensures/ clause is syntactically identical to  /$predicate$/; whilst
\\*
/@ensures(result.$\mathit{methName}$.ensures ==>\ $\mathit{predicate}$)/
\\*
would use a static verifier to ensure that $\mathit{methName}$'s /@ensures/ clause logically implies $\mathit{predicate}$.
With these two features we could annotate the method /generate(exp)/ in line 32 above as:
\begin{lstlisting}
@requires(exp>0)
@ensures(result.exp().ensures ==> (result==exp))
@ensures(result.pow(x).ensures == (result==x**exp()))
Trait generate(Int exp) {...}
\end{lstlisting}

\vspace{-1ex}
In this way, we could statically verify the /generate(exp)/ method, however we fear such verification will be too complex or impractical. 
We could instead automatically check the above postconditions after each call to /generate(exp)/. If /generate(exp)/ is used to define a class (such as /Pow7/ above), we will guarantee that such class has the expected contracts, before it is used. Thus
there is no need to ensure the correctness of the metaprogram itself: such runtime checks are sufficient to ensure that after compilation, the code produced by metaprogramming has its expected behaviour.
%\IODel{In this case we could defer those difficult//novel predicates to run-time checks, without losing much safety:
%Iterative Composition execute metaprogramming code at
%compile time, thus even run-time verification of metaprograms would happen at compile time. This consideration could result in a crucial design decision: code performing metaprogramming does not need to be verified by SV to produce code annotated with the desired contracts; it may be sufficient to apply some type of runtime verification during compile-time execution.} \IOComm{I did a major rewording since we actually have multiple compile-times and run-times, so your version is confusing, hopefully my version makes the point more clear.}
%For example, the following code:
%\vspace{-1ex}
%\begin{lstlisting}[numbers=none]
%@ensures(new Pow7().exp()==7&&Pow7.pow.ensures=="result==x**exp()")
%class Pow7: generate(7)
%\end{lstlisting}
%\vspace{-1ex}
%may require the static verifier to check that the execution of
%/new Pow7().exp()/ will deterministically reduce to /7/, and that the /ensures/ clause of 
%/Pow7.pow/ is syntactically equivalent to 
%/result==x**exp()/. Note how this final step of static verification does not need to re-verify the body of
%/Pow7.pow/ and only needs to do a coarse grained 
%determinism check on the implementation of /Pow7.exp()/, before symbolically executing it.

\section{Conclusion}
By exploiting conventional OO static verification techniques, we have extended the Iterative Composition form of metaprogramming with a simple contract compatibility check, to statically ensure the correctness of code produced by such metaprogramming. In particular, our approach does not require static verification of the result of metaprogramming, but only requires verification of code present directly in source code.
Following general terminology in software verficiation, we say that a trait is \emph{correct} when its methods respect their contracts.
Thus our result is that starting from a set
of well typed and correct traits, 
any code resulting from arbitrary many steps of trait composition will also be correct and well typed.
In this way, the programmer need only provide correct bulding blocks using traits;
code generated by metaprogramming can be integrated with a correct program without needing to use expensive theorem provers or manual verification.
Our example is applied to code specialization of a mathematical function, but our experience suggests that Iterative composition can be used to synthesize arbitrary behaviour.


\catcode\Slash=12% turn of my slash
\bibliographystyle{eptcs}
\bibliography{paper}
%\clearpage
%\appendix
%\appendix
\section{Proof} 
\label{s:proof}

\begin{theorem}[Sound Validation]
	if $c:\Kw{Cap};\emptyset\vdash \e: \T$ and
	$c\mapsto\Kw{Cap}\{\_\}|\e\rightarrow^+ \sigma|\ctx_v[r_l]$, then
	either $valid(\sigma,l)$ or $\mathit{trusted}(\ctx_v,r_l)$.
\end{theorem}

We believe this property captures very precisely our statement in Section~\ref{s:validation}.

It is hard to prove Sound Validation directly,
so we first define a stronger property,
called \emph{Stronger Sound Validation} and
show that it is preserved during reduction by means of conventional 
Progress and Subject Reduction (Progress is one of our assumption,
while Subject Reduction relies heavily on SubjectReductionBase).
That is,
Progress+Subject Reduction $\Rightarrow$ Stronger Sound Validation,
\\*and Stronger Sound Validation $\Rightarrow$ Sound Validation.

\subsection{Stronger Sound Validation $\Rightarrow$ Sound Validation}

Stronger Sound Validation depends on 
$\mathit{wellEncapsulated}$, $\mathit{monitored}$
and $OK$:

\noindent\textbf{Define} $\mathit{wellEncapsulated}(\sigma,\e,l_0)$:\\*
\indent$\forall l \in \mathit{erog}(\sigma,l_0), \text{not}\ \mathit{mutatable}(l,\sigma,\e)$

\noindent The main idea is that an object is well encapsulated if its encapsulated state is safe from
modification. 

\noindent\textbf{Define} $\mathit{monitored}(\e,l)$:\\*
\indent$\e=\ctx_v[M(l;\e_1;\e_2)]$ and either $\e_1=l$ or $l$ is not inside $\e_1$.

\noindent An object is monitored if the execution
is currently inside of a monitor for that object, and
the monitored expression $\e_1$ does not
contains $l$ as a \emph{proper} subexpression.

A monitored object is associated with an expression that can not observe it, but may 
reference its internal representation directly.
In this way, we can safely modify its representation before checking for the invariant.

The idea is that at the start the object will be valid and $\e_1$ will contain $l$;
but during reduction, the $l$ reference will be used in order to
give access to the internal state of $l$; only after that moment, the object may become invalid.


\noindent\textbf{Define} $OK(\sigma,e)$:\\
\indent $\forall l\in\dom(\sigma)$
  either\\
\indent\indent 1. $\mathit{garbage}(l,\sigma,\e)$\\
\indent\indent 2. $\mathit{valid}(\sigma,l)$ and $\mathit{wellEncapsulated}(\sigma,\e,l)$\\
\indent\indent 3. $\mathit{monitored}(\e,l)$

Finally, the system is in a valid state with respect to validation
if for all the objects in the memory, one of these 3 cases apply:
%the class of the object has no invariant method;
the object is not (transitively) reachable from the expression (thus can be garbage collected);
the object is valid, and the object is encapsulated;
or the object is currently monitored.

\begin{theorem}[Stronger Sound Validation]
if $c:\Kw{Cap};\emptyset\vdash \e_0: \T_0$ and
$c\mapsto\Kw{Cap}\{\_\}|\e_0\rightarrow^+ \sigma|\e$, then
$OK(\sigma,\e)$
\end{theorem}
\noindent Starting from only the capability object,
any well typed expression $\e_0$ can be reduced for an arbitrary amount of steps,
and $OK$ will always hold.
\\
\begin{theorem} Stronger Sound Validation $\Rightarrow$ Sound Validation
\end{theorem}
\begin{proof}
\noindent By Stronger Sound Validation, each $l$ in the current redex must be $OK$:
\begin{enumerate}
	\item If $l$ is garbage, it cannot be in the current redex, a contradiction.
	\item If $\mathit{valid}(\sigma,l)$, then $l$ is valid, so thanks to Determinism
	no invalid object could be observed.
	\item Otherwise, if $\mathit{monitored}(\e,l)$ then either:
	\begin{itemize}
	 \item we are executing inside of $\e_1$ thus the current redex is inside of a sub-expression of the monitor that does not contain $l$, a contradiction.
	 \item or we are executing inside $\e_2$:
	 by our reduction rules, all monitor expressions start with 
	 $\e_2=l$\Q@.validate()@, thus the first execution step
	 of $\e_2$ is trusted. Following execution steps are also trusted, since by well formedness the body of invariant methods only use \Q@this@ (now translated to $l$) to access fields.
	\end{itemize}
\end{enumerate}
In any of the possible cases above, Sound Validation holds for $l$, and so it holds for all redexes.
\end{proof}

\subsection{Subject Reduction}

\noindent\textbf{Define} $\text{fieldGuarded}(\sigma,\e)$:\\*
\indent$\forall \ctx$ such that $\e=\ctx[l\singleDot\f] $
and $\Sigma^\sigma(l).f=\Kw{capsule}\,\_$, and $\f\mathrel{\mathit{inside}} \Sigma^\sigma(l).\mathit{validate}$\\*
\indent\indent either 
$\forall T, \forall C, \Sigma^\sigma;\x:\Kw{mut}\,C\,\not\vdash\ctx[\x]:T$, or\\*
\indent\indent $\ctx=\ctx'[$\Q@M(@$l$\Q@;@$\ctx''$\Q@;@$\e$\Q@)@$]$ and $l$ is contained exactly once in $\ctx''$

That is, all \emph{mutating} capsule field accesses are individually guarded by monitors.
Note how we use $C$ in $\x:\Kw{mut}\,C$ to guess the type of the accessed field,
and that we use the full context $\ctx$ instead of the evaluation context $\ctx_v$
to refer to field accesses everywhere in the expression $\e$.


\begin{theorem}[Subject Reduction]
if $\Sigma^{\sigma_0};\emptyset\vdash e_0: T_0$,
$\sigma_0|e_0\rightarrow \sigma_1|e_1$,
$OK(\sigma_0,\e_0)$
and
$\mathit{fieldGuarded}(\sigma_0,\e_0)$
then
$\Sigma^{\sigma_1};\emptyset\vdash e_1: T_1$,
$OK(\sigma_1,e_1)$ and
$\mathit{fieldGuarded}(\sigma_1,\e_1)$
\end{theorem}

\begin{theorem}
	Progress + Subject Reduction $\Rightarrow$ Stronger Sound Validation
\end{theorem}
\begin{proof}
This proof proceeds by induction in the usual manner.

\emph{Base Case}: At the start of the execution, the memory is going to only contain $c$: since $c$ is defined to be initially $\mathit{valid}$, and has only \Q@mut@ fields, and so it is trivially $\mathit{wellEncapsulated}$, thus $OK(c\mapsto\Kw{Cap},e)$.

\emph{Induction}: By Progress we always have another evaluation step to take, by Subject Reduction such a step will preserve $\mathit{OK}$, and so by induction $\mathit{OK}$ holds after any number of steps.

Note how for the proof garbage collection is important: 
when the \Q@validate()@ method evaluates to \Q@false@, 
execution can continue only if the offending object is classified as garbage.
\end{proof}

\subsection{Expose Instrumentation}
We first introduce a lemma derived from well formedness and the type system:
\begin{Lemma}[ExposerInstrumentation]
If $\sigma_0 | \e_0\rightarrow \sigma_1 |\e_1$ and
$\text{fieldGuarded}(\sigma_0,\e_0)$
\\*
then $\text{fieldGuarded}(\sigma_1,\e_1)$
\end{Lemma}
\begin{proof}
The only rule that can 
introduce a new field access is \textsc{mcall}.
In that case, ExposerInstrumentation holds
by well formedness (all field accesses in methods are of the form \Q@this.f@) 
and since \textsc{mcall} inserts a monitor while invoking capsule mutator methods, and not field accesses themselves. If however the method is not a \Q@mut@ method but still accesses a capsule field, by MutField such a field access expression cannot be typed as \Q@mut@ and so no monitor is needed.

Note that \textsc{monitor exit} is fine because monitors are removed only when
 $e_1$ is a value.
\end{proof}

\subsection{Proof of Subject Reduction}
Any reduction step can be obtained
by exactly one application of rule \textsc{ctx} and then one other rule. Thus the proof can simply proceed by cases on such other applied rule.

By SubjectReductionBase and ExposerInstrumentation, 
$\Sigma^{\sigma_1};\emptyset\vdash e_1: T_1$ and  $\mathit{fieldGuarded}(\sigma_1,\e_1)$. So we just need to proceed by cases on the reduction rule applied to verify that $OK(\sigma_1,\e_1)$ holds:


\begin{enumerate}
\item (\textsc{update}) $\sigma|l\singleDot f\equals v\rightarrow \sigma'|\e'$:
	\begin{itemize}
	  \item By \textsc{update} $\e'=\Kw{M}\oR l;l;l\singleDot\text{validate}\oR\cR\cR$, thus $\mathit{monitored}(\e,l)$.
	  \item Every $l_1$ such that $l\in \mathit{rog}(\sigma,l_1)$ will verify the same case as the former step:
	  \begin{itemize}
	  	\item If it was $\mathit{garbage}$, clearly it still is.
	  	\item If it was $\mathit{monitored}$, it also still is.
	    \item Otherwise it was $\mathit{valid}$ and $\mathit{wellEncapsulated}$:
			\begin{itemize}
				\item If $l\in \mathit{erog}(\sigma,l_1)$ we have a contradiction since $mutatable(l, \sigma, e)$, (by MutField)
		    	\item Otherwise, by our well-formedess criteria that \Q@.validate()@ only accesses \Q@imm@ and \Q@capsule@ fields, and by Determinism it is clearly the case that $\mathit{valid}$ still holds;
				By HeadNotCircular it cannot be the case that $l\in \mathit{erog}(\sigma',l_1)$ and so $l_1$ is still $\mathit{wellEncapsulated}$.
		  	\end{itemize}
	  \end{itemize}
	  \item Every other $l_0$ is not in the reachable object graph of $l$
	  thus it being $\mathit{OK}$ could not have been effected by this reduction step.
	\end{itemize}

\item (\textsc{access}) $\sigma|l\singleDot f \rightarrow \sigma|v$:
	\begin{itemize} 
		\item If $l$ was $valid$ and $wellEncapsulated$:
		\begin{itemize}
			\item If we have now broken $wellEncapsulated$ we must have made something in its $erog$  $mutatable$. As we can only type \Q@capsule@ fields as \Q@mut@ and not \Q@imm@ fields, by FieldMut we must have that $f$ is \Q@capsule@ and $l\singleDot f$ is being typed as \Q@mut@. By $\mathit{fieldGuarded}(\sigma_0,\e_0)$ the former step must have been inside a monitor \Q@M(@$l$\Q@;@$\ctx_v[l$\Q@.f@$]$\Q@;@$\e$\Q@)@
		    and the $l$ under reduction was the only occurrence of $l$.
		    Since $f$ is a capsule, we know that $l\notin \text{erog}(\sigma,l)$
		    by HeadNotCircular. Thus in our new step $l$ is not $inside$ $\ctx_v[v]$. Thus $l$ must be $monitored$ and hence it is $OK$.
		    
		    \item Otherwise, $l$ is still $OK$
    	\end{itemize}

		\item Nothing that was $\mathit{garbage}$ could have been made reachable by this expression, since the only value we produced was $v$ and it was reachable through $l$ (and so could not have been garbage), thus $garbage$ is still $OK$.
		
		\item As we don’t change any monitors here, nothing that was $monitored$ could have been made un-$monitored$, and so it is still $OK$.
		
		\item Suppose some $l_0$ was $wellEncapsulated$ and $valid$:
		\begin{itemize}
			\item If $l$ was in the $rog$ of $l_0$, by CapsulaeTree, if $l$ was in the $erog$ of $l$, then $v$ can only be reached from $l_0$ by passing through $l$, and so we could not have made $l_0$ non-$wellEncapsulated$. In addition, since only things in the $erog$ can be referenced by $\singleDot\Kw{validate}\oR\cR$, $l_0$’s validity can not depend on $l$, and by Determinism it is still the case that $l_0$ is $valid$. And so we can’t have effected $l_0$ being $OK$.
			\item Otherwise this reduction step could not have affected $l_0$ so $l_0$ is still $OK$.
		\end{itemize}
\end{itemize}

\item (\textsc{mcall}, \textsc{try enter} and \textsc{try ok}):

	These reduction steps do not modify memory, nor do they modify the memory-locations reachable inside of main-expression, nor do they modify any monitor expressions. Therefore it cannot have any effect on the $garbage$, $wellEncapsulated$, $valid$ (due to Determinism) or $monitored$ properties of any memory locations, thus $\mathit{OK}$ still holds.

\item (\textsc{new}) $\sigma|\Kw{new}\ C\oR\vs\cR\rightarrow \sigma,l\mapsto C\{\vs\}| \Kw{M}\oR l;l;l\singleDot\text{validate}\oR\cR\cR$:

	Clearly the newly created object ($l$) is monitored. As for \textsc{mcall}, other objects and properties are not disturbed, and so $\mathit{OK}$ still holds.


\item (\textsc{monitor exit}) $\sigma|\Kw{M}\oR l; v;\Kw{true}\cR\rightarrow \sigma|v$:
\begin{itemize}
	\item As monitor expressions are not present in the original source code, it must have been introduced by \textsc{update}, \textsc{mcall}, or \textsc{new}. In each case the 3\textsuperscript{rd} expression started of as $l\singleDot\Kw{validate}\oR\cR)$, and it has now (eventually) been reduced to $\Kw{true}$, thus by Determinism $l$ is $valid$.

	\item  If the monitor was introduced by \textsc{update}, then $v = l$. We must have had that $l$ was well encapsulated before \textsc{update} was executed (since it can’t have been garbage and $monitored$), as \textsc{update} itself preserves this property and we haven’t modified memory in anyway, we must still have that $l$ is $wellEncapsulated$. As $l$ is $valid$ and $wellEncapsulated$ it is $OK$.

	\item If the monitor was introduced by \textsc{mcall}. Then it was due to calling a capsule-mutator method that mutated a field $f$.
	\begin{itemize}
		\item A location that was $garbage$ obviously still is, and so is also $OK$.
		\item No location that was $valid$ could have been made non-valid since this reduction rule performs no mutation of memory. If a location was $wellEncapsulated$ before the only way it could be non-$wellEncapsulated$ is if we somehow leaked a \Q@mut@ reference to something, but by our well-formedness rules $v$ cannot be typed as \Q@mut@ and so we can’t have affected $wellEncapsulated$, hence such thing is still $OK$.
		\item The only location that could have been made un-$monitored$ is $l$ itself. By our well-formedness criteria $l$ was only used to modify $l.f$, and we have no parameters by which we could have made $l.f$ non-$wellEncapsulated$, since that would violate CapsuleTree. As nothing else in $l$ was modified, and it must have been $wellEncapsulated$ before the \textsc{mcall}, it still is, and since  $l$ is valid, it is $OK$.
	\end{itemize}
	\item Otherwise the monitor was introduced by \textsc{new}. Since we require that \Q@capsule@ fields and \Q@imm@ fields are only initialised to \Q@capsule@ and \Q@imm@ expressions, by CapsuleTree the resulting value, $l$, must be $wellEncapsulated$, since $l$ is also $valid$ we have that $l$ is $OK$.

\end{itemize}

\item (\textsc{try error}) $\sigma,\sigma_0|\Kw{try}^\sigma\oC \mathit{error}\cC\ \Kw{catch}\ \oC\e\cC\rightarrow \sigma|\e$:

	By StrongExceptionSafety we know that $\sigma_0$ is $\mathit{garbage}$ with respect to $\ctx_v[\e]$. By our well-formedness criteria no location inside $\sigma$ could have been $monitored$.

	Since we don’t modify memory, everything in $\sigma_0$ is $\mathit{garbage}$ and nothing inside $\sigma$ was previously monitored, it is still clearly the case that everything in $\sigma$ is $\mathit{OK}$
\end{enumerate}
\end{document}
