 
Here we show an example illustrating our system in action. Suppose we have a \Q@Cage@ class which contains a \Q@Hamster@; the \Q@Cage@ will move its \Q@Hamster@ along a path. We would like to ensure that the \Q@Hamster@ does not deviate from the path. We can express this as the invariant of \Q@Cage@: the position of the \Q@Cage@'s \Q@Hamster@ must be within the path (stored as a field of \Q@Cage@).

%
% While Spec\# requires specialised \Q@Point@, \Q@Hamster@, and \Q@Cage@ declarations to be able to enforce the invariant, our version manages to capture the required information in just a few annotations on \Q@Cage@ and leaves \Q@Point@ and \Q@Hamster@ unmodified.
%	if(that==null || !(that instanceof Point)){return false;}
% 	return ((Point)that).x==this.x && ((Point)that).y==this.y; 
%  }
\begin{lstlisting}
class Point { Double x; Double y; Point(Double x, Double y) {..}
  @Override read method Bool equals(read Object that) {
    return that instanceof Point &&
      this.x == ((Point)that).x && this.y == ((Point)that).y; }}
class Hamster {Point pos; //pos is imm by default
  Hamster(Point pos) {..}}
class Cage {
  capsule Hamster h;
  List<Point> path; //path is imm by default
  Cage(capsule Hamster h, List<Point> path) {..}
  read method Bool invariant() {
    return this.path.contains(this.h.pos); }
  mut method Void move() {
    Int index = 1 + this.path.indexOf(this.h.pos));
    this.moveTo(this.path.get(index % this.path.size())); }
  mut method Void moveTo(Point p) { this.h.pos = p; }}
\end{lstlisting}

We use the \Q@read@ annotation on the \Q@equals(that)@ method to express that it does not modify either its
receiver or its parameter. In \Q@Cage@ we use 
the \Q@capsule@ annotation to ensure
that the modification of the \Q@Hamster@'s \emph{reachable object graph} (ROG) is fully under the control
of the containing \Q@Cage@. 
We annotated the \Q@move()@
and \Q@moveTo(p)@ methods with \Q@mut@, since they modify
their receivers' ROG. The default annotation is always \Q@imm@, thus \Q@Cage@'s \Q@path@ field is a deeply immutable list of \Q@Point@s.
% Note how we just use \Q@List.contains()@ and \Q@List.indexOf()@
% to check if the hamster position is inside the list.
% The conventional syntax correctly instantiates a \Q@Cage@:
% \Q@new Cage(new Hamster(new Point(..)), List.of(new Point(...))@.
Our system performs runtime checks for the invariant
at the end of \Q@Cage@'s constructor, \Q@moveTo(p)@ method, and after any update to one of its fields.
The \Q@moveTo(p)@ method is the only one that may (directly) break the \Q@Cage@'s invariant. However, there is only a single occurrence of \Q@this@ and it is used to read the \Q@h@ field. We use the guarantees of RCs to ensure that no alias to \Q@this@ could be reachable from either \Q@h@ or the immutable \Q@Point@ parameter. Thus, the potentially broken \Q@this@ object is not visible while the \Q@Hamster@'s position is updated. 
The invariant is checked at the end of the \Q@moveTo(p)@ method, just before \Q@this@ would become visible again.
This technique loosely corresponds to an implicit pack and unpack: we `unpack' \Q@this@ before reading the field, then we work on the field's value while the invariant of \Q@this@ is not known to hold, finally when returning, we `pack' \Q!this! and check its invariant before allowing it to be used again.

Note: since only \Q@Cage@ has an invariant,
 only the code of \Q@Cage@ needs to be handled carefully; allowing the code for \Q@Point@ and \Q@Hamster@ to be unremarkable.
 This is not the case in Spec\#: all code involved in  verification needs to be designed with verification in mind~\cite{barnett2011specification}.
% The best solution we found was to define our own equality for \Q@Point@ instead of relying on \Q@Object.Equals@,
% thus we could not use \Q@List.Contains@ and \Q@List.IndexOf@.

\noindent We now show the hamster example in Spec\#; the system most similar to ours:
%or \small or \footnotesize etc.
\begin{lstlisting}[
language={[Sharp]C}, morekeywords={invariant,ensures,requires,expose,exists}]
// Note: assume everything is `public'
class Point { double x; double y; Point(double x, double y) {..}
  [Pure] bool Equal(double x, double y) {
    return x == this.x && y == this.y; }}
class Hamster{[Peer]Point pos; 
  Hamster([Captured]Point pos){..}}
class Cage {
  [Rep] Hamster h; [Rep, ElementsRep] List<Point> path;
  Cage([Captured] Hamster h, [Captured] List<Point> path)
    requires Owner.Same(Owner.ElementProxy(path), path); {
      this.h = h; this.path = path; base(); }
  invariant exists {int i in (0 : this.path.Count);
    this.path[i].Equal(this.h.pos.x, this.h.pos.y) };
  void Move() {
    int i = 0;
    while(i<path.Count && !path[i].Equal(h.pos.x,h.pos.y)){i++;}
    expose(this) {this.h.pos = this.path[i%this.path.Count];}}}
\end{lstlisting}

In both versions, we designed \Q@Point@ and \Q@Hamster@ in a general way, and not solely to be used by classes with an invariant: thus \Q@Point@ is not an immutable class. However, doing this in Spec\# proved difficult, in particular we were unable to override \Q@Object.Equals(that)@, or even define a usable \Q@equals(that)@ method that takes a \Q@Point@, as such we could not call either \Q@List<Point>.Contains(e)@ or \Q@List<Point>.IndexOf(e)@.
 
\noindent Even with all the above annotations, we needed special care in creating \Q@Cage@s:\vspace{-1.860px}% magic number that prevents the listings background going onto the next page
\begin{lstlisting}[
%basicstyle=\footnotesize,
language={[Sharp]C}, morekeywords={invariant,ensures,requires,expose,exists}]
List<Point> pl = new List<Point>{new Point(0,0),new Point(0,1)};
Owner.AssignSame(pl, Owner.ElementProxy(pl));
Cage c = new Cage(new Hamster(new Point(0, 0)), pl);
\end{lstlisting}

Whereas with our system we can simply write:
\begin{lstlisting}
List<Point> pl = List.of(new Point(0, 0), new Point(0, 1));
Cage c = new Cage(new Hamster(new Point(0, 0)), pl);
\end{lstlisting}

%3 read 2 capsule 3 mut extra method moveTo
%----
In Spec\# we had to add $10$ different annotations, of $8$ different kinds; some of which were quite involved. In comparison, our approach requires only $7$ simple keywords, of $3$ different kinds; however we needed to write 
a separate \Q@moveTo(p)@ method, since we do not want to burden our language with extra constructs such as Spec\#'s \Q@expose@.
%  Moreover we had been unable to reuse 
% \Q@Object.Equals@, \Q@List.IndexOf@ and % \Q@List.Contains@.
% Note: we had to add a new class \Q@PureObject@, since the \Q@Objec@ constructor is not annotated as \Q@[Pure]@.
%3 pure,
%1 peer
%3 captured
%2 rep
%1 ElementsRep
%1 requires Owner.Same(Owner.ElementProxy(path), path);
%1 invariant
%1 exists
%expose(this)
%re implementation of indexOf
%dumb equals(double,double)
%dumb class PureObject { [Pure] PureObject() { } }
%Owner.AssignSame(pl, Owner.ElementProxy(pl));
% manually handle ownership details while instantiating a \Q@new Cage(..)@.
% Note how the \Q@expose@ block cover plays the same role of our \Q@moveTo@ method.

%We evaluate our contribution by means of case studies;