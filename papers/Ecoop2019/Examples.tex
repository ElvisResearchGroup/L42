\section{Redirect applications}

\subsection{Graph example}
We now consider an example where Redirect simplifies the code quite a lot:
We have a \Q@Node@ and \Q@Edge@ concepts for a graph.
The \Q@Node@ have a list of \Q@Edge@s.
A \Q@isConnected@ function takes a list of \Q@Node@s.
A \Q@getConnected@ function takes \Q@Node@ and return a set of \Q@Node@s.
\begin{lstlisting}
graphUtils={
  Edges:list<+{Node start() Node end()}
  Node:{Edges connections()}
  Nodes:set<Elem=Node>//note that we do not specify equals/hash
  static Bool isConnected(Nodes nodes)=
    if(nodes.size()=0) then true
    else getConnected(nodes.asList().head()).size()==nodes.size()
  static Nodes getConnected(Node node)=getConnected(node,Nodes.empty())
  static Nodes getConnected(Node node,Nodes collected)=
    if(collected.contains(node)) then collected
    else connectEdges(node.connections(),collected.add(node))
  static Nodes connectEdges(Edges e,Nodes collected)=
    if( e.isEmpty()) then collected
    else connectEdges(e.tail(),collected.add(e.head().end()))
  }
\end{lstlisting}

We have shown the full code instead of omitting implementations to show that
the code inside of an highly general code like the former is pretty conventional.
Just declare nested classes as if they was the concrete desired types. Note how we can easly create a new \@Nodes@ by doing \Q@Nodes.empty()@.

Here we show how to instantiate \Q@graphUtils@ to a graph representing cities connected by streets,
where the streets are annotated with their length, and \Q@Edges@ is a priority queue, to optimize
finding the shortest path between cities.

\begin{lstlisting}
Map:{
  Street:{City start,City end, Int size}
  City:{}
  Streets:priorityQueue<Elem=Street><+{    
    Int geq(Street e1,Street e2)=e1.size()-e2.size()}
  }<+{
  Streets:{}
  City:{Streets connections, Int index}//index identify the node
  Cities:set<Elem=City><+{
    Bool eq(City e1,City e2) e1.index==e2.index
    Int hash(City e) e.index
    }
  Cities cities
  //more methods
  }
MapUtils=graphUtils<Nodes=Map.Cities>
//infers Nodes.List, Node, Edges, Edge
\end{lstlisting}

In Appending 2 we will show our best attempt to encode this graph example in Java, Rust and Scala.
In short, we discovered...

\subsection{Loading libraries}

Most languages have a standard library.
The standard library have two goals:
\begin{itemize}
\item Be a set of useful features for programmers to use.
\item Be a starting point for third party libraries.
\end{itemize}
While the first point is quite obvious, the second one is a little surprising: third party libraries will communicate with
the user code mostly by using standard library types:
\Q@String@s, \Q@Collection@s and if we are in a pure OO language,
also \Q@Boolean@s, \Q@Integer@s, \Q@Double@s and so on.
The number of types involved in just take input and produce output is much larger then one could expect, since all kind of errors need
to be considered too, and if the language support reflection, all those classes and their errors may end up being transitively required.
The goal of the standard library is.
For example, assume a simple library taking in input a \Q@String@
and producing a \Q@String@: What are its dependencies?

\Q@String@ has an \Q@isEmpty():Boolean@ method, 
the \Q@Boolean@ has a \Q@toString():String@ method, a circular dependency.
\Q@String@ has a \Q@size():Integer@ method, and
a \Q@getChar(Integer pos):Char@ method.
another typical method of strings is @split(String regex):ListString@,
returning a list, that extends some general collection, and so on.
If reflection is not implemented with Mirrors[],
any of those object would have a \Q@class():Class@ method,
and \Q@Class@ would have methods to connect with most other reflection classes.

Those dependencies are usually not a problem because we assume the standard library to be fixed and always available.
Or, if you prefer, we are forced to design programming languages together with their standard library because those dependencies are too hard to manage directly.

With Redirect we can get free from this chain, and every third party library can just declare a the set of dependencies that are really needed.
A single redirect application can then ``load'' the library in the current scope, where a variation of the standard library is available, but not necessarily exactly the library used to develop such third party library.

For example, a library may only need pass indexes around, without directly
doing arithmetic, and may never use ..

\begin{lstlisting}
library={//this code is fully self contained
  N={}
  C={ Bool equal(C x)}
  S={N size(), C getChar(N index)}
  S myLibFunction(S x)=....
  Map=interface{
    S string()
    C char()
    N integer()
    }
  }
...
Int={..}
Char={..}
String={..}
Map=interface{String string() ... }
LoadedLib=library<Map=Map>
\end{lstlisting}
Since the code of $t$ is self contained (do not refer to any class in the outer program, it is possible to just ship it independently of the standard library of the target.
The code of $library$ can be typechecked once, and then
any other program may load it as shown.
Any program defining a \Q@Map@ interface with some types that we expect libraries to rely upon can be used in conjunction with $library$.
In this example, if \Q@Char@ is not a valid structural subtype of \Q@C@,
the redirect would fail with a meaningful error message.

By the stability theorem, we get a good formal characterization of what are the acceptable shapes of the program so that the redirect would succeed.
However, even if the program do not match the expectations of the library,
it could be possible to tweak the code to make it work.
...
