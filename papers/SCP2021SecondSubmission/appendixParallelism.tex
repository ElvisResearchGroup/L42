\section{Safe Parallelism in 42, without destructive reads}
\label{s:parallelism}

This section discuss the relation between our work and the 42 support for safe unobservable parallelism.

From its inception, the work on 42 tried to build a system supporting flexible, elegant and soundly unobservable parallelism without the need of relying on destructive reads.
42 uses  the same kind of reference and object capabilities used by Pony and Gordon, but while they allow for \emph{true} isolated fields with destructive reads, 42 avoids them.
That is: in Pony/Gordon we can easy define a class boxing a capsule references as follows
\begin{lstlisting}[morekeywords={iso}]
class Box{ //pseudocode for clarity
  iso Foo foo;//isolated field
  Box(iso Foo foo){this.foo=foo;}//initialized with an iso reference
  mut method iso Foo getFoo(){return this.foo;}//destructive read here
  }//the getter mutates the box (destructive read) and returns the stored iso reference
//usage
iso Foo myFoo = ..
mut Box box = new Box(myFoo);//the type system ensures this is the only usage of 'myFoo'
//box is now a mut object and can be freely passed around and aliased
iso Foo foo1 = box.getFoo();//this foo1 can now be used for parallelism
iso Foo foo2 = box.getFoo();//either foo2==null or exception is thrown here
\end{lstlisting}
42 does not support destructive reads. Thus, there is no way to declare method \Q@getFoo()@.
This also means that after putting encapsulated data into a field of any kind, it is not possible to extract the data back as encapsulated.
Indeed, if 42 where to somehow allow the initialization of \Q@foo1@ we would then have two ways to reach the 'encapsulated' data, thus breaking the encapsulation guarantee.
Thus, the 42 research has explored various ways to access those fields, that are initialized with encapsulated data but can not soundly release the data as capsule.
From a research perspective, it was interesting to discover many different access patterns that allowed to preserve some encapsulation properties but not others.
On the other side, it is worth of notice that exploring those different kind of encapsulated data does not impact the way capsule references are threated when passed around as method parameters or saved in local variables, nor their parallemism properties.

Indeed, in 42, as in Pony or Gordon, we can make fork-join where the parallel branches only use \Q@capsule@ variables.
The difference is that in 42 those variables will not be able to come from reading encapsulated fields of \Q@mut@ objects.
Note that this is because, as a research question, the 42 developers are trying to understand how far they can go without resorting to destructive reads.
They could easely add a primitive datatype working as a \emph{consumable} mutable box storing a true capsule. Such a primitive data type would behave exactly like the \Q@Box@ type above.
Then, a 42 user could chose to use those boxes to recover all the expressive power (and risks) of destructive reads.
Doing so would however make it much harder to claim that 42 supports expressive automatic parallelism without the need of destructive reads, since destructive reads would now be avaible on demand.

\subsection*{Parallel patterns in 42}

The current version of 42 relies on the annotation \Q|$@$Cache.ForkJoin| and the \Q@Data@ decorator to activate varius forms of (unobservable) automatic parallelism. We show those forms below.
Of course, the invariant protocol described in this paper does work also in the context of \Q|$@$Cache.ForkJoin|, thus it is impossible to observe a broken invariant, even when using parallelism. This is a direct result of the fact that parallelism in 42 is unobservable: the semantic of parallel code is equivalent to the semantic of the corresponding sequential code, only the performance change.


\subsection*{Non-Mutable computation}
This is the simplest 42 parallel pattern.
%This is more expressive than Wcode(Cache.Eager) since it allows us to run parallel code on Wcode(read) references of mutable objects.
Consider the following code in 42:
\begin{lstlisting}[deletekeywords=label]
Example = Data:{
  $@$Cache.ForkJoin class method 
  capsule D foo(capsule A a, capsule B b, imm C c, read D d) = (
    mut A a0=a.op(d)
    mut B b0=b.op(c)
    mut C c0=c.op(d)
    a0.and(b0).and(c0)
    )
  }
\end{lstlisting}
The initialization expressions for \Q@a0@, \Q@b0@, and 
\Q@c0@ are run in parallel, and the final expression is run only when 
all of the initialization expressions are completed.
The method itself can take any kind of parameters, and they can all be used in the final expression, but the initialization expressions need to fit one of the recognized
safe parallel patterns. In \emph{non-mutable computation} only \Q@read@,\Q@capsule@ and \Q@imm@ parameters can be used in the initialization expressions.
The name \emph{non-mutable computation} comes from the fact that, even if the capsules can indeed be mutated, nothing that is visible outside of the fork-join can be mutated while the forkjoin is open; thus parallelism is unobservable.

More in general, \Q|$@$Cache.ForkJoin| works only on methods whose body is exactly a round parenthesis block, with some local variable initialization expressions and a conclusive expression.
Thus, fork-join methods will follow this specific syntactic pattern:
\begin{lstlisting}[deletekeywords=label]
$@$Cache.ForkJoin $\mdf$ method $T_0$ $m$($T_1 x_1,\ldots,T_n x_n$) = (
  $T'_0$ $x_0$=$e_0$
  $\ldots$
  $T'_k$ $x_k$=$e_k$
  $e$
  )
\end{lstlisting}
where the varius expressions $e_0...e_k$ are execuited in parallel, and the final expression $e$ is executed after all of $e_0...e_k$.
Different forms of parallelism impose different requirements on the free variables that expressions $e_0...e_k$ can use.

Some readers find suprising that in the \emph{non-mutable computation} pattern \Q@read@ references can be freely used, since there can be \Q@mut@ references to those same objects.
However, those \Q@mut@ references are all unreachable from inside our fork-join. This is because the whole
mutROG of all the \Q@capsule@ rerefences is encapsulated and other \Q@mut@ references are not allowed.
%How the mutROG of \Q@capsule@ fields can be shared is not important here because in 42 there is no way to go from a \Q@capsule@ field back to a \Q@capsule@ reference.

This form of parallelism is the only one proposed by Gordon; it is very expressive in their setting with destructive reads, but it is quite limited in 42. However 42 offers other forms of parallelism, as shown below.

%The following inductive reasoning can clarify the confusion:
%Base case: we are at top level in the fork-join hierarky, so no code is executing in parallem until we start our fork join.
%Inductive case: we are executing any number of nested fork-joins, but by inductive hypothesis the 
%All the branches of our fork join does not contain any \Q@mut@ free variables.
 %The while mutROG of \Q@capsule@ free variables is encapsulated, thus the
%\Q@read@ variables can not point into those.


\subsection*{Single-Mutable computation}
In this pattern, a single initialization expression can use any kind of parameter, while the other ones can not 
use \Q@mut@, \Q@lent@ (a variation of \Q@mut@ present in 42) or \Q@read@ parameters.
This pattern allows the single initialization expression that can use \Q@mut@ to recursively explore a complex mutable data structure and to update immutable elements arbitrarily nested inside of it.
Consider for example this code computing in parallel 
new immutable string values for all of
the entries in a mutable list:

\begin{lstlisting}[deletekeywords=label]
UpdateList = Data:{
  class method S map(S that) = that++that//could be any user defined code
  class method Void of(mut S.List that) = this.of(current=0I,data=that)  
  class method Void of(I current, mut S.List data) = (
    if current<data.size() 
      this.of(current=current,elem=data.val(current),data=data)
    )
  $@$Cache.ForkJoin class method Void of(I current, S elem, mut S.List data) =(
    S newElem=this.map(elem)
    this.of(current=current+1I,data=data)
    data.set(current,val=newElem)
    )
  }
//usage
mut S.List data = S.List[S"a";S"b";S"c";S"d";S"e";]
UpdateList.of(data)
Debug(data)//["aa"; "bb"; "cc"; "dd"; "ee"]
\end{lstlisting}
As you can see, we do not need to copy the whole list. We can update the elements in place one by one.
If the operation \Q@map(that)@ is complex enough, running it in parallel could be beneficial.
You can equivalently read this code either as a fork join or as sending the computation \Q@this.map(elem)@ to be run on a separate worker, while
the computation     \Q@this.of(current=current+1I,data=data)@ is executed on the current thread.
Indeed, to implement a fork join it is always more effienct to run the last branch in the current thread.
As you can see, it is trivial to adapt that code to explore other kinds of collections, like for example a binary tree.
The visit of the tree will be performed recursivelly but sequentially in the current thread, and workers will be spowned at all recursive layers and their results will be composed at the end of the recursion.


Those two forms of parallelism where already possible on the 42 model before our work on invariants and our \Q@rep@ fields.
We think that it is pretty impressive that this kind of parallelism can be obtained without destructive reads.
Building on top of our the \Q@rep@ fields and on the concept of \emph{rep mutators}, a new form of parallel fork-join computation was recently added:
\emph{This-Mutable computation}.

\subsection*{This-Mutable computation}
In this pattern, the \Q@this@ variable is considered specially.
The method must be declared \Q@mut@, and the 
initialization expressions can not
use \Q@mut@, \Q@lent@ or \Q@read@ parameters.
However, the \Q@mut@ parameter \Q@this@ can be used to directly call
rep mutator methods (marked by \Q|$@$Cache.Clear| in the 42 syntax).
Since a rep mutator can mutate the reachable object graph of a \Q@rep@ field, and the mutROG from different \Q@rep@ fields is disjoint, 
different initialization expressions must use rep mutators updating different \Q@rep@ fields.
In this way, 42 can express parallel computation processing arbitrary complex mutable objects inside well encapsulated data structures.
Consider the following example, where instances of \Q@Foo@ could be arbitrarily complex; containing complex (possibly circular) graphs of mutable objects.
\begin{lstlisting}[deletekeywords=label]
Foo=Data:{.. /*mut method Void op(I a, S b)*/ ..}

Tree={interface [HasToS]    mut method Void op(I a, S b) }

Node = Data:{[Tree] 
  capsule Tree left, capsule Tree right //the 42 syntax uses 'capsule' instead of 'rep'
  $@$Cache.ForkJoin
  mut method Void op(I a, S b) = (
    unused1=this.leftOp(a=a,b=b)
    unused2=this.rightOp(a=a,b=b)
    void
    )
  $@$Cache.Clear
  class method Void leftOp(mut Tree left,I a, S b) = left.op(a=a,b=b)
  $@$Cache.Clear
  class method Void rightOp(mut Tree right,I a, S b) = right.op(a=a,b=b)
  }
Leaf = Data:{[Tree]
  capsule Foo label
  $@$Cache.Clear
  class method Void op(mut Foo label,I a, S b) = label.op(a=a,b=b)
  }
//usage
mut Tree top = Node(
  left=Node(
    left=Leaf(label=..)
    right=Leaf(label=..)
    )
  right=Node(
    left=Leaf(label=..)
    right=Leaf(label=..)
    )
  )
top.op(a=15I b=S"hello")
\end{lstlisting}

This pattern relies on the fact that using \Q@rep@ fields we can define arbitrary complex data structures composed of disjointed mutable object graphs.
Note that \Q@read@ aliases to parts of the data structure can be visible outside.
This is safe since we can not access them when the forkjoin is open. The declarations can not use \Q@read@ parameters.

\subsection*{Non fork-join parallelism in 42}
42 also supports eager caching using the annotation \Q|$@$Cache.Eager|.
This form of parallelism is limited to start only from fully immutable data.
This annotation can be used only on no args methods of objects that are born immutable.
Parallel workers are used to eagerly compute the result of those methods and cache the result in the object.
This form of parallelism allows to express computation in a very declarative style, but it does not interact with capsule references, our rep fields or rep mutators, so an in-dept discussion of \Q|$@$Cache.Eager| is out of scope.

\subsection*{Parallelism in older versions of the 42 type system}
42 has been undergoing many changes across the years.
The earlier version of the 42 type system was based on [??],
where you could see many more modifiers, including
\Q@fresh@ and \Q@baloon@.
In that version \Q@fresh@ is similar to the current 42 \Q@capsule@ references and \Q@baloon@ is similar to one of the various kinds of  \Q@capsule@ fields available in 42 before our work on \Q@rep@ fields.
That work was then summarized in a short 6 pages paper [??], that
refers to both \Q@fresh@ and \Q@baloon@ as \Q@baloon@; using the different context (reference, local variable, field) to make change the meaning of the single keword \Q@baloon@, to cover both roles of \Q@fresh@ and \Q@baloon@ in [..].
Even in those earlier works there was no way to recover a \Q@fresh@ from a \Q@baloon@, and a \Q@baloon@ was basically a kind of encapsulated reference that allowed other restricted kinds of references (\Q@external@ and \Q@external readonly@) to point inside of it. Looking back to those earlier work it is clear to us that the current 42 type system is much more minimal and elegant.
Those works suggested interesting forms of parallelism where the type system could cooperate with a few efficent run time pointer equality checks to decide what to run in parallel.
Such a direction has not been explored further and it is not currently present in modern versions of 42.
