\section{Introduction}
\label{s:intro}
%\newpage
%\LINE
Representation invariants (sometimes called class invariants, object invariants, and type refinements) are
a useful concept when reasoning about software correctness, particularly with OO (Object Oriented) languages. Such invariants are predicates on the state of an object and its ROG (Reachable Object Graph).
They can be presented as documentation, checked as part of static verification, or, as we do in this paper, monitored for violations using runtime verification.
In our system, a class specifies its invariant by defining a method called \Q@invariant()@
that returns a boolean.
We say that an object's invariant holds when its \Q@invariant()@ method would return \Q@true@.\footnote{We do this (as in Dafny~\cite{DBLP:conf/sigada/Leino12}) to minimise the special treatment of invariants, whereas other approaches often treat invariants as a special annotation with its own syntax.}
In a purely functional setting, sound runtime checking is trivial: the runtime simply needs to check the invariant each time a value/object is created (or in the case of refinement types, converted to such a type).

In an impure setting, like most OO languages, 
operations on data structures are often implemented as complex sequences of mutations,
where the invariant is temporarily broken.
To support this behaviour, most invariant protocols present in the literature allow invariants to be broken and observed broken.
The two main forms of invariant protocols are \emph{visible state semantics} \cite{Meyer:1988:OSC:534929} and the \emph{Pack-Unpack/Boogie methodology}~\cite{DBLP:journals/jot/BarnettDFLS04}.
In visible state semantics, invariants can be broken when a method on the object is active (that is, currently executing).
Some interpretations of the visible state are more permissive, requiring the invariants of receivers to hold only before and after every public method call, and after constructors. 
In the pack-unpack approach, objects are either in a `packed' or `unpacked' state, 
the invariant of `packed' objects must hold, whereas unpacked objects can be broken.
To complicate matters further, OO languages often permit rampant aliasing of mutable state, thus any mutation may inadvertently break the invariant of an arbitrary object.
To be feasible, most runtime verification systems only require that an invariant holds of an object immediately after creation, a field update, or a `pack` operation; the invariant of an object may later be broken whenever an arbitrary object in its ROG is mutated.

%------------
In this paper we propose a much stricter invariant protocol: at all times, the invariant of every object involved in execution must hold; thus they can be broken when the object is not (currently) involved in execution. 
An object is \emph{involved in execution} when it is in the ROG of any of the objects mentioned in the method call, field access, or field update that is about to be reduced; we state this more formally later in the paper.

%Our strict invariant protocol clearly supports easier reasoning; however 
Our strict protocol supports easier reasoning: an object can never be observed broken. However 
at first glance it may look overly restrictive, preventing useful program behaviour.
Consider the iconic example of a \Q@Range@ class, with a \Q@min@ and \Q@max@
value, where the invariant requires that \Q@min<max@:
% ISAAC: I changed the example to not use getters and setters as our latter examples often don't and it dosn't add anything extra; in addition, this may fall foul of some visibile state semantics of the setters are considered public.
\begin{lstlisting}
class Range{ 
  private field min; private field max;
  method invariant(){ return min<max; }
  method set(min, max){
    if(min>=max){ throw new Error(/**/); }
    this.min = min;
    this.max = max;
  }
}
\end{lstlisting}
In this example we omit types to focus on the runtime semantics.
The code of \Q@set@ does not violate visible state semantics:
\Q@this.min@ \Q@=@ \Q@min@ may temporarily break the invariant of \Q!this!, however it will be fixed after executing \Q@this.max@ \Q@=@ \Q@max@. Visible state allows such temporary breaking of invariants since we are inside a method on \Q!this!, and by the time it returns, the invariant will be re-established.
However, if \Q@min@ is $\geq$ \Q@this.max@, \Q@set@ will violate our stricter approach. The execution of
\Q@this.min@ \Q@=@ \Q@min@ will break the invariant of \Q@this@ and \Q@this.max@ \Q@=@ \Q@max@ would then involve a broken object. If we were to inject a call
\Q@Do.stuff(this);@ between the two field updates, arbitrary user code could observe a broken object; 
  adding such a call is however allowed by visible state semantics.

In this paper, we illustrate a \emph{box pattern}, where we can provide a modified
\Q@Range@ class with the desired client interface, while respecting the principles of our strict protocol:
\begin{lstlisting}
class BoxRange{//no invariant in BoxRange
  field min; field max;
  BoxRange(min, max){ this.set(min, max); }
  method Void set(min, max){
    if(min>=max){ throw new Error(/**/); }
    this.min = min; this.max = max;
  }
}
class Range{
  private field box; //box contains a BoxRange
  Range(min, max){ this.box = new BoxRange(min, max); }
  method invariant(){ return this.box.min < this.box.max; }
  method set(min, max){ return this.box.set(min,max); }
}
\end{lstlisting}
The code of \Q@Range.set(min,max)@ does not violate our protocol.
% since \Q@this@ is not in the ROG of \Q@this.box@, \Q!min!, or \Q!max!. 
The call to
\Q@BoxRange.set(min,max)@ works in a context where the \Q@Range@ object is
unreachable, and thus not involved in execution.
That is, the \Q@Range@ object is not in the ROG of the receiver or the parameters of \Q@BoxRange.set(min,max)@.
 Thus \Q@Range.set(min,max)@ can temporarily break the \Q@Range@'s invariant.
By using the \Q@box@ field as an extra level of indirection, we restrict the set of objects involved in execution while the state of the object \Q@Range@ is modified.%
\footnote{Due to its simplicity and versatility, we do not claim this pattern to be a contribution of our work, as we expect others to have used it before. We have however not been able to find it referenced with a specific name in the literature, though technically speaking, it is a simplification of the Decorator pattern, but with a different goal.
While in very specific situations the overhead of creating such additional box object may be unacceptable, we designed our work for environments where such fine performance differences are negligible.
Also note that many VMs and compilers can optimize away wrapper objects in many circumstances.~\cite{Bolz:2011:ARP:1929501.1929508}
This is even more applicable in languages with inlined structs, like C++ or C\#.
}
With appropriate type annotations, the code of \Q@Range@ and \Q@BoxRange@ is accepted as correct by our system: no matter how \Q@Range@ objects are used, a broken \Q@Range@ object will never be involved in execution.
In particular, this has the advantage of clearly delineating when the \Q!Range! invariant needs to hold: after every method modifying the \Q!box!. In contrast, the \Q!BoxRange! cannot see the enclosing \Q!Range!, thus its code cannot assume that the invariant holds.

\subheading{Contributions}
Invariant protocols allow for objects to make necessary changes that might make their invariant temporarily broken.
In visible state semantics any object that has an active method call anywhere on the call stacks is potentially invalid;
arguably not a very useful guarantee as observed by
Gopinathan \etal's work.~\cite{Gopinathan:2008:RMO:1483018.1483028}
Approaches such as \textit{pack/unpack}~\cite{DBLP:journals/jot/BarnettDFLS04} 
represent potentially invalid objects in the type system; this
encumbers the type system and the syntax with features whose only purpose is to distinguish objects with broken invariants. %, while (at least in the case of Spec\#) still not soundly supporting I/O and exceptions.
The core insight behind our work 
is that we can use a small number of decorator-like design patterns to avoid exposing those potentially invalid objects 
in the first place, thus avoiding the need for representing them at the type level.

%In this paper we present  
%a general purpose language
% that does not require any \emph{invariant-specific} language mechanisms
% but instead we show how a clever use of capabilities
% is sufficient to capture invariant guarantees
% that are comparable to the state of the art object-oriented verification systems.

%Reasoning about class invariants in object-oriented verification needs to allow for objects to make necessary changes that might make their invariant temporarily broken. 
%The state of the art ranges from \textit{visible state semantics} that makes \textit{any} object that has an active method call anywhere on the call stacks potentially \textit{invalid} --- arguably not a very useful guarantee as observed by Gopinathan et al.~\cite{Gopinathan}.
%On the other hand, approaches such as \textit{pack/unpack}~\cite{SpecSharp} modify the type system with features that serve no general purpose other than help expose the semantics behind broken object invariants.
%The core insight behind our work is that we can design a general purpose language as presented in this paper that does not require any \emph{invariant-specific} language mechanisms but instead we show how a clever use of capabilities and small number of design patterns (Decorator-like \textit{Box} and \textit{Transformer}) is sufficient to capture invariant guarantees that are comparable to the state of the art object-oriented verification systems.

In this paper, we discuss how to combine runtime checks and capabilities
to soundly enforce our strict invariant protocol.
Our solution only requires 
that all code is well-typed, and works in the presence of mutation, I/O, non-determinism, and exceptions, all under an open world assumption.

We formalise and prove our approach sound, and have fully implemented our protocol in L42\footnote{
Our implementation is implemented by checking that a given class conforms to our protocol, and injecting invariant checks in the appropriate places.
An anonymised version of L42, supporting the protocol described in this paper, together with the full code of our case studies, is available at \url{http://l42.is/InvariantArtifact.zip}. %We believe it would be easy to port our work on Pony and Gordon \etal's language.
}, and used it to run our various case studies.
It is important to note that unlike most prior work, we soundly handle catching of invariant failures and I/O.
%In our case study we show that
%we can still encode most of the examples explored in ~\cite{???} (including for example mutable collections of immutable objects) whilst having a significantly lower annotation-burden.
%--I think we can avoid this to save space
%Section \ref{s:TMsAndOCs} explains the pre-existing \emph{type modifier} features we use for this work.
%Section \ref{s:protocol} explains the details of our invariant protocol, and Section \ref{s:formalism} formalises a language enforcing this protocol.
%Sections \ref{s:immutable} and \ref{s:encapsulated} explain and motivate how our protocol can handle invariants over immutable and encapsulated mutable data, respectively.
%Section \ref{s:case-study} presents our GUI case study and compares it against visible state semantics and Spec\#: they performed 5 orders of magnitude more invariant checks, and required 60\% more annotations, respectively.
%Sections \ref{s:related} and \ref{s:conclusion} provide related work and conclusions.

The remainder of this paper proceeds as follows:
\begin{itemize}
	\item Section \ref{s:TMsAndOCs} explains background information necessary to understand our approach
	\item Section \ref{s:protocol} fully explains our novel invariant protocol, and our novel field type for mutable data
	\item Section \ref{s:immutable} demonstrates why the soundness of our protocol depends on the properties of the L42 (and similar) type systems
	\item Section \ref{s:formalism} formalises our runtime invariant checking, and what it means it to soundly enforce our invariant protocol
	\item Section \ref{s:case-studyAll} many case studies, showing that our protocol is more succinct than the pack/unpack approach and much more efficient then the visible state semantic
	\item Section \ref{s:patterns} show how our approach does not hamper expressiveness, by showing programming patterns that can be used to perform batch mutation operations with a single invariant check, and how the state of a `broken' object can be safely passed around
	\item Section \ref{s:L42} summarises how we have implemented our protocol in L42
	\item Section \ref{s:related} presents related work, and Section \ref{s:conclusion} concludes
	\item Section \ref{s:proof} formally specifies the properties a type system needs to guarantee, and proves the formalism in Section \ref{s:formalism} sound
	\item Section \ref{s:typesystem} presents a simple L42-inspired type system and proves that it satisfies the requirements in Section \ref{s:proof}	
\end{itemize}

%We proposed our approach for integration in the L42 language, and after minor reworking it has been accepted, and the current version of L42 integrated our 
%invariant checking protocol in a cohesive way with the support for caching and parallelism.
%Section ~\ref{s:integration} discusses the details of this integration.

% In Appendix \ref{s:runtime-verification}, we discuss more related work on runtime verification.


%you see this already later on... I wanted to avoid repating it
%to perform batch operations with a single invariant check, as well as how the state of `broken' objects can be passed around.}	
%http://www.cs.cmu.edu/~NatProg/papers/p496-coblenz-Glacier-ICSE-2017.pdf
