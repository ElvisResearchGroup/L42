We would like to thank all the reviewers for their insightful reviews!

In this response document we first summarise our major changes, then we list response points specific to individual reviewers.

1.  We added Appendix B, containing a type system for our formal language and a proof showing that such type system
supports all the requirements of Appendix A. This is a concrete example showing that the requirements of Appendix A
can be satisfied by a practical type system. The presented type system is a simplification of the current L42 type system.
We thank the reviewers for pushing us in this direction. This makes our work much stronger from a formal perspective.

Our previouse representation of the typing judgement was confusing. In this new version we instead instrument the
reduction by encodeing reference capabilities directly in the reduction semantics. This also required us to add an explicit "as" expression.

2. We changed our terminology:
Before we used "capsule fields" and "capsule references" and we spent quite some text explaining how 
"capsule fields" can not easily be used to produce a capsule reference. Given the reaction of the reviewers, this terminology must have been intrinsically confusing.
For example reviewer 1 asked "why capsule fields never hold capsule references"
Conceptually, the answer would have been "because that is how we are defining it to be".
That is, in L42 the keyword "capsule" means different things when applied to fields or local variables.
This is similar to what happens in Java with the keyword "final", that means different things when it is applied to classes, methods and fields.
Since the language we use in our presentation is already quite far from the concrete 42 syntax, there is no need to stick to the confusing 42 terminology and we have now changed our terminology.
We now refer to "rep fields" for what we used to call "capsule fields",
while "capsule references" and "capsule local variables" are the same as before.
That is, we havn't changed our approach, we have just changed how we explain it.

3. We added Appendix C, containing an in depth explanation of parallelism in L42 and on how the restrictions and relaxations
 about capsule references and capsule/rep fields do not prevent safe parallelism.
We are unsure if this Appendix should stay in the final version of this work. 
We inserted it because reviewers were concerned that our novel field type was causing difficulties in connection with safe parallelism.
To properly clarify that this was not the case, we had to explain how safe parallelism works in L42, and that required quite some text.

About the 4 main points listed at the top of the reviews
1. "better introduction of the various concepts used"
Following the detaild comments of the revs we have improved the explanation in many points, in particular we have clarified much more how the try-catch and the corresponding restrictions work. The change in terminology (rep fields) is also designed to make the explanation more clear.

2. "improve the formalisation, and make it more precise, and more self-contained"
This is where we spent most of our effort. Appendix B now makes our work self contained.

3. "improve discussion of related work"
We integrated the papers suggested by the reviewers in our discussion, we recognize there are many more works we could discuss, and we are eager to lissen to more suggestions from the revs; however our work already has 83 references.
This is because our work sits at the border between many areas: runtime verification, static verification, object capabilities, and reference capabilities.
 
4. they also all have suggestions to improve the presentation of the paper.
We have done our best to address all the minor comments.

We are worried that reviewer 1 and reviewer 2 seems to ponder that we may be willingly making an unsound system:
>>(reviewer 1) "It seems very dangerous to trust the developer to correctly mark which methods may invalidate the invariant..."
>>(reviewer 2) "Option 2: the paper intends to update enclosing variables from inside the try block, opening up the unsoundness I noted above."
We are very confused about those remarks. We believe our work to be sound. 
We are not assuming a "cooperative programmer" that does not exploit holes.
We belive that there is no hole to exploit.
Where in the paper have we suggested otherwise?


---------
Answers to Reviewer #1: 

>> accidental sharing of the `BoxRange` object could still be a problem[..] It could perhaps be implemented as[..] Or maybe there is [..] ownership [..]?
>> Would aliasing of `this` be  a problem though? Is it a problem in the current system?
Aliasing of the object with invariant is not a problem, our system is sound.
Using the Box pattern, aliasing of the internal encapsulated object (as mut) is only possible when the outer object (the one with the invariant) is not visible.
Aliasing of the internal object (as read) is possible everywhere, but if the outer object is visible, then the constraints induced by its invariant will be enforced on the internal object too. See section 7, the "subInvariant pattern" and when the "subInvariant" method can be observed as false.


>> How big is the actual gain between having `Range` and  `BoxRange` vs. having `Range` and `unpacked Range`?
In a language design only focusing on invariant support, having a language level 'unpacked' keyword feels tempting.
But, in a fully fledged language it would be yet another language feature to consider, and it would interact with all the other features.
Moreover, adding this feature seams to invite even more features into the language:
'(un)packed' is a kind of type state, so we could immagine more states objects could be in, and we may even want to define a metalanguage to support creation of user defined typestates.
In this work we aimed at a minimal type system: when a feature could be expressed as a pattern we do not lift it to the type level.
It would also be tempting to lift the 'sub-invariant' pattern to the type/language level.
Having many patterns lifted would expose them more to the programmers, and encourage their use.
However, we think that there is greater value in minimality.
Indeed, using patterns to express concepts also shows programmers how using patterns can help them encode all kinds of features they need, instead of having to rely on a limited set of language features.
>> I am not sure it would be a bad thing with type-system support for reasoning about when an object is in an inconsistent state.
>> In some sense, you are doing the same thing with the box-pattern: a `FooBox` is (the representation of) a `Foo` whose invariant may be temporarily broken.
>> If this distinction is necessary for reasoning about programs, I would think language support is a *good* thing.
>> There is the static inner class I thought about before
As reviwer 1 notices, the box pattern is more flexible than a language supported 'unpacked' feature:
by making the `Boxed` type private we can avoid client code every seeing said type.
In contrast, if 'unpacked'/'invalid' was a type system feature than any user defined type would intrisically expose the existence of both variants in the public API.
>> A nice thing  about the latter approach is that I know that I am indeed  handling a `Range` and that there are
>> no discrepancies between the classes representing the different states (packed and  unpacked) of the object.
While sometimes a guarancee of no discrepancy would be nice, one of the main pillars of OO is the concept of encapsulation, so that for example methods looking like getters and setters do not need to directly correspond to concrete fields in the object.
Code enforcing their encapsulation boundaries would likelly not want to expose their true internal state as a Box object, they would likelly want to expose a wrapper whose
API would be stable, while keeping the internal representation open to change for future maintanence.

We included some of these considerations in the sub-invariant pattern subsection.


>>there is a story for verification in a concurrent setting as well, since that is  the context for a lot of the work on reference capabilities.
There could be interesting patterns merging parallelism and invariant based verification. We would like to explore this in future work.
Right now, L42 does support unobservable parallelism, so invariants are already enforced in a parallel setting.
We would not consider our work sound if this was not the case. To the best of our understanding, there is absolutelly no way to observe broken invariants in L42, even when using all of the full features of L42 including parallelism and interaction with Java code.
The details on how the L42 runtime is protected agains malicious Java code (and C code loaded from that Java) are outside of the scope of this work.

>>"This is different with respect to many other approaches"  -- Which approaches are these? 
We refer to former work about aliasing and immutablity before Gordon, so fractional permissions, universe types, islands, raw and cooked types, external uniqueness and so on.


>> I don't know what you mean by "the result [of accessing a capsule variable] will not be a capsule reference".[..] it seems like enforcing affineness is more restrictive.[..]
>>  Since Gordon et al. use flow sensitive typing,[..](with some restrictions on parameter and return types).
>>  Note that it is always possible to permanently lose the capsule  capability following an arbitrary operation requiring a `mut` capability.
Yes, that is exacly what we mean: while in 42 any way to read a capsule local variable gives you a capsule reference, in the other approaches the situation is more involved
 (in order to be more expressive) and sometime reading the capsule variable can give you some other kind of reference, as you said.
Also, you suggests that the other approaches also have a much simpler core 
>>I believe some recovery/promotion is happening under the hood here
but they have some syntactic sugar or programming patterns lifted at the language level that make the surface language more usable.
Most likelly, those surface language features could all transparently keep working if (for example) Pony developers wanted to integrate our work in their language.


>> Then, at a later stage,[...]
We changed the sentence a little to clarify the distinction between instances and classes(lines 213-216 in the new version). Indeed, as discussed just two lines above
"any expression ([..]) of a mut type that has no mut or read free variables can be implicitly promoted to capsule",
thus no matter how complex and full of internally aliased mut objects a data structure is, if we can create an instance without using any mut, the whole instance can be a capsule/imm.


>> My intuition for capsule mutators is that they "pretend"
Yes, this is a nice summary, we reworked and included it in the text. 
The concept of the focus operator is very similar to the concept of view-point adaptation, we cited both works there.


>>`return counter++;` should not be in a comment? Or is it because the example is invalid? (there is also no `counter` field in the class)
Yes, the idea is that just doing counter++ would probably not be enough, that there should be some hypotetical unexistant 'backdoor syntax' around it.
We changed the example with some hypothetical syntax to make it more concrete.

>> [..]sequence expressions in the language[..]you could abuse the fact that method calls have multiple arguments[..]  Featherweight Java also has no sequencing[..]
Yes, in minimalistic FJ inspired languages we often use methods with multiple arguments to encode sequencing.


>> "Our setting does ensure termination of the invariant of[..]so, this is a less  interesting property (albeit not uninteresting) in my opinion.
Yes, solving the halting problem would have been much more interesting but would deserve a paper of its own.
Here, more modesly, we just show that it terminates because of determinism and the fact it has already terminated.
We clarified this in a footnote.


>>[..]Would you get the same pathological numbers if you removed the checks after pure methods? In that case, this experiment is a point in
>> favour of tracking (im)mutability, but not necessarily the whole reference capability approach.
Yes, any kind of purity verification solves the exponential explosion, as you can see later for Spec# doing the same exact number of checks of 42.
The RC approach is a kind of purity verification that is so much simpler to use than the Spec# approach.

>> I think this whole section [..] provides a strong argument for using reference capabilities in a setting outside of reasoning about parallelism and data races
Thanks for this consideration. We edited the paper to made this point explicit.


>> "Note how in the lambda in connectWith(other,g), [..]" -- I think this is the same reasoning as in [..], but this is still an assumption.
>> It would be nice with a clarification here or earlier.
Yes, our informal handling of lambdas was very confusing. Now we discuss how differently typed lambdas can capture different kinds of variables.


>> "the overall idea is that an invariant is seen as a Void  cached [..]" -- I do not understand this at all. Why is the  invariant cached if it is always expected to return `true`?
>> The  only times it is interesting to calculate the invariant is when the value may have changed, and then having a cached value is not helpful.
We were unsure how much of this to insert in the paper since it is more of an extension of the current work: the point is that a generalisation of invariants is caching: 
-both caches and invariants have to be recomputed every time something in the ROG changes.
-So if we allow for a meaninful result instead of just 'true' we can use the exact same mechanism of invariants in order to encode caching.
We clarified this in the text.

---------
Answers to Reviewer #2: 

>>The formal calculus is pure
The calculus we have shown have a mutable memory, so it is not pure.

>>The paper gives a nice overview of [..] but stopping short of enough detail to fully explain the differences with regard to that work's notions of "capsule" (including earlier versions of L42)
>>I assumed this fork of L42 had given up on safe concurrency.
A full explanation of how L42 handles safe concurrency is out of the scope of the main text of the paper. 
In this revision, we added Appendix C, where we explain in the details how 42 handle unobservable parallelism.
We are open to either keep this appendix in the final version of the paper or to remove it if it is considered out of scope. 

>and thus obscuring what seem to be subtle reliance on a novel twist to L42's already-unique view of what capsule means
We hope that our new terminology (rep fields) clarifies how we do not rely on a L42 specific interpretation of capsulness and how any system supporting iso of Gordon/Pony would work be able to support our rep fields just fine.


>>I think this paper really needs to (1) include a concrete type system for the core language,
Done in Appendix B.

>> and (2) switch the informal "Java-like" intuitions in the start to be closer to the "Featherweight L42."
This would require a full rewriting of most of the informal text and would make our work much less valuable.
Our aim here is not to glorify 42 but to show a strategy that can be implemented by many different languages.
We hope that Section 8 and Appendix C can sufficiently show how our work looks like when it is applied to the exact environment of L42.

>>Without a formalization, of this change, it's not possible to fully understand the details of this paper's assumptions on capsule fields,[..]
>>I'd settle for a clear formalization of an example type system
We have now added Appendix B with a full type system and a proof showing that such type system satisfies our assumptions.
Appendix C would also help to clarify the former work on L42.


>>How is this possible without *major* modifications or restrictions?
>>the idea of shipping unique/isolated/capsule references to other threads
>>What's to stop someone from writing "read C x = foo.myCapsuleField; send(swap/dread(foo.myCapsuleField)); x.someField" and causing a data race?
Gordon and L42 aims to ensure the correctness of fork--joins. This means that we do not just 'ship work to another worker' and continue our work.
We rather control the scope of the whole fork--join, and what variables can be seen in the various branches. Please refer to Appendix C for more details.


>>Assumption 6 is underspecified partly because the paper remains vague on what exactly it means for an *expression* to have a capsule type.
>>This is a very subtle aspect of the work in this area. Gordon et al. avoided it entirely by formalizing something closer to a compiler IR.
>>Pony tackles it, but with a distinction in the type system between iso references that are stored and iso references that are "in flight" as part of an intermediate reduction (in the implementation; the formal calculus from AGERE'15 just uses recovery).
>>I'm not as familiar with L42's details, but this paper under review doesn't say anything about it. The paper spends a bit of time distinguishing the treatments of capsule locals and fields, but it's not clear what the in-flight use case means.

We hope this is going to be more clear with the new terminology: capsules variables are now only ever on the stack (and linear), and capsule references are "in-flight".
L42 does not support destructive reads. Adding such iso fields would not impact our approach in any way, more details are in Appendix B and C.

>>The paper's formal definition of encapsulated [..]  doesn't directly control heap structure.
Sorry for the confusion, "encapsulated" was only about "capsule references" on the stack.
We have altered the terminology of "rep fields" so that they are instead required to be "confined".
"confined" controls the heap structe of such a field, but it is a weaker guarantee then "encapsulated".
The full guarantee of "encapsulated" (that there is no other way to reach the non-immutable state) is not needed for invariant checking of "rep fields".


>>I'm not fundamentally opposed to the way this paper tries to make its results depend on the *semantics* of types rather than a specific type system;[..]
>>That is a great strategy that I'm tempted to try myself.
>> But I think because the paper hasn't actually applied this to a fully worked out semantics for types, the paper ends up at the very least not explaining its assumptions clearly,
>> and is possibly making mutually inconsistent assumptions. 
Yes, thanks to this comment instead of removing our assumptions and presenting a direct proof we have chosen to keep our assumptions based proof (Appendix A)
and to prove that those assumptions can be satisfied by a practical type system (Appendix B).


>>[..]So, I can see two possibilities for fixing this in a Java-like language:
>>Option 1: the references from outside the try-catch are also marked final.
Yes, this was already part of how 42 dealt with strong error safety fron its inception. We now make it clear also in the informal part.
The formal part is silent on this aspect since we do not encode local variable updates.
Formally, there is no need to encode them directly; it is well known that they can be emulated by a final local variable storing a box object with a single updatable field.


>>[..]there's *no* way to actually get data out of a try-catch directly
>>This is okay (clearly there's a way to get a value out of the try block!)
L42 is an expression based language, and try-catches are expressions (including an 'else part' similarly to python's try-catch-else statements).
Those expressions can return mutable data if such mutable data is created inside one of the catch bodies or if the try block completes successfully.
Even in a statement based language with traditional try-catch statements, we still could get data out of a try-catch by using a 'return'.
Again, try-catching on checked exceptions have none of these strong restrictions, they only apply to unchecked exceptions/errors.
We clarified this in the text.

>>For example, there are some similarities between this work and Müller et al.'s work [59]
We added a couple of sentences about the relation with [59].

>>Please explain in more detail why distinguishing checked vs. unchecked exceptions makes any difference in the restrictions on try blocks [..]
>>I see clearly how the restrictions in the formal calculus ensure no broken invariants are exposed,
>>but I cannot see any reason a checked exception would justify relaxing all of those restrictions.
>>""an exception thrown while an invariant doesn't hold has the potential to""
We make sure that checked exceptions never leak outside of a scope where an object with a possibly broken invariant is involved in execution: 
  - 42 constructors (called factories) can not leak checked exceptions. In a language where they could,
   we would need to guarantee that if the constructor leaks a checked exception, the created object is now unreachable.
  - setters, rep mutators and invariant methods can not throw checked exceptions. This is part of our explicit restrictions.
We clarified this in the paper.

>>"Pony does not guarantee that capsule fields contain a capsule reference at all times, as it provides non-destructive reads" 
>>This is either incorrect, or a highly misleading characterization of Pony's 'tag' RC. In Pony, reading an iso field non-destructively results in a 'tag' RC,
>> which may not be dereferenced, and therefore does not affect reachability paths.
Reviewer 1 pointed out about `iso!`, that is more similar to the case we are discussing. 
As we can see, more complex treatement of isolated fields can create confusion.

>>I find this terminology VERY confusing. A capsule mutator may read the field but not modify the capsule field? [..]le mutator."
We changed the text a little to make it more direct.

>>This bit about getting a read reference from reading a capsule field seems like it would allow violating capsule properties, unless L42 disallows storing read RCs in the heap?
Indeed, L42 allows storing read RCs only in very specific cases, by making wrapper objects that are born 'read'.
This is needed for the most important promotion rule of L42: any expression that does not use any "mut" variables but returns a "mut" is promoted to "capsule".
If read RCs could easily be saved in fields of mutable objects, this would be unsound.



---------
Answers to Reviewer #3:

>>The contribution are not clearly presented. It should be crystal clear what is the current contribution on top of the author's previous work.
We have now made a clear bullet point list of the contributions in each section at the end of the introduction.
More generally, all the content about invariants and their verification is a novel contribution, and all the content about reference capabilities and the way L42 works is not.

>>your formal language should be a bit more expressive and include the necessary expressions to cover your example 
>>(a bit of integer arithmetic and Boolean expressions beyond 'true' and 'false', conditionals and loops).
>>As such, your language is not even Turing-complete...
>>The formal part of the paper is not self-contained since neither it defines a turing-complete language, nor explains all the introduce notations and rules.
It is well known that FJ is turing complete, we have now provided a citation for this. You can easly encode lambda calculus in FJ.
We have now made our work self contained by adding Appendix B.


>>representation invariants are not limited to OO languages. They also exist in other languages in which they are usually called type invariants.
We added a mention of "refinment types" in the introduction.


>>Promotion and Recovery: at this point, it is not clear why this paragraph is relevant for this paper
We added a sentence explaining why this is important in our context.

>>Purity: what about (deterministic) I/O?
If you consider a single running process on any operating system, there is no such thing as deterministic IO:
what we read always depends on what other processes may be doing, what we write can still non deterministically go in error if the HD fails or it is full, for example.
It is possible to talk about deterministic I/O only if we assume control over the whole system, not just our individual process.

>>"it is guaranteed to be garbage collectable" --> "so it is out of scope when checking the invariant" 
>>being guaranteed to be garbage collectable does not offer any strong guarantee since one cannot be sure that the GC will indeed collect it
True, but actual collection is not important here. We are certain that it is never going to be involved in execution ever again.


>>The verifier available online [...] behaves differently: please add explanations here (and maybe as plain text and not as a footnote).
Unfortunately it was a long time ago so we do not rember the exact details, but the online verifier was accepting incorrect programs that the offline verifier was appropriately rejecting.

>>the 'moveTo' method seems to break the invariant (when not called from 'move'). Could you elaborate here?
It would not be able to break the invariant.
By perfoning a field update, it is triggering an invariant check.
Our system is not statically preventing attempted breaking of invariants, it just prevents broken objects from being involved in execution.
For example, in the GUI case study, nothing prevents us from attempting to move the widgets over eachother.
That attempt woulds cause an invariant error to raise (and we would not see any overlapping widgets).

>>if you use a list for 'parents' (instead having two fields), the invariant should verify that every child has 2 parents
We wrote this example to accommodate for any kind of family, we wish not to offend any alternative form of family organisation.


>>"a study [19] discovered that developers expect specification languages to follow the semantics of the underlying language"
>>written this way, this statement is just wrong even
>>though the two examples that follow are correct.
We are confused about this remark. We checked again [19] (now called [69]) and it seams to confirm our sentence (for example in pg10 Table 2).
We wonder if this is simply an English problem. What should we say to be more clear?

>>these references are quite old, see for instance
We added this citation

>>three times less annotation burden than [...] Spec# --> I do not agree with this conclusion if you only look at the number of annotations (which is the most meaningful measure IMO)
Some Spec# annotations are very involved, consider this single annotation:
"Owner.Same(Owner.ElementProxy(children), children)" In our understanding this is a single annotation, so when we count annotations, that whole string would count as 1.
On the other hand, when we count tokens, that string would count as 6.
(note how it does not count 13 since we do not count " ()[]{},;" as tokens)
That is why we considered tokens to be a better metric.

