\section{Patterns}
\label{s:patterns}
\lstset{morekeywords={invalid}}

In this section we show programming patterns that allow various kinds of invariants.
Our goal is not to verify existing code or patterns,
but to create a simple system that allows soundly verifying the correctness of data structures.
In particular, as we show, in order to use our approach to ensure invariants, one has to program in an uncommon and very defensive style.
%The trade off is that programmers may need to spend more effort in understanding their invariants and encoding them, but less effort in understanding the underlying programming language.

\subheading{The SubInvariant Pattern}
We showed how the box pattern can be used to write invariants over cyclic mutable object graphs, the latter also shows how a complex mutation can be done in an `atomic' way, with a single invariant check. However the box pattern is much more powerful.

 Suppose we want to pass a temporarily `broken' object to other code as well as perform multiple field updates with a single invariant check. 
Instead of adding new features to the language, like an \Q!invalid! modifier (denoting an object whose invariant does not need to hold), and an \Q!expose! statement like Spec\#, we can use a `box' class and a rep mutator to the same effect:
\begin{lstlisting}
interface Person{ mut method Bool accept(read Account a,read Transaction t); }
interface Transaction{ mut method List<Transfer> compute(); }
//Here List<T> represents a list of immutable Ts.
class Transfer{ Int money;
  method Void execute(mut AccountBox that){// Gain some money, or lose some money
    if(this.money>0){ that.income+=money; }
    else{ that.expenses -= money; }
  }
}
class AccountBox{
  UInt income=0; UInt expenses=0;
  read method Bool subInvariant(){ return this.income >= this.expenses; }
  //An `AccountBox' is like a `potentially invalid Account':
  //we may observe income >= expenses
}
class Account{
  rep AccountBox box; mut Person holder;
  read method Bool invariant(){ return this.box.subInvariant(); }
  // `h' could be aliased elsewhere in the program
  Account(mut Person h){ this.holder=h; this.box=new AccountBox(); }
  mut method Void transfer(mut Transaction ts){
    if(this.holder.accept(this, ts)){ this.transferInner(ts.compute()); }
  }
  // rep mutator, like an `expose(this)' statement
  private mut method Void transferInner(List<Transfer> ts){
     mut AccountBox b = this.box;
     for (Transfer t : ts) { t.execute(b); }
  }// check the invariant here
}
\end{lstlisting}
The idea here is that \Q!transfer(ts)! will first check to see if the account holder wishes to accept the transaction, it will then compute the full transaction (which could cache the result and/or do some I/O), and then execute each transfer in the transaction. We specifically want to allow an individual \Q!Transfer! to raise the \Q!expenses! field by more than the \Q!income!, however we don't want an entire \Q!Transaction! to do this. 
Our rep mutator (\Q!transferInner!) allows this by behaving like a Spec\# \Q!expose! block: during its body (the \Q!for! loop) we don't know or care if \Q!this.invariant()! is \Q!true!, but at the end it will be checked. For this to make sense, we make \Q!Transfer.execute! take an \Q!AccountBox! instead of an \Q!Account!: it cannot assume that the invariant of \Q!Account! holds, and it is allowed to modify the fields of \Q!that! without needing to check it. Though rep mutators can be used to perform batch operations like the above, they can only take immutable and capsule objects. This means that they can perform no non-deterministic I/O (due to our object capabilities system), and other externally accessible objects (such as a \Q!mut Transaction!) cannot be mutated during such a batch operation.

As you can see, adding support for features like \Q!invalid! and \Q!expose! is unnecessary, and would likely require making the type system significantly more complicated as well as burdening the language with more core syntactic forms.

In particular, the above code demonstrates that our system can:
\begin{itemize}
\item Have useful objects that are not entirely encapsulated: the \Q!Person holder! is a \Q!mut! field; this is fine since it is not mentioned in the \Q!invariant()! method.
\item Wrap normal methods over rep mutators: \Q!transfer! is not a rep mutator, so it can use \Q!this! multiple times and take a \Q!mut! parameter.
\item Perform multiple state updates with only a single invariant check: the loop in \Q!transferInner(ts)! can perform multiple field updates of \Q!income! and \Q!expenses!, however the \Q!invariant()! will only be checked at the end of the loop.
\item Temporarily break an invariant: it is fine if during the \Q!for! loop, \Q!expenses > income!, provided that this is fixed before the end of the loop.
\item Pass the state of an `invalid' object around, in a safe manner: an \Q!AccountBox! contains the state of \Q!Account!, but not the invariant method.
Note how programmers can use conventional private types to control how such `invalid' versions of objects are exposed in the public API, for example by 
declaring \Q@AccountBox@ as a private nested class.
In contrast, if \Q@invalid@ was a type system feature, then any user defined type would intrinsically expose the existence of both variants in the public API.
\end{itemize}

Under our strict invariant protocol, the invariant holds for all reachable objects.
The sub invariant pattern allows to control when an object is required to be valid.
Instead, other protocols strive to allow the invariant to be observed broken in controlled conditions defined by the protocol itself.

The sub invariant pattern offers interesting guarantees:
any object `\Q@a@' with a \Q@subInvariant()@ method that is checked by the \Q@invariant()@ method of an object `\Q@b@'
will respect its \Q@subInvariant()@ in all contexts where `\Q@b@' is involved in execution.
This is because whenever `\Q@b@' is involved in execution, its invariant holds.
Moreover, \Q@a@'s \Q@subInvariant()@ can be observed as \Q@false@ only if a rep mutator of `\Q@b@' is currently active (that is, being executed),
or \Q@b@ is now garbage collectable.
Thus, even when there is no reachable reference to \Q@b@ in the current stack frame,
if no rep mutator on \Q@b@ is active, \Q@a@'s \Q@subInvariant()@ will hold.

In the former example, this means that
if you can refer to an \Q!Account!, you can be sure that its \Q!income >= expenses!;
if you have an \Q!AccountBox! then you can be sure that either \Q!income >= expenses! or 
a rep mutator of the corresponding \Q@Account@ object is currently active.
This closely resembles some visible state semantic protocols, aiming to ensure that  
either an object's invariant holds, or one of its methods is currently active.


Another interesting and natural application of the sub invariant pattern would be to support a version of the GUI such that, when a \Q@Widget@'s position is updated, the \Q@Widget@ can in turn update the coordinates of its parent \Q@Widget@s, in order to re-establish their \Q@subInvariants@.
This would also make the GUI follow the versions of the composite pattern were objects have references to their `parent' nodes.
The main idea is to define an interface \Q@HasSubInvariant@, that denotes \Q@Widgets@ with a \Q@subInvariant()@ method. Then, \Q@WidgetWithInvariant@ is a decorator over a \Q@Widget@; the invariant method of a \Q@WidgetWithInvariant@ checks the \Q@subInvariant()@ of each contained widget.

%How to apply "Widget; the invariant method of a Widget This Invariant-->Widget and its invariant method" here?

We define \Q@SafeMovable@ as a \Q@Widget@ and \Q@HasSubInvariant@. Since \Q@subInvariant()@ methods don't have the restrictions of invariant methods, it allows \Q@SafeMovable@ to be significantly simpler than the version shown before in \autoref{s:case-study}.
\begin{lstlisting}
interface HasSubInvariant{ read method Bool subInvariant(); }

class SafeMovable implements Widget,HasSubInvariant {
  Int width = 300; Int height = 300;
  Int left; Int top;  // Here we do not use a box, thus all the state
  mut Widgets c;      // is in SafeMovable.
  mut Widget parent;//We add a parent field

  @Override read method Int left(){ return this.left; }
  @Override read method Int top(){ return this.top; }
  @Override read method Int width(){ return this.width; }
  @Override read method Int height(){ return this.height; }
  @Override read method read Widgets children(){ return this.c; }
  @Override mut method Void dispatch(Event e){
    for(mut Widget w :this.c){ w.dispatch(e); }
  }
  @Override read method Bool subInvariant(){ /*same of original GUI*/ }

  SafeMovable(mut Widget parent,mut Widgets c){
    this.c=c;          //SafeMovable no longer has an invariant,
    this.left=5;       //so we impose no restrictions on its constructor
    this.top=5;
    this.parent=parent;
    c.add(new Button(0,0,10,10,new MoveAction(this));
  }
}

class MoveAction implements Action{
  mut SafeMovable o;
  MoveAction(mut SafeMovable o){ this.o = o; }
  mut method Void process(Event e){
    this.o.left+=1;
    Widget p = this.o.parent;
    ... // mutate p to re-establish its subInvariant
  }
}
class WidgetWithInvariant implements Widget{
  rep Widget w;
  @Override read method Int left(){ return this.w.left; }
  @Override read method Int top(){ return this.w.top; }
  @Override read method Int width(){ return this.w.width; }
  @Override read method Int height(){ return this.w.height; }
  @Override read method read Widgets children(){ return this.w.c; }
  @Override mut method Void dispatch(Event e){ w.dispatch(e); }
  @Override read method Bool invariant(){ return wInvariant(w); }
  static method Bool wInvariant(read Widget w){
    for(read Widget wi:w.children()){ if(!wInvariant(wi)){ return false; } }
    //Check that the subInvariant of all of w's descendants holds
    if(!(w instanceof HasSubInvariant)){ return true; }
    HasSubInvariant si = (HasSubInvariant)w;
    return si.subInvariant();
  }
  WidgetWithInvariant(capsule Widget w){ this.w = w; }
}
... // main expression
//#$\$$ is a capability operation making a Gui object
mut Widget top=new WidgetWithInvariant(new SafeMovable(...))
Gui.#$\$$().display(top);
\end{lstlisting}
In this way, the method \Q@WidgetWithInvariant.dispatch()@ is the only rep mutator, hence the only invariant checks will be at the end of \Q@WidgetWithInvariant@'s constructor and dispatch methods.

Importantly, this allows the graph of widgets to be cyclic and for each to freely mutate each
other, even if such mutations (temporarily) violate their \Q@subInvariant@'s.
In this way a widget can access its parent (whose \Q@subInvariant()@ may not hold) in order to re-establish it.
Note that this trade off is logically unavoidable:
in order to manipulate a parent in order to fix it, the parent must be reachable, but
by mutating a \Q@Widget@'s position, its parent may become invalid.
Thus if \Q@Widget@s were to encode their validity in their \Q@invariant()@ methods they could not have access to their parents.
Instead, by encoding their validity in a \Q@subInvariant()@ method,
they can access invalid widgets, but this comes at a cost: the programmer must
reason as to when \Q@Widgets@ are valid, as we described above.


\subheading{The Transform Pattern}
Recall the GUI case study from \autoref{s:case-study}, where we had a \Q!Widget! interface and a \Q!SafeMovable! (with an invariant) that implements \Q!Widget!.
% A rep mutator method is essentially a mutation of a field, which is guaranteed to not see the \Q@this@ object.
% Thus, if \Q@this@ is made invalid during  the method's execution, we could not observe it until after the method completes.
Suppose we want to allow \Q@Widget@s to be scaled, we could add \Q@mut@ setters for \Q@width()@, \Q@height()@, \Q@left()@, and \Q@top()@ in the \Q@Widget@ interface. However, if we also wish to scale its children we have a problem, since \Q@Widget.children()@ returns a \Q@read Widgets@, which does not allow mutation. We could of course add a \Q@mut@ method \Q@zoom(w)@ to the \Q@Widget@ interface, however this does not scale if more operations are desired. If instead \Q@Widget.children@ returned a \Q@mut Widgets@, it would be difficult for \Q@Widget@ implementations, such as \Q@SafeMovable@, 
to mention their \Q!children()! in their \Q!invariant()!.
% In the above \Q@SafeMovable@ we only had one rep mutator: \Q@dispatch@. However suppose a \Q@Widget@ wants to directly mutate it's descendents, however it can't do that since \Q@Widget.children@ returns a \Q@read Widgets@, if it returned a \Q@mut Widgets@ then \Q@SafeMovable@ could not be implement, as it's children are contained inside a capsule-field. 
% At first glance, it may seem that rep mutators allow only very limited kinds %of mutation.
% This is however not the case. 
% Consider the following
% simple pattern to allow flexible use of encapsulated content: define a
A simple and practical solution would be to define a \Q@transform(t)@ method in \Q@Widget@, and a \Q@Transformer@ interface 
like so:
%\footnote{A more general transformer could return a generic \Q@read R@.}
\begin{lstlisting}[escapechar=\%]
interface Transformer<T> { capsule method Void apply(mut T elem); }

interface Widget { ...
  mut method Void top(Int that); // setter for immutable data
  // transformer for possibly encapsulated data
  mut method read Void transform(capsule Transformer<Widgets> t);
}
class SafeMovable implements Widget { ...
  // A well typed rep mutator
  mut method Void transform(capsule Transformer<Widgets> t) {t.apply(this.box.c);}}
\end{lstlisting}
% Note that the code above does not access a \Q!capsule! field but merely calls a method that does; thus  it is \emph{not} a rep mutator method, so it is not constrained by the restrictions on them. Code like the above would also allow one to mutate multiple \Q!capsule! fields in one method.
%Our pattern cooperates with the language’s restrictions to ensure each mutation is completed as a separate operation, that is perceived by the rest of the system %as if it was atomic.%
%,  i.e. they can't see or update the other \Q!capsule! fields.
The \Q@transform@ method offers an expressive power similar to \Q@mut@ getters, but prevents \Q@Widgets@ from leaking out.  With a \Q@Transformer@, a \Q@zoom(w)@ function could be simply written as:
\begin{lstlisting}
static method Void zoom(mut Widget w) {
  w.transform(ws -> { for (wi : ws) { zoom(wi); } });
  w.width(w.width() / 2); ...; w.top(w.top() / 2); 
}
\end{lstlisting}

In the context of reference capabilities, \Q@capsule@ lambdas/closures will only be allowed to capture \Q@imm@ and \Q@capsule@ local variables.
Note that the \Q@Transformer@ parameter to \Q@transform@ is \Q@capsule@ and the method \Q@Trasformer.apply@ takes a \Q@capsule@ receiver. In particular, this means that \Q@transform@ will be able to call the lambda at most once,
and that those lambdas cannot be saved and passed to multiple calls to \Q@transform@.
However, we could instead make \Q@transform@ take an \Q@imm@ \Q@Transformer@, and make \Q@Transformer.apply@ be an \Q@imm@ method. This would allow those lambdas to be freely copied and called multiple times, however they would only be able to capture \Q@imm@ local variables.
%not be able to capture pre-existing mutable objects.

Here, we assume lambdas, as in Java, are sugar for normal objects that implement an interface with a single abstract method.
As an example, we could use the following sound rules to determine what lambdas are allowed:
\Q!imm! lambda objects implementing an interface with an \Q!imm! method which only captures final \Q!imm! variables,
\Q!mut! lambdas implementing a \Q!mut! method which only captures final \Q!imm! and \Q!mut! variables,
and \Q!capsule! lambdas implementing a \Q!capsule! method which only captures final \Q!imm! and \Q!capsule! variables.

%The comment below was confusing when doing src/replace,
%Please, check that the new version of capsule lambdas is OK

% \begin{lstlisting}[escapechar=\%]
%// Lambda Expression that creates a new Transformer<...>
%this.transform(l -> l.add(new MyWidget(..)))
%\end{lstlisting}
%//`i' is captured by the closure.
%// `imm' and `capsule' varaibles can be captured.

%    %\Comment{}%this.items.add(i);
%    // Cant instead capture `this': it can't be typed %as `imm'
%    // since `ItemTransformer.transform()' is an %`imm' method
%  })
%}
%  // instead of:
%\Comment{}%this.exposeItems().add(i)

%Note that the code above does not access a \Q!capsule! field but merely calls a method that does; thus
%it is \emph{not} a rep mutator method, so it is not constrained by the restrictions on them. Code like the above would also allow one to mutate multiple \Q!capsule! fields in one method.
%Our pattern cooperates with the language’s restrictions to ensure each mutation is completed as a separate operation, that is perceived by the rest of the system
%as if it was atomic.%
%,  i.e. they can't see or update the other \Q!capsule! fields.

\subheading{Using Patterns Together: A General and Flexible Graph Class}
Here we rely on all the patterns shown above to encode a general library for \Q@Graph@s
of \Q@Node@s.
Users of this library can define personalised kinds of nodes,
with their own personalised sub invariant.
The library will ensure that no matter how the library is used, for any accessible \Q@Graph@, each user defined sub invariant of its \Q@Node@s holds.
Note that those sub invariants are not restricted to the local state of a node; since they can explore the state of all reachable nodes, they may even depend upon the whole graph.

The \Q@Node@s are guaranteed to be encapsulated by the \Q@Graph@, however they can be arbitrarily modified by user defined transformations using the Transform Pattern.
%interface Transform<T>{ method read T apply(mut Nodes nodes); }//can capture only imm
\begin{lstlisting}
interface Transform<T>{ capsule method read T apply(mut Nodes nodes); }

interface Node{
  read method Bool subInvariant(read Nodes nodes)
  mut method mut Nodes directConnections()
}
class Nodes{//just an ordered set of nodes 
  mut method Void add(mut Node n){..}
  read method Int indexOf(read Node n){..}
  mut method Void remove(read Node n){..}
  mut method mut Node get(Int index){..}
}
class Graph{ 
  rep Nodes nodes; //box pattern
  Graph(capsule Nodes nodes){..}
  read method read Nodes getNodes(){ return this.nodes; }
  <T> mut method read T transform(capsule Transform<T> t){
    mut Nodes ns=this.nodes;//rep mutator with a single use of 'this'
    return t.apply(ns);//single call of the capsule lambda
  }
  read method Bool invariant(){
    for(read Node n: this.nodes){if(!n.subInvariant(this.nodes)){return false;}}
    return true;
  }
}
\end{lstlisting}
We now show how our \Q@Graph@ library allows the invariant of the various \Q@Node@s to be customised by the library user, and arbitrary transformations can be performed on the \Q@Graph@s. This is a generalisation of the example proposed by~\cite{Summers:2009:NFO:1562154.1562160}(section 4.2) as one of the hardest problems when it comes to enforcing invariants.

Note how there are only a minimal set of operations defined in the above code, 
others can be freely defined by the user code, as demonstrated below:

\begin{lstlisting}
class MyNode{
  mut Nodes directConnections;
  mut method mut Nodes directConnections(){ return this.directConnections; }
  MyNode(mut Nodes directConnections){../*presented later*/..}
  read method Bool subInvariant(read Nodes nodes){
    /* any user defined condition on this or nodes */}  
  capsule method read MyNode addToGraph(mut Graph g){../*presented later*/..}
  read method Void connectWith(read Node other, mut Graph g){..}
}
...
mut Graph g = new Graph(new Nodes());
read MyNode n1 = new MyNode(new Nodes())).addToGraph(g);
read MyNode n2 = new MyNode(new Nodes())).addToGraph(g);
//lets connect our two nodes
n1.connectWith(n2,g);
\end{lstlisting}
Here we define a \Q@MyNode@ class, where the \Q@subInvariant(nodes)@ can express any property over \Q@this@ and \Q@nodes@, such as properties over their direct connections, or any other reachable node.

We can define methods in \Q@MyNode@ to add our nodes
to graphs and to connect them with other nodes.
Note that the method \Q@addToGraph(g)@ is marked as \Q@capsule@: this ensures that the node is not in any other graph.
In contrast, the method \Q@connectWith(other, g)@ is marked as \Q@read@, even though it is clearly intend to modify the reachable object graph of \Q@this@.
It works by recovering a \Q@mut@ reference to \Q@this@ from the \Q@mut Graph@.

These methods can be implemented like this:
\begin{lstlisting}
read method Void connectWith(read Node other,mut Graph g){
	Int i1=g.getNodes().indexOf(this);
	Int i2=g.getNodes().indexOf(other);
	if(i1==-1 || i2==-1){throw /*error nodes not in g*/;}
	g.transform(ns->{
		mut Node n1=ns.get(i1);
		mut Node n2=ns.get(i2);
		n1.directConnections().add(n2);
	});
}
capsule method read MyNode addToGraph(mut Graph g){
	return g.transform(ns->{
	mut MyNode n1=this;//single usage of capsule 'this'
	ns.add(n1);
	});
}
\end{lstlisting}
As you can see, both methods rely on the transform pattern.

These transformation operations are very general since they
can access the \Q@mut Nodes@ of the \Q@Graph@ and 
any \Q@rep@ or \Q@imm@ data from outside.
Note how the body of the \Q@capsule@ lambda in \Q@connectWith(other,g)@, can not capture the \Q@read this@ or the \Q@read other@, but we get their (immutable) indexes 
and recover the concrete objects from the \Q@mut Nodes ns@ object.
In this way, we also obtain more useful \Q@mut@ references to those nodes.
On the other hand, note how in \Q@addToGraph(g)@ we use the reference to the \Q@capsule this@ within the lambda, this allows the lambda to be safely typed as \Q@capsule@, since there can be no other aliases to \Q@this@, and the \Q@this@ variable cannot be used again in the method.