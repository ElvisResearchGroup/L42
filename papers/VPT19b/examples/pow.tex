\subsection{Recursive Composition Example}
We now extend our /pow_generate/ example from \sref{bic} to generate correct code, without needing to run a static verifier each time it is called:
\begin{lstlisting}
Trait pow_generate(Nat exp) {
	if (exp == 0)
		return static class { 
			@ensures(result == 0)
			Nat exp() { return 0; }

			@requires(x != 0)
			@ensures(result == x**exp())
			Int pow(Int x) { return 1; }
		};
	else if (exp == 1)
		return static class {
			... // similar to the above, but without the @requires
		};
	else if (exp % 2 == 0)
		return pow_generate(exp/2).extend(static class {
			Nat super_exp();

			@ensures(result == x**super_exp())
			Int super_pow(Int x);
			
			@ensures(result == 2*super_exp())
			Nat exp() { return 2*super_exp(); }

			@ensures(result == x**exp())
			Int pow(Int x) { return super_pow(x*x); }
		});
	else 
		return pow_generate(exp - 1).extend(static class {
			... // similar to the above
		});}
\end{lstlisting}

Each of the trait literals above (the /static class \{...\}/ expressions) can be statically verified as being correct. The function /pow_generate/ then combines these trait literals with our /extend/ operator, guranteeing that the result (if any) is also correct. The idea is that /exp/ and /pow/ represent the current exponent and power function we are generating, /super_exp/ and /super_pow/ correspond to the exponent and power function of the recursive call. Though our trait literals are more verbose than in our non verified version, they ensure that the contract matching performed by /extend/ will succeed. 

The idea is that /pow_generate($e$)/ (where $e > 0$) will return a literal of the following form:
\begin{lstlisting}
static class {
	@ensures(...)
	Nat exp() { ... } // Will return $e$ when called

	@ensures(result == x**exp())
	Int pow(Int x) { ... }
	
	... // and perhaps some private super_ methods
}
\end{lstlisting}
Then /pow_generate($e^\prime$)/ (where $e^\prime > e$) will /extend/ this trait with one of the form:
\begin{lstlisting}
static class {
	Nat super_exp();

	@ensures(result == x**super_exp())
	Int super_pow(Int x);
	
	@ensures(...)
	Nat exp() { ... } // Will return $e^\prime$ when called

	@ensures(result == x**exp())
	Int pow(Int x) { ... }
}
\end{lstlisting}

Because /exp/ and /pow/ are implemented in both /pow_generate($e$)/ and /pow_generate($e^\prime$)/, /extend/ will add a /super_/ prefix to the ones in /pow_generate($e$)/. Thus /pow_generate($e^\prime$)/ will return:
\begin{lstlisting}
static class {
	@ensures(...)
	Nat super_exp() { ... } // Will return $e$ when called

	@ensures(result == x**super_exp())
	Int super_exp(Int x) { ... }
	
	... // And some private methods (which will not clash with the above)
}.extend(static class {
	Nat super_exp();

	@ensures(result == x**super_exp())
	Int super_pow(Int x);
	
	@ensures(...)
	Nat exp() { ... } // Will return $e^\prime$ when called

	@ensures(result == x**exp())
	Int pow(Int x) { ... }
})
\end{lstlisting}

The contracts of both operands of the /extend/ are compatible, and so the above will succeed and produce something of the form:
\begin{lstlisting}
static class {
	@ensures(...)
	Nat exp() { ... } // Will return $e^\prime$ when called

	@ensures(result == x**exp())
	Int pow(Int x) { ... }

	@ensures(...)
	private Nat super_exp() { ... } // Will return $e$ when called

	@ensures(result == x**super_exp())
	private Int super_exp(Int x) { ... }
	
	... // maybe some more private methods
}
\end{lstlisting}

Note that /pow_generate/ never recursively calls /pow_generate(0)/, so the /@requires/ clause in the result of /pow_generate(0)/ will not cause any problems with our contract matching.

Though our system guarantees that the result of /pow_generate/ is `correct' (i.e. the methods in the return trait satisfy their contracts), it does not say what methods will be in the result or what their contracts will be. We could statically verify /pow_generate/ itself, however an easier option would be to check at runtime that the result of /pow_generate($e$)/ is of the form:
\begin{lstlisting}
static class {
	@requires(true)
	Nat exp() { ... }

	@requires(x != 0) // if $e$ == 0
	@requires(true) // otherwise
	@ensures(result == x**exp())
	Int pow(Int x) { ... }
}
\end{lstlisting}
This could be done with introspection on the AST of the resulting trait. We also need to check that calling the above /exp()/ method produces $e$. However, static methods in traits cannot be directly called, instead we could dynamically create a class from the trait and call /exp()/ on the result. Alternatively, we could manually `interpret' the method's AST.