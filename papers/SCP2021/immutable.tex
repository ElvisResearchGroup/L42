\section{Essential Language Features}
\label{s:immutable}
Our invariant protocol relies on many different features and requirements. In this section 
we will show examples of using our system, and how relaxing any of our requirements would break the soundness of our protocol.
In our examples and in L42, the reference capability \Q@imm@ is the default, and so it can be omitted.
Many verification approaches take advantage of the separation between primitive/value types and objects, since the former are immutable and do not support reference equality.
However, our approach works in a pure OO setting without such a distinction. Hence we write all type names in \Q@BoldTitleCase@ to emphasise this. 
To save space we omit the bodies of constructors that simply initialise fields with the values of the constructor's parameters, but we show their signature in order to show any annotations.

First we consider
\Q@Person@: it has a single immutable (and non final) field \Q@name@.
\begin{lstlisting}
class Person {
  read method Bool invariant() { return !name.isEmpty(); }
  private String name;//the default RC imm is applied here
  read method String name() { return this.name; }
  mut method Void name(String name) { this.name = name; }
  Person(String name) { this.name = name; }
}
\end{lstlisting}
The \Q@name@ field is not final: \Q@Person@s can change state during their lifetime. The ROGs of all of a \Q@Person@'s fields are immutable, but \Q@Person@s themselves may be mutable.
We enforce \Q@Person@'s invariant by generating checks on the result of calling \Q@this.invariant()@: immediately after each field update, and at the end of the constructor.
Such checks are generated/injected, and not directly written by the programmer.
\begin{lstlisting}
class Person { .. // Same as before
  mut method String name(String name) {
    this.name = name; // check after field update
    if (!this.invariant()) { throw new Error(...); }
  }
  Person(String name) {
    this.name = name; // check at end of constructor
    if (!this.invariant()) { throw new Error(...); }
    }
}
\end{lstlisting}
%Programmers often manually check for invariants by placing assertions in the same places we did in this example. % ISAAC: Deleted because unnecesary, and makes the next sentence misunderstandable
We now show how if we were to relax (as in Rust), or even eliminate (as in Java), the support for OCs, RCs, or strong exception safety, the above checks would not be sufficient to enforce our invariant protocol.

\subheading{Unrestricted Access to Capability Objects?} Allowing \Q@invariant()@ methods to (indirectly) perform non-deterministic operations by creating new capability objects or mutating existing ones would break our guarantee that (manually) calling \Q@invariant()@ always returns \Q!true!.
Consider this use of person; where \Q@myPerson.invariant()@ may randomly return \Q!false!:
\begin{lstlisting}[morekeywords={assert}]
class EvilString extends String {//INVALID EXAMPLE
  @Override read method Bool isEmpty() { return new Random().bool(); }
}//Creates a new capability out of thin air
...
method mut Person createPersons(String name) {
  // we can not be sure that name is not an EvilString
  mut Person schrodinger = new Person(name); // exception here?
  assert schrodinger.invariant(); // will this fail?
  ...}
\end{lstlisting}
%//  mut Person schrodinger2 = new Person(name); // what about here?
Despite the code for \Q@Person.invariant()@ intuitively looking correct and deterministic (\Q@!name.isEmpty()@), the above call to it is not. Obviously this breaks any reasoning and would make our protocol unsound. 
In particular, note how in the presence of dynamic class loading, we have no way of knowing what the type of \Q@name@ could be. Since our system allows non-determinism only through capability objects, and 
restricts their creation, the above example is prevented.

Moreover, since our system allows  
non-determinism only through \Q@mut@ methods on capability objects, 
even if an object has a \Q@capsule@ field referring to a ``file'' object, it would be unable to read such file during an invariant, since a \Q@mut@ reference would be required, but only a \Q@read@ reference would be available.

\subheading{Allowing Internal Mutation Through Back Doors?}
Rust~\cite{matsakis2014rust} and Javari~\cite{TschantzErnst05}
allow interior mutability:
the ROG of an `immutable' object can be mutated through back doors.
Such back doors would allow \Q@invariant()@ methods to store and read information about previous calls.
The example class \Q@MagicCounter@ breaks determinism by
remotely breaking the invariant of \Q@person@ without any interaction with the \Q@person@ object itself:
\begin{lstlisting}
class MagicCounter {//INVALID EXAMPLE
  method Int incr(){/*return counter++; using internal mutability*/}}
class NastyS extends String {..
  MagicCounter c = new MagicCounter(0);
  @Override read method Bool isEmpty(){return this.c.incr()!=2;}}
...
NastyS name = new NastyS(); //RCs believe name's ROG is immutable
Person person = new Person(name); // person is valid, counter=1
name.incr(); // counter == 2, person is now broken
person.invariant(); // returns false, counter == 3
person.invariant(); // returns false, counter == 4
\end{lstlisting}
Such back doors are usually motivated by performance reasons, however in~\cite{GordonEtAl12} they
discuss how a few trusted language primitives can be used to perform caching and other needed optimisations,
without the need for back doors.


%mine: yes, too strong: For validation we need the language to guarantee true deep immutability.
%your: just points outside: It would require some powerful static or dynamic analysis to keep track of every case the ROG of \Q@Person@ could be indirectly mutated, and insert validity checks appropriately, however ensuring deep mutability trivialises this for simple classes.
% Allowing such back-doors could also be used to break the determinism of the \Q@invariant()@ method:
% information can be stored about previous calls.
% In this example you can see how the invariant get to be \Q@false@ and then \Q@true@ again.
%In our simple example, \Q@Person@ objects can be mutated using the setter, and exposed using the getter.
%We may consider the getter to be safe since in modern languages we expect strings to be immutable objects.
%\footnote{While we can update the field \Q@name@ to point to another string, we cannot mutate the string object itself.
%To obtain  \Q@"Hello"@ from \Q@"hello"@ we need to create a whole new string object that looks like the old one except for the first character. This would be different in older languages like C, where strings are just mutable arrays of characters.}
%
%Again, the assumption that they are immutable depends on the correctness of the code inside \Q@String@: if there was a bug in the \Q@String@ class, or any \Q@String@ subclass, then executing 
%\Q@println(bob.name())@ may change \Q@bob@ by quietly changing a part of its ROG.
%Again, checking
%what methods mutate states cannot be responsibility of the \Q@Person@ programmer.
%For Validation we need a language supporting aliasing and mutability control.
%\begin{comment}
%\item Sample Bug 1:
%Suppose there was a bug in \Q@String.isEmpty()@, causing the method to non-deterministically return \Q@true@ or \Q@false@.
%What would it mean for Validation?
%Would a \Q@Person@ be at the same time 
%valid and invalid?
%
%Only deterministic methods can be used for validation.
%Ensuring this cannot be responsibility of the \Q@Person@ programmer, since it may depend on third party code, as shown in this example.
%However, statically checking if a method is deterministic is hard/impossible in most imperative object-oriented languages.
%
%While we may not expect the presence of bugs in the standard library class \Q@String@, the same behaviour can be achieved with subtyping:
%\saveSpace
%\begin{lstlisting}
%class EvilStr extends String{
%  method Bool isEmpty(){
%    return new Random().bool();
%  }}
%...
%String name=...$\Comment{can this be an EvilStr?}$
%Person bob=new Person(name);
%\end{lstlisting}
%\saveSpace
%As you can see, it is hard to make sound claims about Validation.
%
%\item Sample Bug 2:
%In our simple example, \Q@Person@ objects can be mutated using the setter, and exposed using the getter.
%We may consider the getter to be safe since in modern languages we expect strings to be immutable objects.
%\footnote{While we can update the field \Q@name@ to point to another string, we cannot mutate the string object itself.
%To obtain  \Q@"Hello"@ from \Q@"hello"@ we need to create a whole new string object that looks like the old one except for the first character. This would be different in older languages like C, where strings are just mutable arrays of characters.}
%
%Again, the assumption that they are immutable depends on the correctness of the code inside \Q@String@: if there was a bug in the \Q@String@ class, or any \Q@String@ subclass, then executing 
%\Q@println(bob.name())@ may change \Q@bob@ by quietly changing a part of its ROG.
%
%Again, checking
%what methods mutate states cannot be responsibility of the \Q@Person@ programmer.
%For Validation we need a language supporting aliasing and mutability control.
%\end{comment}

\subheading{No Strong Exception Safety?}
The ability to catch and recover from invariant failures allows programs to take corrective action.
Since we represent invariant failures by throwing unchecked exceptions, programs can recover from them with a conventional \Q@try@--\Q@catch@.
%\REVComm{
%	Due to the guarantees of strong exception safety, the only trace that an invalid object existed is the exception thrown; any object that has been mutated/created during the \Q@try@ block is now unreachable (as happens in alias burying~\cite{boyland2001alias}).
	Due to the guarantees of strong exception safety, any object that has been mutated during a \Q@try@ block is now unreachable, as happens in alias burying~\cite{boyland2001alias}.
% In addition, since unchecked exceptions are immutable, they can not contain a \Q@read@ reference to any object (such as the \Q@this@ reference seen by \Q@invariant()@ methods).
%NOTE the text above was pointless since the invariant method can not see 'this' directly anywhay!
% ISAAC: What two properties where you refering to?
 This property ensures that an object whose invariant fails will be unreachable after the invariant failure has been captured. %in a \Q@catch@.	
%}{3}{\label{SES2} [see footnote \ref{SES1}]}.
If instead we were to not enforce strong exception safety, an invalid object could be made reachable.
The following code is ill-typed since we try to mutate \Q@bob@ in a \Q@try-catch@ block that captures all unchecked exceptions; thus also including invariant failures:
%\saveSpace
\begin{lstlisting}[morekeywords={assert}]
mut Person bob = new Person("Bob");//INVALID EXAMPLE
// Catch and ignore invariant failure:
try { bob.name(""); } catch (Error t) { }// bob mutated
assert bob.invariant(); // fails!
\end{lstlisting}
The following variant is instead well typed, since \Q@bob@ is now declared inside of the \Q@try@ and it is guaranteed to be garbage collectable after the \Q@try@ is completed.
\begin{lstlisting}[morekeywords={assert}]
try { mut Person bob = new Person("Bob");    bob.name(""); } 
catch (Error t) { }
\end{lstlisting}

%Recovering from an invariant failure in this way is unsound and would break our protocol.
%Strong exception safety is a useful property to enforce, but for the specific purpose of validation this could be relaxed by restricting only \Q@try-catch@ blocks that could capture unchecked exceptions.
%Since calls to \Q@invariant()@ may only throw unchecked-exceptions, violating strong exception safety within a \Q@try-catch@ that cannot catch unchecked-exceptions would not break our protocol.


%LATER: This means that we could relax our Strong Exception Safety to hold only on unchecked exceptions (by restricting only \Q@try-catch@ blocks that capture unchecked exceptions.



% One of the advantages of checking Validation at run time, is that
% we can allow the program can take corrective actions if a property is violated.
% This may be implemented with a conventional \Q@try-catch@ if violations are represented by throwing errors.
% However, there is an issue with exceptions modelling invalid objects: they can be captured when the invalid object is still in scope. For example:


%As you can see, if we can capture validation failures as normal exceptions %(very desirable feature) then we may end up using invalid objects.
%Moreover,
% as shown before with the example of transferring cargo between two boats,
%after an invariant has been violated, even objects with valid invariant may be in an unexpected state.

% This situation is a general issue about reasoning on the state after recovering from exceptions.
% In particular, as shown in the example this prevent sound validation.

% Note how this produces a different semantics with respect to static verification, where violations
% never happened. However this will not necessarily lead to a broken semantics:
%Thanks to Strong exception safety we have a system where either the application terminate
%when an invalid object is detected, or where any witness of the execution causing the invalid object is erased from history
%those objects and all the witnesses will be garbage collected
% (as happens in alias burying~\cite{boyland2001alias}).
%In our example, this means that to continue execution after a detected bug, 
%we would require to garbage collect the overloaded boat, their cargo and probably most of the commercial port too.








%\subsubheading*{Solving Issue 3: Constructors}
%\saveSpace
%Exposing \Q@this@ during construction is a generally recognized problem~\cite{gil2009we}.
%A simple solution is to require all constructors to 
%simply take a parameter for each field and to just initialise the fields.
%An advantage of such approach is syntactic brevity: constructors are implicitly defined
%by the set of fields and thus there is no need to define them manually.
%\textbf{Expressive initialisation operations can still be performed, by following the factory pattern.}
%\saveSpace


%\subsubheading*{Recap}
%By utilising type modifiers (\Q@imm@, \Q@mut@ and \Q@read@), object capabilities and immutable exceptions we obtain sound runtime verification for immutable classes/UML data types.
