%-----------------------------------------------------------
%define simple objects
%show solution  for simple person: requires 3 properties
%show solution is sound --> proof in appendix
%naive is unsound - person 3 bugs
\saveSpace
\section{Our Invariant Protocol}
\label{s:validate}
\saveSpace
Our invariant protocol guarantees that the whole ROG of any object involved in execution (formally, in a redex) is valid.
This implies that if you can call methods on an object, calling \Q@invariant@ on it is guaranteed to return \Q@true@.
Of course the bodies of both \Q@invariant@ methods and constructors may interact with an invalid \Q@this@.
%Logically, there are two reasons to access a field: we may wish to read the information stored in such object or we wish to mutate the object contained in the field.
%For the first case, we can type the field access as \Q@read@, but in the second case we
%need to type it as \Q@mut@. 
%We call `capsule mutators' a method accessing as \Q@mut@ a capsule field referenced in the invariant.
%We will show how capsule mutators are analogous of the pack/unpack/expose~\cite{???}.
In order for a class to have an invariant under our protocol,
its \Q@invariant@ method must be of the form \Q@read method Bool invariant() {..}@
and only use \Q@this@ to read \Q@imm@ and \Q@capsule@ fields.
Moreover, its constructors may only use \Q@this@ to initialise fields.\footnote{A factory method could be used when more flexible initialisation is required.} This prevents objects from escaping their constructor partially initialised, and of course prevents explicitly calling \Q@invariant@ inside of \Q@invariants@ and constructors bodies, but most importantly, together with our other restrictions, it prevents \Q@this@, or an alias to \Q@this@, from being passed around to arbitrary code.

While mutating the ROG of a \Q@capsule@ field mentioned in the invariant, the containing object may be invalid.
To enforce that the containing object is inaccessible, an access to such field is typed as \Q@mut@ only inside of methods following a specific pattern: the field access is of the form \Q@this.f@ and there is no other occurrence of \Q@this@, the method receiver is \Q@mut@, none of the other parameters are \Q@mut@ or \Q@read@, and the return type is not \Q@mut@.
We call such methods \emph{capsule mutators}. These additional restrictions do not apply if the field is not read in the invariant,
%Finally, we require \Q@f@ to be declared \Q@private@.%, and only accessed as \Q@this.f@.

Our invariant protocol injects calls to \Q@invariant@ on newly created objects, after constructor calls; on the receiver, after a field update; and \Q@this@ at the end of a capsule-mutator.
%\end{itemize}

% Our invariant protocol injects call to \Q@this.invariant@ at the end of constructors, after every field update and at the end of every method containing such \Q@this.f@ access. We call such methods `capsule mutators'.

%Our guarantee is much stronger then the visible state semantic, where in the presence of call-backs,
%invalid objects may be passed around and become visible to any other code~\cite{??}.


%\footnote{we could of course require zero-initialized objects be valid, but this is too strong},
%so in those cases we do not require \Q@this@ to be valid; however one can only use "this" to read/initialise fields, preventing \Q@this@, or an alias, from being passed around to arbitrary code.
%This implies that whenever the user is allowed to call x.invariant(), such call would return true.
%The only points where the user is unable to call .invariant() is inside the constructor
%and inside the invariant method itself.
%If you can refer to a variable \Q@x@, then \Q@x.invariant()@ would return \Q@true@. 
%However, we prevent directly using \Q@this@ inside constructors, invariant itself, and capsule mutators (see \ref{}),
%rather one can only use \Q@this@ to access fields.
% We guarantee that invariant holds in more points, for more objects, but we check it a million times less



%\IO{The Visible-State Semantics is bad.} Here we have a much stronger property, called \emph{operational
% validity}. This means that any object involved in execution (such as by accessing a field, calling a method on it, or) must be valid. As we use runtime-verification to detect invalid objects, it must be possible to access potentially invalid objects within their own \Q@invariant@ method, so we make an exception to the above rule in this case: \Q@this@ itself may be invalid within runtime-injected calls to \Q@invariant@.




%\section{\REVComm{The invariant Method}{2}{remind of the goals for validate (i.e. that it be pure); What about possible non-termination of validate()}}
%\label{s:validate}

%We require the receiver to be \Q@read@ and to have no arguments.
%The method is \Q@read@: thus the method body will see the \Q@this@ object as a \Q@read@ reference; and has no other parameters. 
%By starting from a \Q@read@ reference and nothing else, we are guaranteed that the method is pure:

Our invariant protocol relies on TMs and OCs to ensure that the invariant method is deterministic\footnote{If the invariant were not deterministic, it would be nearly impossible to ensure that it would return \Q@true@ at any point.}.
As this method is declared as \Q@read@, and only takes the implicit parameter \Q@this@ (as \Q@read@), we can guarantee the method is pure:
\begin{itemize}
\item the ROG of \Q@this@ is only accessed as \Q@read@ (or \Q@imm@), thus it cannot be mutated\footnote{This is even true in the concurrent environments of Pony and Gorden, since they ensure that no other thread/actor has access to a \Q@mut@/\Q@capsule@ alias of \Q@this@.},
\item if a capability object is in the ROG of \Q@this@, then it can only be accessed as \Q@read@, preventing calling non-deterministic (capability) methods,
\item no other pre-existing objects are accessible (as L42 does not have global variables).
\end{itemize}

\noindent Also note that \Q@invariant@ is declared as not throwing any exceptions,
	thus only unchecked exceptions can be propagated out.
	% \REVComm{thus it can only leak unchecked exceptions.}{3}{Can ``leaking'' of an unchecked exceptions have negative consequences for validity? Leaking is usually a negative word used in a negative context.}


Clearly the \Q@invariant@ method must be able to take an invalid \Q@this@, since the purpose of such method is to distinguish valid and invalid objects.
On a first look this may seem an open contradiction
with the aim of this work, however only calls to \Q@invariant@ inserted by the language semantics can take an invalid \Q@this@. As for any other method, when the application code calls \Q@invariant@,
\Q@this@ is guaranteed to be valid.
We require that \Q@invariant@
only uses \Q@this@ to access fields,
in order to prevent passing an invalid \Q@this@ to other methods.
%other 
%methods can not 
%However, if we allow the method to use \Q@this@ directly (e.g. storing it in a local variable or passing it to a method), we would break the guarantee of our invariant protocol.
In L42 we implemented a sound relaxation of this restriction, by allowing calling instance methods that in turn only use \Q@this@ to access fields, or call other such instance methods. With this relaxation, the semantics of \Q@invariant@ needs to be understood with the body of those methods inlined; thus the semantic of the inlined code need to be logically reinterpreted in the context of \Q@invariant@, where \Q@this@ may be invalid.
In some sense, those inlined methods and field accesses can be thought of as macro-expanded, and hence are not dynamically dispatched. 
Finally, the code of \Q@invariant@ cannot access \Q@mut@ or read fields, because their content can be changed by unrelated code.

%With the modifiers presented so far, \validate{} can only access \Q@imm@ fields.
%We will later introduce a \Q@capsule@ modifier to allow more flexible validation.


L42 does not have \Q@extends@ so we have not handled inheriting invariants.
In a language with sub-classing, invariant methods would implicitly start with a check that \Q@super.invariant()@ returns \Q@true@.
Note that invariant checks would not be performed at the end of \Q@super(..)@ constructor calls, but only at the end of \Q@new@ expressions, as happens in~\cite{feldman2006jose}.

%In a language with sub-classing, runtime-invariant checks would need to occur on the base-class subobjects, before doing the check on a derived class. In addition, the runtime would only do an invariant check after construction of complete objects.

% containing the \Q@invariant@ method has a super-class, we would automatically check \Q@super.invariant()@ at the beginning of the sub-class’s \Q@invariant@ method.
%, this is required as otherwise `invalid' objects could easily be created by simply overriding \validate.


% JUSTIFY that the fields are valid...

%validable objects are not circular (do not belong in their ROG of any of its fields)
%validate as a predicate on object fields never really see this.
%
%clarifications:
%a validate check is never needed/generated/injected when working on a read x
%multi threading is not relevant/supported
%validable objects are not circular (do not belong in their ROG of any of its fields)

\section{Invariants over immutable state}
\label{s:immState}
In this section we consider validation over fields of \Q@imm@ types.
%\footnote{
%In a real language, for conciseness one could make the \Q@imm@ modifier the default, allowing it to be omitted and our \Q@Person@ example class would only use 3 type modifiers; however we explicitly use it here for clarity.
%}
In the next section we detail our technique for \Q@capsule@ fields.

In the following code \Q@Person@ has a single immutable (non final) field \Q@name@:
\begin{lstlisting}
class Person {
  read method Bool invariant() {return !name.isEmpty();}
  private String name;//the default modifier imm is applied here
  read method String name() {return this.name;}
  mut method String name(String name) {this.name = name;}
  Person(String name) {this.name = name;} }
\end{lstlisting}
\Q@Person@ only has immutable fields and its constructor only uses \Q@this@ to initialise fields.
%; we say such a class is \emph{simple}.
%\Q@Person@, only has immutable fields and the constructor 
%uses the parameters to directly initialize (all) the fields.
% We say such a class is \emph{simple}.%
%\footnote{
%We consider only standard contractors for simplicity of exposition.
%More complex constructors could be supported, provided that \Q@this@ is only used to access fields, we do discuss them for simplicity.}
% The difference with respect to UML DataTypes 
%immutable types (like UML DataTypes)
%UML datatypes are aclass property. immutable types are often an instance one (so no final fields) 
Note that the \Q@name@ field is not final, thus \Q@Person@ objects can change state during their lifetime. This means that the ROGs of all \Q@Person@s fields are immutable, but \Q@Person@s themselves may be mutable.
%Of course UML DataTypes
%immutable types
%No, a type is not a class
% are just a special case of simple classes.
We can easily enforce \Q@Person@'s invariant by generating checks on the result of \Q@this.invariant()@ immediately after each field update, and at the end of the constructor:%
%\footnote{Since the constructor only initialises fields; as with the \validate{} method itself, we allow field uses since \Q@this@ is not directly reached.}
%would require the initial/default value of \Q@this@ to be valid.}

% If a simple class provides a \validate{} method, then validation will be enforced.
% For \Q@Person@, intuitively, the code would behave as follow:

%\Comment{if we made this public, all users who update the field need to call validate}%
%There are many interpretations for your comment
%why you deleted my code comments?
\begin{lstlisting}
class Person { .. // Same as before
  mut method String name(String name) {
    this.name = name; // check after field update
    if (!this.invariant()) {throw new Error(...);} 
  }
  Person(String name) {
    this.name = name; // check at end of constructor
    if (!this.invariant()) {throw new Error(...);}
}} // Generated code, not directly written by the programmer
\end{lstlisting}
%... $\MComment{validation error}$ 

% Many programmers attempted to write similar code in mainstream languages like Java to ensure  that some property always holds. Indeed, at first look, this code seems to correctly enforce validation. Sadly, without relying on TM and OC, the former code would be broken: just making the fields private and checking the \validate{} method at the \textbf{end of the constructor} and at the \textbf{end of mutator methods} is not enough to enforce validation.
% The trick is that our intuition relies not on statically verified properties, or on the semantics of the language, but on the expectations about `correct' behaviour of \Q@String@. We need to enforce Validation without assuming the behaviour of other objects.

\noindent If we were to relax (as in Rust) or even eliminate (as in Java) the support for TMs or OCs, the enforcement of our invariant protocol for the \Q@Person@ class would become harder, or even impossible. 

\textit{Unrestricted use of object-capabilities:}
Allowing \validate{} to (indirectly) use an object capability could allow for it to be non deterministic. For example consider this simple and contrived (mis)use of person:
\begin{lstlisting}[morekeywords={assert}]
class EvilString extends String {
  @Override read method Bool isEmpty() {
    // Create a new capability object out of thin air
    return new Random().bool(); }}
..
method mut Person createPersons(String name) {
  // we can not be sure that name is not an EvilString
  mut Person schrodinger = new Person(name); // exception here?
  assert schrodinger.invariant(); // will this fail?
  ..}
\end{lstlisting}
%//  mut Person schrodinger2 = new Person(name); // what about here?
Despite the code for \Q@Person.invariant@ intuitively looking correct and deterministic, the above call to it is not. Obviously this breaks any reasoning and would make our protocol unsound. 
In particular, note how in the presence of dynamic class loading, we have no way of knowing what the type of \Q@name@ could be.
%???
%Even if we disallow subtyping the same problem could still occur if we had a strange implementing of \Q@String@, or \Q@Person.validate()@ itself.

\textit{Allowing internal mutation through back-doors:}
Suppose we relax our rules by allowing interior mutability
as in Rust and Javari, where sneaky mutation
of the ROG of an immutable object is allowed.
Those back-doors are usually motivated by performance reasons, however in~\cite{GordonEtAl12} they
briefly discuss how a few trusted language primitives can be used to perform caching and other needed optimizations
without the need for back-doors.

Our example shows that such back-doors can be used to break determinism of  \Q@invariant@ methods by allowing the invariant to store and read information about previous calls. We use \Q@MagicCounter@ as a back-door to remotely break the invariant of \Q@person@ without any interaction with the \Q@person@ object itself.

\begin{lstlisting}
class MagicCounter {
  method Int increment(){
    //Magic mutation through an imm reciever, equivalent to i++
}}
class NastyS extends String {..
  MagicCounter evil = new MagicCounter(0);
  @Override read method Bool isEmpty() {
    return this.evil.increment() != 2; 
}}
...
NastyS name = new NastyS("bob"); //TMs belive name's ROG is imm
Person person = new Person(name); // person is valid, counter=1
name.increment(); //counter=2, person is now broken
person.invariant(); // returns false! //counter=3
person.invariant(); // returns true //counter=4
\end{lstlisting}

%mine: yes, too strong: For validation we need the language to guarantee true deep immutability.
%your: just points outside: It would require some powerful static or dynamic analysis to keep track of every case the ROG of \Q@Person@ could be indirectly mutated, and insert validity checks appropriately, however ensuring deep mutability trivialises this for simple classes.
% Allowing such back-doors could also be used to break the determinism of the \validate{} method:
% information can be stored about previous calls.
% In this example you can see how the invariant get to be \Q@false@ and then \Q@true@ again.
%In our simple example, \Q@Person@ objects can be mutated using the setter, and exposed using the getter.
%We may consider the getter to be safe since in modern languages we expect strings to be immutable objects.
%\footnote{While we can update the field \Q@name@ to point to another string, we cannot mutate the string object itself.
%To obtain  \Q@"Hello"@ from \Q@"hello"@ we need to create a whole new string object that looks like the old one except for the first character. This would be different in older languages like C, where strings are just mutable arrays of characters.}
%
%Again, the assumption that they are immutable depends on the correctness of the code inside \Q@String@: if there was a bug in the \Q@String@ class, or any \Q@String@ subclass, then executing 
%\Q@println(bob.name())@ may change \Q@bob@ by quietly changing a part of its ROG.
%Again, checking
%what methods mutate states cannot be responsibility of the \Q@Person@ programmer.
%For Validation we need a language supporting aliasing and mutability control.
%\begin{comment}
%\item Sample Bug 1:
%Suppose there was a bug in \Q@String.isEmpty()@, causing the method to non-deterministically return \Q@true@ or \Q@false@.
%What would it mean for Validation?
%Would a \Q@Person@ be at the same time 
%valid and invalid?
%
%Only deterministic methods can be used for validation.
%Ensuring this cannot be responsibility of the \Q@Person@ programmer, since it may depend on third party code, as shown in this example.
%However, statically checking if a method is deterministic is hard/impossible in most imperative object-oriented languages.
%
%While we may not expect the presence of bugs in the standard library class \Q@String@, the same behaviour can be achieved with subtyping:
%\saveSpace
%\begin{lstlisting}
%class EvilStr extends String{
%  method Bool isEmpty(){
%    return new Random().bool();
%  }}
%...
%String name=...$\Comment{can this be an EvilStr?}$
%Person bob=new Person(name);
%\end{lstlisting}
%\saveSpace
%As you can see, it is hard to make sound claims about Validation.
%
%\item Sample Bug 2:
%In our simple example, \Q@Person@ objects can be mutated using the setter, and exposed using the getter.
%We may consider the getter to be safe since in modern languages we expect strings to be immutable objects.
%\footnote{While we can update the field \Q@name@ to point to another string, we cannot mutate the string object itself.
%To obtain  \Q@"Hello"@ from \Q@"hello"@ we need to create a whole new string object that looks like the old one except for the first character. This would be different in older languages like C, where strings are just mutable arrays of characters.}
%
%Again, the assumption that they are immutable depends on the correctness of the code inside \Q@String@: if there was a bug in the \Q@String@ class, or any \Q@String@ subclass, then executing 
%\Q@println(bob.name())@ may change \Q@bob@ by quietly changing a part of its ROG.
%
%Again, checking
%what methods mutate states cannot be responsibility of the \Q@Person@ programmer.
%For Validation we need a language supporting aliasing and mutability control.
%\end{comment}

\textit{Strong Exception Safety:}
The ability to catch and recover from invariant failures is extremely useful as it allows programs to take corrective action.
Since we represent invariant failures by throwing unchecked exceptions, programs can recover from them with a conventional \Q@try-catch@.
%\REVComm{
%	Due to the guarantees of strong exception safety, the only trace that an invalid object existed is the exception thrown; any object that has been mutated/created during the \Q@try@ block is now unreachable (as happens in alias burying~\cite{boyland2001alias}).
	Due to the guarantees of strong exception safety, any object that has been mutated during a \Q@try@ block is now unreachable (as happens in alias burying~\cite{boyland2001alias}). In addition, since unchecked-exceptions are immutable, they can not contain a \Q@read@ reference to any object (such as the \Q@this@ reference seen by \Q@invariant@). Thse two properties ensure that an object whose invariant fails will be unreachable after the invariant failure has been captured. %in a \Q@catch@.	
%}{3}{\label{SES2} [see footnote \ref{SES1}]}.
If instead we where to not enforce strong exception safety, an invalid object could be made reachable:
\saveSpace
\begin{lstlisting}[morekeywords={assert}, escapechar=\%]
mut Person bob = new Person("bob");
// Catch and ignore invariant failure:
try {bob.name("");} catch (Error t) { }//ill-typed in L42
assert bob.name().isEmpty(); // bob is invalid!
\end{lstlisting}
\saveSpace
As you can see, recovering from an invariant failure in this way is unsound and would break our protocol.
%Strong exception safety is a useful property to enforce, but for the specific purpose of validation this could be relaxed by restricting only \Q@try-catch@ blocks that could capture unchecked exceptions.
%Since calls to \validate{} may only throw unchecked-exceptions, violating strong exception safety within a \Q@try-catch@ that cannot catch unchecked-exceptions would not break our protocol.


%LATER: This means that we could relax our Strong Exception Safety to hold only on unchecked exceptions (by restricting only \Q@try-catch@ blocks that capture unchecked exceptions.



% One of the advantages of checking Validation at run time, is that
% we can allow the program can take corrective actions if a property is violated.
% This may be implemented with a conventional \Q@try-catch@ if violations are represented by throwing errors.
% However, there is an issue with exceptions modelling invalid objects: they can be captured when the invalid object is still in scope. For example:


%As you can see, if we can capture validation failures as normal exceptions %(very desirable feature) then we may end up using invalid objects.
%Moreover,
% as shown before with the example of transferring cargo between two boats,
%after an invariant has been violated, even objects with valid invariant may be in an unexpected state.

% This situation is a general issue about reasoning on the state after recovering from exceptions.
% In particular, as shown in the example this prevent sound validation.

% Note how this produces a different semantics with respect to static verification, where violations
% never happened. However this will not necessarily lead to a broken semantics:
%Thanks to Strong exception safety we have a system where either the application terminate
%when an invalid object is detected, or where any witness of the execution causing the invalid object is erased from history
%those objects and all the witnesses will be garbage collected
% (as happens in alias burying~\cite{boyland2001alias}).
%In our example, this means that to continue execution after a detected bug, 
%we would require to garbage collect the overloaded boat, their cargo and probably most of the commercial port too.








%\subsubsection*{Solving Issue 3: Constructors}
%\saveSpace
%Exposing \Q@this@ during construction is a generally recognized problem~\cite{gil2009we}.
%A simple solution is to require all constructors to 
%simply take a parameter for each field and to just initialize the fields.
%An advantage of such approach is syntactic brevity: constructors are implicitly defined
%by the set of fields and thus there is no need to define them manually.
%\textbf{Expressive initialization operations can still be performed, by following the factory pattern.}
%\saveSpace


%\subsubsection*{Recap}
%By utilising type modifiers (\Q@imm@, \Q@mut@ and \Q@read@), object capabilities and immutable exceptions we obtain sound runtime verification for immutable classes/UML data types.