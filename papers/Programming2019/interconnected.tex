\saveSpace
\subsection{Interconnected Objects}
\label{s:interconnected}
\saveSpace
An interconnected object is an object that is not covered by the previous three categories.
These are objects that contains mutable fields (in addition to immutable and capsule ones).
Such objects can use mutable state to freely refer to each other, and they can easily change the way they are interconnected during the program execution.

It is not obvious how to predict their aliasing relations, and thus to decide the right moments to check the invariant.
One solution is offered by Gopinathan et al.~\cite{Gopinathan:2008:RMO:1483018.1483028}, where aspect orientation is used to keep track of every field that is read by an invariant method:
if \Q@x.invariant()@ access (directly or indirectly) a field of \Q@y@, then we store the information \Q@y->x@ in a monitor object.
Then, whenever a field update happens on an object present in the domain of our (multi-)map, we check all the invariants that may have been broken (\Q@x.invariant()@ in this case).

This solution is very laborious, and has a run time impact that is clearly an order of magnitude larger than the other solutions we have proposed.

Gopinathan et al. shows real examples
where ensuring invariants for this category of objects is important. In order to handle those cases, we could extend our approach to employ their technique.
However, there is empirical evidence~\cite{nelson2012profiling,nelson2012Thesis} 
that most objects are immutable for most of their life time,
and  we believe that most other objects fall in the simple object category.
Using a languages offering type modifiers would probably
further encourage the use of immutable, simple and encapsulated objects.
Stephen~\cite{nelson2012Thesis}  introduces the concept of `Object settling',
and it seams that most objects stop changing after a first initialization phase.
In many cases, those objects could be recognized as immutable
by an expressive type system like the one of 42.

We would like to emphasise that interconnected objects in our system have
a minimal invariant that is much richer than in the other approaches.
This is not just because of the stronger type system offering immutability
and aliasing control, but also because we can safely assume the invariant of all reachable objects.
\saveSpace
\subsection{Encapsulating Interconnected Objects}
\saveSpace
The best thing of encapsulated objects is that their \Q@capsule@ fields can point to 
arbitrary complicated graphs of interconnected objects.
For example, we may have a 
\begin{lstlisting}
class Node{ imm String label, mut ListNode nodes }
\end{lstlisting}
clearly, a \Q@Node@ is an interconnected object, that can freely refer to other \Q@Node@s.
We cannot put invariants over \Q@Node@s.
However, we can have a \Q@Graph@ class, that encapsulate \Q@Node@s.

\begin{lstlisting}
class Graph{
  capsule ListNode nodes
  read method imm Bool invariant(){...}
}
\end{lstlisting}

And now we can enforce invariants over \Q@Graph@s, in turn enforcing properties over the
whole ROG of the \Q@Graph@, that is mostly composed by interconnected \Q@Node@s objects.