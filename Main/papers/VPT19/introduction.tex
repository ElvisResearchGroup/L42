
\section{Introduction}

\paragraph{Quasi Quotation.} Lisp~\cite{pitman1980special}, MetaML~\cite{moggi1999idealized}, Template Haskell~\cite{sheard2002template} and many other approaches use Quasi Quotation (QQ).
This can be supported by two kinds of special parenthesis as a syntactic sugar to manipulate Abstract Syntax Trees (ASTs).
Lisp uses (\Q@`@) and (\Q@,@), while here we use
\Q@[|  |]@  and \Q@$\$$(  )@ (as Template Haskell) 
for better readability.

\noindent
The following example explains their meaning: 

\saveSpace
\begin{lstlisting}
Int res0=x*x $\Comment{normal code}$
Expr<x:Int$\vdash$Int> res1=[| x*x |]
  $\Comment{new Mul(new Var("x"),new Var("x"))}$
Expr<x:Int$\vdash$Int> res2=[| x* $\$$(12+3) |]
  $\Comment{new Mul(new Var("x"),new Lit(15))}$
\end{lstlisting}
\saveSpace

\noindent
Here \Q@res1@ is initialized using a ``quotation'' of code.
This is equivalent to generating the abstract syntax tree by hand, as shown in the comment.
\Q@res2@ is initialized using a ``quasi-quotation'' of code: a chunk of code with a hole, that is filled by executing an expression.

There are different ways to type QQ.
In an expression based language, 
the simplest way is to just have a primitive \Q@Expr@ type,
representing every type of code.
This ensures the result is syntactically well formed, but
it allows for the generation of ill-typed code.
Another option, for example used by MetaML,
is to have a parameterized type.
Here we use \Q@Expr<@$\Gamma\vdash T$\Q@>@; where
$\Gamma$ keeps track of the free variables and $T$ is the expected type
of the result.
This approach is restrictive (see~\cite{servetto2014meta})
 but ensures that all the resulting code is well typed.

Usually programming with QQ requires thinking about the desired method body,
 and often allows generating a more efficient body by generating code specialized for some input value.
A typical example is about generating a \Q@pow@ function, where the exponent is well known.
The ``inefficient'' version would be:

\saveSpace
\begin{lstlisting}[language=ML]
fun power(x:Int,n:Int):Int 
  = if (n=0) then 1 else x*power(x,n-1);
power7_a=$\lambda$ x:Int. power x 7;
\end{lstlisting}
\saveSpace

\noindent A more ``efficient'' version using QQ would be:

\saveSpace
\begin{lstlisting}[language=ML]
fun powerAux(n:Int):Expr<x:Int$\vdash$Int> 
  = if (n=0) then [|1|] else [|x * $\$$(powerAux(n-1)) |];

fun powerGen(n:Int): Int->Int
  = compile([| $\lambda$ x. $\$$(powerAux(n)) |]);

power7_b=powerGen 7;
\end{lstlisting}
\saveSpace

\noindent As you can see, by generating the abstract syntax tree, we can obtain exactly:

\saveSpace
\begin{lstlisting}[language=ML]
power7_b=$\lambda$ x.x*x*x*x*x*x*x*1;
\end{lstlisting}
\saveSpace

\noindent On most machines, \Q@power7_b@ runs faster than \Q@power7_a@.
Metaprogramming applications include more than just speed boosts, but we start with this example because it is very popular and simple.

The code generator above is quite compact, but it is actually \textbf{hiding} (not removing) the complexity of meta-programming.
A common approach to make the code more explicit is to extract
logical concepts as functions.
We can see that the code is proceeding in an inductive fashion:
we know the code for \Q@pow 0@, and given the code for
\Q@pow n@  we can create the code for \Q@pow (n+1)@.
Thus we define \Q@base@ and \Q@inductive@ functions, and we
use them inside \Q@powerAux@:

\saveSpace
\begin{lstlisting}[language=ML]
fun base():Expr< x:Int$\vdash$Int > 
  = [| 1 |]
fun inductive(code:Expr< x:Int$\vdash$Int >):Expr< x:Int$\vdash$Int >
  = [| x * $\$$(code) |]

fun powerAux(n:Int):Expr< x:Int$\vdash$Int >
  = if (n=0) then base()
   else inductive ( powerAux(n-1) );
\end{lstlisting}
\saveSpace

\noindent Then, we have to bind \Q@x:Int@ to a parameter in a function.
This is an important conceptual action and thus we make it a function:

\saveSpace
\begin{lstlisting}[language=ML]
fun lambdaX(code:Expr< x:Int$\vdash$Int >):Expr<$\vdash$Int->Int > 
  = [| $\lambda$ x. $\$$( code ) |]

fun powerGen(n: Int):Int->Int
  = compile(lambdaX(powerAux(n) ))
\end{lstlisting}
\saveSpace

The code we obtain is much larger, but is not logically more complex --- it is just showing the logical structure better.
Note how since QQ works near the code representation,
a function \Q@Int->Int@ is radically different from
code with a free variable \Q@x:Int@$\vdash$\Q@Int@, while they are 
logically similar concepts.


\paragraph{Iterative Composition.} We propose Iterative Composition (IC):
while the unit of composition in QQ is the single AST node, 
IC enforces a higher level of abstraction and does not work directly on the AST.
The unit of composition in IC is a \emph{Library}:
a class body, containing methods and possibly nested classes.
Libraries are self contained in the sense that they contain no free variables.
This avoids all scope-extrusion related problems, and (as shown later) enforces local reasoning.

%% (Alex) The following is commented out because it is just confusing to the reader.
%IC has already been presented in other work~\cite{servetto2014meta};
% IC's expressive power is shown by examples,
%but is not compared with QQ; moreover such works suggested IC's expressive power is inferior to QQ.

The core idea of IC~\cite{servetto2014meta} is to  \emph{rely on  operators of code composition inspired by normal
code reuse}, but lifted to the expression level.
As a concrete example, in Java operators \Q@+@ and \Q@*@ can be used in the expression \Q@1+2*3@,
but the operator \Q@extends@ can only be used in the specific context of class declaration, as in

\saveSpace
\begin{lstlisting}[language=Java]
class A extends B{/*some code*/ int m(){return 1+super.m();}}
\end{lstlisting}
\saveSpace

In our proposed approach, we lift \Q@extends@ and code literals to operator and constants
that can be used in conventional expressions.
Class declarations associate a class name with the result of an expression of type \Q@Library@. 
 We would write the former example as:

\saveSpace
\begin{lstlisting}
A = Override[m()<-superM()]( 
  {/*code of B*/},
  { /*some code*/ int m(){return 1+this.superM();}}
  )
\end{lstlisting}
\saveSpace

\noindent We support the conventional super call mechanism by annotating the operator with
the expected super call name: \Q@Override[m()<-superM()](...)@.


\noindent
We can rewrite our \Q@pow@ example 
in IC:

\saveSpace
\begin{lstlisting}
Pow = {
  static method Library base()
   ={ method Num pow(Num x)= 1 }$\Comment{Code literal with 1 method "pow(x)"}$

  static method Library inductive()
   ={$\Comment{Code literal with 2 methods: "pow(x)", "superPow(x)"}$
    method Num pow(Num x)= x*this.superPow(x)
    method Num superPow(Num x)$\Comment{no body: it is an abstract method}$
    }
  static method Library inductive(Library code)
   = Override[pow(x)<-superPow(x)](code, this.inductive())
  
  static method Library generate(Num y)
   = if (y==0) then this.base();
     else this.inductive(generate(y-1))
  }
...
Pow7 = Pow.generate(7)
$\Comment{That would reduce into the desired code as follows:}$
Pow7 ={method Num pow(Num x)=x.x*x*x*x*x*x*x*1}
\end{lstlisting}
\saveSpace

\noindent In more detail:
\begin{itemize}
\item
\Q@base()@ is a method with no parameter and a \Q@Library@ return type.
This is equivalent to a non-parameterized version of \Q@Expr@ in QQ.
However, our approach still guarantees that all the results are well typed.
\Q@base()@ returns a class with a single method \Q@pow(x)@,
returning 1.
\item
For the inductive case, the method \Q@pow(x)@ of \Q@inductive()@ is defined in terms of
another method (\Q@superPow(x)@), representing the delegation to
the former case in the inductive reasoning.
%Note that the declaration of \Q@superPow(x)@ is an abstract method: a method without body.
\item Method \Q@inductive(code)@ builds
the code for \Q@x+1@ from the code for \Q@x@.
Note how we use \Q@Override@ inside of a normal method body.
This \Q@Override@ will implement
\Q@superPow(x)@ using
the \Q@pow(x)@ body from the induction premise: the \Q@code@ parameter.
Then, \Q@superPow(x)@ is inlined.

\item Method \Q@generate(y)@ uses recursion to \textbf{iteratively compose} the result, using induction starting from
the base case.
Note how this method is logically identical to \Q@powerAux@. However,
since we always work on the actual self contained code neither \Q@lambdaX@ nor \Q@compile@ are needed.
\end{itemize}


\paragraph{Contributions.} Our approach builds on top of code composition operations like multiple inheritance and generics.
The literature offers \cite{barnett2004spec,burdy2005overview,muller2016viper} many successful efforts about proving the
\emph{semantic correctness} 
of code containing inheritance and generics.
On the other hand, static verification of code generated with meta-programming is an open research problem.
We speculate our approach may offer the opportunity to solve this problem,
and by construction generate statically verified code
by reusing techniques originally developed to verify normal object-oriented code.
The contributions of our work are as follows:
\begin{itemize}
\item
In Section~\ref{s:verification} we use examples to demonstrate how to apply conventional object-oriented verification techniques to IC.

\item
In Section~\ref{s:pattern1} we use examples to show that IC is as expressive as QQ, and that
generating code using a composition algebra
is flexible and simple technique, as long as it is combined with
reasonable programming patterns.
\item In Section~\ref{s:study}, we discuss our experience implementing metaprogramming libraries for a language where IC is the only support for code reuse. 
\end{itemize}

In this paper we do not present a formal language semantics. This is partly due to space reasons
and partly because a similar semantic has been formalized in former work~\cite{servetto2014meta}.
Here we aim to show programming patterns that use this expressive power in surprising and novel ways.


%-------------------------------------------------------------------------------------
%-------------------------------------------------------------------------------------

