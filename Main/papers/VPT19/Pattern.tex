 \documentclass[submission,copyright,creativecommons]{eptcs}
\providecommand{\event}{MiniPLOP 2019} % Name of the event you are submitting to
\usepackage{breakurl}             % Not needed if you use pdflatex only.
\usepackage{underscore}           % Only needed if you use pdflatex.
\usepackage{listings}
\usepackage{xcolor}
\usepackage{letltxmacro}
\usepackage{mathtools}
\usepackage{mathpartir}
%\usepackage{stix}

\definecolor{darkRed}{RGB}{100,0,10}
\definecolor{darkBlue}{RGB}{10,0,100}
\newcommand*{\ttfamilywithbold}{\fontfamily{pcr}\selectfont}
%\newcommand*{\ttfamilywithbold}{\ttfamily}

%found on http://tex.stackexchange.com/questions/4198/emphasize-word-beginning-with-uppercase-letters-in-code-with-lstlisting-package
%\lstset{language=FortyTwo,identifierstyle=\idstyle}
%
\makeatletter
\newcommand*\idstyle{%
        \expandafter\id@style\the\lst@token\relax
}
\def\id@style#1#2\relax{%
        \ifcat#1\relax\else
                \ifnum`#1=\uccode`#1%
                        \ttfamilywithbold\bfseries
                \fi
        \fi
}
\makeatother

\lstset{language=Java,
  basicstyle=\upshape\ttfamily\footnotesize,%\small,%\scriptsize,
  keywordstyle=\upshape\bfseries\color{darkRed},
  showstringspaces=false,
  mathescape=true,
  xleftmargin=0pt,
  xrightmargin=0pt,
  breaklines=false,
  breakatwhitespace=false,
  breakautoindent=false,
 identifierstyle=\idstyle,
 morekeywords={method,Use,This,constructor,as,into,rename},
 deletekeywords={double},
 literate=
  {\%}{{\mbox{\textbf{\%}}}}1
  {~} {$\sim$}1
%  {<}{$\langle$}1
%  {>}{$\rangle$}1
}

\newcommand*{\SavedLstInline}{}
\LetLtxMacro\SavedLstInline\lstinline
\DeclareRobustCommand*{\lstinline}{%
	\ifmmode
	\let\SavedBGroup\bgroup
	\def\bgroup{%
		\let\bgroup\SavedBGroup
		\hbox\bgroup
	}%
	\fi
	\SavedLstInline
}

\newcommand\saveSpace{\vspace{-2pt}}

\newcommand\Rotated[1]{\begin{turn}{90}\begin{minipage}{12em}#1\end{minipage}\end{turn}}

\newcommand{\Q}{\lstinline}
\newenvironment{bnf}{$\begin{aligned}}{\end{aligned}$}
\newcommand{\production}[3]{\textit{#1}&\Coloneqq\textit{#2}&\text{#3}}
\newcommand{\prodNextLine}[2]{&\quad\quad\textit{#1}&\text{#2}}
\newenvironment{defye}{\\\indent$\begin{aligned}}{\end{aligned}$\\}
\newcommand{\defy}[2]{\!\!\!\!\!\!&&#1&\coloneqq#2\\}
%\newcommand{\defyc}[1]{&\phantom{\coloneqq}\ \ #1\\}
\newcommand{\defyc}[1]{\!\!\!\!\!\!\rlap{\quad \quad #1}&&\\}
\newcommand{\defya}[2]{#1&\!\!\!\!\!\!&\coloneqq#2\\}

%\newcommand{\prodFull}[3]{#1&::=&\mbox{#2}&\mbox{#3}}
\newcommand{\prodInline}[2]{#1\Coloneqq#2}
\newcommand{\terminal}[1]{\ensuremath{$\texttt{#1}$}}
%\newcommand{\metavariable}[1]{\ensuremath{\mathit{#1}}}

\newcommand{\Rulename}[1]{{\textsc{#1}}}
\newcommand{\ctx}[1]{\ensuremath{\mathcal{E}_#1}\!}
\newcommand{\libi}[2]{\Q@\{@\Q!interface!\ #1\Q{;} #2\Q@\}@}
\newcommand{\lib}[3]{\Q!interface!\ensuremath{?}\ \libc{#1}{#2}{#3}}
\newcommand{\libc}[3]{\,\Q@\{@\!#1\Q{;}\ #2 \Q{;}\ #3\Q@\}@\!\!}

\newcommand{\rp}[1]{\Q{(}\!#1\Q{)}}
\newcommand{\eq}[1]{\,\Q{=}#1}
\newcommand{\red}[3]{#1\,\Q{<}#2\eq#3\,\Q{>}}
\newcommand{\summ}[2]{#1\ \Q{<+}\ #2}
\newcommand{\from}[2]{#1\ensuremath{[}#2\ensuremath{]}}
\newcommand{\mmid}{{\ensuremath{{\mid}}}\!}
\newcommand{\hole}{\ensuremath{\square}}
\newcommand{\s}[1]{\ensuremath{\mathit{#1s}}}
\makeatletter
\newcommand{\This}[1]{\Q!This!#1\nextpath}
\newcommand{\Cs}[1]{#1\nextpath}
\newcommand{\nextpath}{\@ifnextchar\bgroup{\gobblenextpath}{}}
\newcommand{\gobblenextpath}[1]{\Q!.!#1\@ifnextchar\bgroup{\gobblenextpath}{}}
\makeatother



%--------------------------
\newcommand{\mynotes}[3]{{\color{#2} {\sc #1}: #3}}
\newcommand\isaac[1]{\mynotes{Isaac}{blue}{#1}}

\newcommand\IO[1]{\color{blue}{#1}}
\newcommand\marco[1]{\mynotes{Marco}{green}{#1}}


\providecommand*{\code}[1]{\Q`#1`}
\newcommand{\saveSpace}{\vspace{-3px}}
\newcommand{\loseSpace}{\vspace{1ex}}

\title{Dependency Extraction}
\author{Marco Servetto
\institute{ECS VUW\\ Wellington, New Zealand}
\institute{Engineering and Computer Science\\
Victoria University of Wellington}
\email{marco.servetto@ecs.vuw.ac.nz}
}
\def\titlerunning{Dependency Extraction}
\def\authorrunning{M. Servetto}
\begin{document}
\maketitle

\begin{abstract}
Dependency injection is a very heavy weight pattern,
where new interfaces have to be defined, and new classes
have to be added.
Here we propose an alternative, that have the same benefits
but with much less cost.
We call it Dependency Extraction.
\end{abstract}

\section{Context}
Testing sections of a large programs in isolation is very useful since it
allows to isolate the bugs and to test components while other components 
are not ready yet.
It also allows to connect bugs with the part of the program causing them.

When a code artefact starts to become large, and have many 
interconnected parts, testing parts in isolation becomes difficult.

Here we present a solution that works in Java8 (and further versions).
We think a minor variation of this pattern could be used on many other languages supporting generics and
either interfaces with default methods, multiple inheritance or trait composition.

Many aspects of the problem and the solution only apply to nominally typed object oriented programming languages, 
and much of the difficulty is into 
coding in such a style where small components 
can be compiled and type checked independently.

\section{Example problem}

A common motivating example for Dependency Injection is the following:
Alice and Bob are designing a videogame, where there
is a \Q@Map@ containing \Q@Item@s that are located in certain
\Q@Point@s.
Various kinds of \Q@Item@s exists: \Q@Wall@s, \Q@Rock@s,
\Q@Tree@s and many more.
Alice will implement the \Q@load@ functionality:
\begin{quote}
The \Q@load@ method will parse a formatted text file
and use the information to fill a \Q@Map@.
This requires to instantiate an empty map, and to instantiate various kinds of
\Q@Item@s defined by Bob.
\end{quote}
Bob will implement the rest of the game, this includes: 
\begin{quote}
The \Q@Map@ class, the interface \Q@Item@ and
all the classes implementing the \Q@Item@ interface.
To load maps from files, Bob need to cal the \Q@load@ function of Alice.
\end{quote}
Bob plans to make an efficient implementation of the \Q@Map@ as a tricky
sparse matrix.
Thus, bugs are likely to be present in the \Q@Map@ implementation.
The various \Q@Item@s will contains a lot of code related to the game logic.
It would be great if Alice could sidestep such bugs and such functionalities when
implementing her code.

\section{Forces}
\begin{itemize}
\item Alice code need to instantiate classes that are in Bob component.
\item Alice code need to call methods on classes that are are in Bob component.
\item The code of Alice need to compile and be testable when nothing
inside Bob component is available.
\end{itemize}

How can Alice test her code and check that it is correct independently 
of the code of Bob?
Ideally, she should be able to do so even before Bob have written
his part of code.




\section{The conventional solution}
In Java, we can use dependency injection to we write the code so that Alice can 
completely write and test \Q@load@ before
Bob completes the rest of the game.
This requires introducing a lot of boilerplate code, and 
to program in a very unnatural way:
Alice can not write \Q@new Map(..)@
since the \Q@Map@ class may not be available yet.
We need to introduce an \Q@IMap@ interface and an \Q@IMapFactory@ interface,
and Alice need to provide
a mock implementation designed for testing.
This makes the code of Alice very involved and indirect:
Alice \Q@load@ method needs to access an \Q@IMapFactory mapFactory@
and use it to create new maps, as in \Q@mapFactory.makeMap()@.
Same for \Q@Item@s: Alice can not write \Q@new Rock(...)@,
but she needs to use an \Q@IItemFactory@.
When \Q@load@ is called in the context of a running game,
the factories will create actual \Q@Map@s, \Q@Rock@s
and so on, but in the testing environment, 
mock factories will create mock versions.
While in Java this is very verbose and involved,
it still has great advantages over using the
possibly buggy \Q@Map@ of Bob: Alice can just make a \emph{logging}
implementation of \Q@Map@, that just store information about the added objects.

Professional programmers rely heavily on dependency
injection: it seams like the benefits overcome the costs.

Here in the following, you can find an example of how such Java code may look like.
Note how every class would be a file of its own.
Even if we tried to omit all the non fundamental bits,
the code is still very long, once the reader is convinced is just a mess, they
cans skip to the next section.
\begin{lstlisting}
//--common code, this need to be written beforehand ------------
class Point{..}//most of these require their own file
interface Map{Item get(Point that); void set(Item that);}
interface Item{ Point point(); Item hit();}//methods about the game logic
interface Rock extends Item {..}
interface Wall extends Item {..}
interface ItemFactory{Rock makeRock(Point point, int weight); ..}
interface MapFactory{Map makeMap();}

//--Bob code------------
interface BobRock extends Rock {Point point; int weight; ..}
interface BobWall extends Wall {Point point; int height; ..}

class BobMap implements Map{..//map implementation by Bob
  static Map empty(){..}
  Item get(Point that){..}
  Void set(Item that){..}
  }
class Game{
  static Void run()=..Loader.load(..).. //implemented by Bob
  }

//--Alice code----------------------------------
class MapLoader{
  ItemFactory items; MapFactory maps;
  MapLoader(ItemFactory i,MapFactory m){items=i;maps=m;}
  Map load(String fileName){..maps.makeMap()..}
  static Map load(String fileName){
    Map map=maps.makeMap();//we can not write BobMap.empty();
    //read from file and divide in lines,
    for(String line: lines) {load(map,line);}
    }
  static Void load(Map map,String line){
    //example line: S"Rock 23 in 12, 7"
    List<Integer> ns=readInts(line);
    if (line.startsWith("Rock"))
      map.set(items.makeRock(ns.get(0),new Point(ns.get(1),ns.get(2))));
    if (line.startsWith("Wall")) ..
    ..
    }
  }
//--Alice testing code-----------------------------
  
class MockMap implements Map{
  Item get(Point that){throw new Error("Not used by Alice code");}
  void set(Item that){..}
  String toString(){..}
  ..
  }
class MockMapFactory implements MapFactory{
  public Map makeMap(){return new MockMap();} }
class MockItemFactory implements ItemFactory{
  public Rock makeRock(..){return new MockRock(..);}
  public Wall makeWall(..){return new MockWall(..);} }
class MockRock implements Rock{..}
class MockWall implements Wall{..}

class Tester{
  static void test(String fileName,String expected){
    MapLoader m=new MapLoader(new MockMapFactory(),new MockItemFactory());
    Map map=m.load(fileName)
    assert map.toString().equals(expected);
  }}
\end{lstlisting}

\section{Liabilities}
Dependency injection passes the requirement exposed before, however it have many
drawbacks:
\begin{itemize}
\item It is very verbose.
\item A lot of interfaces must be defined ahead of time.
\item Those interface will contain all the needed features of those objects,
so for example the \Q@Rock@ interface will have all the methods all \Q@Item@s need
to have to satisfy the game logic of \Q@Bob@, while logically Alice only needed the constructor
and a \Q@toString@.
\item New fields have to be added to store factory objects.
\end{itemize}


\section{The proposed solution}
Alice can declare a generic interface, where generic parameters will encode
the dependencies, and abstract methods are used to create instances and calling
methods on those dependencies.
Here we show the complete code of Alice, it is compact enough that we do
not need to omit any details.

\begin{lstlisting}
interface Alice<Item, Rock extends Item, Wall extends Item, Map>{
  Map newMap();
  Rock newRock(int weight, Point p);
  Wall newWall(int height, Point p);
  void set(Map map, Item item);
  default Map load(String fileName) throws IOException{
    Map map=newMap();
    Files.lines(Paths.get(fileName))
      .forEach(line->load(map,line));
    return map;
    }
  default void load(Map map,String line) {
    //example line: S"Rock 23 in 12, 7"
    List<Integer> ns=readInts(line);
    if (line.startsWith("Rock")){
      set(map,newRock(ns.get(0),new Point(ns.get(1),ns.get(2))));}
    if (line.startsWith("Wall")){
      set(map,newWall(ns.get(0),new Point(ns.get(1),ns.get(2))));}
    /*..*/
    }
  default List<Integer>readInts(String line){
    try(Scanner s=new Scanner(line)){
      List<Integer> res=new ArrayList<>();
      while (s.hasNextInt()) {res.add(s.nextInt());}
      return res;
    }}
  }
\end{lstlisting}
As you can see, Alice just defines an interface with generic arguments
to encode all her dependencies with Bob code.
Methods defined inside the \Q@Alice@ interface allows to work with such 
generic arguments, that otherwise are just black boxes.
Note how subtyping requirements between dependencies are expressed
using F-Bound polymorphism.

Notice how the code of Alice is very similar to what she would have written
if she was not using dependency injection. The
only differences are as follow:
Instead of \Q@new Map()@ she have to write \Q@newMap()@; same for
\Q@Rock@, \Q@Wall@.
Instead of \Q@map.set(item)@ she have to write \Q@set(map,item)@.

And this is how Alice may test her code:
\begin{lstlisting}
//                               Item   Rock   Wall   Map
class AliceTest implements Alice<String,String,String,StringBuilder>{
  StringBuilder newMap(){return new StringBuilder();}
  String newRock(int weight, Point p){return "Rock:"+p+"->"+weight;}
  String newWall(int height, Point p){return "Wall:"+p+"->"+height;}
  void set(StringBuilder map, String item){map.append(item+"\n");}
  void test(String fileName, String expected){
    try {Assert.assertEquals(load(fileName).toString(),expected);}
    catch (IOException e) {throw new Error(e);}
    }
  @Test void test1(){test("justARock.txt","Rock:Point(5,6)->35\n");}
  @Test void test2(){..}
  ..
  }
\end{lstlisting}
At first, it may look bewildering that Alice is able to instantiate
\Q@Item@, \Q@Rock@ and \Q@Wall@ with \Q@String@ and \Q@Map@ with
\Q@StringBuilder@. However, this is perfect for a testing environment, where
the only need is to log what happened and then to check that the result
of the logs is what we expect.

Bob can write his code in a pretty conventional style too:
\begin{lstlisting}
interface Bob{
  Map load(String fileName) throws IOException;
  /*code using load, Map, Item, Rock, Wall and so on*/
  }
interface Map{Item get(Point that); void set(Item that);}
interface Item{ Point point(); Item hit();}//methods about the game logic
interface Rock extends Item {..}
interface Wall extends Item {..}
\end{lstlisting}
And this is how Alice and Bob can put their code together:

\begin{lstlisting}
class Game implements Bob, Alice<Item,Rock,Wall,Map>{
  Map newMap(){return new Map();}
  Rock newRock(int weight, Point p){return new Rock(weight,p);}
  Wall newWall(int height, Point p){return new Wall(height,p);}
  void set(Map map, Item item){map.set(item);}
  static void main(String args[]){new Game().start();}
  }
\end{lstlisting}

\section{Benefits}
Our solution is much more natural to use than conventional dependency injection.
\begin{itemize}
\item No need of factory classes.
\item No need to store factory objects in extra fields.
\item No need to define communication interfaces before the work can start.
\item Much shorter code.
\item Testing is much more direct and natural, there is no need
to wrap logger objects inside Mock objects implementing specific interfaces.
\item It is easy to refactor non modularized code into following this pattern.
\end{itemize}

\section{Liability: Implementing Generics}
Java do not support extending/implementing generic arguments, thus
certain code modularization would not be supported by this pattern;
for example if Bob wanted to let Tim implement some kinds of \Q@Item@s,
Tim code will need to see the \Q@Item@
 interface as declared in Bob code.
We do not think there is a work around to this issue.

\section{Liability: many classes relying on dependencies}
Our example looks quite nice since Alice is only defining a single
method, that conceptually lives in a top level object of a class that is
instantiated at the top level.

If she had to define many interconnected classes that have
to use \Q@Alice@ methods to access the generic arguments,
the situation gets more complicated.

To keep our explanation connected to the videogame example, let assume
Bob asks Tim to implement the \Q@Item@ interface and all the \Q@Item@s
classes.
There are two solutions to this problem: one is
to use a little bit of conventional dependency injection,
the other one is to use a form of singleton.
Here we show the revised code of Bob, that uses conventional dependency injection.
It is needed since the map needs an instance of \Q@Bob@ (or \Q@Game extends Bob@)
to extract the Point from the generic \Q@Item@.
\begin{lstlisting}
interface Bob<Item>{
  Map<Item> load(String fileName) throws IOException;
  Point getPoint(Item item);
  default void runGame() {}
}
class Map<Item>{
  Bob<Item> top;//this fields is a dependency, the map need Bob
  Item get(Point that) {return null;}
  void set(Item item) {/*..*/top.getPoint(item);/*..*/}
  List<Item>near(Point that){return null;}
}
\end{lstlisting}
Here instead we show the code of Tim.
He use a class \Q@Top@ that could be defined 
just once for all the project.
When the test or the game is run, \Q@Top.set(..)@ will be called, with either
just a \Q@Tim@ (for testing) or a complete \Q@Game extends Tim@ for
deployment.
\begin{lstlisting}
class Top{
  static Object instance;
  static void set(Object instance){Top.instance=instance;}
  @SuppressWarnings("unchecked")
  static <T> T instance(T obj){return (T) instance;}
  }
interface Tim<Map>{
  List<Item>near(Map map,Point that);
}
interface TimTop<Map> extends Tim<Map>{
  default List<Item>near(Map map,Point that){
    return Top.instance(this).near(map, that);}
}
interface Item{Point p();/*..*/}
class Rock<Map> implements Item, TimTop<Map>{
  Point p;public Point p() {return p;}
  public void interact(Map map) {/*..*/near(map,p);/*..*/}
  }
\end{lstlisting}
The idea is that all the concrete \Q@Item@s will implement \Q@TimTop@.
In this way, they will be able to call all the methods of \Q@Tim@ over \Q@this@,
without the need of having any additional fields.
Of course this work around have some safety issues, since we need to somehow ensure
\Q@Top.instance@ contains the right value;
moreover it relies on a static fields, so we are performing an ugly
global set up that influence the whole running application.

\section{Related patterns}
\begin{itemize}
\item Dependency Injection.
\item Abstract Factory.
\item Mocking.
\end{itemize}

\end{document}