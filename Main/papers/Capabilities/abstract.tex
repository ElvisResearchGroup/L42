%% LyX 2.3.1 created this file.  For more info, see http://www.lyx.org/.
%% Do not edit unless you really know what you are doing.
\documentclass[10pt,twoside,british]{scrartcl}
\usepackage{amstext}
\usepackage{fontspec}
\usepackage{unicode-math}
\setmainfont[Mapping=tex-text]{TeX Gyre Schola}
\setsansfont[Scale=1.04,Mapping=tex-text]{Latin Modern Sans}
\setmonofont[Scale=1.09]{Latin Modern Mono}
\usepackage[a4paper]{geometry}
\geometry{verbose,tmargin=2.54cm,bmargin=2.54cm,lmargin=2.54cm,rmargin=2.54cm,headheight=0.7cm,headsep=0.7cm,footskip=1.05cm}
\setlength{\parskip}{\medskipamount}
\setlength{\parindent}{0pt}
\usepackage[xetex]{xcolor}
\usepackage{calc}
\usepackage{endnotes}

\makeatletter
%%%%%%%%%%%%%%%%%%%%%%%%%%%%%% Textclass specific LaTeX commands.
\let\footnote=\endnote

\@ifundefined{date}{}{\date{}}
%%%%%%%%%%%%%%%%%%%%%%%%%%%%%% User specified LaTeX commands.
\usepackage{xcolor}
\usepackage{scrlayer-scrpage}

\newcommand{\lfoot}[1]{\ifoot{\textnormal{#1}}}
\newcommand{\rfoot}[1]{\ofoot{\textnormal{#1}}}

\definecolor{AccentB80}{RGB}{197, 213, 255}
\definecolor{AccentB60}{RGB}{139, 171, 255}
\definecolor{AccentB40}{RGB}{81,  130, 255}
\definecolor{Accent}   {RGB}{0,   63,  221}
\definecolor{AccentD25}{RGB}{0,   46,  165}
\definecolor{AccentD50}{RGB}{0,   31,  110}

%\renewcommand\Huge{\@setfontsize\Huge{10pt}{26}} % Was 25?
%\renewcommand\Large{\@setfontsize\Large{10pt}{14}} % Was 14?
%\large	 10.5 % was 11?
\newcommand{\StyleTitle}[1]{{\Huge{\textcolor{Accent}{{#1}}}}} % Expanded +0.5pt
\newcommand{\StyleSubTitle}[1]{{\large{\textcolor{AccentB40}{{#1}}}\medskip}} % 25pt after , Expanded +0.5pt
\renewcommand{\maketitle}{\StyleTitle{\@title}\\\StyleSubTitle{\@author}}
\newcommand{\zwnbsp}{}
%\renewcommand{\ensuremath}[1]{#1}

\renewcommand{\thanks}[1]{\footnote{#1}}
\definecolor{blue}{HTML}{0000F0} % 
\definecolor{purple}{HTML}{700090}
\definecolor{orange}{HTML}{F07000}
\definecolor{teal}{HTML}{0090B0}
\definecolor{brown}{HTML}{A00000}
\definecolor{green}{HTML}{008000}
\definecolor{pink}{HTML}{F000F0}

\usepackage{enumitem}
\setlist{nolistsep}

\usepackage{fontspec} % To set xetex fonts
\setmainfont[Ligatures={Common, Discretionary}]{TeX Gyre Schola}
\setmonofont[Ligatures={Discretionary}, Scale=1.090909090909090909090909]{Latin Modern Mono} % 12/11
\setsansfont[Numbers={Monospaced},Ligatures={Common, Discretionary}, Scale=1.045454545454545]{Latin Modern Sans} %11.5/11

\unimathsetup{math-style=ISO}
\setmathfont{TeX Gyre Schola Math}

\newcommand{\parj}{\setlength{\parskip}{0pt}}
\newcommand{\pars}{\setlength{\parskip}{\medskipamount}}
\usepackage[para]{footmisc}
\usepackage{hyperref}
\hypersetup{colorlinks,urlcolor=[RGB]{0, 155, 240}}
\newcommand{\email}[1]{%
	\href{mailto:#1}{\texttt{#1}}
}
\raggedright

\makeatother

\usepackage{polyglossia}
\setdefaultlanguage[variant=british]{english}
\begin{document}
\title{Callability Control}
\author{By Isaac Oscar Gariano\textsuperscript{1} and Marco Servetto\textsuperscript{2}
(Victoria University of Wellington)}

\maketitle
\global\long\def\empty{\zwnbsp}%

\global\long\def\k#1{\textcolor{blue}{\texttt{#1}}}%
\global\long\def\t#1{\textcolor{teal}{#1}}%
\global\long\def\f#1{\textcolor{purple}{#1}}%
\global\long\def\l#1{\textcolor{brown}{#1}}%
\global\long\def\v#1{\textcolor{orange}{#1}}%
\global\long\def\m#1{\textcolor{green}{#1}}%

\global\long\def\c#1{\texttt{#1}}%
\global\long\def\ck#1{\c{\k{#1}}}%
\global\long\def\ct#1{\c{\t{#1}}}%
\global\long\def\cf#1{\c{\f{#1}}}%
\global\long\def\cv#1{\c{\v{#1}}}%
\global\long\def\cm#1{\c{\m{#1}}}%

\global\long\def\tab{\texttt{\ \ \ \ }}%

\global\long\def\calls#1{\ck{calls[}#1\ck ]}%

How does one know whether a called function will access arbitrary
files on your system? One common option is to say that if the called
function is not declared impure (e.g. with Haskell’s $\ct{IO}$
monad) it will not read any files (or do any other I/O). Another option
is to raise a runtime error or trap if it tries to do so (e.g. with
Java’s $\ct{SecurityManager}$). The first option is overly restrictive
and heavy weight (what if the function only needs access to standard
out?) whilst the second hampers static and modular reasoning. We call
the operations a function may call its \emph{callability}; we propose
a simple and flexible type-system feature that allows one to soundly
statically reason about functions callability on a fine-grained level.
Importantly our system only restricts \emph{calling} functions, it
does not restrict what \emph{objects} can be created, passed around,
or aliased.

In our systems, all functions are annotated with a base-callability
using the syntax $\calls{\f{\f x}_{1},\dots,\f x_{n}}$, where each
$\f x_{i}$ indicates the name of a function (or set of functions)
that can be called. Note how here a function declares what \emph{it
can access}, compare this with conventional accessibility, (like Scala’s
$\k{private[}\t C\k ]$ modifier) where a function declares what \emph{can
access it}. Our system has three core rules:\parj{}
\begin{itemize}
\item Any function can call call itself.
\item A function can call each function in its base-callability.
\item If a function $\f f$ can call everything in the base-callability
of a function $\f g$, then $\f f$ can call $\f g$.
\end{itemize}
\pars{}With these rules harmless language/library primitives/intrinsics
(such as integer addition) would be marked with $\calls{\empty}$,
allowing any function to call it. On the other hand, if a primitive
$\f f$ should be restricted (e.g. a function to exit the program),
one could annotate it with $\calls{\f f}$, thus allowing only specifically
marked functions to call it. Other systems can be envisioned, such
as having a dummy $\cf{io}$ function, and than marking all I/O primitives
with $\calls{\cf{io}}$.

We also support callability generics and dynamic dispatch, thus allowing
the system to become quite flexible; consider for example the following
(where $\cf{'a}$ denotes a generic callability parameter):\parj{}

\fbox{\begin{minipage}[t]{1\columnwidth - 2\fboxsep - 2\fboxrule}%
$\texttt{\ensuremath{\ck{static}} \ensuremath{\ck{extern}} \ensuremath{\ct{long}} \ensuremath{\cf{posix\_read(}\ct{int}} \ensuremath{\cv{file\_descriptor}}, \ensuremath{\ct{void*}} \ensuremath{\cv{buffer}}, \ensuremath{\ct{ulong}} \ensuremath{\cv{count}\cf )}}$\\
$\texttt{\ensuremath{\tab\calls{\cf{posix\_read}}}; \ensuremath{\cm{// system function defined by the OS}}}$\\
$\texttt{\ensuremath{\ck{static}} \ensuremath{\ct{long}} \ensuremath{\cf{stdin\_read(}\ct{void*}} \ensuremath{\cv{buffer}}, \ensuremath{\ct{ulong}} \ensuremath{\cv{count}\cf )} \ensuremath{\calls{\cf{posix\_read}}} \{}$\\
$\texttt{\ensuremath{\tab\ck{return}} \ensuremath{\cf{posix\_read(}\cv{STDIN\_FILENO}}, \ensuremath{\cv{buffer}}, \ensuremath{\cv{count}\cf )}; \}}$\\
\\
$\texttt{\ensuremath{\ck{interface}} \ensuremath{\ct{Input\_Stream<}\cf{'a}\ct >} \{}$\\
$\texttt{\ensuremath{\tab\ct{char}} \ensuremath{\cf{get\_char()}} \ensuremath{\calls{\cf{'a}}};}$\\
$\texttt{\ensuremath{\tab}… \} \ensuremath{\cm{// other usefull functions}}}$\\
$\texttt{\ensuremath{\ck{static}} \ensuremath{\ct{void}} \ensuremath{\cf{do\_stuff<'a>(}\ct{Input\_Stream<}\cf{'a}\ct >} \ensuremath{\cv{stream}\cf )} \ensuremath{\calls{\cf{'a}}} \{}$\\
$\tab\texttt{… \} \ensuremath{\cm{//}} \ensuremath{\cm{could have any well-typed code}}}$%
\end{minipage}}

\pars{}$\c{\t{Input\_Stream<}\f{'a}\t >}$ is conceptually similar
to a generic $\c{\t{Input\_Stream<T>}}$, where $\ct T$ denotes a
type-parameter.

For any base-callability $\f c$ one can then call $\cf{do\_stuff<}\f c\cf >$and
pass it an instance of a class that implements $\ct{\t{Input\_Stream<}}\f c\ct{\t >}$.
For example, a call $\cf{do\_stuff<[]>(}\v{a\_stream}\cf )$ cannot
(even indirectly) read from a file, since it doesn’t have the callability
$\cf{posix\_read}$ or any callability which could indirectly call
it. Another important benefit of our system is that we can even reason
over a call like $\cf{do\_stuff<[stdin\_read]>(}\v{a\_stream}\cf )$,
it can read-files, but only standard-input (the file identified by
$\cv{STDIN\_FILENO}$). As is usual in OO languages, for such a call
to type-check, $\v{a\_stream}$’s static-type must be a subtype
of $\ct{Input\_Stream<}\cf{[stdin\_read]}\ct >$ but one could always
perform a run-time cast, but risk getting an exception.

This system is designed to soundly enforce reasoning: by looking at
the declarations of functions $\f f$ and $\f g$, and all functions
transitively referenced in their base-capabilities, one can statically
determine whether $\f f$ can call $\f g$. In addition, it is also
designed to work in an environment with dynamic-code loading, if you
have reasoned that one function cannot call another, no matter what
additional code you load or dynamically invoke, that guarantee still
holds. We have also looked at adding other features to reduce the
burden of our annotations: such as wildcards and annotation inference.

\lfoot{\textsuperscript{1}\email{isaac@ecs.vuw.ac.nz} \textsuperscript{2}\email{marco.servetto@ecs.vuw.ac.nz} }

\rfoot{}
\end{document}
