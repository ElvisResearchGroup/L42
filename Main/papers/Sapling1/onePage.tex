\documentclass[a4paper,twoside,british,9pt]{extarticle}
\usepackage[a4paper,bindingoffset=0ex,%
            left=15ex,right=15ex,top=10ex,bottom=15ex,%
            footskip=5ex]{geometry}
\setlength{\parskip}{\medskipamount}
\setlength{\parindent}{0pt}
\usepackage{listings}
\usepackage{xcolor}
\usepackage{letltxmacro}
\usepackage{mathtools}
\usepackage{mathpartir}
%\usepackage{stix}

\definecolor{darkRed}{RGB}{100,0,10}
\definecolor{darkBlue}{RGB}{10,0,100}
\newcommand*{\ttfamilywithbold}{\fontfamily{pcr}\selectfont}
%\newcommand*{\ttfamilywithbold}{\ttfamily}

%found on http://tex.stackexchange.com/questions/4198/emphasize-word-beginning-with-uppercase-letters-in-code-with-lstlisting-package
%\lstset{language=FortyTwo,identifierstyle=\idstyle}
%
\makeatletter
\newcommand*\idstyle{%
        \expandafter\id@style\the\lst@token\relax
}
\def\id@style#1#2\relax{%
        \ifcat#1\relax\else
                \ifnum`#1=\uccode`#1%
                        \ttfamilywithbold\bfseries
                \fi
        \fi
}
\makeatother

\lstset{language=Java,
  basicstyle=\upshape\ttfamily\footnotesize,%\small,%\scriptsize,
  keywordstyle=\upshape\bfseries\color{darkRed},
  showstringspaces=false,
  mathescape=true,
  xleftmargin=0pt,
  xrightmargin=0pt,
  breaklines=false,
  breakatwhitespace=false,
  breakautoindent=false,
 identifierstyle=\idstyle,
 morekeywords={method,Use,This,constructor,as,into,rename},
 deletekeywords={double},
 literate=
  {\%}{{\mbox{\textbf{\%}}}}1
  {~} {$\sim$}1
%  {<}{$\langle$}1
%  {>}{$\rangle$}1
}

\newcommand*{\SavedLstInline}{}
\LetLtxMacro\SavedLstInline\lstinline
\DeclareRobustCommand*{\lstinline}{%
	\ifmmode
	\let\SavedBGroup\bgroup
	\def\bgroup{%
		\let\bgroup\SavedBGroup
		\hbox\bgroup
	}%
	\fi
	\SavedLstInline
}

\newcommand\saveSpace{\vspace{-2pt}}

\newcommand\Rotated[1]{\begin{turn}{90}\begin{minipage}{12em}#1\end{minipage}\end{turn}}

\newcommand{\Q}{\lstinline}
\newenvironment{bnf}{$\begin{aligned}}{\end{aligned}$}
\newcommand{\production}[3]{\textit{#1}&\Coloneqq\textit{#2}&\text{#3}}
\newcommand{\prodNextLine}[2]{&\quad\quad\textit{#1}&\text{#2}}
\newenvironment{defye}{\\\indent$\begin{aligned}}{\end{aligned}$\\}
\newcommand{\defy}[2]{\!\!\!\!\!\!&&#1&\coloneqq#2\\}
%\newcommand{\defyc}[1]{&\phantom{\coloneqq}\ \ #1\\}
\newcommand{\defyc}[1]{\!\!\!\!\!\!\rlap{\quad \quad #1}&&\\}
\newcommand{\defya}[2]{#1&\!\!\!\!\!\!&\coloneqq#2\\}

%\newcommand{\prodFull}[3]{#1&::=&\mbox{#2}&\mbox{#3}}
\newcommand{\prodInline}[2]{#1\Coloneqq#2}
\newcommand{\terminal}[1]{\ensuremath{$\texttt{#1}$}}
%\newcommand{\metavariable}[1]{\ensuremath{\mathit{#1}}}

\newcommand{\Rulename}[1]{{\textsc{#1}}}
\newcommand{\ctx}[1]{\ensuremath{\mathcal{E}_#1}\!}
\newcommand{\libi}[2]{\Q@\{@\Q!interface!\ #1\Q{;} #2\Q@\}@}
\newcommand{\lib}[3]{\Q!interface!\ensuremath{?}\ \libc{#1}{#2}{#3}}
\newcommand{\libc}[3]{\,\Q@\{@\!#1\Q{;}\ #2 \Q{;}\ #3\Q@\}@\!\!}

\newcommand{\rp}[1]{\Q{(}\!#1\Q{)}}
\newcommand{\eq}[1]{\,\Q{=}#1}
\newcommand{\red}[3]{#1\,\Q{<}#2\eq#3\,\Q{>}}
\newcommand{\summ}[2]{#1\ \Q{<+}\ #2}
\newcommand{\from}[2]{#1\ensuremath{[}#2\ensuremath{]}}
\newcommand{\mmid}{{\ensuremath{{\mid}}}\!}
\newcommand{\hole}{\ensuremath{\square}}
\newcommand{\s}[1]{\ensuremath{\mathit{#1s}}}
\makeatletter
\newcommand{\This}[1]{\Q!This!#1\nextpath}
\newcommand{\Cs}[1]{#1\nextpath}
\newcommand{\nextpath}{\@ifnextchar\bgroup{\gobblenextpath}{}}
\newcommand{\gobblenextpath}[1]{\Q!.!#1\@ifnextchar\bgroup{\gobblenextpath}{}}
\makeatother



%--------------------------
\newcommand{\mynotes}[3]{{\color{#2} {\sc #1}: #3}}
\newcommand\isaac[1]{\mynotes{Isaac}{blue}{#1}}

\newcommand\IO[1]{\color{blue}{#1}}
\newcommand\marco[1]{\mynotes{Marco}{green}{#1}}


\providecommand*{\code}[1]{\Q`#1`}


\begin{document}

\title{Nested Traits composition for modular software development}
\date{}
\maketitle
\vspace{-10ex}
Dependency Injection is a common but verbose technique
used to divide code OO code into independently testable units.
Here we propose how an extension to the Trait model,
supporting independently testable units without
the verbosity of Dependency injection.
We present our solution in the language L42.

Traits are units of code reuse, traditionally containing
only abstract and implemented methods.
When two traits \Q@t1@ and \Q@t2@ are composed, the resulting trait contains
all the methods of \Q@t1@ and \Q@t2@.
It is an error if a method \Q@m@ is implemented in both \Q@t1@ and \Q@t2@.
On the other side, an abstract method can be composed with
another abstract or implemented method.
In former work we extended traits to include nested classes, with the intuitive
idea that when two traits are composed, nested classes with the same name are recursively composed.


A common motivating example for Dependency Injection is the following:
Alice and Bob are designing a videogame, where there
is a \Q@Map@ containing \Q@Item@s that are located in certain
\Q@Point@s.
Various kinds of \Q@Item@s exists: \Q@Wall@s, \Q@Rock@s,
\Q@Tree@s and many more. Alice will implement the \Q@load@ functionality:
\Q@load@ will parse a formatted text file
and use the information to fill a \Q@Map@.
Bob will implement the rest of the game, 
including all the kinds of \Q@Item@s and
the \Q@Map@.
An efficient implementation of the \Q@Map@ is tricky,
since it needs to be a sparse matrix.
Thus, bugs are likely to be present in the \Q@Map@ implementation.
The various \Q@Item@s will contains a lot of code related to the game logic.

In Java, we can use dependency injection to we write the code so that Alice can 
completely write and test \Q@load@ before
Bob completes the rest of the game.
This requires to introduce a lot of boilerplate code, and 
to program in a very unnatural way:
Alice can not write \Q@new Map(..)@
since the \Q@Map@ class may not be available yet.
We need to introduce an \Q@IMap@ interface and an \Q@IMapFactory@ interface,
and Alice need to provide
a mock implementation designed for testing.
This makes the code of Alice very involved and indirect:
Alice \Q@load@ method need to access a \Q@IMapFactory mapFactory@
and use it to create new maps, as in \Q@mapFactory.makeMap()@.
Same for \Q@Item@s: Alice can not write \Q@new Rock(...)@,
but she need to use an \Q@IItemFactory@.
When \Q@load@ is called in the context of a running game,
the factories will create actual \Q@Map@s, \Q@Rock@s
and so on, but in the testing environment, 
mock factories will create mock versions.
While in Java this is very verbose and involved,
it still has great advantages over using the
possibly buggy \Q@Map@ of Bob: Alice can 
write a simple but inefficient version of map relying
on a \Q@HashMap<Point,Item>@.
The industry rely on dependency injection a lot, proving
that the benefits overcome the difficulties.

We would like to present the Java code for this example, but
Dependency Injection make it so verbose that it would
never fit into 1 page. On the other side, our solution 
is quite compact and is presented below:
\vspace{-1ex}
\begin{lstlisting}
Alice: Trait({//this is the code wrote by Alice.
  Map: {class method Map()   method Void set(Item that)}//Map has two abstract methods
  Item: {interface} //Alice just declares all the abstract signature she need.
  Rock: {implements Item     class method Rock(Num weight, Point point) }
  class method Map load(S fileName)={
    Map m=Map() //L42 do not use 'new' but instantiate objects calling
    ..//static methods with empty names, as it happens in Python and other languages.
    if line.startsWith(S"Rock") (map.set(Rock(weight:.., point: Point(x:..,y:..))))
    ..//Map, Item and Rock are local declaration in Alice code.
    }//Trait composition & merges the members with the same name.
  })//This will include nested classes Map, Item and Rock present in Bob trait.
Game: Alice & Bob & {}          
\end{lstlisting}

And this is how Alice may test her code:
\begin{lstlisting}
AliceMock: Alice & {
  Item: {interface     Point point}
  Rock: Data <>< {implements Item     Num weight}
  Map: Trait(Collections.hashmap(key: Point, val: Item)) & {
    method Void set(Item that)=this.put(key: that.point(),val: that)
    }
  class method Void test(S fileName,S expected)={
    map=this.load(fileName: fileName)
    Debug.test(map, expected: expected)
    }
Test1: AliceMock.test(S"justARock.txt",S"HashMap[Point(..)->Rock(..)]"
Test2: ..
..
\end{lstlisting}
\vspace{-1ex}
Concluding, our solution is much more natural to use than conventional Dependency injections,
and allows to write code more naturally, without using factories.

\end{document}
