\usepackage{listings}
\usepackage{xcolor}
\usepackage{letltxmacro}
\usepackage{mathtools}
\usepackage{mathpartir}
%\usepackage{stix}

\definecolor{darkRed}{RGB}{100,0,10}
\definecolor{darkBlue}{RGB}{10,0,100}
\newcommand*{\ttfamilywithbold}{\fontfamily{pcr}\selectfont}
%\newcommand*{\ttfamilywithbold}{\ttfamily}

%found on http://tex.stackexchange.com/questions/4198/emphasize-word-beginning-with-uppercase-letters-in-code-with-lstlisting-package
%\lstset{language=FortyTwo,identifierstyle=\idstyle}
%
\makeatletter
\newcommand*\idstyle{%
        \expandafter\id@style\the\lst@token\relax
}
\def\id@style#1#2\relax{%
        \ifcat#1\relax\else
                \ifnum`#1=\uccode`#1%
                        \ttfamilywithbold\bfseries
                \fi
        \fi
}
\makeatother

\lstset{language=Java,
  basicstyle=\upshape\ttfamily\footnotesize,%\small,%\scriptsize,
  keywordstyle=\upshape\bfseries\color{darkRed},
  showstringspaces=false,
  mathescape=true,
  xleftmargin=0pt,
  xrightmargin=0pt,
  breaklines=false,
  breakatwhitespace=false,
  breakautoindent=false,
 identifierstyle=\idstyle,
 morekeywords={then,This,This0,This1,This2,This3,This4,This5},
 %deletekeywords={double},
 literate=
  {\%}{{\mbox{\textbf{\%}}}}1
  {~} {$\sim$}1
%  {<}{$\langle$}1
%  {>}{$\rangle$}1
}

\newcommand*{\SavedLstInline}{}
\LetLtxMacro\SavedLstInline\lstinline
\DeclareRobustCommand*{\lstinline}{%
	\ifmmode
	\let\SavedBGroup\bgroup
	\def\bgroup{%
		\let\bgroup\SavedBGroup
		\hbox\bgroup
	}%
	\fi
	\SavedLstInline
}

\newcommand\saveSpace{\vspace{-2pt}}

\newcommand\Rotated[1]{\begin{turn}{90}\begin{minipage}{12em}#1\end{minipage}\end{turn}}

\newcommand{\Q}{\lstinline}
\newenvironment{bnf}{$\begin{aligned}}{\end{aligned}$}
\newcommand{\production}[3]{\textit{#1}&\Coloneqq\textit{#2}&\text{#3}}
\newcommand{\prodNextLine}[2]{&\quad\quad\textit{#1}&\text{#2}}

%
%\newcommand{\pushright}[1]{\ifmeasuring@#1\else\omit\hfill$\displaystyle#1$\fi\ignorespaces}
%

%\llap{\text{#3}} 

\usepackage{array,tabularx}

% Version that aligns everything on the :=
%\newenvironment{defs}{\setlength{\tabcolsep}{0pt}\tabularx{\textwidth}{r>{}l>{\hfill}X}}{\endtabularx}
%\newcommand{\defi}[3]{\ensuremath{#1\,}&\ensuremath{\coloneqq#2}&\llap{\text{#3}}\\}


\newenvironment{defye}{\\\indent$\begin{aligned}}{\end{aligned}$\\}
\newcommand{\defy}[2]{\!\!\!\!\!\!&&#1&\coloneqq#2\\}
%\newcommand{\defyc}[1]{&\phantom{\coloneqq}\ \ #1\\}
\newcommand{\defyc}[1]{\!\!\!\!\!\!\rlap{\quad \quad #1}&&\\}
\newcommand{\defya}[2]{#1&\!\!\!\!\!\!&\coloneqq#2\\}

%\newcommand{\prodFull}[3]{#1&::=&\mbox{#2}&\mbox{#3}}
\newcommand{\prodInline}[2]{#1\Coloneqq#2}
\newcommand{\terminal}[1]{\ensuremath{$\texttt{#1}$}}
%\newcommand{\metavariable}[1]{\ensuremath{\mathit{#1}}}

\newcommand{\Rulename}[1]{{\textsc{(#1)}}}
\newcommand{\ctx}[1]{\ensuremath{\mathcal{E}_#1}\!}
\newcommand{\libi}[2]{\Q@\{@\Q!interface!\ #1\Q{;} #2\Q@\}@}
\newcommand{\libc}[3]{\Q@\{@#3\Q{;}\ #1\Q{;}\ #2\,\Q@\}@\!}
\newcommand{\lib}[3]{\libc{#2}{#3}{#1}}

\newcommand{\nc}[2]{\Q{private}?\,#1\Q{=}#2}
\newcommand{\cd}[2]{#1\Q{=}#2}
\newcommand{\me}[4]{\Q{static}\ensuremath{?} #1\ #2\rp{#3}\ #4}

\newcommand{\DVs}{\textit{DVs}}
\newcommand{\DV}{\textit{DV}}
\newcommand{\Ds}{\textit{Ds}}
\newcommand{\DLs}{\textit{DLs}}



%\newcommand{\red}[3]{#1\,\Q{<}#2\eq#3\,\Q{>}}


\newcommand{\opName}[1]{{\ensuremath{\footnotesize{\textbf{#1}}}}}
%\newcommand{\from}[2]{#1\op{from}{#2}}
\newcommand{\mmid}{{\ensuremath{{\mid}}}\!}
\newcommand{\hole}{\ensuremath{\square}}
\newcommand{\s}[1]{\ensuremath{\mathit{#1s}}}

\makeatletter



\makeatother
%--------------------------

\newcommand{\NoteColour}[2]{\textcolor{#1}{#2}}
\newcommand{\NoteText}[1]{{#1}}
\newcommand{\NoteComm}[3]{\NoteText{\scriptsize \textcolor{#1}{[{\sc #2}{:} #3]}}}
\newcommand{\NoteDel}[2]{\NoteText{\textcolor{#1}{[#2]}}}

\newcommand{\HideNotes}{%
	\renewcommand{\NoteColour}[2]{##2}%
	
	% Makes \NoteText give the same 'spacing' as a space character
	% i.e. preceding or suceeding this command by any number of spaces
	% should be equivalent to a single space
	\renewcommand{\NoteText}[1]{\unskip\space\ignorespaces}
}

\newcommand{\EZ}[1]{\NoteColour{blue}{#1}}
\newcommand{\EZComm}[1]{\NoteComm{blue}{Elenna}{#1}}
\newcommand{\MS}[1]{\NoteColour{green}{#1}}
\newcommand{\MSComm}[1]{\NoteComm{green}{Marco}{#1}}
\newcommand{\IO}[1]{\NoteColour{blue}{#1}}
\newcommand{\IOComm}[1]{\NoteComm{blue}{Isaac}{#1}}
\newcommand{\IODel}[1]{\NoteDel{blue}{#1}}
