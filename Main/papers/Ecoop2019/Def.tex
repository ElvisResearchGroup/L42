\usepackage{listings}
\usepackage{xcolor}
\usepackage{letltxmacro}
\usepackage{mathtools}
\usepackage{mathpartir}
%\usepackage{stix}

\definecolor{darkRed}{RGB}{100,0,10}
\definecolor{darkBlue}{RGB}{10,0,100}
\newcommand*{\ttfamilywithbold}{\fontfamily{pcr}\selectfont}
%\newcommand*{\ttfamilywithbold}{\ttfamily}

%found on http://tex.stackexchange.com/questions/4198/emphasize-word-beginning-with-uppercase-letters-in-code-with-lstlisting-package
%\lstset{language=FortyTwo,identifierstyle=\idstyle}
%
\makeatletter
\newcommand*\idstyle{%
        \expandafter\id@style\the\lst@token\relax
}
\def\id@style#1#2\relax{%
        \ifcat#1\relax\else
                \ifnum`#1=\uccode`#1%
                        \ttfamilywithbold\bfseries
                \fi
        \fi
}
\makeatother

\lstset{language=Java,
  basicstyle=\upshape\ttfamily\footnotesize,%\small,%\scriptsize,
  keywordstyle=\upshape\bfseries\color{darkRed},
  showstringspaces=false,
  mathescape=true,
  xleftmargin=0pt,
  xrightmargin=0pt,
  breaklines=false,
  breakatwhitespace=false,
  breakautoindent=false,
 identifierstyle=\idstyle,
 morekeywords={method,Use,This,constructor,as,into,rename},
 deletekeywords={double}
}

\newcommand*{\SavedLstInline}{}
\LetLtxMacro\SavedLstInline\lstinline
\DeclareRobustCommand*{\lstinline}{%
	\ifmmode
	\let\SavedBGroup\bgroup
	\def\bgroup{%
		\let\bgroup\SavedBGroup
		\hbox\bgroup
	}%
	\fi
	\SavedLstInline
}

\newcommand\saveSpace{\vspace{-2pt}}

\newcommand\Rotated[1]{\begin{turn}{90}\begin{minipage}{12em}#1\end{minipage}\end{turn}}

\newcommand{\Q}{\lstinline}
\newenvironment{bnf}{$\begin{aligned}}{\end{aligned}$}
\newcommand{\production}[3]{\textit{#1}&\Coloneqq\textit{#2}&\text{#3}}

\newenvironment{defye}{\\\indent$\begin{aligned}}{\end{aligned}$\\}
\newcommand{\defy}[2]{\!\!\!\!\!\!&&#1&\coloneqq#2\\}
%\newcommand{\defyc}[1]{&\phantom{\coloneqq}\ \ #1\\}
\newcommand{\defyc}[1]{\!\!\!\!\!\!\rlap{\quad \quad #1}&&\\}
\newcommand{\defya}[2]{#1&\!\!\!\!\!\!&\coloneqq#2\\}

%\newcommand{\prodFull}[3]{#1&::=&\mbox{#2}&\mbox{#3}}
\newcommand{\prodInline}[2]{#1\Coloneqq#2}
\newcommand{\prodNextLine}[2]{&&#1&\mbox{#2}}
\newcommand{\terminal}[1]{\ensuremath{$\texttt{#1}$}}
%\newcommand{\metavariable}[1]{\ensuremath{\mathit{#1}}}

\newcommand{\Rulename}[1]{{\textsc{#1}}}
\newcommand{\ctx}[1]{\ensuremath{\mathcal{E}_#1}\!}
\newcommand{\libi}[3]{\Q!interface!\ \libc{#1}{#2}{#3}}
\newcommand{\lib}[3]{\Q!interface!\ensuremath{?}\ \libc{#1}{#2}{#3}}
\newcommand{\libc}[3]{\Q@\{@\!#1\Q{;}\ #2 \Q{;}\ #3\Q@\}@\!\!}

\newcommand{\rp}[1]{\Q{(}\!#1\Q{)}}
\newcommand{\eq}[1]{\,\Q{=}#1}
\newcommand{\red}[3]{#1\rp{#2\eq#3}}
\newcommand{\summ}[2]{#1\ \Q{<+}\ #2}
\newcommand{\from}[2]{#1\ensuremath{[}#2\ensuremath{]}}
\newcommand{\mmid}{{\ensuremath{{\mid}}}\!}
\newcommand{\hole}{\ensuremath{\square}}
\newcommand{\s}[1]{\ensuremath{\mathit{#1s}}}
\makeatletter
\newcommand{\This}[1]{\Q!This!#1\nextpath}
\newcommand{\Cs}[1]{#1\nextpath}
\newcommand{\nextpath}{\@ifnextchar\bgroup{\gobblenextpath}{}}
\newcommand{\gobblenextpath}[1]{\Q!.!#1\@ifnextchar\bgroup{\gobblenextpath}{}}
\makeatother



%--------------------------
\newcommand{\mynotes}[3]{{\color{#2} {\sc #1}: #3}}
\newcommand\isaac[1]{\mynotes{Isaac}{blue}{#1}}

\newcommand\IO[1]{\color{blue}{#1}}
\newcommand\marco[1]{\mynotes{Marco}{green}{#1}}

