\section{Case study: evaluation in the 42 language}
\label{s:study}
42
is an ambitious language, aiming to 
 allow thousands of libraries to work together in a safe and maintainable manner.
 To this aim
 the actual 42 syntax is a little different from the one presented here, that is focused on making IC easier to understand to a reader experienced in Java.
 
For example, operations are properly packaged under libraries: indeed most operators
are accessed under \Q@Refactor@, as in \Q@Refactor.rename(...)@.
The super call mechanism is different: instead of user specified names (as \Q@superToS1()@), override
uses conventional names: for \Q@toS()@ they would be \Q@#1toS()@ and \Q@#2toS()@.

Another very important difference is that 42 sightly relaxes the requirement that all library literals are well typed whenever they are involved in execution.
 This carefully designed relaxation does not weaken the formal properties of the language, and provides two advantages:
 it facilitates the use of mutually recursive type declarations, as very common in OO languages, and it allows 
 to omit some abstract method declarations.
%A clear explanation of this technique would deserve a paper of its own. 

The core primitive operators include symmetric sum of code, redirect and rename.
Programmers rarely use such operators directly, favouring derived and more high level operators, similar
to \Q@Override@ as presented in this paper.

42 is \textbf{designed around meta-programming}, where
IC is the \textbf{only} way for code reuse and code adaptation.



Large practical experiments give us confidence that IC can be successfully and conveniently used
to replace both QQ and conventional core reuse features, as \Q@extends@ or generics.
An (anonymized) 42 tutorial can be found at \url{l42.is/tutorial.xhtml}, while
the GitHub project (not anonymized) can be reached at \url{github.com/ElvisResearchGroup/L42}.

\noindent
At URL \url{github.com/ElvisResearchGroup/L42/tree/master/Tests/src}
you can find about 10k lines of 42 code.

To show that we can replace conventional code reuse,
we have implemented a minimal `collections' library
(\url{adamsTowel01/libProject/Collections/})\footnote{
For this and other links the full url looks like \url{github.com/ElvisResearchGroup/L42/tree/master/Tests/src/adamsTowel01/libProject/Collections}
}
 and a large `introspection' library
(\url{adamsTowel01/libProject/Location}), 
allowing to examine library literals.
The collection library uses IC instead of generics,
while the introspection library has many classes reusing common code, and this reuse is obtained using IC instead of
\Q@extends@.
To show that we can replace QQ, we developed
a (quite compact thanks to IC) `units of measure' library
(\url{adamsTowel02/libProject/Units})
, and a sophisticated \Q@Data@ decorator
(\url{adamsTowel02/libProject/Data})
that adds equality, \Q@toS()@, run time invariant checking and other features to library literals.

Finally, to show that IC can scale to work with large units of code,
we have implemented a non-trivial `library loading' library
(\url{adamsTowel02/libProject/Load} and
\url{adamsTowel02/libProject/DeployLibrary}), that 
automatically tweaks libraries to use different implementations for their dependencies;
this allows, for example, to change what kind of numbers and 
strings are internally used by a 42 third party library.
This is obtained by smart usage of the \Q@Redirect@ and \Q@Rename@ operators.


\section{Related work}

Many researchers have explored variations of QQ, as for example
Template Haskell~\cite{sheard2002template} and MetaML \cite{moggi1999idealized}.
Of course those works have profound differences between each other, but they all
focus their attention on AST manipulation.

MorphJ \cite{huang2008expressive} is an interesting outlier, in the sense that is not QQ, but 
it still focuses on AST manipulation. It is another domain-specific language which takes in input well-typed Java classes. Syntactically, it consists in a template-like mechanism added on top of Java. This mechanism is called  
\emph{class morphing} and allows a class to abstract over the structure of other types. For instance, one can define a parametric class \lstinline{Log<X>} which provides, for each method of \lstinline{X}, a method with the same signature which invokes the original method and logs its result on a database.
A theorem prover ensures that, for any well-typed class in input, a well-typed class is produced.

We are not the first work proposing metaprogramming that do not rely on QQ and do not
focus on AST manipulation.
A very relevant concept is meta-traits \cite{reppy2007metaprogramming}:
 traits that can have \emph{place-holders} to be later filled with types, 
values, or method names.
To generate an instance of a meta-trait, one 
gives actual values for the place-holders. 
Formally, a meta-trait is a function from some parameters to traits.
This technique allows to emulate rename operators as well as generic classes/traits.
This proposal supports separate typechecking of trait definitions. That is, there is no need to perform meta-trait expansion
 in order to typecheck classes which use a trait.


%Spec#

Due to space limitations we cannot include an extensive related work section,
but we tried to discuss other approaches during the exposition.
The interested reader may refer to the good survey by Smaragdakis et al.~\cite{smaragdakis2015structured}.

The very popular Scala library LMS~(\url{scala-lms.github.io}), in addition to conventional QQ offers
more abstract techniques for AST rewriting and manipulation/simplification,
to go towards the kind of abstraction offered by IC;
thus recognizing QQ is, at least sometimes, too low level.

We wonder if Ur~\cite{chlipala2010ur} could be extended to represent our \Q@inductive()@ concept:
Ur is a QQ system focused on guaranteeing generation of well-typed records,
and it may be possible to extend it to records with abstract members.


\section{Conclusion}


Quasi Quotation offers the maximum possible expressive power, since it can generate any possible AST.
We conjecture that our approach can generate any behaviour, but not any AST.
For example, we have no direct control on the way local variables, private methods 
and in-lining are handled.
We believe this is a good thing: fine control of the structure of expressions requires
the (meta-)programmer to understand and handle scope, scope-extrusion, variable hiding and similar
representation related issues.

%In our approach, every code literal is self contained, and could be understood as an independent piece of
%code.
%This means that while we can referred to other classes in the program, all variables always have a precise scope,
%and code literals are \textbf{not} closures; that is, the following is invalid:
%\begin{lstlisting}
%C={
%  method Library foo(Num x)=
%    {method Num getX()=x}
%  }$\Comment{ill typed: x not a parameter of method getX()}$
%\end{lstlisting}

Practical experience shows that by using nested classes our approach allows generation of large chunks of code,
by generating many interconnected classes at the same time. On the other side QQ is mostly used to generate single method bodies/functions.

We show that IC can be used to design highly abstract code generation, 
while QQ requires (by design) to keep in mind the concrete shape of the generated code.
We also speculate how conventional OO verification techniques can be
used to verify code generated with IC. To the best of our knowledge, there is no
equivalent verification for QQ.

We believe the (meta)programming style and structure allowed by IC
can support metaprogramming in the large, where the result of metaprogramming can be not just
a single expression or function, but a large section of the program composed by multiple
classes and interfaces. This relies on the fact that a code literal can contain nested classes/interfaces,
that can freely refer to each other and even have subtype relations.
Metaprogramming operators
can then be applied on such code, 
and they would preserve those relations.
One of the biggest technical challenge we encounter while designing the primitive composition operator of 42 was to soundly preserve those relations, especially subtyping ones.
A document illustrating formal descriptions of the operators presented in this work building upon~\cite{servetto2014meta} and formally proving 
their soundness is currently in progress.

To have a glimpse of those difficulties, consider how to directly use QQ to perform method rename:
the natural approach would be to extends QQ allowing
 holes in the place of method names.
Filling those holes may introduce clashes in case
the hole is replaced with the name of an existing method,
either directly declared in the quasi quoted class body or indirectly declared in any implemented interface.
This issue clearly make local reasoning challenging, in order to decide to perform a
rename we need to access the type signature of all the implemented interfaces.

