% This is samplepaper.tex, a sample chapter demonstrating the
% LLNCS macro package for Springer Computer Science proceedings;
% Version 2.20 of 2017/10/04
%
\documentclass[runningheads]{llncs}
%
\usepackage{graphicx}
\usepackage{url}
% Used for displaying a sample figure. If possible, figure files should
% be included in EPS format.
%
% If you use the hyperref package, please uncomment the following line
% to display URLs in blue roman font according to Springer's eBook style:
% \renewcommand\UrlFont{\color{blue}\rmfamily}

\usepackage{listings}
\usepackage{xcolor}
\usepackage{letltxmacro}
\usepackage{mathtools}
\usepackage{mathpartir}
%\usepackage{stix}

\definecolor{darkRed}{RGB}{100,0,10}
\definecolor{darkBlue}{RGB}{10,0,100}
\newcommand*{\ttfamilywithbold}{\fontfamily{pcr}\selectfont}
%\newcommand*{\ttfamilywithbold}{\ttfamily}

%found on http://tex.stackexchange.com/questions/4198/emphasize-word-beginning-with-uppercase-letters-in-code-with-lstlisting-package
%\lstset{language=FortyTwo,identifierstyle=\idstyle}
%
\makeatletter
\newcommand*\idstyle{%
        \expandafter\id@style\the\lst@token\relax
}
\def\id@style#1#2\relax{%
        \ifcat#1\relax\else
                \ifnum`#1=\uccode`#1%
                        \ttfamilywithbold\bfseries
                \fi
        \fi
}
\makeatother

\lstset{language=Java,
  basicstyle=\upshape\ttfamily\footnotesize,%\small,%\scriptsize,
  keywordstyle=\upshape\bfseries\color{darkRed},
  showstringspaces=false,
  mathescape=true,
  xleftmargin=0pt,
  xrightmargin=0pt,
  breaklines=false,
  breakatwhitespace=false,
  breakautoindent=false,
 identifierstyle=\idstyle,
 morekeywords={method,Use,This,constructor,as,into,rename},
 deletekeywords={double},
 literate=
  {\%}{{\mbox{\textbf{\%}}}}1
  {~} {$\sim$}1
%  {<}{$\langle$}1
%  {>}{$\rangle$}1
}

\newcommand*{\SavedLstInline}{}
\LetLtxMacro\SavedLstInline\lstinline
\DeclareRobustCommand*{\lstinline}{%
	\ifmmode
	\let\SavedBGroup\bgroup
	\def\bgroup{%
		\let\bgroup\SavedBGroup
		\hbox\bgroup
	}%
	\fi
	\SavedLstInline
}

\newcommand\saveSpace{\vspace{-2pt}}

\newcommand\Rotated[1]{\begin{turn}{90}\begin{minipage}{12em}#1\end{minipage}\end{turn}}

\newcommand{\Q}{\lstinline}
\newenvironment{bnf}{$\begin{aligned}}{\end{aligned}$}
\newcommand{\production}[3]{\textit{#1}&\Coloneqq\textit{#2}&\text{#3}}
\newcommand{\prodNextLine}[2]{&\quad\quad\textit{#1}&\text{#2}}
\newenvironment{defye}{\\\indent$\begin{aligned}}{\end{aligned}$\\}
\newcommand{\defy}[2]{\!\!\!\!\!\!&&#1&\coloneqq#2\\}
%\newcommand{\defyc}[1]{&\phantom{\coloneqq}\ \ #1\\}
\newcommand{\defyc}[1]{\!\!\!\!\!\!\rlap{\quad \quad #1}&&\\}
\newcommand{\defya}[2]{#1&\!\!\!\!\!\!&\coloneqq#2\\}

%\newcommand{\prodFull}[3]{#1&::=&\mbox{#2}&\mbox{#3}}
\newcommand{\prodInline}[2]{#1\Coloneqq#2}
\newcommand{\terminal}[1]{\ensuremath{$\texttt{#1}$}}
%\newcommand{\metavariable}[1]{\ensuremath{\mathit{#1}}}

\newcommand{\Rulename}[1]{{\textsc{#1}}}
\newcommand{\ctx}[1]{\ensuremath{\mathcal{E}_#1}\!}
\newcommand{\libi}[2]{\Q@\{@\Q!interface!\ #1\Q{;} #2\Q@\}@}
\newcommand{\lib}[3]{\Q!interface!\ensuremath{?}\ \libc{#1}{#2}{#3}}
\newcommand{\libc}[3]{\,\Q@\{@\!#1\Q{;}\ #2 \Q{;}\ #3\Q@\}@\!\!}

\newcommand{\rp}[1]{\Q{(}\!#1\Q{)}}
\newcommand{\eq}[1]{\,\Q{=}#1}
\newcommand{\red}[3]{#1\,\Q{<}#2\eq#3\,\Q{>}}
\newcommand{\summ}[2]{#1\ \Q{<+}\ #2}
\newcommand{\from}[2]{#1\ensuremath{[}#2\ensuremath{]}}
\newcommand{\mmid}{{\ensuremath{{\mid}}}\!}
\newcommand{\hole}{\ensuremath{\square}}
\newcommand{\s}[1]{\ensuremath{\mathit{#1s}}}
\makeatletter
\newcommand{\This}[1]{\Q!This!#1\nextpath}
\newcommand{\Cs}[1]{#1\nextpath}
\newcommand{\nextpath}{\@ifnextchar\bgroup{\gobblenextpath}{}}
\newcommand{\gobblenextpath}[1]{\Q!.!#1\@ifnextchar\bgroup{\gobblenextpath}{}}
\makeatother



%--------------------------
\newcommand{\mynotes}[3]{{\color{#2} {\sc #1}: #3}}
\newcommand\isaac[1]{\mynotes{Isaac}{blue}{#1}}

\newcommand\IO[1]{\color{blue}{#1}}
\newcommand\marco[1]{\mynotes{Marco}{green}{#1}}


%\usepackage{makeidx}  % allows for indexgeneration

\begin{document}
%
\title{Metaprogramming in 42:
Generating imperative behaviour by functional reasoning}
%
\titlerunning{Metaprogramming in 42}  % abbreviated title (for running head)
% If the paper title is too long for the running head, you can set
% an abbreviated paper title here
%
%%-Authors omitted for double blind submission
%%-\author{Hrshikesh Arora\inst{1} %\orcidID{0000-1111-2222-3333} 
%%-\and
%%-Marco Servetto\inst{1} %\orcidID{1111-2222-3333-4444} 
%%-}
%
%%-\authorrunning{H. Arora, M. Servetto}
% First names are abbreviated in the running head.
% If there are more than two authors, 'et al.' is used.
%
%%-\institute{Victoria University of Wellington\\
%%-\email{arorahrsh@ecs.vuw.ac.nz},
%%-\email{marco.servetto@ecs.vuw.ac.nz}
%%-}
%\url{http://www.springer.com/gp/computer-science/lncs} \and
%ABC Institute, Rupert-Karls-University Heidelberg, Heidelberg, Germany\\
%\email{\{abc,lncs\}@uni-heidelberg.de}}
%
\maketitle              % typeset the header of the contribution
%
\begin{abstract}
Quasi Quotation~\cite{moggi1999idealized,pitman1980special,sheard2002template} is a very expressive metaprogramming technique: it allows expressing arbitrary behaviour by
generating arbitrary abstract syntax trees.
However it is hard to reason about Quasi Quotation statically,
the process is fundamentally low-level, imperative and bug prone.
In this paper we present Iterative Composition as
an \textbf{effective alternative} to Quasi Quotation.
Iterative Composition describes a code composition algebra over established code reuse techniques,
similar to trait composition and generics.
Our main contribution is that by applying functional reasoning (such as induction and folds)
over those well-known operators we can generate arbitrary behaviour.
This allows us to reuse the extensive body of verification research in the 
context of object-oriented languages to verify the properties
of the code generated by Iterative Composition.
Finally, we present a prototype implementation of Iterative Composition in the context of the 42 language.


%Today the most common forms of metaprogramming 
%use quasi quotation to generated arbitrarily shaped ASTs.
%This approach ensures absolute power, but is also 
%very difficult to use this power correctly.
%We believe this is an intrinsically ``imperative'' approach,
%and as such is frail and bug-prone.
%
%In the past many code reuse techniques, like 
%subclassing and generics was also considered Metaprogramming.
%We show in this paper how code reuse primitives can generate arbitrary behaviour.
%
%This radically different approach can be safer and simpler, since using
%it requires combining declarative operations using abstraction typical
%of functional reasoning (induction and folds)
%
%Deadline November 13, 2017


%\keywords{First keyword  \and Second keyword \and Another keyword.}
\end{abstract}



\section{Introducing Quasi Quotation}

Lisp~\cite{pitman1980special}, MetaML~\cite{moggi1999idealized}, Template Haskell~\cite{sheard2002template} and many other approaches use Quasi Quotation (QQ).
This can be supported by two kinds of special parenthesis as a syntactic sugar to manipulate Abstract Syntax Trees (ASTs).
Lisp uses (\Q@`@) and (\Q@,@), while here we use
\Q@[|  |]@  and \Q@$\$$(  )@ (as Template Haskell) 
for better readability.

\noindent
The following example explains their meaning: 

\begin{lstlisting}
Int res0=x*x $\Comment{normal code}$
Expr<x:Int$\vdash$Int> res1=[| x*x |]
  $\Comment{new Mul(new Var("x"),new Var("x"))}$
Expr<x:Int$\vdash$Int> res2=[| x* $\$$(12+3) |]
  $\Comment{new Mul(new Var("x"),new Lit(15))}$
\end{lstlisting}

\noindent
Here \Q@res1@ is initialized using a ``quotation'' of code.
This is equivalent to generating the abstract syntax tree by hand, as shown in the comment.
\Q@res2@ is initialized using a ``quasi-quotation'' of code: a chunk of code with a hole, that is filled by executing an expression.

There are different ways to type QQ.
In an expression based language, 
the simplest way is to just have a primitive \Q@Expr@ type,
representing every type of code.
This ensures the result is syntactically well formed, but
it allows for the generation of ill-typed code.
Another option, for example used by MetaML,
is to have a parameterized type.
Here we use \Q@Expr<@$\Gamma\vdash T$\Q@>@; where
$\Gamma$ keeps track of the free variables and $T$ is the expected type
of the result.
This approach is restrictive (see~\cite{servetto2014meta})
 but ensures that all the resulting code is well typed.

Usually programming with QQ requires thinking about the desired method body,
 and often allows generating a more efficient body by generating code specialized for some input value.
A typical example is about generating a \Q@pow@ function, where the exponent is well known.
The ``inefficient'' version would be:

\begin{lstlisting}[language=ML]
fun power(x:Int,n:Int):Int 
  = if (n=0) then 1 else x*power(x,n-1);
power7_a=$\lambda$ x:Int. power x 7;
\end{lstlisting}

\noindent A more ``efficient'' version using QQ would be:

\begin{lstlisting}[language=ML]
fun powerAux(n:Int):Expr<x:Int$\vdash$Int> 
  = if (n=0) then [|1|] else [|x * $\$$(powerAux(n-1)) |];

fun powerGen(n:Int): Int->Int
  = compile([| $\lambda$ x. $\$$(powerAux(n)) |]);

power7_b=powerGen 7;
\end{lstlisting}

\noindent As you can see, by generating the abstract syntax tree, we can obtain exactly:

\begin{lstlisting}[language=ML]
power7_b=$\lambda$ x.x*x*x*x*x*x*x*1;
\end{lstlisting}

\noindent On most machines, \Q@power7_b@ runs faster than \Q@power7_a@.
Metaprogramming applications include more than just speed boosts, but we start with this example because it is very popular and simple.

The code generator above is quite compact, but it is actually \textbf{hiding} (not removing) the complexity of meta-programming.
A common approach to make the code more explicit is to extract
logical concepts as functions.
We can see that the code is proceeding in an inductive fashion:
we know the code for \Q@pow 0@, and given the code for
\Q@pow n@  we can create the code for \Q@pow (n+1)@.
Thus we define \Q@base@ and \Q@inductive@ functions, and we
use them inside \Q@powerAux@:

\begin{lstlisting}[language=ML]
fun base():Expr< x:Int$\vdash$Int > 
  = [| 1 |]
fun inductive(code:Expr< x:Int$\vdash$Int >):Expr< x:Int$\vdash$Int >
  = [| x * $\$$(code) |]

fun powerAux(n:Int):Expr< x:Int$\vdash$Int >
  = if (n=0) then base()
   else inductive ( powerAux(n-1) );
\end{lstlisting}

\noindent Then, we have to bind \Q@x:Int@ to a parameter in a function.
This is an important conceptual action and thus we make it a function:

\begin{lstlisting}[language=ML]
fun lambdaX(code:Expr< x:Int$\vdash$Int >):Expr<$\vdash$Int->Int > 
  = [| $\lambda$ x. $\$$( code ) |]

fun powerGen(n: Int):Int->Int
  = compile(lambdaX(powerAux(n) ))
\end{lstlisting}

The code we obtain is much larger, but is not logically more complex --- it is just showing the logical structure better.
Note how since QQ works near the code representation,
a function \Q@Int->Int@ is radically different from
code with a free variable \Q@x:Int@$\vdash$\Q@Int@, while they are 
logically similar concepts.


We propose Iterative Composition (IC):
while the unit of composition in QQ is the single AST node, 
IC enforces a higher level of abstraction and does not work directly on the AST.
The unit of composition in IC is a \emph{Library}:
a class body, containing methods and possibly nested classes.
Libraries are self contained in the sense that they contain no free variables.
This avoids all scope-extrusion related problems, and (as shown later) enforces local reasoning.

IC has already been presented in other work~\cite{servetto2014meta};
 IC's expressive power is shown by examples,
but is not compared with QQ; moreover such works suggested IC's expressive power is inferior to QQ.
The core idea of IC is to  \emph{rely on  operators of code composition inspired by normal
code reuse}, but lifted to the expression level.
As a concrete example, in Java operators \Q@+@ and \Q@*@ can be used in the expression \Q@1+2*3@,
but the operator \Q@extends@ can only be used in the specific context of class declaration, as in

\begin{lstlisting}[language=Java]
class A extends B{/*some code*/ int m(){return 1+super.m();}}
\end{lstlisting}

In our proposed approach, we lift \Q@extends@ and code literals to operator and constants
that can be used in conventional expressions.
Class declarations associate a class name with the result of an expression of type \Q@Library@. 
 We would write the former example as:

\begin{lstlisting}
A = Override[m()<-superM()]( 
  {/*code of B*/},
  { /*some code*/ int m(){return 1+this.superM();}}
  )
\end{lstlisting}

\noindent We support the conventional super call mechanism by annotating the operator with
the expected super call name: \Q@Override[m()<-superM()](...)@.


\noindent
We can rewrite our \Q@pow@ example 
in IC:

\begin{lstlisting}
Pow = {
  static method Library base()
   ={ method Num pow(Num x)= 1 }$\Comment{Code literal with 1 method "pow(x)"}$

  static method Library inductive()
   ={$\Comment{Code literal with 2 methods: "pow(x)", "superPow(x)"}$
    method Num pow(Num x)= x*this.superPow(x)
    method Num superPow(Num x)$\Comment{no body: it is an abstract method}$
    }
  static method Library inductive(Library code)
   = Override[pow(x)<-superPow(x)](code, this.inductive())
  
  static method Library generate(Num y)
   = if (y==0) then this.base();
     else this.inductive(generate(y-1))
  }
...
Pow7 = Pow.generate(7)
$\Comment{That would reduce into the desired code as follows:}$
Pow7 ={method Num pow(Num x)=x.x*x*x*x*x*x*x*1}
\end{lstlisting}

\noindent In more detail:
\begin{itemize}
\item
\Q@base()@ is a method with no parameter and a \Q@Library@ return type.
This is equivalent to a non-parameterized version of \Q@Expr@ in QQ.
However, our approach still guarantees that all the results are well typed.
\Q@base()@ returns a class with a single method \Q@pow(x)@,
returning 1.
\item
For the inductive case, the method \Q@pow(x)@ of \Q@inductive()@ is defined in terms of
another method (\Q@superPow(x)@), representing the delegation to
the former case in the inductive reasoning.
%Note that the declaration of \Q@superPow(x)@ is an abstract method: a method without body.
\item Method \Q@inductive(code)@ builds
the code for \Q@x+1@ from the code for \Q@x@.
Note how we use \Q@Override@ inside of a normal method body.
This \Q@Override@ will implement
\Q@superPow(x)@ using
the \Q@pow(x)@ body from the induction premise: the \Q@code@ parameter.
Then, \Q@superPow(x)@ is inlined.

\item Method \Q@generate(y)@ uses recursion to \textbf{iteratively compose} the result, using induction starting from
the base case.
Note how this method is logically identical to \Q@powerAux@. However,
since we always work on the actual self contained code neither \Q@lambdaX@ nor \Q@compile@ are needed.
\end{itemize}
Our approach builds on top of code composition operations like multiple inheritance and generics.
The literature offers \cite{barnett2004spec,burdy2005overview,muller2016viper} many successful efforts about proving the
\emph{semantic correctness} 
of code containing inheritance and generics.
On the other hand, static verification of code generated with meta-programming is an open research problem.
We speculate our approach may offer the opportunity to solve this problem,
and by construction generate statically verified code
by reusing techniques originally developed to verify normal object oriented code.
The contributions of our work are as follows:
\begin{itemize}
\item
In Section~\ref{s:verification},
by examples we demonstrate how to apply conventional object oriented verification techniques to IC.

\item
In Section~\ref{s:pattern1},
by examples we show that IC is as expressive as QQ, and that
generating code using a composition algebra
is a flexible and simple technique, if combined with
reasonable programming patterns.
\item In Section~\ref{s:study}, we discuss our experience implementing metaprogramming libraries for a language where IC is the only support for code reuse. 
\end{itemize}

In this paper we do not present a formal language semantic. This is partly due to space reasons
and partly because a similar semantic has been formalized in former work~\cite{servetto2014meta}.
Here we aim to show programming patterns that use this expressive power in surprising and novel ways.


%-------------------------------------------------------------------------------------
%-------------------------------------------------------------------------------------


\section{Compile-time verification: Pow with contracts}
In this section we show how conventional verification techniques for OO can be applied on top of IC.
While IC is already presented in~\cite{servetto2014meta},
the idea and the techniques to use IC for verification are novel contributions.

Looking back at the code to generate \Q@pow@ in QQ and in IC,
we can
note how the method \Q@inductive(n)@ in IC is equivalent to the method
\Q@inductive(n)@ in QQ.
However,
the method \Q@inductive()@ in IC has no equivalent in QQ.
\Q@inductive()@ returns the complete code of a class with an abstract method,
that is then used as a handler to inject behaviour.
There is no way in traditional QQ to express this same concept,
and this is the crucial point that makes our approach easier to verify.
QQ is based on \emph{quasi} quotations, that is, \emph{parametric code} that
will become \emph{complete code} when you fill in the holes.
While in IC, every code literal is \emph{complete code} (with
the usual OO semantics where methods can be overridden).
Unsurprisingly, proving correctness of parametric code is much harder
than for complete code.

We show this in the next example, where we handle \Q@pow@ as before, but while verifying that the
result encodes the right \Q@pow@ function. We will use similar notation to JML, with \Q!@ensure! and \Q!@result!.
We include \Q!@ensureRV! for conditions that we expect to be verified at run time, instead of statically verified.
Many static verification approaches use some automatic theorem proving to figure out the proofs. Here, for completeness, we introduce a 
\Q! @proof! notation, where we insert the full proof, to show what we expect a theorem proving to generate.
To make the verification easier, we generate both the \Q@pow(x)@ method
and a \Q@exp()@ method, keeping track of the accumulated exponent.
This time we will use explicit iteration instead of recursion, just to show that our approach is not bound to any of those styles.
\newcommand\thisExp{\ensuremath{{}^{\textbf{this.exp()}}}}
\newcommand\thisSuperExp{\ensuremath{{}^{\textbf{this.superExp()}}}}
\newcommand\oneThisSuperExp{\ensuremath{{}^{\textbf{1+this.superExp()}}}}
\newcommand\powerY{\ensuremath{{}^{\textbf{y}}}}

\begin{lstlisting}
Pow:{
  static method Library base()
    ={
   $\Comment{@ensures exp()=0}$
    method Num exp()=0 
    
   $\Comment{@ensures pow(x)=x\thisExp}$
    method Num pow(Num x)= 1 
    }
  ...
  }
\end{lstlisting}

We start by examining the base case of our induction: we annotated our 
code with trivial specifications.
A code verifier could statically verify this code 
once and for all when \Q@Pow@ is type-checked.
Then, let us look at the \Q@inductive()@ step:

\begin{lstlisting}
Pow:{
  static method Library base() =...$\Comment{as before}$
  static method Library inductive() 
    ={
   $\Comment{@ensures @result=1+this.superExp()}$
    method Num exp()=1+this.superExp() 
      
    method Num superExp()$\Comment{no body: it is an abstract method}$
      
   $\Comment{@ensures @result=x\thisExp}$
   $\Comment{@proof:}$
   $\Comment{ @result->x*this.superPow(x)//left side}$
   $\Comment{  =x* x\thisSuperExp = x\oneThisSuperExp}$
   $\Comment{ x\thisSuperExp->x\oneThisSuperExp//right side}$
    method Num pow(Num x)= x*this.superPow(x)
      
   $\Comment{@ensures pow(x)=x\thisSuperExp}$
    method Num superPow(Num x)$\Comment{no body: it is an abstract method}$
    }
  ...
  }
\end{lstlisting}

Here the annotations are much heavier.
We show the full proof that a theorem proving may generate.
Note that we need to rely on the contract of \Q@superPow(x)@ (a concept completely hidden in the QQ approach).
Note how also this code, and its possibly non-trivial proof, can be verified once and for all when the code of \Q@Pow@ is compiled.

While the library literals above have contracts,
neither methods \Q@base()@ nor \Q@inductive()@ have any contract:
statically we only know they return a \Q@Library@.

Here we put the pieces together:

\begin{lstlisting}
Pow:{...
 $\Comment{@requiresRV y>=0}$
 $\Comment{@ensuresRV @result.exp().ensures = (@result = y)}$
 $\Comment{and @result.pow(x).ensures = (@result = x\powerY)}$
  static method Library generate(Num y){
    var Library res=this.base()
    for(i in Range(y)){
      res=Override[exp<-superExp, pow<-superPow](res, this.inductive())
     $\Comment{Override will check the composition is correct.}$
     $\Comment{that is: renaming in res pow(x) and exp() as superPow(x) and superExp(),}$
     $\Comment{ res.superPow(x).ensures  satisfyAtLeast inductive.superPow(x).ensures}$
      }
   $\Comment{here we statically know that res is statically verified;}$
    return res }  }$\Comment{RV happens here}$
\end{lstlisting}

Note how when the \Q@Override@ operation  runs, it will
ensure the resulting class is statically verified:
every \Q@Library@ value has statically verified contracts. Operations 
that generate \Q@Library@ values (as \Q@Override@) will also compose those contracts, while verifying such
contract composition is sound.

This may, of course fail.
We think of those operations as compile time operations lifted at execution.
\Q@Override@ producing an error is  equivalent to
getting compilation error about invalid \Q@extends@.
Note how this is not a form of run-time verification, but just the expected semantics of those operators.
\Q@Override@ does not guarantee to produce correct code all the time, in the same way a parser does not guarantee to produce a correct AST for all possible strings: producing errors is part of \Q@Override@ expected behaviour.

In our example \Q@Override@ checks that \Q@res.superPow(x).ensures@
 satisfy at least \Q@inductive().superPow(x).ensures@.
Finally, the \Q@return@ step triggers the run-time verification, and
checks the contracts of \Q@res@.
Indeed before the return, we statically know that \Q@res@ is statically verified,
 but we have no guarantee of what its contracts are saying.
Indeed we are aiming for a run-time verified \Q@Pow.generate(y)@.
We believe this strikes a fundamental balance and is analogous to what a
verifying compiler~\cite{hoare2003verifying} should do.

This separation of concerns in our verification proposal is a key simplification
in our model:
all library values are well typed and statically verified all of the time,
but what those contracts entail can be only verified at run-time, and
the (meta-)programmer can choose when to do it.




To summarize:
\begin{itemize}
\item
At compile time, once and for all, we statically verify that all
method bodies respect the (static) contracts on the method headers.
This can be handled exactly as static verification is normally handled in OO languages.
\item
During code composition we match contracts (\emph{satisfyAtLeast}).
While sound and complete contract matching could be infeasible, 
there are simple restrictions making it trivial.
For example any contract \emph{satisfyAtLeast} the empty contract,
and two syntactically equivalent contracts \emph{satisfyAtLeast} each others.
Expressiveness of matching is probably not important since the (meta-)programmer
is going to write those contracts on purpose to make them matchable.
\item
We guarantee all library literals used by the execution are statically verified; but
statically we do not know the details of their contracts.
\item
Run-time verification is a simple and effective way to close this last gap, verifying
that the result is not only \emph{self-consistent} but also the expected one.
\end{itemize}


%\section{OLD Introduction}
%
%Lisp, MetaML and other languages support metaprogramming by quasi quotation (QQ).
%It is a low-level technique: programmers think in terms of the desired AST;
%QQ can generate arbitrary shaped ASTs, thus allowing generation of \textbf{arbitrary behaviour}.
%
%We consider another technique, that we call incremental composition (IC), that
%also allows to generate arbitrary behaviour.
%IC enforces a high level of abstraction and do not work directly on the AST.
%While the unit of composition in QQ is the single AST node, 
%the unit of composition in IC is a \emph{Library}:
%a class or interface body, containing methods and possibly nested classes.
%Libraries are self contained in the sense that they contains no free variable.
%This cut at the root all scope-extrusion related problems, and (as shown later) enforces local reasoning.
%
%IC has been already presented in other works [...];
%in those works, IC expressive power is show by examples, but is not compared with QQ; moreover the tone  suggested IC power was inferior to QQ.
%
%Here we show that \textbf{IC expressive power is the same of QQ}.
%
%Moreover, 
%code generated by QQ need to be manually proved correct (after is generated).
%IC can be a little more verbose, but the correctness is verified during composition,
%thus result is, quite literally, correct by construction.
%In former work IC correctness was considered only in the limited sense of typing:
%Starting from well typed code, if IC produces a result, it is well typed by construction.
%
%Here we speculated how to reuse conventional Object oriented verification techniques so that
%\textbf{IC construct verified code}.






%-------------------------------------------------------------------------------------
%-------------------------------------------------------------------------------------

\section{IC is as expressive as QQ}
So far we have presented how using pre-post conditions benefits the safety of our approach 
with respect to QQ.
\textbf{Independently} from those benefits, 
we now aim to show that IC is as expressive as QQ, and possibly easier to use on the large scale.

To this aim, we will show how, from a very specific point of view, OO languages are declarative.
We will then show that \Q@Override@ and other composition operators are
declarative operators, and that they can be combined in a general purpose way to synthesize new declarative operators.

Except for the algebra of composition operators, that is already presented in~\cite{servetto2014meta},
all the considerations, ideas and techniques presented in this section are novel contributions.


\subsection*{Object-orientation (OO) points to declarative languages}
Imperative programming asks you to write down the
detailed computational steps your machine should perform.
This allows the programmer to reason about their
programs by ``emulating computer'' in their mind.
As we all know too well, this tempting approach to understand 
 program behaviour does not scale.
We can see OO
as taking imperative programming towards a more
declarative style:
\subsubsection*{Subtyping}
 A dynamically dispatched method invocation
gives us no certainty of the detailed computational steps
that are going to be performed, and
requires programmers to reason in terms of the
abstract contract of the method: that is
the desired properties of the result as function of the arguments.
The programmers are still in control of the detailed behaviour,
but such control is delegated to the programmer that
instantiated such object.
The receiver object takes up the role of the \emph{reasoning engine} that takes
care of transforming the programmer request into a result.
In many OO languages, this reasoning engine amounts to just following a pointer. However,
 as the Visitor Pattern and other  design patterns show us, this is sufficient to encode very interesting behaviour resolution.
Moreover, some languages implement multiple dispatch~\cite{clifton2000multijava}, making this reasoning engine a lot more expressive.
\subsubsection*{Subclassing}
From this point of view, inheritance
represents a similar loss/transfer of control:
The programmer does not know
the full set of methods a class offers:
this depends on the methods offered by the base class,
and if in a future release the heir is enriched with
more methods, such methods will also be injected into
the subclass.

This allows a limited declarative programming style.
For example, in

\begin{lstlisting}[language=Java]
class Rounded extends Button{ ... }
\end{lstlisting}

We declare \Q@Rounded@s to be buttons in a way that
is parametric on what \Q@Button@s concretely are.
The programmer of \Q@Rounded@ just has to specify little bits
of behaviour to personalize the kind of buttons.
The concrete code is then automatically derived
by composing it with the existing code of \Q@Button@.
You can see \Q@Button@s as
following the double role of both a class and a class generator/decorator.
You can think of this as giving some suggestions to \Q@Button@
on what code to generate when creating the class \Q@Rounded@.

It is a very limited declarative language aimed to code composition.
In Java,  the \emph{code composition reasoning engine} is part of the language.
In Java is a very simple minded reasoning, but it gets much more expressive in other languages, like C++.

Since the behaviour of inheritance
is set in stone, \Q@Button@ is not very active in deciding
about \Q@Rounded@.
However, this point of view lets us interpret many approaches
(traits~\cite{scharli2003traits}, mixins~\cite{smaragdakis2000mixin}, generics~\cite{igarashi2001featherweight}, family polymorphism~\cite{ernst2001family})
as ways to enrich what \Q@Button@ can do to generate the
required \Q@Rounded@ from the hints provided by the programmer:
a way to enrich the declarative, domain specific language
that the programmer uses to instruct base classes into generating their heirs.

In this article we wish to temporarily forget that base classes can be used as class or as types, and
focus on their active role in the generation
of code. In this context we will call them Class Decorators.
Compile-time meta-programming~\cite{sheard2002template} is a good way to give Decorators
a mind of their own, so that they can perform arbitrary
complicated steps while generating their heirs.

%Conventional meta-programming techniques, based on AST rewriting by quote/unquote syntax
%are very flexible, but let/force/inactivate the user to generate the behaviour at the
%source code level of detail.

%Instead here we will use meta-programming only as a way to decide what declarative operators to apply
%and how. Meta-programming will not directly generate code, but delegate this responsibility to operators.

%In this paper we show:
%\begin{itemize}
%\item The declarative DSL composed by well established class composition operators
%\item can generate heir code containing arbitrary imperative behaviour,
%\item and the generation process uses inductive functional reasoning.
%\end{itemize}

\subsection*{An algebra of composition operators}

In many class based object oriented languages, a class can inherit the code
from one or more parents.
In order to lift this capability as a metaprogramming operation 
we define an algebra of code composition, where the values
are \Q@Library@ literals and the operations are the composition operators.

\begin{itemize}
\item A \Q@Library@ is a code literal:
a pair of balanced curly brackets with methods and nested classes inside.
Thanks to nested classes, a code literal can contain a large portion of code
with cooperating classes, possibly encapsulating a whole library.
\item A \emph{composition operator}
is a functionality taking in input \Q@Library@s and producing a \Q@Library@
result, or a \emph{composition error}.
\end{itemize}


In the following example, class \Q@C@ contains a single static 
method \Q@foo()@ returning a \Q@Library@ literal.
We can use class \Q@C@ and the \Q@Override@ composition operator to
attempt creating classes \Q@D1,D2,D3@.

\begin{lstlisting}
C:{
  static method Library foo()={method Num m()=2  }  
  }
D1:C.foo()
D2:Override[](C.foo(),{method Num n()=0 })
D3:Override[m()<-superM1(),superM2()](C.foo(),C.foo(),{
  method Num m()=3+this.superM1()+this.superM2()
  $\Comment{superM1 and superM2 used for super calls}$
  method Num superM1()
  method Num superM2()
  })
\end{lstlisting}

By a process known as flattening~\cite{scharli2003traits}, we get the following results for \Q@D1,D2,D3@:

\begin{lstlisting}
D1:{method Num m()=2  }  
\end{lstlisting}

\Q@C.foo()@ is \emph{executed at compile time}
and we use the result to initialize \Q@D1@.

\begin{lstlisting}
D2:{method Num m()=2  method Num n()=0  }  
\end{lstlisting}

\Q@Override@ works like inheritance; in this case since there is no conflict between
the two code literals, the result is a library literal containing both
%\begin{lstlisting}
%D3://composition error:
%//two implementations for method m()  
%\end{lstlisting}
%Here we try to compose the same code literal twice.
%This fails since both (identical) sides implement method m.

\begin{lstlisting}
D3:{method Num m()=3+2+2 } $\Comment{super calls may be inlined}$
\end{lstlisting}

We write \Q@Override[m()<-superM1(),superM2()](..)@ to support calling the multiple conflicting versions
for method \Q@m()@.
Here we try to compose the same code literal twice.
This may fail, since both (identical) sides implement method m.
However, \Q@Override@ treats the last library value in a special
 and preferential way:
if there are multiple conflicting implementations of the same method 
(\Q@m()@ in this case)
the last value can redefine such method and avert the conflict.

The \Q@Override@ operator we are proposing works as 
trait composition~\cite{scharli2003traits}:
we can compose multiple \Q@Library@ values in
an associative and commutative way;
except for the last parameter, which
enjoys preferential composition:
its methods can override methods in the other
libraries. This privilege is equivalent to the 
one \emph{glue code} enjoys in the original trait model 
~\cite{scharli2003traits}(section 3.3).


Many variations of this composition operator has been presented in the literature,
and an exhaustive understanding of its behaviour is not needed to
understand this paper.

In addition to \Q@Override@, we will use many other composition operators.
We believe it is fair to consider these high level operations on code as \emph{declarative}:
 the programmer does not specify the details of the source code
involved, but only reasons at the method call level.
Concretely, in the former example this means that the implementation of method \Q@m()@
inside class \Q@C@ is not relevant: only the behaviour/contract of such method is important.



\subsection*{Patterns for generating arbitrary behaviour}
The general idea is to define a new declarative operator
using other declarative operators.
%(like Override for code composition and Refactor.redirect for generics).
Operators can generate imperative code, but such code is not directly
specified, it is synthesized by inductive reasoning.
%Instead of a detailed explanation of the behaviour of the operators we are going to use, we think it is more engaging
%to dive into the following code example:
As an example of a new operator, we will
consider \Q@Stringable@;
it can generate an opportune \Q@toS()@/\Q@toString()@ method inside of code literals.
In our example, the resulting code is going to be the definition
of class \Q@A@.
Note the declarative feel of this operation: we just specify the desired properties of the result (having a \Q@toS()@ method), and the detailed implementation is  
 automatically derived by the shape of the library literal provided.

\begin{lstlisting}
A: Stringable <>< { .. }
\end{lstlisting}

\subsection*{Babel fish operator \Q@<><@}
Since the main goal of our approach is to generate code, we introduce a \emph{code generation operator} \Q@<><@, called Babel fish.
It is a binary operator, taking a Decorator and a \Q@Library@, and producing another \Q@Library@.

We consider a language with normal operator overriding thus \Q@Stringable <>< { .. }@ is equivalent to \Q@Stringable.babelFish({ .. })@ 

%Here \Q@Stringable@
%can generate a \Q@toS()@/\Q@toString()@ method inside of the code literal \Q@{ .. }@,
%the resulting code is going to be the definition
%of class \Q@A@.

We choose \Q@Stringable@ since its behaviour is not obvious: it is required to examine \Q@{ .. }@ in order to 
propagate \Q@toS()@ to all the fields.
This is also an important example: other operations like \Q@equals(that)@, \Q@hashCode()@ and \Q@compare(that)@
also rely on propagating the operation over the fields.
That is, all of these operations can be implemented by
following the same programming pattern.

We can also see \Q@Override@ as a class decorator, and
write \Q@Override(a,b,c)<><d@ to highlight that the last parameter enjoys preferential composition.

\subsection*{Objective code}
In order to have a clear goal,
let's imagine a tentative code we would like to obtain:

\begin{lstlisting}
Stringable<><{S name Num age method S sayHi() ="Hi, i'm "++this.name()}
\end{lstlisting}

\noindent should evaluate into

\begin{lstlisting}
{S name Num age
 method S toS()=
   "["++this.name().toS()++", "++this.age().toS()++"]"
 method S sayHi()=
   "Hi, i'm "++this.name()
}
\end{lstlisting}

How to obtain this kind of result by using IC instead of QQ?
The main idea is to try to not think about the actual code but about
its behaviour, and how to decompose it into functions, and how to
put these functions together.

%We will now show how the code of \Q@Stringable@ may look like.


\subsection*{Top level operation}
We are going to show the implementation of \Q@Stringable@ in a Java like language where
we are going to repeat parameter names in the call site when this can improve readability.
The following is our entry point: we define a babelFish static method
returning a \Q@Library@.

\begin{lstlisting}
Stringable:{
  static method Library <><(Library that) = (
    libs=this.baseCases(that) $\Comment{1}$
    acc=this.fold(libs,acc:{method S toS()= ""})$\Comment{2}$
    res=this.close(acc)$\Comment{3}$
    Override[](res)<><that$\Comment{4}$
    )
...
\end{lstlisting}


% The method is called \Q@<><@ and thus can be invoked as binary operator.
% We use the name \Q@Library@  for our code literals since they can contain nested classes, thus they potentially encapsulate a whole library. 

The method uses inductive reasoning and is divided into 4 meaningful steps:
\begin{itemize}
\item[1] Using the input, we generate the base cases;
in this case computing \Q@toS()@ for a class with a \textbf{single specific field}.
\item[2] Then we  \textbf{fold} our base cases into a single solution; in this case we specify how to merge two \Q@toS()@ implementations.
\item[3] Finally we \textbf{close/wrap} our implementation (adding  \Q@[]@).
\item[4]
Our result \Q@res@ would now expose \textbf{only} the \Q@toS()@ method.
%(plus needed abstract methods and some private utility methods).
We then use \Q@Override@ to compose our \Q@toS()@ with the original input.
\end{itemize}

%We will see more of the details of \Q@Override@ while examing the rest of the code.


\subsection*{Phase 1: base cases}
To define our base case, we first 
 declare a method returning a constant \Q@Library@ value: an approximation for a class
 whose  \Q@toS()@ method delegates to a field.

\begin{lstlisting}
  static method Library baseTrait() = {
    T f
    T:{method S toS()}$\Comment{nested class}$
    method S toS()= this.f().toS()
    }
\end{lstlisting}

It has a field called \Q@f@ of type \Q@T@.
\Q@T@ is a type with an abstract \Q@toS()@ method returning a string \Q@S@.
The \Q@toS()@ method of the top level class, just propagates out
the behaviour of \Q@T.toS()@.

Note how we call such method a ``trait'': inspired by trait composition, 
this method represents a reusable piece of code, where all the dependencies are explicit.
That is, this code is \emph{general} (\Q@T@ offers only the required \Q@toS()@ signature)
but is not \emph{generic} in the Java/C\# sense.

Now we can define our \Q@baseCases(that)@ method.

\begin{lstlisting}
  static method Libs baseCases(Library that) = Libs[
      RedirectType(path:"T" into:fi.type())<><
      Rename(selector:"f" into:fi.selector())<><
      this.baseTrait()
      | with fi in Fields(that)]
\end{lstlisting}

We extract the fields by observing with introspection
our input class.
As available in both Haskell and Python, we assume we can use the common syntax for list comprehension:
\Q@CollectionType[e| x in list]@
Where the expression \Q@e@ denotes the entries in the newly generated collection.
In this case we generate our entries with the following interesting code:

\begin{lstlisting}
      RedirectType(path:"T" into:fi.type())<><
      Rename(selector:"f" into:fi.selector())<><
      this.baseTrait()
\end{lstlisting}

Here we apply two decorators to our \Q@baseTrait@. First
we rename our field name from \Q@f@ to the current field name, then
 we redirect \Q@T@ into the current field type.
 This redirect is not an obvious operation, and is the kind of code manipulation where IC shines
with respect to QQ, since it allows one to express logic problems that would end up hidden in QQ.
 The operator \Q@RedirectType@ is more general than \Q@Redirect@ as shown in~\cite{servetto2014meta}.
 \Q@Redirect@ is a powerful operator:
 it deletes the nested class \Q@T@ and replaces all
 the occurrences of \Q@T@ with occurrences of such external type.
 \Q@Redirect@ is indeed a powerful form of generics,
 where generic types can be expressed as nested classes with abstract members.
 However, logically \Q@Redirect(path:"T" into:fi.type())@
 would work only in case of \Q@fi.type()@ being a type
 defined \textbf{outside} of the decorated class.
 In case of \Q@fi.type()@ denoting a \textbf{nested class} inside of the decorated class,
 we would need to perform \Q@Rename(path:"T" into:fi.type())@ instead.
 \Q@RedirectType@ just switches between the two options and calls the right operation internally.

\Q@Rename@ and \Q@Redirect@ leverage on the
conventional nominal type system to avoid errors:
 \Q@fi.type()@ returns a general \Q@Type@ type, and offers operations to extract either
 the internal path (typed \Q@Path@, that would be a well typed argument for \Q@Rename@)
 or the external class (typed \Q@Class@, that would be a well typed argument for \Q@Redirect@).

 Note how the convenient and expressive operator \Q@RedirectType@ is a \textbf{derived operator},
 exactly as \Q@Stringable@,
 and is expressed in the language itself. Once and for all the programmer can understand
 the complexity of a specific problem and encapsulate it in a solution.

 On the other hand, while coding this kind of program manipulation with QQ, most programmers
 will just forget to handle the case where the type of a field is a nested class of the input.
 Depending on the target language, this could be a big problem later on.


To give an example of the result of \Q@baseCases(that)@, if we were to call 

\begin{lstlisting}
baseCases({
  S name Num age
  method S sayHi() = "Hi, i'm "++this.name()
  })
\end{lstlisting}

 we would obtain

\begin{lstlisting}
Libs[
  {S name   method S toS() = this.name().toS()};
  {Num age  method S toS() = this.age().toS()};
  ]
\end{lstlisting}

Note how there is no trace of the \Q@sayHi()@ method, and the
nested classes named \Q@T@ have been removed to point at the
external types \Q@S@ and \Q@Num@


\subsection*{Phase 2: folding}
Next we fold all our base cases into a single \Q@toS()@ method.
Traditionally, to fold a list of values into a single value, a binary operation is needed.
However, here we fold code representing methods,
so we need a lifted version of fold:
 we supply a \Q@Library@ value where a method
uses two alternative versions of itself. 

\begin{lstlisting}
    {
    method S toS()= this.superToS1()++", "++this.superToS2()
    method S superToS1()
    method S superToS2()
    }
\end{lstlisting}

This corresponds both to multiple inheritance, where a new version of a method is obtained by composing the two super implementations,
but also to a binary fold operation from \Q@S superToS1()@ and \Q@S superToS2()@
into a \Q@S toS()@ result.
Thus, to apply this folding, we use \Q@Override@ and we leverage on the multiple inheritance interpretation:

\begin{lstlisting}
  static method Library fold(Libs that,Library acc) = {
    if that.isEmpty() (return acc)
    newAcc=Override[toS()<-superToS1(),superToS2()](that.left(), acc)<><{
      method S toS()= this.superToS1()++", "++this.superToS2()
      method S superToS1()
      method S superToS2()
      }
    return this.fold(that.withoutLeft(),acc:newAcc)
    }
\end{lstlisting}

The base case of this recursive code is the empty list, where \Q@acc@ is returned,
otherwise \Q@Override@ provides us with multiple inheritance where
\Q@superToS1()@ and \Q@superToS2()@
allow us to call super from the first/second parameter.
Of course we can not just compose \Q@acc@ with the top of our list: they both offer
a \Q@toS()@ method. We need to provide extra code to override the conflicting implementation
and provide new behaviour, in this case, the two strings separated by a comma.

Continuing with the example from before, starting from the same code literal we would now obtain:

\begin{lstlisting}
{S name Num age
 method S toS() = this.name().toS()++", "++this.age().toS()
}
\end{lstlisting}

\subsection*{Phase 3: wrapping}
We could be satisfied with such result, but often we wish to wrap our final result
to present it better.

\begin{lstlisting}
  static method Library close(Library that) =
    Override[toS()<-superToS()](that)<><{
      method S toS() = "["++this.superToS()++"]"
      method S superToS()
      }
  }
\end{lstlisting}

  In the \Q@close(that)@ method here, \Q@Override@ adds \Q@[]@ around the string.

Starting from

\begin{lstlisting}
{S name Num age method S sayHi() "Hi, I'm "++this.name()}
\end{lstlisting}

we would now obtain

\begin{lstlisting}
{S name Num age
 method S toS() = "["++this.name().toS()++", "+this.age().toS()++"]"
}
\end{lstlisting}

\subsection*{Phase 4: composition}
The last operation in the top level method, is composing our generated \Q@toS()@ behaviour with the
original code (containing also \Q@sayHi@) in order to obtain the final result.

We wish to stress how this example code could easily be adapted for \Q@equals(that)@ and
a plethora of other field dependent operations.
With slightly more adaptation it could generate any pattern based on method shape/names in
any code source.
This is clearly supporting all the expression power of MorphJ~\cite{huang2008expressive}.

It could be possible to ``abstract'' over this code, using the template method pattern so that
to write generators for, let's say, \Q@equals(that)@ and \Q@toS()@ the programmer may reuse the logic of \Q@fold(that,acc)@ and the top level method.
Here we preferred to show the concrete code of the Decorator \Q@Stringable@ for the sake of a more
direct example.

To conclude, in this section we have a shown how a combination of
trait multiple inheritance, redirect/generics and rename
can be used as a starting point to synthesize arbitrary behaviour for a
code literal using an inductive mindset.
Trait multiple inheritance, redirectand rename
are high level declarative class composition and adaptation operators,
and the obtained decorator also has this declarative `feel'.




%\section{Discussion}
%
%The approach presented here is very general, and could be adapted to fit many different languages.
%Those adaptations may have great impact on the general safety on the approach.
%
%The programming language 42 is based on this approach, and couple it with
%a strong nominal type system, where all the literal recursively contained in any executable code, are required to be well typed.
%However, operators still run on .... meta safety ...
%
%Note how composition operators can always fail with an exception, that ....
%
%Software verification conceptually try to match software with specifications.
%Software verification rarely consider metaprogramming,
%and is traditionally divided in two broad category:
%\begin{itemize}
%\item
%Static verification (the code is guaranteed to always give the right answer)
%and 
%\item Run-time verification (the code is guaranteed to never give the wrong answer, however it may non provide an answer, either by non-termination or by termination with an error).
%\end{itemize}
%When metaprogramming is took under consideration, those two category are not sufficient to express all the reasonable verification options;
%In particular, there is the issue of how the specification for the generated code (meta-specification) is obtained/generated.
%Here in the following some more possibilities:
%\begin{itemize}
%\item Full Static verification:
%The code is statically verified, thus always produce 
%correct code, that is also statically verified and 
%follow the Full Static verification criteria when generating
%new code.
%The specification contains a way to generated the meta-specification.
%This is the most hard core strategy and possibly the hardest to realize.
%
%\item Verified Model driven development:
%Meta-Specification are input of the (unverified) code generation process,
%the generated code is then statically verified to respect those specifications.  
%If the output is incorrect, it may be very hard to understand ``why''.
%
%\item Post-mortem static verification:
%A non verified system produce code and its meta-specification.
%Those specifications are then used to statically verify the result.
%Meta-specifications may be incorrect, but is easier to test that correctness of specifications instead of correctness of actual code.
%This is, for example, the approach of Gallina tactics for the Coq Theorem proving.
%
%\item Compile-time verification:
%The process of generating code is run-time verified,
%but this process happens in a phase before the generated code is executed.
%Thus this phase may fail, and the error may be regarded as a compilation error.
%However, if this phase succeed, the generated code is statically verified.
%(against correct meta-specifications, since  produced by a run-time verified process)
%In this approach, usually the generated code is unable to perform further steps of code generations.
%This approach is very powerful: the first phase may be considered as a phase of ``input validation'', and after that, the correct result is ensured.
%
%\item Deep-Run-time verification:
%The code is run-time verified, the generated code is run-time verified and so on.
%This is probably the simpler strategy to implement on top of an existing Run-time verification system
%
%\item Gradual verification:
%Like Deep-Run-time verification, but some algorithms may be simple enough that static verification is performed.
%In particular, the first layer of code generation may be statically verified.
%While gradual verification do not need to use the meta-generation
%boundaries to switch from Static to Run time verification, it may be a very common boundary.
% 
%\end{itemize}

%Conclusions? future work?
%@StrongExceptionSafety is 
%a very strong property,
%and some languages may be unwilling to commit to always preserve it.
%In particular, depending on the details of a specific language
% releasing resources as in \Q@finally@ blocks may require
%some relaxation of @StrongExceptionSafety. Sound releasing of resources could be interesting
%future work.

\section{Conclusions and Future Work}
\label{s:conclusion}
Our approach follows the principles of \emph{offensive programming}
~\cite{stephens2015beginning}\IODel{,} wher\IO{e:}
no attempt to fix or recover an invalid object is performe\IO{d, a}nd
%	\begin{itemize}
%\item
 failures (unchecked exceptions)
		are raised close to their cause: at the end of constructors creating invalid objects and immediately after field updates and instance methods that invalidate their receivers.

%}{3}{[meaning] is not clear} (the operation creating an invalid object), i.e. we ``fail-fast''.    
%		\item
%	\end{itemize}


%The aim of our work is only to enforce object invariants, so we do not present complexities unnecessary for this purpose.
Our work builds on a specific form of TMs and OCs, whose
popularity is growing, and we expect future languages to support some variation of these.
Crucially, any language already designed with such TMs and OCs
can also support our invariant protocol with minimal added complexity.


We demonstrated the applicability and simplicity of our approach with a GUI example.
Our invariant protocol performs several orders of magnitude less checks than visible state semantics, and requires much less annotation 
than Spec\#, (the system with the most comparable goals). In Section~\ref{s:formalism} we formalised our invariant protocol and in Appendix~\ref{s:proof} we prove it sound.
%In appendix~\ref{s:formalism} we formalise our invariant protocol and prove it sound. 
To stay parametric over the various existing type systems which provably enforce the properties we require for our proof (and much more), we do not formalise any specific type system.


% a method could be declared as taking a class whose invariant corresponds to the method's pre-condition,  and returning a class whose invariant corresponds to the pos-condition.


% Such approach may be quite verbose, but would ensure that the precondition on the argument holds for the whole execution of the method, instead of just holding at the beginning.

%It could be worthwhile formalising the minimal type system required by validation.



%However the restrictions we make to ensures that \Q@validate@ is deterministic, namely those the type-system enforces due to its signature, seem quite flexible and reasonable;

%%%%%examples of things that future work may investigate allowing are deterministic I/O and multi-threading. 

The language we presented here restricts the forms of \Q@invariant@ and capsule mutator methods;
such strong restrictions allow for sound and efficient injection of invariant checks. 
\IOBlock{Merge this and the next paragraph \& compact them!}{These restrictions do not get in the way of writing invariants over immutable data, but the box pattern is required for verifying complex mutable data structures. We believe this pattern, although verbose, is simple and understandable. While it may be possible for a more complex and fragile type system to reduce the need for the pattern whilst still ensuring our desired semantics, we prioritize simplicity and generality. }

\IOBlock{Merge into the previous paragraph \& compact}{In order to obtain safety, simplicity, and efficiency we traded some expressive power:
the \Q@invariant@ method can only refer to immutable and encapsulated state.
This means that while we can easily verify that a doubly linked list of immutable elements
is correctly linked up,
we can not do the same for a doubly linked lists of mutable elements. Our approach does not prevent correctly implementing such data structures, but the \Q@invariant@ method would be unable to access the list's nodes, since they would contain \Q@mut@ references to shared objects.
In order to verify such data structures we could add a special kind of field which cannot be (transitively) accessed by invariants; such fields could freely refer to any object. We are however unsure if such complexity would be justified.} \IOComm{Mention flexible ownership types as a potential solution? (Assuming it is)}

% To verify those data-structures, in future work
% we may investigate a special kind of field that
% could be accessed only using a \Q@mut@ receiver.
% Such fields would be allowed to refer to not encapsulated state, 
% and they would be unreachable from the invariant code,that starts from a \Q@read this@.

% \LINE
% The language we presented here restricts the form of \Q@invariant@ and capsule mutator methods. 
% We have shown that such restrictions, albeit strong, allow sound and efficient injection of invariant checks. 
% While our restrictions do not hamper writing invariants over immutable data, invariants over complex mutable % data require the box pattern.  We believe the box pattern, although verbose, is simple and understandable, but % could be improved with syntax sugar.

% Our goals of simplicity and efficiency come at the cost of expressivity: we are unable to express invariants % over non-encapsulated mutable structures, 
% though a more complex and fragile type system may reduce such limitations. However we believe we have % demonstrated that our limitations are not too severe and that we have achieved our goals.
% \LINE

%, however such a language is unlikely to be easily understood by programmers;
%being able to predict whether code would be well typed allows programmers
%to better take advantage of the language.

For an implementation of our work to be sound, catching exceptions like stack overflows or out of memory
cannot be allowed in \Q@invariant@ methods, since they are not deterministically thrown.
%For an implementation of our work to be sound, non-deterministic exceptions like stack overflows or out of memory
%errors cannot be caught in invariants.
%this
%use exception catching as a non deterministic conditional choice, 
%allowing non deterministic behaviour.
Currently L42 never allows catching them, however we could also write a (native) capability method (which can't be used inside an invariant) that enables catching them. Another option worth exploring would be to make such exceptions deterministic, perhaps by giving invariants fixed stack and heap sizes.

Other directions that could be investigated to improve our work include the addition of syntax sugar to ease the burden of the box and the transform patterns; type modifier inference\IODel{, and support for flexible ownership types}.

%Our work, in comparison to previous RV techniques,
%aims to be efficient by limiting the number of validation calls, however we have \REVComm{no empirical evaluation of our approach's performance}{3}{\label{CONTRA1}contradictory to [see footnote \ref{CONTRA2}]}.
%To improve efficiency it could be worth investigating elision of unnecessary validation calls
%or even only validating parts of objects (by running the part of \Q@validate@ that could fail).

\begin{comment}
In literature, both static and runtime verification discuss
the correctness of common programming patterns in conventional languages.
Their struggle is proof of how hard it is to deal with the expressive power of unrestricted imperative object-oriented programming.
 Here instead we build on languages using TMs and OCs to tame the use of imperative features. In this way
we have a fresh start where static variables are disallowed, unchecked exceptions require care to be captured, and I/O is allowed only when an opportune capability object is reachable.
Following those restrictions allow simpler reasoning.
The philosophy of our approach is to be like an extended type system: 
It is the programmer's decision
to annotate a field with a certain type,
or the class with a certain \validate.
If the program is well-typed, they are not questioned in their intent.
During execution the system is solely responsible for soundly enforcing the invariant protocol.
This is in sharp contrast with most work in RV, that is often conceived more as a tool to ease debugging:
both deciding properties and enforcing them is controlled by the programmers.
This is also different from static verification,
%where the properties are ensured instead of enforced.
where the properties are ensured ahead of time instead of being enforced during execution.
%Static verification is very heavy weight, and often impractical/restrictive.

%Both static and runtime verification
%aim to monitor a wide range of properties; to this aim they accept a 
%great deal of complexity, and require the programmer to develop a deep understanding
%over the behaviour and the structure of code.
%For example, the specification of method’s pre and post-conditions
%encode a generalization of the program behaviour in the dedicated specification language.
%This means that, even in the best case scenario, 
%using pre/post-conditions the user is required to specify the program semantics twice:
%first in the specification language and then in the underlying programming language.
%In comparison, our approach aims to only verify conditions on immutable or well encapsulated state.
%This makes our approach \emph{ultra-lightweight}:
%the programmer specifies only the desired \Q@invariant@ method.

Moreover, our approach does not aim to replace static or run-time verification,
but is a building block they can rely upon.
\end{comment}

%\noindent\textit{Our approach:}



%
% ---- Bibliography ----
%
% BibTeX users should specify bibliography style 'splncs04'.
% References will then be sorted and formatted in the correct style.
%
% \bibliographystyle{splncs04}
% \bibliography{mybibliography}
%

\bibliographystyle{plain}
%\bibliographystyle{splncs04}
\bibliography{main}

\end{document}
