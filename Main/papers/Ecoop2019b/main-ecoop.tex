
\documentclass[a4paper,UKenglish,cleveref, autoref]{lipics-v2019}
%This is a template for producing LIPIcs articles. 
%See lipics-manual.pdf for further information.
%for A4 paper format use option "a4paper", for US-letter use option "letterpaper"
%for british hyphenation rules use option "UKenglish", for american hyphenation rules use option "USenglish"
%for section-numbered lemmas etc., use "numberwithinsect"
%for enabling cleveref support, use "cleveref"
%for enabling cleveref support, use "autoref"

%\graphicspath{{./graphics/}}%helpful if your graphic files are in another directory

\bibliographystyle{plainurl}% the mandatory bibstyle

\title{Sound Invariant Checking Using Type Modifiers and Object Capabilities.}

%\titlerunning{Dummy short title}%optional, please use if title is longer than one line

%\author{John Q. Public}{Dummy University Computing Laboratory, [optional: Address], Country \and My second affiliation, Country \and \url{http://www.myhomepage.edu} }{johnqpublic@dummyuni.org}{https://orcid.org/0000-0002-1825-0097}{(Optional) author-specific funding acknowledgements}%mandatory, please use full name; only 1 author per \author macro; first two parameters are mandatory, other parameters can be empty. Please provide at least the name of the affiliation and the country. The full address is optional
%\author{Joan R. Public\footnote{Optional footnote, e.g. to mark corresponding author}}{Department of Informatics, Dummy College, [optional: Address], Country}{joanrpublic@dummycollege.org}{[orcid]}{[funding]}
\author{Isaac Oscar Gariano}{Victoria University of Wellington, New Zealand}{isaac@ecs.vuw.ac.nz}{}{}
\author{Marco Servetto}{Victoria University of Wellington, New Zealand}{marco.servetto@ecs.vuw.ac.nz}{}{}
\author{Alex Potanin}{Victoria University of Wellington, New Zealand}{alex@ecs.vuw.ac.nz}{}{}


\authorrunning{Isaac O.\,G., M. Servetto, and A. Potanin}%mandatory. First: Use abbreviated first/middle names. Second (only in severe cases): Use first author plus '\etall 

\Copyright{Isaac Oscar Garion, Marco Servetto, and Alex Potanin}%mandatory, please use full first names. LIPIcs license is "CC-BY";  http://creativecommons.org/licenses/by/3.0/

\ccsdesc[500]{Theory of computation~Invariants}
\ccsdesc[500]{Theory of computation~Program verification}
\ccsdesc[500]{Software and its engineering~Object oriented languages}%mandatory: Please choose ACM 2012 classifications from https://dl.acm.org/ccs/ccs_flat.cfm 

\keywords{type modifiers, object capabilities, runtime verification, class invariants}%mandatory; please add comma-separated list of keywords

%\category{}%optional, e.g. invited paper

%\relatedversion{}%optional, e.g. full version hosted on arXiv, HAL, or other respository/website
%\relatedversion{A full version of the paper is available at \url{...}.}

\supplement{http://l42.is/EcoopArtifact.zip}%optional, e.g. related research data, source code, ... hosted on a repository like zenodo, figshare, GitHub, ...

%\funding{(Optional) general funding statement \dots}%optional, to capture a funding statement, which applies to all authors. Please enter author specific funding statements as fifth argument of the \author macro.

%\acknowledgements{I want to thank \dots}%optional

%\nolinenumbers %uncomment to disable line numbering

%\hideLIPIcs  %uncomment to remove references to LIPIcs series (logo, DOI, ...), e.g. when preparing a pre-final version to be uploaded to arXiv or another public repository

%Editor-only macros:: begin (do not touch as author)%%%%%%%%%%%%%%%%%%%%%%%%%%%%%%%%%%
\EventEditors{John Q. Open and Joan R. Access}
\EventNoEds{2}
\EventLongTitle{42nd Conference on Very Important Topics (CVIT 2016)}
\EventShortTitle{CVIT 2016}
\EventAcronym{CVIT}
\EventYear{2016}
\EventDate{December 24--27, 2016}
\EventLocation{Little Whinging, United Kingdom}
\EventLogo{}
\SeriesVolume{42}
\ArticleNo{23}
%%%%%%%%%%%%%%%%%%%%%%%%%%%%%%%%%%%%%%%%%%%%%%%%%%%%%%

\usepackage{verbatim}
%\addbibresource{main.bib}
\usepackage{wrapfig}
\usepackage{etoolbox}

\newtoggle{ECOOP}

%
\makeatletter
\DeclareOldFontCommand{\rm}{\normalfont\rmfamily}{\mathrm}
\DeclareOldFontCommand{\sf}{\normalfont\sffamily}{\mathsf}
\DeclareOldFontCommand{\tt}{\normalfont\ttfamily}{\mathtt}
\DeclareOldFontCommand{\bf}{\normalfont\bfseries}{\mathbf}
\DeclareOldFontCommand{\it}{\normalfont\itshape}{\mathit}
\DeclareOldFontCommand{\sl}{\normalfont\slshape}{\@nomath\sl}
\DeclareOldFontCommand{\sc}{\normalfont\scshape}{\@nomath\sc}
\makeatother

\usepackage{mathpartir}
\usepackage{amsmath}
\usepackage{amsthm}

\theoremstyle{plain}

\makeatletter
\newcommand{\providecounter}[1]{%
  \ifcsname c@#1\endcsname % do nothing, counter allready defined
  \else
    \newcounter{#1}%
  \fi
}
\makeatother


\providecounter{definition}
\newtheorem{Definition}[definition]{Definition}
\newcounter{assumption}
\newtheorem{Assumption}[assumption]{Assumption}
\providecounter{lemma}
\newtheorem{Lemma}[lemma]{Lemma}

\usepackage{xspace}
\usepackage{listings}
\usepackage{xcolor}
\usepackage{letltxmacro}
\usepackage{mathtools}
\usepackage{mathpartir}
%\usepackage{stix}

\definecolor{darkRed}{RGB}{100,0,10}
\definecolor{darkBlue}{RGB}{10,0,100}
\newcommand*{\ttfamilywithbold}{\fontfamily{pcr}\selectfont}
%\newcommand*{\ttfamilywithbold}{\ttfamily}

%found on http://tex.stackexchange.com/questions/4198/emphasize-word-beginning-with-uppercase-letters-in-code-with-lstlisting-package
%\lstset{language=FortyTwo,identifierstyle=\idstyle}
%
\makeatletter
\newcommand*\idstyle{%
        \expandafter\id@style\the\lst@token\relax
}
\def\id@style#1#2\relax{%
        \ifcat#1\relax\else
                \ifnum`#1=\uccode`#1%
                        \ttfamilywithbold\bfseries
                \fi
        \fi
}
\makeatother

\lstset{language=Java,
  basicstyle=\upshape\ttfamily\footnotesize,%\small,%\scriptsize,
  keywordstyle=\upshape\bfseries\color{darkRed},
  showstringspaces=false,
  mathescape=true,
  xleftmargin=0pt,
  xrightmargin=0pt,
  breaklines=false,
  breakatwhitespace=false,
  breakautoindent=false,
 identifierstyle=\idstyle,
 morekeywords={method,Use,This,constructor,as,into,rename},
 deletekeywords={double}
}

\newcommand*{\SavedLstInline}{}
\LetLtxMacro\SavedLstInline\lstinline
\DeclareRobustCommand*{\lstinline}{%
	\ifmmode
	\let\SavedBGroup\bgroup
	\def\bgroup{%
		\let\bgroup\SavedBGroup
		\hbox\bgroup
	}%
	\fi
	\SavedLstInline
}

\newcommand\saveSpace{\vspace{-2pt}}

\newcommand\Rotated[1]{\begin{turn}{90}\begin{minipage}{12em}#1\end{minipage}\end{turn}}

\newcommand{\Q}{\lstinline}
\newenvironment{bnf}{$\begin{array}{lcll}}{\end{array}$}
\newcommand{\production}[3]{%
	\text{\itshape #1}&%
	\!\!\!\!\!\Coloneqq\!\!\!\!\!&%
	\text{\itshape #2}&%
	\!\!\!\!\!\mbox{#3}}
%\newcommand{\prodFull}[3]{#1&::=&\mbox{#2}&\mbox{#3}}
\newcommand{\prodInline}[2]{#1\Coloneqq#2}
\newcommand{\prodNextLine}[2]{&&#1&\mbox{#2}}
\newcommand{\terminal}[1]{\ensuremath{$\texttt{#1}$}}
%\newcommand{\metavariable}[1]{\ensuremath{\mathit{#1}}}

\newcommand\Rulename[1]{{\textsc{#1}}}
\newcommand\ctx[1]{\ensuremath{\mathcal{E}_#1}\!}
\newcommand{\lib}[3]{\Q@\{@\!#1\Q{;}\ #2 \Q{;}\ #3\Q@\}@}
\newcommand{\rp}[1]{\Q{(}\!#1\Q{)}}
\newcommand{\red}[3]{#1\rp{#2\Q{=}#3}}
\newcommand{\summ}[2]{#1\ \Q{<+}\ #2}
\newcommand{\mmid}{\ensuremath{\mid}}
\newcommand{\hole}{\ensuremath{\square}}
%--------------------------
\newcommand{\mynotes}[3]{{\color{#2} {\sc #1}: #3}}
\newcommand\isaac[1]{\mynotes{Isaac}{red}{#1}}
\newcommand\marco[1]{\mynotes{Marco}{blue}{#1}}




% The following code defines a macro, \markLine that puts it’s argument in the margins on either side of the next line
% it can be called multiple times for the same line, causing the arguments to placed next to eachother
\begin{comment} 
\usepackage{lineno}
\linenumbers
\renewcommand\makeLineNumber{}
\newbox\LeftMarkers \newbox\RightMarkers
\newcommand{\markLine}[2]{%
\setbox\LeftMarkers\hbox{#1\unhbox\LeftMarkers}%
\setbox\RightMarkers\hbox{\unhbox\RightMarkers#1}}

\newcommand{\markLineD}[1]{\markLineD{#1}{#1}} % just a shortcut
\renewcommand{\makeLineNumber}{%
\ifvoid\LeftMarkers%
\else \hss\unhcopy\LeftMarkers\ \rlap{\hskip\textwidth\ \unhbox\RightMarkers}%
\fi}
\end{comment}

\newcommand{\saveSpace}{\vspace{-3px}}
\newcommand{\loseSpace}{\vspace{1ex}}
\newcommand{\etal}{\emph{et~al.}\xspace}
\newcommand{\REV}[3]{%
	\NoteColour{red}{#1\NoteText{\footnote{%
				\textcolor{red}{\textbf{REV#2{:} #3}}}}}}
			
\newcommand{\REVm}[1]{\NoteColour{red}{#1\NoteText{\footnotemark}}}
\newcommand{\REVt}[2]{\footnotetext{\textcolor{red}{\textbf{REV#1{:} #2}}}}
			
\newcommand{\subheading}[1]{%
	\loseSpace%
	\noindent\textsf{\textbf{\large#1\\\noindent}}
}



\let\origSingleDot=\singleDot % reduce spacing arround the dot operator
\renewcommand{\singleDot}{\kern-1.2pt\origSingleDot\kern-1.8pt} 

% Hack (see https://tex.stackexchange.com/questions/299798/make-characters-active-via-macro-in-math-mode)
% this makes '.' in math mode an alias for \singleDot
\newcommand{\defActiveMathChar}[2]{%
	\begingroup\lccode`~=`#1\relax%
	\lowercase{\endgroup\def~}{#2}%
	\AtBeginDocument{\mathcode`#1="8000}%
}
\defActiveMathChar{.}{\singleDot}

\newcommand{\M}[3]{\ensuremath{\Kw{M}\oR{}#1\semiColon{}#2\semiColon{}#3\cR}}

\let\origMapsTo=\mapsto % put proper spacing arround the mapsto arrow
\renewcommand{\mapsto}{\mathrel{\origMapsTo}}
\newcommand{\invariant}{\Kw{invariant}\oR\cR}
\lstset{morekeywords={assert, expose, iso, isolated, baloon}}
\newcommand{\thm}[1]{\scalebox{0.9}[0.9]{\sf #1}}

%\HideNotes
\toggletrue{ECOOP}

\begin{document}

\maketitle

%TODO mandatory: add short abstract of the document
\begin{abstract}
In this paper we use pre-existing language support for type modifiers and object capabilities to enable a system for sound runtime verification of invariants.
Our invariant protocol is stricter then other  protocols, since it guarantees that class invariants hold for all objects involved in execution.
Invariants are specified simply as methods whose execution is statically guaranteed to be deterministic and not access any externally mutable state.
%We automatically call such invariant methods only when objects are created or the state they refer to may have been mutated.
Our design restricts the range of expressible invariants, but improves upon the usability and performance of prior work.
We soundly support mutation, dynamic dispatch, exceptions, and non-deterministic I/O, while requiring only a modest amount of annotation.
%\IOComm{Mention how we introduce a novel form of 'capsule' field, that prevents rep exposure}

We present case studies showing that our system requires a lower annotation burden compared to Spec\#, and  performs orders of magnitude less runtime invariant checks compared to the widely used `visible state semantics' protocols of D and Eiffel.
We formalise our approach and prove that our protocol soundly enforce invariants.
%that such pre-existing type modifier support is sufficient to ensure its soundness. %\IOComm{Mention other case studies?}
\end{abstract}

\section{Introduction}
\label{s:intro}
%\newpage
%\LINE
Representation invariants (sometimes called class invariants or object invariants) are
a useful concept when reasoning about software correctness in OO (Object Oriented) languages. Such invariants are predicates on the state of an object and its ROG (Reachable Object Graph).
They can be presented as documentation, checked as part of static verification, or, as we do in this paper, monitored for violations using runtime verification.
In our system, a class specifies its invariant by defining a method called \Q@invariant()@
that returns a boolean.
We say that an object's invariant holds when its \Q@invariant()@ method would return \Q@true@.\footnote{We do this (as in Dafny~\cite{DBLP:conf/sigada/Leino12}) to minimise the special treatment of invariants, whereas other approaches often treat invariants as a special annotation with its own syntax.}

Invariants are designed to hold most of the time, however it is commonly required to (temporarily) violate invariants while performing complex sequences of mutations.
To support this behaviour, most invariant protocols present in the literature allow invariants to be broken and observed broken.
The two main protocols are the \emph{visible state semantics} \cite{Meyer:1988:OSC:534929} and the \emph{Pack-Unpack/Boogie methodology}~\cite{DBLP:journals/jot/BarnettDFLS04}.
In the visible state semantics, invariants can be broken when a method on the object is active (that is, currently executing).
Some interpretations of the visible state are more permissive, requiring the invariants of receivers to hold only before and after every public method call, and after constructors. 
In the pack-unpack approach, objects are either in a `packed' or `unpacked' state, 
the invariant of `packed' objects must hold, whereas unpacked objects can be broken.

%------------
In this paper we propose a much stricter invariant protocol: at all times, the invariant of every object involved in execution must hold; thus they can be broken when the object is not (currently) involved in execution. 
An object is \emph{involved in execution} when it is in the ROG of any of the objects mentioned in the method call, field access, or field update that is about to be reduced; we state this more formally later in the paper.

%Our strict invariant protocol clearly supports easier reasoning; however 
Our strict protocol supports easier reasoning: an object can never be observed broken. However 
at first glance it may look overly restrictive, preventing useful program behaviour.
Consider the iconic example of a \Q@Range@ class, with a \Q@min@ and \Q@max@
value, where the invariant requires that \Q@min<=max@:
% ISAAC: I changed the example to not use getters and setters as our latter examples often don't and it dosn't add anything extra; in addition, this may fall foul of some visibile state semantics of the setters are considered public.
\begin{lstlisting}
class Range{ private field min; private field max;
  method invariant(){return min<max;}
  method set(min, max){
    if(min>=max){throw new Error(/**/);}
    this.min = min;
    this.max = max; }}
\end{lstlisting}
In this example we omit types to focus on the runtime semantics.
The code of \Q@set@ does not violate visible state semantics:
\Q@this.min@ \Q@=@ \Q@min@ may temporarily break the invariant of \Q!this!, however it will be fixed after executing \Q@this.max@ \Q@=@ \Q@max@. Visible state allows such temporary breaking of invariants since we are inside a method on \Q!this!, and by the time it returns, the invariant will be re-established.
However, if \Q@min@ is $\geq$ \Q@this.max@, \Q@set@ will violate our stricter approach. The execution of
\Q@this.min@ \Q@=@ \Q@min@ will break the invariant of \Q@this@ and \Q@this.max@ \Q@=@ \Q@max@ would then involve a broken object. If we were to inject a call
\Q@Do.stuff(this);@ between the two field updates, arbitrary user code could observe a broken object; 
  adding such a call is however allowed by visible state semantics.

Using the \emph{box pattern}, we can provide a modified
\Q@Range@ class with the desired client interface, while respecting the principles of our strict protocol:
\begin{lstlisting}
class BoxRange{//no invariant in BoxRange
  field min; field max;
  BoxRange(min, max){ this.set(min, max); }
  method Void set(min, max){
    if(min>=max){throw new Error(/**/);}
    this.min = min; this.max = max;
    }  }
class Range{ private field box; //box contains a BoxRange
  Range(min, max){ this.box = new BoxRange(min, max); }
  method invariant(){ return this.box.min < this.box.max; }
  method set(min, max){ return this.box.set(min,max); }
  }
\end{lstlisting}
The code of \Q@Range.set(min,max)@ does not violate our protocol.
% since \Q@this@ is not in the ROG of \Q@this.box@, \Q!min!, or \Q!max!. 
The call to
\Q@BoxRange.set(min,max)@ works in a context where the \Q@Range@ object is
unreachable, and thus not involved in execution.
That is, the \Q@Range@ object is not in the ROG of the receiver or the parameters of \Q@BoxRange.set(min,max)@.
 Thus \Q@Range.set(min,max)@ can temporarily break the \Q@Range@'s invariant.
By using the \Q@box@ field as an extra level of indirection, we restrict the set of objects involved in execution while the state of the object \Q@Range@ is modified.%
\footnote{Due to its simplicity and versatility, we do not claim this pattern to be a contribution of our work, as we expect others to have used it before. We have however not been able to find it referenced with a specific name in the literature, though technically speaking, it is a simplification of the Decorator, but with a different goal.
While in very specific situations the overhead of creating such additional box object may be unacceptable, we designed our work for environments where such fine performance differences are negligible.
Also note that many VMs and compilers can optimize away wrapper objects in many circumstances.~\cite{Bolz:2011:ARP:1929501.1929508}}
With appropriate type annotations, the code of \Q@Range@ and \Q@BoxRange@ is accepted as correct by our system: no matter how \Q@Range@ objects are used, a broken \Q@Range@ object will never be involved in execution.

\subheading{Contributions}
Invariant protocols allow for objects to make necessary changes that might make their invariant temporarily broken.
In visible state semantics any object that has an active method call anywhere on the call stacks is potentially invalid;
arguably not a sufficent guarantee as observed by
Gopinathan \etal's.~\cite{Gopinathan:2008:RMO:1483018.1483028}
Approaches such as \textit{pack/unpack}~\cite{DBLP:journals/jot/BarnettDFLS04} 
represent potentially invalid objects in the type system; this
encumbers the type system and the syntax with features whose only purpose is to distinguish objects with broken invariants.
%, while (at least in the case of Spec\#) still not soundly supporting I/O and exceptions.
The core insight behind our work 
is that we can use a small number of decorator-like design patterns to avoid exposing those potentially invalid objects 
in the first place, thus avoiding the need of representing them at the type level.

%In this paper we present  
%a general purpose language
% that does not require any \emph{invariant-specific} language mechanisms
% but instead we show how a clever use of capabilities
% is sufficient to capture invariant guarantees
% that are comparable to the state of the art object-oriented verification systems.

%Reasoning about class invariants in object-oriented verification needs to allow for objects to make necessary changes that might make their invariant temporarily broken. 
%The state of the art ranges from \textit{visible state semantics} that makes \textit{any} object that has an active method call anywhere on the call stacks potentially \textit{invalid} --- arguably not a very useful guarantee as observed by Gopinathan et al.~\cite{Gopinathan}.
%On the other hand, approaches such as \textit{pack/unpack}~\cite{SpecSharp} modify the type system with features that serve no general purpose other than help expose the semantics behind broken object invariants.
%The core insight behind our work is that we can design a general purpose language as presented in this paper that does not require any \emph{invariant-specific} language mechanisms but instead we show how a clever use of capabilities and small number of design patterns (Decorator-like \textit{Box} and \textit{Transformer}) is sufficient to capture invariant guarantees that are comparable to the state of the art object-oriented verification systems.

In the remainder of this paper, we discuss how to combine runtime checks and capabilities
to soundly enforce our strict invariant protocol.
Our solution only requires 
that all code is well-typed, and works in the presence of mutation, I/O, non-determinism, and exceptions, all under an open world assumption.

We formalise our approach and, in Appendix~\ref{s:proof}, prove that our use of Reference and Object Capabilities soundly enforces our invariant protocol.

We have fully implemented our protocol in L42\footnote{
Our implementation is implemented by checking that a given class conforms to our protocol, and injecting invariant checks in the appropriate places.
An anonymised version of L42, supporting the protocol described in this paper, together with the full code of our case studies, is available at \url{http://l42.is/InvariantArtifact.zip}. %We believe it would be easy to port our work on Pony and Gordon \etal's language.
}, we used this implementation to implement many case studies, showing that our protocol is more succinct than the pack/unpack approach and much more efficient then the visible state semantic.
It is important to note that unlike most prior work, we soundly handle catching of invariant failures and I/O.
%In our case study we show that
%we can still encode most of the examples explored in ~\cite{???} (including for example mutable collections of immutable objects) whilst having a significantly lower annotation-burden.
%--I think we can avoid this to save space
%Section \ref{s:TMsAndOCs} explains the pre-existing \emph{type modifier} features we use for this work.
%Section \ref{s:protocol} explains the details of our invariant protocol, and Section \ref{s:formalism} formalises a language enforcing this protocol.
%Sections \ref{s:immutable} and \ref{s:encapsulated} explain and motivate how our protocol can handle invariants over immutable and encapsulated mutable data, respectively.
%Section \ref{s:case-study} presents our GUI case study and compares it against visible state semantics and Spec\#: they performed 5 orders of magnitude more invariant checks, and required 60\% more annotations, respectively.
%Sections \ref{s:related} and \ref{s:conclusion} provide related work and conclusions.
We describe our case studies in Section~\ref{s:case-studyAll}.
Our approach may seem very restrictive;
the programming patterns in Section~\ref{s:patterns} show how our approach does not hamper expressiveness; in particular we show how batch mutation operations can be performed with a single invariant check, and how the state of a `broken' object can be safely passed around.
% In Appendix \ref{s:runtime-verification}, we discuss more related work on runtime verification.


%you see this already later on... I wanted to avoid repating it
%to perform batch operations with a single invariant check, as well as how the state of `broken' objects can be passed around.}	
%http://www.cs.cmu.edu/~NatProg/papers/p496-coblenz-Glacier-ICSE-2017.pdf

\section{Type Modifiers and Object Capabilities Background}
\label{s:TMsAndOCs}
Reasoning about imperative object-oriented (OO) programs is a non trivial task,
made particularly difficult by mutation, aliasing, dynamic dispatch, I/O, and exceptions. There are many ways to perform such reasoning, here we use the type system to restrict, but not prevent such behaviour in order to be able to soundly enforce invariants with runtime verification (RV).
% [dynamic class loading],

\subheading{Type Modifiers (TMs)}
TMs, as used in this paper, are a type system feature that allows reasoning about aliasing and mutation. Recently a new design for them has emerged that radically improves their usability;
three different research languages are being independently developed relying on this new design: the language of Gordon \etal~\cite{GordonEtAl12}, Pony~\cite{clebsch2015deny,clebsch2017orca}, and L42~\cite{ServettoZucca15,ServettoEtAl13a,JOT:issue_2011_01/article1,GianniniEtAl16}.
These projects are quite large: several million lines of code are written in Gordon \etal's language and are used by a large private Microsoft project; Pony and L42 have large libraries and are active open source projects. In particular the TMs of these languages are used to provide automatic and correct parallelism~\cite{GordonEtAl12,clebsch2015deny,clebsch2017orca,ServettoEtAl13a}.

While we focus on the specific TMs provided by L42, Pony, and Gordon \etal, type modifiers
 are a well known language mechanism~\cite{TschantzErnst05,BirkaErnst04,OstlundEtAl08,clebsch2015deny,GianniniEtAl16,GordonEtAl12}
 that allow static reasoning about mutability and aliasing properties of objects.
With slightly different names and semantics, the four \IODel{most common} modifiers \IO{we use} for \IO{object} references \IO{(i.e. expressions and variables)} \IODel{to objects} are:
\begin{itemize}
\item Mutable (\Q@mut@): the referenced object can be mutated\IO{, and freely shared/aliased}, as in most imperative languages without modifiers.
If all types are \Q@mut@, there is no restriction on aliasing/mutation.
\item \IODel{Readonly (\Q@read@): \ldots}
\item Immutable (\Q@imm@): the referenced object \IODel{can never mutate} \IO{cannot mutate, not even through other aliases}. \IODel{Like \Q@read@ references, one cannot mutate through an \Q@imm@ reference, however \Q@imm@ references also guarantee that the referenced object will not mutate through any other alias.} \IO{We call an object referred to by such a reference, an \emph{immutable object}; note that we don't prevent such an object from being mutated \emph{before} an immutable reference to it is made.} \IOComm{I'm not using the ROG definition of immutability, as we don't say our modifiers are deep untill later.}
\item Readonly (\Q@read@): the referenced object cannot be mutated by such references, but there may \IO{also} be mutable aliases to \IODel{such} \IO{the same} object, thus mutation can still be observed. \IO{Readonly references can refer to both mutable and immutable objects, since \Q!read! is a supertype of both \Q!imm! and \Q!mut!.}
\item Encapsulated (\Q@capsule@):
 \IODel{everything} \IO{every non-immutable object} in the reachable object graph (ROG) of a capsule reference (including itself) is \IODel{mutable} only \IO{reachable} through that reference\IODel{; \REV{however immutable references can be freely shared across capsule boundaries}{B}{what does [this] mean}}. \IO{This means that for any expression $e$ and $e'$, if $e$ is typable as \Q!capsule!, then the result of $e'$ is not in the ROG of the result of $e$, unless $e'$ evaluates to an immutable object (only possible if $e'$ is not typeable as \Q!mut! or \Q!capsule!).} \IO{An encapsulated reference can be freely promoted as mutable or immutable, since there could have been no other references to it.}
\end{itemize}
%In the context of object-oriented programming, type modifiers may also apply to the implicit \Q@this@ parameter in method declarations, restricting the type of references the method can be called on. In addition, due to the deep meanings we type field access on object references to be the most restrictive of the object references modifier and the field’s. As \Q@read@ references impose no assumptions about aliasing, any \Q@imm@ or \Q@mut@ expression can be safely implicitly promoted to \Q@read@, whereas other conversions are not generally safe.
%\loseSpace

\noindent TMs are different to field or variable modifiers like Java's \Q@final@: TMs apply to references, whereas \Q@final@ applies to fields themselves. Unlike a variable/field of a \Q@read@ type, a \Q@final@ variable/field cannot be reassigned, it always refers to the same object, however the variable/field can still be used to mutate the referenced object.
On the other hand, an object cannot be mutated through a \Q@read@ reference, however a \Q@read@ variable can still be reassigned.\footnote{In C, this is similar to the difference between \Q@A* const@ (like \Q@final@) and \Q@const A*@ (like \Q@read@), where \Q@const A* const@ is like \Q@final read@.}
%\end{itemize}

Consider the following  example usage of \Q@mut@, \Q@imm@, and \Q@read@, where we can observe a change in \Q@rp@ caused by a mutation inside \Q@mp@.
\begin{lstlisting}
 mut Point mp = new Point(1, 2);   mp.x = 3; // ok
 imm Point ip = new Point(1, 2); $\Comment{}$ip.x = 3; // type error
read Point rp = mp;              $\Comment{}$rp.x = 3; // type error
$\text{\IODel{// ok, read is a common supertype of imm/mut}}$
mp.x = 5; // ok, now we can observe rp.x == 5
ip = new Point(3, 5); // ok, ip is not final
\end{lstlisting} \IOComm{I made the above take less lines}


There are several possible interpretations of the semantics of type modifiers.
Here we assume the full/deep meaning~\cite{ZibinEtAl10,Potanin2013}:
\begin{itemize}
  \item the objects in the ROG of an immutable object are immutable,
  \item a mutable field accessed from a \Q@read@ reference produces a \Q@read@ reference,
%  \item no \emph{down}-casting is allowed between different type modifiers.
  \item no casting/promotion from \Q@read@ to \Q@mut@ is allowed.
%  \item promotion, is a type-system feature allowing implicit and safe casting from \Q@read@ and \Q@mut@ to \Q@imm@.
\end{itemize}

\noindent There are many different existing techniques and type systems that handle the modifiers above~\cite{ZibinEtAl10,ClarkeWrigstad03,HallerOdersky10,GordonEtAl12,ServettoZucca15}.
The main progress in the last few years is with the flexibility of such type systems:
 where the programmer should use \Q@imm@ when  representing immutable data
and \Q@mut@ nearly everywhere else. The system will be able to transparently promote/recover~\cite{GordonEtAl12,clebsch2015deny,ServettoZucca15} the type modifiers, adapting them to their use context.
To see a glimpse of this flexibility, consider the following example:
%//the same expression can create mut, imm or capsule
%\saveSpace
\begin{lstlisting}
    mut Circle mc = new Circle(new Point(0, 0), 7);
capsule Circle cc = new Circle(new Point(0, 0), 7);
    imm Circle ic = new Circle(new Point(0, 0), 7);
\end{lstlisting}

Here \Q@mc@, \Q@cc@, and \Q@ic@ are syntactically initialised with the same expression: \Q@new Circle(..)@.
The \Q@new@ expression returns a \Q@mut@, so \Q@mc@ is obviously ok.
%\footnote{Capsules must encapsulate their entire ROG, thus a \Q@new@ expression
%can not directly return \Q@capsule@ in the case of objects with \Q@mut@ fields.}
Moreover, the expression does not use any \Q@mut@ local variables, thus the flexible TM system
allows the \Q@mut@ result to be promoted to \Q@capsule@, thus \Q@cc@ is ok. 
Additionally, a \Q@capsule@ can be implicitly converted to \Q@imm@, thus \Q@ic@ is also ok.
We want to emphasise that this is not a special feature of \Q@new@ expressions:
any expression of a \Q@mut@ type that uses no free \Q@mut@ variables declared outside can be implicitly promoted to \Q@capsule@/\Q@imm@.\footnote{%
This requires some restrictions on \Q@read@ fields not discussed in detail for lack of space.
} This is the main improvement on the flexibility of TMs in recent literature~\cite{ServettoEtAl13a,ServettoZucca15,GordonEtAl12,clebsch2015deny,clebsch2017orca}.
Former work~\cite{Boyland10,boyland2003checking,Hogg91,Smith:2000:AT:645394.651903,DBLP:conf/pldi/AikenFKT03}, which eventually enabled the work of Gordon \etal's,  does not consider promotion and 
infers uniqueness/isolation/immutability only when starting from references that have been tracked with restrictive annotations along their whole lifetime.
From a usability perspective, this improvement means that
these TMs are opt-in: a programmer can write large sections of code
mindlessly using \Q@mut@ types and be free to have rampant aliasing. 
Then, at a later stage, another programmer may still 
be able to encapsulate those data structures into an \Q@imm@ or \Q@capsule@ reference.

%\saveSpace
%\begin{lstlisting}
%mc.radius = 3; // ok
%imm Point ip = ic.center; // ok, ROG immutable
%read Circle rc = mc
%read Point rp = rc.center; // ok, fields of read Circle are read
%$\Comment{}$mut Point mp = rc.center; // type error
%\end{lstlisting}
%\saveSpace
%Such flexibility is also visible where \Q@rc.center@ returns a \Q@read@ but \Q@ic.center@ returns an \Q@imm@: any expression typed as \Q@read@ that only
%uses immutable variables can safely be promoted to \Q@imm@ or \Q@capsule@.


 %(since \Q@ic@ is \Q@imm@, and \Q@imm@ is a deep modifier).
%true fact but not sufficient?

%With this kind of type system, we can ensure immutable classes by just declaring all their fields as final and immutable.%
%\footnote{
%In Java,  to ensure a class is immutable we need:
%the class must be final, all the fields must be final of immutable
%classes (thus no interface fields, final classes all the way down),
%and the SecurityManager need to properly tame reflection.}

% Not sure about this paragraph:

The \Q@capsule@ modifier (sometimes called isolated/\Q@iso@) is possibly the one whose details differ the most in the literature. Here we refer to the interpretation of~\cite{GordonEtAl12}, that introduced the concept of recovery/promotion.
This concept is the basis for L42, Pony, and Gordon \etal's type systems~\cite{GordonEtAl12,ServettoEtAl13a,ServettoZucca15,ServettoEtAl13a,clebsch2015deny,clebsch2017orca}. 

%\begin{itemize}
%	\item A capsule local variable can only be used once. %as \Q@capsule@ or \Q@mut@.
%	\item Only a capsule expression can be used to initialize or update a \Q@capsule@ field.
%	\item A capsule field access has the same type modifier as the receiver.
%	\item An expression of a \Q@mut@ type that uses no \Q@mut@ variables declared outside can be implicitly promoted to \Q@capsule@. Promotion/recovery is the main improvement on the flexibility of TM in recent literature~\cite{ServettoEtAl13a,ServettoZucca15,GordonEtAl12,clebsch2015deny,clebsch2017orca}
%	(this requires some restrictions on \Q@read@ fields that we do not discuss in detail for lack of space).
% \end{itemize}

%This is to ensure the capsule doesn't `leak', potentially violating it's exclusivity,

The capsule/isolated fields of Gordon \etal and Pony rely on destructive reads~\cite{GordonEtAl12,clebsch2015deny}: in order to read them, a new value (such as \Q!null!) will be assigned to them. In contrast, L42~\cite{ServettoEtAl13a,ServettoZucca15} does not require such destructive reads, thus \Q@capsule@ fields can be accessed many times, and their content can be seen from outside; but only in controlled ways.
Both Gordon \etal and Pony restrict how \Q!capsule! local variables can be used by changing the type they are seen as, however both allow the local variable to be `consumed', allowing them to be used as normal capsule/isolated expressions, at the cost of being unable to use the variable again. L42 however uses a simpler approach where all accesses to \Q!capsule! local variables consume them: they are expressed using linear/affine types~\cite{boyland2001alias}, thus they can only be used once.
%Both the capsule/isolated fields and variables of Pony and Gordon \etall rely on destructive reads~\cite{GordonEtAl12,clebsch2015deny}: reading such fields replaces their values with \Q@null@.+++ and a static analysis ensures capsule local variables can be accessed only once after the initialization and every update.

%In contrast,  L42~\cite{ServettoEtAl13a,ServettoZucca15} do not require destructive reads, and treat %\Q@capsule@ local variables and \Q@capsule@ fields differently:
%\Q@capsule@ local variables are expressed using linear/affine types~\cite{boyland2001alias}, thus they can only be used once;
%\Q@capsule@ fields can be accessed many times,
%and thus their content can be seen from outside; but only in controlled ways.

%while M\# and Pony requires both capsule fields and capsule variables to be `balloons'~\cite{Almeida97,ServettoEtAl13a} in the object graph.

%Destructive reads would be a bad idea for validation as they would likely invalidate objects.

%x=this.#f()
%..do all you want with x...
%// invariant check here!!
%.
%this.f(x -> ..)
%this.f=transform(this.f)
%invariantCheck()
%transform(this.#f())
%invariantCheck()

%\loseSpace
\subheading{Exceptions}\label{s:exceptions}
In most languages exceptions may be thrown at any point; combined with mutation this complicates reasoning about the state of programs after exceptions are caught: if an exception was thrown whilst mutating an object, what state is that object in? Does its invariant hold?
The concept of \emph{strong exception safety} (SES)~\cite{Abrahams2000,JOT:issue_2011_01/article1} simplifies reasoning:
if a \Q@try@--\Q@catch@ block caught an exception, the state visible before execution of the \Q@try@ block is unchanged, and the exception object does not expose any object that was being mutated.
%\LINE
%\noindent{\textit{Exceptions:}}
% M\#, L42 and Pony rely on SES for all unchecked exceptions to ensure safe and transparent parallelism,
% They wish to ensure the code behave as if the execution was fully sequential.
% Exceptions create additional difficulties in such context: if two operations are running in parallel in
% a fork-join, and the first one produces an exception, it should be safe to cancel the other operation and
% to propagate the exception outwards. The system need to guarantee
% the progress the second operation accumulated is not observable.
% Pony avoids this problem simply by not supporting exceptions;
% while
%M\# and L42 will parallelize only expressions that do not leak checked exceptions,
%and they enforce Strong Exception Safety(SES)~\cite{Abrahams2000} for unchecked exceptions.
%Other authors have identified the concept of SES as
% a general issue when reasoning about objects state after catching an exception.
% while we need it to soundly capture invariant failures.
L42 already enforces SES for unchecked exceptions.\footnote{%
This is needed to support safe parallelism. Pony takes a more drastic approach and does not support exceptions in the first place. 
We are not aware of how Gordon \etal handles exceptions, however in order for it to have sound unobservable parallelism it must have some restrictions.%
%We do not know how M\# conciliate deterministic parallelism and unchecked exceptions, we suspect some variation of SES must be in place.
}
L42 enforces SES using TMs in the following way:\footnote{Transactions are another way of enforcing strong exception safety, but they require specialized and costly run time support.}\footnote{A formal proof of why these restriction are sufficient is presented in the work of Lagorio~\cite{JOT:issue_2011_01/article1}.}
\begin{itemize}
\item Code inside a \Q@try@ block capturing unchecked exceptions is typed as if all \Q@mut@ variables declared outside of the block are \Q@read@.
\item Only \Q@imm@ objects may be thrown as unchecked exceptions.
\end{itemize} 
%Of course this has the effect that even if no-exception is thrown, no mutation could have occured, which is an even stronger property than SES, other work is more flexible~\cite{?}, at the cost of more complicated typing rules.
%With SES we can soundly capture invariant-failures as an exception, since any mutation that caused the invariant failure cannot be observed. However, we also need to prevent a broke-object from being reachable from the exception object; since the only way a broken-object can be seen is within the \Q@read@ \Q@invariant@  method, it follows that if the exception-object contains no \Q@read@ references in its ROG it cannot leak a broken object. Preventing this in the-typsystem is non-trivial, so instead we simply require that:

\noindent This strategy does not restrict throwing exceptions, but only catching unchecked ones.
SES allows us to soundly capture invariant failures as unchecked exceptions: 
the broken object is guaranteed to be garbage collectable when the exception is captured. For the purposes of soundly catching invariant failures, it would be sufficient to enforce SES only when capturing exceptions caused by such failures.
%The ability to catch and recover from such failures is extremely useful as it allows the program to take corrective action.(DUPLICATED)

% We think this restriction is acceptable for run time verification, other works are much more restrictive,



%The above rules need only be enforced for catch blocks that could catch invariant-failures (including exceptions thrown within execution of \Q@invariant@) itself;
%, and since \Q@invariant@ declares no checked exceptions, this includes all exceptions throw-able by it.

% TMs are very useful in restricting the scope of mutation. 
% Any expression that does not use any \Q@mut@ 
% variable declared outside of such expression does not modify objects visible outside.
% With this observation in mind, we can use TMs to enforce SES in the following way:\footnote{
% 

% \begin{itemize}
% \item all thrown exceptions are immutable objects,
% \item 
% \end{itemize}

% For the aim of enforcing invariants, we could relax SES to hold only when capturing exceptions caused by invariant failures; but we are building on approaches that enforce SES on all unchecked exceptions .


% Intro to OCs

\subheading{Object Capabilities (OCs)}
OCs, which L42, Pony, and Gordon \etal's work have, are a widely used~\cite{miller2003capability,
noble2016abstract,karger1988improving} programming style that allows associating resources with objects. When this style
is respected, code that does not possess an alias to such an object cannot use its associated resource.
%Object capabilities are programming style used to control and restrict use of operations such as access to external resources
Here, as in Gordon \etal's work, we use OCs to reason about determinism and I/O. To properly enforce this, the OC style needs to be respected while implementing the primitives of the standard library and when performing foreign function calls that could be non deterministic, such as operations that read from files or generate random numbers. Such operations would not be provided by static methods, but instead instance methods of classes whose instantiation is kept under control. 
% \noindent\REVComm{\textit{Object Capabilities:}}{2}{Citations here?}
% While type modifiers are statically verified properties of references, object capabilities are run time characteristics of specific objects.

% Conceptually, an object capability is a communicable, unforgeable token of authority, a key to access special functionality: only certain objects with `special' powers can do `special' actions, and those objects are obtained in a controlled way. We call such objects `capability objects'.


% Their main use case is to allow for fine grained control over what sections of code are allowed to do. 

\lstset{language=Java}
 For example, in Java, \Q@System.in@
 \lstset{language=FortyTwo} 
  is a \emph{capability object} that provides access to the standard input resource, however, as it is globally accessible it completely prevents reasoning about determinism. 
 % In contrast, in the object capability style, one would not have-global variables but have the main por

% a capability object (it has the capability to read input); however it is globally accessible: thus any code could use it, preventing reasoning about determinism.
In contrast, if Java were to respect the object capability style, the \Q@main@ method could take a \Q@System@ parameter, as in
 \Q@main(mut System s)@
 \lstset{language=Java}
\Q@{.. s.in.read() ..}@. \lstset{language=FortyTwo}%
Calling methods on that \Q@System@ instance would be the only way to perform I/O;
moreover, the only \Q@System@ instance would be the one created by the runtime system before calling \Q@main@. % would have no usable constructor, and all its I/O methods would require a mutable (\Q@mut@) receiver.
% Other non deterministic operations would also work this.
%may just take a \Q@mut System@ object as a parameter.
% could also work this way.
This design has been explored by Joe-E~\cite{finifter2008verifiable}.
OCs are typically not part of the type system nor do they require runtime checks or special support beyond that provided by a memory safe language. However, since
L42 allows user code to perform foreign calls without going through a predefined standard library, its type system enforces the OC pattern over such calls:
%To reason about determinism, L42 connects TMs with the OC style as follows: % style by requiring:
\begin{itemize}
\item Foreign methods (which have not been whitelisted as deterministic) and methods whose names start with \texttt{\#\$} are \emph{capability methods}.%
\item Constructors of classes declared as \emph{capability classes} are also capability methods.
\item Capability methods can only be called by other capability-methods or \Q@mut@/\Q@capsule@ methods of capability classes.
\item In L42 there is no \Q@main@ method, rather it has several main expressions; such expressions can also call capability methods, thus they can instantiate capability objects and pass them around to the rest of the program.
% \item Any method that uses non deterministic primitive operations (such as native calls or access to global variables\footnote{ Even just allowing unrestricted access to \Q@imm@ global variables would prevent reasoning over determinism due to the possibility of global variable updates; however constant/final globals of an \Q@imm@ type would not cause such problems.
% }) must be an instance method requiring a \Q@mut@ receiver.
% Classes having such methods are \emph{capability classes}, and their instances are \emph{capability objects}.
% \item A capability object can only be created inside a \Q@mut@ method of a capability class; or
% by the runtime system, and passed to the main method.

% \item If the language has global variables, they should only be 
%\item There are no global variables.\footnote{}
\end{itemize}

\noindent L42 expects capability methods to be used mostly internally by capability classes, whereas user code would call normal methods on already existing capability objects.

For the purposes of invariant checking, we only care about the effects that methods could have on the running program and heap. As such, \emph{output} methods (such as a \Q@print@ method) can be whitelisted as `deterministic', provided they do not affect program execution, such as by non deterministically throwing I/O errors.

\subheading{Purity}\label{s:purity}
TMs and OCs together statically guarantee that any method with only \Q!read! or \Q!imm! parameters (including the receiver) is \emph{pure}; we define pure
as being deterministic and not mutating existing memory. Such methods are pure because:
\begin{itemize}
	\item the ROG of the parameters (including \Q!this!) is only accessible as \Q@read@ (or \Q@imm@), thus it cannot be mutated\footnote{This is even true in the concurrent environments of Pony and Gordon \etal, since they ensure that no other thread/actor has access to a \Q@mut@/\Q@capsule@ alias of \Q@this@. 
	Thus, since such methods do not write to memory accessible by another thread, nor read memory that could be mutated by another thread, they are atomic.},
	\item if a capability object is in the ROG of any of the arguments (including the receiver), then it can only be accessed as \Q@read@, preventing calling any non deterministic (capability) methods,
	\item no other preexisting objects are accessible (as L42 does not have global variables).\footnote{%
		If L42 did have static variables, getters and setters for them would be capability methods.
		Even allowing unrestricted access to \Q@imm@
		static variables would prevent reasoning over
		determinism, due to the possibility of global variable
		updates; however constant/final globals of an 
		\Q@imm@ type would not cause such problems.%
	}
\end{itemize}

%Methods that perform non-deterministic \emph{input} shouldn't be white-listed.%, including methods that read information passed to white-listed output methods.


%Here we combine TMs with OCs to guarantee 
%\MS{determinism of} any method that can not access a \Q@mut@ reference to a capability object:
%all non deterministic primitive operations are instance methods requiring a \Q@mut@ receiver, and
%	\item all non-deterministic primitives (like native calls) require a \Q@mut@ receiver,
%instances of capability classes containing such methods can only be created by a \Q@mut@ method of another capability class
%	\begin{itemize}
%		\item the runtime-system\footnote{as 42 has no standard-library, we treat meta-code as the runtime-system} before main begins,
%		\item within a \Q@mut@ method on such a class
%	\end{itemize}
%	\item all non-deterministic operations require a \Q@mut@ receiver,
%	\item all classes 
%	\item there are no global variables\footnote{Note: even just allowing \Q@imm@
%global variables would prevent reasoning over determinism due to the possibility of global variable updates; however constant/final globals of an \Q@imm@ type would not cause such problems.},
% \item user code cannot directly create a capability object: they can only indirectly do so through an existing \Q@mut@ capability object reference.

% NOTE: SOMEWHERE MAKE IT CLEAR THAT NON-DETERMINISM CAN ONLY OCCUR THROUGH A CAPABILITY OBJECT
%\end{itemize}

%-----------------------------------------------------------
%define simple objects
%show solution  for simple person: requires 3 properties
%show solution is sound --> proof in appendix
%naive is unsound - person 3 bugs
\saveSpace
\section{Our Invariant Protocol}
\label{s:protocol}
\saveSpace
\IODel{In this section we will formalize our approach over a core language, and we formally state our soundness property (proved in the Appendix).
In the next section, by examples, we will show that all our restrictions and
requirements are actually needed, and that just violating any one of them would cause our system to be unsound.}

Our invariant protocol guarantees that the whole ROG of any object involved in execution (formally, in a redex) is \emph{valid}:\IODel{if you can call methods on an object,} calling \Q@invariant@ on it is guaranteed to return \Q@true@ in a finite number of steps. However, calls to \Q!invariant! that are generated by our runtime monitoring (see below) can access the fields of a potentially-invalid \Q!this!. This is necessary to allow for the \Q!invariant! method to do its job: namely distinguish between valid and invalid objects. However, as for any other method, calls to \Q!invariant! written explicitly by users are guaranteed to have a valid receiver. 

% However, the \Q!invariant! method itself needs to be able to operate on a potentially invalid \Q!this!, this will only happen when it is automatically called by the language itself, not by explicit calls present in the source code.
%\IODel{Clearly the \Q@invariant@ method must be able to take an invalid \Q@this@, since the purpose of such method is to distinguish valid and invalid objects. On a first look this may seem an open contradiction
% with the aim of this work, however only calls to \Q@invariant@ inserted by the language semantics can take an invalid \Q@this@. As for any other method, when the application code can call \Q@invariant@, \Q@this@ is guaranteed to be valid.
%\IODel{Also the bodies of constructors may interact with an invalid \Q@this@; we restrict their shape so that neither invariant methods nor constructors can observe the invalid object directly, but only its fields. }

%Logically, there are two reasons to access a field: we may wish to read the information stored in such object or we wish to mutate the object contained in the field.
%For the first case, we can type the field access as \Q@read@, but in the second case we
%need to type it as \Q@mut@. 
%We call `capsule mutators' a method accessing as \Q@mut@ a capsule field referenced in the invariant.
%We will show how capsule mutators are analogous of the pack/unpack/expose~\cite{???}.
%In order for a class to have an invariant under our protocol,
%\IODel{its }\Q@invariant@ method the form 
% Can  a program write say mut method invariant or is it syntactically [???]

We require that all classes contain a \Q@read method Bool invariant() {..}@, if no \Q!invariant! method is present, a trivial one returning \Q!true! will be assumed. As this method only takes a \Q!read! parameter (the receiver), we can be sure that it is pure \footnote{If the invariant were not pure, it would be nearly impossible to ensure that it would return \Q@true@ at any point.} (see Section \ref{s:purity}).\IOComm{Note about only leaking unchecked exceptions?}

We require that \Q@invariant@ methods only use \Q@this@ to read \Q@imm@ and \Q@capsule@ fields. This restriction ensures that unrelated code cannot break the invariants of arbitrary objects, since a \Q!read! or \Q!mut! field could be modified through arbitrary aliases (see Section \ref{s:immutable}). 
%To ensure that invariants cannot be broken by unrelated code (see Section \ref{s:immutable})  %\IODel{Access \Q@mut@ fields is forbidden because their ROG can be changed by unrelated code. In order to prevent passing an invalid \Q@this@ to other methods.}

In order to ensure that a broken object is not visible whilst mutating one of their capsule fields (see Section \ref{s:encapsulated}), 

A method that directly reads a \Q!capsule! field of a \Q!mut! receiver is said to be a \emph{capsule mutator}, unless the field is not read in the \Q!invariant! method of the enclosing class. We place the following static restrictions on capsule mutators:
\begin{itemize}
	\item the receiver of the \Q!capsule! field access must be \Q!this!,
	\item they must have no other (implicit) occurrences of \Q!this!,
	\item they cannot have any \Q!mut! or \Q!read! parameters (excluding the receiver), and
	\item they must not have a \Q!mut! return type.
	\item the method must not be declared to throw any checked exceptions\footnote{If we did allow this, we would have to inject invariant checks whenever checked-exceptions are thrown from such methods. However, this would make the runtime semantics of checked exceptions inconsistent with unchecked ones.},
\end{itemize}
Our type system will ensure that such methods are \Q!mut method!s, and the \Q!capsule! field will be seen as \Q!mut!. These restrictions allow \Q!capsule! fields to be freely mutated, potentially breaking the \Q!invariant! of the containing object, however as we cannot use \Q!this! and have no \Q!read! or \Q!mut! parameters, we can be sure that such broken object is not accessible during the execution. Preventing \Q!mut! return types ensures that the method cannot leak out a mutable alias to the \Q!capsule! field. Note that these restrictions do not apply when the receiver of the field access is \Q!capsule!, since we guarantee that the receiver is not in the ROG of such a field, and hence can never bee seen afterwards.
%\IOComm{mention these don't apply to \Q!capsule! recievers..}
%\IODel{\noindent Also note that \Q@invariant@ is declared as not throwing any exceptions, thus only unchecked exceptions can be propagated out.}

We do not allow explicit constructor definitions, rather we treat them as having the form \Q@$C$($T_1 x_1$,$\ldots$,$T_n x_n$) {this.$f_1$=$x_1$;$\ldots$;this.$f_n$=$x_n$;}@, where the fields of $C$ are $f_1,\ldots,f_n$ and have types $T_1,\ldots,T_n$. This ensures that partial-uninitialised (and likely invalid) objects are not passed around or used.

\subheading{Monitoring}
The language runtime will insert automatic calls to \Q!invariant!, if such a call returns \Q!false!, an unchecked-exception will be thrown. Such calls are inserted in the following points:
\begin{itemize}
	\item After a constructor call, on the newly created object.
	\item After a field update, on the receiver.
	\item After a capsule mutator method returns, on the receiver of the method\footnote{The invariant is not checked if the call was terminated via an an unchecked exception, since strong exception safety guarantees the object will be unreachable anyway.}.
\end{itemize}
In Appendix \ref{s:proof}, we show that these checks, together with our aforementioned restrictions, are sufficient to ensure our guarantee that all objects involved in execution (except as part of an invariant check) are valid.

\subheading{Relaxations}
The above restrictions can be partially relaxed without breaking soundness, however this would not make the proof more interesting. In particular:
\begin{itemize}
	\item \Q!invariant! methods can be allowed to call instance methods that in turn only use \Q@this@ to read \Q!imm! or \Q!capsule!, or call other such instance methods. With this relaxation, the semantics of \Q@invariant@ needs to be understood with the body of those methods inlined; thus the semantic of the inlined code needs to be logically reinterpreted in the context of \Q@invariant@, where \Q@this@ may be invalid. In some sense, those inlined methods and field accesses can be thought of as macro expanded, and hence are not dynamically dispatched. Such inlining has been implemented in L42.

	\item Unrestricted readonly access to \Q!capsule! fields can be allowed by automatically generate getters of the form \Q!read method read C f() { return this.f; }!. Such getters are already a fundamental part of the L42 language.
	
	\item Java style constructors could be allowed, provided that \Q!this! is only (implicitly) used as the receiver of field initialisations. L42 does not provide such constructors, but one can always write a static factory method that behaves equivalently.
\end{itemize}
\IOComm{Put the following paragraph somewhere else? Perhaps future work?}
Both L42, and our formal language (see Section~\ref{s:formalism}) do not have traditional subclassing, rather all `classes' are either interfaces (which only have abstract methods), or are final (which cannot be subtyped). In a language with subclassing, invariant methods would implicitly start with a check that \Q@super.invariant()@ returns \Q@true@. Note that invariant checks would not be performed at the end of \Q@super(..)@ constructor calls, but only at the end of \Q@new@ expressions, as happens in~\cite{feldman2006jose}.




\section{Background}
\label{s:background}
\noindent\textit{Type Modifiers:}
Type modifiers are a well known language mechanism~\cite{TschantzErnst05,BirkaErnst04,OstlundEtAl08,clebsch2015deny,GianniniEtAl16,GordonEtAl12} allowing static verification of mutability and aliasing properties of objects.
With sightly different names and semantics, the three most common modifiers for object references are:
\begin{itemize}
\item Mutable (\Q@mut@): the referenced object can be mutated, as in most languages without modifiers.
\item Readonly (\Q@read@): the referenced object cannot be mutated by such reference, but in the program there may be mutable references to this same object, so mutation can still be observed. 
\item Immutable (\Q@imm@): the referenced object can never mutate. Like \Q@read@ references, one cannot mutate through an \Q@imm@ reference, however \Q@imm@ references also guarantees that the referenced object will never be mutated, not even through another reference.
\end{itemize}
%In the context of object-oriented programming, type modifiers may also apply to the implicit \Q@this@ parameter in method declarations, restricting the type of references the method can be called on. In addition, due to the deep meanings we type field access on object references to be the most restrictive of the object references modifier and the field’s. As \Q@read@ references impose no assumptions about aliasing, any \Q@imm@ or \Q@mut@ expression can be safely implicitly promoted to \Q@read@, whereas other conversions are not generally safe.
%\loseSpace
TM are different to field or variable modifiers like Java’s \Q@final@: TM applies to references,  \Q@final@ specifies what can be done to the field itself. In comparison to \Q@imm@:

\begin{itemize}
\item A \Q@final@ variable/field cannot be reassigned, it always refers to the same object; however, the referenced object itself may be mutated.
\item A reference of an \Q@imm@ type however refers to an object that will never be mutated, and neither will its ROG. However, a field of type \Q@imm@ may be updated to another \Q@imm@ reference.
\footnote{In C, this is similar to the difference between \Q@const *A@ and \Q@*const A@, where a \Q@final imm@ variable would be like \Q@const *const A@.}
\end{itemize}



\noindent Consider the following  example usage of \Q@mut@, \Q@imm@ and \Q@read@:
\begin{lstlisting}
mut Point mp = new Point(1, 2);
mp.x = 3; // ok
imm Point ip = new Point(1, 2);
$\Comment{}$ip.x = 3; // type error
read Point rp = mp; // ok read is a common supertype of imm/mut
$\Comment{}$rp.x = 3; // type error
mp.x = 5; // ok, and now we can observe rp.x == 5
ip = new Point(3, 5); // ok, ip is not final
\end{lstlisting}
\noindent We cannot use a \Q@read@ reference to cause mutation, but we have no guarantee of the absence of mutation; in our example we can observe a change in \Q@rp@ caused by a mutation inside \Q@mp@.


There are several possible interpretations of the semantics of type modifiers.
Here we assume the full/deep meaning:
\begin{itemize}
  \item all the objects in the ROG of an immutable object are immutable;
  this corresponds to UML DataTypes,
  \item a mutable field accessed from a \Q@read@ reference produce a \Q@read@ reference,
  \item no \emph{down}-casting is allowed between different type modifiers.
\end{itemize}


\noindent There are many different existing techniques and type systems that handle the modifiers above~\cite{ZibinEtAl10,ClarkeWrigstad03,HallerOdersky10,GordonEtAl12,ServettoZucca15}.
The main progress in the last couple of years is with the flexibility of such type systems: where the programmer should use \Q@imm@ to represent objects that would obviously be modelled as UML DataTypes, and \Q@mut@ nearly everywhere else; the system will be able to transparently promote/recover~\cite{GordonEtAl12,clebsch2015deny,ServettoZucca15} the type modifiers, adapting them to their use context.
To see a glimpse of this flexibility, consider the following example:
\saveSpace
\begin{lstlisting}
mut Point mCenter = new Point(1, 2);
mut Circle mc = new Circle(mCenter, /*radius*/7);
mc.radius = 3; // ok
imm Circle ic = new Circle(new Point(0, 0), 7); // ok imm
imm Point ip = ic.center; // ok, ROG immutable
read Circle rc = mc
read Point rp = rc.center; // ok, fields of read Circle are read
$\Comment{}$mut Point mp = rc.center; // type error
\end{lstlisting}
\saveSpace

Here \Q@imm Circle ic@ and \Q@mut Circle mc@ are both initialized with \Q@new Circle(...)@.
This is not a special feature of \Q@new@ expressions: since \Q@new@ returns a \Q@mut@ and any expression typed as \Q@mut@ that only uses immutable variables can safely be promoted to \Q@imm@.
% (since the returned value could not possibly be aliased).
%FALSE: it can be internally aliased!
Such flexibility is also visible where \Q@rc.center@ returns a \Q@read@ but \Q@ic.center@ returns an \Q@imm@: any expression typed as read that only
uses immutable variables can safely be promoted to \Q@imm@.

 %(since \Q@ic@ is \Q@imm@, and \Q@imm@ is a deep modifier).
%true fact but not sufficient?

%With this kind of type system, we can ensure immutable classes by just declaring all their fields as final and immutable.%
%\footnote{
%In Java,  to ensure a class is immutable we need:
%the class must be final, all the fields must be final of immutable
%classes (thus no interface fields, final classes all the way down),
%and the SecurityManager need to properly tame reflection.}

% Not sure about this paragraph:



\loseSpace

TM are very useful in restricting the scope of mutation. 
Any expression that does not use any \Q@mut@ 
variable declared outside of such expression does not modify objects visible outside. This consideration is particularly useful to understand code in the presence of exceptions. Other authors have identified the concept of Strong Exception Safety~\cite{Abrahams2000} as a general issue when reasoning about objects state after catching an exception:
when a \Q@try-catch@ catches an exception, the visible objects must be the same as before the \Q@try@ block started its execution.
This can be obtained leveraging TM in the following way:
\begin{itemize}
\item all thrown exceptions are immutable objects,
\item code inside a \Q@try@ block is typed as if all \Q@mut@ variables declared outside of the block are \Q@read@.
\end{itemize}

\loseSpace
\noindent\textit{Object Capabilities:}
While type modifiers are statically verified properties of references, object capabilities are run time characteristics of specific objects.

Conceptually, an object capability is a communicable, unforgeable token of authority, a key to access special functionality: only certain objects with `special' powers can do `special' actions, and those objects are obtained in a controlled way. We call such objects `capability objects'.
Their main use case is to allow for fine-grained control over what sections of code are allowed to do. For example, in Java \Q@System.in@ is a capability object (it has the capability to read input); however it is globally accessible: thus any code could use it, preventing reasoning about determinism.
In a language enforcing object capabilities the \Q@main@ method could take a \Q@System@ object as a parameter, and using that object is the only way to perform I/O, as in \Q@mySystem.println("hello")@.
Moreover, the \Q@System@ class would have no visible constructor, and all its I/O methods would require a mutable (\Q@mut@) receiver.
Many other operations, like random numbers and file management 
%may just take a \Q@mut System@ object as a parameter.
could also work this way.
\noindent This design has been explored in literature by Joe-E~\cite{finifter2008verifiable}.

Here we use TM to guarantee that any method that is not (indirectly) passed a \Q@mut@ reference to a capability object will not use any capabilities:
\begin{itemize}
\item all capability-methods must require a \Q@mut@ receiver,
\item there are no global variables\footnote{Note: even just allowing \Q@imm@
global variables would prevent reasoning over determinism due to the possibility of global variable updates; however constant/final globals of an \Q@imm@ type would not cause such problems.},
\item user code cannot directly create a capability object: they can only indirectly do so through an existing \Q@mut@ capability object reference

% NOTE: SOMEWHERE MAKE IT CLEAR THAT NON-DETERMINISM CAN ONLY OCCUR THROUGH A CAPABILITY OBJECT
\end{itemize}

%-----------------------------------------------------------


%define simple objects
%show solution  for simple person: requires 3 properties

%show solution is sound --> proof in appendix
%naive is unsound - person 3 bugs


\section{The \Q@.validate()@ Method}
\label{s:validate}
Thanks to TM and OC, we can now express the signature for \Q@.validate()@ method as follows:
\saveSpace
\begin{lstlisting}
read method imm Bool validate();
\end{lstlisting}
\saveSpace
%The method is \Q@read@: thus the method body will see the \Q@this@ object as a \Q@read@ reference; and has no other parameters. 
%By starting from a \Q@read@ reference and nothing else, we are guaranteed that the method is pure:
If the class containing the validate method has a super-class, we would automatically check \Q@super.validate()@ at the beginning of the sub-class’s \Q@.validate()@ method, this is required as otherwise `invalid' objects could easily be created by simply using subtyping to redefine \Q@.validate()@.
As this method is declared as \Q@read@, and only takes the implicit parameter \Q@this@ (as \Q@read@), we can guarantee the method is pure:
\begin{itemize}
\item the ROG from \Q@this@ is only accessed as \Q@read@ (or \Q@imm@), thus it cannot be mutated\footnote{
This can even be safe in a multi-thread environment: TM are often used to ensure correct parallelism; for example threads may be required to not share \Q@mut@ data, thus a \Q@read@ reference could only be mutated by a \Q@mut@ reference under the control of the same thread.
},
\item if a capability object (such as \Q@System@) is in the ROG of \Q@this@, then it can only be accessed as \Q@read@, preventing use of its capability (such as I/O),
\item nothing else is accessible (we do not have global variables).
\end{itemize}

\noindent Also note \Q@.validate()@ is not declared as throwing any exceptions, thus it can only leak unchecked exceptions.


Clearly the \Q@.validate()@ method must be able to take an invalid \Q@this@, since the purpose of such method is to distinguish valid and invalid objects.\footnote{
At a first look this may seam an open contradiction
with the aim of this work, however only calls to validate inserted by the language semantic can take an invalid \Q@this@. As for any other method, when the application code calls \Q@.validate()@,
\Q@this@ is guaranteed to be valid.
} However, if we allow the method to use \Q@this@ directly (e.g. storing it in a local variable or passing it to a method), we would break the guarantees of validation (namely: `invalid objects are not reachable by application code'); as such we enforce the simple restriction that \Q@.validate()@ may only use \Q@this@ to access fields.
As a relaxation, we could allow calling instance methods that in turn only use \Q@this@ to access fields, or call other such instance methods. With this relaxation, the semantics of \Q@.validate()@ need to be understood with the body of those methods inlined; thus the semantic of the inlined code need to be logically reinterpreted in the context of \Q@.validate()@, where \Q@this@ may be invalid.
In some sense, those inlined methods and field accesses can be thought as macro-expanded.


Finally,
the code of \Q@.validate()@ can not access  \Q@mut@ and \Q@read@ fields, because their content can be changed by unrelated code.
Thus, with the modifiers presented so far, we can only access \Q@imm@ fields.
We will later introduce a \Q@capsule@ modifier to allow more flexible validation.


% JUSTIFY that the fields are valid...

%validable objects are not circular (do not belong in their ROG of any of its fields)
%validate as a predicate on object fields never really see this.
%
%clarifications:
%a validate check is never needed/generated/injected when working on a read x
%multi threading is not relevant/supported
%validable objects are not circular (do not belong in their ROG of any of its fields)

\section{Validating immutable state}
\label{s:immState}
In this section we consider validation over fields of \Q@imm@ types.\footnote{
In a real language, for conciseness one could make the \Q@imm@ modifier the default, allowing it to be omitted and our \Q@Person@ example class would only use 3 type modifiers; however we explicitly use it here for clarity.
}
In the next section we expand our technique.

In the following code \Q@Person@ has a single immutable (non final) field \Q@name@:
\begin{lstlisting}
class Person {
  read method imm Bool validate() {return !name.isEmpty();}
  private imm String name;
  read method imm String name() {return this.name;}
  mut method imm Strinig name(imm String name) {this.name = name;}
  Person(imm String name) {this.name = name;} }
\end{lstlisting}
\Q@Person@, only has immutable fields and the constructor only uses \Q@this@ to access/update fields.%; we say such a class is \emph{simple}.
%\Q@Person@, only has immutable fields and the constructor 
%uses the parameters to directly initialize (all) the fields.
% We say such a class is \emph{simple}.%
%\footnote{
%We consider only standard contractors for simplicity of exposition.
%More complex constructors could be supported, provided that \Q@this@ is only used to access fields, we do discuss them for simplicity.}
The difference with respect to UML DataTypes 
%immutable types (like UML DataTypes)
%UML datatypes are aclass property. immutable types are often an instance one (so no final fields) 
 is that fields are not required to be final, thus the object can change state during its lifetime. This means that the ROGs of all the \emph{fields} are immutable, but the object itself may be mutable.
%Of course UML DataTypes
%immutable types
%No, a type is not a class
% are just a special case of simple classes.
Validation for such a class can easily be enforced by generating checks on the result of \Q@.validate()@, immediately after each field update, and at the end of the constructor\footnote{since the constructor only initialises fields, as with the \Q@.validate()@ method itself, we don't check for validity during the constructor as \Q@this@ is not directly reached, doing so would require the initial/default value of \Q@this@ to be valid.}:
% If a simple class provides a \Q@.validate()@ method, then validation will be enforced.
% For \Q@Person@, intuitively, the code would behave as follow:

%\Comment{if we made this public, all users who update the field need to call validate}%
%There are many interpretations for your comment
%why you deleted my code comments?
\begin{lstlisting}
class Person {
  read method imm Bool validate() {return !name.isEmpty();}
  private imm String name;
  read method imm String name() {return this.name;}
  mut method imm String name(imm String name) {
    this.name = name; // check after field update
    if (!this.validate()) {throw new Error(...);} 
  }
  Person(imm String name) {
    this.name = name; // check at end of constructor
    if (!this.validate()) {throw new Error(...);}
}}// Generated code, not directly written by the programmer
\end{lstlisting}
%... $\MComment{validation error}$ 

% Many programmers attempted to write similar code in mainstream languages like Java to ensure  that some property always holds. Indeed, at first look, this code seems to correctly enforce validation. Sadly, without relying on TM and OC, the former code would be broken: just making the fields private and checking the \Q@.validate()@ method at the \textbf{end of the constructor} and at the \textbf{end of mutator methods} is not enough to enforce validation.
% The trick is that our intuition relies not on statically verified properties, or on the semantics of the language, but on the expectations about `correct' behaviour of \Q@String@. We need to enforce Validation without assuming the behaviour of other objects.

If we were to relax (as in Rust) or even eliminate (as in Java) the support for TM or OC, the validation of this simple \Q@Person@ class would become harder, or even impossible. We now proceed to show examples where
relaxation of TM or OC breaks our validation. 

\loseSpace
\noindent\textit{Unrestricted use of object-capabilities:}
Allowing \Q@.validate()@ to (indirectly) use an object capability could allow for it to be non deterministic, by either:
\begin{itemize}
\item allowing \Q@.validate()@ to (indirectly) access a \Q@mut@ reference to a capability-object,
\item relaxing the rule that capability-methods must have a \Q@mut@ receiver, or
\item allowing capability objects to be created out of thin air.
\end{itemize}

For example consider this simple and contrived (mis)use of person:
\begin{lstlisting}
class EvilString extends String {
  @Override read method Bool isEmpty() {
    // Creates a new capability object out of thin air
    return new Random().bool();
} }
...
imm method mut Person createPersons(imm String name){
  // we can not be sure wether name is an `EvilString'
  mut Person schrodinger1 = new Person(name); // exception here?
  mut Person schrodinger2 = new Person(name); // what about here?
  ...}
\end{lstlisting}
Despite the code for \Q@Persion.validate()@ intuitively looking correct and deterministic, the above calls to it are not. Obviously this breaks any reasoning and makes our validation unsound. 
In particular, note how in the presence of dynamic class loading, 
we can not make any assumption on the dynamic type of \Q@name@.
%???
%Even if we disallow subtyping the same problem could still occur if we had a strange implementing of \Q@String@, or \Q@Persion.validate()@ itself.

\loseSpace
\noindent\textit{Allowing internal mutations/back-doors:}
% TODO: Come up with better title
Suppose we relax our mutation rules, by allowing interior mutability
as in Rust and Javari: thus allowing mutation of the ROG of an immutable object through back-doors:
\begin{lstlisting}
class AtomicBool {
  imm method imm Void store(imm Bool val){
    ... // Mutate through an imm reciever
  }
}

class NastyString extends String {
  imm AtomicBool evil = new AtomicBool(false);
  imm method imm Void nasty() {
    this.evil.store(true); // this imm method can do mutation
  }

  @Override read method Bool isEmpty() {
    return this.evil.load() ? false : super.isEmpty();
  }
  ...
}
...
imm NastyString name = new NastyString("bob");
imm Person person = new Person(name); // person is valid
name.nasty(); // mutate the ROG of person, without it noticing
// person is now invalid!
\end{lstlisting}

In this example we use \Q@AtomicBool@ as a back-door to remotely break the invariant of \Q@person@ without any interaction with the \Q@person@ object itself.
%mine: yes, too strong: For validation we need the language to guarantee true deep immutability.
%your: just points outside: It would require some powerful static or dynamic analysis to keep track of every case the ROG of \Q@Person@ could be indirectly mutated, and insert validity checks appropriately, however ensuring deep mutability trivialises this for simple classes.


Allowing such back-doors could also be used to break the determinism of the \Q@.validate()@ method, by allowing it to store and read information about previous calls.

%In our simple example, \Q@Person@ objects can be mutated using the setter, and exposed using the getter.
%We may consider the getter to be safe since in modern languages we expect strings to be immutable objects.
%\footnote{While we can update the field \Q@name@ to point to another string, we cannot mutate the string object itself.
%To obtain  \Q@"Hello"@ from \Q@"hello"@ we need to create a whole new string object that looks like the old one except for the first character. This would be different in older languages like C, where strings are just mutable arrays of characters.}
%
%Again, the assumption that they are immutable depends on the correctness of the code inside \Q@String@: if there was a bug in the \Q@String@ class, or any \Q@String@ subclass, then executing 
%\Q@println(bob.name())@ may change \Q@bob@ by quietly changing a part of its ROG.
%Again, checking
%what methods mutate states cannot be responsibility of the \Q@Person@ programmer.
%For Validation we need a language supporting aliasing and mutability control.
%\begin{comment}
%\item Sample Bug 1:
%Suppose there was a bug in \Q@String.isEmpty()@, causing the method to non-deterministically return \Q@true@ or \Q@false@.
%What would it mean for Validation?
%Would a \Q@Person@ be at the same time 
%valid and invalid?
%
%Only deterministic methods can be used for validation.
%Ensuring this cannot be responsibility of the \Q@Person@ programmer, since it may depend on third party code, as shown in this example.
%However, statically checking if a method is deterministic is hard/impossible in most imperative object-oriented languages.
%
%While we may not expect the presence of bugs in the standard library class \Q@String@, the same behaviour can be achieved with subtyping:
%\saveSpace
%\begin{lstlisting}
%class EvilStr extends String{
%  method Bool isEmpty(){
%    return new Random().bool();
%  }}
%...
%String name=...$\Comment{can this be an EvilStr?}$
%Person bob=new Person(name);
%\end{lstlisting}
%\saveSpace
%As you can see, it is hard to make sound claims about Validation.
%
%\item Sample Bug 2:
%In our simple example, \Q@Person@ objects can be mutated using the setter, and exposed using the getter.
%We may consider the getter to be safe since in modern languages we expect strings to be immutable objects.
%\footnote{While we can update the field \Q@name@ to point to another string, we cannot mutate the string object itself.
%To obtain  \Q@"Hello"@ from \Q@"hello"@ we need to create a whole new string object that looks like the old one except for the first character. This would be different in older languages like C, where strings are just mutable arrays of characters.}
%
%Again, the assumption that they are immutable depends on the correctness of the code inside \Q@String@: if there was a bug in the \Q@String@ class, or any \Q@String@ subclass, then executing 
%\Q@println(bob.name())@ may change \Q@bob@ by quietly changing a part of its ROG.
%
%Again, checking
%what methods mutate states cannot be responsibility of the \Q@Person@ programmer.
%For Validation we need a language supporting aliasing and mutability control.
%\end{comment}

\loseSpace
\noindent\textit{Strong Exception Safety:}
The ability to catch and recover from validation failures is extremely useful as it allows the program to take corrective action.
This may be implemented with a conventional \Q@try-catch@, since violations are represented by throwing unchecked exceptions. Due to the guarantees of strong exception safety, the only trace that the invalid object existed is the exception thrown; any object that has been mutated/created during the \Q@try@ block is now unreachable (as happens in alias burying~\cite{boyland2001alias}).

However, if instead we chose not to enforce strong exception safety, an invalid object could be easily made reachable:
\saveSpace
\begin{lstlisting}[morekeywords={assert}, escapechar=\%]
mut Person bob = new Person("bob");
// Catch and ignore validation failure:
try {bob.name("");} catch (imm Error t){}
assert bob.name().isEmpty(); // now we have a rechable invalid object!
\end{lstlisting}
\saveSpace
As you can see, recovering from a validation failure in this way is unsound and breaks the guarantees of validation.
Strong exception safety is a useful property to enforce, but for the specific purpose of validation this could be relaxed by restricting only \Q@try-catch@ blocks that could capture unchecked exceptions.
Since calls to \Q@.validate()@ may only throw unchecked-exceptions, violating strong exception safety within a \Q@try-catch@ that cannot catch unchecked-exceptions would not break validation.

%LATER: This means that we could relax our Strong Exception Safety to hold only on unchecked exceptions (by restricting only \Q@try-catch@ blocks that capture unchecked exceptions.



% One of the advantages of checking Validation at run time, is that
% we can allow the program can take corrective actions if a property is violated.
% This may be implemented with a conventional \Q@try-catch@ if violations are represented by throwing errors.
% However, there is an issue with exceptions modelling invalid objects: they can be captured when the invalid object is still in scope. For example:


%As you can see, if we can capture validation failures as normal exceptions %(very desirable feature) then we may end up using invalid objects.
%Moreover,
% as shown before with the example of transferring cargo between two boats,
%after an invariant has been violated, even objects with valid invariant may be in an unexpected state.

% This situation is a general issue about reasoning on the state after recovering from exceptions.
% In particular, as shown in the example this prevent sound validation.

% Note how this produces a different semantics with respect to static verification, where violations
% never happened. However this will not necessarily lead to a broken semantics:
%Thanks to Strong exception safety we have a system where either the application terminate
%when an invalid object is detected, or where any witness of the execution causing the invalid object is erased from history
%those objects and all the witnesses will be garbage collected
% (as happens in alias burying~\cite{boyland2001alias}).
%In our example, this means that to continue execution after a detected bug, 
%we would require to garbage collect the overloaded boat, their cargo and probably most of the commercial port too.








%\subsubsection*{Solving Issue 3: Constructors}
%\saveSpace
%Exposing \Q@this@ during construction is a generally recognized problem~\cite{gil2009we}.
%A simple solution is to require all constructors to 
%simply take a parameter for each field and to just initialize the fields.
%An advantage of such approach is syntactic brevity: constructors are implicitly defined
%by the set of fields and thus there is no need to define them manually.
%\textbf{Expressive initialization operations can still be performed, by following the factory pattern.}
%\saveSpace


%\subsubsection*{Recap}
%By utilising type modifiers (\Q@imm@, \Q@mut@ and \Q@read@), object capabilities and immutable exceptions we obtain sound runtime verification for immutable classes/UML data types.
\subheading{Relaxing Restrictions on Rep Fields?}
%\section{Invariants over encapsulated state}
%\label{s:encapsulated}
Rep fields allow expressing invariants over mutable object graphs.
Consider managing the shipment of items, where there is a maximum combined weight:
\begin{lstlisting}
class ShippingList {
  rep Items items;
  read method Bool invariant(){ return this.items.weight()<=300; }
  ShippingList(capsule Items items) {
    this.items = items;
    if (!this.invariant()){ throw Error(...); }//injected check
  }
  mut method Void addItem(Item item) {
    this.items.add(item);
    if (!this.invariant()){ throw Error(...); }//injected check
  }
}
\end{lstlisting}
We inject calls to \Q@invariant()@ at the end of the constructor and the \Q@addItem(item)@ method.
This is safe since the \Q@items@ field is declared \Q@rep@.
Relaxing our system to allow a \Q@mut@ reference capability for
the \Q@items@ field and the corresponding constructor parameter would 
make the above checks insufficient:
it would be possible for external code with no knowledge of the \Q@ShippingList@ to mutate its items. 
%Conventional ownership solves these problems by requiring a deep clone of all the data the constructor takes as input, as well as all exposed data (possibly through getters). % Isaac: I'm not sure this is correct, ownership transfer is a thing I've seen before, also freshly created objects would also be fine
In order to write correct library code in mainstream languages like Java and C++, defensive cloning~\cite{Bloch08,Detlefs98wrestlingwith} is needed.
For performance reasons, this is hardly done in practice and is a continuous source of bugs and unexpected behaviour.%

%\saveSpace
\begin{lstlisting}
mut Items items = ...;//INVALID EXAMPLE
mut ShippingList l = new ShippingList(items); // l is valid
items.addItem(new HeavyItem()); // l is now invalid!
\end{lstlisting}
If we were to allow \Q@x.items@ to be seen as \Q@mut@, where \Q@x@ is not \Q@this@, then  even if the \Q@ShippingList@ has full control of \Q!items! at initialisation time, such control may be lost later, and code unaware of the \Q@ShippingList@ could break it:
\begin{lstlisting}
//INVALID EXAMPLE: l.items can be exposed as mut
mut ShippingList l = new ShippingList(new Items()); // l is ok
mut Items evilAlias = l.items; // here l loses control
evilAlias.addItem(new HeavyItem()); // now l is invalid!
\end{lstlisting}
Relaxing our requirements for rep mutators
would break our protocol: if rep mutators could have a \Q@mut@ return type the following would be accepted:
\begin{lstlisting}
//INVALID EXAMPLE: rep mutator expose(c) return type is mut
mut method mut Items expose(C c) {return c.foo(this.items);}
\end{lstlisting}
Depending on dynamic dispatch, \Q@c.foo()@ may just be the identity function, thus
we would get in the same situation as the former example.
%Static analysis is usually unable/unwilling to track precise behaviour of dynamic dispatch.


%In addition to the above we put restrictions on any \Q@mut@ and \Q@capsule@ methods using a \Q@capsule@ field (we call such methods `rep mutators'):
%\begin{itemize}
%\item only a single use of \Q@this@ is allowed (and is the one that uses the field),
%\item no \Q@mut@ or \Q@read@ parameters are allowed (apart from the implicit \Q@this@ parameter)
%\item and the return type cannot be \Q@mut@.
%\end{itemize}
%\noindent  Moreover, if the used \Q!capsule! field is referenced in \validate, a \Q@this.validate()@ call is injected at the end of the method.


Allowing \Q@this@ to be used more than once 
would allow the following code, where 
\Q@this@ may be reachable from \Q@f@, thus \Q@f.hi()@ may observe an object that does not satisfying its invariant:
\begin{lstlisting}
mut method Void multiThis(C c) {//INVALID EXAMPLE: two `this'
  read Foo f = c.foo(this);
  this.items.add(new HeavyItem());
  f.hi(); }//`this' could be observed here if it is in ROG(f)
\end{lstlisting}
\noindent In order to ensure that a second reference to \Q@this@ is not reachable through arguments to such methods, we only allow \Q@imm@ and \Q@capsule@ parameters.
Accepting a \Q@read@ parameter, as in the example below,
would cause the same problems as before, where \Q@f@ may contain
a reference to \Q@this@:
\begin{lstlisting}
mut method Void addHeavy(read Foo f) {//INVALID EXAMPLE
  this.items.add(new HeavyItem());
  f.hi(); }//`this' could be observed here if it is in ROG(f)
...
mut ShippingList l = new ShippingList(new Items());
read Foo f = new Foo(l);
l.addHeavy(f); // We pass another reference to `l' through f
\end{lstlisting}%
%
%, we would have the same problem with a \Q@read@ paramater. ... justify why we ned capsule
% The boat will sink if the weight of the cargo goes over 300. However, 
% \Q@Item@ and \Q@Items@ come from a third party library,  are not annotated with contracts and the authors may change their behaviour in the future. 
% All the code using \Q@Boat@  (client code) would like to  assume the boat has not sunk yet.
% In turn, that depends on the behaviour of \Q@Items.weight()@, thus the meaning of the \Q@Boat@ invariant is parametric on the everchanging meaning of  \Q@Items.weight()@.
% Can the code in the \Q@Boat@ class somehow enforce that for every possible well typed \Q@Item@ and \Q@Items@, client code will interact only with valid (non sunk)  boats?
% That is, we are unable or unwilling to constrain \Q@Item@ and \Q@Items@ to
% cooperate into making \Q@Boat@s unsinkable; 
% we aim to make so that \Q@Boat@s can be correct independently of
% possibly buggy, possibly even malicious \Q@Item@ and \Q@Items@ implementations.
% Indeed, thanks to the encapsulation, any kind of check in the language,
% as in `\Q@if(cargo.weight()>=300){..}@', would delegate the 
% behaviour to untrusted code in \Q@Items@.
%
% \textbf{without any knowledge about the behaviour of \Q@add()@ and \Q@weight()@}
% \footnote{A statically verified system with contracts on all methods may have this kind of knowledge.}
% there is no way we can discover the invariant violation without actually adding the objects and checking the 
% weight after the fact; thus in the general case violations can only be detected 
% when a broken object is already present in the system.
% Remember that to keep our approach lightweight,
% we do not rely on pre-post conditions; thus
% the behaviour of \Q@Items.weight()@ and \Q@Items.add(item)@ is uncertain.
% The names may suggest a specific behaviour, but there is no contract annotated on such methods.
%
% Note also that in the general case there is no way to fix a broken object,
%or to perform a deep clone and to test the operation on the clone first.
%
%
%REWRITE THIS BIT
%Here \Q!capsule! fields 
%as input to our code-generation / \Q@validate()@-injection; that is, \Q@capsule T f@ is expanded by the language into:
%\begin{itemize}
%\item Induce a \Q@capsule@ parameter for the generated %constructor.
%\item Require to be updated with a \Q@capsule@ expression.
%\item Are accessed as a \Q@mut@ field.
%Access is \textbf{not} a destructive read.
% However methods accessing them are kept under
%strict control; either
%\begin{itemize}
%\item they have \Q@read this@: they act like a normal %getter, and can not propagate
%writing permission over the reachable object graph of that field.
%Indeed, with \Q@read this@, any field access \Q@this.f@ will be typed \Q@read@ or \Q@imm@.
%\item they have \Q@mut this@, no parameter is \Q@mut@ or \Q@read@,
%the return type is not \Q@mut@ and \Q@this@ appear exactly one time in
%the method body: we call those methods \textbf{exposers}, and the invariant is going to be checked at the end of
%the exposers.
%\end{itemize}
%
%
%\end{itemize}
%Exposers are the key part of our solution.
%
%
% Those restrictions also enforce that while executing a rep mutator no object outside the reachable object graph of \Q@this@ can be mutated, and thus capability objects cannot be usedI/O can not be performed: the capability objects are externally visible mutable objects and thus the type system will never place them into a \Q@capsule@.
%\subheading{The true expressibility of capsule modifiers}
%A rep mutator method is a wrapper of a logical operation on a field, which is guaranteed to not see the \Q@this@ object.
%Thus, if \Q@this@ where to become broken during 
%the method's execution, we could not observe it until after. At first glance, it may seems that capsule %mutators allows for limited kinds of mutations.
%This is however not the case, consider the following
%general rep mutator method that allows to apply any possible transformation over the content of a capsule %field:
%At first glance it mayseem from
%

\saveSpace
\section{Proof of Theorem 1 and axiomatic type properties}
\label{s:meaning}
\saveSpace

\textit{Axiomatic type properties:}
As previously discussed, instead of providing a concrete set of type rules, we provide a set of properties
that the type system needs to respect.
To express these properties, we first need some auxiliary definitions.

%\noindent\textbf{Define}
%$\mathit{encapsulatedObj}(C)$:\\*
%${}_{}$\quad\quad \Q@class @$C$\,\Q@implements @$\Many{C}$\Q@{@$\,\Many{F}\,\Many{M}$\Q@}@
% and $\forall \mdf\,C\,\f \in \Many{F},\ \mdf \in \{\Kw{imm},\Kw{capsule}\}$\\*
%\noindent As we discussed, only encapsulated objects can support invariants;
%their class declarations only have immutable or capsule fields. Note how here we see immutable
%and simple objects as special cases of encapsulated ones.

The encapsulated ROG of $l_0$ is composed of all the objects
in the ROG of its immutable and capsule fields:\\*
\indent $l \in \mathit{erog}(\sigma,l_0)
\text{ if } \Sigma^\sigma(l_0).f \in \{\Kw{imm}\,\_,\Kw{capsule}\,\_\}
\text{ and } l \in \mathit{rog}(\sigma,\sigma(l_0).f)
$\loseSpace

\noindent An object is \emph{mutatable} in a $\sigma$ and  $\e$ if there is an occurrence of 
$l$ in $e$, that when seen as \Q@imm@ makes the expression ill typed:\\*
$\mathit{mutatable}(l,\sigma,\e) \text{iff}$ for some $T=\Kw{imm}\,\Sigma^\sigma(l)$ and $\ctx[l]=\e$,\\*
\indent $\Sigma^\sigma;\x:T\vdash\ctx[\x]:T'$ does not hold for any $T'$.\loseSpace

%if $\ \sigma_0|e_0\rightarrow \sigma|e$ then $\sigma_1|\e_1=\sigma|\e$
% $\exists! \sigma_1|\e_1$ such that $\sigma_0|\e_0\rightarrow \sigma_1|\e_1$\\*

%We can now assume the following properties over the type system:

\begin{Assumption}[Progress]
if $\Sigma^{\sigma_0};\emptyset\vdash e_0: T_0$,
and $e_0$ is not a value or $\mathit{error}$, then
$\sigma_0|e_0\rightarrow \sigma_1|e_1$.
\end{Assumption}


\begin{Assumption}[Subject Reduction Base]
if $\Sigma^{\sigma_0};\emptyset\vdash e_0: T_0$,
$\sigma_0|e_0\rightarrow \sigma_1|e_1$,
then
$\Sigma^{\sigma_1};\emptyset\vdash e_1: T_1$.
\end{Assumption}


\noindent If the result of a field access is mutable,
the receiver is also mutable:\saveSpace\saveSpace
\begin{Assumption}[Mut Field]
\ \\
\indent(1)\ if $\Sigma;\Gamma\vdash\e\singleDot\f:\Kw{mut}\,\_$
then $\Sigma;\Gamma\vdash\e:\Kw{mut}\,\_$
 and 
\\*\indent(2)
if $\Sigma;\Gamma\vdash\e_0\singleDot\f\equals\e_1:T$
then $\Sigma;\Gamma\vdash\e_0:\Kw{mut}\,\_$.
\end{Assumption}

\noindent An object is not part of the ROG of its immutable or capsule fields:\saveSpace\saveSpace
\begin{Assumption}[Head Not Circular]
if
$\Sigma^\sigma;\Gamma\vdash l:T$,
then $l\notin\text{erog}(\sigma,l)$.
\end{Assumption}


\noindent In a well typed $\sigma$ and $e$, if mutatable $l_2$ is reachable from
$l_1$, and $l_1$ is reachable from $l_0$,
then all the paths connecting $l_0$ and $l_2$ pass trough $l_1$; thus
if we were to remove $l_1$ from the object graph, $l_0$ would no longer reach $l_2$:
\saveSpace\saveSpace
\begin{Assumption}[Capsule Tree]
If   $\Sigma^\sigma;\Gamma\vdash \e:\T$,
$l_2\in\text{erog}(\sigma,l_1)$,
$l_1\in\text{erog}(\sigma,l_0)$,\\*
and
$\mathit{mutatable}(l_2,\sigma,\e)$
then 
$l_2\notin\text{erog}(\sigma\setminus l_1,l_0)$.
\end{Assumption}


Capsule Tree and Head Not Circular together 
imply that capsule fields section the object graph into a tree of nested `balloons',
where nodes are mutable encapsulated objects and
edges are given by reachability between those objects in the original memory:
$l_2$ is in the encapsulated ROG of $l_1$;
$l_2$ is mutatable and reachable through $l_1$, thus
it must be reachable by a \Q@capsule@ field.
Thanks to Head Not Circular and $l_1\in\text{erog}(\sigma,l_0)$ we can derive 
$l_0\notin\text{erog}(\sigma,l_1)$.

The execution of an expression
with no \Q@mut@ free variables is deterministic and does not
mutate pre existing memory (and thus does not not perform I/O by mutating pre existing $c$):
\begin{Assumption}[Determinism]
if $\emptyset;\Gamma\vdash \e:\T$, 
$\forall x \Gamma(x)\neq\Kw{mut}\,\_$, and
$\sigma | \e'\rightarrow^+ \sigma' | \e''$
then 
$\sigma | \e'\Rightarrow^+ \sigma,\_ | \e''$,
where $\e'=\e[x_1=l_1,\ldots,x_n=l_n]$ and $\Sigma^\sigma;\emptyset\vdash \e':\T$
\end{Assumption}

\begin{Assumption}[StrongExceptionSafety]
if $\Sigma^{\sigma,\sigma'};\emptyset\vdash \ctx[\Kw{try}^\sigma\oC\e_0\cC\ \Kw{catch}\ \oC\e_1\cC]:\T$
and\\*
$
\sigma,\sigma'|\ctx[\Kw{try}^\sigma\oC\e_0\cC\ \Kw{catch}\ \oC\e_1\cC]\rightarrow 
\sigma''|\ctx[\Kw{try}^\sigma\oC\e'\cC\ \Kw{catch}\ \oC\e_1\cC]
$
then 
$\sigma''=\sigma,\_$
and
$\Sigma^\sigma;\emptyset\vdash \ctx[\e_1]:\T$
\end{Assumption}
\noindent
For each \Q@try-catch@, execution preserves the memory needed to continue the execution in case of error
(the memory visible outside of the \Q@try@).%

%Thanks to how our reduction rules are designed, especially \textsc{try error},
%@Progress will need to rely on @StrongExceptionSafety internally.

Note that our last well formedness rule requires 
\textsc{update} and \textsc{mcall} to introduce
monitor expressions only over locations
that are not preserved by \Q@try@ blocks.
This can be achieved, since monitors are introduced
around mutation operations
(and \Q@new@ expression),
and Strong Exception Safety ensures no mutation happens on preserved memory.

% To the best of our knowledge, only the type system of 42~\cite{ServettoEtAl13a,ServettoZucca15}
%  supports all these assumptions out of the box,
% while both Gordon~\cite{GordonEtAl12} and Pony~\cite{clebsch2015deny,clebsch2017orca} supports all except StrongExceptionSafety,
% however it should be trivial to modify them to support it:
% the \Q@try-catch@ rule could be modified to
% $\emptyset;\Gamma\vdash\Kw{try}\ \oC\e_0\cC\ \Kw{catch}\ \oC\e_1\cC:\T$
% if\\* $\emptyset;
% \Gamma,\{x:\Kw{read}\,C | x:\Kw{mut}\,C\,\in\Gamma\}
% \vdash\e_0:\T$ and $\emptyset;\Gamma\vdash\e_1:\T$,
% i.e. $e_0$ can be typed when seeing all externally defined mutable references as \Q@read@.



\subsection{Proof of Theorem 1}
\label{s:proof}

It is hard to prove Soundness directly,
so we first define a stronger property,
called \emph{Stronger Soundness} and
show that it is preserved during reductions by means of conventional
Progress and Subject Reduction (Progress is one of our assumptions,
while Subject Reduction relies heavily upon Subject Reduction Base).
That is,
Progress $\wedge$ Subject Reduction $\Rightarrow$ Stronger Sound Validation,
\\*Stronger Soundness $\Rightarrow$ Soundness.

%The structure of the proof is interesting:
%It is hard to prove Sound Validation directly,
%so we first define a stronger property,
%called Stronger Sound Validation and
%we show that it is preserved during reduction by mean of conventional Progress and Subject Reduction.
%That is,
%Progress+Subject Reduction $\Rightarrow$ Stronger Sound Validation
%and Stronger Sound Validation $\Rightarrow$ Sound Validation.

%\saveSpace
\section{GUI Case study}
\label{s:case-study}
\saveSpace

%interface Foo{ma mb}
%
%class Raw implements Foo{
%  no validate 
%}
%class ValidFoo1 implemens Foo{
%  private capsule Foo inner;
%  ma(){
%    this.inner.ma();
%  }
%  validate
%}
%
%
%class Raw {
%  mut method mut A stuff(read A a) {
%      if a.bar() {
%         x = ...
%      }
%      return new A(x)
%  }
%
%  read method imm Object stuffPre(read A a) {
%     return a.bar()
%  }
%  read method mut A stuffPost(imm Object o) {
%     return new A(x)
%  }
%  mut method imm Object stuffInner(imm Object) {
%      x = ...
%  }
%}
%class ValidFoo1 {
%	cpasule Raw r;
%  mut method mut A stuff(read A a) {
%     imm Object p = this.r.stuffPre(a);
%     imm Object res = this.transformR(r -> r.stuffInner(p));
%      return this.r.stuffPost(res)
%  }
%}
%
%
%
%class ValidFoo2 implemens Foo{
%  private capsule Foo inner;
%  validate
%}
Here we show that we are able to verify classes with circular mutable object graphs, that interact with the real world using I/O.
%Here we discuss how to use conventional OO programming patterns to take advantage of our system. Using those
%patterns allows to circumvent some of the apparent limitations of our system.
% At first glance it would seem that by only being able to validate immutable and encapsulated state one could not create validated classes with complicated, mutable interconnected object graphs.
%We show that this is not the case by encoding
%Our invariant is that everything in every movable container (and the top level component) should 
%-be inside the container 
%-not overlap with anyhing else inside
%Our containers represent boxes that completley contain other non-overlapping boxes.
Our case study involves a GUI with containers (\Q@SafeMovable@s) and \Q@Button@s;
the \Q@SafeMovable@ class has an invariant to ensure that its children are completely contained within it and do not overlap. The \Q@Button@s move their \Q@SafeMovable@ when pressed. We have a \Q@Widget@ interface which provides methods to get \Q!Widget!s' size and position as well as children (a list of \Q@Widget@s). Both \Q@SafeMovable@s and \Q@Button@s implement \Q@Widget@. Crucially, since the children of \Q@SafeMovable@ is a list of \Q@Widget@s it can contain other \Q@SafeMovable@s, and all queries to their size and position are dynamically dispatched, such queries are also used in \Q@SafeMovable@'s invariant.
Here we show a simplified version\footnote{The full version, written in L42, which uses a different syntax, is available in our artifact at\\ \url{http://l42.is/EcoopArtifact.zip}}, where  \Q@SafeMovable@ has just one \Q@Button@, and certain sizes and positions are fixed. Note that \Q@Widgets@ is a class representing a mutable list of \Q@mut@ \Q@Widget@s.

\begin{lstlisting}[mathescape=false]
class SafeMovable implements Widget { capsule Box box;
  @Override read method Int left() { return this.box.l; }
  @Override read method Int top() { return this.box.t; }
  @Override read method Int width() { return 300; }
  @Override read method Int height() { return 300; }
  @Override read method read Widgets children() {
    return this.box.c; }
  @Override  mut method Void dispatch(Event e) {
    for (Widget w:this.box.c) { w.dispatch(e); }}
  read method Bool invariant() {..}
  SafeMovable(capsule Widgets cs) { this.box = makeBox(c); }
  static method capsule Box makeBox(capsule Widgets c) {
    mut Box b = new Box(5, 5, cs);
    b.c.add(new Button(0, 0, 10, 10, new MoveAction(b));
    return b; } //mut b is soundly promoted to capsule
}
class Box { Int l; Int t; mut Widgets c;
  Box(Int l, Int t, mut Widgets c) {..}
}
class MoveAction implements Action { mut Box outer;
  MoveAction(mut Box outer) { this.outer = outer; }
  mut method Void process(Event event) { this.outer.l += 1; }
}
..
//main expression; #$ is a capability method making a Gui object
Gui.#$().display(new SafeMovable(..));
\end{lstlisting}\saveSpace
As you can see, \Q@Box@es encapsulate the state of the \Q@SafeMovable@s that can change over time:
\Q@left@, \Q@top@, and \Q@children@. Also note how the ROG of \Q@Box@ is circular: since
the \Q@MoveAction@s inside \Q@Button@s need a reference to the containing \Q@Box@ in order to move it.
Even though the children of \Q@SafeMovable@s are fully encapsulated, we can still easily dispatch events to them using \Q@dispatch@. Once a \Q@Button@ receives an \Q@Event@ with a matching ID, it will call its \Q@Action@'s \Q@process@ method. 

%Our main function uses a capability-object to display the top-level \Q@Widget@ and its \Q@children@, as well as dispatch events to it. 
Our example shows that the restrictions of TMs and OCs are flexible enough to encode interactive GUI programs, where widgets may circularly reference other widgets.
In order to perform this case study we had to first implement a simple GUI Library in L42. This library uses object capabilities to draw the widgets on screen, as well as fetch and dispatch the events. Importantly, neither our application, nor the underlying GUI library require back doors into either our type modifier or capability system to function, demonstrating the practical usability of our restrictions.


\subheading{The Invariant}
\Q@SafeMovable@ is the only class in our GUI that has an invariant, our system automatically checks it in two places: the end of its constructor and the end of its \Q@dispatch@ method (is a capsule mutator). There are no other checks inserted since we never do a field update on a \Q@SafeMovable@. The code for the invariant is just a couple of simple nested loops:
\begin{lstlisting}
read method Bool invariant() {
  for(Widget w1 : this.box.c) {
    if(!this.inside(w1)) { return false; }
    for(Widget w2 : this.box.c) {
      if(w1!=w2 && SafeMovable.overlap(w1, w2)){return false;}}}
  return true;}
\end{lstlisting}
Here \Q@SafeMovable.overlap@ is a static method that simply checks that the bounds of the widgets don't overlap. The call to \Q@this.inside(w1)@ similarly checks that the widget is not outside the bounds of \Q@this@; this instance method call is allowed as \Q@inside@ only uses \Q@this@ to access its fields.%its \Q@width@ and \Q@height@ fields.


% This code is a simplified version of our first case study. In the full code the \Q@SafeMovable@ constructors take \Q@left@, \Q@top@, \Q@width@ and \Q@height@ parameters and can either take a \Q@box@ directly or will generate one with $4$ \Q@Button@s}
% we we have $4$ buttons, each button moves in one of the $4$ cardinal directions.
\newlength{\imglength}%
\setlength{\imglength}{0.4\textwidth}%
\columnsep=0.5em%
\begin{wrapfigure}{l}{\imglength}%
	\setbox0=\hbox{\strut}%
	\vspace{-1.3\ht0}%
    \includegraphics[width=\imglength]{GuiImg}%
	\vspace{-1.5\ht0}%
\end{wrapfigure}
\subheading{Our Experiment}
As shown in the figure to the left, counting both \Q@SafeMovable@s and \Q@Button@s, our main method creates $21$ widgets: a top level (green) \Q@SafeMovable@ without buttons, containing $4$ (red, blue, and black) \Q@SafeMovable@s with
$4$ (gray) buttons each. When a button is pressed it moves the containing \Q@SafeMovable@ a small amount in the corresponding direction.
%Each container has 4 gray buttons one for each cardinal direction.
% In our set up
% the top level \Q@SafeMovable@
% contains a big red \Q@SafeMovable@
%  containing $2$ smaller blue \Q@SafeMovable@. One of those contains a tiny black \Q@SafeMovable@.
%Our invariant is that everything in every movable container (and the top level component) should 
%-be inside the container 
%-not overlap with anyhing else inside
%Our containers represent boxes that completley contain other non-overlapping boxes.
This set up is not overly complicated, the maximum nesting level of \Q@Widget@s is $5$.
Our main method automatically presses each of the $16$ buttons once. In L42, using the approach of this paper, this resulted in $77$ calls to \Q@SafeMovable@'s invariant.

\subheading{Comparison With Visible State Semantics}
As an experiment, we set our implementation to generate invariant checks following the  visible state semantics approaches of D and Eiffel~\cite{Alexandrescu:2010:DPL:1875434,DRef},
where the invariant of the receiver is instead checked at the start and end of \emph{every} 
public (in D) and qualified\footnote{That is, the receiver is not \Q!this!.} (in Eiffel) method calls.
In our \Q!SafeMovable! class, all methods are public, and all calls are qualified, thus this difference is irrelevant. Neither protocol performs invariant checks on field accesses or updates,
however due to the `uniform access principle'%\footnote{%
%L42 also follows the Eiffel uniform access principle: field accesses are the same as method calls.%
%}
, Eiffel allows fields to directly implement methods, allowing the \Q!width! and \Q!height! \emph{fields} to directly implement \Q!Widget!s \Q!width! and \Q!height! \emph{methods}. On the other hand in D, one would have to write getter \emph{methods}, which would invoke invariant checks.
%though 42 represents field accesses as method calls, for a fair comparison with conventional OO approach, we do not treat field accesses on \Q@SafeMovable@s within the \Q@SafeMovable@ class itself as public method calls
% D and Eiffel have slightly different interpretation of
% visible state semantic when qualified/unqualified method calls or field accesses are performed.
% However, in our GUI example those corner cases are
% not relevant. To be sure, we implemented both of their strategies and obtained the same results.
When we ran our test case following the D approach, the \Q!invariant! method was called $52,734,053$ times, whereas the Eiffel approach `only' called it $14,816,207$ times;%\footnote{%
%We expect all (sound) runtime approaches based on visible state semantics ~\cite{feldman2006jose,fahndrich2010embedded,abercrombie2002jcontractor,tran2003design} to produce similar results.}
in comparison our invariant protocol only performed $77$ calls. The number of checks is exponential in the depth of the GUI: the invariant of a \Q@SafeMovable@ will call the \Q@width@, \Q@height@, \Q@left@, and \Q@top@ methods of its children, which may themselves be \Q@SafeMovable@s, and hence such calls may invoke further invariant checks. Note that \Q!width! and \Q!height! are simply getters for fields, whereas the other two are non trivial \emph{methods}. % Since we only use \Q@this@ to perform method calls, the invariant is not recursively checked on \Q@this@, if it were we would get an infinite recursion.
% Note that 42 represents field accesses as method calls, this is similar to the Eiffel uniform access principle; but for a fair comparison with D we treat private accesses to them like field accesses.


% It can be surprising to see such extreme difference. We ran our example with less widgets, and our results suggest an exponential growth in the cost of the conventional approach. For example by removing 2 containers (and their 8 buttons) we get....
% We now explain how this exponential explosion happens:
% For the outer-most box to check its invariant, it needs to call methods \Q@left,top,height@ and \Q@with@ to all its contained widgets.
% If one of those widget has \Q@invariant@, when such public method is called,
% its \Q@invariant@ is checked (twice). This requires to call \Q@left,top,height@ and \Q@with@ to all its contained widgets, some of those may also have \Q@invariant@s.

% In literature there is attention to prevent method called from an invariant to call public methods on \Q@this@; it would cause the system to go in loop.
% However when calling methods on other objects is allowed, if those objects have invariant, this cause the aforementioned explosion. 

% Our example also shows that the restrictions of TM's and OC's are flexible enough to encode interactive GUI programs, where widgets may circularly reference other widgets.
% In order to perform this case study, we had to first implemented a simple Gui Library in L42. This library uses object capabilities to draw the widgets on screen, as well as fetch and dispatch the events.
% The gui library abstract away all the details of drawing and events; the user code need only to provide concrete classes implementing the \Q@Widget@ interface.

\subheading{Spec\# Comparison}
We also encoded our example in Spec\#\footnote{We compiled Spec\# using the latest available source (from 19/9/2014). The verifier available online at \url{rise4fun.com/SpecSharp} behalves differently.}, which like L42, statically verifies aliasing/ownership properties, as well as the admissibility of invariants.
% To keep the Spec\# code aligned with the L42 one, 
% we do not use a .NET GUI library, but we just simulate the behaviour of L42, without actually opening a window.
The backend of the L42 GUI library is written in Java, we did not port it to Spec\#, rather we just simulate the backend, and don't actually display a GUI in Spec\#.

We ran our code through the Spec\# verifier (powered by Boogie~\cite{DBLP:conf/fmco/BarnettCDJL05}), which only gave us $2$ warnings\footnote{We used \Q@assume@ statements, equivalent to Java's \Q@assert@, to dynamically check array bounds. %. and value presence.
This aligns the code with L42, which also performs such checks at runtime.}: that the invariant of \Q@SafeMovable@ was not known to hold at the end of its constructor and \Q@dispatch@ method. Like our system however, Spec\# checks the invariant
at those two points at runtime. Thus the code is equivalently verified in both Spec\# and L42; in particular it performed exactly the same number ($77$) of runtime invariant checks.\footnote{%
We also encoded our GUI in Microsoft Code Contracts~\cite{DBLP:conf/sac/FahndrichBL10}, whose unsound heuristic also calls the invariant $77$ times; however Code Contract does not enforce the
encapsulation of \Q@children@, thus their approach would not be sound in our context.}

% Assuming that Spec\#'s verifier is sound, this means that our code is equally verified in Spec\# and L42, providing a reasonable comparison.
% Both Spec\# and L42 did the same thing:
% they statically verify ownership/aliasing annotations,
% they check the admissibility/valididty of a the invariant code and finally 
% they perform sufficient runtime-checking of the invariant.
% We used the Boogie static checker to verify all the aliasing/ownership properties needed to
% ensure that the $77$ run-time invariant checks soundly enforce that the invariant holds when is expected.
% This of course includes preventing the Gui to ever display two overlapping Widget. 

%\end{itemize}
%The result is the same of L42: the invariant is checked $77$ times, and in exactly the same locations of L42.

We found it quite difficult to encode the GUI in Spec\#, due to its unintuitive and rigid ownership discipline. In particular we needed to use many more annotations, which were larger and had greater variety. In the following table we summarise the annotation burden,
for the \emph{program} that defines and displays the \Q@SafeMovable@s and our GUI; as well as the \emph{library} which defines \Q@Button@s, \Q@Widget@, and event handling.\footnote{We only count constructs Spec\# adds over C\# as annotations, we also do not count annotations related to array bounds or null checks.}:
\begin{center}\saveSpace\saveSpace
\begin{tabular}{ c  c  c  c  c}
 & Spec\# & Spec\# & L42 & L42 \\ 
 & \!\!program\!\! & library & \!\!program\!\! & library \\
\hline
 
\!\!\!Total number of annotations 
 	& $40$ & $19$ & $19$ & $18$ \\ \hline
% Totals 	 $59$ $37
\!\!\!Tokens (except \Q@.,;(){}[]@ and whitespace)\!\!\!
%(ecluding \Q@;,@ characters, white-space, or parentheses/brackets.) 
	& $106$ & $34$ & $18$ & $18$  \\  \hline
% $140$  & $37$
Characters (with minimal whitespace) 
	& $619$ & $207$ & $74$ & $60$ \\ \hline
%  $826$ $134$
\end{tabular}
\end{center}

To encode the GUI example in L42, the only annotations we needed were the 3 type modifiers: \Q@mut@, \Q@read@, and \Q@capsule@.
% , for a total of 
% $19$ annotations (one token each, $74$ characters in total).
Our Spec\# code requires things such as, purity, immutability, ownership, method pre/post conditions and method modification annotations. In addition, it requires the use of 4 different ownership functions including explicit ownership assignments. In total we used 18 different kinds of annotations in Spec\#.
Together these annotations can get quite long, such as the following precondition on \Q@SafeMovable@'s constructor: \\
\begin{tabular}{r}
\Q@requires Owner.Same(Owner.ElementProxy(children), children);@
\end{tabular}

% methods and field attributes as well as requires, ensures and modifies clauses, and finally also %explicit ownership assignment statements.
% Spec\# annotation can be involved, as for example \Q@requires Owner.Same(Owner.ElementProxy(children), children);@}

% To estimate the annotation burden we count the number of tokens (excluding \Q@.;,@ and parenthesis).
% This gave us $113$ tokens, that is more then $5$ times the amount needed in L42.
% The total annotation character count is $830$; $10$ times more then L42.


% 40   106+17=113 619+211
%main 11 annotations, 28 + 11 tokens, 118+65 characters
%safemovable 29 annotations, 78+6 tokens, 501+146 characters

%19   34   207+44
% auxLib // 14 annotations, 25 tokens, 155+44 characters
%guiLib// 5 annoations, 9 tokens, 52 characters

% Moreover, in L42 we only use $3$ different kinds of annotations, while in Spec\# we use $15$ kinds of annotations.
\noindent The Spec\# code also required us to deviate from the style of code we showed in our simplified version: we could not write a usable \Q@children@ method in \Q@Widget@ that returns a list of children, instead we had to write \Q@children_count()@ and \Q@children(int i)@ methods; we also needed to create a trivial class with a \Q@[Pure]@ constructor (since \Q@Object@'s one is not marked as such). In contrast, the only strange thing we had to in L42 was creating \Q@Box@es by using 
an additional variable in a nested scope.
This is needed to delineate scopes for promotions.
Based on these results, we believe our system is significantly simpler and easier to use.
% On the basis of these results we believe  that our system is easier to use for programmers that are not experts in software verification.

\subheading{The Box Pattern}
Our design, using an inner \Q@Box@ object, is a common pattern in static verification: where one encapsulates all relevant mutating state into an encapsulated sub object which is not exposed to users.

% We also found natural to use the box pattern also in Spec\# requires.
Both our L42 and Spec\# code required us to use the box pattern for our \Q@SafeMovable@, due to the circular object graph caused by the \Q@Action@s of \Q@Button@s needing to change their enclosing \Q@SafeMovable@'s position.
% Also the implementation of the minimal GUI library (that our \Q@SaveMovable@ builds upon) has a much lower annotation burden.
%\LINE

\subheading{The Transform Pattern}
% A capsule mutator method is essentially a mutation of a field, which is guaranteed to not see the \Q@this@ object.
% Thus, if \Q@this@ is made invalid during  the method's execution, we could not observe it until after the method completes.
Suppose we want to scale a \Q@Widget@, we could add \Q@mut@ setters for \Q@width@, \Q@height@, \Q@left@, and \Q@top@ in the \Q@Widget@ interface. However, if we also wish to scale its children we have a problem, since \Q@Widget.children@ returns a \Q@read Widgets@, which does not allow mutation. We could of course add a \Q@mut@ method \Q@zoom@ to the \Q@Widget@ interface, however this does not scale if more operations are desired. If instead \Q@Widget.children@ returned a \Q@mut Widgets@, it would be difficult for \Q@Widget@ implementations, such as \Q@SafeMovable@, to keep control of their ROGs.

% In the above \Q@SafeMovable@ we only had one capsule mutator: \Q@dispatch@. However suppose a \Q@Widget@ wants to directly mutate it's descendents, however it can't do that since \Q@Widget.children@ returns a \Q@read Widgets@, if it returned a \Q@mut Widgets@ then \Q@SafeMovable@ could not be implement, as it's children are contained inside a capsule-field. 
% At first glance, it may seem that capsule mutators allow only very limited kinds %of mutation.
% This is however not the case. 

% Consider the following
% simple pattern to allow flexible use of encapsulated content: define a

A simple and practical solution would be to define a \Q@transform@ method in \Q@Widget@, and a \Q@Transformer@ interface 
like so:\footnote{A more general transformer could return a generic \Q@read R@.}
\saveSpace
\begin{lstlisting}
interface Transformer<T> { method Void apply(mut T elem); }
interface Widget { ..
  mut method Void top(Int that); // setters for immutable data
  mut method read Void transform(Transformer<Widgets> t);
} // transformers for possibly encapsulated data
class SafeMovable { ..
  mut method Void transform(Transformer<Widgets> t) {
    return t.apply(this.box.c); }} // Well typed capsule mutator
\end{lstlisting}\saveSpace
% Note that the code above does not access a capsule field but merely calls a method that does; thus  it is \emph{not} a capsule mutator method, so it is not constrained by the restrictions on them. Code like the above would also allow one to mutate multiple capsule fields in one method.
%Our pattern cooperates with the language’s restrictions to ensure each mutation is completed as a separate operation, that is perceived by the rest of the system %as if it was atomic.%
%,  i.e. they can't see or update the other capsule fields.
The \Q@transform@ method offers an expressive power similar to \Q@mut@ getters, but prevents \Q@Widgets@ from leaking out.  With a \Q@Transformer@, a \Q@zoom@ function could be simply:
\saveSpace\begin{lstlisting}
static method Void zoom(mut Widget w) {
  w.transform(ws -> { for (wi : ws) { zoom(wi,scale); }});
  w.width(w.width() / 2); ..; w.top(w.top() / 2); }
\end{lstlisting}\saveSpace
\saveSpace
\saveSpace
% One of the advantages of this approach is that a the \@zoom@ method can be written by anyone anywhere

% \begin{lstlisting}[escapechar=\%]
%// Lambda Expression that creates a new Transformer<...>
%this.transform(l -> l.add(new MyWidget(..)))
%\end{lstlisting}
%//`i' is captured by the closure.
%// `imm' and `capsule' varaibles can be captured.

%    %\Comment{}%this.items.add(i);
%    // Cant instead capture `this': it can't be typed %as `imm'
%    // since `ItemTransformer.transform()' is an %`imm' method
%  })
%}
%  // instead of:
  %\Comment{}%this.exposeItems().add(i)

%Note that the code above does not access a capsule field but merely calls a method that does; thus
%it is \emph{not} a capsule mutator method, so it is not constrained by the restrictions on them. Code like the above would also allow one to mutate multiple capsule fields in one method.
%Our pattern cooperates with the language’s restrictions to ensure each mutation is completed as a separate operation, that is perceived by the rest of the system
%as if it was atomic.%
%,  i.e. they can't see or update the other capsule fields. %s:case-study

\saveSpace
\section{Related work}
\label{s:related}
\saveSpace

\textit{Type Modifiers:}
We rely on a combination of modifiers that are supported by at least 3 languages/lines of research:
L42~\cite{ServettoZucca15,ServettoEtAl13a,JOT:issue_2011_01/article1,GianniniEtAl16},
Pony~\cite{clebsch2015deny,clebsch2017orca}, and Gordon's~\cite{GordonEtAl12}; 
each of these works is accompanied by proofs about the properties of those modifiers.
Since such proofs have already been done, in this work we just assume the required properties.
Those approaches all support deep/strong interpretation, without back-doors.

TM approaches like Javari~\cite{TschantzErnst05,Boyland06} and Rust~\cite{matsakis2014rust} are unsuitable since they introduce back-doors which are not easily verifiable as being used properly.
Many approaches just try to preserve purity (as for example~\cite{pearce2011jpure}), but here we also need aliasing control.
Ownership~\cite{ClarkeEtAl13,ZibinEtAl10,DietlEtAl07} is another popular form of aliasing control that can be used as a building block for static verification~\cite{%
muller2002modular,%
barnett2011specification%
}.%On ownership verification
%Peter Mueller and Arnd Peotzsch Heffter,  eg Müller, P.: Modular Specification and Verification of Object-Oriented Programs, 2002.
%M. Barnett and M. Fähndrich and K. R. M. Leino and P. Müller and W. Schulte and H. Venter: Specification and Verification: The Spec# Experience. Communications of the ACM, 2011.
\MSComm{add discussion on the work reviewrs pointed out}

%\noindent\textit{Strong Exception Safety:}
%Exception safety seems at first glance a smaller issue with respect 
%to the other two, but is the final piece that lets the whole system work in a real world setting.
%Note that state of the art type systems to enforce exception safety
% do not restrict code that do not capture errors, and
%only the point of error capturing is constrained.

\textit{Object Capabilities:}
Object capabilities~\cite{RobustComposition}, in conjunction with type modifiers, are able to
 enforce purity of code in a modular way, without requiring following a monadic style.
The Joe-E language~\cite{finifter2008verifiable} explores how to use object capabilities to ensure pure behaviour for methods.
However, in order to express Joe-E as a subset of Java, they leverage on a simplified model of immutability:
immutable classes must be final with only final fields that refer to immutable classes.
%Instances of immutable classes are immutable objects.
In Joe-E every method that only takes instances of immutable classes is pure.
%\IOComm{Worth mentioning Wyvern? Since alex mentioned that it enforces purirty? Perhaps he can write that section?}
Their model however would not allow the verification of purity for invariant methods of mutable objects.
In contrast our model has a more fine grained representation of immutable/readonly: it is \emph{reference based} instead of \emph{class based}. \IOComm {Redundent?} This means that in our model, every method taking only \Q@read@ or \Q@imm@ references as input is pure,
both in the sense that no object visible outside of the method is mutated, but also that no I/O is performed.

\textit{Class invariants protocols:}
Class invariants are a fundamental part of the design by contract methodology. 
Invariant protocols differ wildeley and can be unsound or complicated, particular due to re-entrancy and aliasing
\cite{leino2004object,drossopoulou2008unified}\IOComm{Cite bertrand meyer? He refers to these as 'referemce leaking' and 'furtive accesses'}.

%literature on class invariant accepts that sometime the object invariant may not hold,
%and that is exacerbated because of 
%Leino, K. R. M. and Müller, P.: Object Invariants in Dynamic Contexts (ECOOP), 2004.
%S. Drossopoulou and A. Francalanza and  P. Müller and A. J. Summers: A Unified Framework for Verification Techniques for Object Invariants ECOOP 2008. 
There are different options as to what object-invariants are known to hold:
\begin{itemize}
\item  when the object is in a \textit{steady} state:
 the execution is not inside any of its (non `helper'~\cite{JML}) methods~\cite{Gopinathan:2008:RMO:1483018.1483028};
 constantly maintained between calls to public methods~\cite{WikiInvariant},
\item at the start and at the end of every public method (which may or may not include recursive method calls)~\cite{Burdy2005};\MSComm{add more citations},
\item at the start and end of every \emph{qualified} call~\cite{?},
\item only when explicitly required (such as in a method contract or implied by another object's invariant) and after an explicit check (`pack') operation~\cite{?},
%\url{https://en.wikipedia.org/wiki/Class_invariant}};
%\item
%constantly maintained when the object is \textit{closed};
%invariant can be manually opened and closed by using special operations; % Add cite here!
\item or, as in this work, when an object could be \emph{involved} in execution.
\end{itemize}
Those different protocols are deceivingly similar, and 
some approaches like JML suggest verifying a simpler approach (that method calls preserve the invariant of the \emph{receiver}) but assume a more powerful one (the invariant of \emph{every} object except the receiver holds).

% use the unsound option of assuming one protocol, but actually checking another.

%DONE IN INTRO breaking class invariants = bug in class code
%braking validation= DEPEND.

%To encode this range of invariant semantics
%in our approach we can add a boolean \Q@isOpen@ field and add \Q@this.isOpen || ..@
%in front of the validity condition.
%Validation can be used to manually encode complex scenarios,
%for example if a method called on an object needs to break the invariant of another object,
%it can do so by manually setting the \Q@isOpen@ flag on the other object.


%On ownership verification
%Peter Mueller and Arnd Peotzsch Heffter,  eg Müller, P.: Modular Specification and Verification of Object-Oriented Programs, 2002.
%M. Barnett and M. Fähndrich and K. R. M. Leino and P. Müller and W. Schulte and H. Venter: Specification and Verification: The Spec# Experience. Communications of the ACM, 2011.

\newcommand\sepItems{\saveSpace\saveSpace\saveSpace\\*${}_{}$\\*${}_{}\,\bullet\,$}

\LINE

\textit{Runtime verification tools:}
Many languages and tools support some form of runtime invariant checking (e.g. Eiffel~\cite{Meyer:1992:EL:129093}, D~\cite{Alexandrescu:2010:DPL:1875434}
and JML~\cite{Burdy2005}).
By looking to a survey by Voigt et al.~\cite{Voigt2013} and the extensive MOP project~\cite{meredith2012overview},
it seems that most runtime verification tools (RV) empower users
to implement what kind of monitoring they see fit for their specific problem at hand. This means that users are responsible for deciding, designing, and encoding both the logical properties and the instrumentation criteria\cite{meredith2012overview}.
In the context of object-invariants, this means the user defines the invariant protocol, and the soundness of such protocol is not checked by the tool.

In practice, this means that the logic, instrumentation, and implementation end up connected:
a specific instrumentation strategy is only good to test certain logic properties in certain applications.
No guarantee is given that the implemented instrumentation strategy is able to support the required logic in the monitored application.
Some of those tools are designed to support class invariants: for example InvTS~\cite{gorbovitski08efficient} lets you write Python conditions that are verified on a set of Python objects, but the programmer needs to be able
to predict which objects are in need of being checked and to use a simpler domain specific language to target them. Hence if a programmer makes a mistake while using this domain specific language, invariant checking
will not be triggered.
Some tools are intentionally unsound and just perform invariant checks following some heuristic that is expected to catch most of failures: jmlrac~\cite{Burdy2005} and Microsoft Code Contracts~\cite{fahndrich2010embedded}.

\IOComm{Put this in Gui section a footnote: whose heuristic also run the invariant checking $77$ times on our GUI case study}

%In particular, the heuristic of 
%We encoded our GUI example also on Microsoft Code Contract; their system also ran the invariant checking $77$ times. Their system is easy to use, but it is unsound since it is built over an unsound/incomplete static verifier~\cite{?}.






%
%In this work we define a language where a minimal, standardized,
%efficient and completely general purpose instrumentation strategy can soundly verify conditions
%expressible as a\\* \Q@read method imm Bool invariant()@, for any well-typed program; with open world assumption
%and possible Byzantine behaviour of any object in the system.
%
%By seeing class invariant as a part of the type of the object,
%the `RV tool' philosophy is akin to letting the programmer customize the behaviour of the
%type system: the programmer implementation may be unsound; while our philosophy is
%to give the user a way to represent complex and expressive types (in the form of arbitrary code in 
%the \Q@invariant()@ method), but 
%the type system implementation is fixed in stone by the language designer.

Many works attempt to move out of the `RV tool' philosophy to ensure RV monitors work as expected, as for example%
%\sepItems
%In avionics, where memory allocation is disallowed, making reasoning about aliasing much simpler~\cite{laurent2015assuring}:
%``\emph{Runtime Verification (RV) can act as the last line of defense to
%protect the public safety, but only if the RV system itself is trusted.}''.
%\sepItems
%In domain specific languages~\cite{ferrari2002guardians}:
%``\emph{Proof techniques for establishing security properties}''.
%\sepItems
%On assertions over restrictive domain specific languages, to tame some of the C/C++
%undefined behaviour~\cite{agten2015sound}:
%``\emph{no verified assertion in the verified
%module will ever fail at runtime, even if the module runs as part of
%a vulnerable application thSound and Unsound monitorsat is subject to code injection attacks}''.
the study of contracts as refinements of types~\cite{findler2001contract}, focusing only on pre and post conditions in OO languages.

Gopinathan \&al.'s.~\cite{Gopinathan:2008:RMO:1483018.1483028} approach keeps invariants under tight control:
relying on powerful aspect-oriented support, they detect any field update in the whole ROG of any object, and check all the invariants that such update may have violated.
They argue against any variation of visible state semantic, where  methods still have to assume that any object may be broken; in such case calling any public method would trigger an error, but while the object is just passed around (and for example stored in collections), the broken state will not be detected.
``\emph{there are many instances where $o$'s invariant is violated by the programmer inadvertently changing the state of $p$ when $o$ is in a steady state. Typically, $o$ and $p$ are objects exposed by the API, and the programmer (who is the user of the API), unaware of the dependency between $o$ and $p$, calls a method of $p$ in such a way that $o$'s invariant is violated. The fact that the violation occurred is detected much later, when a method of $o$ is called again, and it is difficult to determine exactly where such violations occur.}''

However, their approach addresses neither exceptions nor non-determinism caused by I/O, so their work is unsound when those aspects are took in consideration.

Their approach is very computationally intensive, but we think it is powerful enough that it could even be used to roll-back the very field update that caused the invariant to fail, making the object valid again.
We considered a roll-back approach, however rolling back a single-field update is likley to be completley unexpected, rather we should roll back more meaningful operations, similarly to what happens
with transational memory. As with transactional-memory, this is likely to be very hard to support efficiently.
%However we think roll-back this would be a 
%\REVComm{\REVComm{terrible}{2}{It seems in poor taste to complain of ``terrible'' ideas, especially without attempting to demonstrate the improvements of the proposed approach.}}{3}{Nontechnical term. It is not a great idea to label previous work as ``terrible''}
% ideally not only the field-update breaking the invariant should be reverted, %the roll-back should 
Using TMs to enforce strong exception safety is a much simpler alternative, providing the same level of safety, albeit being more restrictive (namely that if the operation did succeed it is still effectively rolled-back).

%: for example
%assume that we are moving object between two boats:
%the overflowing object may be removed from the \Q@cargo@ of the second boat, but it would not
%be placed back in the first boat. It would look like the object has disappeared.
%The important pTheir approach is very computationally intensive, but we think it is powerful enough that it could even be used to roll-back the very field update that caused oint here is that the program would be in an unexpected state
%even if no object invariants are violated, and this would happen \textbf{because} of the 
%invariant checking/fixing behaviour, not because of code written by the programmer.
%We believe that the only viable option is to detect violations after the fact.
\LINE
\IOComm{ADD THOSE TO PERFORMANCE EIFFEL,D\cite{feldman2006jose,fahndrich2010embedded,abercrombie2002jcontractor,tran2003design}}

%\noindent\textit{Performance}
%Our case study shows that our sound approach can monitor programs
%for a fraction of the cost of many other approaches.
%Many other works%
%~\cite{feldman2006jose,fahndrich2010embedded,abercrombie2002jcontractor,tran2003design}
% check/run
%the invariant code at the start and end of every public
%method; this even include trivial getters.
%In  our approach, we call the \validate{} method
%one time at the end of each setter, capsule mutator method and constructor.
%We do not inject it at the end of other methods, which are usually more numerous and invoked much more often.
%Of course, \validate{} can still be called indirectly, for example by calling a setter.
%We expect our approach to result in a dramatic reduction over the number of required checks,
%except for cases when public methods just update many fields directly (without using setters).


\LINE


\noindent\textit{Security and DMZ:}
Static verification lets us reason about a complete program
and verify its correctness.
Traditional static verification is like a mathematical proof: a program is valid if it is all correct,
but a single error invalidates all the claims.
Thus, it is hard to perform verification on large programs, or when independently
maintained third party libraries are involved.
%\REVComm{
%To solve this issue, static verification systems are %starting to
%}{2}{[is this correct?] verification of reference %monitors, gradual typing, and contracts have been %explored for longer}
To soundly verify code embedded in an untrusted 
environment, it is possible to 
consider a verified core
and a run-time verified boundary.
You can see our approach as an extremely modularized version of such system:
every class is its own demilitarized zone, and the rest of the code 
could have Byzantine behaviour.
Our formal proofs 
shown that every class that compiles/type checks is soundly validated
independently of the code that uses this class or any other surrounding code.
Our approach works both with an open world assumption and in a library setting.
Consider for example the work of Parkinson~\cite{parkinson2007class}:
in his short paper he verified an \Q@Observer@ class invariant over
a \Q@Subject/Observer@ pattern.
However, the proof relies on the method \Q@Subject.register(Observer)@ respecting its contract.
Such assumption is unrealistic in a real system with dynamic class loading,
and this invariant could trivially be broken by a user defined \Q@EvilSubject@.


\noindent\textit{Dedicated specification language or underling language:}
Using a specification languages near to the logic and disjointed from a specific language's
semantics may seem attractive, however
a study~\cite{chalin2007logical} discovered that developers expect
the specification language to use the semantics of the underling language, including
short circuit semantics and arithmetic exceptions; thus for example
\Q@1/0 || 2>1@
should not hold, while 
\Q@2>1 || 1/0@ should hold thanks to short circuit semantics.
This study was influential enough to convince JML to change its interpretation of logical expressions
accordingly~\cite{chalin2008jml}.
We believe this is evidence that using a method in the underlying language to encode the validation is
a developers-friendly solution.



${}_{}$\sepItems
Works over C\# recognize the need
for purity/determinism when method calls are allowed in contracts~\cite{barnett200499}
``\emph{There are three main current approaches: a) forbid the use of functions in specifications, b) allow only provably pure functions, or c) allow programmers free use
	of functions. The first approach is not scalable, the second overly restrictive and
	the third unsound.}''\\*
They recognize that many tools unsoundly use option (c), such as AsmL~\cite{barnett2003runtime}.
They propose a concept of observational purity, that if completely fleshed out
could possibly be a great addition to our proposed type system.
We speculate that some 
primitive language support may be needed, for example implementing the Flyweight pattern 
as part of the language semantics.


%1 aliasing control
%  example hamster can be broken with those 2 lines
%2 I/O /determinism
%  hamster EvilPoint with random equal is accepted
%3 exceptions
%  spec sharp is happy to be unsound with capturing %unchecked exceptions
%---
%*TMs, OCs
%
%*expand on invariant protocol
%
%*RV tool
%------------
%*spec# unsond, parkinson critique, and static %verification is like math proof
%
%*Soundness or not
%
%*C#purity, dedicate spec language



%\noindent\textit{Theorem provers and SAT solvers}
%Rather than providing a simple set of rules as to what a \Q@validate@ method can contain,
%and where to insert calls to it, we could instead rely on implementation-specific static analysis:
% in which a \Q@validate@ method is valid iff the compiler can prove that it is deterministic
% and that it’s generated \Q@validate()@ calls are sufficient to enforce validation.
%Though approaches like this are frequently used such as with unifying Java’s generic-wildcards [], Rust’s ‘borrow checker’, …; we believe that would not produce a good result for our purposes: 
%\begin{itemize}
%	\item it would mean that a programmer would have no way of telling whether their code would compile, in particular code compiling would depend on the specific compiler (version) used.
%	\item the runtime cost of validation would be completely unpridictibable; since it is deterministic there is nothing stopping the compiler from calling \Q@validate@ any number of times, and at any point in time.
%	\item When a validation error could be throw would likewise be unpredictable, though it should happen after an object is made invalid\footnote{technically our definition of validation technically allows the error to happen sooner, as long as it’s not too late; however pre-emptive errors like this would be extremely hard to debug}, it could happen any time before it’s use. Making matters worse, if multiple object’s would be invalidated before either is used, which one’s error would be thrown is unconstrained
%	\item This approach will not work well in the pressence of dynamic code loading, in particular it woud likley significantly slow down such loading or spurioslly fail depending on what other code has been loaded
%\end{itemize}



%Conclusions? future work?
%@StrongExceptionSafety is 
%a very strong property,
%and some languages may be unwilling to commit to always preserve it.
%In particular, depending on the details of a specific language
% releasing resources as in \Q@finally@ blocks may require
%some relaxation of @StrongExceptionSafety. Sound releasing of resources could be interesting
%future work.

Related work

Class invariants are a fundamental part of the design by contract methodology. 
Many languages and tools support some form of invariant verification (e.g. Eiffel~\cite{Meyer:1992:EL:129093}, D~\cite{Alexandrescu:2010:DPL:1875434}, JML~\cite{Burdy2005}, Spec\#~\cite{Barnett:2004:SPS:2131546.2131549}).
%In order to be verified, the invariant needs to be expressed in some formal way.
Here we focus on multi-object invariants: the class invariant of a given object may depend upon the observable behaviour of any object referenced in its Reachable Object Graph (ROG).

\noindent\textit{Security and DMZ:}
Static verification let us reason about a complete program
and verify its correctness.
Traditional static verification is like a mathematical proof: is valid if it is \textbf{all correct},
but a single error invalidates all the claims.
Thus, it is hard to perform verification on large programs, or when independently maintained third party libraries
are involved.
To solve this issue, static verification systems are starting to consider a verified core
and a run-time verified boundary.

You can see our approach as an extremely modularized version of such system:
every class is its own demilitarized zone, and the rest of the code 
could have Byzantine behaviour.

Every class that compiles/type checks should be protected against breakage,
 independently of the code that uses this class or any other surrounding code.
 That is, our approach works both with open world assumption and in a library setting.


\saveSpace
\section{Conclusion}
\label{s:conclusion}
\saveSpace

Static verification requires great effort, but can ensures all invariants \textbf{always} holds, thus all objects are always coherent.

However, Static verification is very heavy weight, and often impractical.
In the context of a conventional OO language with imperative features,
we propose an \textbf{ultra-lightweight} verification approach,
where the programmer specifies \textbf{only} the desired class invariants as an 
\Q@invariant()@ method written in the language itself.
This is much more convenient with respect to requiring the specification of methods pre and post conditions,
since the number of classes is usually order of magnitude smaller then the number of methods,
and a fully annotated program requires to write down 
pre-post conditions for each methods, encoding a generalization of its behaviour
in the dedicated specification language.
This means that, even in the best case scenario, 
using pre-post conditions
the user is required to specify the program semantic twice:
first in the specification language and then in the underlying programming language.


With just invariants, our system will then 
\textbf{soundly ensure invariants of all objects involved in the execution}.
Our approach do not rely on assumption over the behaviour of methods/classes;
except for the language semantics and the type system guarantees.
Methods are just treated as black-boxes, producing a result or throwing an error.

Of course, there is a catch: this result is obtained by modifying/instrumenting the
semantic of the language, so that (as for type casts) \textbf{violations are detected at run-time}, and exceptions
are throw in order to stop the execution before involving any broken object.

\noindent\textit{SIC as extended type system:}
The philosophy of our approach is to be like an extended type system: 
\begin{itemize}
\item The programmer decides to annotate a field with a certain type, or the class with a certain invariant.
\item If that is a valid type, or a valid invariant, the user is not questioned in its intent.
\item The system enforce that field will only contain values or that type, or that instance of that class
will respect that invariant.
\end{itemize}
This is in sharp contrast with most work in RV, that is often conceived more as a tool to ease debugging:
both deciding the invariant and enforcing it is controlled by the programmers.

\bibliography{main} % The OOPSLA template has this at the end...

\appendix
\section{Invariant Protocol Proof and Type System Requirements}

\lstset{language=FortyFour} % Make all code bold
\label{s:proof}
As previously discussed, we provide a set of requirements that the type system needs to ensure, and prove the soundness of our invariant protocol over these,
in this way we are parametric over the concrete type system. In \autoref{s:typesystem}, we present an example type system and prove that it satisfies these requirements.

\subheading{Auxiliary Definitions}
To express our type system assumptions, we first need some auxiliary definitions. 
\LS

\noindent First, we inductively define the set of objects in the reachable object graph (\rog) of a location $l$:\\
\indent $l' \in \rog(\s, l)$ iff:%
\begin{iitemize}
	\item $l' = l$, or\SS
	\item $\exists f$ such that $l' \in \rog(\s, \s[l.f])$
\end{iitemize}
\LS
We define the \mrog of an $l$ to be the locations reachable from $l$ by traversing through any number of \Q!mut! and \Q!rep! fields:\\
\indent $l' \in \mrog(\s, l)$ iff:%
\begin{iitemize}
	\item $l' = l$, or\SS
	\item $\exists f$ such that $\C{l}.f = \field{\fmdf}{\_}{f}$, $\fmdf \in \{\Kw{mut},\Kw{rep}\}$, and $l' \in \mrog(\s, \s[l.f])$
\end{iitemize}
Thus the \mrog of $l$ are the objects that could be mutated via a reference to $l$.

\LS
\noindent We define what it means for an $l$ to be \reach from an expression or context:
\begin{iitemize}
\item $\reach(\s, e, l)$ iff $\exists l' \in e$ such that $l \in \rog(\s, l')$\SS
\item $\reach(\s, \E, l)$ iff $\exists l' \in \E$ such that $l \in \rog(\s, l')$
\end{iitemize}

\LS

We now define what it means for an object to be \immut: it is in the \rog of an \Q!imm! reference or a \reach \Q!imm! field:\\*
\indent $\immut(\s, e, l)$ iff $\exists l'$ such that:
\begin{iitemize}
\item $\Kw{imm}\,l' \in e$, and $l \in \rog(\s, l')$, or\SS
\item $\reach(\s, e, l')$, $\C{l'}.f = \field{\Kw{imm}}{\_}{f}$, and $l \in \rog(\s, \s[l'.f])$, for some $f$
\end{iitemize}

Now we can define what it means for an $l$ to be \muty\footnote{We use the term \muty and not `\emph{mutable}' as an object might be neither \muty nor im\emph{mutable}, e.g. if there are only \Q!read! references to it.} by an expression $e$: something reachable from $l$ can also be reached by using a \Q!mut! or \Q!capsule! reference in $e$, and traversing through any number of \Q!mut! or \Q!rep! fields:\\
\indent $\muty(\s, e, l)$ iff $\exists l', l''$ such that:
\begin{iitemize}
	\item $l' \in \rog(\s,l)$\SS
	\item $\mdf\,l'' \in e$ with $\mdf \in \{\Kw{mut}, \Kw{capsule}\}$\SS
	\item $l' \in \mrog(\s,l'')$
\end{iitemize}
\noindent The idea is that $e$ could mutate something reachable from $l$: by using $l''$ to get a \Q!mut! reference to $l'$, and then performing a field update on it; the new field value for $l'$ would then be observable through $l$. In particular, we will require the type system to ensure that $e$ can only mutate state observable from $l$ if $l$ is \muty.

\LS

\noindent Finally, we model the \encap property of \Q!capsule! references:\\
\indent $\encap(\s, \E, l)$ iff $\forall l' \in \rog(\s, l)$, if $\muty(\s, \E[\Kw{capsule}\,l], l')$, then  not $\reach(\s, \E, l')$.

\noindent
That is, a location $l$ found in a context $\E$ is encapsulated if all $\muty$ objects in its $\rog$ would be unreachable with that single use of $l$ removed.
That single use of $l$ is the connection preventing those $\muty$ objects from being garbage collectable.

\subheading{Type System Requirements}
\IO[8]{As we do not want to require a specific concrete type system, we instead assume some properties about the expressions that it admits.}
Rather than requiring each expression during reduction to be well-typed, we instead let the type-system impose restrictions on method bodies, and type-check the initial expression, we then require properties on all future memories and expressions (i.e. $\VS$s).
In \autoref{s:typesystem} we show such a type-system and prove it satisfies these requirements, but these requirements do \emph{not} hold for arbitrary well-typed $\s|e$ pairs, only for $\VS$s.
This allows the type-system to be simpler, in particular, as the initial main expression can only have \Q!mut! references to $c$ (an object with no fields(), the type-system need not check that the heap structure and reference capabilities in the main expressions are consistent.

First we require that fields and methods are only given values with the correct reference capabilities, i.e. the field initialisers of \Q!new! expressions, the right hand sides of update expressions, and the receiver and parameters of method calls have the capabilities required by the field declarations/method signatures:
\SS\begin{restatable}[Type Consistency]{Requirement}{REQTypeCons}\
\LSitem
\begin{ienumerate}
	\item If $\VS(\E[\new{C}{\drangex{\mdf}{\_}}])$, then:
	\begin{nitemize}
		\item there is a $\clazz{C}{\_}{\Fs}{\_}$
		\item $\Fs = \drangex{\fmdf}{\_\,\_}$
		\item $\drange[\text{, }]{\mdf}{\!\leq \derep{\fmdf}}$
	\end{nitemize}

	\item If $\VS(\E[\_\,l\D f \equals \mdf\,\_])$, then:
	\begin{nitemize}
		\item $\C{l}.f = \field{\fmdf}{\_}{f}$\SS[0.15]
		\item $\mdf \leq \derep{\fmdf}$
	\end{nitemize}

	\item If $\VS(\E[\call{\mdf_0\,l}{m}{\drangex{\mdf}{\_}}])$, then:
	\begin{nitemize}
		\item $\C{l}.m = \method{\mdf'_0}{\_}{m}{\drangex{\mdf'}{\_\,\_}}{\_}$\SS[0.15]
		\item $\drange[\text{, }][0]{\mdf}{\!\leq\mdf'}$
	\end{nitemize}
\end{ienumerate}
\end{restatable}%
\SS\noindent This requirement also ensure that objects are created with the appropriate number of fields, and that fields and methods that are accessed/updated/called actually exist.

\LS

Now we define formal properties about our reference capabilities, thus giving them meaning. First we require that
an \immut object can not also be \muty: i.e. if an object is reachable from an \Q!imm! reference or field, then no part of its \rog can be reached by starting at a \Q!mut! or \Q!capsule! reference, and then traversing through \Q!mut! and \Q!rep! fields:
\SS\begin{restatable}[Imm Consistency]{Requirement}{REQImmCons}\ \\
	\indent If $\VS(\s, \E[e])$ and $\immut(\s, e, l)$, then not $\muty(\s, e, l)$.
\end{restatable}
\SS\noindent Thus $e$ cannot use field accesses to obtain a \Q!mut! or \Q!capsule! reference to anything reachable from an \immut $l$.
Note that this does not prevent \emph{promotion} from a \Q!mut! to an \Q!imm!: an \Q!as! expression can change a reference from \Q!mut! to \Q!imm!, provided that in the new state there are no longer any \Q!mut! references to the \rog of $l$. Note that from the definition of \muty and \immut, it follows that if $l$ is \immut in any $e$,
then it is \immut in $\E[e]$, and not \muty in any $e' \in \E[e]$.

\LS 

\noindent We require that if something was not \muty, it remains that way:%
\SS\begin{restatable}[Mut Consistency]{Requirement}{REQMutCons}\ \\
	\indent If $\VS(\s, \E[e])$, $l \in \dom(\s)$, not $\muty(\s, e, l)$, and $\s|e \rightarrow^{*} \s'|e'$, then not $\muty(\s', e', l)$.
\end{restatable}
\SS\noindent Note that this holds even if $l$ is \muty through $\E$, thus an \Q!as! expression cannot change a \Q!read! or \Q!imm! reference to \Q!mut!, as the associated location will not be \muty within the body of the \Q!as! expression, even if there are \Q!mut! references to the same object outside the \Q!as!.

\LS

We require that any \Q!capsule! reference is \encap, i.e. that no \muty part of its \rog is reachable through any other reference:%
\SS\begin{restatable}[Capsule Consistency]{Requirement}{REQCapCons}\ \\
\indent If $\VS(\s,\E[\Kw{capsule}\,l])$, then $\encap(\s, \E, l)$.
\end{restatable}%
\SS\noindent As all objects are created as \Q!mut!, the only way to actually get a \Q!capsule! reference is via an \Q!as! expression.
As our reduction rules impose no constraints on such expressions, the type-system must ensure that it only accepts a $\Kw{as}\,\Kw{capsule}$ expression if it is guaranteed to 
return an $\encap$ reference. Note that a specific type system's idea of ``capsuleness'' may in fact be stronger then \encap, but \encap is sufficient for our invariant protocol.

\LS

\noindent We require that field updates are only performed on \Q!mut!/\Q!capsule! receivers:%
\SS\begin{restatable}[Mut Update]{Requirement}{REQMutUpd}\ \\
	\indent If $\VS(\E[\mdf\_\D\_ \equals \_])$, then $\mdf \leq \Kw{mut}$.
\end{restatable}
\LS
%Finally we require strong exception safety: the body of a \Q!try! block does not mutate objects that existed before the enclosing \Q!try!--\Q!catch! began executing:
%\SS\begin{restatable}[Strong Exception Safety]{Requirement}{REQSES}\ \\
%	\indent  If $\VS(\s', \E[\trys{\s}{e}{\_}])$, then $\s \subseteq \s'$.
%\end{restatable}
Finally we require strong exception safety: the body of a \Q!try! block does not mutate objects that existed before the enclosing \Q!try!--\Q!catch! began executing and are reachable outside the \Q!try! block:
\SS\begin{restatable}[Strong Exception Safety]{Requirement}{REQSES}\ \\
	\indent  If $\VS(\s', \EV[\trys{\s}{e}{e'}])$, then $\forall l \in \dom(\s)$, if $\reach(\s, \EV[e'], l)$, then $\s(l) = \s'(l)$.
\end{restatable}
\SS\noindent
Note that this strong requirement \emph{only} needs to hold because our \Q@try@--\Q@catch@ can catch invariant failures: in L42, \Q@try@--\Q@catch@'s that catch \emph{checked} exceptions do not need this restriction.
Note that as our reduction rules never modify the body of a \Q!catch!, it follows that if $\VS(\s', \EV[\trys{\s}{\_}{e}])$, then for any $l \in \dom(\s')$, if $l \notin \dom(\s)$, then $l$ is not \reach in $\EV[e]$.

%Finally we require a form of strong exception safety: that the body of a \Q!try! block will never contain a field update or rep mutator call to any object that existed before the 
%enclosing \Q!try!--\Q!catch! began executing:
%	%the memory preserved by each \Q@try@--\Q@catch@ execution is never \muty within the \Q!try!:%
%\SS\begin{restatable}[Strong Exception Safety]{Requirement}{REQSES}\ \\
%	\indent If $\VS(\s', \E[\try{\s}{e}{\_}])$, then $\forall l \in \dom(\s)$:
%	\begin{iitemize}
%	\item $\_\,l.f \equals \_ \notin e$\SS
%	\item forall $\call{\_\,l}{m}{e} \in e$, there is no $f$ with $C.f = \field{\Kw{rep}}{\_}{f}$ and $C.m = \method{\Kw{mut}}{\_}{m}{\_}{\E[\Kw{this}.f]}$
%	\end{iitemize}
%\end{restatable}
%\SS\noindent By induction, it follows that the reduction of the try block never performs a field update or rep mutator call on an $l \in \dom(\s)$. By the fact that only the \textsc{update} reduction rule modifies memory, it follows that $\s \subseteq \s'$, i.e. $\s'$ is the same as $\s$, except it may have newly created objects. We explicitly require the absence of rep mutator calls, as even if the method's body does not perform a field update, the call will still insert a monitor over $l$. Note that \thm{Strong Exception Safety} \emph{only} needs to hold because our \Q@try@--\Q@catch@ can catch %invariant failures: in L42, \Q@try@--\Q@catch@'s that catch \emph{checked} exceptions do not need this restriction.
%\LS
%We use strong exception safety to prove that locations preserved by \Q@try@ blocks are never monitored (this is important as it means that a \Q!catch! that catches a monitor failure will not be able to see the responsible object),%
%\SS\begin{Lemma}[Unmonitored Try]\ \\
%	%If $\VS(\s, e)$, then $\forall \E$, $e = \E[\trys{\s_0}{\E'[\M{l}{\_}{\_}]}{\_}]$ implies $l\notin\s_0$
%	\indent If $\VS(\s, e)$ and $e = \E[\trys{\s'}{\E'[\M{l}{\_}{\_}]}{\_}]$, then $l \notin \s'$.
%\end{Lemma}
%\begin{proof}
%By $\VS$ we have $c\mapsto\Kw{Cap}\{\}|e_0\rightarrow^+ \s|e$, so we proceed by induction on the number of ``$\rightarrow$''s: in the base case, $e = e_0$ and so it cannot contain a monitor expression by %the definition of \VS. If this property holds for $\VS(\s, e)$ but not for $\s'|e'$ with $\s|e\rightarrow \s'|e'$, we must have applied the \textsc{new}, \textsc{update}, or \textsc{call} rules; since no %other reduction steps introduce a monitor expression. If the reduction was a \textsc{new}, $l$ will be fresh, so it could not have been in $\s'$. If the reduction was an \textsc{update}, by \thm{Mut %Update}, $\mdf\,l \in e$, with $\mdf \leq \Kw{mut}$, similarly (by our well-formedness rules on method bodies) \textsc{call} will only introduce a monitor over a call to a \Q!mut! method, so by \thm{Type %Consistency}, $\mdf\,l \in e$, with $\mdf \leq \Kw{mut}$; either way we have that $l$ was \muty, since our reductions never change the $\s'$ annotation, by \thm{Strong Exception Safety}, we have that $l %\notin \s'$.
%\end{proof}

\subheading{Useful Lemmas}
	First we prove a few useful lemmas about the properties of references in our language.

	\LS
	
	By the definition of \VS and the reduction rules themselves, we can show that the main expression and heap never contain dangling references:
	
\SS\begin{Lemma}[No Dangling]\ \\
		\indent If $\VS(\s, e)$ then:
		\begin{iitemize}
			\item $\forall l \in e$, $l \in \dom(\s)$\SS
			\item $\forall l \in \dom(\s)$, if $\s(l) = C\{\ls\}$ then $\{\ls\} \subseteq \dom(\s)$
		\end{iitemize}
	\end{Lemma}\SS
	\begin{proof}
		The proof is by definition of \VS, and induction on the number of reductions since the initial memory and main-expression.
		In the base case, by definition of \VS, the only $l$ in the main-expression and memory is $c$, which is defined in the memory.
		In the inductive case, each reduction rule only introduces $l$s into the memory or main-expression that were either already there, or in the case of \textsc{new/new true}, that are simultaneously added to the \dom of the memory.
		%By induction, in the base case, by definition of $\VS$, $\dom(\s) = \{c\}$ and $\s(c) = \Kw{Cap}\{\}$, and the only $l \in e$ is $c$, so the statement trivially holds.
		%In the inductive case, we have $\s'|e' \rightarrow \s|e$ for some $\s'$ and $e'$ satisfying the lemma, each reduction rule only introduces $l$s into $\s$ and $e$ that were either already in $\s'$ or $e'$, or in the case of \textsc{new}, are simultaneously added to the \dom of $\s$, thus by the inductive hypothesis, every $l$ in $\s$ or $e$ is also in $\dom(\s)$. Note that this holds only because our well-formedness rules prevent $l$s from being in method bodies, otherwise the \textsc{call} rule could introduce dangling references.
	\qed\end{proof}
	\noindent As a simple corollary of this, we have that if $l \in \dom(\s)$, then $\rog(\s, l) \subseteq \dom(\s)$, similarly with \mrog.

	\LS
	Similarly, we show that once an $l$ becomes un-\reach, it remains that way:
	
\SS\begin{Lemma}[Lost Forever]\ \\
		\indent If $\VS(\s, \E[e])$, and $\s|e \rightarrow^* \s'|e'$, then $\forall l \in \dom(\s)$, if not $\reach(\s, e, l)$, then not $\reach(\s', e', l)$.
	\end{Lemma}\SS
	\begin{proof}
		The proof follows from induction on the number of reductions, and the fact that each reduction either does not introduce an $l$ into the main expression or heap,
		or only introduces $l$s that were already \reach (in the case of \textsc{update} and \textsc{access}), or only introduces an $l \notin \dom(\s)$ (in the case of \textsc{new/new true}).
	\qed\end{proof}
	\LS
	
	\noindent We show that a sub-expression can mutate an object only if it is \muty:
	
	\SS\begin{Lemma}[Non-Mutating]\ \\
		\indent If $\VS(\s,\E[e])$, $l \in \dom(\s)$, not $\muty(\s, e, l)$, and $\s|e \rightarrow^* \s'|e'$, then $\s'(l) = \s(l)$.
	\end{Lemma}
	\SS\begin{proof}
		By \thm{No Dangling}, $l$ is always in the $\dom$ of memory, so 
		by \thm{Mut Consistency}, $l$ never becomes $\muty$, and so we never obtain a $\Kw{mut}$ or $\Kw{capsule}$ reference to it, thus by \thm{Mut Update}, we never update the fields of $l$, and there are no reduction rules that remove from $\s$.
		\qed\end{proof}
	\LS
	
		We can use our object capability discipline (described in \autoref{s:formalism}) to prove that the \Q!invariant()! method is deterministic and does not mutate existing memory:%
	
\SS\begin{Lemma}[Determinism]\ \\
	\indent If $\VS(\s, \E[\invariant{l}])$ and $\s|\invariant{l} \rightarrow^{n} \s'|e'$, for some $n \geq 0$, then:
	\begin{iitemize}
		%\item $\forall l' \in \dom(\s)$, not $\muty(\s', e', l')$
		\item $\s \subseteq \s'$\SS
		\item $\s|\invariant{l} \Rightarrow^{n} \s'|e'$
	\end{iitemize}
	\end{Lemma}\SS
	\begin{proof}
	As the only reference in $\invariant{l}$ is $\Kw{read}\,l$, it follows from the definition of \muty, that there is no $l'$ with $\muty(\s, \invariant{l}, l')$, thus by \thm{Mutatatable Update} we have that for all $l \in \dom(\s)$, $\s(l) = \s(l')$, i.e. $\s \subseteq \s'$

	We show the second part by induction on $n$: if $n = 0$, then no reduction was performed, $e' = \invariant{l}$, and it trivially holds that $\s|\invariant{l} \Rightarrow^{0} \s|\invariant{l}$. In the inductive case, we have some $\s''$ and $\e''$ with $\s|\invariant{l} \rightarrow^{n - 1} \s''|e'' \rightarrow \s'|e'$, and assume our inductive hypothesis that $\s|\invariant{l} \Rightarrow^{n - 1} \s''|e''$.
	As $c$ is not \muty in $\invariant{l}$, by \thm{Mut Consistency}, $\Kw{mut}\,c \notin e''$ and $\Kw{capsule}\,c \notin e''$. Since, by definition, there are never any other instances of \Q!Cap!, it follows from \thm{Type Consistency} that the reduction $\s''|\e'' \rightarrow \s'|e'$ was not due to \textsc{call/call mutator} reducing a call to a \Q!mut! method of \Q!Cap!.
	As all other methods are uniquely defined, the reduction must have been deterministic, i.e.  $\s''|\e'' \Rightarrow \s'|e'$, and so by the inductive hypothesis, we have $\s|\invariant{l} \Rightarrow^{n} \s''|e''$.
	\qed\end{proof}

	\LS
\subheading{Rep Field Soundness}
Now we define and prove important properties about our novel \Q!rep! fields. We first start with a few core auxiliary definitions.
To simplify the notation, we define the \rf of an $l$ to be the set of \Q!rep! field names for $l$:\\
\indent $\rf(\s,l) = \{f  \text{ where } \C{l}.f = \field{\Kw{rep}}{\_}{f}\}$
\LS

\noindent We say that an $l$ and $f$ is \CR if $l$ is reachable from $l.f$:\\
\indent $\CR(\s, l, f)$ iff $l \in \rog(\s, \s[l.f])$.\\
\noindent We say that an $l$ is \RCR if any its \Q!rep! fields are \CR:\\
\indent $\exists f \in \rf(\s,l)$ such that $\CR(\s, l, f)$.
\LS

\noindent We say that an $l$ and $f$ is \CN if $l.f$ is not \muty without passing through $l$:\\
\indent $\CN(\s, l, f)$ iff not $\muty(\s\setminus l, e, \s[l.f])$.\\
\noindent We say that an $l$ is \RCN if each of its \Q!rep! fields are \CN:\\
\indent $\forall f \in \rf(\s,l)$ we have $\CN(\s, l, f)$.
\LS
%not \RCR if it is not reachable from any of its \Q!rep! fields:\\
%\indent $not \RCR(\s, l, f)$ iff $\isrep{l}{f}$ implies that $l \notin \rog(\s, \s[l.f])$
%\indent $not \RCR(\s, l)$ iff $\forall f$, $not \RCR(\s, l, f)$
%\LS 
%
%\noindent We say that an $l$ is \RCN if none of its \Q!rep! fields are \muty without passing through $l$:\\
%\indent $\RCN(\s, e, l, f)$ iff $\isrep{l}{f}$ implies that not $\muty(\s\setminus l, e, \s[l.f])$.
%\indent $\RCN(\s, e, l)$ iff $\forall f$, $\RCN(\s, e, l, f)$.
%\LS 

\noindent We say that an $l$ is \RM if we are in a monitor for $l$ which must have been introduced by \textsc{call mutator}:\\
\indent $\RM(\s, e, l)$ iff $e = \E[\M{l}{e'}{\_}]$, with $e' \neq \Kw{mut}\,l$.
%\item $e = \E[\Kw{mut}\,l\D f]$, with $f \in \rf(\s,l)$
%\end{iitemize}
\LS 

\noindent Finally we say that $l$ is \HNO if we are in a monitor introduced for a call to a rep mutator, and $l$ is not reachable from inside this monitor, except perhaps through a single \Q!rep! field access:\\
\indent $\HNO(\s, e, l)$ iff $e = \EV[\M{l}{e'}{\_}]$, and either:
\begin{iitemize}
\item not $\reach(\s, e', l)$, or\SS
\item $e' = \E[\Kw{mut}\,l\D f]$, $f \in \rf(\s,l)$, and not $\reach(\s, \E, l)$
\end{iitemize}

\LS 

\noindent Now we formally state the core properties of our \Q!rep! fields (informally described in \autoref{s:protocol}):%
\SS\begin{theorem}[Rep Field Soundness]\ \\
	\indent If $\VS(\s, e)$ then $\forall l$ with $\reach(\s, e, l)$, we have:
	\begin{iitemize}
		\item not $\RCR(\s, l, f)$, and\SS
		\item one of the following holds:
		\begin{iitemize}
			\item $\RCN(\s, l)$ and not $\RM(\s, e, l)$, or\SS[0.15]
			\item $\HNO(\s, e, l)$.
		\end{iitemize}
	\end{iitemize}
\end{theorem}
\SS\noindent That is, for every reachable object $l$: 
	$l$ is not reachable through any of its \Q!rep! fields, 
	and either we are in a rep mutator for $l$ and $l$ is not observable (except perhaps through a single \Q!rep! field access),
	or we are not \RM $l$, and each of $l$s \Q!rep! fields are \CN.
\begin{proof}
By $\VS$ we have $c\mapsto\Kw{Cap}\{\}|e_0\rightarrow^{m} \s|e$, so we proceed by induction on $m$, the number of reductions. The base case when $m = 0$ is trivial, since \Q!Cap! has no \Q!rep! fields and the initial main expression $e_0$ cannot contain monitors.

In the inductive case, where $m > 0$, we have $\drange[\rightarrow][0][m - 1]{\s}{\!|e} \rightarrow \s|e$, for some  $\range[,][0][m - 1]{\s}$ and $\range[,][0][m - 1]{e}$, where $\s_0|e_0$ is a valid initial memory and expression.
Our inductive hypothesis is then that the conclusion of our theorem holds for each $\s_i|e_i$, for $i \in [0, m - 1]$.
We then proceed by cases on the reduction rule applied, and prove the theorems conclusion for $\s|e$:
\begin{ienumerate}
	\item $\rrule{new/new true}{\s'}{\new{C}{\drange{\mdf}{l}}}{\s}{e'}$, 
	\\where $\s = \s',l_0\mapsto C\{\range{l}\}$; by \thm{Type Cosnsistency}, we have $\clazz{C}{\_}{\drange{\fmdf}{\_\,f}}$.
	\begin{enumerate}
		\item We have that $l_0$ is not \RCR:
			by \thm{No Dangling}, we have that $\forall l' \in \dom(\s')$, $\rog(\s', l') \subseteq \dom(\s')$.
			By our notational conventions for ``$,$'', it follows that $l_0 \notin \dom(\s')$.
			Now consider each $i \in [1, n]$, since the pre-existing $\s'$ was not modified, it follows that $\rog(\s', l_i) = \rog(\s, \s[l_0.f_i])$.
			By \thm{No Dangling} we have that $\rog(\s, \s[l_0.f_i]) \subseteq \dom(\s)$, and so $l_0 \notin \rog(\s, \s[l_0.f_i])$, thus each $l_0.f_i$ is not \CR.
			
		\item Ever \reach $l' \neq l_0$ is not \RCR: 
			Since reduction didn't modify the fields of any pre-existing $l'$, by the inductive hypothesis, we have that $l'$ is still not \RCR.
		
		\item The new $l_0$ is \RCN and not \RM:
		\begin{itemize}
			\item Consider each $i \in [1, n]$ with $\fmdf_i = \Kw{rep}$.
				By \thm{Type Consistency} and \thm{Capsule Consistency}, $l_i$ was \encap and so $\rog(\s', l_i)$ cannot be \muty from \EV.
				Thus, we don't have $\muty(\s\setminus l_0, \EV[e'], l_i)$, and so each of $l_0$s \Q!rep! fields is \CN.
			\item We trivially have that $l_0$ is not \RM since $l_0 \notin \dom(\s')$, by \thm{No Dangling}, there can't be any monitor expressions for it in $\EV$.
		\end{itemize}

		\item Every \reach $l' \neq l_0$ that was \RCN and not \RM still is:
		\begin{itemize}
			\item Suppose we have made it so that for some $f' \in \rf(\s',l')$, $l'.f'$ is no longer \CN.
				
			Since we didn't modify the \rog of $l'$ nor the \rog of any other pre-existing $l''$, we must have that $\s'[l'.f']$ is now \muty through $l_0.f_i$, for some $i \in [1, n]$.
			This requires that $l_i$ is an initialiser for a \Q!mut! or \Q!rep! field, which by \thm{Type Consistency} means that $\mdf_i \leq \Kw{mut}$.
			But then $\s'[l'.f']$ was already \muty through $\mdf_i\,l_i$, so $l'.f'$ can't have already been \CN, a contradiction.

			\item We can't have caused $l'$ to be \RM since we haven't introduced any monitor expressions, nor modified any existing ones.
		\end{itemize}

		\item Every \reach $l' \neq l_0$ is \HNO: 
			by \thm{No Dangling}, $l' \in \dom(\s')$, so by \thm{Lost Forever}, $l'$ must have already been \reach.
			Thus, by the inductive hypothesis, $l'$ must be \HNO, but we haven't removed any monitor expression or field accesses (because the arguments to the constructor are all fully reduced values), thus $l'$ is still \HNO.
	\end{enumerate}
	
	\item $\rrule{access}{\s}{\mdf\,l\D f}{\s}{\rmdf{\mdf}{\fmdf}\,\s[l.f]}$, where $\C{l}.f = \field{\fmdf}{\_}{f}$:
	\begin{enumerate}
		\item No \reach $l'$ is \RCR: 
			this holds by the inductive hypothesis and the fact that we haven't mutated memory.
		
		\item If $l$ is \reach and it was \RCN and not \RM, than it still is:
		\begin{itemize}
			\item If $\fmdf \neq \Kw{rep}$, then we can't have broken \CN for any $f' \in \rf(\s,l)$, since by definition of \RCN, $\s[l.f']$ can't have been \muty through $\s[l.f]$.
			\item If $\fmdf = \Kw{rep}$, since $l'$ was not \RM, this field access can't have been inside a rep mutator (or else we would be inside a monitor).
			As fields are instance private, we have $\mdf \neq \Kw{mut}$, or else the field access would have come from a rep mutator.
			
			If $\mdf = \Kw{capsule}$, then by \thm{Capsule Consistency} and \RCR, $l$ is not \reach from $\EV[\rmdf{\mdf}{\fmdf}\,\s[l.f]]$, so it is irrelevant if $l$ is no longer \RCN.
			Otherwise, since $\mdf \notin \{Kw{capsule}, \Kw{mut}\}$, we have $\rmdf{\mdf}{\fmdf} \nleq \Kw{mut}$, so $l.f$ is still \CN.
			By the above case for $\fmdf \neq \Kw{rep}$, every other $f' \in \rf(\s,l)$ is \CN.

			\item We can't have made $l'$ \RM since we have introduced any monitor expressions.
		\end{itemize}

		\item If $l$ was \RM or not \RCN, than it is \HNO: 
			by the inductive hypothesis, $l$ was \HNO before this reduction, thus $\EV = \EV'[\M{l}{\EV''}{\_}]$.
			As $l$ is clearly \reach in $\EV''[\mdf\,l\D f]$, by definition of \HNO we must have that $l$ is not \reach from $\EV''$, and $\fmdf = \Kw{rep}$.
			By \RCR, $l$ is not in the \rog of $\s[l.f]$, and so $l$ is not \reach from $\EV''[\rmdf{\mdf}{\fmdf}\,\s[l.f]]$, and so it is still \HNO.

		\item Every \reach $l' \neq l$ that was \RCN and not \RM, still is:
		\begin{itemize}
			\item Since this reduction doesn't modify memory, and $\rmdf{\mdf}{\fmdf} \leq \Kw{mut}$ only if $\mdf \leq \Kw{mut}$, we can't have made the \rog of any \Q!rep! field $f'$ of $l'$ \muty without going through $l'$, so \RCN is preserved.
			\item As in the \textsc{new/new true} case above, we can't have made \RM hold as we haven't introduced any monitor expressions.
		\end{itemize}

		\item If $l$ was \RM or not \RCN, than it is \HNO:
			if $f \in \rf(\s, l)$, with $\EV$ of form $\EV'[\M{l}{\EV''}{\_}$ and $l$ not \reach through $\EV''$,
			then $e$ is of form $\EV'[\M{l}{\EV''[\s[l.f]]}]$. By the above, $l$ is not \RCR, and so 
			$l$ is not \reach through $\s[l.f]$, thus $l$ is not \reach through $\EV''[\s[l.f]]$, and so $l$ is \HNO.
			Otherwise, by the inductive hypothesis, $l$ was \HNO, by definition of \HNO, since the above case does not hold,
			then $\EV$ is of form $\EV'[\M{l}{\EV''}{\_}]$ with $l$ not \reach through $\EV''[\mdf\,l\D f]$, thus by \thm{Lost Forever}, $l$ is not \reach through $\EV''[\s[l.f]]$, thus $l$ is still \HNO.
		
		\item Every \reach $l' \neq l$ that was \RM or not \RCN is \HNO:
			as this reduction doesn't create any new objects, by \thm{No Dangling} and \thm{Lost Forever}, anything \reach was already \reach, thus by the inductive hypothesis, $l'$ must have been \HNO.
			but we haven't removed any monitor expression or field accesses on $l'$, thus $l'$ must still be \HNO.
	\end{enumerate}

	\item $\rrule{update}{\s'}{\mdf\,l\D f\equals \mdf'\,l'}{\s'[l.f=l']}{\M{l}{\Kw{mut}\,l}{\invariant{l}}}$:
	\begin{enumerate}
		\item For each $f' \in \rf(\s, l)$, $l.f'$ is still not \RCR: 
		\begin{itemize} 
			\item if $f' = f$, then by \thm{Type Consistency} and \thm{Capsule Consistency}, $\encap(\s', \EV[\mdf\,l\D f = \hole], l')$.
			Hence $l$ is not \reach from $l'$, and so after the update, $l.f'$ cannot be \CR.
			\item otherwise, by the inductive hypothesis, $l.f'$ was not \RCR, so $l \notin \rog(\s', \s'[l.f'])$,
			and so this update couldn't have change the \rog of $l.f'$, and so it is still \RCR.
		\end{itemize}
		
		\item For every \reach $l'' \neq l$, and $f' \in \rf(\s,l'')$, $l''.f'$ is still not \CR:
		\begin{itemize}
			\item By the inductive hypothesis, $l''.f'$ was not \CR.
			\item 
				If $l''$ was \RCN, by \thm{Mut Update}, $\mdf \leq \Kw{mut}$. 
				By \RCN, the \rog of $\s'[l''.f']$ is not \muty, except through a \emph{field} access on $l''$,
				but this rule doesn't perform a field access, 
				so since $l'' \neq l$, we must have that $l \notin \rog(\s', \s'[l''.f'])$.
				Since we can't have modified the \rog of $\s'[l''.f']$, $l''.f'$ is still not \CR.
			
			\item Otherwise, by the inductive hypothesis, $l''$ was \HNO, and so $l'' \notin \rog(\s', l')$, so we can't have added $l''$ to the \rog of anything,
			thus $l''.f'$ is still not \CR.
		\end{itemize}

		\item Any \reach $l''$ that was \RCN and not \RM still is:
		\begin{itemize}
			\item Suppose $l'' = l$ and $f \in \rf(\s', l)$, by \thm{Type Consistency} and \thm{Capsule Consistency}, $l'$ is \encap, thus $l'$ is not \muty from \EV, and $l$ is not \reach from $l'$.
				Hence $l'$ is still \encap, and so $l.f$ is still \CN.
			\item Now consider any $f' \in \rf(\s', l'')$, with $l''.f' \neq l.f$; by the above, $l$ is not \RCR and so $l \notin \rog(\s', \s'[l''.f'])$.
				If $f$ was a \Q!mut! or \Q!rep! field, by \thm{Type Consistency}, $\mdf' \leq \Kw{mut}$, so by \RCN, $l' \notin \rog(\s', \s'[l''.f'])$; thus we can't have made $\rog(\s', \s'[l''.f'])$ \muty through $l.f$; so $\s'[l''.f']$ can't now be \muty through $\Kw{mut}\,l$. 
				By \thm{Mut Consitency}, we couldn't have have made $\s'[l''.f']$ \muty some other way, so $l''$ is still \RCN.
			\item As in the above cases for \textsc{new/new true}, $l''$ is still not \RM as we haven't introduced any monitor expressions.
		\end{itemize}
		
		\item Every \reach $l'$ that was \RM or not \RCN is \HNO:
			similarly to the above case for \textsc{access},
			as this reduction doesn't create any new objects, by by \thm{No Dangling} and \thm{Lost Forever}, anything \reach was already \reach, thus by the inductive hypothesis, $l'$ must have been \HNO.
			but we haven't removed any monitor expression or field accesses, thus $l'$ must still be \HNO.
	\end{enumerate}

	\item $\rrule{call/call mutator}{\s}{\call{\mdf_0\,l_0}{m}{\drange{\mdf}{l}}}{\s}{e}$
	\begin{enumerate}
		\item Every \reach $l'$ is not \RCR:
			as this rule doesn't mutate memory, by the inductive hypothesis, every \reach $l'$ is still not \RCR.
		

		\item If $l_0$ was \RCN and not \RM, it either still is, or is now \HNO:
		\begin{itemize}
			\item As we haven't modified memory, and by our well-formedness rules on method bodies, we haven't introduce any new $l$s into the main-expression, we must have that $l_0$ is still \RCN.
			\item Suppose the rule applied was \textsc{call}, by our well-formedness rules for method bodies, $e$ doesn't contain a monitor. 
				Moreover, by the \textsc{call} rule, $e$ is not a rep mutator, if $e = \E[\mdf'\,l_0\D f]$, for some $f \in \rf(\s, l_0)$, we must have that $m$ was not a \Q!mut! method.
				Since fields are instance-private, we must have $\mdf' \nleq \Kw{mut}$, and by our well-formedness rules on method bodies, $e$ doesn't contain any monitors, thus we can't have caused $l_0$ to be \RM.

			\item Otherwise, the rule applied was \textsc{call mutator}, and $m$ is a rep mutator, and hence we have $e = \M{l_0}{e'}{\invariant{l_0}}$.
				By our rules for rep mutators, $m$ must be a \Q!mut! method with only \Q!imm! and \Q!capsule! parameters, thus by \thm{Type Consistency}, 
				$\mdf_0 \leq \Kw{mut}$, and for each $i \in [1, n]$, $\mdf_i \in \{\Kw{imm},\Kw{capsule}\}$. 
				By \thm{Imm Consistency} and \thm{Capsule Consistency}, $l_0$ can't be reachable from any $l_i$. Since rep mutators use \Q!this! only once, to access a \Q!rep! field, $e' = \E[\Kw{mut}\,l_0\D f]$, for some $f \in \rf(\s, l_0)$. 
				By our rules for rep mutators, $l_0 \notin \E$, and $l_0$ is not \reach from any $l_i$, and by our well-formedness rules for method bodies, 
				there are no other $l$s in $\E$, thus we have that $l_0$ is not \reach from any $\E$, thus \HNO now holds for $l$.
		\end{itemize}

		\item Every $l' \neq l_0$ that was \RCN and not \RM, still is:
		\begin{itemize}
			\item By the above, since we haven't modified memory or introduced any new $l$s, $l'$ must still be \RCN.
			\item Since $l' \neq l_0$ and fields are instance-private, we must have that there is no $\mdf'\,l'\D f \in e$.
				Moreover, by our well-formedness rules on method bodies, and the \textsc{call/call mutator} rules, the only monitor that could be in $e$ is a monitor on $l_0$, thus we can't have made $l'$ \RM.
		\end{itemize}
		

		\item Every \reach $l'$ that was \RM or not \RCN is \HNO:
			as in the \textsc{update} case above, by the inductive hypothesis, $l'$ must have been \HNO, as we haven't removed any monitor expressions or field accesses, $l'$ is still \HNO.
	\end{enumerate}

	\item $\rrule{try error}{\s}{\trys{\s'}{e}{e'}}{\s}{e'}$, where $\error(\s, e)$
	\begin{enumerate}
		\item Every \reach $l$ is not \RCR:
			as in the \textsc{call/call mutator} case above, since this rule doesn't mutate memory, by the inductive hypothesis, every \reach $l$ is still not \RCR.
	
		\item Every \reach $l$ that was \RCN and not \RM still is: 
			by \thm{Mut Consistency} and the fact that we haven't modified memory, $l$ must still be \RCN. Since we haven't introduced any monitor expressions or field accesses, $l$ cannot now be \RM.
		
		\item If $l$ is still \reach, and was \RM or not \RCN then it is now \RCN and not \RM:
		\begin{itemize}
			\item By definition of \error, we have $e = \EV'[\M{l}{v}{v'}]$.
			
			\item If the monitor was introduced by \textsc{new} or \textsc{update}, then $v = \Kw{mut}\,l$. And so \HNO can't have held for $l$ since $l = l'$, and $v$ was not the receiver of a field access.
			Thus by the inductive hypothesis, $l$ must have been \RCN and not \RM, a contradiction.
			\item By definition of \VS and our well-formedness rules on method bodies, we must have that monitor must introduced by \textsc{call mutator}, due to a call to a rep mutator on $l$.\footnote{
				A type-system will likely prevent this case from happening, 
				as this would require calling a \Q!mut! method on $l$, but $l$ is \reach outside the \Q!try! block. However, if the type system can prove that said \Q!mut! method will not actually mutate $l$, this would not violate our requirements.
				Thus we still need to ensure that \thm{Rep Field Soundness} holds in this case.}
			\item From our reduction rules, it follows that we were previously in a state $\s_i|e_i$, where $i \in [1, m - 1]$, $e_i$ is of form $\EV''[e'']$, and the next state was obtained by said application of the \textsc{call mutator} rule to $e''$.
			\item Moreover, it follows that $\EV''= \EV[\trys{\s'}{\EV'}{e'}]$, as no reduction rules modify the $\EV$.
			\item We must not have had that $l$ was \HNO, since $e''$ would contain $l$ as the receiver of a method call. Thus, by our inductive hypothesis, in state $i$, $l$ was \RCN and not \RM.
		
			\item By \thm{Strong Exception Safety} and \thm{No Dangling}, every $l'$ \reach from $\EV[e']$ has not been mutated, i.e. $\s(l') = \s_i(l') = \s'(l)$.
			\item Since nothing \reach has been mutated, it follows that $l$ is still \RCN.
			\item By \VS and our well-formedness rules on method bodies, it follows that $e'$ contains no monitor expressions.
			\item Moreover, since $l$ was not \RM in $\EV[\trys{\s'}{\EV'[e'']}{e'}]$, and $e'$ contains no monitors, $l$ it follows that $l$ is not \RM in $\EV[e']$.
		\end{itemize}

		\item Every \reach $l'' \neq l$ that was \RM or not \RCN is \HNO: 
			as in the above case for \textsc{update}, by the inductive hypothesis, $l''$ must have been \HNO, as we haven't removed any monitor expressions on $l''$, or any field accesses, $l''$ is still \HNO.
	\end{enumerate}

	\item $\rrule{monitor exit}{\s}{\M{l}{\mdf\,l'}{\_}}{\s}{\mdf\,l'}$
	\begin{enumerate}
		\item Every \reach $l''$ is not \RCR:
			as in the \textsc{call/call mutator} case above, since this rule doesn't mutate memory, by the inductive hypothesis, every \reach $l''$ is still not \RCR.
		
		\item Every \reach $l''$ that was \RCN and not \RM still is:
			as in the \textsc{try error} case above, by \thm{Mut Consistency} and the fact that we haven't modified memory, $l''$ must still be \RCN. Since we haven't introduced any monitor expressions or field accesses, $l''$ cannot now be \RM.

		\item If $l$ is still \reach, and $l$ was \RM or not \RCN then it is now \RCN and not \RM:
		\begin{itemize}
			%Change e'' to e' and \EV'' to \EV' and other notations...
			\item If the monitor was introduced by \textsc{new} or \textsc{update}, then $\mdf\,l' = \Kw{mut}\,l$. And so \HNO can't have held for $l$ since $l = l'$, and $v$ was not the receiver of a field access.
			Thus by the inductive hypothesis, $l$ must have been \RCN and not \RM, a contradiction.
			\item By definition of \VS and our well-formedness rules on method bodies, we must have that monitor must introduced by \textsc{call mutator}, due to a call to a rep mutator on $l$.
			\item From our reduction rules, it follows that we were previously in a state $\s_i|e_i$, where $i \in [1, m - 1]$, $e_i$ is of form $\EV'[e']$, and the next state was obtained by said application of the \textsc{call mutator} rule to $e'$.
			\item Moreover, it follows that $\EV'= \EV$, as no reduction rules modify the $\EV$.
			\item We must not have had that $l$ was \HNO, since $e'$ would contain $l$ as the receiver of a method call. Thus, by our inductive hypothesis, in state $i$, $l$ was \RCN and not \RM.
		
			
			\item As with the above case for \thm{try error}, by the inductive hypothesis, $l$ must have been \HNO, and so the monitor must have been introduced by \textsc{call mutator}.
			\item Thus, we were previously in a state $\s_i|e_i$ where $i \in [1, m - 1]$, $e_i$ is of form $\EV[e']$, and the next state was obtained by said application of the \textsc{call mutator} rule to $e'$.
			\item Thus, by the inductive hypothesis, in state $i$, $l$ must have been \RCN and not \RM.
			
			\item Because $l$ was not \RM in $\s_i|\EV[e']$, and $\mdf\,l'$ contains no monitors, $l$ is not \RM in $\EV[\mdf\,l']$.
			
			\item Since a rep mutator cannot have any \Q!mut! parameters, by \thm{Type Consistency} and \thm{Non-Mutating}, the body of the method can only modify things \muty through $l$, or a \Q!capsule! parameter.
			\item By \thm{Type Consistency}, and \thm{Capsule Consistency}, every capsule parameter is \encap, and so anything mutated through such a parameter must have been un\reach outside the call.
			\item Thus, forall $l' \in \dom(\s_i)$, if $\reach(\s_i, \EV, l')$ and $l' \notin \mrog(\s_i, l)$, then $ \s(l) = \s_i(l)$.
		
			\item If $\mdf = \Kw{capsule}$, then by \thm{Capsule Consistency}, not part of the \mrog of any \Q!rep! field of $l$ can be in the \rog of $l'$ (or else $l$ would have to be un\reach), so we can't have made such a field \muty.
			\item If $\mdf \neq \Kw{capsule}$, then since a rep mutator cannot have a \Q!mut! return type, 
				and our \textsc{call mutator} rule wraps the method body in a \Q!as! expression,
				we must have that $\mdf \nleq \Kw{mut}$.
				Thus  $\mdf \in \{\Kw{read}, \Kw{imm}\}$, and so by $l$ is not \muty through $\mdf\,l'$.
			
			\item As $l$ was \RCN in $\s_i|\EV[e']$, 
				and we haven't modified anything \reach through $\s \setminus l$, 
				nor have we made the \rog of $l$ \muty through $\mdf\,l'$,
				it follows that $l$ is also \RCN in $\EV[\mdf\,l']$.
		\end{itemize}

		\item Every \reach $l'' \neq l$ that was \RM or not \RCN is \HNO: 
			as in the \textsc{update} case above, by the inductive hypothesis, $l''$ must have been \HNO, as we haven't removed any monitor expressions on $l''$, or any field accesses, $l''$ is still \HNO.
	\end{enumerate}

	\item (\textsc{as}, \textsc{try enter}, and \textsc{try ok}) these are trivial, since as in the above cases:
	\begin{enumerate}
		\item Every \reach $l$ is not \RCR:
			as in the \textsc{call/call mutator} case above, since these rules don't mutate memory, by the inductive hypothesis, every \reach $l$ is still not \RCR.

		\item Every \reach $l$ that was \RCN and not \RM still is:
			as in the \textsc{try error} case above, by \thm{Mut Consistency} and the fact that these rules don't modified memory, $l$ must still be \RCN. Since this rules don't introduce any monitor expressions or field accesses, $l$ cannot now be \RM.

		\item Every \reach $l$ that was \RM or not \RCN is \HNO:
			as in the \textsc{update} case above, by the inductive hypothesis, $l$ must have been \HNO, as these rules don't remove any monitor expressions or field accesses, $l''$ is still \HNO.
	\qed\end{enumerate}
\end{ienumerate}
\end{proof}

\subheading{Stronger Soundness}
It is hard to prove \thm{Soundness} directly,
so we first define a stronger property,
called \thm{Stronger Soundness}.

\LS

We say that an object is \mony if execution
is currently inside of a monitor for that object, and
the monitored expression $e_1$ does not
contain a reference to $l$ as a \emph{proper} sub-expression:\\
\indent $\mony(e,l)$ iff
$e=\EV[\M{l}{e'}{\_}]$ and $l \in e'$ only if $e' = \_\,l$.\\
\noindent A monitored object is associated with an expression that cannot observe it, but may
reference its internal representation directly.
In this way, we can safely modify its representation before checking its invariant.
The idea is that at the start the object will be valid and $e'$ will reference $l$;
but during reduction, $l$ will be used to
modify the object, but not observe it; only after that moment, the object may become invalid.

\LS

\thm{Stronger Soundness} says that starting from a well-typed and well-formed $\s_0|e_0$, and performing any number of reductions, every \reach object is either \valid or \mony:%
\SS\begin{theorem}[Stronger Soundness]\ \\
\indent If $\VS(\s, e)$ then $\forall l$, if $\reach(\s, e, l)$, then $\valid(\s, l)$ or $\mony(e, l)$.
\end{theorem}\SS
\begin{proof}
	As with the above proof of \thm{Rep Field Soundness}, we will prove this inductively on the number of reductions.
	By $\VS$ we have $c\mapsto\Kw{Cap}\{\}|e_0\rightarrow^{m} \s|e$, 
		The base case when $m = 0$ is trivial, 
			from our requirements for the \Q!Cap! class,
			$\s|\invariant{c} \rightarrow \s|\new{\Kw{True}}{} \rightarrow  \s,l\mapsto \Kw{True}\{\}|l$, for some $l$, thus by \thm{Determinism}, it follows that $c$ (the only thing in the memory) is \valid.

	In the inductive case, where $m > 0$, we have $\drange[\rightarrow][0][m - 1]{\s}{\!|e} \rightarrow \s|e$, for some  $\range[,][0][m - 1]{\s}$ and $\range[,][0][m - 1]{e}$, where $\s_0|e_0$ is a valid initial memory and expression.
	Our inductive hypothesis is then that that everything \reach from the previous \VS is \valid or \mony. We then proceed by cases on the reduction rule that gets us to $\s|e$:
\begin{ienumerate}
	\item $\rrule{new}{\s'}{\new{C}{\range{\_\,l}}}{\s',l_0\mapsto C\{\range{l}\}}{\M{l_0}{\Kw{mut}\,l_0}{\invariant{l_0}}}$:
	\begin{itemize}
		\item Clearly the newly created object, $l$, is \mony.
		\item This rule does not modify pre-existing memory, introduce pre-existing $l$s into the main expression, nor remove monitors on other $l$s, by the inductive hypothesis, every $l' \neq l_0$ is still \valid (due to \thm{Determinism}), or \mony.
	\end{itemize}
	
	\item $\rrule{new true}{\s'}{\new{\Kw{True}}{}}{\s',l_0\mapsto \Kw{True}\{\}}{\Kw{mut}\,l_0}$:
	\begin{itemize}
		\item The \Q!True! class is required to have an invariant of \Q!new True()!,
		so as with $c$ in the base case above, we have that $l_0$ is \valid.
		\item As in the above case for \textsc{new}, since we didn't modify pre-existing memory, introduce pre-existing $l$s into the main expression, nor remove monitors, by the inductive hypothesis, every $l' \neq l_0$ is still \valid or \mony.
	\end{itemize}

	\item $\rrule{update}{\s'}{\mdf\,l\D f\equals v}{\s}{e'}$, where $e' = \M{l}{\Kw{mut}\,l}{\invariant{l}}$:
	\begin{itemize}
		\item Clearly $l$ is now $\mony$.
		\item Consider any other $l'$, where $l \in \rog(\s',l')$ and $l'$ was \valid; now suppose we just made $l'$ in\valid. 
			By our well-formedness criteria, \Q@invariant()@ can only accesses \Q@imm@ and \Q@rep@ fields, thus by \thm{Non-Mutating}, 
			and \thm{Determinism}, we must have that $l$ was in the \rog of $\s'[l'.f']$, for some $f' \in \rf(\s', l')$.
			
			Since $l \neq l'$, $l'$ can't have been \RCN. Thus, by \thm{Rep Field Soundness}, $l'$ was \HNO, and so $\EV[\mdf\,l\D f\equals v]$ is of form $\EV'[\M{l'}{e''}{e'''}]$:
			\begin{itemize}
				\item As the \rog of $l'$ has just been mutated, and since $e'''$ must have started off as $\invariant{l'''}$, if follows from \thm{Determinism}, that we cannot currently be inside $e'''$.
				\item Thus, $\EV = \EV'[\M{l'}{\EV''}{e'''}]$, where $\EV''[\mdf\,l\D f\equals v] = e''$.
				
				\item Suppose that $l'$ was not \reach in $e''$, then clearly $l' \notin e''$, since $l' \neq l$, it follows that $l' \notin \EV''[e']$, and so $l'$ is \mony.

				\item Otherwise, by definition of \HNO, we have that $e'' = \E[\Kw{mut}\,l'\D f'']$ for some $f'' \in \rf(\s', l')$, and where $l'$ is not \reach in $\E$.
				
				\item By the proof for the \textsc{try error} case of \thm{Rep Field Soundness}, the monitor must have come from a call to a rep mutator, in a state where $l'$ was \RCN.
					Thus, we were previously in a state $\s_i|e_i$, for some $i \in [0, m - 1]$, immediately after a \textsc{call mutator};
					moreover, $e_i$ is of form $\EV'[\M{l'}{e'_i}{\_}]$, immediately after a \textsc{call mutator}, where $e'_i$ is of form $\E'[\Kw{mut}\,l' \D f''']$.
				
				\item By \thm{Rep Field Soundness}, $l'$ is not \reach through $\s'[l'.f''']$,.
					By the proof for the \textsc{call/call mutator} case of \thm{Rep Field Soundness}, we have that $l'$ is not \reach through $\E'$.
					Thus, by \thm{Lost Forever}, once $\Kw{mut}\,l' \D f'''$ has been reduced, $l'$ must be un\reach, and it follows that $\Kw{mut}\,l'\D f'' = \Kw{mut}\,l' \D f'''$
			
				\item By \thm{Mut Update}, $l$ is \muty in the current state, thus by \thm{Mut Consistency}, we have that it was also \muty when \textsc{call mutator} rule was applied.
					 But we have that $l'$ was \RCN, so since $l \in \rog(\s', \s'[l'.f'])$, we have that $l$ can only be \muty through $l'$.

				\item By \thm{Lost Forever}, the only way we could have obtain a reference to $l$ was by reducing $\Kw{mut}\,l' \D f''$, but we haven't done that yet, a contradiction.
			\end{itemize}

		\item Every other \valid $l'$, where $l \notin \rog(\s',l')$ is still \valid by \thm{Determinism}.
		\item As in the above case, since we don't remove any monitors, any other $l'$ that was \mony, is still \mony.
	\end{itemize}

			
	\item $\rrule{try error}{\s}{\trys{\s'}{e}{e'}}{\s}{e'}$, where $\error(\s, e) = \EV'[\M{l}{\_}{\_}]$:
	\begin{itemize}
		\item As with the case for \textsc{try error} in the proof of \thm{Rep Field Soundnes}, we were previously in a state $\s_i|e_i$, where $e_i = \EV[\trys{\s'}{\_}{\_}]$, and $\s_i = \s'$.
		\item By definition of \error, we have that $l$ is not \valid in $\s$, since monitor expressions always start of as an invariant calls.
		\item Suppose $l$ is still \reach in $\s|\EV[e']$, by \thm{Strong Exception Safety}, we have $l \in \dom(\s')$. Thus by the inductive hypothesis, we have that $l$ was \valid or \mony in the state $\s'|e_i$.
		\item If $l$ was \mony, then by \VS and our well-formedness rules on method bodies, said monitor must have been introduced by the \textsc{new}, \textsc{update}, or \textsc{call mutator} reduction rules.
		\item The \textsc{new} and \textsc{update} rules monitor a value, which cannot reduce to a \Q!try!--\Q!catch!, so the monitor must have been introduced by \textsc{call mutator}.
		\item But by our well-formedness rules on rep mutators, the body of the called method cannot mention $l$ except to read a field, 
		as shown in the case for \textsc{update} above, $l$ will be un\reach once the field access has been reduced, which by \thm{Lost Forever} is a contradiction, as $l$ is \reach through $e$.
		\item Thus, $l$ can't have been \mony in $\s'|e_i$, so it must have been \valid.

		\item Also by \thm{Strong Exception Safety}, we have that nothing reachable from $l$ could have been modified, that is $\forall l' \in \rog(\s', l)$, we have $\s'(l') = \s(l')$.
			By \thm{Lost Forever}, and our reduction rules, any memory location not \reach from a call $\invariant{l}$ cannot affect its reduction.
		\item Thus, by \thm{Determinism}, and the fact that $l$ was valid in \s', we have that $l$ is still \valid, a contradiction.
		\item Thus, $l$ cannot be \reach, so the fact that it is in\valid is irrelevant.
		
		\item As in the above case for \textsc{new}, since we didn't modify any memory, or remove any other monitors, by the inductive hypothesis every $l' \neq l$ is still \valid or \mony.
	\end{itemize}

		
	\item $\rrule{monitor exit}{\s}{\M{l}{v}{\Kw{imm}\,l'}}{\s}{v}$, where $\C{l'} = \Kw{True}$:
	\begin{itemize}
		\item By \VS and our well-formedness requirements on method bodies, the monitor expression must have been introduced by \textsc{update}, \textsc{call mutator}, or \textsc{new}. In each case the third expression started off as $\invariant{l}$, and it has now (eventually) been reduced to $\Kw{imm}\,l'$, thus by \thm{Determinism} $l$ is \valid.
		
		\item As in the above case for \textsc{new}, since we didn't modify any memory, or remove any other monitors, by the inductive hypothesis every \reach $l' \neq l$ is still \valid or \mony.
	\end{itemize}

	\item (\textsc{access}, \textsc{call/call mutator}, \textsc{as}, \textsc{try enter}, and \textsc{try ok}) these are trivial:
	\begin{itemize}
		\item As in the above case for \textsc{new}, since these rules don't modify memory or remove monitors, by the inductive hypothesis, every \reach $l$ is still \valid or \mony.
	\qed\end{itemize}
\end{ienumerate}
\end{proof}

\subheading{Proof of Soundness}
First we need to prove that an object is not reachable from one of its \Q!imm! fields; if it were, \Q!invariant()! could access such a field and observe a potentially broken object:

\SS\begin{Lemma}[Imm Not Circular]\ \\
\indent If $\VS(\s, e)$, $\forall f,l$, if $\reach(\s, e, l)$, $\C{l}.f = \field{\Kw{imm}}{\_}{f}$, then $l \notin \rog(\s, \s[l.f])$.
\end{Lemma}\SS
\begin{proof}
The proof is by induction; obviously the property holds in the initial $\s|e$, since $\s = c\mapsto \Kw{Cap}\{\}$. Now suppose it holds in a $\VS(\s', e')$ where $\s'|e' \rightarrow \s|e$:
\begin{ienumerate}
	\item Consider any pre-existing \reach $l$ and $f$ with $\C[\s']{l}.f = \field{\Kw{imm}}{\_}{f}$, by \thm{Imm Consistency} and \thm{Non-Mutating}, the only way $\rog(\s, \s[l.f])$ could have changed is if $e' = \EV[\mdf\,l\D f = \mdf'\,l']$, where $\mdf \leq \Kw{mut}$, i.e. we just applied the \textsc{update} rule. By \thm{Type Consistency}, $\mdf' \leq \Kw{imm}$, so by \thm{Imm Consistency}, $l \notin \rog(\s, l')$. Since $l' = \s[l.f]$, we now have $l \notin \rog(\s, \s[l.f])$.
	\item The only rules that make an $l$ \reach are \textsc{new/new true}. So consider $e = \EV[\new{C}{\range{\_\,l}}]$, and each $i$ with $C.i = \field{\Kw{imm}}{\_}{f}$. But each of $\range{l}$ existed in the previous state and $l \notin \dom(\s')$; so by \VS and our reduction rules, $l \notin \rog(\s', l_i) = \rog(\s, \s[l.f])$.
\qed\end{ienumerate}
\end{proof}
\noindent Note that the above only applies to \Q!imm! \emph{fields}: \Q!imm! \emph{references} to cyclic objects can be created by promoting a \Q!mut! reference, however the cycle must pass through a \emph{field} declared as \Q!read! or \Q!mut!, but such fields cannot be referenced in the invariant method.

\LS

We can now finally prove the soundness of our invariant protocol:

\setcounter{theorem}{0}% We want to force it to use the original number
\SS\THMSoundness\SS
\setcounter{theorem}{3}% So further theorems are numbered appropriately

\begin{proof}
\noindent Suppose $\VS(\s, e)$, and $e = \ER[\_\,l]$. Suppose $l$ is not $\valid$; since $l$ is \reach, by \thm{Stronger Soundness}, $\mony(e,l)$, $e = \E[\M{l}{e_1}{e_2}]$, and either:
\begin{iitemize}
	\item $\ER = \E[\M{l}{\E'}{e_2}]$, that is $l$ was found inside of $e_1$, but by definition of $\ER$, we can't have $e_1 = \mdf\,l$, this contradicts the definition of \mony, or
	\item $\ER = \E[\M{l}{e_1}{\E'}]$, and thus $l$ was found inside $e_2$. By our reduction rules, all monitor expressions start with $e_2=\invariant{l}$; if this has yet to be reduced, then $\E' = \E''[\call{\hole}{\Kw{invariant}}{}]$, thus $\trusted(\ER, l)$. By our well-formedness rules for \Q!invariant()!, the next reduction step will be a \textsc{call}, $e_2$ will only contain $l$ as the receiver of a field access; so if we just performed said \textsc{call}, $\E' = \E''[\hole\D f]$: hence $\trusted(\ER, l)$. Otherwise, by \thm{Imm Not Circular}, \thm{Rep Field Soundness}, and \RCR, no further reductions of $e_2$ could have introduced an occurrence of $l$, so we must have that $l$ was introduced by the \textsc{call} to \Q!invariant()!, and so $\trusted(\ER, l)$.
\end{iitemize}
Thus either $l$ is $\valid$ or $\trusted(\ER, l)$.
\qed\end{proof}

\lstset{language=FortyThree} % Back to default

\section{The Hamster Example, in L42 and in Spec\#}
\label{s:hamster}

 
Suppose we have a \Q@Cage@ class which contains a \Q@Hamster@; the \Q@Cage@ will move its \Q@Hamster@ along a path. We would like to ensure that the \Q@Hamster@ does not deviate from the path. We can express this as the invariant of \Q@Cage@: the position of the \Q@Cage@'s \Q@Hamster@ must be within the path (stored as a field of \Q@Cage@).
This example is interesting since it relies on \Q@List@s and \Q@Point@ that are not designed with \Q@Hamster/Cage@s in mind.
We show a detailed analysis on our Spec\# implementation, showing all the effort we put in trying to find better solutions.

%
% While Spec\# requires specialised \Q@Point@, \Q@Hamster@, and \Q@Cage@ declarations to be able to enforce the invariant, our version manages to capture the required information in just a few annotations on \Q@Cage@ and leaves \Q@Point@ and \Q@Hamster@ unmodified.
%	if(that==null || !(that instanceof Point)){return false;}
% 	return ((Point)that).x==this.x && ((Point)that).y==this.y; 
%  }
\begin{lstlisting}
class Point { Double x; Double y; Point(Double x, Double y) {..}
  @Override read method Bool equals(read Object that) {
    return that instanceof Point &&
      this.x == ((Point)that).x && this.y == ((Point)that).y; }}
class Hamster {Point pos; //pos is imm by default
  Hamster(Point pos) {..}}
class Cage {
  capsule Hamster h;
  List<Point> path; //path is imm by default
  Cage(capsule Hamster h, List<Point> path) {..}
  read method Bool invariant() {
    return this.path.contains(this.h.pos); }
  mut method Void move() {
    Int index = 1 + this.path.indexOf(this.h.pos));
    this.moveTo(this.path.get(index % this.path.size())); }
  mut method Void moveTo(Point p) { this.h.pos = p; }}
\end{lstlisting}

We use the \Q@read@ annotation on the \Q@equals(that)@ method to express that it does not modify either its
receiver or its parameter. In \Q@Cage@ we use 
the \Q@capsule@ annotation to ensure
that the modification of the \Q@Hamster@'s ROG is fully under the control
of the containing \Q@Cage@. 
We annotated the \Q@move()@
and \Q@moveTo(p)@ methods with \Q@mut@, since they modify
their receivers' ROG. The default annotation is \Q@imm@, thus \Q@Cage@'s \Q@path@ field is a deeply immutable list of \Q@Point@s.
% Note how we just use \Q@List.contains()@ and \Q@List.indexOf()@
% to check if the hamster position is inside the list.
% The conventional syntax correctly instantiates a \Q@Cage@:
% \Q@new Cage(new Hamster(new Point(..)), List.of(new Point(...))@.
Our system performs runtime checks for the invariant
at the end of \Q@Cage@'s constructor, \Q@moveTo(p)@ method, and after any update to one of its fields.
The \Q@moveTo(p)@ method is the only one that may (directly) break the \Q@Cage@'s invariant. However, there is only a single occurrence of \Q@this@ and it is used to read the \Q@h@ field. We use the guarantees of RCs to ensure that no alias to \Q@this@ could be reachable from either \Q@h@ or the immutable \Q@Point@ parameter. Thus, the potentially broken \Q@this@ object is not visible while the \Q@Hamster@'s position is updated. 
The invariant is checked at the end of the \Q@moveTo(p)@ method, just before \Q@this@ would become visible again.
This technique loosely corresponds to an implicit pack and unpack: we `unpack' \Q@this@ before reading the field, then we work on the field's value while the invariant of \Q@this@ is not known to hold, finally when returning, we `pack' \Q!this! and check its invariant before allowing it to be used again.

Note: since only \Q@Cage@ has an invariant,
 only the code of \Q@Cage@ needs to be handled carefully; allowing the code for \Q@Point@ and \Q@Hamster@ to be unremarkable.
 This is not the case in Spec\#: all code involved in  verification needs to be designed with verification in mind~\cite{barnett2011specification}.
% The best solution we found was to define our own equality for \Q@Point@ instead of relying on \Q@Object.Equals@,
% thus we could not use \Q@List.Contains@ and \Q@List.IndexOf@.

\noindent We now show the hamster example in Spec\#; the system most similar to ours:
%or \small or \footnotesize etc.
\begin{lstlisting}[
language={[Sharp]C}, morekeywords={invariant,ensures,requires,expose,exists}]
// Note: assume everything is `public'
class Point { double x; double y; Point(double x, double y) {..}
  [Pure] bool Equal(double x, double y) {
    return x == this.x && y == this.y; }}
class Hamster{[Peer]Point pos; 
  Hamster([Captured]Point pos){..}}
class Cage {
  [Rep] Hamster h; [Rep, ElementsRep] List<Point> path;
  Cage([Captured] Hamster h, [Captured] List<Point> path)
    requires Owner.Same(Owner.ElementProxy(path), path); {
      this.h = h; this.path = path; base(); }
  invariant exists {int i in (0 : this.path.Count);
    this.path[i].Equal(this.h.pos.x, this.h.pos.y) };
  void Move() {
    int i = 0;
    while(i<path.Count && !path[i].Equal(h.pos.x,h.pos.y)){i++;}
    expose(this) {this.h.pos = this.path[i%this.path.Count];}}}
\end{lstlisting}

In both versions, we designed \Q@Point@ and \Q@Hamster@ in a general way, and not solely to be used by classes with an invariant: thus \Q@Point@ is not an immutable class. However, doing this in Spec\# proved difficult, in particular we were unable to override \Q@Object.Equals(that)@, or even define a usable \Q@equals(that)@ method that takes a \Q@Point@, as such we could not call either \Q@List<Point>.Contains(e)@ or \Q@List<Point>.IndexOf(e)@.
 
\noindent Even with all the above annotations, we needed special care in creating \Q@Cage@s:\vspace{-1.860px}% magic number that prevents the listings background going onto the next page
\begin{lstlisting}[
%basicstyle=\footnotesize,
language={[Sharp]C}, morekeywords={invariant,ensures,requires,expose,exists}]
List<Point> pl = new List<Point>{new Point(0,0),new Point(0,1)};
Owner.AssignSame(pl, Owner.ElementProxy(pl));
Cage c = new Cage(new Hamster(new Point(0, 0)), pl);
\end{lstlisting}

Whereas with our system we can simply write:
\begin{lstlisting}
List<Point> pl = List.of(new Point(0, 0), new Point(0, 1));
Cage c = new Cage(new Hamster(new Point(0, 0)), pl);
\end{lstlisting}

%3 read 2 capsule 3 mut extra method moveTo
%----
In Spec\# we had to add $10$ different annotations, of $8$ different kinds; some of which were quite involved. In comparison, our approach requires only $7$ simple keywords, of $3$ different kinds; however we needed to write 
a separate \Q@moveTo(p)@ method, since we do not want to burden our language with extra constructs such as Spec\#'s \Q@expose@.
%  Moreover we had been unable to reuse 
% \Q@Object.Equals@, \Q@List.IndexOf@ and % \Q@List.Contains@.
% Note: we had to add a new class \Q@PureObject@, since the \Q@Objec@ constructor is not annotated as \Q@[Pure]@.
%3 pure,
%1 peer
%3 captured
%2 rep
%1 ElementsRep
%1 requires Owner.Same(Owner.ElementProxy(path), path);
%1 invariant
%1 exists
%expose(this)
%re implementation of indexOf
%dumb equals(double,double)
%dumb class PureObject { [Pure] PureObject() { } }
%Owner.AssignSame(pl, Owner.ElementProxy(pl));
% manually handle ownership details while instantiating a \Q@new Cage(..)@.
% Note how the \Q@expose@ block cover plays the same role of our \Q@moveTo@ method.

%We evaluate our contribution by means of case studies;


\lstset{language={[Sharp]C}, morekeywords={invariant,ensures,requires,expose,exists,capsule}}

\subheading{Is that really the best way to use Spec\#?}  
Here we describe exactly why encoded the
hamster example in the way we did. For brevity, we will assume the default accessibility is \Q@public@, whilst in both Spec\# and C\#, it is actually \Q@private@.

\subheading{The \Q@Point@ Class} 
The typical way of writing a \Q@Point@ class in C\# is as follows:
\begin{lstlisting}
class Point {
	double x, y;
	Point(double x, double y) { this.x = x; this.y = y; }
}
\end{lstlisting}

This works exactly as is in Spec\#, however we have difficulty if we want to define equality of \Q@Point@s (see below).

\subheading{The \Q@Hamster@ Class} 
The \Q@Hamster@ class in C\# would simply be:
\begin{lstlisting}
class Hamster {
	Point pos;
	Hamster(Point pos) { this.pos = pos; }
}
\end{lstlisting}

Though this is legal in Spec\#, it is practically useless. Spec\# has no way of knowing whether \Q@pos@ is \emph{valid} or \emph{consistent}. If \Q@pos@ is not known to be valid, one will be unable to pass it to almost any method, since by default methods implicitly require their receivers and arguments to be valid (compare this with our invariant protocol, which guarantees that any reachable object is valid).
If \Q@pos@ is not known to be consistent, one will be unable to mutate it, by updating one of its fields or by passing it as an argument (or receiver) to a non-\Q@Pure@ method.
Though we don't want \Q@pos@ to ever mutate, Spec\# currently has no way of enforcing that an \emph{instance} of a non-immutable class is itself immutable\footnote{There is a paper~\cite{DBLP:conf/vstte/LeinoMW08} that describes a simple solution to this problem: assign ownership of the object to a special predefined `freezer' object, which never gives up mutation permission, however this does not appear to have been implemented; this would provide similar flexibility to the RC system we use, which allows an initially mutable object to be promoted to immutable.}, as such we will simply refrain from mutating it.

To enable Spec\# to reason about \Q@pos@'s validity, we will require that it be a \emph{peer} of the enclosing \Q@Hamster@; we can do this by annotating \Q@pos@ with \Q@[Peer]@. Peers are objects that have the same owner, implying that  whenever one is valid and/or consistent, the other one also is. This means that if we have a \Q@Hamster@, we can use its \Q@pos@, in the same ways as we could use the \Q@Hamster@.

To simplify instantiation of \Q@Hamster@s, their constructors will take unowned \Q@Point@s; Spec\# will then automatically make such \Q@Point@ a peer. This is achieved by taking a \Q@[Captured]@ \Q@Point@ in the constructor (note how similar this is to taking a \Q@capsule@ \Q@Point@). Note that unlike our system, this prevents multiple \Q@Hamster@s from sharing the same \Q@Point@, unless both \Q@Hamster@s have the same owner, if \Q@Point@ were immutable, there would be no such restriction.

With the aforementioned modifications, the \Q@Hamster@ becomes:
\begin{lstlisting}
class Hamster {
  [Peer] Point pos;
  Hamster([Captured] Point pos) { this.pos = pos; }
}
\end{lstlisting}

%If however, we did want \Q@Point@ to be an immutable/value type, the original unannotated version would not have any problems.

\subheading{The \Q@Cage@ Class} 
The natural way to write this class in C\#, if it had native support for class invariants like Spec\#, would be:
\begin{lstlisting}
class Cage {
  Hamster h;
  List<Point> path;
  Cage(Hamster h, List<Point> path){this.h=h; this.path=path;}
  invariant this.path.Contains(this.h.pos);
  void Move() { 
    int index = this.path.IndexOf(this.h.pos);
    this.h.pos = this.path[index % this.path.Count]; } 
}
\end{lstlisting}

However for the above \Q@invariant@ to be admissible in Spec\#, \Q@this.path@ and \Q@this.h@ must both be owned by \Q@this@. In addition, the \emph{elements} of \Q@this.path@ need to be owned by \Q@this@, since \Q@this.path.Conatains@ will read them. Note that \Q@this.h.pos@ also needs to be owned by \Q@this@, however since \Q@pos@ is declared as \Q@[Peer]@, if \Q@this@ owns \Q@this.h@, it also owns \Q@this.h.pos@. To fix the invariant, we will declare \Q@h@, \Q@path@, and the elements of \Q@path@ as \emph{reps} (i.e. they are owned by the containing object). Finally, since \Q@Move@ modifies \Q@this.h@, \Q@this.h@ needs to be made consistent, which requires that the owner (\Q@this@) be made invalid; this can be achieved by using an \Q@expose(this)@ statement. \Q@expose(this){@\emph{body}\Q@}@ marks \Q@this@ as invalid, executes \emph{body}, checks that the invariant of \Q@this@ holds, and then marks \Q@this@ valid again.
As we did with the \Q@Hamster@, we will simply take unowned \Q@h@ and \Q@path@ values, however we also need the elements of \Q@path@ to be unowned; since Spec\# has no \Q@[ElementsCaptured]@ annotation, we will require \Q@path@ to be unowned, and its elements (denoted by \Q@Owner.ElementProxy(path)@) to be owned by the same owner as \Q@path@ (which is \Q@null@).
\begin{lstlisting}
class Cage {
  [Rep] public Hamster h;
  [Rep, ElementsRep] List<Point> path;
	
  Cage([Captured] Hamster h, [Captured] List<Point> path)
    requires Owner.Same(Owner.ElementProxy(path), path);
  { this.h = h; this.path = path; }
	
  invariant this.path.Contains(this.h.pos);
  void Move() { 
    int index = this.path.IndexOf(this.h.pos);
    expose(this){this.h.pos=this.path[index%this.path.Count]; }} 
}
\end{lstlisting}

The above constructor now fails to verify, since Boogie is unconvinced that its pre-condition actually holds when we initialise \Q@this.path@. This is because the constructor for \Q@Object@ (the default base class if none is provided) is not marked as \Q@[Pure]@; since it is (implicitly) called upon entry to \Q@Cage@'s constructor, Boogie has no idea as to what memory could've mutated, and so it doesn't know whether the pre-condition still holds. The solution is to explicitly call it, but at the end of the constructor: \Q@{this.h = h; this.path = path; base();}@.

The above \Q@Cage@ code however does not work, since \Q@List@ operations, such as \Q@Contains@ and \Q@IndexOf@, will call the virtual \Q@Object.Equals@ method to compute equality of \Q@Point@s. However \Q@Object.Equals@ implements \emph{reference} equality, whereas we want \emph{value} equality.

\subheading{Defining Equality of \Q@Point@s}
The obvious solution in C\# is to just override \Q@Object.Equals@ accordingly, and let dynamic dispatch handle the rest:
\begin{lstlisting}
class Point {
  .. // as before
  override bool Equals(Object? o) {
    Point? that = o as Point;
    return that!=null && this.x == that.x && this.y == that.y;}
}
\end{lstlisting}

However this fails in Spec\# since \Q@Object.Equals@ is annotated with \Q@[Pure]@\\*\Q@[Reads(ReadsAttribute.Reads.Nothing)]@, and of course every overload of it must also satisfy this. The \Q@Reads@ annotations specifies that the method cannot read fields of \emph{any} object, not even the receiver, this makes overloading the method useless.
% Our best guess as to why \Q@Object.Equals@ is annotated like that is that they expect it to be the default reference-equality, annotating it like this could aid static verification as it implies that whether or not two objects are equal cannot change, even if their fields are modified.

We resorted to making our own \Q@Equal@ method. Since it is called in \Q@Cage@'s invariant, Spec\# requires it to be annotated as \Q@[Pure]@, and either annotated with \Q@[Reads(ReadsAttribute.Reads.Nothing)]@ or \\* \Q@[Reads(ReadsAttribute.Reads.Owned)]@ (the default, if the method is \Q@[Pure]@). The latter annotation means it can only read fields of objects owned by the \emph{receiver} of the method, so a \Q@[Pure] bool Equal(Point that)@ method can read the fields of \Q@this@, but not the fields of \Q@that@. Of course this would make the method unusable in \Q@Cage@ since the \Q@Point@s we are comparing equality against do not own each other. As such, the simplest solution is to just pass the fields of the other point to the method:
\begin{lstlisting}
[Pure] bool Equal(double x, double y) {
  return x == this.x && y == this.y;}
\end{lstlisting}

Sadly however this mean we can no longer use \Q@List@'s \Q@Contains@ and \Q@IndexOf@ methods, rather we have to expand out their code manually.
\lstset{language=FortyTwo}

%\subsection{Expressiveness}
%\noindent\textit{Expressiveness:}
%Finally, in our this third case study we 
%shown that even if we do not aim to expressiveness, but to simplicity, soundness and efficiency, we are still able to express a reasonable amount of cases.
%We encoded in L42 all the examples present in papers~\cite{??}.
%We can express all the examples except ....
%Again, we quantify the annotation burden and we discover....

%\subsection{The transform pattern}


%\begin{lstlisting}[escapechar=\%]
%class List {
%  mut List prev;
%  mut List next;
%  Object elem;  
%  read method Bool ok() {
%    return this.next.prev==this && this.prev.next==this &&..;}
%  read method Int size(){
%    if(next==this){return 1;} return next.size()+1;}}
%\end{lstlisting}
%Clearly the \Q@mut@ fields of \Q@List@ cannot be marked as \Q@capsule@.
%However, only \Q@capsule@ and \Q@imm@
%fields can be accessed in \validate.
%Thus, \Q@.innerValidate()@ can not be the \Q@invariant@ method for \Q@List@.
%The solution is to use a `box' over our \Q@List@, and to validate our `box':

%\loseSpace\loseSpace\loseSpace
%\begin{lstlisting}[escapechar=\%]
%class ListBox { 
%  capsule List inner;
%  read method imm Bool invariant() {return this.inner.ok();}
%  read method Int size(){ return this.inner.size();}
%\end{lstlisting}
%\saveSpace
%Encoding this example in Spec\# would be much more verbose (see case study XX) and still require
%a \Q@ListBox@ object,
%while the visible state semantic of Eiffel or D would cause an large amount of \Q@invariant@ checks
%if the list had any recursive method; consider for example the \Q@size@ method:
%if there was an invariant with visible state semantic on \Q@List@, calling \Q@List.size()@ would require 
%calling \Q@List.invariant()@ before and after the method execution. If the list has more then one element, the recursive \Q@size@ call would also call the invariant twice.
%We would also want to create forwarding methods in \Q@ListBox@ for all public methods defined in \Q@List@. This approach allows the validation of many interesting and practically useful data-structures.
%However the limitations of capsule mutator methods mean that any \Q@mut@ methods in \Q@ListBox@ taking \Q@read@ or \Q@mut@ parameters, or returning \Q@mut@, cannot be trivially forwarded.
%% since they necessitate mutating a \Q@capsule@, instead complicated and involved forwarding would be needed, if it is even possible.
%Our example is about a list of immutable objects.
%To instead validate a list of \Q@mut@ objects we would need to use our box pattern not just around the list,
%but around a section of data encapsulating both the list and all the contained elements.
%This is because our simple \Q@capsule@ modifier requires the whole ROG to be encapsulated.
%Conceptually, it would be better for the list (of mutable objects) to be validated by its
%head, since the behaviour of the contained objects is transparent to the validation criteria. 
%Our limitation relates to full encapsulation and contrasts with flexible encapsulation as in 
%ownership~\cite{ClarkeEtAl98}. However, neither traditional flexible encapsulation/ownership, nor our language are capable of verifying that \Q@List.elem@ is not (indirectly) referenced within \Q@ListBox.validate()@.


%\subsection{Family, a worst case scenario for L42}
%\noindent\textit{Family, a worst case scenario for L42}
%For our second case study, we wished to make an example where the performance of L42 and the conventional approach was similar. We forged an example when a \Q@Family@ has a list of parents and a list of children;
%all the children need to be younger then their parents and every \Q@Person@ need to have a non empty name and a positive age.  
%We model the pass of time with a \Q@processDay@ method, and we simulate $3$ years of life (that is, $3\times365$ days) of a family of $4$.
%The age of a \Q@Person@ grow when its birthday is processed.
%Notably, \Q@processDay@ is a \Q@mut@ method that can potentially mutate any person in the system, thus
%L42 have to run a lot of invariant checks. The object graph here is very shallow: the \Q@Family@ holds the \Q@Person@s and that is it.
%However, even in this case we get about $9$ times less invariant calls: $19403$ with visible state semantic  and $2210$ in L42.
%Also the \Q@Family@ example uses the box pattern.

%\subsection{Family}
%We wished to make an example where the performance of L42 and the conventional approach 
%was similar. We forged an example when a Family has a list of parents and a list of children;
%all the childrens need to be younger then their parents and every Person need to have a non empty name and a positive age.  
%We model the pass of time with a \Q@processDay@ method, and we simulate 3 years of life (that is, 3*365 days) of a family of 4.
%The age of a Person grow when its birthday is processed.
%Notably, \Q@processDay@ is a mut method that can potentially mutate any person in the system, thus
%L42 have to run a lot of invariant checks. The object graph here is very shallow: the Family holds the Persons and that is it.
%However, even in this case we get about 10 times less invariant calls: Num in the conventional approach and Num in L42.

%\subsection{Spec\# 2papers}
%Our goal in this third case study was to show that even if we do not aim to expressiveness, but to simplicity, soudness and efficiency, we are still able to express a reasonable amount of cases.
%We can express all the examples except ....
%Again, we quantify the annotation burden and we discover....

%\section{Stack overflow and Out of memory}
%For our system to be sound,
%Stack overflow and Out of memory errors need to be modeled specially.
%If they are just (unchecked) exceptions then they could be catched to 
%generate non deterministic behaviour inside invariant code.

%However, it is possible to use capaility objects to capture them as special system events/signals.
%In this way we can maintain the soundness of our system even in this corner case.
%Of course, another option would be to make them into unrecoverable fatal errors.

\section{More Case Studies}

\subsection{Family}
The following test case was designed to produce a worst case in the number of invariant checks. We have a \Q!Family! that (indirectly) contains a list of \Q!parents! and \Q!children!. The \Q!parents! and \Q!children! are of type \Q!Person!. Both \Q!Family! and \Q!Person! have an invariant, the invariant of \Q!Family! depends on its contained \Q!Person!s.

% TODO: Talk to mark about code style

% TODO: Swap parent/child updates in aritifact
\begin{lstlisting}
class Person { 
  final String name;
  Int daysLived;
  final Int birthday;
  Person(String name, Int daysLived, Int birthday) { .. }
  mut method Void processDay(Int dayOfYear) {
  	this.daysLived += 1;
    if (this.birthday == dayOfYear) {
    	Console.print("Happy birthday " + this.name + "!");}}
  read method Bool invariant() {
    return !this.name.equals("") && this.daysLived >= 0 &&
      this.birthday >= 0 && this.birthday < 365; }
}
class Family { 
  static class Box { 
    mut List<Person> parents;
    mut List<Person> children;
    Box(mut List<Person> parents, mut List<Person> children){..}
    mut method Void processDay(Int dayOfYear) {
      for(Person c : this.children){c.processDay(dayOfYear);}
      for(Person p : this.parents){p.processDay(dayOfYear);}
    }
  }
  capsule Box box;
  Family(capsule List<Person> ps,capsule List<Person> cs) {
    this.box = new Box(ps, cs); }
  mut method Void processDay(Int dayOfYear) { 
    this.box.processDay(dayOfYear); }
  mut method Void addChild(capsule Person child) { 
    this.box.children.add(child); }
  read method Bool invariant() {
    for (Person p : this.box.parents) {
      for (Person c : this.box.children) {
        if (p.daysLived <= c.daysLived) { 
          return false; }}}
    return true; }
}
\end{lstlisting}
Note how we created a \Q!Box! class to hold the \Q!parents! and \Q!children!.
Thanks to this pattern, the invariant only needs to hold at the end of \Q!Family.processDay!, after all the \Q!parents! and \Q!children! have been updated. Thus \Q!Family.processDay! is atomic: it updates all its contained \Q!Person!s together.
%This capture the intention of consistently calling \Q@Person.processDay@ once for all the persons as an atomic operation.
%This capture the intention of atomically and consistently calling \Q@Person.processDay@ once for all the persons.
Had we instead made the \Q!parents! and \Q!children! \Q!capsule! fields of \Q!Family!, the invariant would be required to also hold between modifying the two lists. This could cause problems if, for example, a child was updated before their parent.

\noindent We have a simple test case that calls \Q!processDay! on a \Q!Family!, $1{,}095$ ($3\times365$) times.
%$3\times365$ times ($1{,}095$ days):
\begin{lstlisting}
// 2 parents (one 32, the other 34), and no children
var fam = new Family(List.of(new Person("Bob", 11720, 40),
    new Person("Alice", 12497, 87)), List.of());
    
for (Int day = 0; day < 365; day++) { // Run for 1 year
  fam.processDay(day);
}
for (Int day = 0; day < 365; day++) { // The next year
  fam.processDay(day);
  if (day == 45) {
    fam.addChild(new Person("Tim", 0, day)); }}

for (Int day = 0; r < 365; day++) { // The 3rd year
  fam.processDay(day);
  if (day == 340) {
    fam.addChild(new Person("Diana", 0, day)); }}
\end{lstlisting}
% The counts (including the invariant keyword, and read on the invariant method)
% Spec# 14 family, 2 main   = 16
% L42 	12 family, 1 person = 13
% Fake 42 = 14 (+2 for box ctor, -1 for family ctor)

The idea is that everything we do with the \Q!Family! is a mutation, in addition the \Q!fam.processDay! calls also mutate the contained \Q!Person!s.

This is a worst case scenario for our approach compared to visible state semantics since it reduces our advantages:
our approach avoids invariant checks when objects are not mutated
but in this example most operations are mutations; 
similarly, our approach prevents the exponential explosion of nested invariant checks\footnote{see section \ref{s:case-study}} when deep object graphs are involved, but in this example the object graph of \Q!fam! is very shallow.
\loseSpace

We ran this test case using several different languages: L42 (using our protocol) performs $4{,}000$ checks, D performs $7{,}995$, Eiffel performs $19{,}335$, and finally, Spec\# performs only $1{,}104$.

Our protocol performs a single invariant check at the end of each constructor,  \Q!processDay! and \Q!addChild! call (for both \Q!Person! and \Q!Family!). 

The visible state semantics of both D and Eiffel perform additional invariant checks at the beginning of each call to \Q!processDay! and \Q!addChild!. In addition, due to Eiffel's `uniform access principle', the field accesses of \Q!Person.daysLived! (in  \Q!Family!'s invariant) are not distinguished from ordinary method calls, thus each one performs two invariant checks.
% TODO: Actually implement it in eiffel (this will be a nightmare...)

The results for Spec\# are very interesting, since it performs less checks than L42.
This is the case since \Q!processDay! in \Q!Person! just does a simple field update, which in Spec\# do not invoke runtime invariant checks. Instead, Spec\# tries to statically verify that the update cannot break the invariant, if it is unable to do this it requires that the update be wrapped in an \Q!expose! block. In Spec\#, integer arithmetic is not allowed to overflow\footnote{A compilation option allows overflow to be check at runtime, by throwing  where unchecked exceptions (just as invariant failures) if it occurs}, as such it verifies that the field increment in \Q!processDay! cannot break the invariant.

%This is the case since \Q!processDay! in \Q!Person! just does a simple field increment, thus the Spec\# verifier is able to statically verify that this wont break the invariant, and so it does not require a corresponding \Q!expose! block, and hence does not perform a runtime invariant check.
%The Spec\# verifier is able to do this as it works on a language semantic where arithmetic overflow does not occur. Such semantic can be enforced by
%a compilation option (disable on default for performance reasons).
%%%%however one can turn on runtime checking for overflow.
%%%and check overflow errors at run time
%With this option turned on, eliding the invariant check is sound since overflow will have the same result as a runtime invariant check failure, namely it will throw an unchecked exception.

% Concluding Spec\# is able to replaces some runtime invariant checks with more efficient runtime overflow checks.

%This static reasoning is performed under the assumption that arithmetic overflow will not occur, thus Spec\# is considering a different semantic for \Q@Int@ then L42. Spec\# can inject run-time checks to enforce such arithmetic semantic.


The annotation difficulty with Spec\#\footnote{see the artifact for the full code} was similar to our previous examples, however since the fields of \Q!Person! all have immutable classes/types, we did not have to have any special annotations for that class. The \Q!Family! class was similar to our \Q!Cage! example (see section \ref{s:intro}), however in order to implement the \Q!addChild! method we were forced to do a shallow clone of the new child (this also caused a couple of extra runtime invariant checks).

\section{Patterns}
\label{s:patterns}
\lstset{morekeywords={invalid}}
In Section~\ref{s:case-study} and Appendix~\ref{s:MoreCaseStudies} we showed how the box pattern can be used to write invariants over cyclic mutable object graphs, the latter also shows how a complex mutation can be done in an `atomic' way, with a single invariant check. However the box pattern is much more powerful. Suppose we want to pass a temporarily `broken' object to other code as well as perform multiple field updates with a single invariant check. 
Instead of adding new features to the language, like an \Q!invalid! TM (denoting an object whose invariant need not hold), and an \Q!expose! statement like Spec\#, we can use a `box' class and a capsule mutator to the same effect:
\begin{lstlisting}
interface Person {
  mut method Bool accept(read Account a, read Transaction t); }

interface Transaction { 
  // Here ImmList<T> represents a list of immutable Ts.
  mut method ImmList<Transfer> compute(); }

class Transfer { Int money;
  // An `AccountBox' is like an `invalid Account':
  //   `that' need not have income > expenses
  method Void execute(mut AccountBox that) {
    // Gain some money, or lose some money
    if (this.money > 0) { that.income += money; }
    else { that.expenses -= money; }}}

class AccountBox { UInt income = 0; UInt expenses = 0; }
class Account {
  capsule AccountBox box; mut Person holder;
  read method Bool invariant() {
    return this.box.income > this.box.expenses; }

  // `h' could be aliased elsewehere in the program    
  Account(mut Person h) { 
    this.holder = h; this.box = new AccountBox(); }

  mut method Void transfer(mut Transaction ts) {
    if (this.holder.accept(this, ts)) {
	  this.transferInner(ts.compute()); }}

  // capsule mutator, like an `expose(this)' statement
  private mut method Void transferInner(ImmList<Transfer> ts) {
     mut AccountBox b = this.box;
     for (Transfer t : ts) { t.execute(b); }
     // check the invariant here
}}
\end{lstlisting}
The idea here is that \Q!transfer(ts)! will first check to see if the account holder wishes to accept the transaction, it will then compute the full transaction (which could cache the result and/or do some I/O), and then execute each transfer in the transaction. We specifically want to allow an individual \Q!Transfer! to raise the \Q!expenses! field by more than the \Q!income!, however we don't want an entire \Q!Transaction! to do this. 
Our capsule mutator (\Q!transferInner!) allows this by behaving like a Spec\# \Q!expose! block: during its body (the \Q!for! loop) we don't know or care if \Q!this.invariant()! is \Q!true!, but at the end it will be checked. For this to make sense, we make \Q!Transfer.execute! take an \Q!AccountBox! instead of an \Q!Account!: it cannot assume that the invariant of \Q!Account! holds, and it is allowed to modify the fields of \Q!that! without needing to check it. As you can see, adding support for features like \Q!invalid! and \Q!expose! is unnecessary, and would likely require making the type system significantly more complicated as well as burdening the language with more core syntactic forms.

In particular, the above code demonstrates that our system can:
\SSI\begin{itemize}
\item Have useful objects that are not entirely encapsulated: the \Q!Person holder! is a \Q!mut! field; this is fine since it is not mentioned in the \Q!invariant! method.
\item Perform multiple state updates with only a single invariant check: the loop in \Q!transferInner! can perform multiple field updates of \Q!income! and \Q!expenses!, however the \Q!invariant! will only be checked at the end of the loop.
\item Temporarily break an invariant: it is fine if during the \Q!for! loop, \Q!expenses > income!, provided that this is fixed before the end of the loop.
\item Pass the state of an `invalid' object around, in a safe manner: an \Q!AccountBox! contains the state of \Q!Account!, but not its invariant: if you have an \Q!Account!, you can be sure that its \Q!income > expenses!, but not if you have an \Q!AccountBox!.
\item Wrap normal methods over capsule mutators: \Q!transfer! is not a capsule mutator, so it can use \Q!this! multiple times and take a \Q!mut! parameter.
\end{itemize}

\noindent Though capsule mutators can be used to perform batch operations like the above, they can only take immutable and capsule objects. This means that they can perform no non-deterministic I/O (due to our OC system), and other externally accessible objects (such as a \Q!mut Transaction!) cannot be mutated during such a batch operation.

\subheading{The Transform Pattern}
Recall the GUI case study in Section~\ref{s:case-study}, where we had a \Q!Widget! interface and a \Q!SafeMovable! (with an invariant) that implements \Q!Widget!.
% A capsule mutator method is essentially a mutation of a field, which is guaranteed to not see the \Q@this@ object.
% Thus, if \Q@this@ is made invalid during  the method's execution, we could not observe it until after the method completes.
Suppose we want to allow \Q@Widget@s to be scaled, we could add \Q@mut@ setters for \Q@width@, \Q@height@, \Q@left@, and \Q@top@ in the \Q@Widget@ interface. However, if we also wish to scale its children we have a problem, since \Q@Widget.children@ returns a \Q@read Widgets@, which does not allow mutation. We could of course add a \Q@mut@ method \Q@zoom@ to the \Q@Widget@ interface, however this does not scale if more operations are desired. If instead \Q@Widget.children@ returned a \Q@mut Widgets@, it would be difficult for \Q@Widget@ implementations, such as \Q@SafeMovable@, 
to mention their \Q!children! in their \Q!invariant!.

% In the above \Q@SafeMovable@ we only had one capsule mutator: \Q@dispatch@. However suppose a \Q@Widget@ wants to directly mutate it's descendents, however it can't do that since \Q@Widget.children@ returns a \Q@read Widgets@, if it returned a \Q@mut Widgets@ then \Q@SafeMovable@ could not be implement, as it's children are contained inside a capsule-field. 
% At first glance, it may seem that capsule mutators allow only very limited kinds %of mutation.
% This is however not the case. 
% Consider the following
% simple pattern to allow flexible use of encapsulated content: define a

A simple and practical solution would be to define a \Q@transform@ method in \Q@Widget@, and a \Q@Transformer@ interface 
like so:\footnote{A more general transformer could return a generic \Q@read R@.}
\begin{lstlisting}[escapechar=\%]
interface Transformer<T> { method Void apply(mut T elem); }
interface Widget { ...
  mut method Void top(Int that); // setter for immutable data
  // transformer for possibly encapsulated data
  mut method read Void transform(Transformer<Widgets> t);
}

class SafeMovable { ...
  // A well typed capsule mutator
  mut method Void transform(Transformer<Widgets> t) {
    t.apply(this.box.c); }}
\end{lstlisting}
% Note that the code above does not access a \Q!capsule! field but merely calls a method that does; thus  it is \emph{not} a capsule mutator method, so it is not constrained by the restrictions on them. Code like the above would also allow one to mutate multiple \Q!capsule! fields in one method.
%Our pattern cooperates with the language’s restrictions to ensure each mutation is completed as a separate operation, that is perceived by the rest of the system %as if it was atomic.%
%,  i.e. they can't see or update the other \Q!capsule! fields.
The \Q@transform@ method offers an expressive power similar to \Q@mut@ getters, but prevents \Q@Widgets@ from leaking out.  With a \Q@Transformer@, a \Q@zoom@ function could be simply written as:
\begin{lstlisting}
static method Void zoom(mut Widget w) {
  w.transform(ws -> { for (wi : ws) { zoom(wi,scale); }});
  w.width(w.width() / 2); ...; w.top(w.top() / 2); }
\end{lstlisting}

% One of the advantages of this approach is that a the \@zoom@ method can be written by anyone anywhere

% \begin{lstlisting}[escapechar=\%]
%// Lambda Expression that creates a new Transformer<...>
%this.transform(l -> l.add(new MyWidget(..)))
%\end{lstlisting}
%//`i' is captured by the closure.
%// `imm' and `capsule' varaibles can be captured.

%    %\Comment{}%this.items.add(i);
%    // Cant instead capture `this': it can't be typed %as `imm'
%    // since `ItemTransformer.transform()' is an %`imm' method
%  })
%}
%  // instead of:
%\Comment{}%this.exposeItems().add(i)

%Note that the code above does not access a \Q!capsule! field but merely calls a method that does; thus
%it is \emph{not} a capsule mutator method, so it is not constrained by the restrictions on them. Code like the above would also allow one to mutate multiple \Q!capsule! fields in one method.
%Our pattern cooperates with the language’s restrictions to ensure each mutation is completed as a separate operation, that is perceived by the rest of the system
%as if it was atomic.%
%,  i.e. they can't see or update the other \Q!capsule! fields.

\section{Related Work on Runtime Verification Tools}
\label{s:runtime-verification}
By looking to a survey by Voigt \etal~\cite{Voigt2013} and the extensive MOP project~\cite{meredith2012overview},
it seems that most runtime verification tools (RV) empower users
to implement the kind of monitoring they see fit for their specific problem at hand. This means that users are responsible for deciding, designing, and encoding both the logical properties and the instrumentation criteria~\cite{meredith2012overview}.
In the context of class invariants, this means the user defines the invariant protocol and the soundness of such protocol is not checked by the tool.

In practice, this means that the logic, instrumentation, and implementation end up connected:
a specific instrumentation strategy is only good to test certain logic properties in certain applications.
No guarantee is given that the implemented instrumentation strategy is able to support the required logic in the monitored application.
Some of these tools are designed to support class invariants: for example InvTS~\cite{gorbovitski08efficient} lets you write Python conditions that are verified on a set of Python objects, but the programmer needs to be able
to predict which objects are in need of being checked and to use a simpler domain specific language to target them. Hence if a programmer makes a mistake while using this domain specific language, invariant checking
will not be triggered.
Some tools are intentionally unsound and just perform invariant checking following some heuristic that is expected to catch most failures: such as jmlrac~\cite{Burdy2005} and Microsoft Code Contracts~\cite{fahndrich2010embedded}.

%In particular, the heuristic of 
%We encoded our GUI example also on Microsoft Code Contract; their system also ran the invariant checking $77$ times. Their system is easy to use, but it is unsound since it is built over an unsound/incomplete static verifier~\cite{?}.






%
%In this work we define a language where a minimal, standardized,
%efficient and completely general purpose instrumentation strategy can soundly verify conditions
%expressible as a\\* \Q@read method imm Bool invariant()@, for any well-typed program; with open world assumption
%and possible Byzantine behaviour of any object in the system.
%
%By seeing class invariant as a part of the type of the object,
%the `RV tool' philosophy is akin to letting the programmer customize the behaviour of the
%type system: the programmer implementation may be unsound; while our philosophy is
%to give the user a way to represent complex and expressive types (in the form of arbitrary code in 
%the \Q@invariant()@ method), but 
%the type system implementation is fixed in stone by the language designer.

Many works attempt to move out of the `RV tool' philosophy to ensure RV monitors work as expected, as for example
%\sepItems
%In avionics, where memory allocation is disallowed, making reasoning about aliasing much simpler~\cite{laurent2015assuring}:
%``\emph{Runtime Verification (RV) can act as the last line of defense to
%protect the public safety, but only if the RV system itself is trusted.}''.
%\sepItems
%In domain specific languages~\cite{ferrari2002guardians}:
%``\emph{Proof techniques for establishing security properties}''.
%\sepItems
%On assertions over restrictive domain specific languages, to tame some of the C/C++
%undefined behaviour~\cite{agten2015sound}:
%``\emph{no verified assertion in the verified
%module will ever fail at runtime, even if the module runs as part of
%a vulnerable application thSound and Unsound monitorsat is subject to code injection attacks}''.
the study of contracts as refinements of types~\cite{findler2001contract}.
However, such work is only interested in pre and post-conditions, not invariants.

Our invariant protocol is much stronger than
visible state semantics, and keeps the invariant under tight control.
Gopinathan \etal's.~\cite{Gopinathan:2008:RMO:1483018.1483028} approach keeps
a similar level of control:
relying on powerful aspect-oriented support, they detect any field update in the whole ROG of any object, and check all the invariants that such update may have violated.
We agree with their criticism of visible state semantics, where  methods still have to assume that any object may be broken; in such case calling any public method would trigger an error, but while the object is just passed around (and for example stored in collections), the broken state will not be detected; Gopinathan \etal says ``\emph{there are many instances where $o$'s invariant is violated by the programmer inadvertently changing the state of $p$ when $o$ is in a steady state. Typically, $o$ and $p$ are objects exposed by the API, and the programmer (who is the user of the API), unaware of the dependency between $o$ and $p$, calls a method of $p$ in such a way that $o$'s invariant is violated. The fact that the violation occurred is detected much later, when a method of $o$ is called again, and it is difficult to determine exactly where such violations occur.}''

However, their approach addresses neither exceptions nor non-determinism caused by I/O, so their work is unsound if those aspects are taken into consideration.

Their approach is very computationally intensive, but we think it is powerful enough that it could even be used to roll back the very field update that caused the invariant to fail, making the object valid again.
We considered a rollback approach for our work, however rolling back a single field update is likely to be completely unexpected, rather we should roll back more meaningful operations, similarly to what happens
with transactional memory, and so is likely to be very hard to support efficiently.
%However we think roll-back this would be a 
%\REVComm{\REVComm{terrible}{2}{It seems in poor taste to complain of ``terrible'' ideas, especially without attempting to demonstrate the improvements of the proposed approach.}}{3}{Nontechnical term. It is not a great idea to label previous work as ``terrible''}
% ideally not only the field-update breaking the invariant should be reverted, %the roll-back should 
Using RCs to enforce strong exception safety is a much simpler alternative, providing the same level of safety, albeit being more restrictive (namely that if the operation did succeed it is still effectively rolled back).

%: for example
%assume that we are moving object between two boats:
%the overflowing object may be removed from the \Q@cargo@ of the second boat, but it would not
%be placed back in the first boat. It would look like the object has disappeared.
%The important pTheir approach is very computationally intensive, but we think it is powerful enough that it could even be used to roll-back the very field update that caused oint here is that the program would be in an unexpected state
%even if no object invariants are violated, and this would happen \textbf{because} of the 
%invariant checking/fixing behaviour, not because of code written by the programmer.
%We believe that the only viable option is to detect violations after the fact.

%\LINE
%Another approach used in the dynamic language Racket is to interpose on primitive operations like procedure-calls and field updates; this allows one to enforce visible-state semantics by wrapping invariant operations around such operations (as is done in aspect-oriented systems like Jose~{?}). This technique can be used with gradual typing to dynamically enforce `types' of mutable structures in a safe way.

%\LINE
%\subheading{Chaperones and impersonators}
Chaperones and impersonators~\cite{DBLP:conf/oopsla/StricklandTFF12} lifts the techniques of gradual typing~\cite{takikawa2015towards,DBLP:conf/oopsla/TakikawaSDTF12,DBLP:conf/popl/WrigstadNLOV10}
to work on general purpose predicates, where
values can be wrapped to ensure an invariant holds.
This technique is very powerful and can be used to enforce pre and post-conditions by wrapping function arguments and return values.
This technique however does not monitor the effects of aliasing, as such they may notice if a contract has been broken, but not when or why. In addition, due to the difficulty of performing static analysis in weakly typed languages, they need to inject runtime checking code around every user-facing operation.
Aspect oriented systems like Jose~\cite{feldman2006jose}, similarly wrap invariant checks around method bodies.
%
%One of the advantages of their system is that is transparent to the user. 
%\LINE
%
%\LINE
%
%\noindent\textit{Performance}
%Our case study shows that our sound approach can monitor programs
%for a fraction of the cost of many other approaches.
%Many other works%
%~\cite{feldman2006jose,fahndrich2010embedded,abercrombie2002jcontractor,tran2003design}
% check/run
%the invariant code at the start and end of every public
%method; this even include trivial getters.
%In  our approach, we call the \Q@invariant()@ method
%one time at the end of each setter, capsule mutator method and constructor.
%We do not inject it at the end of other methods, which are usually more numerous and invoked much more often.
%Of course, \Q@invariant()@ can still be called indirectly, for example by calling a setter.
%We expect our approach to result in a dramatic reduction over the number of required checks,
%except for cases when public methods just update many fields directly (without using setters).

\end{document}
