\section{Introduction}
\label{s:intro}
%\newpage
%\LINE

Object-oriented programming languages provide great flexibility through subtyping and dynamic dispatch: they
allow code to be adapted and specialised to behave differently in different contexts.
%%, which is made even more complex by dynamic class loading (supported by many mainstream OO languages).
However this flexibility hampers code reasoning, since object behaviour is usually nearly completely
unrestricted. This is further complicated by the support OO languages typically have for exceptions,
memory mutation, and I/O.
% Class invariants are a well known technique to help write correct code, however there are various different interpretations of when they should hold.
% invariant protocols, specifying when the invariant is expected to hold and when is checked. 
%% In the absence of on the fly static verification of dynamically loaded code, it is difficult for programmers to write code that is correct in a library setting.


Class invariants are an important concept when reasoning about software correctness.
They can be presented as documentation, checked as part of static verification, or, as we do in this paper, monitored for violations using runtime verification.
In our system, a class specifies its invariant by defining a boolean method called \Q@invariant@.
We say that an object's invariant holds when its \Q@invariant@ method would return \Q@true@. 
We do this, like Dafny~\cite{DBLP:conf/sigada/Leino12}, to minimise special syntactic and type-system treatment of invariants, making them easier to understand for users. Whereas most other approaches treat invariants as a special annotation with its own syntax.

An \emph{invariant protocol}~\cite{FlexibleInvariants} specifies when invariants need to be checked, and when they can be assumed; if such checks guarantee said assumptions, the protocol is sound.
The two main sound invariant protocols present in literature are \emph{visible state semantics} \cite{Meyer:1988:OSC:534929} and the \emph{Boogie/Pack-Unpack methodology}~\cite{DBLP:journals/jot/BarnettDFLS04}. The visible state semantics expect the invariants of receivers to hold before and after every public method call, and after constructors. Invariants are simply checked at all such points, thus this approach is obviously sound; however this can be incredibly inefficient, even in simple cases.
In contrast, the pack/unpack methodology marks all objects as either \emph{packed} or \emph{unpacked}, where a packed object is one whose invariant is expected to hold.
In this approach, an object's invariant is checked only by the pack operation.
In order for this to be sound, some form of aliasing and/or mutation control is necessary. For example, Spec\#~\cite{Barnett:2004:SPS:2131546.2131549}, which follows the pack/unpack methodology, uses a theorem prover, together with source code annotations.
While Spec\# can be used for full static verification, it conveniently allows invariant checks to be performed
at runtime, 
whilst statically verifying aliasing, purity and other similar standard properties.
This allows us to closely compare our approach with Spec\#.

Instead of using automated theorem proving, 
it is becoming more popular to verify aliasing and immutability using a type system.
For example, three languages: L42~\cite{ServettoZucca15,ServettoEtAl13a,JOT:issue_2011_01/article1,GianniniEtAl16}, Pony~\cite{clebsch2015deny,clebsch2017orca}, and the language of Gordon \etal~\cite{GordonEtAl12} use Type Modifiers (TMs) and Object Capabilities (OCs) to ensure safe and deterministic parallelism.%
\footnote{TMs are called \emph{reference capabilities} in other works. We use the term TM here
to not confuse them with object capabilities, another technique we also use in this paper.}
While studying those languages, we discovered an elegant way to enforce invariants.

We use the guarantees provided by these systems to ensure that at all times, if an object is usable in execution, its invariant holds. What this means is that if you can do anything with an object, such as by using it as an argument/receiver of a method call, we know that the invariant of it, and all objects reachable from it, holds. In order to achieve this, we use TMs and OCs to restrict how the result of invariant methods may change, this is done by restricting I/O as well as what state the invariant can refer to and what can alias/mutate such state.  We use these restrictions to reason as to when an object’s invariant could have been violated, and when such object can next be used, we then inject a runtime check between these two points. See Section \ref{s:protocol} for the exact details of our invariant protocol.

\subheading{Example}
Here we show an example illustrating our system in action. Suppose we have a \Q@Cage@ class which contains a \Q@Hamster@; the \Q@Cage@ will move its \Q@Hamster@ along a path. We would like to ensure that the \Q@Hamster@ does not deviate from the path. We can express this as the invariant of \Q@Cage@: the position of the \Q@Cage@'s \Q@Hamster@ must be within the path (stored as a field of \Q@Cage@).

%
% While Spec\# requires specialised \Q@Point@, \Q@Hamster@, and \Q@Cage@ declarations to be able to enforce the invariant, our version manages to capture the required information in just a few annotations on \Q@Cage@ and leaves \Q@Point@ and \Q@Hamster@ unmodified.
%	if(that==null || !(that instanceof Point)){return false;}
% 	return ((Point)that).x==this.x && ((Point)that).y==this.y; 
%  }
\begin{lstlisting}
class Point { Double x; Double y; Point(Double x, Double y) {..}
  @Override read method Bool equals(read Object that) {
    return that instanceof Point &&
      this.x == ((Point)that).x && this.y == ((Point)that).y; }
}
class Hamster {Point pos; //pos is imm by default
  Hamster(Point pos) {..} 
}
class Cage {
  capsule Hamster h;
  List<Point> path; //path is imm by default
  Cage(capsule Hamster h, List<Point> path) {..}
  read method Bool invariant() {
    return this.path.contains(this.h.pos); }
  mut method Void move() {
    Int index = 1 + this.path.indexOf(this.h.pos));
    this.moveTo(this.path.get(index % this.path.size())); }
  mut method Void moveTo(Point p) { this.h.pos = p; }
}
\end{lstlisting}

Many verification approaches take advantage of the separation between primitive/value types and objects, since the former are immutable and do not support reference equality.
However, our approach works in a pure OO setting without such a distinction. Hence we write all type names in \Q@BoldTitleCase@ to underline this. Note: to save space, here and in the rest of the paper we omit the bodies of constructors that simply initialise fields with the values of constructor parameters, but we show their signature in order to show any annotations.

We use the \Q@read@ annotation on \Q@equals@ to express that it does not modify either the
receiver or the parameter. In \Q@Cage@ we use 
the \Q@capsule@ annotation to ensure
that the \Q@Hamster@'s \emph{reachable object graph} (ROG) is fully under the control
of the containing \Q@Cage@. 
We annotated the \Q@move@
and \Q@moveTo@ methods with \Q@mut@, since they modify
their receivers' ROG. The default annotation is always \Q@imm@, thus \Q@Cage@'s \Q@path@ field is a deeply immutable list of \Q@Point@s.
% Note how we just use \Q@List.contains()@ and \Q@List.indexOf()@
% to check if the hamster position is inside the list.
% The conventional syntax correctly instantiates a \Q@Cage@:
% \Q@new Cage(new Hamster(new Point(..)), List.of(new Point(...))@.
Our system performs runtime checks for the invariant
at the end of \Q@Cage@'s constructor, \Q@moveTo@ method, and after any update to one of its fields.
The \Q@moveTo@ method is the only one that may (directly) break the \Q@Cage@'s invariant. However, there is only a single occurrence of \Q@this@ and it is used to read the \Q@h@ field. We use the guarantees of TMs to ensure that no alias to \Q@this@ could be reachable from either \Q@h@ or the immutable \Q@Point@ parameter. Thus, the potentially broken \Q@this@ object is not visible while the \Q@Hamster@'s position is updated. 
The invariant is checked at the end of the \Q@moveTo@ method, just before \Q@this@ would become visible again.
This technique loosely corresponds to an implicit pack and unpack: we use \Q@this@ only to read the field value, then we work on its value while the invariant of \Q@this@ is not known to hold, finally we check the invariant before allowing the object to  be used again.

Note: since only \Q@Cage@ has an invariant,
 only \Q@Cage@ has special restrictions, allowing the code for \Q@Point@ and \Q@Hamster@ to be unremarkable.
 This is not the case in Spec\#: all code involved in  verification needs to be designed with verification in mind~\cite{barnett2011specification}.
% The best solution we found was to define our own equality for \Q@Point@ instead of relying on \Q@Object.Equals@,
% thus we could not use \Q@List.Contains@ and \Q@List.IndexOf@.

\subheading{Spec\# Example} Here we show the previous example in Spec\#, the system most similar to ours (see appendix \ref{s:hamster} for a more detailed discussion about this solution):
%or \small or \footnotesize etc.
\begin{lstlisting}[
language={[Sharp]C}, morekeywords={invariant,ensures,requires,expose,exists}]
// Note: assume everything is `public'
class Point { double x; double y; Point(double x, double y) {..}
  [Pure] bool Equal(double x, double y) {
    return x == this.x && y == this.y; }
}
class Hamster{[Peer]Point pos; 
  Hamster([Captured]Point pos){..}
}
class Cage {
  [Rep] Hamster h; [Rep, ElementsRep] List<Point> path;
  Cage([Captured] Hamster h, [Captured] List<Point> path)
    requires Owner.Same(Owner.ElementProxy(path), path); {
      this.h = h; this.path = path; base(); }
  invariant exists {int i in (0 : this.path.Count);
    this.path[i].Equal(this.h.pos.x, this.h.pos.y) };
  void Move() {
    int i = 0;
    while(i<path.Count && !path[i].Equal(h.pos.x,h.pos.y)){i++;}
    expose(this) {this.h.pos = this.path[i%this.path.Count];}}
}
\end{lstlisting}

In both versions, we designed \Q@Point@ and \Q@Hamster@ in a general way, and not solely to be used by classes with an invariant, in particular \Q@Point@ is not an immutable class. However, doing this in Spec\# proved difficult, in particular we were unable to override \Q@Object.Equals@, or even define a usable \Q@equals@ method that takes a \Q@Point@, as such we could not call either \Q@List<Point>.Contains@ or \Q@List<Point>.IndexOf@.
 
Even with all of the above annotations, we still needed special care in creating \Q@Cage@s:\vspace{-1.860px}% magic number that prevents the listings background going onto the next page
\begin{lstlisting}[
%basicstyle=\footnotesize,
language={[Sharp]C}, morekeywords={invariant,ensures,requires,expose,exists}]
List<Point> pl = new List<Point>{new Point(0,0),new Point(0,1)};
Owner.AssignSame(pl, Owner.ElementProxy(pl));
Cage c = new Cage(new Hamster(new Point(0, 0)), pl);
\end{lstlisting}

Whereas with our system we can simply write:
\begin{lstlisting}
List<Point> pl = List.of(new Point(0, 0), new Point(0, 1));
Cage c = new Cage(new Hamster(new Point(0, 0)), pl);
\end{lstlisting}

%3 read 2 capsule 3 mut extra method moveTo
%----
In Spec\# we had to add $10$ different annotations, of $8$ different kinds; some of which were quite involved. In comparison, our approach requires only $7$ simple keywords, of $3$ different kinds; however we needed to write 
a separate \Q@moveTo@ method, since we do not want to burden our language with extra constructs such as Spec\#'s \Q@expose@.
%  Moreover we had been unable to reuse 
% \Q@Object.Equals@, \Q@List.IndexOf@ and % \Q@List.Contains@.
% Note: we had to add a new class \Q@PureObject@, since the \Q@Objec@ constructor is not annotated as \Q@[Pure]@.
%3 pure,
%1 peer
%3 captured
%2 rep
%1 ElementsRep
%1 requires Owner.Same(Owner.ElementProxy(path), path);
%1 invariant
%1 exists
%expose(this)
%re implementation of indexOf
%dumb equals(double,double)
%dumb class PureObject { [Pure] PureObject() { } }
%Owner.AssignSame(pl, Owner.ElementProxy(pl));
% manually handle ownership details while instantiating a \Q@new Cage(..)@.
% Note how the \Q@expose@ block cover plays the same role of our \Q@moveTo@ method.
%\begin{lstlisting}
%class SafeMovable implements Widget{
%  @Override read method Widgets children(){return %this.box.cs;}
%  @Override read method Int left(){return this.box.l;}
%  @Override read method Int top(){return this.box.t;}
%  @Override read method Int width(){return this.w;}
%  @Override read method Int height(){return this.h;}
%  capsule Box box;
%  Int w; Int h;
%  read method Bool invariant(){//iterate on box.cs, check:
%    //not overlap with each other, are inside the widget bounds
%  }
%  SafeMovable(Int w,Int h,capsule List<Widget> cs) {
%    this.w=w; this.h=h; this.box=boxWithButton(cs);}
%  static method capsule Box boxWithButton(capsule Widgets cs){
%    mut Box b=new Box(5,5,cs);
%    b.cs.add(new Button(0,0,10,10,new MoveAction(b));
%    return b;//b is declare mut, but it is soundly returned capsule
%  }}
%\end{lstlisting}

%\begin{lstlisting}
%class Box{
%  Int l; Int t; mut List<Widget> cs;
%  Box(Int l, Int t, mut List<Widget> cs){...} }
%
%class MoveAction implements Action{
%  mut Box outer; MoveAction(mut Box outer){this.outer=outer;}
%  mut method Void process(Event event) {this.outer.l+=1;} }
%\end{lstlisting}

%Ideally, we would like the invariant to be dynamically checked 
%at the end of the constructor and at the end of the \Q@move@ method.
%However, this would be unsound without some form of aliasing control over \Q@Hamster@,
%the \Q@List@ and all the \Q@Point@s, as shown in the following example\footnote{
%The visible state semantic prevent \Q@toString()@ from producing a non nonsensical result
%by checking the invariant and throwing an invariant error.
%}
%\begin{lstlisting}
%List<Point>ps=Arrays.asList(new Point(2,3),new Point(4,5));
%Cage c=new Cage(new Hamster(new Point(2,3),ps);
%//invariant holds here
%ps.get(0).x=8;//invariant is broken here, since ps
%//was accessible from outside the cage
%c.toString()//the hamster is in a position that is not on the list
%c.invariant();//return false!!
%\end{lstlisting}

%Moreover, this is unsound also if we can not ensure determinism of the invariant method;
%for example we could have an \Q@EvilList@

%\begin{lstlisting}
%class EvilList<T> extends ArrayList<T>{..
%  @Override boolean contains(T elem){return new Random().bool();}
%}
%..
%List<Point>ps=Arrays.asList(new Point(2,3),new Point(4,5));
%Cage c=new Cage(new Hamster(new Point(2,3),ps);
%//invariant happens to holds at the end of the constructor by chance,
%c.invariant();//here instead it return false!!
%\end{lstlisting}

%Despite the code for \Q@Cage.invariant()@ intuitively looking correct and deterministic, the above calls to it are not. Obviously this breaks any reasoning and should be considered unsound. 
%In particular, note how in the presence of dynamic class loading, 
%we can not make any assumption on the dynamic type of \Q@path@.
%A:TrustInvariant<><{ int num}
%B:Invariant<><{A a}


%\begin{lstlisting}
%class Point {Int x; Int y;
%  /*... constructor, equals and other obvious utility methods*/}
%class Hamster { Point pos; Hamster(Point pos){this.pos=pos;} }
%class Cage {
%  capsule Hamster h;  List<Point> path;
%  Cage(capsule Hamster h, List<Point> path){ this.h=h; this.path=path; }
%  read method Bool invariant(){
%    return this.path.contains(this.h.pos);
%  }
%  mut method Void move() {
%    Int index=1+this.path.indexOf(this.h.pos);
%    if (index>=this.path.size()) {index=0;}
%    this.moveTo(this.path.get(index));
%  }
%  mut method Void moveTo(Point p){  this.h.pos=p }//new method
%  @Override method String toString(){
%    return "hPos:"+this.h.pos+", path:"+this.path;
%  }
%}
%\end{lstlisting}

%\begin{lstlisting}
%Point:Data<><{var Num x, var Num y}
%Points:Collections.vector(ofMut:Point)
%Hamster:Data<><{var Point pos}
%Cage:Data<><{
%  capsule Hamster h
%  Points path
%  read method Bool #invariant()
%    this.path().contains(this.h().pos())
%  mut method Void move() (
%    var Size index=1Size+this.path().indexOfLeft(val:this.h().pos())
%    if index>=this.path().size() (index:=0Size)
%    this.move(newPos:this.path().val(index))
%    )
%  mut method Void move(Point newPos)
%    this.#h().pos(newPos)
%  }
%\end{lstlisting}
%In L42 we only need to add 
%1 \Q@read@, 2 \Q@mut@ and 1 \Q@capsule@ annotations.
%We also need to add a new method \Q@moveTo@. This is equivalent to the explicit \Q@expose(this){..}@
%block required in Spec\#.

%To keep the syntax familiar, we present our code example in a tweaked Java syntax using type modifiers.
%If a method override an interface method, we inherit the modifiers from the interface.
%Any non annotated type is implicitly immutable.
%Note how we just added 1 \Q@read@, 2 \Q@mut@ and 2 \Q@capsule@ annotations; in L42 constructors can be automatically generated, this would remove the need for 1 of the\Q@capsule@ annotations.
%We also need to add a new method \Q@moveTo@. This is equivalent to the explicit \Q@expose(this){..}@
%block required in Spec\#.

%In L42 any non annotated type is implicitly immutable.
%We also need to add a new method \Q@moveTo@. This is equivalent to the explicit \Q@expose(this){..}@
%block required in Spec\#.
%In this paper we will show how those minor code modifications are sufficient
%to statically verify that runtime verification is needed only
%after the constructor and after the \Q@moveTo@ method.
%We will also show how designing the standard library in OCs style
%ensures that any \Q@read@ method with no parameters (as \Q@invariant()@) is
%deterministic.
%We obtain our results thanks to coarse grained type system support and a careful design of the standard library, where all the possible sources of non-determinism follow the OCs style.
%---------------------

%We evaluate our contribution by means of case studies;
\subheading{Summary}
We have fully implemented our protocol in L42\footnote{A suitably anonymised, experimental version of L42, supporting the protocol described in this paper, together with the full code of our case studies, is available at \url{http://l42.is/EcoopArtifact.zip}.}\footnote{We also believe it would be easy to implement our protocol in Pony and Gordon \etal's language.}, we used this implementation to implement and test an interactive GUI involving a class with an invariant. On a test case with $5$ objects with an invariant, 
our protocol performed only $77$ invariant checks, whereas the visible state semantic invariant protocols of D and Eiffel perform $53$ and $14$ million checks (respectively). See Section \ref{s:case-study} for an explanation of these result.
% \MS{The difference between D and Eiffel is an effect of Eiffel's `uniform access principle', where fields can be used to implements `interface methods', while not triggering invariant checks.}
We also compared with Spec\#, whose invariant protocol performs the same number of checks as ours, however the annotation burden was almost $4$ times higher than ours.


In this paper we argue that our protocol is not only more succinct than the pack/unpack approach, but is also easier and safer to use.
Moreover, our approach deals with more scenarios than most prior work: we allow sound catching of invariant failures and also carefully handle non deterministic operations like I/O.
%In our case study we show that
%we can still encode most of the examples explored in ~\cite{???} (including for example mutable collections of immutable objects) whilst having a significantly lower annotation-burden.
Section \ref{s:TMsAndOCs} explains the \emph{type modifier} and \emph{object capability} support we use for this work.
Section \ref{s:protocol} explains the details of our invariant protocol, and Section \ref{s:formalism} formalises a language enforcing this protocol.
Sections \ref{s:immutable} and \ref{s:encapsulated}, respectively, explain and motivate how our protocol can handle invariants over immutable and encapsulated data.
Section \ref{s:case-study} presents our GUI case study and compares it against visible state semantics and Spec\#.
Sections \ref{s:related} and \ref{s:conclusion} provide related work and conclusions.

Appendix \ref{s:proof} provides a proof that our invariant protocol is sound. Appendices \ref{s:hamster} and \ref{s:MoreCaseStudies} provide further
case studies and comparisons against Spec\#, D and Eiffel.

%http://www.cs.cmu.edu/~NatProg/papers/p496-coblenz-Glacier-ICSE-2017.pdf
