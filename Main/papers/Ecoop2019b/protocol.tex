%-----------------------------------------------------------
%define simple objects
%show solution  for simple person: requires 3 properties
%show solution is sound --> proof in appendix
%naive is unsound - person 3 bugs
\saveSpace
\section{Our Invariant Protocol}
\label{s:validate}
\saveSpace
Our invariant protocol guarantees that the whole ROG of any object involved in execution (formally, in a redex) is \emph{valid}, i.e. its invariant holds.
This implies that if you can call methods on an object, calling \Q@invariant@ on it is guaranteed to return \Q@true@.
Of course the bodies of both \Q@invariant@ methods and constructors may interact with an invalid \Q@this@.
%Logically, there are two reasons to access a field: we may wish to read the information stored in such object or we wish to mutate the object contained in the field.
%For the first case, we can type the field access as \Q@read@, but in the second case we
%need to type it as \Q@mut@. 
%We call `capsule mutators' a method accessing as \Q@mut@ a capsule field referenced in the invariant.
%We will show how capsule mutators are analogous of the pack/unpack/expose~\cite{???}.
%In order for a class to have an invariant under our protocol,
%\IODel{its }\Q@invariant@ method the form 
% Can  a program write say mut method invariant or is it syntactically [???]
Our protocol enforces that all classes have a method of the form \Q@read method Bool invariant() {..}@ that only uses \Q@this@ to read \Q@imm@ and \Q@capsule@ fields. If an \Q@invariant@ method is not defined by the programmer, one that always returns \Q@true@ will be generated for them.

Moreover, constructors may only use \Q@this@ to initialise fields.\footnote{A factory method could be used when more flexible initialisation is required.} This prevents objects from escaping their constructor partially initialised, and of course prevents explicitly calling \Q@invariant@ inside of \Q@invariants@ and constructors bodies, but most importantly, together with our other restrictions, it prevents \Q@this@, or an alias to \Q@this@, from being passed around to arbitrary code.

While mutating the ROG of a \Q@capsule@ field mentioned in the invariant, the containing object may be invalid.
To enforce that the containing object is inaccessible, an access to such field is typed as \Q@mut@ only inside of methods following a specific pattern: the field access is of the form \Q@this.f@ and there is no other occurrence of \Q@this@, the method receiver is \Q@mut@, none of the other parameters are \Q@mut@ or \Q@read@, and the return type is not \Q@mut@.
We call such methods \emph{capsule mutators}. These additional restrictions do not apply if the field is not read in the invariant,
%Finally, we require \Q@f@ to be declared \Q@private@.%, and only accessed as \Q@this.f@.

Our invariant protocol injects calls to \Q@invariant@ on newly created objects, after constructor calls; on the receiver, after a field update; and \Q@this@ at the end of a capsule mutator.
%\end{itemize}

% Our invariant protocol injects call to \Q@this.invariant@ at the end of constructors, after every field update and at the end of every method containing such \Q@this.f@ access. We call such methods `capsule mutators'.

%Our guarantee is much stronger then the visible state semantic, where in the presence of call-backs,
%invalid objects may be passed around and become visible to any other code~\cite{??}.


%\footnote{we could of course require zero-initialized objects be valid, but this is too strong},
%so in those cases we do not require \Q@this@ to be valid; however one can only use "this" to read/initialise fields, preventing \Q@this@, or an alias, from being passed around to arbitrary code.
%This implies that whenever the user is allowed to call x.invariant(), such call would return true.
%The only points where the user is unable to call .invariant() is inside the constructor
%and inside the invariant method itself.
%If you can refer to a variable \Q@x@, then \Q@x.invariant()@ would return \Q@true@. 
%However, we prevent directly using \Q@this@ inside constructors, invariant itself, and capsule mutators (see \ref{}),
%rather one can only use \Q@this@ to access fields.
% We guarantee that invariant holds in more points, for more objects, but we check it a million times less



%\IO{The Visible-State Semantics is bad.} Here we have a much stronger property, called \emph{operational
% validity}. This means that any object involved in execution (such as by accessing a field, calling a method on it, or) must be valid. As we use runtime-verification to detect invalid objects, it must be possible to access potentially invalid objects within their own \Q@invariant@ method, so we make an exception to the above rule in this case: \Q@this@ itself may be invalid within runtime-injected calls to \Q@invariant@.




%\section{\REVComm{The invariant Method}{2}{remind of the goals for validate (i.e. that it be pure); What about possible non-termination of validate()}}
%\label{s:validate}

%We require the receiver to be \Q@read@ and to have no arguments.
%The method is \Q@read@: thus the method body will see the \Q@this@ object as a \Q@read@ reference; and has no other parameters. 
%By starting from a \Q@read@ reference and nothing else, we are guaranteed that the method is pure:

Our invariant protocol relies on TMs and OCs to ensure that the invariant method is deterministic\footnote{If the invariant were not deterministic, it would be nearly impossible to ensure that it would return \Q@true@ at any point.}.
As this method is declared as \Q@read@, and only takes the implicit parameter \Q@this@ (as \Q@read@), we can guarantee the method is pure:
\begin{itemize}
\item the ROG of \Q@this@ is only accessed as \Q@read@ (or \Q@imm@), thus it cannot be mutated\footnote{This is even true in the concurrent environments of Pony and Gorden, since they ensure that no other thread/actor has access to a \Q@mut@/\Q@capsule@ alias of \Q@this@.},
\item if a capability object is in the ROG of \Q@this@, then it can only be accessed as \Q@read@, preventing calling non deterministic (capability) methods,
\item no other pre existing objects are accessible (as L42 does not have global variables).
\end{itemize}

\noindent Also note that \Q@invariant@ is declared as not throwing any exceptions,
	thus only unchecked exceptions can be propagated out.
	% \REVComm{thus it can only leak unchecked exceptions.}{3}{Can ``leaking'' of an unchecked exceptions have negative consequences for validity? Leaking is usually a negative word used in a negative context.}


Clearly the \Q@invariant@ method must be able to take an invalid \Q@this@, since the purpose of such method is to distinguish valid and invalid objects.
On a first look this may seem an open contradiction
with the aim of this work, however only calls to \Q@invariant@ inserted by the language semantics can take an invalid \Q@this@. As for any other method, when the application code calls \Q@invariant@,
\Q@this@ is guaranteed to be valid.
We require that \Q@invariant@
only uses \Q@this@ to access fields,
in order to prevent passing an invalid \Q@this@ to other methods.
%other 
%methods can not 
%However, if we allow the method to use \Q@this@ directly (e.g. storing it in a local variable or passing it to a method), we would break the guarantee of our invariant protocol.
In L42 we implemented a sound relaxation of this restriction, by allowing calling instance methods that in turn only use \Q@this@ to access fields, or call other such instance methods. With this relaxation, the semantics of \Q@invariant@ needs to be understood with the body of those methods inlined; thus the semantic of the inlined code need to be logically reinterpreted in the context of \Q@invariant@, where \Q@this@ may be invalid.
In some sense, those inlined methods and field accesses can be thought of as macro expanded, and hence are not dynamically dispatched. 
Finally, the code of \Q@invariant@ cannot access \Q@mut@ or read fields, because their content can be changed by unrelated code.

%With the modifiers presented so far, \Q@invariant@ can only access \Q@imm@ fields.
%We will later introduce a \Q@capsule@ modifier to allow more flexible validation.


L42 does not have \Q@extends@ so we have not handled inheriting invariants.
In a language with subclassing, invariant methods would implicitly start with a check that \Q@super.invariant()@ returns \Q@true@.
Note that invariant checks would not be performed at the end of \Q@super(..)@ constructor calls, but only at the end of \Q@new@ expressions, as happens in~\cite{feldman2006jose}.

%In a language with sub-classing, runtime-invariant checks would need to occur on the base-class subobjects, before doing the check on a derived class. In addition, the runtime would only do an invariant check after construction of complete objects.

% containing the \Q@invariant@ method has a super-class, we would automatically check \Q@super.invariant()@ at the beginning of the sub-class’s \Q@invariant@ method.
%, this is required as otherwise `invalid' objects could easily be created by simply overriding \validate.


% JUSTIFY that the fields are valid...

%validable objects are not circular (do not belong in their ROG of any of its fields)
%validate as a predicate on object fields never really see this.
%
%clarifications:
%a validate check is never needed/generated/injected when working on a read x
%multi threading is not relevant/supported
%validable objects are not circular (do not belong in their ROG of any of its fields)