\section{Type Modifiers and Object Capabilities}
\label{s:TMsAndOCs}
Reasoning about imperative object oriented (OO) programs is a non trivial task,
made particularly difficult by mutation, aliasing, dynamic dispatch, I/O, and exceptions. There are many ways to perform such reasoning, here we use the type system to restrict, but not prevent such behaviour in order to be able to soundly enforce invariants with runtime verification (RV).
% [dynamic class loading],

\subheading{Type Modifiers (TMs)}
TMs as used in this paper are a type system feature that allows reasoning about aliasing and mutation. Recently a new design for them has emerged that radically improves their usability;
three different research languages are being independently developed relying on this new design: the language of Gordon et.~al.~\cite{GordonEtAl12}, Pony~\cite{clebsch2015deny,clebsch2017orca}, and L42~\cite{ServettoZucca15,ServettoEtAl13a,JOT:issue_2011_01/article1,GianniniEtAl16}.
These projects are quite large: several million lines of code are written in Gordon et.~al.'s language and are used by a large private Microsoft project; Pony and L42 have large libraries and are active open source projects. In particular the TMs of these languages are used to provide automatic and correct parallelism~\cite{GordonEtAl12,clebsch2015deny,clebsch2017orca,ServettoEtAl13a}.

While we focus on the specific TMs provided by L42, Pony, and Gordon et.~al., type modifiers
 are a well known language mechanism~\cite{TschantzErnst05,BirkaErnst04,OstlundEtAl08,clebsch2015deny,GianniniEtAl16,GordonEtAl12}
 that allow static reasoning of mutability and aliasing properties of objects.
With slightly different names and semantics, the four most common modifiers for object references are:
\begin{itemize}
\item Mutable (\Q@mut@): the referenced object can be mutated, as in most imperative languages without modifiers.
If all types are \Q@mut@, there is no restriction on aliasing/mutation.
\item Readonly (\Q@read@): the referenced object cannot be mutated by such references, but there may be mutable aliases to such object, thus mutation can still be observed. 
\item Immutable (\Q@imm@): the referenced object can never mutate. Like \Q@read@ references, one cannot mutate through an \Q@imm@ reference, however \Q@imm@ references also guarantee that the referenced object will not mutate through any other alias.
\item Encapsulated (\Q@capsule@):
 everything in the ROG of a capsule reference (including itself) is mutable only through that reference; however immutable references can be freely shared across capsule boundaries.
\end{itemize}
%In the context of object-oriented programming, type modifiers may also apply to the implicit \Q@this@ parameter in method declarations, restricting the type of references the method can be called on. In addition, due to the deep meanings we type field access on object references to be the most restrictive of the object references modifier and the field’s. As \Q@read@ references impose no assumptions about aliasing, any \Q@imm@ or \Q@mut@ expression can be safely implicitly promoted to \Q@read@, whereas other conversions are not generally safe.
%\loseSpace

\noindent TMs are different to field or variable modifiers like Java's \Q@final@: TMs apply to references, whereas \Q@final@ applies to fields themselves. Unlike a variable/field of an \Q@imm@ type, a \Q@final@ variable/field cannot be reassigned, it always refers to the same object, however the referenced object itself may be mutated.
On the other hand, an \Q@imm@ reference refers to an object that will never be mutated, but an \Q@imm@ variable can be reassigned.\footnote{In C, this is similar to the difference between \Q@const *A@ and \Q@*const A@, where a \Q@final read@ variable would be like a \Q@const *const A@.}
%\end{itemize}

Consider the following  example usage of \Q@mut@, \Q@imm@, and \Q@read@, where we can observe a change in \Q@rp@ caused by a mutation inside \Q@mp@.
\begin{lstlisting}
mut Point mp = new Point(1, 2);
mp.x = 3; // ok
imm Point ip = new Point(1, 2);
$\Comment{}$ip.x = 3; // type error
read Point rp = mp; // ok, read is a common supertype of imm/mut
$\Comment{}$rp.x = 3; // type error
mp.x = 5; // ok, now we can observe rp.x == 5
ip = new Point(3, 5); // ok, ip is not final
\end{lstlisting}


There are several possible interpretations of the semantics of type modifiers.
Here we assume the full/deep meaning~\cite{ZibinEtAl10}:
\begin{itemize}
  \item the objects in the ROG of an immutable object are immutable,
  \item a mutable field accessed from a \Q@read@ reference produces a \Q@read@ reference,
%  \item no \emph{down}-casting is allowed between different type modifiers.
  \item no casting/promotion from \Q@read@ to \Q@mut@ is allowed.
%  \item promotion, is a type-system feature allowing implicit and safe casting from \Q@read@ and \Q@mut@ to \Q@imm@.
\end{itemize}

\noindent There are many different existing techniques and type systems that handle the modifiers above~\cite{ZibinEtAl10,ClarkeWrigstad03,HallerOdersky10,GordonEtAl12,ServettoZucca15}.
The main progress in the last few years is with the flexibility of such type systems:
 where the programmer should use \Q@imm@ when  representing immutable data
and \Q@mut@ nearly everywhere else. The system will be able to transparently promote/recover~\cite{GordonEtAl12,clebsch2015deny,ServettoZucca15} the type modifiers, adapting them to their use context.
To see a glimpse of this flexibility, consider the following example:
%//the same expression can create mut, imm or capsule
%\saveSpace
\begin{lstlisting}
    mut Circle mc = new Circle(new Point(0, 0), 7);
capsule Circle cc = new Circle(new Point(0, 0), 7);
    imm Circle ic = new Circle(new Point(0, 0), 7);
\end{lstlisting}

Here \Q@mc@, \Q@ic@, and \Q@cc@ are syntactically initialised with the same expression: \Q@new Circle(..)@.
The \Q@new@ expression returns a \Q@mut@, so \Q@mc@ is obviously ok.
%\footnote{Capsules must encapsulate their entire ROG, thus a \Q@new@ expression
%can not directly return \Q@capsule@ in the case of objects with \Q@mut@ fields.}
Moreover, the expression does not use any \Q@mut@ local variables, thus the flexible TM system
allows the \Q@mut@ result to be promoted to \Q@capsule@, thus \Q@cc@ is ok. 
Additionally, a \Q@capsule@ can be implicitly converted to \Q@imm@, thus \Q@ic@ is also ok.
We want to underline that this is not a special feature of \Q@new@ expressions:
any expression of a \Q@mut@ type that uses no free \Q@mut@ variables declared outside can be implicitly promoted to \Q@capsule@/\Q@imm@.\footnote{%
This requires some restrictions on \Q@read@ fields not discussed in detail for lack of space.
} This is the main improvement on the flexibility of TMs in recent literature~\cite{ServettoEtAl13a,ServettoZucca15,GordonEtAl12,clebsch2015deny,clebsch2017orca}.
Former work~\cite{Boyland10,boyland2003checking,Hogg91,Smith:2000:AT:645394.651903,DBLP:conf/pldi/AikenFKT03}, which eventually enabled the work of Gordon et.~al., does not consider promotion and 
infers uniqueness/isolation/immutability only when starting from references that have been tracked with restrictive annotations along their whole lifetime.
From a usability perspective, this improvement means that
these TMs are opt-in: a programmer can write large sections of code
mindlessly using \Q@mut@ types and be free to have rampant aliasing. 
Then, at a later stage, another programmer may still 
be able to encapsulate those data structures into an \Q@imm@ or \Q@capsule@ reference.

%\saveSpace
%\begin{lstlisting}
%mc.radius = 3; // ok
%imm Point ip = ic.center; // ok, ROG immutable
%read Circle rc = mc
%read Point rp = rc.center; // ok, fields of read Circle are read
%$\Comment{}$mut Point mp = rc.center; // type error
%\end{lstlisting}
%\saveSpace
%Such flexibility is also visible where \Q@rc.center@ returns a \Q@read@ but \Q@ic.center@ returns an \Q@imm@: any expression typed as \Q@read@ that only
%uses immutable variables can safely be promoted to \Q@imm@ or \Q@capsule@.


 %(since \Q@ic@ is \Q@imm@, and \Q@imm@ is a deep modifier).
%true fact but not sufficient?

%With this kind of type system, we can ensure immutable classes by just declaring all their fields as final and immutable.%
%\footnote{
%In Java,  to ensure a class is immutable we need:
%the class must be final, all the fields must be final of immutable
%classes (thus no interface fields, final classes all the way down),
%and the SecurityManager need to properly tame reflection.}

% Not sure about this paragraph:

The \Q@capsule@ modifier (sometimes called isolated/\Q@iso@) is possibly the one whose details differ the most in literature. Here we refer to the interpretation of~\cite{GordonEtAl12}, that introduced the concept of recovery/promotion.
This concept is the basis for L42, Pony, and Gordon et.~al.'s type systems~\cite{GordonEtAl12,ServettoEtAl13a,ServettoZucca15,ServettoEtAl13a,clebsch2015deny,clebsch2017orca}. 

%\begin{itemize}
%	\item A capsule local variable can only be used once. %as \Q@capsule@ or \Q@mut@.
%	\item Only a capsule expression can be used to initialize or update a \Q@capsule@ field.
%	\item A capsule field access has the same type modifier as the receiver.
%	\item An expression of a \Q@mut@ type that uses no \Q@mut@ variables declared outside can be implicitly promoted to \Q@capsule@. Promotion/recovery is the main improvement on the flexibility of TM in recent literature~\cite{ServettoEtAl13a,ServettoZucca15,GordonEtAl12,clebsch2015deny,clebsch2017orca}
%	(this requires some restrictions on \Q@read@ fields that we do not discuss in detail for lack of space).
% \end{itemize}

%This is to ensure the capsule doesn't `leak', potentially violating it's exclusivity,

The capsule/isolated fields of Gordon et.~al. and Pony rely on destructive reads~\cite{GordonEtAl12,clebsch2015deny}: in order to read them, a new value (such as \Q!null!) will be assigned to them. In contrast, L42~\cite{ServettoEtAl13a,ServettoZucca15} does not require such destructive reads, thus \Q@capsule@ fields can be accessed many times, and their content can be seen from outside; but only in controlled ways.
Both Gordon et.~al. and Pony restrict how \Q!capsule! local variables can be used by changing the type they are seen as, however both allow the local variable to be `consumed', allowing them to be used as normal capsule/isolated expressions, at the cost of being unable to use the variable again. L42 however uses a simpler approach where all accesses to \Q!capsule! local variables consume them: they are expressed using linear/affine types~\cite{boyland2001alias}, thus they can only be used once.
%Both the capsule/isolated fields and variables of Pony and Gordon et.~al. rely on destructive reads~\cite{GordonEtAl12,clebsch2015deny}: reading such fields replaces their values with \Q@null@.+++ and a static analysis ensures capsule local variables can be accessed only once after the initialization and every update.

%In contrast,  L42~\cite{ServettoEtAl13a,ServettoZucca15} do not require destructive reads, and treat %\Q@capsule@ local variables and \Q@capsule@ fields differently:
%\Q@capsule@ local variables are expressed using linear/affine types~\cite{boyland2001alias}, thus they can only be used once;
%\Q@capsule@ fields can be accessed many times,
%and thus their content can be seen from outside; but only in controlled ways.

%while M\# and Pony requires both capsule fields and capsule variables to be `balloons'~\cite{Almeida97,ServettoEtAl13a} in the object graph.

%Destructive reads would be a bad idea for validation as they would likely invalidate objects.

%x=this.#f()
%..do all you want with x...
%// invariant check here!!
%.
%this.f(x -> ..)
%this.f=transform(this.f)
%invariantCheck()
%transform(this.#f())
%invariantCheck()

%\loseSpace
\subheading{Exceptions}\label{s:exceptions}
In most languages exceptions may be thrown at any point; combined with mutation this complicates reasoning about the state of programs after exceptions are caught: if an exception was thrown whilst mutating an object, what state is that object in? Does its invariant hold?
The concept of \emph{strong exception safety} (SES)~\cite{Abrahams2000,JOT:issue_2011_01/article1} simplifies reasoning:
if a \Q@try@--\Q@catch@ block caught an exception, the state visible before execution of the \Q@try@ block is unchanged, and the exception object does not expose any object that was being mutated.
%\LINE
%\noindent{\textit{Exceptions:}}
% M\#, L42 and Pony rely on SES for all unchecked exceptions to ensure safe and transparent parallelism,
% They wish to ensure the code behave as if the execution was fully sequential.
% Exceptions create additional difficulties in such context: if two operations are running in parallel in
% a fork-join, and the first one produces an exception, it should be safe to cancel the other operation and
% to propagate the exception outwards. The system need to guarantee
% the progress the second operation accumulated is not observable.
% Pony avoids this problem simply by not supporting exceptions;
% while
%M\# and L42 will parallelize only expressions that do not leak checked exceptions,
%and they enforce Strong Exception Safety(SES)~\cite{Abrahams2000} for unchecked exceptions.
%Other authors have identified the concept of SES as
% a general issue when reasoning about objects state after catching an exception.
% while we need it to soundly capture invariant failures.
L42 already enforces SES for unchecked exceptions.\footnote{%
This is needed to support safe parallelism. Pony takes a more drastic approach and does not support exceptions in the first place. 
We are not aware of how Gordon et.~al. handles exceptions, however in order for it to have sound unobservable parallelism it must have some restrictions.%
%We do not know how M\# conciliate deterministic parallelism and unchecked exceptions, we suspect some variation of SES must be in place.
}
L42 enforces SES using TMs in the following way:\footnote{Transactions are another way of enforcing strong exception safety, but they require specialized and costly run time support.}\footnote{A formal proof of why these restriction are sufficient is presented in the work of Lagorio~\cite{JOT:issue_2011_01/article1}.}
\begin{itemize}
\item Code inside a \Q@try@ block capturing unchecked exceptions is typed as if all \Q@mut@ variables declared outside of the block are \Q@read@.
\item Only \Q@imm@ objects may be thrown as unchecked exceptions.
\end{itemize} 
%Of course this has the effect that even if no-exception is thrown, no mutation could have occured, which is an even stronger property than SES, other work is more flexible~\cite{?}, at the cost of more complicated typing rules.
%With SES we can soundly capture invariant-failures as an exception, since any mutation that caused the invariant failure cannot be observed. However, we also need to prevent a broke-object from being reachable from the exception object; since the only way a broken-object can be seen is within the \Q@read@ \Q@invariant@  method, it follows that if the exception-object contains no \Q@read@ references in its ROG it cannot leak a broken object. Preventing this in the-typsystem is non-trivial, so instead we simply require that:

\noindent This strategy does not restrict throwing exceptions, but only catching unchecked ones.
SES allows us to soundly capture invariant failures as unchecked exceptions: 
the broken object is guaranteed to be garbage collectable when the exception is captured. For the purposes of soundly catching invariant failures, it would be sufficient to enforce SES only when capturing exceptions caused by such failures.
%The ability to catch and recover from such failures is extremely useful as it allows the program to take corrective action.(DUPLICATED)

% We think this restriction is acceptable for run time verification, other works are much more restrictive,



%The above rules need only be enforced for catch blocks that could catch invariant-failures (including exceptions thrown within execution of \Q@invariant@) itself;
%, and since \Q@invariant@ declares no checked exceptions, this includes all exceptions throw-able by it.

% TMs are very useful in restricting the scope of mutation. 
% Any expression that does not use any \Q@mut@ 
% variable declared outside of such expression does not modify objects visible outside.
% With this observation in mind, we can use TMs to enforce SES in the following way:\footnote{
% 

% \begin{itemize}
% \item all thrown exceptions are immutable objects,
% \item 
% \end{itemize}

% For the aim of enforcing invariants, we could relax SES to hold only when capturing exceptions caused by invariant failures; but we are building on approaches that enforce SES on all unchecked exceptions .


% Intro to OCs

\subheading{Object Capabilities (OCs)}
OCs, which L42, Pony, and Gordon et.~al.'s work have, are a widely used~\cite{miller2003capability,
noble2016abstract,karger1988improving} programming style that allows associating resources with objects. When this style
is respected, code that does not possess an alias to such an object cannot use its associated resource.
%Object capabilities are programming style used to control and restrict use of operations such as access to external resources
Here, as in Gordon et.~al.'s work, we use OCs to reason about determinism and I/O. To properly enforce this, the OC style needs to be respected while implementing the primitives of the standard library and when performing foreign function calls that could be non deterministic, such as operations that read from files or generate random numbers. Such operations would not be provided by static methods, but instead instance methods of classes whose instantiation is kept under control. 
% \noindent\REVComm{\textit{Object Capabilities:}}{2}{Citations here?}
% While type modifiers are statically verified properties of references, object capabilities are run time characteristics of specific objects.

% Conceptually, an object capability is a communicable, unforgeable token of authority, a key to access special functionality: only certain objects with `special' powers can do `special' actions, and those objects are obtained in a controlled way. We call such objects `capability objects'.


% Their main use case is to allow for fine grained control over what sections of code are allowed to do. 

\lstset{language=Java}
 For example, in Java, \Q@System.in@
 \lstset{language=FortyTwo} 
  is a \emph{capability object} that provides access to the standard input resource, however, as it is globally accessible it completely prevents reasoning about determinism. 
 % In contrast, in the object capability style, one would not have-global variables but have the main por

% a capability object (it has the capability to read input); however it is globally accessible: thus any code could use it, preventing reasoning about determinism.
In contrast, if Java were to respect the object capability style, the \Q@main@ method could take a \Q@System@ parameter, as in
 \Q@main(mut System s)@
 \lstset{language=Java}
\Q@{.. s.in.read() ..}@. \lstset{language=FortyTwo}%
Calling methods on that \Q@System@ instance would be the only way to perform I/O;
moreover, the only \Q@System@ instance would be the one created by the runtime system before calling \Q@main@. % would have no usable constructor, and all its I/O methods would require a mutable (\Q@mut@) receiver.
% Other non deterministic operations would also work this.
%may just take a \Q@mut System@ object as a parameter.
% could also work this way.
This design has been explored by Joe-E~\cite{finifter2008verifiable}.
OCs are typically not part of the type system nor do they require runtime checks or special support beyond that provided by a memory safe language. However, since
L42 allow user code to perform foreign calls without going through a predefined standard library, its type system enforces the OC pattern over such calls:
%To reason about determinism, L42 connects TMs with the OC style as follows: % style by requiring:
\begin{itemize}
\item Foreign methods (which have not been white listed as deterministic) and methods whose name start with \texttt{\#\$} are \emph{capability methods}.%
\item Constructors of classes declared as \emph{capability classes} are also capability methods.
\item Capability methods can only be called by other capability-methods or \Q@mut@/\Q@capsule@ methods of capability classes.
\item In L42 there is no \Q@main@ method, rather it has several main expressions; such expressions can also call capability methods, thus they can instantiate capability objects and pass them around to the rest of the program.
% \item Any method that uses non deterministic primitive operations (such as native calls or access to global variables\footnote{ Even just allowing unrestricted access to \Q@imm@ global variables would prevent reasoning over determinism due to the possibility of global variable updates; however constant/final globals of an \Q@imm@ type would not cause such problems.
% }) must be an instance method requiring a \Q@mut@ receiver.
% Classes having such methods are \emph{capability classes}, and their instances are \emph{capability objects}.
% \item A capability object can only be created inside a \Q@mut@ method of a capability class; or
% by the runtime system, and passed to the main method.

% \item If the language has global variables, they should only be 
%\item There are no global variables.\footnote{}
\end{itemize}

\noindent L42 expects capability methods to be used mostly internally by capability classes, whereas user code would call normal methods on already existing capability objects.

For the purposes of invariant checking, we only care about the effects that methods could have on the running program and heap. As such, \emph{output} methods (such as a \Q@print@ method) can be white listed as `deterministic', provided they do not effect program execution, such as by non deterministically throwing I/O errors.

\subheading{Purity}\label{s:purity}
TMs and OCs together statically guarantee that any method with only \Q!read! or \Q!imm! parameters (including the receiver) is \emph{pure}; we define pure
as being deterministic and not mutating existing memory. Such methods are pure because:
\begin{itemize}
	\item the ROG of the parameters (including \Q!this!) is only accessible as \Q@read@ (or \Q@imm@), thus it cannot be mutated\footnote{This is even true in the concurrent environments of Pony and Gorden, since they ensure that no other thread/actor has access to a \Q@mut@/\Q@capsule@ alias of \Q@this@.},
	\item if a capability object is in the ROG of any of the arguments (including the receiver), then it can only be accessed as \Q@read@, preventing calling any non deterministic (capability) methods,
	\item no other preexisting objects are accessible (as L42 does not have global variables).\footnote{%
		If L42 did have static variables, getters and setters for them would be capability methods.
		Even allowing unrestricted access to \Q@imm@
		static variables would prevent reasoning over
		determinism, due to the possibility of global variable
		updates; however constant/final globals of an 
		\Q@imm@ type would not cause such problems.%
	}
\end{itemize}

%Methods that perform non-deterministic \emph{input} shouldn't be white-listed.%, including methods that read information passed to white-listed output methods.


%Here we combine TMs with OCs to guarantee 
%\MS{determinism of} any method that can not access a \Q@mut@ reference to a capability object:
%all non deterministic primitive operations are instance methods requiring a \Q@mut@ receiver, and
%	\item all non-deterministic primitives (like native calls) require a \Q@mut@ receiver,
%instances of capability classes containing such methods can only be created by a \Q@mut@ method of another capability class
%	\begin{itemize}
%		\item the runtime-system\footnote{as 42 has no standard-library, we treat meta-code as the runtime-system} before main begins,
%		\item within a \Q@mut@ method on such a class
%	\end{itemize}
%	\item all non-deterministic operations require a \Q@mut@ receiver,
%	\item all classes 
%	\item there are no global variables\footnote{Note: even just allowing \Q@imm@
%global variables would prevent reasoning over determinism due to the possibility of global variable updates; however constant/final globals of an \Q@imm@ type would not cause such problems.},
% \item user code cannot directly create a capability object: they can only indirectly do so through an existing \Q@mut@ capability object reference.

% NOTE: SOMEWHERE MAKE IT CLEAR THAT NON-DETERMINISM CAN ONLY OCCUR THROUGH A CAPABILITY OBJECT
%\end{itemize}
