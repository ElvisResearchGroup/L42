\usepackage{verbatim}
%\addbibresource{main.bib}
\usepackage{wrapfig}

%
\makeatletter
\DeclareOldFontCommand{\rm}{\normalfont\rmfamily}{\mathrm}
\DeclareOldFontCommand{\sf}{\normalfont\sffamily}{\mathsf}
\DeclareOldFontCommand{\tt}{\normalfont\ttfamily}{\mathtt}
\DeclareOldFontCommand{\bf}{\normalfont\bfseries}{\mathbf}
\DeclareOldFontCommand{\it}{\normalfont\itshape}{\mathit}
\DeclareOldFontCommand{\sl}{\normalfont\slshape}{\@nomath\sl}
\DeclareOldFontCommand{\sc}{\normalfont\scshape}{\@nomath\sc}
\makeatother

\usepackage{mathpartir}
\usepackage{amsmath}
\usepackage{amsthm}

\theoremstyle{plain}

\makeatletter
\newcommand{\providecounter}[1]{%
  \ifcsname c@#1\endcsname % do nothing, counter allready defined
  \else
    \newcounter{#1}%
  \fi
}
\makeatother


\providecounter{definition}
\newtheorem{Definition}[definition]{Definition}
\newcounter{assumption}
\newtheorem{Assumption}[assumption]{Assumption}
\providecounter{lemma}
\newtheorem{Lemma}[lemma]{Lemma}

\usepackage{xspace}
\usepackage{listings}
\usepackage{xcolor}
\usepackage{letltxmacro}
\usepackage{mathtools}
\usepackage{mathpartir}
%\usepackage{stix}

\definecolor{darkRed}{RGB}{100,0,10}
\definecolor{darkBlue}{RGB}{10,0,100}
\newcommand*{\ttfamilywithbold}{\fontfamily{pcr}\selectfont}
%\newcommand*{\ttfamilywithbold}{\ttfamily}

%found on http://tex.stackexchange.com/questions/4198/emphasize-word-beginning-with-uppercase-letters-in-code-with-lstlisting-package
%\lstset{language=FortyTwo,identifierstyle=\idstyle}
%
\makeatletter
\newcommand*\idstyle{%
        \expandafter\id@style\the\lst@token\relax
}
\def\id@style#1#2\relax{%
        \ifcat#1\relax\else
                \ifnum`#1=\uccode`#1%
                        \ttfamilywithbold\bfseries
                \fi
        \fi
}
\makeatother

\lstset{language=Java,
  basicstyle=\upshape\ttfamily\footnotesize,%\small,%\scriptsize,
  keywordstyle=\upshape\bfseries\color{darkRed},
  showstringspaces=false,
  mathescape=true,
  xleftmargin=0pt,
  xrightmargin=0pt,
  breaklines=false,
  breakatwhitespace=false,
  breakautoindent=false,
 identifierstyle=\idstyle,
 morekeywords={method,Use,This,constructor,as,into,rename},
 deletekeywords={double},
 literate=
  {\%}{{\mbox{\textbf{\%}}}}1
  {~} {$\sim$}1
%  {<}{$\langle$}1
%  {>}{$\rangle$}1
}

\newcommand*{\SavedLstInline}{}
\LetLtxMacro\SavedLstInline\lstinline
\DeclareRobustCommand*{\lstinline}{%
	\ifmmode
	\let\SavedBGroup\bgroup
	\def\bgroup{%
		\let\bgroup\SavedBGroup
		\hbox\bgroup
	}%
	\fi
	\SavedLstInline
}

\newcommand\saveSpace{\vspace{-2pt}}

\newcommand\Rotated[1]{\begin{turn}{90}\begin{minipage}{12em}#1\end{minipage}\end{turn}}

\newcommand{\Q}{\lstinline}
\newenvironment{bnf}{$\begin{aligned}}{\end{aligned}$}
\newcommand{\production}[3]{\textit{#1}&\Coloneqq\textit{#2}&\text{#3}}
\newcommand{\prodNextLine}[2]{&\quad\quad\textit{#1}&\text{#2}}
\newenvironment{defye}{\\\indent$\begin{aligned}}{\end{aligned}$\\}
\newcommand{\defy}[2]{\!\!\!\!\!\!&&#1&\coloneqq#2\\}
%\newcommand{\defyc}[1]{&\phantom{\coloneqq}\ \ #1\\}
\newcommand{\defyc}[1]{\!\!\!\!\!\!\rlap{\quad \quad #1}&&\\}
\newcommand{\defya}[2]{#1&\!\!\!\!\!\!&\coloneqq#2\\}

%\newcommand{\prodFull}[3]{#1&::=&\mbox{#2}&\mbox{#3}}
\newcommand{\prodInline}[2]{#1\Coloneqq#2}
\newcommand{\terminal}[1]{\ensuremath{$\texttt{#1}$}}
%\newcommand{\metavariable}[1]{\ensuremath{\mathit{#1}}}

\newcommand{\Rulename}[1]{{\textsc{#1}}}
\newcommand{\ctx}[1]{\ensuremath{\mathcal{E}_#1}\!}
\newcommand{\libi}[2]{\Q@\{@\Q!interface!\ #1\Q{;} #2\Q@\}@}
\newcommand{\lib}[3]{\Q!interface!\ensuremath{?}\ \libc{#1}{#2}{#3}}
\newcommand{\libc}[3]{\,\Q@\{@\!#1\Q{;}\ #2 \Q{;}\ #3\Q@\}@\!\!}

\newcommand{\rp}[1]{\Q{(}\!#1\Q{)}}
\newcommand{\eq}[1]{\,\Q{=}#1}
\newcommand{\red}[3]{#1\,\Q{<}#2\eq#3\,\Q{>}}
\newcommand{\summ}[2]{#1\ \Q{<+}\ #2}
\newcommand{\from}[2]{#1\ensuremath{[}#2\ensuremath{]}}
\newcommand{\mmid}{{\ensuremath{{\mid}}}\!}
\newcommand{\hole}{\ensuremath{\square}}
\newcommand{\s}[1]{\ensuremath{\mathit{#1s}}}
\makeatletter
\newcommand{\This}[1]{\Q!This!#1\nextpath}
\newcommand{\Cs}[1]{#1\nextpath}
\newcommand{\nextpath}{\@ifnextchar\bgroup{\gobblenextpath}{}}
\newcommand{\gobblenextpath}[1]{\Q!.!#1\@ifnextchar\bgroup{\gobblenextpath}{}}
\makeatother



%--------------------------
\newcommand{\mynotes}[3]{{\color{#2} {\sc #1}: #3}}
\newcommand\isaac[1]{\mynotes{Isaac}{blue}{#1}}

\newcommand\IO[1]{\color{blue}{#1}}
\newcommand\marco[1]{\mynotes{Marco}{green}{#1}}




% The following code defines a macro, \markLine that puts it’s argument in the margins on either side of the next line
% it can be called multiple times for the same line, causing the arguments to placed next to eachother
\begin{comment} 
\usepackage{lineno}
\linenumbers
\renewcommand\makeLineNumber{}
\newbox\LeftMarkers \newbox\RightMarkers
\newcommand{\markLine}[2]{%
\setbox\LeftMarkers\hbox{#1\unhbox\LeftMarkers}%
\setbox\RightMarkers\hbox{\unhbox\RightMarkers#1}}

\newcommand{\markLineD}[1]{\markLineD{#1}{#1}} % just a shortcut
\renewcommand{\makeLineNumber}{%
\ifvoid\LeftMarkers%
\else \hss\unhcopy\LeftMarkers\ \rlap{\hskip\textwidth\ \unhbox\RightMarkers}%
\fi}
\end{comment}

\newcommand{\saveSpace}{\vspace{-3px}}
\newcommand{\loseSpace}{\vspace{1ex}}

\newcommand{\REV}[3]{%
	\NoteColour{red}{#1\NoteText{\footnote{%
				\textcolor{red}{\textbf{REV#2{:} #3}}}}}}
			
\newcommand{\REVm}[1]{\NoteColour{red}{#1\NoteText{\footnotemark}}}
\newcommand{\REVt}[2]{\footnotetext{\textcolor{red}{\textbf{REV#1{:} #2}}}}
			
\newcommand{\subheading}[1]{%
	\loseSpace%
	\noindent\textsf{\textbf{\large#1\\\noindent}}
}



\let\origSingleDot=\singleDot % reduce spacing arround the dot operator
\renewcommand{\singleDot}{\kern-1.2pt\origSingleDot\kern-1.8pt} 

% Hack (see https://tex.stackexchange.com/questions/299798/make-characters-active-via-macro-in-math-mode)
% this makes '.' in math mode an alias for \singleDot
\newcommand{\defActiveMathChar}[2]{%
	\begingroup\lccode`~=`#1\relax%
	\lowercase{\endgroup\def~}{#2}%
	\AtBeginDocument{\mathcode`#1="8000}%
}
\defActiveMathChar{.}{\singleDot}

\newcommand{\M}[3]{\ensuremath{\Kw{M}\oR{}#1\semiColon{}#2\semiColon{}#3\cR}}

\let\origMapsTo=\mapsto % put proper spacing arround the mapsto arrow
\renewcommand{\mapsto}{\mathrel{\origMapsTo}}
\newcommand{\invariant}{\Kw{invariant}\oR\cR}
\lstset{morekeywords={expose, iso}}
\newcommand{\thm}[1]{\scalebox{0.9}[0.9]{\sf #1}}
