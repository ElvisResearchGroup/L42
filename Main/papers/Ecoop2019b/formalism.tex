\saveSpace
\section{Formalisation}
\label{s:meaning}
\saveSpace
In order to model our system, we need to formalise an imperative object oriented language
with exceptions and object capabilities,  and rich type system
support for \Q@mut@, \Q@imm@, \Q@read@, \Q@capsule@, and strong exception safety.
Formally modelling the semantics of such a language is easy, but 
modelling and proving correctness of it would deserve a paper
of its own, and indeed many such papers exist in literature%
~\cite{ServettoEtAl13a,ServettoZucca15,GordonEtAl12,clebsch2015deny,JOT:issue_2011_01/article1}.
Thus we are going to assume that there is an expressive and sound type system enforcing
those properties, and instead focus on invariant checking.
To provide good modularisation for reasoning, 
we will clearly list the properties we need to rely upon, so that \emph{every} type
system supporting those properties can support our invariant protocol.

We strive to keep our small step semantics as conventional as possible; following Pierce~\cite{pierce2002types} and Featherweight Java~\cite{IgarashiEtAl01} we assume:
\begin{itemize}
\item An implicit program/class-table.
\item Memory $\sigma\Coloneqq l\mapsto C\{\Many{v}\}$ is a finite map from locations $l$ to annotated tuples $C\{\Many{v}\}$: representing objects,
where $C$ is the class name and $\Many{v}$ are the field values.
We use the notation $\sigma[l.f=v]$ to update an object's field and $\sigma[l.f]$ to access it.
\item A main expression that is reduced in the context of such memory and program.
\item A type system $\Sigma;\Gamma\vdash\e:T$, where 
the expression $\e$ can contain locations and free variables;
the type of locations is encoded in $\Sigma\Coloneqq l\mapsto C$,
while the types of free variables are encoded in $\Gamma\Coloneqq x\mapsto T$.
\item We use $\Sigma^\sigma$ to trivially extract the corresponding $\Sigma$ from $\sigma$.
\end{itemize}
In addition, to encode object capabilities and I/O, we assume a special location  $c$ of class \Q@Cap@. This location would refer to an object whose fields model thins like the content of input and output files. In order to simplify our proof, we additionally assume that:
\begin{itemize}
	\item instances of \Q@Cap@ cannot be created with a \Q@new@ expression,
	\item all methods of \Q@Cap@ must be \Q@mut@, and mutate the ROG of their receiver,
	\item \Q@Cap@ can only have \Q@mut@ fields, and
	\item \Q@Cap@'s \Q@invariant@ method is defined to return \Q@true@.
\end{itemize}
\noindent To keep our formalization focused on
the challenges of invariant checking, 
there are some
tweaks with respect to our informal description of our approach.
From a formal perspective 
these changes do not change expressivity:
\begin{itemize}
\item All fields are instance private, as opposed to only \Q@capsule@ ones used in \Q@invariant@. One could always provide getters and setters to simulate public fields.
\item There are no explicit constructor definitions, rather all constructors are of the canonical form
\Q@$C$($T_1 x_1$,$\ldots$,$T_n x_n$) {this.$f_1$=$x_1$;$\ldots$;this.$f_n$=$x_n$;}@,
 where $T_1, \ldots, T_n$ are the types (including modifiers) of $C$.
To provide more flexible initialisation one could always make a factory method.
\item We require \Q@invariant@ methods to only use \Q@this@ to access fields,
this can be achieved by inlining method calls.
This does prevent the calling of recursive methods,
allowing them would not cause problems with our protocol,
but it does not increase expressivity.
%or if they are recursive, replacing them with calls to methods that take the fields of \Q@this@ instead of \Q@this@ itself.
\item For simplicity, we do not have actual exception objects,
rather we just have a concept of \emph{errors} with no associated values.
Adding \Q@imm@ exceptions would not cause any interesting variation of our proof.
\end{itemize}


\newcommand{\ctxG}{\myCalBig{G}}
\renewcommand{\vs}{\Many{v}}
\renewcommand{\Opt}[1]{#1?}
\begin{figure}
\!\!\!\!
\begin{grammatica}
\produzione{\e}{\x\mid l\mid\Kw{true}\mid\Kw{false}\mid \e\singleDot\m\oR\es\cR\mid \e\singleDot\f 
\mid\e\singleDot\f\equals\e 
\mid\Kw{new}\ C\oR\es\cR
\mid\Kw{try}\ \oC\e_1\cC\ \Kw{catch}\ \oC\e_2\cC
}{expression}\\
\seguitoProduzione{
\mid \Kw{M}\oR l;\e_1;\e_2\cR\mid\Kw{try}^{\sigma}\oC\e_1\cC\ \Kw{catch}\ \oC\e_2\cC
}{runtime expr.}\\
\produzione{v}{l}{value}\\
\produzione{\ctx_v}{\square
\mid \ctx_v\singleDot m\oR\es\cR
\mid v\singleDot\m\oR\Many{v}_1,\ctx_v,\es_2\cR
%\mid \ctx_v\singleDot\f 
%\mid \ctx_v\singleDot\f\equals\e
\mid v\singleDot\f\equals\ctx_v
}{evaluation context}\\
\seguitoProduzione{
\mid \Kw{new}\ C\oR\Many{v}_1,\ctx_v,\es_2\cR
\mid \Kw{M}\oR l;\ctx_v;\e\cR
\mid \Kw{M}\oR l;v;\ctx_v\cR
\mid \Kw{try}^\sigma\oC\ctx_v\cC\ \Kw{catch}\ \oC\e\cC}{}\\

\produzione{\ctx}{\square\mid\ctx\singleDot m\oR\es\cR\mid\e\singleDot\m\oR\es_1,\ctx,\es_2\cR
%\mid \ctx\singleDot\f 
%\mid \ctx\singleDot\f\equals\e
\mid \e\singleDot\f\equals\ctx
\mid \Kw{new}\ C\oR\es_1,\ctx,\es_2\cR
}{full context}\\
\seguitoProduzione{
\mid
\Kw{M}\oR l;\ctx;\e\cR\mid
\Kw{M}\oR l;\e;\ctx\cR\mid
\Kw{try}^{\sigma?}\oC\ctx\cC\ \Kw{catch}\ \oC\e\cC\mid
\Kw{try}^{\sigma?}\oC\e\cC\ \Kw{catch}\ \oC\ctx\cC

}{}\\


%\produzione{M_l}{\ctx[M\oR l,\e\cR]}{}\\
%\produzione{\ctxG_l}{
%  M_l\singleDot\m\oR\es_1,\ctx,\es_2\cR
% |\e\singleDot\m\oR\es_1, M_l, \es_2, \ctx, \es_3\cR
% |M_l\singleDot\f\equals\ctx
% |\Kw{new}\ C\oR\es_1,M_l,\es_2,\ctx,\es_3\cR
% |\Kw{try}\oC\ctx\cC\ \Kw{catch}\ \oC\e\cC
% |\ctx[\ctxG_l]}{}\\
\produzione{CD}{\Kw{class}\ C\ \Kw{implements}\ \Many{C}\oC\Many{F}\,\Many{M}\cC\mid 
\Kw{interface}\ C\ \Kw{implements}\ \Many{C}\oC\Many{M}\cC
}{class declaration}\\
\produzione{F}{\T\ \f;}{field}\\
\produzione{M}{\mdf\, \Kw{method}\, \T\ \m\oR\T_1\,\x_1,\ldots,\T_n\,\x_n\cR\ \Opt\e}{method}\\
\produzione{\mdf}{\Kw{mut}\mid\Kw{imm}\mid\Kw{capsule}\mid\Kw{read}}{type modifier}\\
\produzione{\T}{\mdf\,C}{type}\\
\produzione{r_l}{
 v\singleDot\m\oR\Many{v}\cR
\mid v\singleDot\f
\mid v_1\singleDot\f\equals v_2
\mid \Kw{new}\,C\oR\Many{v}\cR
,\quad\text{with }l\in \{v,v_1,v_2,\Many{v}\}
}{redex containing $l$}\\
\produzione{\mathit{error}}{
\ctx_v[\Kw{M}\oR l; v;\Kw{false}\cR]
,\quad\text{with }
\ctx_v \text{not of form}\ \ctx_v'[\Kw{try}^{\sigma?}\oC\ctx_v''\cC\ \Kw{catch}\ \oC\_\cC]
}{validation error}
\end{grammatica}
\caption{Grammar}
\end{figure}


\loseSpace
\noindent\textit{Grammar and Well Formedness Criteria:}
The detailed grammar is defined in Figure 1.
The only non standard expression is the monitor. They are not present in the source code, but are inserted by our reduction rules. The object it monitors is the one referenced by $l$, $e_1$ is the expression which is being monitored, and $e_2$ denotes the evaluation of $l.invariant()$.
We annotate the monitor expression with $l$ to ensure
that if the invariant check fails, it is precisely $l$ that is invalid.
We use a failed monitor expression (i.e. when $\e_2$ is the value \Q@false@) to represent throwing unchecked exceptions.
In addition, our reduction rules annotate the body of \Q@try@ expressions with
the original state of memory. This is used to propagate the guarantees of strong exception safety,
that is, the annotated memory will not be mutated by executing the body of the \Q@try@.

 Our well formedness criteria are:
\begin{itemize}
\item All fields are instance private; that is field accesses in method bodies are of the form
\Q@this.@$f$.

\item Field accesses in the main expression
must be of the form $l\singleDot\f$.

\item \Q@invariant@ methods are of the form \Q@read@ \Q@method@ \Q@Bool@ \Q@invariant()@ \emph{e}.
Where the only occurrences of \Q@this@ in $e$ are as the receiver for field accesses.
\item All fields referred to in \Q@invariant@ are either \Q@imm@ or \Q@capsule@.
\item All methods that read \Q@capsule@ fields
are either \Q@read@;
or \Q@mut/capsule@, with no \Q@mut@ or \Q@read@ parameters, no \Q@mut@ result, and 
must use \Q@this@ exactly once in their body. We could relax the restrictions for parameter and return types, since \Q@this@ is \Q@capsule@, it is a unique alias.
\item 
During reduction, locations that are preserved by a \Q@try@ block are
never monitored; formally 
for $\Kw{try}^\sigma\oC\e\cC\_$, $\e$ is not of the form $\ctx[$\Q@M(@$l;\_$\Q@)@$]$, for any $l\in\sigma$.
\end{itemize}

\noindent We model subtyping with interfaces 
and do not consider subclassing.
In particular interfaces do not have an implemented \Q@invariant@ method, rather classes do.
\newcommand{\rowSpace}{\\\vspace{2.5ex}}

\begin{figure}
\!\!
$\!\!\!\!\!\begin{array}{l}
 \inferrule[(update)]{{}_{}}{
\sigma|l.f\equals{}v\rightarrow \sigma[l.f=v]|
\Kw{M}\oR l;l;l\singleDot\Kw{invariant}\oR\cR\cR
 }{}
\quad
 \inferrule[(new)]{{}_{}}{
\sigma|\Kw{new}\ C\oR\vs\cR\rightarrow \sigma,l\mapsto C\{\vs\}|
\Kw{M}\oR l;l;l\singleDot\Kw{invariant}\oR\cR\cR
 }{}
\\
\rowSpace
 \inferrule[(mcall)]{{}_{}}{
\sigma|l\singleDot\m\oR v_1,\ldots,v_n\cR\rightarrow \sigma|
\e'[\Kw{this}=l,\x_1=v_1,\ldots,x_n=v_n]
 }{
  \begin{array}{l}
  \sigma(l)=C\{\_\}\\
  C.m=\mdf\,\Kw{method}\,\T\,\m\oR\T_1\,\x_1\ldots\T_n\x_n\cR\,\e\\

\text{if }\ \exists \f\text{ such that}\ \ C.f=\Kw{capsule}\,\_,
\mdf=\Kw{mut},
\\*\quad\f\, \text{inside}\, C\singleDot\m
\text{ and }
\f\,\text{inside}\, C\singleDot\Kw{invariant}

\\*
\text{then }\e'=\Kw{M}\oR l;\e;l\singleDot\Kw{invariant}\oR\cR\cR\\*
\text{otherwise }\ \e'= \e
  \end{array}
}
\rowSpace
 \inferrule[(monitor exit)]{{}_{}}{
\sigma|\Kw{M}\oR l; v;\Kw{true}\cR\rightarrow \sigma|v
 }{}
\quad

 \inferrule[(ctxv)]{\sigma_0|\e_0\rightarrow\sigma_1|\e_1}{
\sigma_0|\ctx_v[\e_0]\rightarrow \sigma_1|\ctx_v[\e_1]
 }{}

\quad
 \inferrule[(try enter)]{{}_{}}{
\sigma|\Kw{try}\ \oC \e_1\cC\ \Kw{catch}\ \oC\e_2\cC\rightarrow 
\sigma|\Kw{try}^\sigma\oC\e_1\cC\ \Kw{catch}\ \oC\e_2\cC
 }{}
\quad

\rowSpace

 \inferrule[(try ok)]{{}_{}}{
\sigma,\sigma'|\Kw{try}^{\sigma}\oC v\cC\ \Kw{catch}\ \oC\_\cC\rightarrow \sigma,\sigma'|v
 }{}
\quad

 \inferrule[(try error)]{{}_{}}{
\sigma,\_|\Kw{try}^\sigma\oC \mathit{error}\cC\ \Kw{catch}\ \oC\e\cC\rightarrow \sigma|\e
 }
\quad
 \inferrule[(access)]{{}_{}}{
\sigma|l.f\rightarrow \sigma|\sigma[l.f]
 }{}
%\quad
\end{array}$
\caption{Reduction rules}
\end{figure}

\loseSpace
\noindent\textit{Reduction rules:}
Reduction rules are defined in Figure 2.
These rules are pretty standard;
\textsc{mcall}
uses the auxiliary function \emph{inside},
formally defined as follows:

$%\begin{array}{l}
\f\, \textit{inside}\, C\singleDot\m\text{ iff }
C\singleDot\m=\_\,\Kw{method}\_\,\ctx[\Kw{this}\singleDot\f]
%\end{array}
$

%\noindent Inserting the monitor expressions during reduction is convenient for the proof,
%but it could instead be done ahead of time.

Monitors are added for all field updates and \Q@new@ expressions, but
for method calls, are only added if the method is \Q@mut@ and its body accesses a \Q@capsule@ field (of \Q@this@).
There is no need to monitor \Q@capsule@ methods, since their receiver is a unique alias and hence can never be used again.

The interaction with monitors and exceptions is interesting:
a monitor releases the value if the invariant check evaluates to \Q@true@, and produces an error if the 
check evaluates to \Q@false@.
If either $\e_1$ or $\e_2$ are not already values, the execution is propagated inside
by \textsc{ctxv}.
If either $\e_1$ or $\e_2$ evaluate to an error, such error is captured by 
\textsc{try error}.
Thanks to strong exception safety,
we do not need to worry
if the (partial) execution of $\e_1$ broke the object referenced by $l$.
If the language were to support checked exceptions, but offered 
strong exception safety only for unchecked ones,
the type system would need to require that neither $\e_1$ nor $\e_2$ throw
checked exceptions.





%WHERE TO PUT THIS?
%Note that for \Q@capsule@ fields, the constructor and the field update
%will require \Q@capsule@ for the correspoi, while the field access will produce a \Q@mut@.



\loseSpace
\textit{Axiomatic type properties:}
As previously discussed, instead of providing a concrete set of type rules, we provide a set of properties
that the type system needs to respect.
To express these properties, we first need some auxiliary definitions.

%\noindent\textbf{Define}
%$\mathit{encapsulatedObj}(C)$:\\*
%${}_{}$\quad\quad \Q@class @$C$\,\Q@implements @$\Many{C}$\Q@{@$\,\Many{F}\,\Many{M}$\Q@}@
% and $\forall \mdf\,C\,\f \in \Many{F},\ \mdf \in \{\Kw{imm},\Kw{capsule}\}$\\*
%\noindent As we discussed, only encapsulated objects can support invariants;
%their class declarations only have immutable or capsule fields. Note how here we see immutable
%and simple objects as special cases of encapsulated ones.

The encapsulated ROG of $l_0$ is composed of all the objects
in the ROG of its immutable and capsule fields:\\*
\indent $l \in \mathit{erog}(\sigma,l_0)
\text{ if } \Sigma^\sigma(l_0).f \in \{\Kw{imm}\,\_,\Kw{capsule}\,\_\}
\text{ and } l \in \mathit{rog}(\sigma,\sigma(l_0).f)
$\loseSpace

\noindent An object is \emph{mutatable} in a $\sigma$ and  $\e$ if there is an occurrence of 
$l$ in $e$, that when seen as \Q@imm@ makes the expression ill typed:\\*
$\mathit{mutatable}(l,\sigma,\e) \text{iff}$ for some $T=\Kw{imm}\,\Sigma^\sigma(l)$ and $\ctx[l]=\e$,\\*
\indent $\Sigma^\sigma;\x:T\vdash\ctx[\x]:T'$ does not hold for any $T'$.\loseSpace

\noindent We define
a deterministic reduction arrow.
Here we require that there is exactly one reduction possible:\\*
\indent$\ \sigma_0|e_0\Rightarrow \sigma_1|e_1$ iff $\{\sigma_1|\e_1\}=\{\sigma|\e \text{ where } \sigma_0|e_0\rightarrow \sigma|e\}$

%if $\ \sigma_0|e_0\rightarrow \sigma|e$ then $\sigma_1|\e_1=\sigma|\e$
% $\exists! \sigma_1|\e_1$ such that $\sigma_0|\e_0\rightarrow \sigma_1|\e_1$\\*

%We can now assume the following properties over the type system:

\begin{Assumption}[Progress]
if $\Sigma^{\sigma_0};\emptyset\vdash e_0: T_0$,
and $e_0$ is not a value or $\mathit{error}$, then
$\sigma_0|e_0\rightarrow \sigma_1|e_1$.
\end{Assumption}


\begin{Assumption}[Subject Reduction Base]
if $\Sigma^{\sigma_0};\emptyset\vdash e_0: T_0$,
$\sigma_0|e_0\rightarrow \sigma_1|e_1$,
then
$\Sigma^{\sigma_1};\emptyset\vdash e_1: T_1$.
\end{Assumption}


\noindent If the result of a field access is mutable,
the receiver is also mutable:\saveSpace\saveSpace
\begin{Assumption}[Mut Field]
\ \\
\indent(1)\ if $\Sigma;\Gamma\vdash\e\singleDot\f:\Kw{mut}\,\_$
then $\Sigma;\Gamma\vdash\e:\Kw{mut}\,\_$
 and 
\\*\indent(2)
if $\Sigma;\Gamma\vdash\e_0\singleDot\f\equals\e_1:T$
then $\Sigma;\Gamma\vdash\e_0:\Kw{mut}\,\_$.
\end{Assumption}

\noindent An object is not part of the ROG of its immutable or capsule fields:\saveSpace\saveSpace
\begin{Assumption}[Head Not Circular]
if
$\Sigma^\sigma;\Gamma\vdash l:T$,
then $l\notin\text{erog}(\sigma,l)$.
\end{Assumption}


\noindent In a well typed $\sigma$ and $e$, if mutatable $l_2$ is reachable from
$l_1$, and $l_1$ is reachable from $l_0$,
then all the paths connecting $l_0$ and $l_2$ pass trough $l_1$; thus
if we were to remove $l_1$ from the object graph, $l_0$ would no longer reach $l_2$:
\saveSpace\saveSpace
\begin{Assumption}[Capsule Tree]
If   $\Sigma^\sigma;\Gamma\vdash \e:\T$,
$l_2\in\text{erog}(\sigma,l_1)$,
$l_1\in\text{erog}(\sigma,l_0)$,\\*
and
$\mathit{mutatable}(l_2,\sigma,\e)$
then 
$l_2\notin\text{erog}(\sigma\setminus l_1,l_0)$.
\end{Assumption}


Capsule Tree and Head Not Circular together 
imply that capsule fields section the object graph into a tree of nested `balloons',
where nodes are mutable encapsulated objects and
edges are given by reachability between those objects in the original memory:
$l_2$ is in the encapsulated ROG of $l_1$;
$l_2$ is mutatable and reachable through $l_1$, thus
it must be reachable by a \Q@capsule@ field.
Thanks to Head Not Circular and $l_1\in\text{erog}(\sigma,l_0)$ we can derive 
$l_0\notin\text{erog}(\sigma,l_1)$.

The execution of an expression
with no \Q@mut@ free variables is deterministic and does not
mutate pre existing memory (and thus does not not perform I/O by mutating pre existing $c$):
\begin{Assumption}[Determinism]
if $\emptyset;\Gamma\vdash \e:\T$, 
$\forall x \Gamma(x)\neq\Kw{mut}\,\_$, and
$\sigma | \e'\rightarrow^+ \sigma' | \e''$
then 
$\sigma | \e'\Rightarrow^+ \sigma,\_ | \e''$,
where $\e'=\e[x_1=l_1,\ldots,x_n=l_n]$ and $\Sigma^\sigma;\emptyset\vdash \e':\T$
\end{Assumption}

\begin{Assumption}[StrongExceptionSafety]
if $\Sigma^{\sigma,\sigma'};\emptyset\vdash \ctx[\Kw{try}^\sigma\oC\e_0\cC\ \Kw{catch}\ \oC\e_1\cC]:\T$
and\\*
$
\sigma,\sigma'|\ctx[\Kw{try}^\sigma\oC\e_0\cC\ \Kw{catch}\ \oC\e_1\cC]\rightarrow 
\sigma''|\ctx[\Kw{try}^\sigma\oC\e'\cC\ \Kw{catch}\ \oC\e_1\cC]
$
then 
$\sigma''=\sigma,\_$
and
$\Sigma^\sigma;\emptyset\vdash \ctx[\e_1]:\T$
\end{Assumption}
\noindent
For each \Q@try-catch@, execution preserves the memory needed to continue the execution in case of error
(the memory visible outside of the \Q@try@).%

%Thanks to how our reduction rules are designed, especially \textsc{try error},
%@Progress will need to rely on @StrongExceptionSafety internally.

Note that our last well formedness rule requires 
\textsc{update} and \textsc{mcall} to introduce
monitor expressions only over locations
that are not preserved by \Q@try@ blocks.
This can be achieved, since monitors are introduced
around mutation operations
(and \Q@new@ expression),
and Strong Exception Safety ensures no mutation happens on preserved memory.

% To the best of our knowledge, only the type system of 42~\cite{ServettoEtAl13a,ServettoZucca15}
%  supports all these assumptions out of the box,
% while both Gordon~\cite{GordonEtAl12} and Pony~\cite{clebsch2015deny,clebsch2017orca} supports all except StrongExceptionSafety,
% however it should be trivial to modify them to support it:
% the \Q@try-catch@ rule could be modified to
% $\emptyset;\Gamma\vdash\Kw{try}\ \oC\e_0\cC\ \Kw{catch}\ \oC\e_1\cC:\T$
% if\\* $\emptyset;
% \Gamma,\{x:\Kw{read}\,C | x:\Kw{mut}\,C\,\in\Gamma\}
% \vdash\e_0:\T$ and $\emptyset;\Gamma\vdash\e_1:\T$,
% i.e. $e_0$ can be typed when seeing all externally defined mutable references as \Q@read@.

\loseSpace
\noindent\textit{Statement of Soundness:}
%We first need to define what it means for an object to be valid:
An object is \emph{valid} iff calling its \Q@invariant@ method would
deterministically produce \Q@true@ in a finite number of steps, i.e. it does not evaluate to \Q@false@, fail to terminate, or produce an error.
We also require evaluating \Q@invariant@ to preserve existing memory ($\sigma$): but new objects ($\sigma'$) can be created and freely mutated.

\indent$valid(\sigma,l)$ iff $\sigma | l.invariant()\Rightarrow^+ \sigma,\sigma' | \text{\Q@true@}$.\loseSpace

\noindent In order for a class invariant to be meaningful it needs to be possible for it to potentially observe an invalid object. However, invalid objects should not be observed outside of \@@invariant@.
For this purpose we define the set of trusted steps, 
as the call to \Q@invariant@ and the field accesses inside its evaluation.
Note that just the single small step reduction
of calling \Q@invariant@ is trusted, not the whole evaluation of the \Q@invariant@ expression.

\loseSpace
\noindent $\mathit{trusted}(\ctx_v,r_l)$ iff\\*
\indent either
$r_l=l.invariant()$ and
 $\ctx_v=\ctx_v'[$\Q@M(@$l$\Q@;@$v$\Q@;@$\square$\Q@)@$]$,\\*
\indent or
$r_l=l$\Q@.f@ and
 $\ctx_v=\ctx_v'[$\Q@M(@$l$\Q@;@$v$\Q@;@$\ctx_v''$\Q@)@$]$.
\loseSpace

\noindent Finally, we can now define what it means for a language to soundly enforce our invariant protocol: every object involved in any untrusted redex is valid.

\begin{theorem}[Soundness]
if $c:\Kw{Cap};\emptyset\vdash \e: \T$ and
$c\mapsto\Kw{Cap}\{\_\}|\e\rightarrow^+ \sigma|\ctx_v[r_l]$, then
either $valid(\sigma,l)$ or $\mathit{trusted}(\ctx_v,r_l)$.
\end{theorem}


We believe this property captures very precisely our statements in section~\ref{s:protocol}.

It is hard to prove Soundness directly,
so we first define a stronger property,
called \emph{Stronger Soundness} and
show that it is preserved during reductions by means of conventional
Progress and Subject Reduction (Progress is one of our assumptions,
while Subject Reduction relies heavily upon Subject Reduction Base).
That is,
Progress $\wedge$ Subject Reduction $\Rightarrow$ Stronger Sound Validation,
\\*Stronger Soundness $\Rightarrow$ Soundness.

%The structure of the proof is interesting:
%It is hard to prove Sound Validation directly,
%so we first define a stronger property,
%called Stronger Sound Validation and
%we show that it is preserved during reduction by mean of conventional Progress and Subject Reduction.
%That is,
%Progress+Subject Reduction $\Rightarrow$ Stronger Sound Validation
%and Stronger Sound Validation $\Rightarrow$ Sound Validation.
