\documentclass[a4paper,UKenglish]{lipics-v2018}
\setcounter{errorcontextlines}{999}
%This is a template for producing LIPIcs articles. 
%See lipics-manual.pdf for further information.
%for A4 paper format use option "a4paper", for US-letter use option "letterpaper"
%for british hyphenation rules use option "UKenglish", for american hyphenation rules use option "USenglish"
% for section-numbered lemmas etc., use "numberwithinsect"

\usepackage{microtype}%if unwanted, comment out or use option "draft"
%\graphicspath{{./graphics/}}%helpful if your graphic files are in another directory
\bibliographystyle{plainurl}% the recommnded bibstyle

\usepackage{verbatim}
%\addbibresource{main.bib}
\usepackage{wrapfig}


%
\makeatletter
\DeclareOldFontCommand{\rm}{\normalfont\rmfamily}{\mathrm}
\DeclareOldFontCommand{\sf}{\normalfont\sffamily}{\mathsf}
\DeclareOldFontCommand{\tt}{\normalfont\ttfamily}{\mathtt}
\DeclareOldFontCommand{\bf}{\normalfont\bfseries}{\mathbf}
\DeclareOldFontCommand{\it}{\normalfont\itshape}{\mathit}
\DeclareOldFontCommand{\sl}{\normalfont\slshape}{\@nomath\sl}
\DeclareOldFontCommand{\sc}{\normalfont\scshape}{\@nomath\sc}
\makeatother

\usepackage{mathpartir}
\usepackage{amsmath}
\usepackage{amsthm}

\theoremstyle{plain}
\newcounter{definition}
\newtheorem{Definition}[definition]{Definition}
\newcounter{assumption}
\newtheorem{Assumption}[assumption]{Assumption}
\newcounter{lemma}
\newtheorem{Lemma}[lemma]{Lemma}

%
\usepackage{xspace}
\usepackage{listings}
\usepackage{xcolor}
\usepackage{letltxmacro}
\usepackage{mathtools}
\usepackage{mathpartir}
%\usepackage{stix}

\definecolor{darkRed}{RGB}{100,0,10}
\definecolor{darkBlue}{RGB}{10,0,100}
\newcommand*{\ttfamilywithbold}{\fontfamily{pcr}\selectfont}
%\newcommand*{\ttfamilywithbold}{\ttfamily}

%found on http://tex.stackexchange.com/questions/4198/emphasize-word-beginning-with-uppercase-letters-in-code-with-lstlisting-package
%\lstset{language=FortyTwo,identifierstyle=\idstyle}
%
\makeatletter
\newcommand*\idstyle{%
        \expandafter\id@style\the\lst@token\relax
}
\def\id@style#1#2\relax{%
        \ifcat#1\relax\else
                \ifnum`#1=\uccode`#1%
                        \ttfamilywithbold\bfseries
                \fi
        \fi
}
\makeatother

\lstset{language=Java,
  basicstyle=\upshape\ttfamily\footnotesize,%\small,%\scriptsize,
  keywordstyle=\upshape\bfseries\color{darkRed},
  showstringspaces=false,
  mathescape=true,
  xleftmargin=0pt,
  xrightmargin=0pt,
  breaklines=false,
  breakatwhitespace=false,
  breakautoindent=false,
 identifierstyle=\idstyle,
 morekeywords={method,Use,This,constructor,as,into,rename},
 deletekeywords={double}
}

\newcommand*{\SavedLstInline}{}
\LetLtxMacro\SavedLstInline\lstinline
\DeclareRobustCommand*{\lstinline}{%
	\ifmmode
	\let\SavedBGroup\bgroup
	\def\bgroup{%
		\let\bgroup\SavedBGroup
		\hbox\bgroup
	}%
	\fi
	\SavedLstInline
}

\newcommand\saveSpace{\vspace{-2pt}}

\newcommand\Rotated[1]{\begin{turn}{90}\begin{minipage}{12em}#1\end{minipage}\end{turn}}

\newcommand{\Q}{\lstinline}
\newenvironment{bnf}{$\begin{array}{lcll}}{\end{array}$}
\newcommand{\production}[3]{%
	\text{\itshape #1}&%
	\!\!\!\!\!\Coloneqq\!\!\!\!\!&%
	\text{\itshape #2}&%
	\!\!\!\!\!\mbox{#3}}
%\newcommand{\prodFull}[3]{#1&::=&\mbox{#2}&\mbox{#3}}
\newcommand{\prodInline}[2]{#1\Coloneqq#2}
\newcommand{\prodNextLine}[2]{&&#1&\mbox{#2}}
\newcommand{\terminal}[1]{\ensuremath{$\texttt{#1}$}}
%\newcommand{\metavariable}[1]{\ensuremath{\mathit{#1}}}

\newcommand\Rulename[1]{{\textsc{#1}}}
\newcommand\ctx[1]{\ensuremath{\mathcal{E}_#1}\!}
\newcommand{\lib}[3]{\Q@\{@\!#1\Q{;}\ #2 \Q{;}\ #3\Q@\}@}
\newcommand{\rp}[1]{\Q{(}\!#1\Q{)}}
\newcommand{\red}[3]{#1\rp{#2\Q{=}#3}}
\newcommand{\summ}[2]{#1\ \Q{<+}\ #2}
\newcommand{\mmid}{\ensuremath{\mid}}
\newcommand{\hole}{\ensuremath{\square}}
%--------------------------
\newcommand{\mynotes}[3]{{\color{#2} {\sc #1}: #3}}
\newcommand\isaac[1]{\mynotes{Isaac}{red}{#1}}
\newcommand\marco[1]{\mynotes{Marco}{blue}{#1}}


\newcommand{\saveSpace}{\vspace{-3px}}
\newcommand{\loseSpace}{\vspace{1ex}}

\newcommand{\REV}[3]{%
	\textcolor{red}{#1\footnote{
		\textcolor{red}{\textbf{REV#2{:} #3}}}}}


\title{Sound Invariant Checking Using Type Modifiers and Object Capabilities.}
%\titlerunning{Dummy short title}%optional, please use if title is longer than one line
\author{Authors omitted for double-bind review.}{Unspecified Institution.}{}{}{}
\authorrunning{Authors omitted for double-bind review.} %mandatory. First: Use abbreviated first/middle names. Second (only in severe cases): Use first author plus 'et al.'
\Copyright{Authors omitted for double-bind review.} %mandatory, please use full first names. LIPIcs license is "CC-BY";  http://creativecommons.org/licenses/by/3.0/

\begin{comment}
	%mandatory, please use full name; only 1 author per \author macro; first two parameters are mandatory, other parameters can be empty.
	\author{Isaac Oscar Garian}{Victoria University of Wellington}{isaac@ecs.vuw.ac.nz}{}{}
	\author{Marco Servetto}{Victoria University of Wellington}{marco.servetto@ecs.vuw.ac.nz}{}{}
	\author{Alex Potanin}{Victoria University of Wellington}{alex@ecs.vuw.ac.nz}{}{}
	\authorrunning{Isaac O.\,G., M. Servetto, and A. Potanin} %mandatory. First: Use abbreviated first/middle names. Second (only in severe cases): Use first author plus 'et al.'
	\Copyright{Isaac Oscar G. and Marco Servetto} %mandatory, please use full first names. LIPIcs license is "CC-BY";  http://creativecommons.org/licenses/by/3.0/
\end{comment}

\subjclass{Dummy classification}% mandatory: Please choose ACM 2012 classifications from https://www.acm.org/publications/class-2012 or https://dl.acm.org/ccs/ccs_flat.cfm . E.g., cite as "General and reference $\rightarrow$ General literature" or \ccsdesc[100]{General and reference~General lwrapfigureiterature}. 

\begin{comment}
\begin{CCSXML}
<ccs2012>
<concept>
<concept_id>10003752.10010124.10010138.10010139</concept_id>
<concept_desc>Theory of computation~Invariants</concept_desc>
<concept_significance>500</concept_significance>
</concept>
<concept>
<concept_id>10003752.1
%
\makeatletter
\DeclareOldFontCommand{\rm}{\normalfont\rmfamily}{\mathrm}
\DeclareOldFontCommand{\sf}{\normalfont\sffamily}{\mathsf}
\DeclareOldFontCommand{\tt}{\normalfont\ttfamily}{\mathtt}
\DeclareOldFontCommand{\bf}{\normalfont\bfseries}{\mathbf}
\DeclareOldFontCommand{\it}{\normalfont\itshape}{\mathit}
\DeclareOldFontCommand{\sl}{\normalfont\slshape}{\@nomath\sl}
\DeclareOldFontCommand{\sc}{\normalfont\scshape}{\@nomath\sc}
\makeatother
%0010124.10010138.10010142</concept_id>
<concept_desc>Theory of computation~Program verification</concept_desc>
<concept_significance>500</concept_significance>
</concept>
<concept>
<concept_id>10011007.10011006.10011008.10011009.10011011</concept_id>
<concept_desc>Software and its engineering~Object oriented languages</concept_desc>
<concept_significance>500</concept_significance>
</concept>
<concept>
<concept_id>10011007.10010940.10010992.10010998.10011001</concept_id>
<concept_desc>Software and its engineering~Dynamic analysis</concept_desc>
<concept_significance>300</concept_significance>
</concept>
<concept>
<concept_id>10011007.10011006.10011008.10011024.10011032</concept_id>
<concept_desc>Software and its engineering~Constraints</concept_desc>
<concept_significance>300</concept_significance>
</concept>
</ccs2012>
\end{CCSXML}

\ccsdesc[500]{Theory of computation~Invariants}
\ccsdesc[500]{Theory of computation~Program verification}
\ccsdesc[500]{Software and its engineering~Object oriented languages}
\ccsdesc[300]{Software and its engineering~Dynamic analysis}
\ccsdesc[300]{Software and its engineering~Constraints}
\end{comment}

\keywords{type modifiers, object capabilities, runtime verification, class invariants}%mandatory
% \category{}%optional, e.g. invited paper
%\relatedversion{}%optional, e.g. full version hosted on arXiv, HAL, or other respository/website
%\supplement{}%optional, e.g. related research data, source code, ... hosted on a repository like zenodo, figshare, GitHub, ...
%\funding{}%optional, to capture a funding statement, which applies to all authors. Please enter author specific funding statements as fifth argument of the \author macro.
%\acknowledgements{I want to thank \dots}%optional

%Editor-only macros:: begin (do not touch as author)%%%%%%%%%%%%%%%%%%%%%%%%%%%%%%%%%%
\EventEditors{John Q. Open and Joan R. Access}
\EventNoEds{2}
\EventLongTitle{42nd Conference on Very Important Topics (CVIT 2016)}
\EventShortTitle{CVIT 2016}
\EventAcronym{CVIT}
\EventYear{2016}
\EventDate{December 24--27, 2016}
\EventLocation{Little Whinging, United Kingdom}
\EventLogo{}
\SeriesVolume{42}
\ArticleNo{23}
%\nolinenumbers %uncomment to disable line numbering
%\hideLIPIcs  %uncomment to remove references to LIPIcs series (logo, DOI, ...), e.g. when preparing a pre-final version to be uploaded to arXiv or another public repository
%%%%%%%%%%%%%%%%%%%%%%%%%%%%%%%%%%%%%%%%%%%%%%%%%%%%%%

\begin{document}

\maketitle

\begin{abstract}
{{\ensuremath{\mathit{e}}
		\xspace}}
	
	
${}_{}$
\noindent\textit{Context:} % What is the broad context of the work? What is the importance of the general research area?
Object-oriented programming languages through sub-typing and dynamic dispatch provide great flexibility: they
allow code to be adapted/specialised to behave differently in different contexts.
%%, which is made even more complex by dynamic class loading (supported by many mainstream OO languages).
However this flexibility hampers code reasoning since object behaviour is usually nearly completely
unrestricted. This is further complicated with the support OO languages typically have for exceptions,
memory mutation, and I/O.
Class invariants are a well known technique to help write correct code, however
there are various different interpretations of when they should hold.
% invariant protocols, specifying when the invariant is expected to hold and when is checked. 
%% In the absence of on the fly static verification of dynamically loaded code, it is difficult for programmers to write code that is correct in a library setting.

\loseSpace
\noindent\textit{Inquiry:} %What problem or question does the paper address? How has this problem or question
%been addressed by others (if at all)?
We investigate a protocol where class-invariants are expected to hold for all objects involved in execution, but
they are soundly checked only in a few points. Our approach is sound since we ensure that invariants are pure
and checked immediately after an invariant violation could have occurred. This allows for stronger and simpler reasoning, as well as less runtime verification overhead, compared to the
more common approach of `visible state' semantics where only the invariant of \Q@this@ is required to hold, and
only at the start and end of instance methods.


% for all observable objects, their class-invariant holds.
%We wish to guarantee that a class's invariant holds for all its observable instances.

% Most prior work on enforcing class invariants assume unverified/unverifiable restrictions on library code, dynamic class loading and I/O, or even just trust programmers to use the provided tools correctly.
% For example, static verification often restricts dynamic class loading, while run-time verification often unsoundly allows non deterministic code in class-invariants.

%We wish to allow for stronger reasoning over this by ensuring that non boke
%We wish to guarantee that a user defined property holds for all observable instances of a class.
%This is a variation of class invariants, where objects in a broken state can never be observed.

\loseSpace
\noindent\textit{Approach:} %What was done that unveiled new knowledge?
We combine previous works on type modifiers and object capabilities to ensure properties about purity,
determinism, aliasing, and mutability control. Type modifiers and object capabilities are useful in their own
right, but we use these facilities here to ensure our sparsely injected run-time checks are sufficient.
%We have implemented our protocol on top of the L42 language.

\loseSpace
\noindent\textit{Knowledge:} %What new facts were uncovered? If the research was not results oriented, what new capabilities are enabled by the work?
Through examples, we show how hard it is to reason about code behaviour in the context of dynamic dispatch and
I/O. We show how type modifiers and object capabilities lay down the foundation needed to reason about OO code
in this context, and that \REV{without such reasoning, enforcing class-invariants for all objects involved in
execution becomes nearly impossible.
We show that type-modifiers and object capabilities allow for such reasoning to be both simple, modular, and
sound in a context where I/O, exceptions, and dynamic-dispatch are present.}{1}{Not sufficiently substantiated.}

\loseSpace
\noindent\textit{Grounding:} %What argument, feasibility proof, artifacts, or results and evaluation support this work?
We demonstrate, by means of a case study, empirical evidence of the conciseness of our approach in comparison to
Spec\#.
We also show the performance of our protocol in comparison to the conventional runtime-verification approaches
taken by D and \REV{Eiffel}{2}{Old}.

We formally model a class of languages that support our protocol. This model is formally verified by a proof,
parametric on the presence of a type system which guarantees certain properties of type modifiers. Type systems
with such properties have already been explored, formalized, and proved in prior work.

\loseSpace
\noindent\textit{Importance:} %Why does this work matter?
Class-invariants allow programmers to reason as to what states objects might be in. \REV{\REV{Our approach
makes such reasoning both sound and strong}{1}{Not sufficiently substantiated}}{3}{I think this is talking  
about the fact that prior work does not fully deal with I/O, exceptions, etc. in a suitable way, but I'm not sure.}: if a programmer has access to an object, it and 
all objects reachable from it are in a \REV{valid state}{1}{Definition is unclear}.
In addition our approach works independently of the behaviour of potentially buggy or malicious code, provided all code of course type checks. \REV{Since our approach also works on existing emerging languages}{1}{Not sufficiently substantiated.}, and adds only marginal
program complexity, we believe it is also practical and usable.

%TODO
% \REVComm{Our work represents a philosophical shift}{2}{This makes it sound like run-time verification has
% never been considered before.} similar to the jump from unchecked casts in C to checked casts in Java: it is the programmer's responsibility to create valid objects and to preserve validation
% while mutating objects, however a validation failure is soundly detected by a run-time exception,
% and even after capturing such an exception, validation still holds for all objects involved in the execution.
%Class invariants provide guarantees about the state of objects throughout the execution.
%Runtime verification of class invariants is
%a hard problem due to issues with aliasing, exceptions,
%non-deterministic invariants, I/O, subtyping and untrusted code.
%We challenge this problem in the context of
%a Java-like language where the invariants are expressed in the language itself.
%We formally define \textbf{Sound Invariant Checking}
%and formally prove that a combination of carefully selected type modifiers, object capabilities,
% and strong exception safety is sufficient
%to handle Sound Invariant Checking for the most common categories of objects.

 \end{abstract}
\section{Introduction}
\label{s:intro}
%\newpage
%\LINE
Representation invariants (sometimes called class invariants or object invariants) are
a useful concept when reasoning about software correctness in OO (Object Oriented) languages. Such invariants are predicates on the state of an object and its ROG (Reachable Object Graph).
They can be presented as documentation, checked as part of static verification, or, as we do in this paper, monitored for violations using runtime verification.
In our system, a class specifies its invariant by defining a method called \Q@invariant()@
that returns a boolean.
We say that an object's invariant holds when its \Q@invariant()@ method would return \Q@true@.\footnote{We do this (as in Dafny~\cite{DBLP:conf/sigada/Leino12}) to minimise the special treatment of invariants, whereas other approaches often treat invariants as a special annotation with its own syntax.}

Invariants are designed to hold most of the time, however it is commonly required to (temporarily) violate invariants while performing complex sequences of mutations.
To support this behaviour, most invariant protocols present in the literature allow invariants to be broken and observed broken.
The two main protocols are the \emph{visible state semantics} \cite{Meyer:1988:OSC:534929} and the \emph{Pack-Unpack/Boogie methodology}~\cite{DBLP:journals/jot/BarnettDFLS04}.
In the visible state semantics, invariants can be broken when a method on the object is active (that is, currently executing).
Some interpretations of the visible state are more permissive, requiring the invariants of receivers to hold only before and after every public method call, and after constructors. 
In the pack-unpack approach, objects are either in a `packed' or `unpacked' state, 
the invariant of `packed' objects must hold, whereas unpacked objects can be broken.

%------------
In this paper we propose a much stricter invariant protocol: at all times, the invariant of every object involved in execution must hold; thus they can be broken when the object is not (currently) involved in execution. 
An object is \emph{involved in execution} when it is in the ROG of any of the objects mentioned in the method call, field access, or field update that is about to be reduced; we state this more formally later in the paper.

%Our strict invariant protocol clearly supports easier reasoning; however 
Our strict protocol supports easier reasoning: an object can never be observed broken. However 
at first glance it may look overly restrictive, preventing useful program behaviour.
Consider the iconic example of a \Q@Range@ class, with a \Q@min@ and \Q@max@
value, where the invariant requires that \Q@min<=max@:
% ISAAC: I changed the example to not use getters and setters as our latter examples often don't and it dosn't add anything extra; in addition, this may fall foul of some visibile state semantics of the setters are considered public.
\begin{lstlisting}
class Range{ private field min; private field max;
  method invariant(){return min<max;}
  method set(min, max){
    if(min>=max){throw new Error(/**/);}
    this.min = min;
    this.max = max; }}
\end{lstlisting}
In this example we omit types to focus on the runtime semantics.
The code of \Q@set@ does not violate visible state semantics:
\Q@this.min@ \Q@=@ \Q@min@ may temporarily break the invariant of \Q!this!, however it will be fixed after executing \Q@this.max@ \Q@=@ \Q@max@. Visible state allows such temporary breaking of invariants since we are inside a method on \Q!this!, and by the time it returns, the invariant will be re-established.
However, if \Q@min@ is $\geq$ \Q@this.max@, \Q@set@ will violate our stricter approach. The execution of
\Q@this.min@ \Q@=@ \Q@min@ will break the invariant of \Q@this@ and \Q@this.max@ \Q@=@ \Q@max@ would then involve a broken object. If we were to inject a call
\Q@Do.stuff(this);@ between the two field updates, arbitrary user code could observe a broken object; 
  adding such a call is however allowed by visible state semantics.

Using the \emph{box pattern}, we can provide a modified
\Q@Range@ class with the desired client interface, while respecting the principles of our strict protocol:
\begin{lstlisting}
class BoxRange{//no invariant in BoxRange
  field min; field max;
  BoxRange(min, max){ this.set(min, max); }
  method Void set(min, max){
    if(min>=max){throw new Error(/**/);}
    this.min = min; this.max = max;
    }  }
class Range{ private field box; //box contains a BoxRange
  Range(min, max){ this.box = new BoxRange(min, max); }
  method invariant(){ return this.box.min < this.box.max; }
  method set(min, max){ return this.box.set(min,max); }
  }
\end{lstlisting}
The code of \Q@Range.set(min,max)@ does not violate our protocol.
% since \Q@this@ is not in the ROG of \Q@this.box@, \Q!min!, or \Q!max!. 
The call to
\Q@BoxRange.set(min,max)@ works in a context where the \Q@Range@ object is
unreachable, and thus not involved in execution.
That is, the \Q@Range@ object is not in the ROG of the receiver or the parameters of \Q@BoxRange.set(min,max)@.
 Thus \Q@Range.set(min,max)@ can temporarily break the \Q@Range@'s invariant.
By using the \Q@box@ field as an extra level of indirection, we restrict the set of objects involved in execution while the state of the object \Q@Range@ is modified.%
\footnote{Due to its simplicity and versatility, we do not claim this pattern to be a contribution of our work, as we expect others to have used it before. We have however not been able to find it referenced with a specific name in the literature, though technically speaking, it is a simplification of the Decorator, but with a different goal.
While in very specific situations the overhead of creating such additional box object may be unacceptable, we designed our work for environments where such fine performance differences are negligible.
Also note that many VMs and compilers can optimize away wrapper objects in many circumstances.~\cite{Bolz:2011:ARP:1929501.1929508}}
With appropriate type annotations, the code of \Q@Range@ and \Q@BoxRange@ is accepted as correct by our system: no matter how \Q@Range@ objects are used, a broken \Q@Range@ object will never be involved in execution.

\subheading{Contributions}
Invariant protocols allow for objects to make necessary changes that might make their invariant temporarily broken.
In visible state semantics any object that has an active method call anywhere on the call stacks is potentially invalid;
arguably not a sufficent guarantee as observed by
Gopinathan \etal's.~\cite{Gopinathan:2008:RMO:1483018.1483028}
Approaches such as \textit{pack/unpack}~\cite{DBLP:journals/jot/BarnettDFLS04} 
represent potentially invalid objects in the type system; this
encumbers the type system and the syntax with features whose only purpose is to distinguish objects with broken invariants.
%, while (at least in the case of Spec\#) still not soundly supporting I/O and exceptions.
The core insight behind our work 
is that we can use a small number of decorator-like design patterns to avoid exposing those potentially invalid objects 
in the first place, thus avoiding the need of representing them at the type level.

%In this paper we present  
%a general purpose language
% that does not require any \emph{invariant-specific} language mechanisms
% but instead we show how a clever use of capabilities
% is sufficient to capture invariant guarantees
% that are comparable to the state of the art object-oriented verification systems.

%Reasoning about class invariants in object-oriented verification needs to allow for objects to make necessary changes that might make their invariant temporarily broken. 
%The state of the art ranges from \textit{visible state semantics} that makes \textit{any} object that has an active method call anywhere on the call stacks potentially \textit{invalid} --- arguably not a very useful guarantee as observed by Gopinathan et al.~\cite{Gopinathan}.
%On the other hand, approaches such as \textit{pack/unpack}~\cite{SpecSharp} modify the type system with features that serve no general purpose other than help expose the semantics behind broken object invariants.
%The core insight behind our work is that we can design a general purpose language as presented in this paper that does not require any \emph{invariant-specific} language mechanisms but instead we show how a clever use of capabilities and small number of design patterns (Decorator-like \textit{Box} and \textit{Transformer}) is sufficient to capture invariant guarantees that are comparable to the state of the art object-oriented verification systems.

In the remainder of this paper, we discuss how to combine runtime checks and capabilities
to soundly enforce our strict invariant protocol.
Our solution only requires 
that all code is well-typed, and works in the presence of mutation, I/O, non-determinism, and exceptions, all under an open world assumption.

We formalise our approach and, in Appendix~\ref{s:proof}, prove that our use of Reference and Object Capabilities soundly enforces our invariant protocol.

We have fully implemented our protocol in L42\footnote{
Our implementation is implemented by checking that a given class conforms to our protocol, and injecting invariant checks in the appropriate places.
An anonymised version of L42, supporting the protocol described in this paper, together with the full code of our case studies, is available at \url{http://l42.is/InvariantArtifact.zip}. %We believe it would be easy to port our work on Pony and Gordon \etal's language.
}, we used this implementation to implement many case studies, showing that our protocol is more succinct than the pack/unpack approach and much more efficient then the visible state semantic.
It is important to note that unlike most prior work, we soundly handle catching of invariant failures and I/O.
%In our case study we show that
%we can still encode most of the examples explored in ~\cite{???} (including for example mutable collections of immutable objects) whilst having a significantly lower annotation-burden.
%--I think we can avoid this to save space
%Section \ref{s:TMsAndOCs} explains the pre-existing \emph{type modifier} features we use for this work.
%Section \ref{s:protocol} explains the details of our invariant protocol, and Section \ref{s:formalism} formalises a language enforcing this protocol.
%Sections \ref{s:immutable} and \ref{s:encapsulated} explain and motivate how our protocol can handle invariants over immutable and encapsulated mutable data, respectively.
%Section \ref{s:case-study} presents our GUI case study and compares it against visible state semantics and Spec\#: they performed 5 orders of magnitude more invariant checks, and required 60\% more annotations, respectively.
%Sections \ref{s:related} and \ref{s:conclusion} provide related work and conclusions.
We describe our case studies in Section~\ref{s:case-studyAll}.
Our approach may seem very restrictive;
the programming patterns in Section~\ref{s:patterns} show how our approach does not hamper expressiveness; in particular we show how batch mutation operations can be performed with a single invariant check, and how the state of a `broken' object can be safely passed around.
% In Appendix \ref{s:runtime-verification}, we discuss more related work on runtime verification.


%you see this already later on... I wanted to avoid repating it
%to perform batch operations with a single invariant check, as well as how the state of `broken' objects can be passed around.}	
%http://www.cs.cmu.edu/~NatProg/papers/p496-coblenz-Glacier-ICSE-2017.pdf

\section{Type Modifiers and Object Capabilities Background}
\label{s:TMsAndOCs}
Reasoning about imperative object-oriented (OO) programs is a non trivial task,
made particularly difficult by mutation, aliasing, dynamic dispatch, I/O, and exceptions. There are many ways to perform such reasoning, here we use the type system to restrict, but not prevent such behaviour in order to be able to soundly enforce invariants with runtime verification (RV).
% [dynamic class loading],

\subheading{Type Modifiers (TMs)}
TMs, as used in this paper, are a type system feature that allows reasoning about aliasing and mutation. Recently a new design for them has emerged that radically improves their usability;
three different research languages are being independently developed relying on this new design: the language of Gordon \etal~\cite{GordonEtAl12}, Pony~\cite{clebsch2015deny,clebsch2017orca}, and L42~\cite{ServettoZucca15,ServettoEtAl13a,JOT:issue_2011_01/article1,GianniniEtAl16}.
These projects are quite large: several million lines of code are written in Gordon \etal's language and are used by a large private Microsoft project; Pony and L42 have large libraries and are active open source projects. In particular the TMs of these languages are used to provide automatic and correct parallelism~\cite{GordonEtAl12,clebsch2015deny,clebsch2017orca,ServettoEtAl13a}.

While we focus on the specific TMs provided by L42, Pony, and Gordon \etal, type modifiers
 are a well known language mechanism~\cite{TschantzErnst05,BirkaErnst04,OstlundEtAl08,clebsch2015deny,GianniniEtAl16,GordonEtAl12}
 that allow static reasoning about mutability and aliasing properties of objects.
With slightly different names and semantics, the four \IODel{most common} modifiers \IO{we use} for \IO{object} references \IO{(i.e. expressions and variables)} \IODel{to objects} are:
\begin{itemize}
\item Mutable (\Q@mut@): the referenced object can be mutated\IO{, and freely shared/aliased}, as in most imperative languages without modifiers.
If all types are \Q@mut@, there is no restriction on aliasing/mutation.
\item \IODel{Readonly (\Q@read@): \ldots}
\item Immutable (\Q@imm@): the referenced object \IODel{can never mutate} \IO{cannot mutate, not even through other aliases}. \IODel{Like \Q@read@ references, one cannot mutate through an \Q@imm@ reference, however \Q@imm@ references also guarantee that the referenced object will not mutate through any other alias.} \IO{We call an object referred to by such a reference, an \emph{immutable object}; note that we don't prevent such an object from being mutated \emph{before} an immutable reference to it is made.} \IOComm{I'm not using the ROG definition of immutability, as we don't say our modifiers are deep untill later.}
\item Readonly (\Q@read@): the referenced object cannot be mutated by such references, but there may \IO{also} be mutable aliases to \IODel{such} \IO{the same} object, thus mutation can still be observed. \IO{Readonly references can refer to both mutable and immutable objects, since \Q!read! is a supertype of both \Q!imm! and \Q!mut!.}
\item Encapsulated (\Q@capsule@):
 \IODel{everything} \IO{every non-immutable object} in the reachable object graph (ROG) of a capsule reference (including itself) is \IODel{mutable} only \IO{reachable} through that reference\IODel{; \REV{however immutable references can be freely shared across capsule boundaries}{B}{what does [this] mean}}. \IO{This means that for any expression $e$ and $e'$, if $e$ is typable as \Q!capsule!, then the result of $e'$ is not in the ROG of the result of $e$, unless $e'$ evaluates to an immutable object (only possible if $e'$ is not typeable as \Q!mut! or \Q!capsule!).} \IO{An encapsulated reference can be freely promoted as mutable or immutable, since there could have been no other references to it.}
\end{itemize}
%In the context of object-oriented programming, type modifiers may also apply to the implicit \Q@this@ parameter in method declarations, restricting the type of references the method can be called on. In addition, due to the deep meanings we type field access on object references to be the most restrictive of the object references modifier and the field’s. As \Q@read@ references impose no assumptions about aliasing, any \Q@imm@ or \Q@mut@ expression can be safely implicitly promoted to \Q@read@, whereas other conversions are not generally safe.
%\loseSpace

\noindent TMs are different to field or variable modifiers like Java's \Q@final@: TMs apply to references, whereas \Q@final@ applies to fields themselves. Unlike a variable/field of a \Q@read@ type, a \Q@final@ variable/field cannot be reassigned, it always refers to the same object, however the variable/field can still be used to mutate the referenced object.
On the other hand, an object cannot be mutated through a \Q@read@ reference, however a \Q@read@ variable can still be reassigned.\footnote{In C, this is similar to the difference between \Q@A* const@ (like \Q@final@) and \Q@const A*@ (like \Q@read@), where \Q@const A* const@ is like \Q@final read@.}
%\end{itemize}

Consider the following  example usage of \Q@mut@, \Q@imm@, and \Q@read@, where we can observe a change in \Q@rp@ caused by a mutation inside \Q@mp@.
\begin{lstlisting}
 mut Point mp = new Point(1, 2);   mp.x = 3; // ok
 imm Point ip = new Point(1, 2); $\Comment{}$ip.x = 3; // type error
read Point rp = mp;              $\Comment{}$rp.x = 3; // type error
$\text{\IODel{// ok, read is a common supertype of imm/mut}}$
mp.x = 5; // ok, now we can observe rp.x == 5
ip = new Point(3, 5); // ok, ip is not final
\end{lstlisting} \IOComm{I made the above take less lines}


There are several possible interpretations of the semantics of type modifiers.
Here we assume the full/deep meaning~\cite{ZibinEtAl10,Potanin2013}:
\begin{itemize}
  \item the objects in the ROG of an immutable object are immutable,
  \item a mutable field accessed from a \Q@read@ reference produces a \Q@read@ reference,
%  \item no \emph{down}-casting is allowed between different type modifiers.
  \item no casting/promotion from \Q@read@ to \Q@mut@ is allowed.
%  \item promotion, is a type-system feature allowing implicit and safe casting from \Q@read@ and \Q@mut@ to \Q@imm@.
\end{itemize}

\noindent There are many different existing techniques and type systems that handle the modifiers above~\cite{ZibinEtAl10,ClarkeWrigstad03,HallerOdersky10,GordonEtAl12,ServettoZucca15}.
The main progress in the last few years is with the flexibility of such type systems:
 where the programmer should use \Q@imm@ when  representing immutable data
and \Q@mut@ nearly everywhere else. The system will be able to transparently promote/recover~\cite{GordonEtAl12,clebsch2015deny,ServettoZucca15} the type modifiers, adapting them to their use context.
To see a glimpse of this flexibility, consider the following example:
%//the same expression can create mut, imm or capsule
%\saveSpace
\begin{lstlisting}
    mut Circle mc = new Circle(new Point(0, 0), 7);
capsule Circle cc = new Circle(new Point(0, 0), 7);
    imm Circle ic = new Circle(new Point(0, 0), 7);
\end{lstlisting}

Here \Q@mc@, \Q@cc@, and \Q@ic@ are syntactically initialised with the same expression: \Q@new Circle(..)@.
The \Q@new@ expression returns a \Q@mut@, so \Q@mc@ is obviously ok.
%\footnote{Capsules must encapsulate their entire ROG, thus a \Q@new@ expression
%can not directly return \Q@capsule@ in the case of objects with \Q@mut@ fields.}
Moreover, the expression does not use any \Q@mut@ local variables, thus the flexible TM system
allows the \Q@mut@ result to be promoted to \Q@capsule@, thus \Q@cc@ is ok. 
Additionally, a \Q@capsule@ can be implicitly converted to \Q@imm@, thus \Q@ic@ is also ok.
We want to emphasise that this is not a special feature of \Q@new@ expressions:
any expression of a \Q@mut@ type that uses no free \Q@mut@ variables declared outside can be implicitly promoted to \Q@capsule@/\Q@imm@.\footnote{%
This requires some restrictions on \Q@read@ fields not discussed in detail for lack of space.
} This is the main improvement on the flexibility of TMs in recent literature~\cite{ServettoEtAl13a,ServettoZucca15,GordonEtAl12,clebsch2015deny,clebsch2017orca}.
Former work~\cite{Boyland10,boyland2003checking,Hogg91,Smith:2000:AT:645394.651903,DBLP:conf/pldi/AikenFKT03}, which eventually enabled the work of Gordon \etal's,  does not consider promotion and 
infers uniqueness/isolation/immutability only when starting from references that have been tracked with restrictive annotations along their whole lifetime.
From a usability perspective, this improvement means that
these TMs are opt-in: a programmer can write large sections of code
mindlessly using \Q@mut@ types and be free to have rampant aliasing. 
Then, at a later stage, another programmer may still 
be able to encapsulate those data structures into an \Q@imm@ or \Q@capsule@ reference.

%\saveSpace
%\begin{lstlisting}
%mc.radius = 3; // ok
%imm Point ip = ic.center; // ok, ROG immutable
%read Circle rc = mc
%read Point rp = rc.center; // ok, fields of read Circle are read
%$\Comment{}$mut Point mp = rc.center; // type error
%\end{lstlisting}
%\saveSpace
%Such flexibility is also visible where \Q@rc.center@ returns a \Q@read@ but \Q@ic.center@ returns an \Q@imm@: any expression typed as \Q@read@ that only
%uses immutable variables can safely be promoted to \Q@imm@ or \Q@capsule@.


 %(since \Q@ic@ is \Q@imm@, and \Q@imm@ is a deep modifier).
%true fact but not sufficient?

%With this kind of type system, we can ensure immutable classes by just declaring all their fields as final and immutable.%
%\footnote{
%In Java,  to ensure a class is immutable we need:
%the class must be final, all the fields must be final of immutable
%classes (thus no interface fields, final classes all the way down),
%and the SecurityManager need to properly tame reflection.}

% Not sure about this paragraph:

The \Q@capsule@ modifier (sometimes called isolated/\Q@iso@) is possibly the one whose details differ the most in the literature. Here we refer to the interpretation of~\cite{GordonEtAl12}, that introduced the concept of recovery/promotion.
This concept is the basis for L42, Pony, and Gordon \etal's type systems~\cite{GordonEtAl12,ServettoEtAl13a,ServettoZucca15,ServettoEtAl13a,clebsch2015deny,clebsch2017orca}. 

%\begin{itemize}
%	\item A capsule local variable can only be used once. %as \Q@capsule@ or \Q@mut@.
%	\item Only a capsule expression can be used to initialize or update a \Q@capsule@ field.
%	\item A capsule field access has the same type modifier as the receiver.
%	\item An expression of a \Q@mut@ type that uses no \Q@mut@ variables declared outside can be implicitly promoted to \Q@capsule@. Promotion/recovery is the main improvement on the flexibility of TM in recent literature~\cite{ServettoEtAl13a,ServettoZucca15,GordonEtAl12,clebsch2015deny,clebsch2017orca}
%	(this requires some restrictions on \Q@read@ fields that we do not discuss in detail for lack of space).
% \end{itemize}

%This is to ensure the capsule doesn't `leak', potentially violating it's exclusivity,

The capsule/isolated fields of Gordon \etal and Pony rely on destructive reads~\cite{GordonEtAl12,clebsch2015deny}: in order to read them, a new value (such as \Q!null!) will be assigned to them. In contrast, L42~\cite{ServettoEtAl13a,ServettoZucca15} does not require such destructive reads, thus \Q@capsule@ fields can be accessed many times, and their content can be seen from outside; but only in controlled ways.
Both Gordon \etal and Pony restrict how \Q!capsule! local variables can be used by changing the type they are seen as, however both allow the local variable to be `consumed', allowing them to be used as normal capsule/isolated expressions, at the cost of being unable to use the variable again. L42 however uses a simpler approach where all accesses to \Q!capsule! local variables consume them: they are expressed using linear/affine types~\cite{boyland2001alias}, thus they can only be used once.
%Both the capsule/isolated fields and variables of Pony and Gordon \etall rely on destructive reads~\cite{GordonEtAl12,clebsch2015deny}: reading such fields replaces their values with \Q@null@.+++ and a static analysis ensures capsule local variables can be accessed only once after the initialization and every update.

%In contrast,  L42~\cite{ServettoEtAl13a,ServettoZucca15} do not require destructive reads, and treat %\Q@capsule@ local variables and \Q@capsule@ fields differently:
%\Q@capsule@ local variables are expressed using linear/affine types~\cite{boyland2001alias}, thus they can only be used once;
%\Q@capsule@ fields can be accessed many times,
%and thus their content can be seen from outside; but only in controlled ways.

%while M\# and Pony requires both capsule fields and capsule variables to be `balloons'~\cite{Almeida97,ServettoEtAl13a} in the object graph.

%Destructive reads would be a bad idea for validation as they would likely invalidate objects.

%x=this.#f()
%..do all you want with x...
%// invariant check here!!
%.
%this.f(x -> ..)
%this.f=transform(this.f)
%invariantCheck()
%transform(this.#f())
%invariantCheck()

%\loseSpace
\subheading{Exceptions}\label{s:exceptions}
In most languages exceptions may be thrown at any point; combined with mutation this complicates reasoning about the state of programs after exceptions are caught: if an exception was thrown whilst mutating an object, what state is that object in? Does its invariant hold?
The concept of \emph{strong exception safety} (SES)~\cite{Abrahams2000,JOT:issue_2011_01/article1} simplifies reasoning:
if a \Q@try@--\Q@catch@ block caught an exception, the state visible before execution of the \Q@try@ block is unchanged, and the exception object does not expose any object that was being mutated.
%\LINE
%\noindent{\textit{Exceptions:}}
% M\#, L42 and Pony rely on SES for all unchecked exceptions to ensure safe and transparent parallelism,
% They wish to ensure the code behave as if the execution was fully sequential.
% Exceptions create additional difficulties in such context: if two operations are running in parallel in
% a fork-join, and the first one produces an exception, it should be safe to cancel the other operation and
% to propagate the exception outwards. The system need to guarantee
% the progress the second operation accumulated is not observable.
% Pony avoids this problem simply by not supporting exceptions;
% while
%M\# and L42 will parallelize only expressions that do not leak checked exceptions,
%and they enforce Strong Exception Safety(SES)~\cite{Abrahams2000} for unchecked exceptions.
%Other authors have identified the concept of SES as
% a general issue when reasoning about objects state after catching an exception.
% while we need it to soundly capture invariant failures.
L42 already enforces SES for unchecked exceptions.\footnote{%
This is needed to support safe parallelism. Pony takes a more drastic approach and does not support exceptions in the first place. 
We are not aware of how Gordon \etal handles exceptions, however in order for it to have sound unobservable parallelism it must have some restrictions.%
%We do not know how M\# conciliate deterministic parallelism and unchecked exceptions, we suspect some variation of SES must be in place.
}
L42 enforces SES using TMs in the following way:\footnote{Transactions are another way of enforcing strong exception safety, but they require specialized and costly run time support.}\footnote{A formal proof of why these restriction are sufficient is presented in the work of Lagorio~\cite{JOT:issue_2011_01/article1}.}
\begin{itemize}
\item Code inside a \Q@try@ block capturing unchecked exceptions is typed as if all \Q@mut@ variables declared outside of the block are \Q@read@.
\item Only \Q@imm@ objects may be thrown as unchecked exceptions.
\end{itemize} 
%Of course this has the effect that even if no-exception is thrown, no mutation could have occured, which is an even stronger property than SES, other work is more flexible~\cite{?}, at the cost of more complicated typing rules.
%With SES we can soundly capture invariant-failures as an exception, since any mutation that caused the invariant failure cannot be observed. However, we also need to prevent a broke-object from being reachable from the exception object; since the only way a broken-object can be seen is within the \Q@read@ \Q@invariant@  method, it follows that if the exception-object contains no \Q@read@ references in its ROG it cannot leak a broken object. Preventing this in the-typsystem is non-trivial, so instead we simply require that:

\noindent This strategy does not restrict throwing exceptions, but only catching unchecked ones.
SES allows us to soundly capture invariant failures as unchecked exceptions: 
the broken object is guaranteed to be garbage collectable when the exception is captured. For the purposes of soundly catching invariant failures, it would be sufficient to enforce SES only when capturing exceptions caused by such failures.
%The ability to catch and recover from such failures is extremely useful as it allows the program to take corrective action.(DUPLICATED)

% We think this restriction is acceptable for run time verification, other works are much more restrictive,



%The above rules need only be enforced for catch blocks that could catch invariant-failures (including exceptions thrown within execution of \Q@invariant@) itself;
%, and since \Q@invariant@ declares no checked exceptions, this includes all exceptions throw-able by it.

% TMs are very useful in restricting the scope of mutation. 
% Any expression that does not use any \Q@mut@ 
% variable declared outside of such expression does not modify objects visible outside.
% With this observation in mind, we can use TMs to enforce SES in the following way:\footnote{
% 

% \begin{itemize}
% \item all thrown exceptions are immutable objects,
% \item 
% \end{itemize}

% For the aim of enforcing invariants, we could relax SES to hold only when capturing exceptions caused by invariant failures; but we are building on approaches that enforce SES on all unchecked exceptions .


% Intro to OCs

\subheading{Object Capabilities (OCs)}
OCs, which L42, Pony, and Gordon \etal's work have, are a widely used~\cite{miller2003capability,
noble2016abstract,karger1988improving} programming style that allows associating resources with objects. When this style
is respected, code that does not possess an alias to such an object cannot use its associated resource.
%Object capabilities are programming style used to control and restrict use of operations such as access to external resources
Here, as in Gordon \etal's work, we use OCs to reason about determinism and I/O. To properly enforce this, the OC style needs to be respected while implementing the primitives of the standard library and when performing foreign function calls that could be non deterministic, such as operations that read from files or generate random numbers. Such operations would not be provided by static methods, but instead instance methods of classes whose instantiation is kept under control. 
% \noindent\REVComm{\textit{Object Capabilities:}}{2}{Citations here?}
% While type modifiers are statically verified properties of references, object capabilities are run time characteristics of specific objects.

% Conceptually, an object capability is a communicable, unforgeable token of authority, a key to access special functionality: only certain objects with `special' powers can do `special' actions, and those objects are obtained in a controlled way. We call such objects `capability objects'.


% Their main use case is to allow for fine grained control over what sections of code are allowed to do. 

\lstset{language=Java}
 For example, in Java, \Q@System.in@
 \lstset{language=FortyTwo} 
  is a \emph{capability object} that provides access to the standard input resource, however, as it is globally accessible it completely prevents reasoning about determinism. 
 % In contrast, in the object capability style, one would not have-global variables but have the main por

% a capability object (it has the capability to read input); however it is globally accessible: thus any code could use it, preventing reasoning about determinism.
In contrast, if Java were to respect the object capability style, the \Q@main@ method could take a \Q@System@ parameter, as in
 \Q@main(mut System s)@
 \lstset{language=Java}
\Q@{.. s.in.read() ..}@. \lstset{language=FortyTwo}%
Calling methods on that \Q@System@ instance would be the only way to perform I/O;
moreover, the only \Q@System@ instance would be the one created by the runtime system before calling \Q@main@. % would have no usable constructor, and all its I/O methods would require a mutable (\Q@mut@) receiver.
% Other non deterministic operations would also work this.
%may just take a \Q@mut System@ object as a parameter.
% could also work this way.
This design has been explored by Joe-E~\cite{finifter2008verifiable}.
OCs are typically not part of the type system nor do they require runtime checks or special support beyond that provided by a memory safe language. However, since
L42 allows user code to perform foreign calls without going through a predefined standard library, its type system enforces the OC pattern over such calls:
%To reason about determinism, L42 connects TMs with the OC style as follows: % style by requiring:
\begin{itemize}
\item Foreign methods (which have not been whitelisted as deterministic) and methods whose names start with \texttt{\#\$} are \emph{capability methods}.%
\item Constructors of classes declared as \emph{capability classes} are also capability methods.
\item Capability methods can only be called by other capability-methods or \Q@mut@/\Q@capsule@ methods of capability classes.
\item In L42 there is no \Q@main@ method, rather it has several main expressions; such expressions can also call capability methods, thus they can instantiate capability objects and pass them around to the rest of the program.
% \item Any method that uses non deterministic primitive operations (such as native calls or access to global variables\footnote{ Even just allowing unrestricted access to \Q@imm@ global variables would prevent reasoning over determinism due to the possibility of global variable updates; however constant/final globals of an \Q@imm@ type would not cause such problems.
% }) must be an instance method requiring a \Q@mut@ receiver.
% Classes having such methods are \emph{capability classes}, and their instances are \emph{capability objects}.
% \item A capability object can only be created inside a \Q@mut@ method of a capability class; or
% by the runtime system, and passed to the main method.

% \item If the language has global variables, they should only be 
%\item There are no global variables.\footnote{}
\end{itemize}

\noindent L42 expects capability methods to be used mostly internally by capability classes, whereas user code would call normal methods on already existing capability objects.

For the purposes of invariant checking, we only care about the effects that methods could have on the running program and heap. As such, \emph{output} methods (such as a \Q@print@ method) can be whitelisted as `deterministic', provided they do not affect program execution, such as by non deterministically throwing I/O errors.

\subheading{Purity}\label{s:purity}
TMs and OCs together statically guarantee that any method with only \Q!read! or \Q!imm! parameters (including the receiver) is \emph{pure}; we define pure
as being deterministic and not mutating existing memory. Such methods are pure because:
\begin{itemize}
	\item the ROG of the parameters (including \Q!this!) is only accessible as \Q@read@ (or \Q@imm@), thus it cannot be mutated\footnote{This is even true in the concurrent environments of Pony and Gordon \etal, since they ensure that no other thread/actor has access to a \Q@mut@/\Q@capsule@ alias of \Q@this@. 
	Thus, since such methods do not write to memory accessible by another thread, nor read memory that could be mutated by another thread, they are atomic.},
	\item if a capability object is in the ROG of any of the arguments (including the receiver), then it can only be accessed as \Q@read@, preventing calling any non deterministic (capability) methods,
	\item no other preexisting objects are accessible (as L42 does not have global variables).\footnote{%
		If L42 did have static variables, getters and setters for them would be capability methods.
		Even allowing unrestricted access to \Q@imm@
		static variables would prevent reasoning over
		determinism, due to the possibility of global variable
		updates; however constant/final globals of an 
		\Q@imm@ type would not cause such problems.%
	}
\end{itemize}

%Methods that perform non-deterministic \emph{input} shouldn't be white-listed.%, including methods that read information passed to white-listed output methods.


%Here we combine TMs with OCs to guarantee 
%\MS{determinism of} any method that can not access a \Q@mut@ reference to a capability object:
%all non deterministic primitive operations are instance methods requiring a \Q@mut@ receiver, and
%	\item all non-deterministic primitives (like native calls) require a \Q@mut@ receiver,
%instances of capability classes containing such methods can only be created by a \Q@mut@ method of another capability class
%	\begin{itemize}
%		\item the runtime-system\footnote{as 42 has no standard-library, we treat meta-code as the runtime-system} before main begins,
%		\item within a \Q@mut@ method on such a class
%	\end{itemize}
%	\item all non-deterministic operations require a \Q@mut@ receiver,
%	\item all classes 
%	\item there are no global variables\footnote{Note: even just allowing \Q@imm@
%global variables would prevent reasoning over determinism due to the possibility of global variable updates; however constant/final globals of an \Q@imm@ type would not cause such problems.},
% \item user code cannot directly create a capability object: they can only indirectly do so through an existing \Q@mut@ capability object reference.

% NOTE: SOMEWHERE MAKE IT CLEAR THAT NON-DETERMINISM CAN ONLY OCCUR THROUGH A CAPABILITY OBJECT
%\end{itemize}

%%As for pack/unpack we allow objects to become invalid while their reachable object graph (ROG) is being updated. However in our protocol we guarantee 
%that invalid objects are never used or passed around.

%\footnote{If you were to step with a debugger, you may still see invalid objects, and \Q@invariant@ calls failing.}
%\footnote{Spec\# is more general and, as an extra feature, allows specially declared methods to use invalid objects, while we require such objects to never be used.}
%\footnote{In particular, all parameters (including \Q@this@) and all of their ROGs will be valid.}
% systems performs 10,000 less  is up to 10,000 times slower.
% \item Debugging is easier since the invariant failure always happens when the method invalidating the object is still active on the stack trace.
%\end{itemize}



%\IO{In this paper we show how we can add sound enforcement of class-invariants to languages that already have such TMs and OCs.\footnote{\IO{I.e. we are not trying to suggest adding TMs and OCs to conventional languages.}}}
% instead of the 
%expressive but necessarily more involved
%detailed composable annotations that are common in static-verification.


%without 
%\IO{[restricting dynamic class loading] without requiring static-verification}
% or employing a verification language.

\LINE
We propose a way of soundly verifying class invariants by using a minimalistic code instrumentation strategy, which can be applied to languages like M\#, Pony or  L42  without adding any additional syntax, nor changing their type systems. Our approach uses the guarantees already provided by TMs and OCs to ensure that \Q@invariant@ methods are \emph{pure}\footnote{We say a method is \emph{pure} if it is deterministic and does not mutate existing state.} and called before a possibly invalid object becomes reachable. Such calls to \Q@invariant@ are not present in the source code but are instead performed by our instrumentation; in addition, if such a call returns \Q@false@ (indicating the receiver was invalid) an unchecked exception is thrown; this is similar to failed assertions in Java.

% Different verification approaches have different invariant protocols~\cite{FlexibleInvariants}, specifying when the invariant is expected to hold and when it is allowed to be violated.
% The common invariant protocol~\cite{?}
%JML, OOSC..., D, Eiffel
% only enforces the invariant on the receiver before and after every public method call. Sadly, this allows such receivers to be invalidated and passed around to other code. \IO{It also makes it hard to determine the cause of an invariant violation, since any object in the system is allowed to be invalidated by any code, such as could happen with aliases to such an objects field.}

% TODO: Call the invariant protocl 'validation'
%%HERE%%% discussing how to adapt this but not removing
Our invariant protocol follows the mindset of Spec\#~\cite{?}: we allow temporally violating an object's invariant\footnote{If you were to step with a debugger, you may still see invalid objects, and \Q@invariant@ calls failing.} when their reachable object graph (ROG) is being updated; however, we guarantee that such objects are not used during this process.\footnote{Spec\# is more general and, as an extra feature, allows specially declared methods to use invalid objects, while we require such objects to never be used.}
Unlike Spec\#~\cite{?}, our system allows sound catching of invariant failures; this is sound since TMs can guarantee that the invalid object is garbage-collectable.

Our approach has 2 main benefits over the traditional one: it greatly simplifies reasoning since programmers can assume that objects accessible to them are valid\footnote{In particular, all parameters (including \Q@this@) and their ROGs will be valid.}. And it requires that the invariant be checked significantly less often; we fully implemented our approach over L42 and we conducted a case-study which shows that the traditional approach staggeringly performs 1,000,000 times more runtime-checks than ours.
% systems performs 10,000 less  is up to 10,000 times slower.
% \item Debugging is easier since the invariant failure always happens when the method invalidating the object is still active on the stack trace.
%\end{itemize}
Here we present a sound and efficient system, designed to be simple and easy to use for non-verification experts. Our simplicity however comes with a compromise: we can not verify some forms of data-structures, such as collections of mutable elements.\footnote{Our approach does not prevent correctly implement such data-structures, rather we do not support encoding the correctness of such objects as a class-invariant.}

In section \ref{TODO} we compare against a state of the art system, Spec\#, and show how our system can both verify a significant number of the class-invariants they can (including for example mutable collections of immutable objects) whilst having a significantly lower annotation-burden (about 3 times less).
This is because we use TMs for aliasing and mutability control, which are a simple, easy, and coarse grained
technique. On the other hand, the static-analysis employed by Spec\# is more fine grained but requires more frequent and involved annotations.


% Static verification of method contracts can prove the correctness of OO programs, but this is costly and requires defining the desired semantics of the program in a verification language.
% This however requires limiting dynamic class loading, ensuring that only statically verified code can be loaded.
% Without this restriction, even predicting the behaviour of an innocent looking call like 
% \Q@myPoint.getX()@ is impossible: the dynamic type of \Q@myPoint@ can refer to a dynamically loaded class
% whose method \Q@.getX()@ uses I/O to behave non deterministically, or even to format the user’s hard drive.

% Recently, a new design for \emph{type modifiers}(TMs) has emerged, that radically improves their usability.
% Three different research languages are being independently developed relying on this new design: \IO{Gordon}, Pony and 42.
% Those modifiers include \Q@mut@, \Q@imm@ and \Q@capsule@, to represent mutable, immutable and encapsulated data.
% Gordon and Pony need those type informsations to better exploit multi core machine.
% These modifiers provide opt-in restrictions/guarantees: code where everything has a \Q@mut@-type is as flexible as Java.
% However, when \Q@imm@/\Q@capsule@ are used some code may perform better on a multi core machine, either automatically (gordon) or if the program uses actors (pony).
% 42 developers also recognize that those modifiers may provide benefit for parallelism, but they focus on the implications on third party library usage safety.

% Point: new TM is usable, and there are 3 languages using it for parallelism




	% todo let marco right this:
		% breif summary of what they are for
		% citations....
	 	% breif mention of how much their used

% Point:OC and SES are also needed for parallelism and are present in 42, likely present in Gordon and irrelevant in pony

% Gordon briefly discuss that they relies on \emph{Object capabilities} (OCs) to make their approach sound in respect to interaction with I/O. To the best of our knowledge Gordon do not disclose interaction with exceptions, while Pony do not have exceptions in the first place.


% Point: thus, those 3 system are good canditate for our validation

% Their work do not 
% In order to be sound they need OC and exception safety
%  Object capabilities can been used to enforce determinism and the absence of I/O~\cite{finifter2008verifiable}.

%Recent research on \emph{type modifiers} (TMs) and \emph{object capabilities} (OCs) offer us a simpler way to perform high-level reasoning on OO programs, without restricting dynamic class loading or employing a verification language.
%Type modifiers have been used to enforce ownership and automatic/correct parallelism~cite{GordonEtAl12,clebsch2015deny,clebsch2017orca}. Object capabilities can been used to enforce determinism and the absence of I/O~\cite{finifter2008verifiable}.



%Given these benefits, many emerging languages (such as Rust~\cite{matsakis2014rust} and newspeak~\cite{bracha2010modules}) support some form of type modifiers and/or object capabilities.



% As for failed assertions in Java, an unchecked exception is thrown if such a call returns \Q@false@ (indicating that the object is invalid).

% Note that we do not expect invocations of \Q@invariant@ to be present in the code written by the programmer, \Q@invariant@ invocations are injected by our instrumentation.


%That is, we do not ensure the absence of invalid objects, instead we use exceptions to steer execution to a point where the type system ensures the object is not reachable.
% If you were to step with a debugger, you may see invalid objects, and validation checks failing.

% Our approach hijacks the already present TMs and OCs to ensure that a 
% \Q@invariant@ is automatically called at the right times during execution; an unchecked exception is raise
% if such method return false. We use TMs and OCs also to check that the \Q@invariant@ method is `pure', hence they represent logic predicates on the receiver reachable object graph (ROG)%
%We say that an object \Q@o@ is valid when \Q@o.invariant()@ would return \Q@true@, and is invalid otherwise (e.g. if it returns \Q@false@, throws an exception or does not terminate).

%\hrule

%TODO improve?
%simplifies the writing of code that is unrelated to invariant-checking

%------------------------------
%---
% \item \REVComm{It is the responsibility of the class \emph{author} to ensure the invariant is preserved across all public methods, whereas with validation it is the responsibility of the class \emph{user} to not (indirectly) mutate the object in any way that will break it}{2}{What does this mean?}.

%\end{itemize}
% \noindent Our system is more primitive and general than class invariants, and can be used as a building block to encode 
% them.
% and allows 
%global (global is not the right word)
% reasoning through the program.
% \REVComm{Indeed, class invariants can be represented in our system by defining validity as 
%your ``and'' was wrong, is ``or''
% the class invariant holding or a mutator method currently running (this is easily implemented with a boolean % flag).}{3}{Hard to understand setence}
%, and all objects observing modified by the execution
%creating such objects are made be unreachable if the corresponding unchecked exception was to be captured.

%This approach contrasts with an alternative approach, %that of static verification, in which case a program that %could produce invalid objects would fail to compile. Our %approach is 
%more flexible 
%as it allows for a program that may produce invalid objects to still be useful, with the programmer only needing to handle them when they occur.

% \subsection{Random stuff }
% We show that a na\"{i}ve approach that just checks \validate{} at the end of constructors and setters would be insufficient. We take advantage of TM+OC to conservatively identify classes where validation is possible.
%

\begin{comment}
\noindent\textit{Similarity with checked casts:}
%\subsection{Extended type system and similarity with checked casts}
Even if not as good as full static verification, from a formal perspective
% our system clearly aid reasoning.
our system provides sound guarantees that aid reasoning.

Programmers are still responsible for creating and mutating objects in order to preserve their validity;
but once an object has been successfully created/mutated, it is guaranteed to be valid.
Attempting to create a new invalid object, or to mutate an object into an invalid one, causes
a run time error.

This is similar to casts in Java with respect to casts in C:
In both languages the programmer is responsible for casting values correctly;
however, in Java casts are soundly checked: if in a Java program the control flow \REVComm{goes over a cast}{2}{unclear about whether it describes a cast that succeeds or fails/},
we know that in that specific
execution such assumption was correct.
On the other hand incorrect C casts just default to undefined behaviour:
if a C execution continue after a cast, we know absolutely nothing. The same execution
may fail when run on a different machine, 
or even on the same one in another moment.
In the same sense as checked-casts, object construction and ROG mutation are soundly checked by our system,
and success in a mutation means that in that specific execution, validity was preserved.

\end{comment}


%load Item i
%  playCargo(this.cargo(),i)
%
%static playCargo(mut Cargo c, Item i)
%  c.add(i)
%  c.remove(i)
%  
% 
%valid/validation/validity
%
%name of the method: valid/validate
%-----------------------------------
%
%
%%awesome capability objects
%
%%awesome type modifiers
%
%awesome exception safety %%no, we try to sell Exception safety as an application of TM
%
%%with those 3 together we get validation
%
%%is like a extended type system %? yes by comparision with casts?
%
%contrast with class invariants
%
%
%valid only access fields
%
%valid
%
%
%patterns wrapper/state variable allows for flexible invariants
%
%verify the programmer intentions are self-consistent
%
%encoding post conditions and pre conditions as specialized types
%
%code example
%

%%%
%%%May coonect with "OffensiveProgramming/DefensiveProgramming".
%%%


\noindent\textit{Structure of our paper:}
TODO...

\begin{comment}
 We now present how a real language could support
validation. 
We would like to underline that the
features we need have all been presented in the past, but their application to sound validation has never been explored.

%To ease our explanation, we present various categories
%of objects where we can support validation, starting from the simpler one to handle.
In Section \ref{s:background} we will introduce one by one the various
language features cooperating
to support validation. %, and we work under the assumption that there are no static variables.
We will provide a brief introduction to type modifiers, object capabilities, and exception safety.

In Sections \ref{s:validate}, \ref{s:immState}, and \ref{s:encapsulated}, we will show how TM and OC enable validation.
To clearly communicate what kind of checks the language semantics should perform,
 we will write them down as if they were `generated source code'.
However, our proposed approach is independent of the actual technique to insert these checks, and they may instead be inserted directly into bytecode, or they can be part of the underlying semantics of a virtual machine.

In Section \ref{s:meaning} we will provide formal definitions, and in Appendix \ref{s:proof} we provide  a proof that our language soundly enforces validation.
\end{comment}


\LINE


%Points:
%1. L42 et al are 'realistic'
%2. OC's are like Java, and not like haskell...
%3. TM are opt in
%4. Not advocating but merely pointing out that TMs and %OCs can support CI
%5. Modularity of our approach: don't need to look at %other contracts or invariants

%6. PURITY!! (explain why this is important)
%7. Talk and unpack (as a 3rd? invariant protocal)
%\saveSpace\saveSpace
Software verification is the art of reasoning 
on a program: given a precondition, 
we can verify that a specific method or expression
satisfy a postcondition.
Under this lens, class invariants can be seen as implicit pre and post condition.
However, in a real system, it may be impossible to ensure any precondition. For example, 
in the absence of on the fly static verification of dynamically loaded code there is no way to ensure a
subclass respect behavioural subtyping.
In particular, run-time verification of pre conditions can not verify the behaviour of all the methods of the parameters.

We present an approach to enforce class invariants without needing any reasoning.

Classes can have a boolean method \Q@invariant@,
and any call of \Q@invariant@, everywhere in the system is guaranteed to return \Q@true@.
We obtain such feat in the following way:
\begin{itemize}
	\item We have a rich type system, supporting type modifiers \Q@read,mut,imm,capsule@. Similar versions of those modifiers are common in literature.
	\item The standard library is designed using object capabilities.
	\item The invariant method can use this only to access immutable and capsule fields.
	\item The constructor uses this only to initialize the fields.
	\item encapsulated state can be accesses with certain limitations
	\item The invariant method is checked at run time after the constructor, after every field update and
	after methods accessing encapsulated state in mutation.
	\item Mutation need to be performed so that all the mutation appear to be atomic with 
\end{itemize}
\LINE

Three recent language design leverage on TMs and OCs to provide safe parallelism.
Those are realistic languages [..].
The programming style required by OCs, while different from Java, is much more near to Java then Haskell. TMs on the other side are completely opt-in, and code can [..]
Here we show an approach, allowing those 3 language 
to  enforce class invariants with a combination of run-time checks and static reasoning.
We rely on TMs and OCs in they way they are already supported in M\#,Pony and L42.

Here we do not advocate in favour or against adding
TMs and/or OCs to Java, C\# or any other language, but we point out how languages supporting TMs and OCs can
enforce class invariants in a very broad scope, where
each class can be modularly analysed by assuming only that all the other classes are well typed.
There is no need to verify or even consider possible contracts or invariants present on any other class.

\LINE
More in the detail, our approach 

Run-time invariant checks are added only in 3 points in the code: at the end of constructors, after field updates and after methods accessing encapsulated state.


We implement our approach over L42 and we perform a
case study showing that our approach requires 
exponentially less invariant checks with respect to the conventional one: to check before and after every method call.


Here we show an approach, allowing those 3 language 
to  enforce that every observable object respect its class invariants.
That is, classes can have a boolean method \Q@invariant@,
and any call of \Q@invariant@, everywhere in the system is guaranteed to return \Q@true@.
The invariant can be broken by following a programming pattern where the broken object becomes temporarily inaccessible.\footnote{
	This is similar to what happen with pack/unpack in Spec\#.
	Eiffel, D, Jose and other just require the invariant to hold at the start and at the end of every public method. This requirement has been considered too weak by many works~\cite{??}
}





L42 have TMs and OCs
CI can be soundly implemented over TMs and OCs
in an efficient and simple way

Since TMs and OCs allows to soundly/simply support CI,
we should port TMs and OCs in many other languages


\LINE


We only require all the code to be well typed and 


while 
there are approaches handling aliasing and mutation,
this is made particularly difficult by mutation, aliasing, dynamic dispatch, dynamic class loading,
I/O, and exceptions.
We propose here 

In
that the 
is based on the idea that 
given certain precondition


%empty weakest precondPoints:


% 1. L42 et al are 'realistic'
% 2. OC's are like Java, and not like haskell...
% 3. TM are opt in
% 4. Not advocating but merely pointing out that TMs and OCs can support CI
% 5. Modularity of our approach: don't need to look at other contracts or invariants

% 6. PURITY!! (explain why this is important)
% 7. Talk and unpack (as a 3rd? invariant protocal)Points:
% 1. L42 et al are 'realistic'
% 2. OC's are like Java, and not like haskell...
% 3. TM are opt in
% 4. Not advocating but merely pointing out that TMs and OCs can support CI
% 5. Modularity of our approach: don't need to look at other contracts or invariants

% 6. PURITY!! (explain why this is important)
% 7. Talk and unpack (as a 3rd? invariant protocal)ition

%box pattern encodes atomicity of mutation so that
%the invariant is expected to hold after every atomic mutation


%alex: monday 10am
%----------------

 



\section{Background}
\label{s:background}
\noindent\textit{Type Modifiers:}
Type modifiers are a well known language mechanism~\cite{TschantzErnst05,BirkaErnst04,OstlundEtAl08,clebsch2015deny,GianniniEtAl16,GordonEtAl12} allowing static verification of mutability and aliasing properties of objects.
With sightly different names and semantics, the three most common modifiers for object references are:
\begin{itemize}
\item Mutable (\Q@mut@): the referenced object can be mutated, as in most languages without modifiers.
\item Readonly (\Q@read@): the referenced object cannot be mutated by such reference, but in the program there may be mutable references to this same object, so mutation can still be observed. 
\item Immutable (\Q@imm@): the referenced object can never mutate. Like \Q@read@ references, one cannot mutate through an \Q@imm@ reference, however \Q@imm@ references also guarantees that the referenced object will never be mutated, not even through another reference.
\end{itemize}
%In the context of object-oriented programming, type modifiers may also apply to the implicit \Q@this@ parameter in method declarations, restricting the type of references the method can be called on. In addition, due to the deep meanings we type field access on object references to be the most restrictive of the object references modifier and the field’s. As \Q@read@ references impose no assumptions about aliasing, any \Q@imm@ or \Q@mut@ expression can be safely implicitly promoted to \Q@read@, whereas other conversions are not generally safe.
%\loseSpace
TM are different to field or variable modifiers like Java’s \Q@final@: TM applies to references,  \Q@final@ specifies what can be done to the field itself. In comparison to \Q@imm@:

\begin{itemize}
\item A \Q@final@ variable/field cannot be reassigned, it always refers to the same object; however, the referenced object itself may be mutated.
\item A reference of an \Q@imm@ type however refers to an object that will never be mutated, and neither will its ROG. However, a field of type \Q@imm@ may be updated to another \Q@imm@ reference.
\footnote{In C, this is similar to the difference between \Q@const *A@ and \Q@*const A@, where a \Q@final imm@ variable would be like \Q@const *const A@.}
\end{itemize}



\noindent Consider the following  example usage of \Q@mut@, \Q@imm@ and \Q@read@:
\begin{lstlisting}
mut Point mp = new Point(1, 2);
mp.x = 3; // ok
imm Point ip = new Point(1, 2);
$\Comment{}$ip.x = 3; // type error
read Point rp = mp; // ok read is a common supertype of imm/mut
$\Comment{}$rp.x = 3; // type error
mp.x = 5; // ok, and now we can observe rp.x == 5
ip = new Point(3, 5); // ok, ip is not final
\end{lstlisting}
\noindent We cannot use a \Q@read@ reference to cause mutation, but we have no guarantee of the absence of mutation; in our example we can observe a change in \Q@rp@ caused by a mutation inside \Q@mp@.


There are several possible interpretations of the semantics of type modifiers.
Here we assume the full/deep meaning:
\begin{itemize}
  \item all the objects in the ROG of an immutable object are immutable;
  this corresponds to UML DataTypes,
  \item a mutable field accessed from a \Q@read@ reference produce a \Q@read@ reference,
  \item no \emph{down}-casting is allowed between different type modifiers.
\end{itemize}


\noindent There are many different existing techniques and type systems that handle the modifiers above~\cite{ZibinEtAl10,ClarkeWrigstad03,HallerOdersky10,GordonEtAl12,ServettoZucca15}.
The main progress in the last couple of years is with the flexibility of such type systems: where the programmer should use \Q@imm@ to represent objects that would obviously be modelled as UML DataTypes, and \Q@mut@ nearly everywhere else; the system will be able to transparently promote/recover~\cite{GordonEtAl12,clebsch2015deny,ServettoZucca15} the type modifiers, adapting them to their use context.
To see a glimpse of this flexibility, consider the following example:
\saveSpace
\begin{lstlisting}
mut Point mCenter = new Point(1, 2);
mut Circle mc = new Circle(mCenter, /*radius*/7);
mc.radius = 3; // ok
imm Circle ic = new Circle(new Point(0, 0), 7); // ok imm
imm Point ip = ic.center; // ok, ROG immutable
read Circle rc = mc
read Point rp = rc.center; // ok, fields of read Circle are read
$\Comment{}$mut Point mp = rc.center; // type error
\end{lstlisting}
\saveSpace

Here \Q@imm Circle ic@ and \Q@mut Circle mc@ are both initialized with \Q@new Circle(...)@.
This is not a special feature of \Q@new@ expressions: since \Q@new@ returns a \Q@mut@ and any expression typed as \Q@mut@ that only uses immutable variables can safely be promoted to \Q@imm@.
% (since the returned value could not possibly be aliased).
%FALSE: it can be internally aliased!
Such flexibility is also visible where \Q@rc.center@ returns a \Q@read@ but \Q@ic.center@ returns an \Q@imm@: any expression typed as read that only
uses immutable variables can safely be promoted to \Q@imm@.

 %(since \Q@ic@ is \Q@imm@, and \Q@imm@ is a deep modifier).
%true fact but not sufficient?

%With this kind of type system, we can ensure immutable classes by just declaring all their fields as final and immutable.%
%\footnote{
%In Java,  to ensure a class is immutable we need:
%the class must be final, all the fields must be final of immutable
%classes (thus no interface fields, final classes all the way down),
%and the SecurityManager need to properly tame reflection.}

% Not sure about this paragraph:



\loseSpace

TM are very useful in restricting the scope of mutation. 
Any expression that does not use any \Q@mut@ 
variable declared outside of such expression does not modify objects visible outside. This consideration is particularly useful to understand code in the presence of exceptions. Other authors have identified the concept of Strong Exception Safety~\cite{Abrahams2000} as a general issue when reasoning about objects state after catching an exception:
when a \Q@try-catch@ catches an exception, the visible objects must be the same as before the \Q@try@ block started its execution.
This can be obtained leveraging TM in the following way:
\begin{itemize}
\item all thrown exceptions are immutable objects,
\item code inside a \Q@try@ block is typed as if all \Q@mut@ variables declared outside of the block are \Q@read@.
\end{itemize}

\loseSpace
\noindent\textit{Object Capabilities:}
While type modifiers are statically verified properties of references, object capabilities are run time characteristics of specific objects.

Conceptually, an object capability is a communicable, unforgeable token of authority, a key to access special functionality: only certain objects with `special' powers can do `special' actions, and those objects are obtained in a controlled way. We call such objects `capability objects'.
Their main use case is to allow for fine-grained control over what sections of code are allowed to do. For example, in Java \Q@System.in@ is a capability object (it has the capability to read input); however it is globally accessible: thus any code could use it, preventing reasoning about determinism.
In a language enforcing object capabilities the \Q@main@ method could take a \Q@System@ object as a parameter, and using that object is the only way to perform I/O, as in \Q@mySystem.println("hello")@.
Moreover, the \Q@System@ class would have no visible constructor, and all its I/O methods would require a mutable (\Q@mut@) receiver.
Many other operations, like random numbers and file management 
%may just take a \Q@mut System@ object as a parameter.
could also work this way.
\noindent This design has been explored in literature by Joe-E~\cite{finifter2008verifiable}.

Here we use TM to guarantee that any method that is not (indirectly) passed a \Q@mut@ reference to a capability object will not use any capabilities:
\begin{itemize}
\item all capability-methods must require a \Q@mut@ receiver,
\item there are no global variables\footnote{Note: even just allowing \Q@imm@
global variables would prevent reasoning over determinism due to the possibility of global variable updates; however constant/final globals of an \Q@imm@ type would not cause such problems.},
\item user code cannot directly create a capability object: they can only indirectly do so through an existing \Q@mut@ capability object reference

% NOTE: SOMEWHERE MAKE IT CLEAR THAT NON-DETERMINISM CAN ONLY OCCUR THROUGH A CAPABILITY OBJECT
\end{itemize}

%-----------------------------------------------------------


%define simple objects
%show solution  for simple person: requires 3 properties

%show solution is sound --> proof in appendix
%naive is unsound - person 3 bugs


\section{The \Q@.validate()@ Method}
\label{s:validate}
Thanks to TM and OC, we can now express the signature for \Q@.validate()@ method as follows:
\saveSpace
\begin{lstlisting}
read method imm Bool validate();
\end{lstlisting}
\saveSpace
%The method is \Q@read@: thus the method body will see the \Q@this@ object as a \Q@read@ reference; and has no other parameters. 
%By starting from a \Q@read@ reference and nothing else, we are guaranteed that the method is pure:
If the class containing the validate method has a super-class, we would automatically check \Q@super.validate()@ at the beginning of the sub-class’s \Q@.validate()@ method, this is required as otherwise `invalid' objects could easily be created by simply using subtyping to redefine \Q@.validate()@.
As this method is declared as \Q@read@, and only takes the implicit parameter \Q@this@ (as \Q@read@), we can guarantee the method is pure:
\begin{itemize}
\item the ROG from \Q@this@ is only accessed as \Q@read@ (or \Q@imm@), thus it cannot be mutated\footnote{
This can even be safe in a multi-thread environment: TM are often used to ensure correct parallelism; for example threads may be required to not share \Q@mut@ data, thus a \Q@read@ reference could only be mutated by a \Q@mut@ reference under the control of the same thread.
},
\item if a capability object (such as \Q@System@) is in the ROG of \Q@this@, then it can only be accessed as \Q@read@, preventing use of its capability (such as I/O),
\item nothing else is accessible (we do not have global variables).
\end{itemize}

\noindent Also note \Q@.validate()@ is not declared as throwing any exceptions, thus it can only leak unchecked exceptions.


Clearly the \Q@.validate()@ method must be able to take an invalid \Q@this@, since the purpose of such method is to distinguish valid and invalid objects.\footnote{
At a first look this may seam an open contradiction
with the aim of this work, however only calls to validate inserted by the language semantic can take an invalid \Q@this@. As for any other method, when the application code calls \Q@.validate()@,
\Q@this@ is guaranteed to be valid.
} However, if we allow the method to use \Q@this@ directly (e.g. storing it in a local variable or passing it to a method), we would break the guarantees of validation (namely: `invalid objects are not reachable by application code'); as such we enforce the simple restriction that \Q@.validate()@ may only use \Q@this@ to access fields.
As a relaxation, we could allow calling instance methods that in turn only use \Q@this@ to access fields, or call other such instance methods. With this relaxation, the semantics of \Q@.validate()@ need to be understood with the body of those methods inlined; thus the semantic of the inlined code need to be logically reinterpreted in the context of \Q@.validate()@, where \Q@this@ may be invalid.
In some sense, those inlined methods and field accesses can be thought as macro-expanded.


Finally,
the code of \Q@.validate()@ can not access  \Q@mut@ and \Q@read@ fields, because their content can be changed by unrelated code.
Thus, with the modifiers presented so far, we can only access \Q@imm@ fields.
We will later introduce a \Q@capsule@ modifier to allow more flexible validation.


% JUSTIFY that the fields are valid...

%validable objects are not circular (do not belong in their ROG of any of its fields)
%validate as a predicate on object fields never really see this.
%
%clarifications:
%a validate check is never needed/generated/injected when working on a read x
%multi threading is not relevant/supported
%validable objects are not circular (do not belong in their ROG of any of its fields)

\section{Validating immutable state}
\label{s:immState}
In this section we consider validation over fields of \Q@imm@ types.\footnote{
In a real language, for conciseness one could make the \Q@imm@ modifier the default, allowing it to be omitted and our \Q@Person@ example class would only use 3 type modifiers; however we explicitly use it here for clarity.
}
In the next section we expand our technique.

In the following code \Q@Person@ has a single immutable (non final) field \Q@name@:
\begin{lstlisting}
class Person {
  read method imm Bool validate() {return !name.isEmpty();}
  private imm String name;
  read method imm String name() {return this.name;}
  mut method imm Strinig name(imm String name) {this.name = name;}
  Person(imm String name) {this.name = name;} }
\end{lstlisting}
\Q@Person@, only has immutable fields and the constructor only uses \Q@this@ to access/update fields.%; we say such a class is \emph{simple}.
%\Q@Person@, only has immutable fields and the constructor 
%uses the parameters to directly initialize (all) the fields.
% We say such a class is \emph{simple}.%
%\footnote{
%We consider only standard contractors for simplicity of exposition.
%More complex constructors could be supported, provided that \Q@this@ is only used to access fields, we do discuss them for simplicity.}
The difference with respect to UML DataTypes 
%immutable types (like UML DataTypes)
%UML datatypes are aclass property. immutable types are often an instance one (so no final fields) 
 is that fields are not required to be final, thus the object can change state during its lifetime. This means that the ROGs of all the \emph{fields} are immutable, but the object itself may be mutable.
%Of course UML DataTypes
%immutable types
%No, a type is not a class
% are just a special case of simple classes.
Validation for such a class can easily be enforced by generating checks on the result of \Q@.validate()@, immediately after each field update, and at the end of the constructor\footnote{since the constructor only initialises fields, as with the \Q@.validate()@ method itself, we don't check for validity during the constructor as \Q@this@ is not directly reached, doing so would require the initial/default value of \Q@this@ to be valid.}:
% If a simple class provides a \Q@.validate()@ method, then validation will be enforced.
% For \Q@Person@, intuitively, the code would behave as follow:

%\Comment{if we made this public, all users who update the field need to call validate}%
%There are many interpretations for your comment
%why you deleted my code comments?
\begin{lstlisting}
class Person {
  read method imm Bool validate() {return !name.isEmpty();}
  private imm String name;
  read method imm String name() {return this.name;}
  mut method imm String name(imm String name) {
    this.name = name; // check after field update
    if (!this.validate()) {throw new Error(...);} 
  }
  Person(imm String name) {
    this.name = name; // check at end of constructor
    if (!this.validate()) {throw new Error(...);}
}}// Generated code, not directly written by the programmer
\end{lstlisting}
%... $\MComment{validation error}$ 

% Many programmers attempted to write similar code in mainstream languages like Java to ensure  that some property always holds. Indeed, at first look, this code seems to correctly enforce validation. Sadly, without relying on TM and OC, the former code would be broken: just making the fields private and checking the \Q@.validate()@ method at the \textbf{end of the constructor} and at the \textbf{end of mutator methods} is not enough to enforce validation.
% The trick is that our intuition relies not on statically verified properties, or on the semantics of the language, but on the expectations about `correct' behaviour of \Q@String@. We need to enforce Validation without assuming the behaviour of other objects.

If we were to relax (as in Rust) or even eliminate (as in Java) the support for TM or OC, the validation of this simple \Q@Person@ class would become harder, or even impossible. We now proceed to show examples where
relaxation of TM or OC breaks our validation. 

\loseSpace
\noindent\textit{Unrestricted use of object-capabilities:}
Allowing \Q@.validate()@ to (indirectly) use an object capability could allow for it to be non deterministic, by either:
\begin{itemize}
\item allowing \Q@.validate()@ to (indirectly) access a \Q@mut@ reference to a capability-object,
\item relaxing the rule that capability-methods must have a \Q@mut@ receiver, or
\item allowing capability objects to be created out of thin air.
\end{itemize}

For example consider this simple and contrived (mis)use of person:
\begin{lstlisting}
class EvilString extends String {
  @Override read method Bool isEmpty() {
    // Creates a new capability object out of thin air
    return new Random().bool();
} }
...
imm method mut Person createPersons(imm String name){
  // we can not be sure wether name is an `EvilString'
  mut Person schrodinger1 = new Person(name); // exception here?
  mut Person schrodinger2 = new Person(name); // what about here?
  ...}
\end{lstlisting}
Despite the code for \Q@Persion.validate()@ intuitively looking correct and deterministic, the above calls to it are not. Obviously this breaks any reasoning and makes our validation unsound. 
In particular, note how in the presence of dynamic class loading, 
we can not make any assumption on the dynamic type of \Q@name@.
%???
%Even if we disallow subtyping the same problem could still occur if we had a strange implementing of \Q@String@, or \Q@Persion.validate()@ itself.

\loseSpace
\noindent\textit{Allowing internal mutations/back-doors:}
% TODO: Come up with better title
Suppose we relax our mutation rules, by allowing interior mutability
as in Rust and Javari: thus allowing mutation of the ROG of an immutable object through back-doors:
\begin{lstlisting}
class AtomicBool {
  imm method imm Void store(imm Bool val){
    ... // Mutate through an imm reciever
  }
}

class NastyString extends String {
  imm AtomicBool evil = new AtomicBool(false);
  imm method imm Void nasty() {
    this.evil.store(true); // this imm method can do mutation
  }

  @Override read method Bool isEmpty() {
    return this.evil.load() ? false : super.isEmpty();
  }
  ...
}
...
imm NastyString name = new NastyString("bob");
imm Person person = new Person(name); // person is valid
name.nasty(); // mutate the ROG of person, without it noticing
// person is now invalid!
\end{lstlisting}

In this example we use \Q@AtomicBool@ as a back-door to remotely break the invariant of \Q@person@ without any interaction with the \Q@person@ object itself.
%mine: yes, too strong: For validation we need the language to guarantee true deep immutability.
%your: just points outside: It would require some powerful static or dynamic analysis to keep track of every case the ROG of \Q@Person@ could be indirectly mutated, and insert validity checks appropriately, however ensuring deep mutability trivialises this for simple classes.


Allowing such back-doors could also be used to break the determinism of the \Q@.validate()@ method, by allowing it to store and read information about previous calls.

%In our simple example, \Q@Person@ objects can be mutated using the setter, and exposed using the getter.
%We may consider the getter to be safe since in modern languages we expect strings to be immutable objects.
%\footnote{While we can update the field \Q@name@ to point to another string, we cannot mutate the string object itself.
%To obtain  \Q@"Hello"@ from \Q@"hello"@ we need to create a whole new string object that looks like the old one except for the first character. This would be different in older languages like C, where strings are just mutable arrays of characters.}
%
%Again, the assumption that they are immutable depends on the correctness of the code inside \Q@String@: if there was a bug in the \Q@String@ class, or any \Q@String@ subclass, then executing 
%\Q@println(bob.name())@ may change \Q@bob@ by quietly changing a part of its ROG.
%Again, checking
%what methods mutate states cannot be responsibility of the \Q@Person@ programmer.
%For Validation we need a language supporting aliasing and mutability control.
%\begin{comment}
%\item Sample Bug 1:
%Suppose there was a bug in \Q@String.isEmpty()@, causing the method to non-deterministically return \Q@true@ or \Q@false@.
%What would it mean for Validation?
%Would a \Q@Person@ be at the same time 
%valid and invalid?
%
%Only deterministic methods can be used for validation.
%Ensuring this cannot be responsibility of the \Q@Person@ programmer, since it may depend on third party code, as shown in this example.
%However, statically checking if a method is deterministic is hard/impossible in most imperative object-oriented languages.
%
%While we may not expect the presence of bugs in the standard library class \Q@String@, the same behaviour can be achieved with subtyping:
%\saveSpace
%\begin{lstlisting}
%class EvilStr extends String{
%  method Bool isEmpty(){
%    return new Random().bool();
%  }}
%...
%String name=...$\Comment{can this be an EvilStr?}$
%Person bob=new Person(name);
%\end{lstlisting}
%\saveSpace
%As you can see, it is hard to make sound claims about Validation.
%
%\item Sample Bug 2:
%In our simple example, \Q@Person@ objects can be mutated using the setter, and exposed using the getter.
%We may consider the getter to be safe since in modern languages we expect strings to be immutable objects.
%\footnote{While we can update the field \Q@name@ to point to another string, we cannot mutate the string object itself.
%To obtain  \Q@"Hello"@ from \Q@"hello"@ we need to create a whole new string object that looks like the old one except for the first character. This would be different in older languages like C, where strings are just mutable arrays of characters.}
%
%Again, the assumption that they are immutable depends on the correctness of the code inside \Q@String@: if there was a bug in the \Q@String@ class, or any \Q@String@ subclass, then executing 
%\Q@println(bob.name())@ may change \Q@bob@ by quietly changing a part of its ROG.
%
%Again, checking
%what methods mutate states cannot be responsibility of the \Q@Person@ programmer.
%For Validation we need a language supporting aliasing and mutability control.
%\end{comment}

\loseSpace
\noindent\textit{Strong Exception Safety:}
The ability to catch and recover from validation failures is extremely useful as it allows the program to take corrective action.
This may be implemented with a conventional \Q@try-catch@, since violations are represented by throwing unchecked exceptions. Due to the guarantees of strong exception safety, the only trace that the invalid object existed is the exception thrown; any object that has been mutated/created during the \Q@try@ block is now unreachable (as happens in alias burying~\cite{boyland2001alias}).

However, if instead we chose not to enforce strong exception safety, an invalid object could be easily made reachable:
\saveSpace
\begin{lstlisting}[morekeywords={assert}, escapechar=\%]
mut Person bob = new Person("bob");
// Catch and ignore validation failure:
try {bob.name("");} catch (imm Error t){}
assert bob.name().isEmpty(); // now we have a rechable invalid object!
\end{lstlisting}
\saveSpace
As you can see, recovering from a validation failure in this way is unsound and breaks the guarantees of validation.
Strong exception safety is a useful property to enforce, but for the specific purpose of validation this could be relaxed by restricting only \Q@try-catch@ blocks that could capture unchecked exceptions.
Since calls to \Q@.validate()@ may only throw unchecked-exceptions, violating strong exception safety within a \Q@try-catch@ that cannot catch unchecked-exceptions would not break validation.

%LATER: This means that we could relax our Strong Exception Safety to hold only on unchecked exceptions (by restricting only \Q@try-catch@ blocks that capture unchecked exceptions.



% One of the advantages of checking Validation at run time, is that
% we can allow the program can take corrective actions if a property is violated.
% This may be implemented with a conventional \Q@try-catch@ if violations are represented by throwing errors.
% However, there is an issue with exceptions modelling invalid objects: they can be captured when the invalid object is still in scope. For example:


%As you can see, if we can capture validation failures as normal exceptions %(very desirable feature) then we may end up using invalid objects.
%Moreover,
% as shown before with the example of transferring cargo between two boats,
%after an invariant has been violated, even objects with valid invariant may be in an unexpected state.

% This situation is a general issue about reasoning on the state after recovering from exceptions.
% In particular, as shown in the example this prevent sound validation.

% Note how this produces a different semantics with respect to static verification, where violations
% never happened. However this will not necessarily lead to a broken semantics:
%Thanks to Strong exception safety we have a system where either the application terminate
%when an invalid object is detected, or where any witness of the execution causing the invalid object is erased from history
%those objects and all the witnesses will be garbage collected
% (as happens in alias burying~\cite{boyland2001alias}).
%In our example, this means that to continue execution after a detected bug, 
%we would require to garbage collect the overloaded boat, their cargo and probably most of the commercial port too.








%\subsubsection*{Solving Issue 3: Constructors}
%\saveSpace
%Exposing \Q@this@ during construction is a generally recognized problem~\cite{gil2009we}.
%A simple solution is to require all constructors to 
%simply take a parameter for each field and to just initialize the fields.
%An advantage of such approach is syntactic brevity: constructors are implicitly defined
%by the set of fields and thus there is no need to define them manually.
%\textbf{Expressive initialization operations can still be performed, by following the factory pattern.}
%\saveSpace


%\subsubsection*{Recap}
%By utilising type modifiers (\Q@imm@, \Q@mut@ and \Q@read@), object capabilities and immutable exceptions we obtain sound runtime verification for immutable classes/UML data types.
\subheading{Relaxing Restrictions on Rep Fields?}
%\section{Invariants over encapsulated state}
%\label{s:encapsulated}
Rep fields allow expressing invariants over mutable object graphs.
Consider managing the shipment of items, where there is a maximum combined weight:
\begin{lstlisting}
class ShippingList {
  rep Items items;
  read method Bool invariant(){ return this.items.weight()<=300; }
  ShippingList(capsule Items items) {
    this.items = items;
    if (!this.invariant()){ throw Error(...); }//injected check
  }
  mut method Void addItem(Item item) {
    this.items.add(item);
    if (!this.invariant()){ throw Error(...); }//injected check
  }
}
\end{lstlisting}
We inject calls to \Q@invariant()@ at the end of the constructor and the \Q@addItem(item)@ method.
This is safe since the \Q@items@ field is declared \Q@rep@.
Relaxing our system to allow a \Q@mut@ reference capability for
the \Q@items@ field and the corresponding constructor parameter would 
make the above checks insufficient:
it would be possible for external code with no knowledge of the \Q@ShippingList@ to mutate its items. 
%Conventional ownership solves these problems by requiring a deep clone of all the data the constructor takes as input, as well as all exposed data (possibly through getters). % Isaac: I'm not sure this is correct, ownership transfer is a thing I've seen before, also freshly created objects would also be fine
In order to write correct library code in mainstream languages like Java and C++, defensive cloning~\cite{Bloch08,Detlefs98wrestlingwith} is needed.
For performance reasons, this is hardly done in practice and is a continuous source of bugs and unexpected behaviour.%

%\saveSpace
\begin{lstlisting}
mut Items items = ...;//INVALID EXAMPLE
mut ShippingList l = new ShippingList(items); // l is valid
items.addItem(new HeavyItem()); // l is now invalid!
\end{lstlisting}
If we were to allow \Q@x.items@ to be seen as \Q@mut@, where \Q@x@ is not \Q@this@, then  even if the \Q@ShippingList@ has full control of \Q!items! at initialisation time, such control may be lost later, and code unaware of the \Q@ShippingList@ could break it:
\begin{lstlisting}
//INVALID EXAMPLE: l.items can be exposed as mut
mut ShippingList l = new ShippingList(new Items()); // l is ok
mut Items evilAlias = l.items; // here l loses control
evilAlias.addItem(new HeavyItem()); // now l is invalid!
\end{lstlisting}
Relaxing our requirements for rep mutators
would break our protocol: if rep mutators could have a \Q@mut@ return type the following would be accepted:
\begin{lstlisting}
//INVALID EXAMPLE: rep mutator expose(c) return type is mut
mut method mut Items expose(C c) {return c.foo(this.items);}
\end{lstlisting}
Depending on dynamic dispatch, \Q@c.foo()@ may just be the identity function, thus
we would get in the same situation as the former example.
%Static analysis is usually unable/unwilling to track precise behaviour of dynamic dispatch.


%In addition to the above we put restrictions on any \Q@mut@ and \Q@capsule@ methods using a \Q@capsule@ field (we call such methods `rep mutators'):
%\begin{itemize}
%\item only a single use of \Q@this@ is allowed (and is the one that uses the field),
%\item no \Q@mut@ or \Q@read@ parameters are allowed (apart from the implicit \Q@this@ parameter)
%\item and the return type cannot be \Q@mut@.
%\end{itemize}
%\noindent  Moreover, if the used \Q!capsule! field is referenced in \validate, a \Q@this.validate()@ call is injected at the end of the method.


Allowing \Q@this@ to be used more than once 
would allow the following code, where 
\Q@this@ may be reachable from \Q@f@, thus \Q@f.hi()@ may observe an object that does not satisfying its invariant:
\begin{lstlisting}
mut method Void multiThis(C c) {//INVALID EXAMPLE: two `this'
  read Foo f = c.foo(this);
  this.items.add(new HeavyItem());
  f.hi(); }//`this' could be observed here if it is in ROG(f)
\end{lstlisting}
\noindent In order to ensure that a second reference to \Q@this@ is not reachable through arguments to such methods, we only allow \Q@imm@ and \Q@capsule@ parameters.
Accepting a \Q@read@ parameter, as in the example below,
would cause the same problems as before, where \Q@f@ may contain
a reference to \Q@this@:
\begin{lstlisting}
mut method Void addHeavy(read Foo f) {//INVALID EXAMPLE
  this.items.add(new HeavyItem());
  f.hi(); }//`this' could be observed here if it is in ROG(f)
...
mut ShippingList l = new ShippingList(new Items());
read Foo f = new Foo(l);
l.addHeavy(f); // We pass another reference to `l' through f
\end{lstlisting}%
%
%, we would have the same problem with a \Q@read@ paramater. ... justify why we ned capsule
% The boat will sink if the weight of the cargo goes over 300. However, 
% \Q@Item@ and \Q@Items@ come from a third party library,  are not annotated with contracts and the authors may change their behaviour in the future. 
% All the code using \Q@Boat@  (client code) would like to  assume the boat has not sunk yet.
% In turn, that depends on the behaviour of \Q@Items.weight()@, thus the meaning of the \Q@Boat@ invariant is parametric on the everchanging meaning of  \Q@Items.weight()@.
% Can the code in the \Q@Boat@ class somehow enforce that for every possible well typed \Q@Item@ and \Q@Items@, client code will interact only with valid (non sunk)  boats?
% That is, we are unable or unwilling to constrain \Q@Item@ and \Q@Items@ to
% cooperate into making \Q@Boat@s unsinkable; 
% we aim to make so that \Q@Boat@s can be correct independently of
% possibly buggy, possibly even malicious \Q@Item@ and \Q@Items@ implementations.
% Indeed, thanks to the encapsulation, any kind of check in the language,
% as in `\Q@if(cargo.weight()>=300){..}@', would delegate the 
% behaviour to untrusted code in \Q@Items@.
%
% \textbf{without any knowledge about the behaviour of \Q@add()@ and \Q@weight()@}
% \footnote{A statically verified system with contracts on all methods may have this kind of knowledge.}
% there is no way we can discover the invariant violation without actually adding the objects and checking the 
% weight after the fact; thus in the general case violations can only be detected 
% when a broken object is already present in the system.
% Remember that to keep our approach lightweight,
% we do not rely on pre-post conditions; thus
% the behaviour of \Q@Items.weight()@ and \Q@Items.add(item)@ is uncertain.
% The names may suggest a specific behaviour, but there is no contract annotated on such methods.
%
% Note also that in the general case there is no way to fix a broken object,
%or to perform a deep clone and to test the operation on the clone first.
%
%
%REWRITE THIS BIT
%Here \Q!capsule! fields 
%as input to our code-generation / \Q@validate()@-injection; that is, \Q@capsule T f@ is expanded by the language into:
%\begin{itemize}
%\item Induce a \Q@capsule@ parameter for the generated %constructor.
%\item Require to be updated with a \Q@capsule@ expression.
%\item Are accessed as a \Q@mut@ field.
%Access is \textbf{not} a destructive read.
% However methods accessing them are kept under
%strict control; either
%\begin{itemize}
%\item they have \Q@read this@: they act like a normal %getter, and can not propagate
%writing permission over the reachable object graph of that field.
%Indeed, with \Q@read this@, any field access \Q@this.f@ will be typed \Q@read@ or \Q@imm@.
%\item they have \Q@mut this@, no parameter is \Q@mut@ or \Q@read@,
%the return type is not \Q@mut@ and \Q@this@ appear exactly one time in
%the method body: we call those methods \textbf{exposers}, and the invariant is going to be checked at the end of
%the exposers.
%\end{itemize}
%
%
%\end{itemize}
%Exposers are the key part of our solution.
%
%
% Those restrictions also enforce that while executing a rep mutator no object outside the reachable object graph of \Q@this@ can be mutated, and thus capability objects cannot be usedI/O can not be performed: the capability objects are externally visible mutable objects and thus the type system will never place them into a \Q@capsule@.
%\subheading{The true expressibility of capsule modifiers}
%A rep mutator method is a wrapper of a logical operation on a field, which is guaranteed to not see the \Q@this@ object.
%Thus, if \Q@this@ where to become broken during 
%the method's execution, we could not observe it until after. At first glance, it may seems that capsule %mutators allows for limited kinds of mutations.
%This is however not the case, consider the following
%general rep mutator method that allows to apply any possible transformation over the content of a capsule %field:
%At first glance it mayseem from
%


\saveSpace
\section{Related work}
\label{s:related}
\saveSpace

\textit{Type Modifiers:}
We rely on a combination of modifiers that are supported by at least 3 languages/lines of research:
L42~\cite{ServettoZucca15,ServettoEtAl13a,JOT:issue_2011_01/article1,GianniniEtAl16},
Pony~\cite{clebsch2015deny,clebsch2017orca}, and Gordon's~\cite{GordonEtAl12}; 
each of these works is accompanied by proofs about the properties of those modifiers.
Since such proofs have already been done, in this work we just assume the required properties.
Those approaches all support deep/strong interpretation, without back-doors.

TM approaches like Javari~\cite{TschantzErnst05,Boyland06} and Rust~\cite{matsakis2014rust} are unsuitable since they introduce back-doors which are not easily verifiable as being used properly.
Many approaches just try to preserve purity (as for example~\cite{pearce2011jpure}), but here we also need aliasing control.
Ownership~\cite{ClarkeEtAl13,ZibinEtAl10,DietlEtAl07} is another popular form of aliasing control that can be used as a building block for static verification~\cite{%
muller2002modular,%
barnett2011specification%
}.%On ownership verification
%Peter Mueller and Arnd Peotzsch Heffter,  eg Müller, P.: Modular Specification and Verification of Object-Oriented Programs, 2002.
%M. Barnett and M. Fähndrich and K. R. M. Leino and P. Müller and W. Schulte and H. Venter: Specification and Verification: The Spec# Experience. Communications of the ACM, 2011.
\MSComm{add discussion on the work reviewrs pointed out}

%\noindent\textit{Strong Exception Safety:}
%Exception safety seems at first glance a smaller issue with respect 
%to the other two, but is the final piece that lets the whole system work in a real world setting.
%Note that state of the art type systems to enforce exception safety
% do not restrict code that do not capture errors, and
%only the point of error capturing is constrained.

\textit{Object Capabilities:}
Object capabilities~\cite{RobustComposition}, in conjunction with type modifiers, are able to
 enforce purity of code in a modular way, without requiring following a monadic style.
The Joe-E language~\cite{finifter2008verifiable} explores how to use object capabilities to ensure pure behaviour for methods.
However, in order to express Joe-E as a subset of Java, they leverage on a simplified model of immutability:
immutable classes must be final with only final fields that refer to immutable classes.
%Instances of immutable classes are immutable objects.
In Joe-E every method that only takes instances of immutable classes is pure.
%\IOComm{Worth mentioning Wyvern? Since alex mentioned that it enforces purirty? Perhaps he can write that section?}
Their model however would not allow the verification of purity for invariant methods of mutable objects.
In contrast our model has a more fine grained representation of immutable/readonly: it is \emph{reference based} instead of \emph{class based}. \IOComm {Redundent?} This means that in our model, every method taking only \Q@read@ or \Q@imm@ references as input is pure,
both in the sense that no object visible outside of the method is mutated, but also that no I/O is performed.

\textit{Class invariants protocols:}
Class invariants are a fundamental part of the design by contract methodology. 
Invariant protocols differ wildeley and can be unsound or complicated, particular due to re-entrancy and aliasing
\cite{leino2004object,drossopoulou2008unified}\IOComm{Cite bertrand meyer? He refers to these as 'referemce leaking' and 'furtive accesses'}.

%literature on class invariant accepts that sometime the object invariant may not hold,
%and that is exacerbated because of 
%Leino, K. R. M. and Müller, P.: Object Invariants in Dynamic Contexts (ECOOP), 2004.
%S. Drossopoulou and A. Francalanza and  P. Müller and A. J. Summers: A Unified Framework for Verification Techniques for Object Invariants ECOOP 2008. 
There are different options as to what object-invariants are known to hold:
\begin{itemize}
\item  when the object is in a \textit{steady} state:
 the execution is not inside any of its (non `helper'~\cite{JML}) methods~\cite{Gopinathan:2008:RMO:1483018.1483028};
 constantly maintained between calls to public methods~\cite{WikiInvariant},
\item at the start and at the end of every public method (which may or may not include recursive method calls)~\cite{Burdy2005};\MSComm{add more citations},
\item at the start and end of every \emph{qualified} call~\cite{?},
\item only when explicitly required (such as in a method contract or implied by another object's invariant) and after an explicit check (`pack') operation~\cite{?},
%\url{https://en.wikipedia.org/wiki/Class_invariant}};
%\item
%constantly maintained when the object is \textit{closed};
%invariant can be manually opened and closed by using special operations; % Add cite here!
\item or, as in this work, when an object could be \emph{involved} in execution.
\end{itemize}
Those different protocols are deceivingly similar, and 
some approaches like JML suggest verifying a simpler approach (that method calls preserve the invariant of the \emph{receiver}) but assume a more powerful one (the invariant of \emph{every} object except the receiver holds).

% use the unsound option of assuming one protocol, but actually checking another.

%DONE IN INTRO breaking class invariants = bug in class code
%braking validation= DEPEND.

%To encode this range of invariant semantics
%in our approach we can add a boolean \Q@isOpen@ field and add \Q@this.isOpen || ..@
%in front of the validity condition.
%Validation can be used to manually encode complex scenarios,
%for example if a method called on an object needs to break the invariant of another object,
%it can do so by manually setting the \Q@isOpen@ flag on the other object.


%On ownership verification
%Peter Mueller and Arnd Peotzsch Heffter,  eg Müller, P.: Modular Specification and Verification of Object-Oriented Programs, 2002.
%M. Barnett and M. Fähndrich and K. R. M. Leino and P. Müller and W. Schulte and H. Venter: Specification and Verification: The Spec# Experience. Communications of the ACM, 2011.

\newcommand\sepItems{\saveSpace\saveSpace\saveSpace\\*${}_{}$\\*${}_{}\,\bullet\,$}

\LINE

\textit{Runtime verification tools:}
Many languages and tools support some form of runtime invariant checking (e.g. Eiffel~\cite{Meyer:1992:EL:129093}, D~\cite{Alexandrescu:2010:DPL:1875434}
and JML~\cite{Burdy2005}).
By looking to a survey by Voigt et al.~\cite{Voigt2013} and the extensive MOP project~\cite{meredith2012overview},
it seems that most runtime verification tools (RV) empower users
to implement what kind of monitoring they see fit for their specific problem at hand. This means that users are responsible for deciding, designing, and encoding both the logical properties and the instrumentation criteria\cite{meredith2012overview}.
In the context of object-invariants, this means the user defines the invariant protocol, and the soundness of such protocol is not checked by the tool.

In practice, this means that the logic, instrumentation, and implementation end up connected:
a specific instrumentation strategy is only good to test certain logic properties in certain applications.
No guarantee is given that the implemented instrumentation strategy is able to support the required logic in the monitored application.
Some of those tools are designed to support class invariants: for example InvTS~\cite{gorbovitski08efficient} lets you write Python conditions that are verified on a set of Python objects, but the programmer needs to be able
to predict which objects are in need of being checked and to use a simpler domain specific language to target them. Hence if a programmer makes a mistake while using this domain specific language, invariant checking
will not be triggered.
Some tools are intentionally unsound and just perform invariant checks following some heuristic that is expected to catch most of failures: jmlrac~\cite{Burdy2005} and Microsoft Code Contracts~\cite{fahndrich2010embedded}.

\IOComm{Put this in Gui section a footnote: whose heuristic also run the invariant checking $77$ times on our GUI case study}

%In particular, the heuristic of 
%We encoded our GUI example also on Microsoft Code Contract; their system also ran the invariant checking $77$ times. Their system is easy to use, but it is unsound since it is built over an unsound/incomplete static verifier~\cite{?}.






%
%In this work we define a language where a minimal, standardized,
%efficient and completely general purpose instrumentation strategy can soundly verify conditions
%expressible as a\\* \Q@read method imm Bool invariant()@, for any well-typed program; with open world assumption
%and possible Byzantine behaviour of any object in the system.
%
%By seeing class invariant as a part of the type of the object,
%the `RV tool' philosophy is akin to letting the programmer customize the behaviour of the
%type system: the programmer implementation may be unsound; while our philosophy is
%to give the user a way to represent complex and expressive types (in the form of arbitrary code in 
%the \Q@invariant()@ method), but 
%the type system implementation is fixed in stone by the language designer.

Many works attempt to move out of the `RV tool' philosophy to ensure RV monitors work as expected, as for example%
%\sepItems
%In avionics, where memory allocation is disallowed, making reasoning about aliasing much simpler~\cite{laurent2015assuring}:
%``\emph{Runtime Verification (RV) can act as the last line of defense to
%protect the public safety, but only if the RV system itself is trusted.}''.
%\sepItems
%In domain specific languages~\cite{ferrari2002guardians}:
%``\emph{Proof techniques for establishing security properties}''.
%\sepItems
%On assertions over restrictive domain specific languages, to tame some of the C/C++
%undefined behaviour~\cite{agten2015sound}:
%``\emph{no verified assertion in the verified
%module will ever fail at runtime, even if the module runs as part of
%a vulnerable application thSound and Unsound monitorsat is subject to code injection attacks}''.
the study of contracts as refinements of types~\cite{findler2001contract}, focusing only on pre and post conditions in OO languages.

Gopinathan \&al.'s.~\cite{Gopinathan:2008:RMO:1483018.1483028} approach keeps invariants under tight control:
relying on powerful aspect-oriented support, they detect any field update in the whole ROG of any object, and check all the invariants that such update may have violated.
They argue against any variation of visible state semantic, where  methods still have to assume that any object may be broken; in such case calling any public method would trigger an error, but while the object is just passed around (and for example stored in collections), the broken state will not be detected.
``\emph{there are many instances where $o$'s invariant is violated by the programmer inadvertently changing the state of $p$ when $o$ is in a steady state. Typically, $o$ and $p$ are objects exposed by the API, and the programmer (who is the user of the API), unaware of the dependency between $o$ and $p$, calls a method of $p$ in such a way that $o$'s invariant is violated. The fact that the violation occurred is detected much later, when a method of $o$ is called again, and it is difficult to determine exactly where such violations occur.}''

However, their approach addresses neither exceptions nor non-determinism caused by I/O, so their work is unsound when those aspects are took in consideration.

Their approach is very computationally intensive, but we think it is powerful enough that it could even be used to roll-back the very field update that caused the invariant to fail, making the object valid again.
We considered a roll-back approach, however rolling back a single-field update is likley to be completley unexpected, rather we should roll back more meaningful operations, similarly to what happens
with transational memory. As with transactional-memory, this is likely to be very hard to support efficiently.
%However we think roll-back this would be a 
%\REVComm{\REVComm{terrible}{2}{It seems in poor taste to complain of ``terrible'' ideas, especially without attempting to demonstrate the improvements of the proposed approach.}}{3}{Nontechnical term. It is not a great idea to label previous work as ``terrible''}
% ideally not only the field-update breaking the invariant should be reverted, %the roll-back should 
Using TMs to enforce strong exception safety is a much simpler alternative, providing the same level of safety, albeit being more restrictive (namely that if the operation did succeed it is still effectively rolled-back).

%: for example
%assume that we are moving object between two boats:
%the overflowing object may be removed from the \Q@cargo@ of the second boat, but it would not
%be placed back in the first boat. It would look like the object has disappeared.
%The important pTheir approach is very computationally intensive, but we think it is powerful enough that it could even be used to roll-back the very field update that caused oint here is that the program would be in an unexpected state
%even if no object invariants are violated, and this would happen \textbf{because} of the 
%invariant checking/fixing behaviour, not because of code written by the programmer.
%We believe that the only viable option is to detect violations after the fact.
\LINE
\IOComm{ADD THOSE TO PERFORMANCE EIFFEL,D\cite{feldman2006jose,fahndrich2010embedded,abercrombie2002jcontractor,tran2003design}}

%\noindent\textit{Performance}
%Our case study shows that our sound approach can monitor programs
%for a fraction of the cost of many other approaches.
%Many other works%
%~\cite{feldman2006jose,fahndrich2010embedded,abercrombie2002jcontractor,tran2003design}
% check/run
%the invariant code at the start and end of every public
%method; this even include trivial getters.
%In  our approach, we call the \validate{} method
%one time at the end of each setter, capsule mutator method and constructor.
%We do not inject it at the end of other methods, which are usually more numerous and invoked much more often.
%Of course, \validate{} can still be called indirectly, for example by calling a setter.
%We expect our approach to result in a dramatic reduction over the number of required checks,
%except for cases when public methods just update many fields directly (without using setters).


\LINE


\noindent\textit{Security and DMZ:}
Static verification lets us reason about a complete program
and verify its correctness.
Traditional static verification is like a mathematical proof: a program is valid if it is all correct,
but a single error invalidates all the claims.
Thus, it is hard to perform verification on large programs, or when independently
maintained third party libraries are involved.
%\REVComm{
%To solve this issue, static verification systems are %starting to
%}{2}{[is this correct?] verification of reference %monitors, gradual typing, and contracts have been %explored for longer}
To soundly verify code embedded in an untrusted 
environment, it is possible to 
consider a verified core
and a run-time verified boundary.
You can see our approach as an extremely modularized version of such system:
every class is its own demilitarized zone, and the rest of the code 
could have Byzantine behaviour.
Our formal proofs 
shown that every class that compiles/type checks is soundly validated
independently of the code that uses this class or any other surrounding code.
Our approach works both with an open world assumption and in a library setting.
Consider for example the work of Parkinson~\cite{parkinson2007class}:
in his short paper he verified an \Q@Observer@ class invariant over
a \Q@Subject/Observer@ pattern.
However, the proof relies on the method \Q@Subject.register(Observer)@ respecting its contract.
Such assumption is unrealistic in a real system with dynamic class loading,
and this invariant could trivially be broken by a user defined \Q@EvilSubject@.


\noindent\textit{Dedicated specification language or underling language:}
Using a specification languages near to the logic and disjointed from a specific language's
semantics may seem attractive, however
a study~\cite{chalin2007logical} discovered that developers expect
the specification language to use the semantics of the underling language, including
short circuit semantics and arithmetic exceptions; thus for example
\Q@1/0 || 2>1@
should not hold, while 
\Q@2>1 || 1/0@ should hold thanks to short circuit semantics.
This study was influential enough to convince JML to change its interpretation of logical expressions
accordingly~\cite{chalin2008jml}.
We believe this is evidence that using a method in the underlying language to encode the validation is
a developers-friendly solution.



${}_{}$\sepItems
Works over C\# recognize the need
for purity/determinism when method calls are allowed in contracts~\cite{barnett200499}
``\emph{There are three main current approaches: a) forbid the use of functions in specifications, b) allow only provably pure functions, or c) allow programmers free use
	of functions. The first approach is not scalable, the second overly restrictive and
	the third unsound.}''\\*
They recognize that many tools unsoundly use option (c), such as AsmL~\cite{barnett2003runtime}.
They propose a concept of observational purity, that if completely fleshed out
could possibly be a great addition to our proposed type system.
We speculate that some 
primitive language support may be needed, for example implementing the Flyweight pattern 
as part of the language semantics.


%1 aliasing control
%  example hamster can be broken with those 2 lines
%2 I/O /determinism
%  hamster EvilPoint with random equal is accepted
%3 exceptions
%  spec sharp is happy to be unsound with capturing %unchecked exceptions
%---
%*TMs, OCs
%
%*expand on invariant protocol
%
%*RV tool
%------------
%*spec# unsond, parkinson critique, and static %verification is like math proof
%
%*Soundness or not
%
%*C#purity, dedicate spec language



%\noindent\textit{Theorem provers and SAT solvers}
%Rather than providing a simple set of rules as to what a \Q@validate@ method can contain,
%and where to insert calls to it, we could instead rely on implementation-specific static analysis:
% in which a \Q@validate@ method is valid iff the compiler can prove that it is deterministic
% and that it’s generated \Q@validate()@ calls are sufficient to enforce validation.
%Though approaches like this are frequently used such as with unifying Java’s generic-wildcards [], Rust’s ‘borrow checker’, …; we believe that would not produce a good result for our purposes: 
%\begin{itemize}
%	\item it would mean that a programmer would have no way of telling whether their code would compile, in particular code compiling would depend on the specific compiler (version) used.
%	\item the runtime cost of validation would be completely unpridictibable; since it is deterministic there is nothing stopping the compiler from calling \Q@validate@ any number of times, and at any point in time.
%	\item When a validation error could be throw would likewise be unpredictable, though it should happen after an object is made invalid\footnote{technically our definition of validation technically allows the error to happen sooner, as long as it’s not too late; however pre-emptive errors like this would be extremely hard to debug}, it could happen any time before it’s use. Making matters worse, if multiple object’s would be invalidated before either is used, which one’s error would be thrown is unconstrained
%	\item This approach will not work well in the pressence of dynamic code loading, in particular it woud likley significantly slow down such loading or spurioslly fail depending on what other code has been loaded
%\end{itemize}



%Conclusions? future work?
%@StrongExceptionSafety is 
%a very strong property,
%and some languages may be unwilling to commit to always preserve it.
%In particular, depending on the details of a specific language
% releasing resources as in \Q@finally@ blocks may require
%some relaxation of @StrongExceptionSafety. Sound releasing of resources could be interesting
%future work.

Related work

Class invariants are a fundamental part of the design by contract methodology. 
Many languages and tools support some form of invariant verification (e.g. Eiffel~\cite{Meyer:1992:EL:129093}, D~\cite{Alexandrescu:2010:DPL:1875434}, JML~\cite{Burdy2005}, Spec\#~\cite{Barnett:2004:SPS:2131546.2131549}).
%In order to be verified, the invariant needs to be expressed in some formal way.
Here we focus on multi-object invariants: the class invariant of a given object may depend upon the observable behaviour of any object referenced in its Reachable Object Graph (ROG).

\noindent\textit{Security and DMZ:}
Static verification let us reason about a complete program
and verify its correctness.
Traditional static verification is like a mathematical proof: is valid if it is \textbf{all correct},
but a single error invalidates all the claims.
Thus, it is hard to perform verification on large programs, or when independently maintained third party libraries
are involved.
To solve this issue, static verification systems are starting to consider a verified core
and a run-time verified boundary.

You can see our approach as an extremely modularized version of such system:
every class is its own demilitarized zone, and the rest of the code 
could have Byzantine behaviour.

Every class that compiles/type checks should be protected against breakage,
 independently of the code that uses this class or any other surrounding code.
 That is, our approach works both with open world assumption and in a library setting.


\saveSpace
\section{Conclusion}
\label{s:conclusion}
\saveSpace

Static verification requires great effort, but can ensures all invariants \textbf{always} holds, thus all objects are always coherent.

However, Static verification is very heavy weight, and often impractical.
In the context of a conventional OO language with imperative features,
we propose an \textbf{ultra-lightweight} verification approach,
where the programmer specifies \textbf{only} the desired class invariants as an 
\Q@invariant()@ method written in the language itself.
This is much more convenient with respect to requiring the specification of methods pre and post conditions,
since the number of classes is usually order of magnitude smaller then the number of methods,
and a fully annotated program requires to write down 
pre-post conditions for each methods, encoding a generalization of its behaviour
in the dedicated specification language.
This means that, even in the best case scenario, 
using pre-post conditions
the user is required to specify the program semantic twice:
first in the specification language and then in the underlying programming language.


With just invariants, our system will then 
\textbf{soundly ensure invariants of all objects involved in the execution}.
Our approach do not rely on assumption over the behaviour of methods/classes;
except for the language semantics and the type system guarantees.
Methods are just treated as black-boxes, producing a result or throwing an error.

Of course, there is a catch: this result is obtained by modifying/instrumenting the
semantic of the language, so that (as for type casts) \textbf{violations are detected at run-time}, and exceptions
are throw in order to stop the execution before involving any broken object.

\noindent\textit{SIC as extended type system:}
The philosophy of our approach is to be like an extended type system: 
\begin{itemize}
\item The programmer decides to annotate a field with a certain type, or the class with a certain invariant.
\item If that is a valid type, or a valid invariant, the user is not questioned in its intent.
\item The system enforce that field will only contain values or that type, or that instance of that class
will respect that invariant.
\end{itemize}
This is in sharp contrast with most work in RV, that is often conceived more as a tool to ease debugging:
both deciding the invariant and enforcing it is controlled by the programmers.


\appendix

\saveSpace
\section{Formalisation of Validation}
\label{s:meaning}
\saveSpace
In order to model our system, we need to formalise an imperative object-oriented language
with exceptions and object capabilities,  and with a rich type system
supporting \Q@mut, imm, read, capsule@ and strong exception safety.
Formally modelling the semantics of such a language is easy, but 
modelling and proving correctness of such a rich type system would deserve a paper
of its own, and indeed many such papers exist in literature%
~\cite{ServettoEtAl13a,ServettoZucca15,GordonEtAl12,clebsch2015deny,JOT:issue_2011_01/article1}.
Thus, we are going to assume that there is an expressive and sound type system enforcing
those properties, and instead focus on validation.
To provide a good modularisation for our reasoning, 
we will clearly list the properties we need to rely upon, so that \emph{every type
system supporting those properties} supports validation.

To encode object capabilities and I/O, we assume a special location
$c$ of type \Q@Cap@.
This location would refer to an object whose fields model, for example, the content of input and output files.
All its methods must require a \Q@mut this@, and shall mutate the ROG of $c$, and the main expression will start with
such $c$ location in scope. In order to simplify our proof, we assume that $c$ only has \Q@mut@ fields, hence it is always valid (i.e. $c.validate()$ is defined to evaluate to \Q@true@).
We strive to keep our small step semantics as conventional as possible; following \REVRComm{Pierce~\cite{pierce2002types}}{2}{A citation to Featherweight Java (TOPLAS 2001) would be specific than [this]} we assume:
\begin{itemize}
\item An implicit program/class-table.
\item Memory $\sigma:\!:\!= l\mapsto C\{\Many{v}\}$ as a finite map from locations $l$ to annotated tuples $C\{\Many{v}\}$ representing objects,
where $C$ is the class name and $\Many{v}$ contains the values of the fields.
We use the notation $\sigma[l.f=v]$ to update an object field and $\sigma[l.f]$ to access the field.
\item A main expression that is reduced in the context of such a memory and program.
\item A reduction relation $\sigma|\e\rightarrow \sigma'|e'$.
\item A type system $\Sigma;\Gamma\vdash\e:T$, where 
the expression $\e$ can contain locations $l$ and free variables $x$;
the type of locations is encoded in the memory environment $\Sigma:\!:\!= l\mapsto C$
and the type of the free variables is encoded in the variable environment $\Gamma:\!:\!= x\mapsto T$.
\item We use $\Sigma^\sigma$ to trivially extract $\Sigma$ from a memory $\sigma$.
\item The special capability object location $c$ of the \Q@Cap@ class; instances of \Q@Cap@ cannot be created with a \Q@new@ expression.
\item We have a special \emph{monitor} expression \Q@M(@$l$\Q@;@$\e_1$\Q@;@$\e_2$\Q@)@.
Those expressions are not present in the source code but are inserted by the reduction.
\REVRComm{Initially we will have $\e_2= l$\Q@.validate()@;
$\e_1$ will be reduced until it becomes a value, then
$\e_2$ will be reduced to test if $l$ is invalid.}{3}{Hard to understand sentence} We annotate the monitor-expression with $l$ to track
that if the validation check fails, it is precisely $l$ that is invalid.
We use a failed monitor expression (i.e. when $\e_2$ is \Q@false@) to represent an \Q@error@ expression.
\item Before reducing the body of a \Q@try@, we annotate it with a snapshot of 
the state of the memory. This is used to annotate that such state will not be mutated by executing the body of the \Q@try@.
\end{itemize}

To keep our formalization focused on
the challenges of validation, 
there are some
tweaks with respect to our informal description of our approach.
From a formal perspective 
these changes do not change expressiveness:
\begin{itemize}
\item We require that all fields are instance-private, as opposed to only capsule fields. One could always provide getters and setters to simulate public fields.
\item We do not have explicit constructor definitions, rather we assume that all constructors are of the canonical form
\Q@$C$($T_1 x_1$,$\ldots$,$T_n x_n$) {this.$f_1$=$x_1$;$\ldots$;this.$f_n$=$x_n$;}@,
 where $T_1,\ldots\T_n$ are the types (including modifiers) of the fields of $C$.
To provide more flexible initialization one could always make a factory method.
\item We require that \Q@.validate()@ can only use \Q@this@ to access fields,
this can be achieved by inlining method calls.
%or if they are recursive, replacing them with calls to methods that take the fields of \Q@this@ instead of \Q@this@ itself.
\item For simplicity, we do not have actual exception objects,
rather we just have a concept of an \emph{error} with no associated value.
We believe adding traditional exceptions would not cause any interesting variation of our proof.
\end{itemize}


\newcommand{\ctxG}{\myCalBig{G}}
\renewcommand{\vs}{\Many{v}}
\renewcommand{\Opt}[1]{#1?}
\begin{figure}
\!\!\!\!
\begin{grammatica}
\produzione{\e}{\x\mid l\mid\Kw{true}\mid\Kw{false}\mid \e\singleDot\m\oR\es\cR\mid \e\singleDot\f 
\mid\e\singleDot\f\equals\e 
\mid\Kw{new}\ C\oR\es\cR
\mid\Kw{try}\ \oC\e_1\cC\ \Kw{catch}\ \oC\e_2\cC
}{expression}\\
\seguitoProduzione{
\mid \Kw{M}\oR l;\e_1;\e_2\cR\mid\Kw{try}^{\sigma}\oC\e_1\cC\ \Kw{catch}\ \oC\e_2\cC
}{run-time expr.}\\
\produzione{v}{l}{value}\\
\produzione{\ctx_v}{\square
\mid \ctx_v\singleDot m\oR\es\cR
\mid v\singleDot\m\oR\Many{v}_1,\ctx_v,\es_2\cR
%\mid \ctx_v\singleDot\f 
%\mid \ctx_v\singleDot\f\equals\e
\mid v\singleDot\f\equals\ctx_v
}{evaluation ctx}\\
\seguitoProduzione{
\mid \Kw{new}\ C\oR\Many{v}_1,\ctx_v,\es_2\cR
\mid \Kw{M}\oR l;\ctx_v;\e\cR
\mid \Kw{M}\oR l;v;\ctx_v\cR
\mid \Kw{try}^\sigma\oC\ctx_v\cC\ \Kw{catch}\ \oC\e\cC}{}\\

\produzione{\ctx}{\square\mid\ctx\singleDot m\oR\es\cR\mid\e\singleDot\m\oR\es_1,\ctx,\es_2\cR
%\mid \ctx\singleDot\f 
%\mid \ctx\singleDot\f\equals\e
\mid \e\singleDot\f\equals\ctx
\mid \Kw{new}\ C\oR\es_1,\ctx,\es_2\cR
}{full ctx}\\
\seguitoProduzione{
\mid
\Kw{M}\oR l;\ctx;\e\cR\mid
\Kw{M}\oR l;\e;\ctx\cR\mid
\Kw{try}^{\sigma?}\oC\ctx\cC\ \Kw{catch}\ \oC\e\cC\mid
\Kw{try}^{\sigma?}\oC\e\cC\ \Kw{catch}\ \oC\ctx\cC

}{}\\


%\produzione{M_l}{\ctx[M\oR l,\e\cR]}{}\\
%\produzione{\ctxG_l}{
%  M_l\singleDot\m\oR\es_1,\ctx,\es_2\cR
% |\e\singleDot\m\oR\es_1, M_l, \es_2, \ctx, \es_3\cR
% |M_l\singleDot\f\equals\ctx
% |\Kw{new}\ C\oR\es_1,M_l,\es_2,\ctx,\es_3\cR
% |\Kw{try}\oC\ctx\cC\ \Kw{catch}\ \oC\e\cC
% |\ctx[\ctxG_l]}{}\\
\produzione{CD}{\Kw{class}\ C\ \Kw{implements}\ \Many{C}\oC\Many{F}\,\Many{M}\cC\mid 
\Kw{interface}\ C\ \Kw{implements}\ \Many{C}\oC\Many{M}\cC
}{class decl}\\
\produzione{F}{\T\ \f;}{field}\\
\produzione{M}{\mdf\, \Kw{method}\, \T\ \m\oR\T_1\,\x_1,\ldots,\T_n\,\x_n\cR\ \Opt\e}{method}\\
\produzione{\mdf}{\Kw{mut}\mid\Kw{imm}\mid\Kw{capsule}\mid\Kw{read}}{type modifier}\\
\produzione{\T}{\mdf\,C}{type}\\
\produzione{r_l}{
 v\singleDot\m\oR\Many{v}\cR
\mid v\singleDot\f
\mid v_1\singleDot\f\equals v_2
\mid \Kw{new}\,C\oR\Many{v}\cR
\quad\text{with }l\in \{v,v_1,v_2,\Many{v}\}
}{$l$ inside a redex}\\
\produzione{\mathit{error}}{
\ctx_v[\Kw{M}\oR l; v;\Kw{false}\cR]
\quad\text{with }
\ctx_v \text{not of form}\ \ctx_v'[\Kw{try}^{\sigma?}\oC\ctx_v''\cC\ \Kw{catch}\ \oC\_\cC]
}{validation error}
\end{grammatica}
\caption{Grammar}
\end{figure}


\loseSpace
\noindent\textit{Grammar and Well-Formedness Criteria:}
The detailed grammar is exposed in Figure 1.
As explained before, the only non-standard expression is the monitor.
We denote with $r_l$ a redex that contains the location $l$.

\noindent Our well formedness criteria are:
\begin{itemize}
\item All field accesses in method bodies are of the form
\Q@this.@$f$. Thus we require all fields to be instance-private.

\item Field accesses in the main expression, 
must be of the form $l\singleDot\f$.

\item \Q@.validate()@ takes a \Q@read this@, and uses \Q@this@ only to access fields. Even calling methods on \Q@this@
is disallowed.
\item All the fields referred in \Q@.validate()@ are either \Q@imm@ or \Q@capsule@.
\item All the methods that access capsule fields 
either have a \Q@read this@,
or have a \Q@mut/capsule this@, no \Q@mut@ or \Q@read@ parameters, no \Q@mut@ result and 
must use \Q@this@ exactly once in their body.
\item 
During reduction, locations $l$ that are preserved by a \Q@try@ block are
never monitored; formally 
in $\Kw{try}^\sigma\oC\e\cC\_$, $\e$ is not of the form $\ctx[$\Q@M(@$l;\_$\Q@)@$]$ with $l\in\sigma$.
\end{itemize}

We model subtyping with interfaces 
and we do not consider subclassing.
Indeed interfaces do not have an implemented \Q@.validate()@ method, objects implementing those interfaces do.
To enrich our formalism with subclassing, we would need to add the 
well-formed criteria that \Q@validate()@ 
methods start by checking the result of \Q@super@.\Q@validate()@.

\begin{figure}
\!\!
$\!\!\!\!\!\begin{array}{l}
 \inferrule[(update)]{{}_{}}{
\sigma|l.f\equals{}v\rightarrow \sigma[l.f=v]|
\Kw{M}\oR l;l;l\singleDot\Kw{validate}\oR\cR\cR
 }{}
\quad
 \inferrule[(new)]{{}_{}}{
\sigma|\Kw{new}\ C\oR\vs\cR\rightarrow \sigma,l\mapsto C\{\vs\}|
\Kw{M}\oR l;l;l\singleDot\Kw{validate}\oR\cR\cR
 }{}
\\[5ex]
 \inferrule[(mcall)]{{}_{}}{
\sigma|l\singleDot\m\oR v_1,\ldots,v_n\cR\rightarrow \sigma|
\e'[\Kw{this}=l,\x_1=v_1,\ldots,x_n=v_n]
 }{
  \begin{array}{l}
  \sigma(l)=C\{\_\}\\
  C.m=\mdf\,\Kw{method}\,\T\,\m\oR\T_1\,\x_1\ldots\T_n\x_n\cR\e\\

\text{if }\ \exists \f\text{ such that}\ \ C.f=\Kw{capsule}\,\_,
\mdf=\Kw{mut},
\\*\quad\f\, \text{inside}\, C\singleDot\m
\text{ and }
\f\,\text{inside}\, C\singleDot\Kw{validate}

\\*
\text{then }\e'=\Kw{M}\oR l;\e;l\singleDot\Kw{validate}\oR\cR\cR\\*
\text{otherwise }\ \e'= \e
  \end{array}
}
\\[5ex]
 \inferrule[(monitor exit)]{{}_{}}{
\sigma|\Kw{M}\oR l; v;\Kw{true}\cR\rightarrow \sigma|v
 }{}
\quad

 \inferrule[(ctxv)]{\sigma_0|\e_0\rightarrow\sigma_1|\e_1}{
\sigma_0|\ctx_v[\e_0]\rightarrow \sigma_1|\ctx_v[\e_1]
 }{}

\quad
 \inferrule[(try enter)]{{}_{}}{
\sigma|\Kw{try}\ \oC \e_1\cC\ \Kw{catch}\ \oC\e_2\cC\rightarrow 
\sigma|\Kw{try}^\sigma\oC\e_1\cC\ \Kw{catch}\ \oC\e_2\cC
 }{}
\quad


\\[5ex]


 \inferrule[(try ok)]{{}_{}}{
\sigma,\sigma'|\Kw{try}^{\sigma}\oC v\cC\ \Kw{catch}\ \oC\_\cC\rightarrow \sigma,\sigma'|v
 }{}
\quad

 \inferrule[(try error)]{{}_{}}{
\sigma,\_|\Kw{try}^\sigma\oC \mathit{error}\cC\ \Kw{catch}\ \oC\e\cC\rightarrow \sigma|\e
 }
\quad
 \inferrule[(access)]{{}_{}}{
\sigma|l.f\rightarrow \sigma|\sigma[l.f]
 }{}
%\quad
\end{array}$
\caption{Reduction rules}
\end{figure}

\loseSpace
\noindent\REVRComm{\textit{Reduction rules:}}{2}{This discussion is surprisingly short}
Reduction rules are defined in Figure 2.
These rules are pretty standard;
\textsc{mcall}
uses the intuitive auxiliary function \emph{inside}
formally defined as follow:

$%\begin{array}{l}
\f\, \text{inside}\, C\singleDot\m\text{ holds iff }
C\singleDot\m=\_\,\Kw{method}\_\,\ctx[\Kw{this}\singleDot\f]
%\end{array}
$

%\noindent Inserting the monitor expressions during reduction is convenient for the proof,
%but it could instead be done ahead of time.

That is, the monitor is added for all field update and new objects, but;
for method calls the monitor is added only if the method has a \Q@mut@ modifier and its body accesses \Q@capsule@ field.

The interaction with monitors and exceptions is interesting:
a monitor releases the value if the check is \Q@true@, and produces an error if the 
check is \Q@false@.
If either $\e_1$ or $\e_2$ are not values, the execution is propagated inside
by \textsc{ctxv}.
If either $\e_1$ or $\e_2$ evaluate to an error, such error is captured by 
\textsc{try error}.
Thanks to strong exception safety
we do not need to worry
if the (partial) execution of $\e_1$ broke the $l$ object.
If the language were to support checked and unchecked exceptions, but offered 
strong exception safety only for the unchecked ones, then 
the type system should require neither $\e_1$ nor $\e_2$ leak 
checked exceptions.





%WHERE TO PUT THIS?
%Note that for \Q@capsule@ fields, the constructor and the field update
%will require \Q@capsule@ for the correspoi, while the field access will produce a \Q@mut@.



\loseSpace
\noindent\textit{Axiomatic type properties:}
As discussed, instead of providing a concrete set of type rules, we provide a set of properties
that such a type system needs to respect.
To express these properties, we first need some auxiliary definitions:

%\noindent\textbf{Define}
%$\mathit{encapsulatedObj}(C)$:\\*
%${}_{}$\quad\quad \Q@class @$C$\,\Q@implements @$\Many{C}$\Q@{@$\,\Many{F}\,\Many{M}$\Q@}@
% and $\forall \mdf\,C\,\f \in \Many{F},\ \mdf \in \{\Kw{imm},\Kw{capsule}\}$\\*
%\noindent As we discussed, only encapsulated objects can support invariants;
%their class declarations only have immutable or capsule fields. Note how here we see immutable
%and simple objects as special cases of encapsulated ones.

\noindent\textbf{Define} $\mathit{erog}(\sigma,l_0)$:\\*
\indent $l \in \mathit{erog}(\sigma,l_0)
\text{ if } \Sigma^\sigma(l_0).f \in \{\Kw{imm}\,\_,\Kw{capsule}\,\_\}
\text{ and } l \in \mathit{rog}(\sigma,\sigma(l_0).f)
$\\*
\noindent
The encapsulated ROG of $l_0$ is composed by all the objects
in the ROG of its immutable and capsule fields.


\noindent\textbf{Define} $\mathit{mutatable}(l,\sigma,\e)$:\\*
\indent with $T=\Kw{imm}\,\Sigma^\sigma(l)$ and $\e=\ctx[l]$,\\*
\indent $\Sigma^\sigma;\x:T\vdash\ctx[\x]:T'$ does not hold for any $T'$.\\*
\noindent That is, an object is mutatable by a $\sigma,\e$ if there is an occurrence of 
$l$ in $e$ that when seen as immutable makes the expression ill-typed.



\noindent\textbf{Define}$\ \sigma_0|e_0\Rightarrow \sigma_1|e_1$:\\*
\indent iff $\{\sigma_1|\e_1\}=\{\sigma|\e \text{ where } \sigma_0|e_0\rightarrow \sigma|e\}$

%if $\ \sigma_0|e_0\rightarrow \sigma|e$ then $\sigma_1|\e_1=\sigma|\e$
% $\exists! \sigma_1|\e_1$ such that $\sigma_0|\e_0\rightarrow \sigma_1|\e_1$\\*
\noindent We define
a deterministic reduction arrow.
Here we require that there is exactly one reduction possible.


%We can now assume the following properties over the type system:

\begin{Assumption}[Progress]
if $\Sigma^{\sigma_0};\emptyset\vdash e_0: T_0$,
and $e_0$ not a value or $\mathit{error}$, then
$\sigma_0|e_0\rightarrow \sigma_1|e_1$
\end{Assumption}


\begin{Assumption}[SubjectReductionBase]
if $\Sigma^{\sigma_0};\emptyset\vdash e_0: T_0$,
$\sigma_0|e_0\rightarrow \sigma_1|e_1$,
then
$\Sigma^{\sigma_1};\emptyset\vdash e_1: T_1$
\end{Assumption}


\begin{Assumption}[MutField]
\ \\
\indent(1)\ if $\Sigma;\Gamma\vdash\e\singleDot\f:\Kw{mut}\,\_$
then $\Sigma;\Gamma\vdash\e:\Kw{mut}\,\_$
,\ and 
\\*\indent(2)
if $\Sigma;\Gamma\vdash\e_0\singleDot\f\equals\e_1:T$
then $\Sigma;\Gamma\vdash\e_0:\Kw{mut}\,\_$
\end{Assumption}
\noindent If the result of a field access is mutable,
the receiver is mutable too, and the receiver of a field update is always mutable.

\begin{Assumption}[HeadNotCircular]
if
$\Sigma^\sigma;\Gamma\vdash l:T$
then $l\notin\text{erog}(\sigma,l)$
\end{Assumption}
\noindent
\noindent An object is not part of the ROG of its immutable or capsule fields.


\begin{Assumption}[CapsuleTree]
If   $\Sigma^\sigma;\Gamma\vdash \e:\T$,
$l_2\in\text{erog}(\sigma,l_1)$,
$l_1\in\text{erog}(\sigma,l_0)$,\\*
and
$\mathit{mutatable}(l_2,\sigma,\e)$
then 
$l_2\notin\text{erog}(\sigma\setminus l_1,l_0)$
\end{Assumption}
\noindent In a well typed $\sigma,e$, if mutatable $l_2$ is reachable from
$l_1$, and $l_1$ is reachable from $l_0$,
then all the paths connecting $l_0$ and $l_2$ pass trough $l_1$; thus
if we was to remove the node $l_1$ from the object graph, $l_0$ would not reach $l_2$ any more.


CapsuleTree and HeadNotCircular together 
shows that capsule fields section the object graph into a tree of nested `balloons',
where nodes are mutable encapsulated objects and
edges are given by reachability between those objects in the original memory:

$l_2$ is in the encapsulated ROG of $l_1$;
$l_2$ is mutatable and is reached trough $l_1$, thus
it must be reachable by a \Q@capsule@ field.
Thanks to HeadNotCircular and $l_1\in\text{erog}(\sigma,l_0)$ we can derive 
$l_0\notin\text{erog}(\sigma,l_1)$.



\begin{Assumption}[Determinism]
if $\emptyset;\Gamma\vdash \e:\T$, 
$\forall x \Gamma(x)\neq\Kw{mut}\,\_$, and
$\sigma | \e'\rightarrow^+ \sigma' | \e''$
then 
$\sigma | \e'\Rightarrow^+ \sigma,\_ | \e''$,
where $\e'=\e[x_1=l_1,\ldots,x_n=l_n]$ and $\Sigma^\sigma;\emptyset\vdash \e':\T$
\end{Assumption}
\noindent The execution of an expression
with no \Q@mut@ free variables is deterministic and does not
  mutate pre existing memory (and thus does not not perform I/O by mutating pre existing $c$).


\begin{Assumption}[StrongExceptionSafety]
if $\Sigma^{\sigma,\sigma'};\emptyset\vdash \ctx[\Kw{try}^\sigma\oC\e_0\cC\ \Kw{catch}\ \oC\e_1\cC]:\T$
and\\*
$
\sigma,\sigma'|\ctx[\Kw{try}^\sigma\oC\e_0\cC\ \Kw{catch}\ \oC\e_1\cC]\rightarrow 
\sigma''|\ctx[\Kw{try}^\sigma\oC\e'\cC\ \Kw{catch}\ \oC\e_1\cC]
$
then 
$\sigma''=\sigma,\_$
and
$\Sigma^\sigma;\emptyset\vdash \ctx[\e_1]:\T$
\end{Assumption}
\noindent
For each \Q@try-catch@, the execution preserves the memory needed to continue the execution in case of error
(the memory visible outside of the \Q@try@).%

%Thanks to how our reduction rules are designed, especially \textsc{try error},
%@Progress will need to rely on @StrongExceptionSafety internally.

Note that our last well formedness rule requires 
\textsc{update} and \textsc{mcall} to introduce
monitor expressions only over locations
that are not preserved by a \Q@try@ block.
This can be achieved since monitors are introduced
around $\mathit{mutating}$ operations
(and \textsc{new}),
and StrongExceptionSafety ensures no mutation happens on the preserved memory.

To the best of our knowledge, only the type system of 42~\cite{ServettoEtAl13a,ServettoZucca15}
 supports all these assumptions out of the box,
while both Gordon~\cite{GordonEtAl12} and Pony~\cite{clebsch2015deny,clebsch2017orca} supports all except StrongExceptionSafety,
however it should be trivial to modify them to support it:
the \Q@try-catch@ rule could be modified to
$\emptyset;\Gamma\vdash\Kw{try}\ \oC\e_0\cC\ \Kw{catch}\ \oC\e_1\cC:\T$
if\\* $\emptyset;
\Gamma,\{x:\Kw{read}\,C | x:\Kw{mut}\,C\,\in\Gamma\}
\vdash\e_0:\T$ and $\emptyset;\Gamma\vdash\e_1:\T$,
i.e. $e_0$ can be typed when seeing all externally defined mutable references as \Q@read@.

\loseSpace
\noindent\textit{Statement of Validation:}
%We first need to define what it means for an object to be valid:
An object is \emph{valid} iff calling its \Q@.validate()@ method would
deterministically produce \Q@true@ in a finite number of steps, i.e. it does not evaluate to \Q@false@, fail to terminate, or produce an error.
We also require that evaluating \Q@.validate()@ preserve existing memory ($\sigma$), but new objects ($\sigma'$) can be created and freely mutated.

\noindent\textbf{Define} $valid(\sigma,l)$:\\*
\indent $\sigma | l.validate()\Rightarrow^+ \sigma,\sigma’ | \text{\Q@true@}$

\noindent In order for validation to be meaningful it needs to be possible for \Q@.validate()@ to potentially observe an invalid object. However, invalid objects should not be observed outside of \Q@.validate()@.
For this purpose we define the set of trusted steps, 
as the call to \Q@.validate()@ and the field accesses inside a monitor.
Note that just the single small-step reduction
of calling \Q@.validate()@ is trusted, not the whole evaluation of the \Q@.validate()@ expression.


\noindent\textbf{Define} $\mathit{trusted}(\ctx_v,r_l)$:\\*
\indent either
$r_l=l$\Q@.validate()@ and
 $\ctx_v=\ctx_v'[$\Q@M(@$l$\Q@;@$v$\Q@;@$\square$\Q@)@$]$\\*
\indent or
$r_l=l$\Q@.f@ and
 $\ctx_v=\ctx_v'[$\Q@M(@$l$\Q@;@$v$\Q@;@$\ctx_v''$\Q@)@$]$

\noindent Finally, we can now define what it means for a language to soundly enforce validation: every object involved in any untrusted redex is valid.

\begin{theorem}[Sound Validation]
if $c:\Kw{Cap};\emptyset\vdash \e: \T$ and
$c\mapsto\Kw{Cap}\{\_\}|\e\rightarrow^+ \sigma|\ctx_v[r_l]$, then
either $valid(\sigma,l)$ or $\mathit{trusted}(\ctx_v,r_l)$.
\end{theorem}

We believe this property captures very precisely our statement in Section~\ref{s:validation}.
The proof is in Appendix~\ref{s:proof}. 
%The structure of the proof is interesting:
%It is hard to prove Sound Validation directly,
%so we first define a stronger property,
%called Stronger Sound Validation and
%we show that it is preserved during reduction by mean of conventional Progress and Subject Reduction.
%That is,
%Progress+Subject Reduction $\Rightarrow$ Stronger Sound Validation
%and Stronger Sound Validation $\Rightarrow$ Sound Validation.
\appendix
\section{Proof} 
\label{s:proof}

\begin{theorem}[Sound Validation]
	if $c:\Kw{Cap};\emptyset\vdash \e: \T$ and
	$c\mapsto\Kw{Cap}\{\_\}|\e\rightarrow^+ \sigma|\ctx_v[r_l]$, then
	either $valid(\sigma,l)$ or $\mathit{trusted}(\ctx_v,r_l)$.
\end{theorem}

We believe this property captures very precisely our statement in Section~\ref{s:validation}.

It is hard to prove Sound Validation directly,
so we first define a stronger property,
called \emph{Stronger Sound Validation} and
show that it is preserved during reduction by means of conventional 
Progress and Subject Reduction (Progress is one of our assumption,
while Subject Reduction relies heavily on SubjectReductionBase).
That is,
Progress+Subject Reduction $\Rightarrow$ Stronger Sound Validation,
\\*and Stronger Sound Validation $\Rightarrow$ Sound Validation.

\subsection{Stronger Sound Validation $\Rightarrow$ Sound Validation}

Stronger Sound Validation depends on 
$\mathit{wellEncapsulated}$, $\mathit{monitored}$
and $OK$:

\noindent\textbf{Define} $\mathit{wellEncapsulated}(\sigma,\e,l_0)$:\\*
\indent$\forall l \in \mathit{erog}(\sigma,l_0), \text{not}\ \mathit{mutatable}(l,\sigma,\e)$

\noindent The main idea is that an object is well encapsulated if its encapsulated state is safe from
modification. 

\noindent\textbf{Define} $\mathit{monitored}(\e,l)$:\\*
\indent$\e=\ctx_v[M(l,\e_1;\e_2)]$ and either $\e_1=l$ or $l$ is not inside $\e_1$.

\noindent An object is monitored if the execution
is currently inside of a monitor for that object, and
the monitored expression $\e_1$ does not
contains $l$ as a \emph{proper} subexpression.

A monitored object is associated with an expression that can not observe it, but may 
reference its internal representation directly.
In this way, we can safely modify its representation before checking for the invariant.

The idea is that at the start the object will be valid and $\e_1$ will contain $l$;
but during reduction, the $l$ reference will be used in order to
give access to the internal state of $l$; only after that moment, the object may become invalid.


\noindent\textbf{Define} $OK(\sigma,e)$:\\
\indent $\forall l\in\dom(\sigma)$
  either\\
\indent\indent 1. $\mathit{garbage}(l,\sigma,\e)$\\
\indent\indent 2. $\mathit{valid}(\sigma,l)$ and $\mathit{wellEncapsulated}(\sigma,\e,l)$\\
\indent\indent 3. $\mathit{monitored}(\e,l)$

Finally, the system is in a valid state with respect to validation
if for all the objects in the memory, one of these 3 cases apply:
%the class of the object has no invariant method;
the object is not (transitively) reachable from the expression (thus can be garbage collected);
the object is valid, and the object is encapsulated;
or the object is currently monitored.

\begin{theorem}[Stronger Sound Validation]
if $c:\Kw{Cap};\emptyset\vdash \e_0: \T_0$ and
$c\mapsto\Kw{Cap}\{\_\}|\e_0\rightarrow^+ \sigma|\e$, then
$OK(\sigma,\e)$
\end{theorem}
\noindent Starting from only the capability object,
any well typed expression $\e_0$ can be reduced for an arbitrary amount of steps,
and $IOK$ will always hold.
\\
\begin{theorem} Stronger Sound Validation $\Rightarrow$ Sound Validation
\end{theorem}
\begin{proof}
\noindent By Stronger Sound Validation, each $l$ in the current redex must be $OK$:
\begin{enumerate}
	\item If $l$ is garbage, it cannot be in the current redex, a contradiction.
	\item If $\mathit{valid}(\sigma,l)$, then $l$ is valid, so thanks to Determinism
	no invalid object could be observed.
	\item Otherwise, if $\mathit{monitored}(\e,l)$ then either:
	\begin{itemize}
	 \item we are executing inside of $\e_1$ thus the current redex is inside of a sub-expression of the monitor that does not contain $l$, a contradiction.
	 \item or we are executing inside $\e_2$:
	 by our reduction rules, all monitor expressions start with 
	 $\e_2=l$\Q@.validate()@, thus the first execution step
	 of $\e_2$ is trusted. Following execution steps are also trusted, since by well formedness the body of invariant methods only use \Q@this@ (now translated to $l$) to access fields.
	\end{itemize}
\end{enumerate}
In any of the possible cases above, Sound Validation holds for $l$, and so it holds for all redexes.
\end{proof}

\subsection{Subject Reduction}

\noindent\textbf{Define} $\text{fieldGuarded}(\sigma,\e)$:\\*
\indent$\forall \ctx$ such that $\e=\ctx[l\singleDot\f] $
and $\Sigma^\sigma(l).f=\Kw{capsule}\,\_$, and $\f\mathrel{\mathit{inside}} \Sigma^\sigma(l).\mathit{validate}$\\*
\indent\indent either 
$\forall T, \forall C, \Sigma^\sigma;\x:\Kw{mut}\,C\,\not\vdash\ctx[\x]:T$, or\\*
\indent\indent $\ctx=\ctx'[$\Q@M(@$l$\Q@;@$\ctx''$\Q@;@$\e$\Q@)@$]$ and $l$ is contained exactly once in $\ctx''$

That is, all \emph{mutating} capsule field accesses are individually guarded by monitors.
Note how we use $C$ in $\x:\Kw{mut}\,C$ to guess the type of the accessed field,
and that we use the full context $\ctx$ instead of the evaluation context $\ctx_v$
to refer to field accesses everywhere in the expression $\e$.


\begin{theorem}[Subject Reduction]
if $\Sigma^{\sigma_0};\emptyset\vdash e_0: T_0$,
$\sigma_0|e_0\rightarrow \sigma_1|e_1$,
$OK(\sigma_0,\e_0)$
and
$\mathit{fieldGuarded}(\sigma_0,\e_0)$
then
$\Sigma^{\sigma_1};\emptyset\vdash e_1: T_1$,
$OK(\sigma_1,e_1)$ and
$\mathit{fieldGuarded}(\sigma_1,\e_1)$
\end{theorem}

\begin{theorem}
	Progress + Subject Reduction $\Rightarrow$ Stronger Sound Validation
\end{theorem}
\begin{proof}
This proof proceeds by induction in the usual manner.

\emph{Base Case}: At the start of the execution, the memory is going to only contain $c$: since $c$ is defined to be initially $\mathit{valid}$, and has only \Q@mut@ fields, and so it is trivially $\mathit{wellEncapsulated}$, thus $OK(c\mapsto\Kw{Cap},e)$.

\emph{Induction}: By Progress we always have another evaluation step to take, by Subject Reduction such a step will preserve $\mathit{OK}$, and so by induction $\mathit{OK}$ holds after any number of steps.

Note how for the proof garbage collection is important: 
when the \Q@validate()@ method evaluates to \Q@false@, 
execution can continue only if the offending object is classified as garbage.
\end{proof}

\subsection{Proof of Subject Reduction}
We first introduce a lemma derived from well formedness and the type system:
\begin{Lemma}[ExposerInstrumentation]
If $\sigma_0 | \e_0\rightarrow \sigma_1 |\e_1$ and
$\text{fieldGuarded}(\sigma_0,\e_0)$
\\*
then $\text{fieldGuarded}(\sigma_1,\e_1)$
\end{Lemma}
\begin{proof}
The only rule that can 
introduce a new field access is \textsc{mcall}.
In that case, ExposerInstrumentation holds
by well formedness (all field accesses in methods are of the form \Q@this.f@) 
and since \textsc{m call} inserts a monitor while invoking capsule mutator methods, and not field accesses themselves. If however the method is not a \Q@mut@ method but still accesses a capsule field, by MutField such a field access expression cannot be typed as \Q@mut@ and so no monitor is needed.

Note that \textsc{monitor exit} is fine because monitors are removed only when
 $e_1$ is a value.
\end{proof}

\begin{theorem}
	Subject Reduction Base $\Rightarrow$ Subject Reduction
\end{theorem}
\begin{proof}
Any reduction step can be obtained
by exactly one application of rule \textsc{ctx} and then one other rule.



Thus the proof can simply proceed by cases on such other applied rule.

By SubjectReductionBase and ExposerInstrumentation, 
$\Sigma^{\sigma_1};\emptyset\vdash e_1: T_1$ and  $\mathit{fieldGuarded}(\sigma_1,\e_1)$. So we just need to proceed by cases on the reduction rule applied to verify that $OK(\sigma_1,\e_1)$:


\begin{enumerate}
\item \textsc{update:} $\sigma|l\singleDot f\equals v\rightarrow \sigma'|\e'$:
\begin{itemize}
  \item by \textsc{update} $\e'=\Kw{M}\oR l;l;l\singleDot\text{validate}\oR\cR\cR;$, thus $\mathit{monitored}(\e,l)$.
  \item Every $l_1$ such that $l\in \text{rog}(\sigma,l_1)$ will verify the same case
  as the former step:
  \begin{itemize}
  	\item If it was $\mathit{garbage}$, clearly it still is.
  	\item If it was $\mathit{monitored}$, it also still is.
  	\item If can't have been $\mathit{wellEncapsulated}$ since $mutatable(l, \sigma, e)$, (by MutField)
  \end{itemize}
  \item Every other $l_0$ is not reached by $l$ thus it being $\mathit{OK}$ could not have been effected by this reduction step.
\end{itemize}

\noindent\textbf{case field access} $l.f\rightarrow v$:

    If for $l$ $IOK$ holds by (2),  
    it is possible that the next step is not encapsulated.
    This would mean that the field $f$ is a capsule and that we are required
to type it as \Q@mut@ to type the expression for the next step.
By $\mathit{fieldGuarded}(\sigma_0,\e_0)$
    the former step was inside of a monitor \Q@M(@$l$\Q@;@$\ctx_v[l$\Q@.f@$]$\Q@;@$\e$\Q@)@
    and the $l$ under reduction was the only occurrence of $l$.
    since $f$ is a capsule, we know that $l\notin \text{erog}(\sigma,l)$
    by HeadNonCircular.
    Thus in the new step not $l\, \text{inside}\ \ctx_v[v]$.
    Thus for l (3)[monitored] holds.
    
We still need to show that properties $\mathit{monitored}$ and $\mathit{wellEncapsulated}$
 for other objects are
not disturbed. This is the point where our aliasing and mutability control are most crucial:
We know that mutable $v$ is (directly) reachable from
$l$ that have invariant.
Thanks to CapsuleTree we know that for all $l_0$ reaching $l$,
$v$ can be reached by $l_0$ only passing trough $l$.
Thus, we can conclude  $l_0$ is not encapsulated in the former step (containing mutable $l$).
Thus, $l_0$ is either without invariant, garbage or monitored.
None of those 3 cases can be disturbed by a field access.


\noindent\textbf{case meth call}:\\*
  This reduction step does not influence any object in the memory and does not
disturb the properties $\mathit{monitored}$ and $\mathit{wellEncapsulated}$.

\noindent\textbf{case new}:\\*
  If $C$ has invariant, then by @ConstructionInstrumentation the new object is monitored.
As for the method call, other objects and properties are not disturbed.


\noindent\textbf{case monitor exit} \Q@M(@$l;v;$\Q@true)@$\rightarrow v$ :
  \begin{itemize}
\item
    If it was a setter $v=l$, and 
    thanks to Determinism the execution of invariant is deterministic;
    thus for $l$ in the former step both case (2) and (3) holds.
    In the next step (2) will hold for $l$.
\item
    If it was a capsule mutator method, thanks to Determinism the execution
 of \Q@.validate()@ is deterministic;
    thus for $l$ in the former step both $H$ and case (3) holds.
    Thanks to ExposerInstrumentation $v$ is offered without mutation permissions, so
    In the next step $l$ is encapsulated and (2) will hold.
\item
    If it is was a constructor, 
    then $v$ is encapsulated and thanks to Determinism
    the execution of invariant is deterministic, thus in the next step (2) will hold.
\end{itemize}

\noindent\textbf{case try enter and try ok}
This case do not influence any object in the memory and does not
disturb the properties $\mathit{monitored}$ and $\mathit{wellEncapsulated}$.

\noindent\textbf{case try catch} $\sigma,\sigma_0|\Kw{try}^\sigma \oC\mathit{error}\cC\Kw{catch}\, \e\rightarrow \sigma|\e$:\\*
From the premise we know 
$IOK(\sigma,\sigma_0;\ctx_v[\Kw{try}^\sigma \oC\mathit{error}\cC\Kw{catch}\, \e])$;
thus we need to show
$IOK(\sigma;\ctx_v[\e])$.
By StrongExceptionSafety we know that $\sigma_0$ is garbage with respect to $\ctx_v[\e]$.

There could be many $l$ inside $\sigma,\sigma_0$ that are $\mathit{monitored}$
in the former step thanks to monitor expressions inside $\mathit{error}$.
However, all such $l$ are defined inside $\sigma_0$,
for the last well formedness condition.
\end{enumerate}
\end{proof}


\bibliography{main}


\end{document}
