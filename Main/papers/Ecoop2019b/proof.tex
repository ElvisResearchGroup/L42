\clearpage
\section{Proof and Axioms}
\label{s:proof}
\lstset{morekeywords={fwd}}

\subheading{Axiomatic Type Properties}
As previously discussed, instead of providing a concrete set of type rules, we provide a set of properties
that the type system needs to respect.
To express these properties, we first need some auxiliary definitions.

%\noindent\textbf{Define}
%$\mathit{encapsulatedObj}(C)$:\\*
%${}_{}$\quad\quad \Q@class @$C$\,\Q@implements @$\Many{C}$\Q@{@$\,\Many{F}\,\Many{M}$\Q@}@
% and $\forall \mdf\,C\,\f \in \Many{F},\ \mdf \in \{\Kw{imm},\Kw{capsule}\}$\\*
%\noindent As we discussed, only encapsulated objects can support invariants;
%their class declarations only have immutable or capsule fields. Note how here we see immutable
%and simple objects as special cases of encapsulated ones.

The encapsulated ROG of $l_0$ is composed of all the objects
in the ROG of its immutable and capsule fields:\\*
\indent $l \in \mathit{erog}(\sigma,l_0)
\text{ iff } \exists f, \Sigma^\sigma(l_0).f \in \{\Kw{imm}\,\_,\Kw{capsule}\,\_\}
\text{ and } l \in \mathit{rog}(\sigma, \sigma[l_0.f])$% \loseSpace
\IOComm{TODO: Change mutatable to take $(\s, e, l)$}
\noindent An object is \emph{mutatable} in a $\sigma$ and  $\e$ if there is an occurrence of 
$l$ in $e$, that when seen as \Q@imm@ makes the expression ill typed:\\*
\indent $\mathit{mutatable}(l,\sigma,\e)$ iff for some $T=\Kw{imm}\,\Sigma^\sigma(l)$ and $\ctx[l]=\e$,\\*
\indent \indent $\Sigma^\sigma;\x:T\vdash\ctx[\x]:T'$ does not hold for any $T'$.
\IOComm{Should we have an assumption that if an $l$ is not mutatable, its rog won't be mutated? Currently we just use Mut Field vaguely to say essentially the same thing...}

%if $\ \sigma_0|e_0\rightarrow \sigma|e$ then $\sigma_1|\e_1=\sigma|\e$
% $\exists! \sigma_1|\e_1$ such that $\sigma_0|\e_0\rightarrow \sigma_1|\e_1$\\*

%We can now assume the following properties over the type system:

Here we assume the usual \thm{Progress} and \thm{Subject Reduction Base}. Note that \thm{Subject Reduction Base} only ensures properties about type checking, not invariant checking.\saveSpace\saveSpace
\begin{Assumption}[Progress]\rm
	if $\Sigma^{\sigma_0};\emptyset\vdash e_0: T_0$,
	and $e_0$ is not of form $l$ or $\mathit{error}$, then
	$\sigma_0|e_0\rightarrow \sigma_1|e_1$.
\end{Assumption}

\begin{Assumption}[Subject Reduction Base]\rm
	if $\Sigma^{\sigma_0};\emptyset\vdash e_0: T_0$,
	$\sigma_0|e_0\rightarrow \sigma_1|e_1$,
	then
	$\Sigma^{\sigma_1};\emptyset\vdash e_1: T_1$.
\end{Assumption}


If the result of a field access is \Q!mut!,
the receiver is also \Q!mut!, in addition field updates are only allowed on \Q!mut! receivers:\saveSpace\saveSpace
\begin{Assumption}[Mut Field]\rm
	\ \\
	\indent(1)\ if $\Sigma;\Gamma\vdash\e\singleDot\f:\Kw{mut}\,\_$
	then $\Sigma;\Gamma\vdash\e:\Kw{mut}\,\_$
	and 
	\\*\indent(2)
	if $\Sigma;\Gamma\vdash\e_0\singleDot\f\equals\e_1:T$
	then $\Sigma;\Gamma\vdash\e_0:\Kw{mut}\,\_$.
\end{Assumption}%
\IOComm{Have a subtyping assumption? So that whenever we talk about mut, it also aplies to capsule...}

The execution of an expression
with no \Q@mut@ free variables is deterministic and does not
mutate pre-existing memory (and thus does not not perform I/O by mutating the pre-existing $c$):\saveSpace\saveSpace
\begin{Assumption}[Determinism]\rm
	if $\emptyset;\Gamma\vdash \e:\T$, 
	$\forall x\, ( \Gamma(x)\neq\Kw{mut}\,\_$), and
	$\sigma | \e'\rightarrow^+ \sigma' | \e''$
	then 
	$\sigma | \e'\Rightarrow^+ \sigma,\_ | \e''$,
	where $\e'=\e[x_1=l_1,\ldots,x_n=l_n]$ and $\Sigma^\sigma;\emptyset\vdash \e':\T$
\end{Assumption}

For each \Q@try@--\Q@catch@, execution preserves the memory needed to continue the execution in case of an error (the memory visible outside of the \Q@try@).\saveSpace\saveSpace
\begin{Assumption}[Strong Exception Safety]\rm
	if $\Sigma^{\sigma,\sigma'};\emptyset\vdash \ctx[\Kw{try}^\sigma\oC\e_0\cC\ \Kw{catch}\ \oC\e_1\cC]:\T$
	and\\*\indent
	$
	\sigma,\sigma'|\ctx[\Kw{try}^\sigma\oC\e_0\cC\ \Kw{catch}\ \oC\e_1\cC]\rightarrow 
	\sigma''|\ctx[\Kw{try}^\sigma\oC\e'\cC\ \Kw{catch}\ \oC\e_1\cC]
	$
	then 
	$\sigma''=\sigma,\_$, $\Sigma^\sigma;\emptyset\vdash \ctx[\e_1]:\T$,
	\\*\indent and $\forall l \in \dom(\sigma)$, not $\mathit{mutatable}(l, \sigma, e_0)$
\end{Assumption}

%Thanks to how our reduction rules are designed, especially \textsc{try error},
%@Progress will need to rely on @StrongExceptionSafety internally.

SES allows us to prove that locations preserved by \Q@try@ blocks are never monitored:
\begin{Lemma}[Unmonitored Try]\rm
	if $c:\Kw{Cap};\emptyset\vdash \e: \T$ and $c\mapsto\Kw{Cap}\{\_\}|\e\rightarrow^+ \_|\Kw{try}^\sigma\oC\ctx[\M{l}{\_}{\_}]\cC\ \_$, then $l\notin\sigma$
\end{Lemma}
\begin{proof}
The proof is by induction: after 0 reduction steps, $\e$ cannot contain a monitor expression, since they may only be introduced by reduction rules. If this property holds after $n$ reduction steps, but is violated on the $n+1$'th step, we must have applied the \textsc{update}, \textsc{mcal} or \textsc{new} rules, since they are the only ones that can introduce a montior expression. If the rule was a \textsc{new} rule, $l$ will be fresh, so it could not have been in $\sigma$. If the rule was an \textsc{update}, by \ASS{mut-r update}, $l$ must have been \Q!mut!, similarly \textsc{mcall} will only introduce a monitor over a \Q!mut! receiver, either way $l$ is $\mathit{mutatable}$, so by \thm{Strong Exception Safety}, we have that $l \notin \sigma$.
\end{proof}

% To the best of our knowledge, only the type system of 42~\cite{ServettoEtAl13a,ServettoZucca15}
%  supports all these assumptions out of the box,
% while both Gordon~\cite{GordonEtAl12} and Pony~\cite{clebsch2015deny,clebsch2017orca} supports all except StrongExceptionSafety,
% however it should be trivial to modify them to support it:
% the \Q@try-catch@ rule could be modified to
% $\emptyset;\Gamma\vdash\Kw{try}\ \oC\e_0\cC\ \Kw{catch}\ \oC\e_1\cC:\T$
% if\\* $\emptyset;
% \Gamma,\{x:\Kw{read}\,C | x:\Kw{mut}\,C\,\in\Gamma\}
% \vdash\e_0:\T$ and $\emptyset;\Gamma\vdash\e_1:\T$,
% i.e. $e_0$ can be typed when seeing all externally defined mutable references as \Q@read@.



\subheading{Proof of Soundness}
It is hard to prove \thm{Soundness} directly,
so we first define a stronger property,
called \thm{Stronger Soundness}, and
show that it is preserved during reduction by means of conventional
\thm{Progress} and \thm{Subject Reduction} (\thm{Progress} is one of our assumptions,
while \thm{Subject Reduction} relies heavily upon \thm{Subject Reduction Base}).
That is:
\begin{itemize}
\item \thm{Progress} $\wedge$ \thm{Subject Reduction} $\Rightarrow$ \thm{Stronger Soundness}, and
\item \thm{Stronger Soundness} $\Rightarrow$ \thm{Soundness}.
\end{itemize}
%The structure of the proof is interesting:
%It is hard to prove Sound Validation directly,
%so we first define a stronger property,
%called Stronger Sound Validation and
%we show that it is preserved during reduction by mean of conventional Progress and Subject Reduction.
%That is,
%Progress+Subject Reduction $\Rightarrow$ Stronger Sound Validation
%and Stronger Sound Validation $\Rightarrow$ Sound Validation.


\subheading{Stronger Soundness $\Rightarrow$ Soundness}
\thm{Stronger Soundness} depends on $\mathit{monitored}$ and \emph{OK}. 

An object is \emph{monitored} if execution
is currently inside of a monitor for that object, and
the monitored expression $\e_1$ does not
contain $l$ as a \emph{proper} sub-expression:

\indent $\mathit{monitored}(\e,l)$ iff
$\e=\ctx_v[\M{l}{\e_1}{\e_2}]$ and either $\e_1=l$, or $l$ is not inside $\e_1$.%\loseSpace

\noindent A monitored object is associated with an expression that can not observe it, but may
reference its internal representation directly.
In this way, we can safely modify its representation before checking its invariant.

The idea is that at the start the object will be valid and $\e_1$ will reference $l$;
but during reduction, $l$ will be used to
modify the object; only after that moment, the object may become invalid.


\noindent\textbf{Define} $\mathit{OK}(\sigma,e)$:\\
\indent $\forall l\in\dom(\sigma)$
  either $\mathit{valid}(\sigma,l)$ or $\mathit{monitored}(\e,l)$\\

\noindent Finally, the system is in an \emph{OK} state
if all objects in memory, are either
%the class of the object has no invariant method;
not (transitively) reachable from the expression (thus can be garbage collected),
valid and encapsulated,
or currently monitored.

\begin{theorem}[Stronger Soundness]\rm
if $c:\Kw{Cap};\emptyset\vdash \e_0: \T_0$ and
$c\mapsto\Kw{Cap}\{\_\}|\e_0\rightarrow^+ \sigma|\e$, then
$\emph{OK}(\sigma,\e)$.
\end{theorem}
\noindent Starting from only the capability object,
any well typed expression $\e_0$ can be reduced with an arbitrary number of steps,
and \emph{OK} will always hold.

\begin{theorem}\rm \thm{Stronger Soundness} $\Rightarrow$ \thm{Soundness}
\end{theorem}
\begin{proof}
\noindent By \thm{Stronger Soundness}, each $l$ in the current redex must be \emph{OK} \IOComm{This assumes that each $l$ in $e$ must be in dom of sigma! We should make this as an assumption!}:
\IOComm{Expand out the theorem we are proving, such as with a suppose e -> E[rl]...}
\begin{enumerate}
	\item If $\mathit{valid}(\sigma,l)$, then $l$ is valid, so thanks to \thm{Determinism}
	no invalid object could be observed.
	\item Otherwise, if $\mathit{monitored}(\e,l)$ then either:
	\begin{itemize}
	 \item we are executing inside of $\e_1$, thus the current redex is inside of a \IO{sub-}expression of the monitor that does not contain $l$, a contradiction.
	 \item or we are executing inside $\e_2$:
	 by our reduction rules, all monitor expressions start with
	 $\e_2=l.\invariant$, thus the first execution step
	 of $\e_2$ is \emph{trusted}. Further execution steps are also \emph{trusted}, since by well-formedness the body of invariant methods only use \Q@this@ (now replaced with $l$) to read fields.
	\end{itemize}
\end{enumerate}
In any of the possible cases above, \thm{Soundness} holds for $l$, and so it holds for all redexes.
\end{proof}

\subheading{Capsule Field Soundness}
% ....
Some auxilary properties:\\

%Capsule-Non-Circular(s, l, f): l not in rog(s, s[l.f])
\newcommand{\qindent}{\\\indent$\quad$}
\begin{itemize}
\item $\CNC(\s, l)$ iff $\forall f$,
\qindent if  $\S(l).f = \Kw{capsule}\,\_$ then $l \notin \rog(\s, \s[l.f])$.
\item $\CNE(\s, e, l)$ iff $\forall l' \in e,\forall f$:
\qindent  if $\S(l).f = \Kw{capsule}\,\_$ and $\muty(\s, e, l')$, then not $\tmuty(\s\setminus l, e, l', \s[l.f])$.
\item $\CNA(\s, e, l)$ iff $\forall \E$:
\qindent if $e = \E[\M{l}{e'}{\_}]$, then $e' = l$ and
\qindent if $e = \E[l.f]$ and $\S(l).f = \Kw{capsule}\,\_$, then not $\muty(s, E)$.

\item $\CNO(\s, e, l)$ iff $e = \EV[\M{l}{e'}{\_}]$, and either:
\qindent $e' = \E[l.f]$ and not $\reach(\s, \E, l)$ or \IOComm{$f$ will be a capsule field.}
\qindent not $\reach(\s, e', l)$.
%Capsule-Not-Observed(s, e, l): e = EV[M(l; e'; _)], and either:
%		e' = E[l.f] and not reacheable(s, E, l) or,
%		not reacheable(s, e', l)

\end{itemize}

\begin{theorem}[Capsule Field Soundnes]\rm
%	forall VS(s, e), forall l in dom(s), CNC(S, l, f) and either
$\gap{\forall \s|e},\forall l \in \mathit{dom}(\s)$:\\
\indent $\CNC(\s, l, f)$ and either:
\qindent $\CNE(\s, e, l)$ and $\CNA(\s, e, l)$, or
\qindent $\CNO(\s, e, l)$.
\end{theorem}
\begin{proof}
%if $c:\Kw{Cap};\emptyset\vdash \e_0: \T_0$ and
%$c\mapsto\Kw{Cap}\{\_\}|\e_0\rightarrow^+ \sigma|\e$, then
%$\emph{OK}(\sigma,\e)$.
\gap{This trivially holds in the base case when $\s = c\mapsto\Kw{Cap}\{\_\}$, since \Q!Cap! has no capsule fields.

Now suppose it holds for $s|\EV[e]$ and $\s|\EV[e] \rightarrow \s'|\EV[e']$. We will prove the theorem for $s'|\EV[e']$ based on cases for the non-\textsc{ctxv} reduction rule applied:}
\clearpage

\begin{enumerate}

\item (\textsc{mcall}) $\sigma|\EV[l\singleDot\m\oR v_1,\ldots,v_n\cR]\rightarrow \sigma|\EV[\e[\Kw{this}\coloneqq l,\x_1\coloneqq v_1,\ldots,x_n\coloneqq v_n]]$:
\begin{itemize}
\item If $m$ is not a capsule mutator, by \gap{our well-formedness rules for method bodies}, $e$ doesn't contain a monitor:
\begin{itemize}
		\item Thus $e$ wont contain $\Kw{this}.f$ as \muty, where $f \in \cf(\s, l)$; since fields are instance private, we can't have introduced any other field accesses, so we can't have broken \CNA, for any $l$.
		\item We can't have broken \CNC, \CNE or \CNO for any $l$, since we haven't changed the $l$s in the main expression, modified memory, or removed a monitor.
\end{itemize}
\item Otherwise, $e = \M{l}{e'}{l.\invariant}$. By our rules for capsule mutators, $m$ can only have \Q!imm! and \Q!capsule! parameters. Since capsule mutators are \Q!mut! methods, $l$ must have been \muty, so by \ASS{imm not mut} and \ASS{capsule}, $l$ cant be reachable from any $x_i$. By \gap{our well-formedness rules for method bodies}, $e'$ can't contain any $l$s, and since $m$ uses \Q!this! only once, to access a capsule field, $e' = \E[l.f]$, for some $f \in \cf{\s, l}$. Since $l$ is not \reach from any $x_i$, and $l \notin \E$, $l$ is not \reach from any $l' \in \E$, thus we have that \CNO holds.
\item Note that no other reduction rule can introduce an $l.f$ or an $\M{l}{e}{l.\invariant}$ with $e \neq l$; so they trivially preserve \CNA and ensure it for new $l$s.
\end{itemize}

\item (\textsc{Monitor Exit}) $\sigma|\EV[\M{l}{v}{\Kw{true}}]\rightarrow \sigma|\EV[v]$:
\begin{itemize}
\item This reduction doesn't modify memory, so we can't have violated \CNC for any $l'$.
\item Consider any $l' \neq l$:
\begin{itemize}
\item We can't have broken \CNE, since we haven't changed the $l$s in the main expression, nor modified memory.
\item Similarly, we can't have affected \CNO, since this reduction step doesn't introduce any $l$s, and the monitor we are removing is for an $l \neq l'$.
\end{itemize}
\item If this monitor was not introduced by \textsc{mcall}, $v = l$:
\begin{itemize}
\item If \CNE held for $l$, by the above proof for the case for $l' \neq l$, we can't have violated it.
\item \CNO can't have held for $l$ since $l \in v$, and $v$ was not the receiver of a field access.
\end{itemize}

\item Otherwise, this monitor was introduce by \textsc{mcall} due to a call to a capsule mutator on $l$, consider the state $\s_0|\EV[e_0]$ immediately before that \textsc{mcall}:
\begin{itemize}
\item We must not have had that \CNO l, since $e_0$ would contain $l$ as the receiver of a method call. Thus $\CNE(s_0, \EV[e_0], l)$ and $\CNA(s_0, \EV[e_0], l)$. 
\item Because \CNA held in $s_0|\EV[e_0]$, and we aren't introducing a field access or monitor, it still holds.
\item Since a capsule mutator cannot have any \Q!mut! parameters, by \ASS{!tmutatable -!> mutatable} and \ASS{mut-r update}, the body of the method can't have modified $\s_0$, thus $\s = \s_0, \_$. Since no pre-existing memory has changed since the \textsc{mcall}, and a capsule mutator cannot have a \Q!mut! return type, we can't have $\muty(\s, \EV, v)$, so the \textsc{monitor exit} reduction have restored $\CNE(\s_0, \EV[e_0], l)$.
\end{itemize}
\end{itemize}

\item (\textsc{new}) $\sigma|\EV[\Kw{new}\ C\oR v_1,\ldots,v_n\cR]\rightarrow \sigma,l\mapsto C\{v_1,\ldots,v_n\}| \EV[\M{l}{l}{l\singleDot\invariant}]$:
\begin{itemize}
	\item Consider any pre-existing $l'$:
	\begin{itemize}
		\item This reduction rule only adds to \s, so it couldn't have modified the \rog of $l'$ and broken \CNC for $l'$.
		\item Suppose we broke \CNE for $l'$ by making a $f' \in \cf(s, l)$ \tmuty. Since the \rog of $l'$ can't have been modified, nor could the \rog of any other pre-existing $l"$ have be modified we must have that $\s[l'.f]$ is \tmuty through some $v_i$. This requires that $v_i$ be the initiliser for a \Q!mut! or \Q!capsule! field, which by \ASS{mut init} and \ASS{capsule init}, means that $v_i$ must also be \muty. But then the $\s[l'.f]$ was \tmuty through $l'$, so $l'$ can't have already been \CNE.
		\item This reduction does not remove monitors nor introduce $l'$ into the main expression, thus it could not have violated \CNO.
	\end{itemize}
	\item Now consider each $i$ with $C.i = \Kw{capsule}\,\_\,f$:
	\begin{itemize}
	\item Since the pre-existing $\s$ was not modified, by \VS, $l \notin \rog(\s, v_i) = \rog(\s', \s'[l.f])$; thus \CNC holds.
	\item By \ASS{capsule init}, $\cap(\s, v_i, \EV[\Kw{new}\ C\oR v_1,\ldots,v_{i-1},\square,v_{i+1},\ldots,v_n\cR])$ and $\rog(s, v_i)$ is not \tmuty from \EV, and so not $\tmuty(\s'\setminus l, \EV[\M{l}{l}{l.\invariant}], v_i)$, thus \CNE holds.
	\end{itemize}
\end{itemize}


\item (\textsc{update}) $\sigma|\EV[l.f\equals{}v]\rightarrow \sigma[l.f=v]|\EV[\M{l}{l}{l.\invariant}]$:
\begin{itemize}
\item Consider any $l' \neq l$:
\begin{itemize}
		\item If $l'$ was \CNE, by \ASS{mut-r update}, $l$ is \muty. Consider any $f' \in \cf(\s, l')$, by \CNE, the \rog of $l'.f'$ is not \tmuty and $l$ not in $\rog(\s, \s[l'.f'])$, so we can't have modified $\rog(\s, \s[l'.f'])$ hence $l$ is still \CNC. By \ASS{!tmutatable -!> mutatable}, we can't have broken \CNE either.
		\item Otherwise, $l'$ was \CNO. Since this reduction dosen't remove monitors, or introduce any new $l$s into the main expression, we can't have violated \CNO.
\end{itemize}
\item $l$ cant have been $\CNO$, since we had $l.f = v$ inside our evaluation context, and $l.f = v$ is not a field access.
\item If $f \in \cf(\s, l)$:
\begin{itemize} 
\item By \ASS{capsule update}, $\cap(\s, v, \EV[l.f = \square])$ and $l \notin \nirog(v)$, by \ASS{mut-r update} and \ASS{imm not mut}, $l$ can't be \reach from an \Q!imm! field in the \rog of $v$, so $l\notin \rog(s, v)$, hence \CNC holds.
\item By $\cap(\s, v, \EV[l.f = \square])$, $v$ is not \tmuty from \EV, and so not $\tmuty(s\setminus l, \EV[\M{l}{l}{l.\invariant}], v)$, thus \CNE still holds.
\end{itemize}
\item By the proof above for the case $l' \neq l$, we can't have broken \CNE or \CNC for $f \notin \cf(\s, l)$, nor any other field of $l$.
\end{itemize}


\item (\textsc{access}) $\sigma|\EV[l.f]\rightarrow \sigma|\EV[\s[l.f]]$:
\begin{itemize}
	\item This reduction doesn't modify memory, so we can't have violated \CNC for any $l'$ and $f'$.
	\item Consider any $l'$:
	\begin{itemize}
		\item If $l'$ was \CNE and \CNA, then by \ASS{!tmutatable -!> mutatable}, for any $f \in \cf(\s, l')$, we can't have made $l.f'$ \tmuty (even if $l.f$ = $l'.f'$).
		\item Suppose $l'$ was \CNO, with $l' \neq l$, we must have $l' \notin \rog(\s, l)$. Since this reduction doesn't remove a monitor, and $\s[l.f] \in \rog(\s, l)$, we $l' \notin \rog(\s, \s[l.f])$, thus we didn't break \CNO.
		\item Suppose $l'$ = $l$, and $l'$ was \CNO. Then by \CNC, $l \notin \rog(\s, \s[l.f])$, for any $f \in \cf(\s, l)$ of $l$; thus since we haven't made $l'$ \reach from the main expression, nor removed a monitor, $l'$ is still \CNO.
	\end{itemize}
\end{itemize}
\item (\textsc{try enter}, \textsc{try ok}, and \textsc{try error}):
these rules don't change the $l$s in the main expression, modify memory, or remove monitors, thus they can't break \CNC, \CNE, or \CNO for any $l$.
\end{enumerate}
\end{proof}
\vfill
\clearpage
\subheading{Subject Reduction}
\begin{theorem}[Subject Reduction]\rm
if $\Sigma^{\sigma_0};\emptyset\vdash e_0: T_0$,
$\sigma_0|e_0\rightarrow \sigma_1|e_1$,
$\mathit{OK}(\sigma_0,\e_0)$
then
$\Sigma^{\sigma_1};\emptyset\vdash e_1: T_1$,
$\mathit{OK}(\sigma_1,e_1)$
\end{theorem}

\begin{proof}
By \thm{Subject Reduction Base}, $\Sigma^{\sigma_1};\emptyset\vdash e_1: T_1$. Any reduction step can be obtained by exactly one application of the \textsc{ctxv} rule and one other rule. So we just need to proceed by cases on the other reduction rule applied, and verify that each $l \in \dom(\s)$ is \valid or \mony.
\begin{enumerate}
\item (\textsc{update}) $\sigma|\ctx_v[l.f\equals v]\rightarrow \sigma'|\ctx_v[\e']$:
	\begin{itemize}
	  \item By \textsc{update}, $\e'=\M{l}{l}{l.\invariant}$, thus $\mathit{monitored}(\EV[e'],l)$.
	  \item Consider any other $l'$ such that $l\in \mathit{rog}(\sigma,l')$:
	  \begin{itemize}
	  	\item If it was $\mathit{monitored}$, it still is.
	    \item Otherwise it was $\mathit{valid}$:
			\begin{itemize}
				\item Suppose we made $l'$ not valid. By our well-formedness criteria, \Q@invariant@ can only accesses \Q@imm@ and \Q@capsule@ fields, thus by \ASS{imm not mut} and \ASS{mut-r update}, we must have that $l$ was in the \rog of a capsule field $f$ of $l'$; $l'$ can't have been \CNE, since $l$ is \muty. Thus, by \CFS, $l'$ was \CNO, and $\EV = \EV'[\M{l}{\EV''}{\_}]$:
				\begin{itemize}
					\item If $\EV''[l.f \equals v] = \E[l'.f]$ then by \CNO, $l'$ is not reachable from $\E$. The monitor must have been introduced by an \textsc{mcall} on a capsule mutator for $l'$. Since a capsule mutator can take only \Q!imm! and \Q!capsule! parameters, $l$ cannot be reachable from them (since $l$ was in the \rog of $l'$, and $l$ is \muty). Thus the only way to access $l$ is by accessing $l'.f$.
					Since capsule mutators can access \Q!this! only once, and \gap{by the proof of \CFS}, there is no other $l'.f$ in $\E[l'.f]$, nor was there one in a previous stage of reduction; this contradicts the fact that we just updated $l$ (which is only reachable through $l'.f$).
					\item Thus, by \CNO, we must have $\EV''[l.f \equals v] = e$ with $l'$ not reachable from $e$; so $l'$ was, and still is, \mony.
				\end{itemize}
				\item Otherwise, $l'$ is still \emph{valid}.
		  	\end{itemize}
	  \end{itemize}
	  \item $l$ is not reachable from any other $l'$, thus $l'$ being \valid or \mony could not have been effected by this reduction step.
	\end{itemize}

\item (\textsc{access}) $\sigma|\EV[l\singleDot f] \rightarrow \sigma|\EV[v]$:
	\begin{itemize}
		\item Consider any \mony $l'$, then $\EV = \EV'[\M{l'}{\EV''}{l.\invariant}]$:\\
			The monitor can't have been introduced by \textsc{new} or \textsc{update}, since they wrap monitors around $v$s which cannot be reduced; thus the monitor was introduced by \textsc{mcall}.
			Since the monitor for $l'$ contains a $l.f$, it cant have been \CNA, thus by \CFS, we must have that $l'$ is \CNO. 
			Thus $\s[l'.f] \neq l$, and so $l'$ is still \mony.
		\item Otherwise $l'$ was \valid, since this rule doesn't mutate memory, by \thm{Determinism}, it is still \valid.
\end{itemize}

\item (\textsc{mcall}, \textsc{try enter}, \textsc{try ok}, and \textsc{try error}):

	These reduction steps do not modify memory, the memory locations reachable inside of main expression, or any pre-existing monitor expressions. Therefore it cannot have any effect on the \emph{valid} (due to \thm{Determinism}), or $\mathit{monitored}$ properties of any memory locations.

\item (\textsc{new}) $\sigma|\EV[\Kw{new}\ C\oR\vs\cR]\rightarrow \sigma,l\mapsto C\{\vs\}|\EV[ \M{l}{l}{l.\invariant}]$:

	Clearly the newly created object, $l$, is \emph{monitored}. As for \textsc{mcall}, other objects and properties are not disturbed, and so they are still \valid or \mony.


\item (\textsc{monitor exit}) $\sigma|\M{l}{v}{\Kw{true}}\rightarrow \sigma|v$:
\begin{itemize}
	\item By \ASS{no monitors}, the monitor expression must have been introduced by \textsc{update}, \textsc{mcall}, or \textsc{new}. In each case the 3\textsuperscript{rd} expression started of as $l.\invariant$, and it has now (eventually) been reduced to $\Kw{true}$, thus by \thm{Determinism} $l$ is \emph{valid}.
\end{itemize} 
\end{enumerate}
\end{proof}


\begin{theorem}\rm
	\thm{Progress} + \thm{Subject Reduction} $\Rightarrow$ \thm{Stronger Soundness}
\end{theorem}
\begin{proof}
This proof proceeds by induction in the usual manner.

\emph{Base case}: At the start of execution, memory only contains $c$: since $c$ is defined to always be $\mathit{valid}$, and has only \Q@mut@ fields, we trivially have, thus $\mathit{OK}(c\mapsto\Kw{Cap},e)$.

\emph{Induction}: By \thm{Progress}, we always have another evaluation step to take, by \thm{Subject Reduction} such a step will preserve $\mathit{OK}$, and so by induction, $\mathit{OK}$ holds after any number of steps.

Note how for the proof garbage collections is important:
when the \Q@invariant()@ method evaluates to \Q@false@,
execution will only be continued by the \textsc{try error} rule, which by \thm{Unmonitored Try}, garbage collects the offending object: it will no longer be in \s.

\end{proof}

\bigskip
\begin{verbatim}
capsule(s, l, Ctx):
    for all l' in Ctx, rog(s, l') disjoint nirog(s, l)
    where l' in nirog(s, l) iff l' = l or l' in nirog(s, s[l.f]) and S(l).f = imm _

nirog
f in cfields(s, l) iff
    S(l).f = capsule _
	
tmutatable(s, e, l, l'):
    l in rog(s, l') or (tmutatable(s, e, s[l.f], l') for some S(l).f in {capsule _, mut _})

mutatable(s, e, l)
mutatable(s, E)

reacheable(s, E, l) exists l' in E, l in rog(s, l')
similarly with reacheable(s, e, l)
	
Valid-State(s, e):
    exists e0, with M(_; _; _) not in e0, such that:
        c -> Cap |- e0 : T
        and c->Cap{_} | e0 ->+ s|e

C.i (it is usefull! and easy to explain)
\end{verbatim}

%	\indent(1)\  ... \\*

%\Mut Field:
%	if S; G; E |- e.f = e' : T, then S; G; E[[].f = e'] |- e : mut _
%	if S; G; E |- e.f : mut _, then S; G; E[[].f] |- e : mut _

% This is to have an enumerate that is indented!
\newenvironment{ienumerate}{${}$% allows a line break
	\begin{itemize}\item[] % indents like a normal itemize, but with no bullet!
	\begin{enumerate}
}{\end{enumerate}\end{itemize}}

\begin{Assumption}[Type Consistency]\rm
\begin{ienumerate}
%if S; G; E |- e.f = e' : _ C, and C.f = T', then S; G; E[e.f = []] |- e' : T'
\item if \ty{e.f \equals e'}{\_\,C} and $C.f = T'$, then \ty[\e.f \equals \h]{e'}{T'}
\item if \ty{e.m\oR e_1,\ldots,e_n\cR}{T}, \\
\ty[\h.m\oR e_1,\ldots,e_n\cR]{e}{\_\,C}, and
$C.m = \mdf\,\Kw{method}\,\T'\,\oR\T_1\,\x_1\ldots\T_n\x_n\cR\,\_$, then:
\begin{enumerate}
\item \ty{e.m\oR e_1,\ldots,e_n\cR}{T'},
\item \ty[\h.m\oR e_1,\ldots,e_n\cR]{e}{\mdf\,\_}, and
\item \ty[e.m\oR\e_1,\ldots,e_{i-1},\h,e_{i+1},\ldots,e_n\cR]{e_i}{T_i}
%\item hello
\end{enumerate}
\item if \ty{\Kw{new}\ C\oR e_1,\ldots,e_n\cR}{T} and $C.i = T_i\,\_$, then:
\qindent \ty[\Kw{new}\ C\oR \e_1,\ldots,e_{i-1},\h,e_{i+1},\ldots,e_n\cR]{e_i}{T_i}
\end{ienumerate}
\end{Assumption}%

\begin{Assumption}[Mut Field]\rm
\begin{ienumerate}
\item if \ty{e.f \equals e'}{T}, then \ty[\h.f \equals e']{e}{\Kw{mut}\,\_}
\item if \ty{e.f}{\Kw{mut}\,\_}, then \ty[\h.f]{e}{\Kw{mut}\,\_}
\end{ienumerate}
\end{Assumption}%
