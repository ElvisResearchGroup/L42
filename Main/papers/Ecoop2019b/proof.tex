\saveSpace
\section{Proof and Axioms}
\label{s:proof}
\saveSpace

\subheading{Axiomatic Type Properties}
As previously discussed, instead of providing a concrete set of type rules, we provide a set of properties
that the type system needs to respect.
To express these properties, we first need some auxiliary definitions.

%\noindent\textbf{Define}
%$\mathit{encapsulatedObj}(C)$:\\*
%${}_{}$\quad\quad \Q@class @$C$\,\Q@implements @$\Many{C}$\Q@{@$\,\Many{F}\,\Many{M}$\Q@}@
% and $\forall \mdf\,C\,\f \in \Many{F},\ \mdf \in \{\Kw{imm},\Kw{capsule}\}$\\*
%\noindent As we discussed, only encapsulated objects can support invariants;
%their class declarations only have immutable or capsule fields. Note how here we see immutable
%and simple objects as special cases of encapsulated ones.

The encapsulated ROG of $l_0$ is composed of all the objects
in the ROG of its immutable and capsule fields:\\*
\indent $l \in \mathit{erog}(\sigma,l_0)
\text{ \IO{iff} } \Sigma^\sigma(l_0).f \in \{\Kw{imm}\,\_,\Kw{capsule}\,\_\}
\text{ and } l \in \mathit{rog}(\sigma,\sigma(l_0).f)
$\IOComm{put a $\exists f$?} \loseSpace

\noindent An object is \emph{mutatable} in a $\sigma$ and  $\e$ if there is an occurrence of 
$l$ in $e$, that when seen as \Q@imm@ makes the expression ill typed:\\*
\indent $\mathit{mutatable}(l,\sigma,\e)$ iff for some $T=\Kw{imm}\,\Sigma^\sigma(l)$ and $\ctx[l]=\e$,\\*
\indent \indent $\Sigma^\sigma;\x:T\vdash\ctx[\x]:T'$ does not hold for any $T'$.\loseSpace

%if $\ \sigma_0|e_0\rightarrow \sigma|e$ then $\sigma_1|\e_1=\sigma|\e$
% $\exists! \sigma_1|\e_1$ such that $\sigma_0|\e_0\rightarrow \sigma_1|\e_1$\\*

%We can now assume the following properties over the type system:

\IOComm{Add an explanation for the following two assumptions?}
\begin{Assumption}[Progress]\rm
	if $\Sigma^{\sigma_0};\emptyset\vdash e_0: T_0$,
	and $e_0$ is not \IO{form $l$} \IODel{a value} or $\mathit{error}$, then
	$\sigma_0|e_0\rightarrow \sigma_1|e_1$.
\end{Assumption}

\begin{Assumption}[Subject Reduction Base]\rm
	if $\Sigma^{\sigma_0};\emptyset\vdash e_0: T_0$,
	$\sigma_0|e_0\rightarrow \sigma_1|e_1$,
	then
	$\Sigma^{\sigma_1};\emptyset\vdash e_1: T_1$.
\end{Assumption}


\noindent If the result of a field access is \Q!mut!\IODel{able},
the receiver is also mut\Q!mut!\IODel{able}\IO{ field updates are only allowed on \Q!mut! receivers}:\saveSpace\saveSpace
\begin{Assumption}[Mut Field]\rm
	\ \\
	\indent(1)\ if $\Sigma;\Gamma\vdash\e\singleDot\f:\Kw{mut}\,\_$
	then $\Sigma;\Gamma\vdash\e:\Kw{mut}\,\_$
	and 
	\\*\indent(2)
	if $\Sigma;\Gamma\vdash\e_0\singleDot\f\equals\e_1:T$
	then $\Sigma;\Gamma\vdash\e_0:\Kw{mut}\,\_$.
\end{Assumption}

\noindent An object is not part of the ROG of its immutable or capsule fields\IO{\footnote{{This is not strictly true in L42, as 42 allows circular objects with \Q!fwd imm! fields, however such objects cannot have an invariant.}}}:\saveSpace\saveSpace
\begin{Assumption}[Head Not Circular]\rm
	if
	$\Sigma^\sigma;\Gamma\vdash l:T$,
	then $l\notin\text{erog}(\sigma,l)$.
\end{Assumption}


\noindent In a well typed $\sigma$ and $e$, if mutatable $l_2$ is reachable \IODel{from} \IO{through the \emph{erog} of
$l_1$}, and $l_1$ is reachable \IODel{from} \IO{through the \IO{erog} of} $l_0$,
then all the paths connecting $l_0$ and $l_2$ pass trough $l_1$; thus
if we were to remove $l_1$ from the object graph, $l_0$ would no longer reach $l_2$:
\saveSpace\saveSpace
\begin{Assumption}[Capsule Tree]\rm
	If   $\Sigma^\sigma;\Gamma\vdash \e:\T$,
	$l_2\in\text{erog}(\sigma,l_1)$,
	$l_1\in\text{erog}(\sigma,l_0)$,\\*
	and
	$\mathit{mutatable}(l_2,\sigma,\e)$
	then 
	$l_2\notin\text{erog}(\sigma\setminus l_1,l_0)$.
\end{Assumption}


Capsule Tree and Head Not Circular together 
imply that capsule fields section the object graph into a tree of nested `balloons',
where nodes are mutable encapsulated objects and
edges are given by reachability between those objects in the original memory: \IO{if}
$l_2$ is in the encapsulated ROG of $l_1$\IO{, and} \IODel{;}
$l_2$ is mutatable and reachable through $l_1$, \IODel{thus
it} \IO{then $l_2$} must be reachable by a \Q@capsule@ field.
Thanks to Head Not Circular and $l_1\in\text{erog}(\sigma,l_0)$ we can derive \IO{that}
$l_0\notin\text{erog}(\sigma,l_1)$.

The execution of an expression
with no \Q@mut@ free variables is deterministic and does not
mutate pre existing memory (and thus does not not perform I/O by mutating \IO{the} pre existing $c$):
\begin{Assumption}[Determinism]\rm
	if $\emptyset;\Gamma\vdash \e:\T$, 
	$\forall x \Gamma(x)\neq\Kw{mut}\,\_$, and
	$\sigma | \e'\rightarrow^+ \sigma' | \e''$
	then 
	$\sigma | \e'\Rightarrow^+ \sigma,\_ | \e''$,
	where $\e'=\e[x_1=l_1,\ldots,x_n=l_n]$ and $\Sigma^\sigma;\emptyset\vdash \e':\T$
\end{Assumption}

\IO{\noindent For each \Q@try@\IO{--}\Q@catch@, execution preserves the memory needed to continue the execution in case of \IO{an} error
	(the memory visible outside of the \Q@try@).}
\begin{Assumption}[StrongExceptionSafety]\rm
	if $\Sigma^{\sigma,\sigma'};\emptyset\vdash \ctx[\Kw{try}^\sigma\oC\e_0\cC\ \Kw{catch}\ \oC\e_1\cC]:\T$
	and\\*
	$
	\sigma,\sigma'|\ctx[\Kw{try}^\sigma\oC\e_0\cC\ \Kw{catch}\ \oC\e_1\cC]\rightarrow 
	\sigma''|\ctx[\Kw{try}^\sigma\oC\e'\cC\ \Kw{catch}\ \oC\e_1\cC]
	$
	then 
	$\sigma''=\sigma,\_$
	and
	$\Sigma^\sigma;\emptyset\vdash \ctx[\e_1]:\T$
\end{Assumption}
\IODel{\noindent For each \Q@try-catch@, execution preserves the memory needed to continue the execution in case of error
(the memory visible outside of the \Q@try@).}%

%Thanks to how our reduction rules are designed, especially \textsc{try error},
%@Progress will need to rely on @StrongExceptionSafety internally.

Note that our last well formedness rule requires 
\textsc{update} and \textsc{mcall} to introduce
monitor expressions only over locations
that are not preserved by \Q@try@ blocks.
This can be achieved, since monitors are introduced
around mutation operations
(and \Q@new@ expression),
and Strong Exception Safety ensures no mutation happens on preserved memory.

% To the best of our knowledge, only the type system of 42~\cite{ServettoEtAl13a,ServettoZucca15}
%  supports all these assumptions out of the box,
% while both Gordon~\cite{GordonEtAl12} and Pony~\cite{clebsch2015deny,clebsch2017orca} supports all except StrongExceptionSafety,
% however it should be trivial to modify them to support it:
% the \Q@try-catch@ rule could be modified to
% $\emptyset;\Gamma\vdash\Kw{try}\ \oC\e_0\cC\ \Kw{catch}\ \oC\e_1\cC:\T$
% if\\* $\emptyset;
% \Gamma,\{x:\Kw{read}\,C | x:\Kw{mut}\,C\,\in\Gamma\}
% \vdash\e_0:\T$ and $\emptyset;\Gamma\vdash\e_1:\T$,
% i.e. $e_0$ can be typed when seeing all externally defined mutable references as \Q@read@.



\subheading{Proof of Theorem 1}
It is hard to prove Soundness directly,
so we first define a stronger property,
called \emph{Stronger Soundness} and
show that it is preserved during reductions by means of conventional
Progress and Subject Reduction (Progress is one of our assumptions,
while Subject Reduction relies heavily upon Subject Reduction Base).
That is,
Progress $\wedge$ Subject Reduction $\Rightarrow$ Stronger \IODel{Sound Validation} \IO{Soudness} \IO{and}
Stronger Soundness $\Rightarrow$ Soundness.

%The structure of the proof is interesting:
%It is hard to prove Sound Validation directly,
%so we first define a stronger property,
%called Stronger Sound Validation and
%we show that it is preserved during reduction by mean of conventional Progress and Subject Reduction.
%That is,
%Progress+Subject Reduction $\Rightarrow$ Stronger Sound Validation
%and Stronger Sound Validation $\Rightarrow$ Sound Validation.


\subheading{Stronger Soundness $\Rightarrow$ Soundness}
\saveSpace
Stronger Soundness depends on
$\mathit{wellEncapsulated}$, $\mathit{monitored}$
and \emph{OK}:\\
\loseSpace%
\indent $\mathit{wellEncapsulated}(\sigma,\e,l_0)$ iff
$\forall l \in \mathit{erog}(\sigma,l_0), \text{not}\ \mathit{mutatable}(l,\sigma,\e)$.\loseSpace

\noindent The main idea is that an object is well encapsulated if its encapsulated state cannot be
modified by $e$.

An object is \emph{monitored} if execution
is currently inside of a monitor for that object, and
the monitored expression $\e_1$ does not
contain $l$ as a \emph{proper} subexpression:

\indent $\mathit{monitored}(\e,l)$ iff
$\e=\ctx_v[\M{l}{\e_1}{\e_2}]$ and either $\e_1=l$, or $l$ is not inside $\e_1$.\loseSpace

\noindent A monitored object is associated with an expression that can not observe it, but may
reference its internal representation directly.
In this way, we can safely modify its representation before checking its invariant.

The idea is that at the start the object will be valid and $\e_1$ will \IODel{contain} \IO{reference} $l$;
but during reduction, $l$ will be used to
modify the object; only after that moment, the object may become invalid.


\noindent\textbf{Define} $\mathit{OK}(\sigma,e)$:\\
\indent $\forall l\in\textit{dom}(\sigma)$
  either\\
\indent\indent 1. $\mathit{garbage}(l,\sigma,\e)$,\\
\indent\indent 2. $\mathit{valid}(\sigma,l)$ and $\mathit{wellEncapsulated}(\sigma,\e,l)$, or\\
\indent\indent 3. $\mathit{monitored}(\e,l)$.

Finally, the system is in \IODel{a valid} \IO{an \emph{OK}} state
if all objects in memory, are either
%the class of the object has no invariant method;
not (transitively) reachable from the expression (thus can be garbage collected)\IO{,}\IODel{;}
valid and encapsulated\IO{,}\IODel{;}
or currently monitored.

\begin{theorem}[Stronger Soundness]\rm
if $c:\Kw{Cap};\emptyset\vdash \e_0: \T_0$ and
$c\mapsto\Kw{Cap}\{\_\}|\e_0\rightarrow^+ \sigma|\e$, then
$\emph{OK}(\sigma,\e)$.
\end{theorem}
\noindent Starting from only the capability object,
any well typed expression $\e_0$ can be reduced in an arbitrary number of steps,
and \emph{OK} will always hold.

\begin{theorem}\rm Stronger Soundness $\Rightarrow$ Soundness
\end{theorem}
\begin{proof}
\noindent By Stronger Sound Validation, each $l$ in the current redex must be \emph{OK}:
\begin{enumerate}
	\item If $l$ is \emph{garbage}, it cannot be in the current redex, a contradiction.
	\item If $\mathit{valid}(\sigma,l)$, then $l$ is valid, so thanks to Determinism
	no invalid object could be observed.
	\item Otherwise, if $\mathit{monitored}(\e,l)$ then either:
	\begin{itemize}
	 \item we are executing inside of $\e_1$, thus the current redex is inside of a sub expression of the monitor that does not contain $l$, a contradiction.
	 \item or we are executing inside $\e_2$:
	 by our reduction rules, all monitor expressions start with
	 $\e_2=l.\invariant$, thus the first execution step
	 of $\e_2$ is trusted. Further execution steps are also trusted, since by well formedness the body of invariant methods only use \Q@this@ (now \IODel{translated to} \IO{replaced with} $l$) to read fields.
	\end{itemize}
\end{enumerate}
In any of the possible cases above, Soundness holds for $l$, and so it holds for all redexes.
\end{proof}

\subheading{Subject Reduction}

\noindent\textbf{Define} $\mathit{fieldGuarded}(\sigma,\e)$:\\*
\indent$\forall \ctx$ such that $\e=\ctx[l\singleDot\f] $
and $\Sigma^\sigma(l).f=\Kw{capsule}\,\_$, and $\f\mathrel{\mathit{inside}} \Sigma^\sigma(l).\invariant$\\*
\indent\indent either
$\forall T, \forall C, \Sigma^\sigma;\x:\Kw{mut}\,C\,\not\vdash\ctx[\x]:T$, or\\*
\indent\indent $\ctx=\ctx'[\M{l}{\ctx''}{\e}]$ and $l$ is contained exactly once in $\ctx''$.

That is, all \Q@mut@ capsule field accesses are individually guarded by monitors.
Note how we use $C$ in $\x:\Kw{mut}\,C$ to guess the type of the accessed field,
and that we use the full context $\ctx$, instead of the evaluation context $\ctx_v$,
to refer to field accesses everywhere in the expression $\e$.


\begin{theorem}[Subject Reduction]\rm
if $\Sigma^{\sigma_0};\emptyset\vdash e_0: T_0$,
$\sigma_0|e_0\rightarrow \sigma_1|e_1$,
$\mathit{OK}(\sigma_0,\e_0)$
and
$\mathit{fieldGuarded}(\sigma_0,\e_0)$
then
$\Sigma^{\sigma_1};\emptyset\vdash e_1: T_1$,
$\mathit{OK}(\sigma_1,e_1)$ and
$\mathit{fieldGuarded}(\sigma_1,\e_1)$
\end{theorem}

\begin{theorem}\rm
	Progress + Subject Reduction $\Rightarrow$ Stronger Soundness
\end{theorem}
\begin{proof}
This proof proceeds by induction in the usual manner.

\emph{Base case}: At the start of execution, memory only contains $c$: since $c$ is defined to always be $\mathit{valid}$, and has only \Q@mut@ fields, \IODel{hence} it is trivially $\mathit{wellEncapsulated}$, thus $\mathit{OK}(c\mapsto\Kw{Cap},e)$.

\emph{Induction}: By Progress, we always have another evaluation step to take, by Subject Reduction such a step will preserve $\mathit{OK}$, and so by induction, $\mathit{OK}$ holds after any number of steps.

Note how for the proof garbage \IODel{collection} \IO{collectability} is important:
when the \Q@invariant()@ method evaluates to \Q@false@,
execution can continue only if the offending object is classified as \emph{garbage}.
\end{proof}

\subheading{Exposer Instrumentation}
We first introduce a lemma derived from our well formedness criteria and the type system:
\begin{Lemma}[Exposer Instrumentation]\rm
If $\sigma_0 | \e_0\rightarrow \sigma_1 |\e_1$ and
$\text{fieldGuarded}(\sigma_0,\e_0)$
\\*
then $\text{fieldGuarded}(\sigma_1,\e_1)$.
\end{Lemma}
\begin{proof}
The only rule that can
introduce a new field access is \textsc{mcall}.
In that case, Exposer Instrumentation holds
by well formedness (all field accesses in methods are of the form \Q@this.f@)\IO{,}
\IODel{and} since \textsc{mcall} inserts a monitor while invoking capsule mutator methods, and not field accesses themselves. If however the method is not a \Q@mut@ method but still accesses a capsule field, by Mut Field such a field access expression cannot be typed as \Q@mut@ and so no monitor is needed.

Note that \textsc{monitor exit} is fine because monitors are removed only when
 $e_1$ is a value.
\end{proof}

\subheading{Proof of Subject Reduction}\saveSpace
Any reduction step can be obtained
by exactly one application of the \textsc{ctxv} rule and one other rule. Thus the proof can simply proceed by cases on the other applied rule.

By Subject Reduction Base and Exposer Instrumentation,
$\Sigma^{\sigma_1};\emptyset\vdash e_1: T_1$ and  $\mathit{fieldGuarded}(\sigma_1,\e_1)$. So we just need to proceed by cases on the reduction rule applied to verify that $\mathit{OK}(\sigma_1,\e_1)$ holds:


\begin{enumerate}
\item (\textsc{update}) $\sigma|l\singleDot f\equals v\rightarrow \sigma'|\e'$:
	\begin{itemize}
	  \item By \textsc{update} $\e'=\Kw{M}\oR l;l;l\singleDot\Kw{invariant}\oR\cR\cR$, thus $\mathit{monitored}(\e,l)$.
	  \item Every $l_1$ such that $l\in \mathit{rog}(\sigma,l_1)$ will verify the same case as the former step:
	  \begin{itemize}
	  	\item If it was $\mathit{garbage}$, clearly it still is.
	  	\item If it was $\mathit{monitored}$, it still is.
	    \item Otherwise it was $\mathit{valid}$ and $\mathit{wellEncapsulated}$:
			\begin{itemize}
				\item If $l\in \mathit{erog}(\sigma,l_1)$ we have a contradiction since $\mathit{mutatable}(l, \sigma, e)$, (by MutField)
		    	\item Otherwise, by our well formedess criteria that \Q@invariant@ only accesses \Q@imm@ and \Q@capsule@ fields, and by Determinism, it is clearly the case that $\mathit{valid}$ still holds;
				By HeadNotCircular it cannot be the case that \IO{$l\in \mathit{erog}(\sigma',l_1)$,} and so $l_1$ is still $\mathit{wellEncapsulated}$.
		  	\end{itemize}
	  \end{itemize}
	  \item Every other $l_0$ is not in the reachable object graph of $l$,
	  thus it being $\mathit{OK}$ could not have been effected by this reduction step.
	\end{itemize}

\item (\textsc{access}) $\sigma|l\singleDot f \rightarrow \sigma|v$:
	\begin{itemize}
		\item If $l$ was \emph{valid} and \emph{wellEncapsulated}:
		\begin{itemize}
			\item If we have now broken \emph{wellEncapsulated}, we must have made something in its \emph{erog} \emph{mutatable}. As we can only type \Q@capsule@ fields as \Q@mut@ and not \Q@imm@ fields, by Field Mut we must have that $f$ is \Q@capsule@ and $l\singleDot f$ is being typed as \Q@mut@. By $\mathit{fieldGuarded}(\sigma_0,\e_0)$\IO{, t}he former step must have been inside a monitor \M{l}{\ctx_v[l.f]}{\e}
		    and the $l$ under reduction was the only occurrence of $l$.
		    Since $f$ is a capsule, we know that $l\notin \text{erog}(\sigma,l)$
		    by Head Not Circular. Thus in our new step $l$ is not\emph{inside} $\ctx_v[v]$. Thus $l$ must be \emph{monitored} and hence it is $\mathit{OK}$.

		    \item Otherwise, $l$ is still $\mathit{OK}$
    	\end{itemize}

	\item Suppose some other $l_0$ was \emph{wellEncapsulated} and \emph{valid}:
	\begin{itemize}
			\item If $l$ was in the \emph{rog} of $l_0$, by Capsule Tree, if $l$ was in the \emph{rog} of $l$, then $v$ can only be reached from $l_0$ by passing through $l$, and so we could not have made $l_0$ non \emph{wellEncapsulated}. In addition, since only things in the \emph{erog} can be referenced by \Q@invariant@'s validity can not depend on $l$, and by Determinism it is still the case that $l_0$ is \emph{valid}. And so we can't have effected $l_0$ being $\mathit{OK}$.
			\item Otherwise, this reduction step could not have affected $l_0$, so $l_0$ is still $\mathit{OK}$.
	\end{itemize}


	\item Nothing that was $\mathit{garbage}$ could have been made reachable by this expression, since the only value we produced was $v$ and it was reachable through $l$ (and so could not have been $\mathit{garbage}$), thus $\mathit{garbage}$ is still $\mathit{OK}$.

	\item As we don't change any monitors here, nothing that was $\mathit{monitored}$ could have been made un-$\mathit{monitored}$, and so it is still $\mathit{OK}$.
\end{itemize}

\item (\textsc{mcall}, \textsc{try enter} and \textsc{try ok}):

	These reduction steps do not modify memory, nor do they modify the memory locations reachable inside of main expression, nor do they modify any monitor expressions. Therefore it cannot have any effect on the $\mathit{garbage}$, \emph{wellEncapsulated}, \emph{valid} (due to Determinism), or $\mathit{monitored}$ properties of any memory locations, thus $\mathit{OK}$ still holds.

\item (\textsc{new}) $\sigma|\Kw{new}\ C\oR\vs\cR\rightarrow \sigma,l\mapsto C\{\vs\}| \Kw{M}\oR l;l;l\singleDot\Kw{invariant}\oR\cR\cR$:

	Clearly the newly created object ($l$) is \emph{monitored}. As for \textsc{mcall}, other objects and properties are not disturbed, and so $\mathit{OK}$ still holds.


\item (\textsc{monitor exit}) $\sigma|\Kw{M}\oR l; v;\Kw{true}\cR\rightarrow \sigma|v$:
\begin{itemize}
	\item As monitor expressions are not present in the original source code, it must have been introduced by \textsc{update}, \textsc{mcall}, or \textsc{new}. In each case the 3\textsuperscript{rd} expression started of as $l\singleDot\Kw{invariant}\oR\cR$, and it has now (eventually) been reduced to $\Kw{true}$, thus by Determinism $l$ is \emph{valid}.

	\item  If the monitor was introduced by \textsc{update}, then $v = l$. We must have had that $l$ was well encapsulated before \textsc{update} was executed (since it can't have been \emph{garbage} and \emph{monitored}, as \textsc{update} itself preserves this property and we haven't modified memory in anyway, we must still have that $l$ is \emph{wellEncapsulated}. As $l$ is \emph{valid} and \emph{wellEncapsulated} it is $\mathit{OK}$.

	\item If the monitor was introduced by \textsc{mcall}, then it was due to calling a capsule mutator method that mutated a field $f$.
	\begin{itemize}
		\item A location that was \emph{garbage} obviously still is, and so is also $\mathit{OK}$.
		\item No location that was \emph{valid} could have been made invalid since this reduction rule performs no mutation of memory. If a location was \emph{wellEncapsulated} before, the only way it could be non \emph{wellEncapsulated} is if we somehow leaked a \Q@mut@ reference to something, but by our well-formedness rules, $v$ cannot be typed as \Q@mut@ and so we can't have affected \emph{wellEncapsulated}, hence such thing is still $\mathit{OK}$.
		\item The only location that could have been made un \emph{monitored} is $l$ itself. By our well formedness criteria, $l$ was only used to modify $l.f$, and we have no parameters by which we could have made $l.f$ non \emph{wellEncapsulated}, since that would violate Capsule Tree. As nothing else in $l$ was modified, and it must have been \emph{wellEncapsulated} before the \textsc{mcall}, it still is, and since  $l$ is valid, it is $\mathit{OK}$.
	\end{itemize}
	\item Otherwise the monitor was introduced by \textsc{new}. Since we require that \Q@capsule@ fields and \Q@imm@ fields are only initialised to \Q@capsule@ and \Q@imm@ expressions, by Capsule Tree the resulting value, $l$, must be \emph{wellEncapsulated}, since $l$ is also \emph{valid} we have that $l$ is $\mathit{OK}$.

\end{itemize}

\item (\textsc{try error}) $\sigma,\sigma_0|\Kw{try}^\sigma\oC \mathit{error}\cC\ \Kw{catch}\ \oC\e\cC\rightarrow \sigma|\e$:

	By Strong Exception Safety, we know that $\sigma_0$ is $\mathit{garbage}$ with respect to $\ctx_v[\e]$. By our well formedness criteria, no location inside $\sigma$ could have been \emph{monitored}.
	Since we don't modify memory, everything in $\sigma_0$ is $\mathit{garbage}$ and nothing inside $\sigma$ was previously monitored, it is still clearly the case that everything in $\sigma$ is $\mathit{OK}$. 
\end{enumerate}