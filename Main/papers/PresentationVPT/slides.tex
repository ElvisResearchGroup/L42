
\begin{frame}[fragile]
\frametitle{Pow}

\begin{lstlisting}
@requires(exp > 0)
@ensures(result = x**exp)
Int pow(Int x, Int exp) {
	if (exp == 1) return x;
	if (exp $\%$2 == 0) return pow(x*x, exp/2);//even
	return x*pow(x, exp-1); }  //odd
\end{lstlisting}
If the exponent is known at compile time,
unfolding the recursion produces even more efficient code:
\vspace{-1ex}
\begin{lstlisting}
@ensures(result = x**7) Int pow7(Int x) { 
  Int x2 = x*x; // x**2
  Int x4 = x2*x2; // x**4
  return x*x2*x4; } // Since 7 = 1 + 2 + 4
\end{lstlisting}

\end{frame}


\begin{frame}[fragile]
\frametitle{Pow}
\begin{itemize}
\item OO languages supporting static verification usually extends method
declaration with syntax supporting pre/post conditions. During
compilation, directly after typing, an automatic theorem prover
checks the constraints.
\item Very slow even on fast hardware!
\item \Q@pox(x,y)@ execution is slower than an hand written \Q@pow7(x)@
\item Manually declaring \Q@pow@$_n$ methods: repetitive, boring.
\item Running the static verifier on all \Q@pow@$_n$ versions: time consuming
\end{itemize}
\end{frame}


\begin{frame}[fragile]
\frametitle{Pow}

What if we could programmatically generate code at compile time?
For example, we could write \Q@generate(y)@ as a method
generating a class body with a method \Q@pow(x)@ computing
\Q@pow(x,y)@

\Q@class Pow7: generate(7)@

\Q@...@

\Q@Pow7.pow(4) == pow7(4)@

\end{frame}


\begin{frame}[fragile]
\frametitle{Iteratively building up \Q@pow@}
\vspace{-2ex}
\begin{itemize}
\item Verification supports code reuse / class inheritance.
\item What if metaprogramming was based on code reuse?
\item For example, here we use inheritance to encode statically verified
\Q@pow@s while reusing part of the already verified contract
\item \tiny{A code optimizer could inline nested calls, producing 
the same efficient code of the hand written versions of before}
\end{itemize}
\vspace{-1ex}
\begin{lstlisting}
class Pow1{
  @ensures(result=x**1)
  Int pow1(Int x){return x;}  }

class Pow2 extends Pow1{
  @ensures(result=x**2)
  Int pow2(Int x){return x*pow1(x);}  }

class Pow3 extends Pow2{
  @ensures(result=x**3)
  Int pow3(Int x){return x*pow2(x);}  }
\end{lstlisting}
\end{frame}



\begin{frame}[fragile]
\frametitle{Pow and Exp}
\vspace{-4ex}
\begin{lstlisting}
class Pow1{
  Int exp1(){return 1;}
  @ensures(result=x**exp1())
  Int pow1(Int x){return x;}  }

class Pow2 extends Pow1{
  Int exp2(){return exp1()+1;}
  @ensures(result=x**exp2())
  Int pow2(Int x){return x*pow1(x);}  }

class Pow3 extends Pow2{
  Int exp3(){return exp2()+1;}
  @ensures(result=x**exp3())
  Int pow3(Int x){return x*pow2(x);}  }
\end{lstlisting}
\end{frame}


\begin{frame}[fragile]
\frametitle{Traits}
\vspace{-3ex}
\begin{columns}
    \column{\dimexpr\paperwidth-10pt}
\begin{lstlisting}[basicstyle=\small]
//induction base case: pow(x)=x**1
Trait base=class {
  @ensures(result>0)@ensures(result=1)Int exp(){return 1;}  
  @ensures(result=x**exp())Int pow(Int x){return x;}
  }
//if _pow(x)=x**_exp(), pow(x)=x**(1+_exp())
Trait odd=class {
  @ensures(result>0)Int $\_$exp();
  @ensures(result=1+$\_$exp())Int exp(){return 1+$\_$exp();}
  @ensures(result=x**$\_$exp())Int $\_$pow(Int x);
  @ensures(result=x**exp())Int pow(Int x){return x*$\_$pow(x);}
}
//if _pow(x)=x**_exp(), pow(x)=x**(2*_exp())
Trait even=class {
  @ensures(result>0)Int $\_$exp();
  @ensures(result=2*$\_$exp())Int exp(){return 2*$\_$exp();}
  @ensures(result=x**$\_$exp())Int $\_$pow(Int x);
  @ensures(result=x**exp())Int pow(Int x){return $\_$pow(x*x);}
}
\end{lstlisting}
\end{columns}
\end{frame}

\begin{frame}[fragile]
\frametitle{Trait composition}
\vspace{-3ex}
\begin{columns}
    \column{\dimexpr\paperwidth-10pt}
\begin{lstlisting}[basicstyle=\small]
//`compose' performs a step of iterative composition
Trait compose(Trait current, Trait next){
  current = current[rename exp()->$\_$exp(), pow(x)->$\_$pow(x)];
  return (current+next)[hide $\_$exp(), $\_$pow(x)];
  }

@requires(exp>0)//the entry point for our metaprogramming
Trait generate(Int exp) {
  if (exp==1) return base;
  if (exp$\%$2==0) return compose(generate(exp/2),even);
  return compose(generate(exp-1),odd);
  }

class Pow7: generate(7)
//the body of class Pow7 is the result of generate(7)


/*example usage:*/new Pow7().pow(3)==2187//Compute 3**7
\end{lstlisting}
\end{columns}
\end{frame}

\begin{frame}[fragile]
\frametitle{Comparing code}
\vspace{-3ex}
\begin{lstlisting}[basicstyle=\small]
//conventional code
@requires(exp > 0)
@ensures(result = x**exp)
Int pow(Int x, Int exp) {
	if (exp == 1) return x;
	if (exp $\%$2 == 0) return pow(x*x, exp/2);//even
	return x*pow(x, exp-1); }  //odd

//metaprogramming
@requires(exp>0)
Trait generate(Int exp) {
  if (exp==1) return base;
  if (exp$\%$2==0) return compose(generate(exp/2),even);
  return compose(generate(exp-1),odd);
  }

class Pow7:{//generated code
@ensures(result>0) exp(){return 1+1+1+1+1+1+1;}
@ensures(result = x**exp()) Int pow(Int x) {??}
\end{lstlisting}
\end{frame}

\begin{frame}[fragile]
\frametitle{Concluding}
\begin{itemize}
\item Idea: merge conventional verification and trait composition.
\item Iterative composition (metaprogramming) allows to generate
code that is correct by construction.
\item Theorem prover/manual verification  only for the basic building blocks.
\item Contracts are syntactically matched during the sum operation.
\item What is the contract of the result?
\item We are searching for relevant related work!
\end{itemize}
\end{frame}


\begin{frame}[fragile]
\frametitle{Thanks}
${}_{}$\\*
${}_{}$\\*
${}_{}$\\*
${}_{}$\\*
Questions?
\end{frame}


%\begin{frame}[fragile]
%\frametitle{This leaking in our approach}
%\end{frame}
