\section{Introduction}
\label{s:intro}
%\newpage
%\LINE
Class invariants are
a useful concept when reasoning about software correctness in OO (object oriented) languages, and are predicates on the state of an object and its ROG (reachable object graph).
They can be presented as documentation, checked as part of static verification, or, as we do in this paper, monitored for violations using runtime verification.
In our system, a class specifies its invariant by defining a boolean method called \Q@invariant@.
We say that an object's invariant holds when its \Q@invariant@ method would return \Q@true@. 
We do this, like Dafny~\cite{DBLP:conf/sigada/Leino12}, to minimise special syntactic and type-system treatment of invariants.%, making them easier to understand for users, whereas most other approaches treat invariants as a special annotation with its own syntax.

Class invariants are designed to hold in most moments, but they can be (temporarily) broken and observed broken.
%An \emph{invariant protocol}~\cite{FlexibleInvariants} specifies when invariants need to be checked, and when they can be assumed; if such checks guarantee said assumptions, the protocol is sound.
The two main sound invariant protocols present in literature are \emph{visible state semantics} \cite{Meyer:1988:OSC:534929} and the \emph{Boogie/Pack-Unpack methodology}~\cite{DBLP:journals/jot/BarnettDFLS04}.
In the visible state semantics, they can be broken when a method on the object is active (that is, one of the object methods is currently in execution somewhere in the stack trace).
Some interpretations of the visible state semantic are more permissive, requiring the invariants of receivers to hold before and after every public method call, and after constructors. 
In pack-unpack, objects can be either packed or unpacked, and only the invariant of unpacked objects can be broken.



%Object-oriented programming languages provide great flexibility through subtyping and dynamic dispatch: they
%allow code to be adapted and specialised to behave differently in different contexts.
%%%, which is made even more complex by dynamic class loading (supported by many mainstream OO languages).
%However this flexibility hampers code reasoning, since OO languages typically support nearly unrestricted use of exceptions,
%memory mutation, object aliasing, and I/O.

%------------
In this paper we propose a much stricter invariant protocol: the invariant of every observable object must hold. We formally define \emph{observable} later in the paper, but for example it requires that
 at the point of a method call, say \Q@a.foo(b)@,
the invariant of all the objects in the ROG of the receiver and all
the arguments (\Q@a@ and \Q@b@) must hold.
%In a pure OO context where every operation is mediated by a method
%call, including object creation, field access and field update on
%'this', this intuition covers near every case of where 

Note that this is much stronger than just saying that the invariant
should hold every time an object is actually observed (for example,
every time a field is accessed).
This is more flexible that refinement types~\cite{?}: 
objects that are not visible can be be broken;
including objects ready for garbage collection.

This stricter invariant protocol would clearly support easier reasoning; however 
at a first glance it may look too restrictive, preventing us to express useful program behaviour.
Consider the iconic example of a \Q@Range@ class, with a \Q@min@ and \Q@max@
value, where the invariant requires that \Q@min<=max@.
%We assume to be in a language where field declaration induces getters and setters.
%We also assume a static factory method taking the fields as parameters and initializing a new object.
\begin{lstlisting}
class Range{ field min; field max;//assumed getters and setters
  method invariant(){return min<max;}
  method set(min, max){
    if(min>=max){throw new Error(/**/);}
    this.min(min);//setters for min/max
    this.max(max); }  }
\end{lstlisting}
In this first example we omit types to focus on the behaviour.
Under the visible state semantic, this code of \Q@set@ is ok:
\Q@min@ may temporarily break the invariant, that is fixed the moment
after by \Q@max@. It is ok to break the invariant in that point, since
we are inside the method \Q@set@ of \Q@Range@; thus there is an active method.
However, under our stricter approach, we consider this code to be intrinsically wrong. The moment
\Q@this.max(max)@ is called, the invariant of \Q@this@ may be broken, and
the invariant of an observable object can never be broken. For example, if we was to inject a call
\Q@Do.stuff(this);@ between the two setters call,
arbitrary user code could observe a broken object.

Using the \emph{box pattern}\footnote{
This pattern is obvious enough that we do not wish to claim it as a contribution of our work,
but we are unable to find it referenced with a specific name in literature. Technically speaking, it is a simplification of the Decorator, but with a different goal in mind.}, we can provide a modified
\Q@Range@ class with the desired client interface, while respecting the principles of our strict protocol:
\begin{lstlisting}
class BoxRange{//no invariant in BoxRange
  field min; field max;//assumed getters and setters
  BoxRange(min, max){this.min=min;this.max=max;}
  method Void set(min, max){
    if(min>=max){throw new Error(/**/);}
    this.min(min); this.max(max); }  }
class Range{ field box; //a BoxRange
  method invariant(){
    return this.box().min<this.box().max;}
  method set(min, max){
    return this.box().set(min,max); }  }
\end{lstlisting}
The code of \Q@Range.set@ does not violate our invariant protocol:
  since \Q@this@ $\notin \mathit{ROG}($\Q@this.box()@$)$, the call
\Q@BoxRange.set@ works in an environment where the \Q@Range@ object is
not observable, thus its invariant can be temporarily broken.
While in very specific situations the overhead of creating such additional box object may be unacceptable, 
we designed our work for environments where such fine performance differences are negligible.\footnote{
Many VMs and compilers optimize away wrapper objects in many circumstances.\cite{help}}

With appropriate type annotations, the code of \Q@Range@ and \Q@BoxRange@ is accepted as correct by our system, and no matter how \Q@Range@ objects are used, a broken \Q@Range@ objects will never be observable.
In the reminder of this work, we discuss how to combine runtime checks, object capabilities  and
type modifiers to create a convenient language where our strict invariant protocol can
be soundly enforced; even in the presence of mutations, I/O, non
determinism and exceptions, all under the open world assumption, when
we only need to assume that all code is well typed.

\subheading{Summary of our contributions}
We have fully implemented our protocol in L42\footnote{
Our implementation is implemented by checking that a given class conforms to our protocol, and injecting invariant checks in the appropriate places.
A suitably anonymised, experimental version of L42, supporting the protocol described in this paper, together with the full code of our case studies, is available at \url{http://l42.is/InvariantArtifact.zip}. We believe it would be easy to port our work on Pony and Gordon \etal's language.}, we used this implementation to implement many case studies; 
to show that our work supports real world applications one of those case studies models an interactive GUI.
In our case study, our protocol performed only $77$ invariant checks, whereas the visible state semantic invariant protocols of D and Eiffel perform $53$ and $14$ million checks (respectively). See Section \ref{s:case-study} for an explanation of these result.
We also compare with Spec\#; in the GUI case Spec\# did $77$ checks, as our approach; however the annotation burden was almost $4$ times higher than ours.
In pack/unpack, an object's invariant is checked only by the pack operation.
In order for this to be sound, some form of aliasing and/or mutation control is necessary. For example, Spec\#~\cite{Barnett:2004:SPS:2131546.2131549}, which follows the pack/unpack methodology, uses a theorem prover, together with source code annotations.
While Spec\# can be used for full static verification, it conveniently allows invariant checks to be performed
at runtime, whilst statically verifying aliasing, purity and other similar standard properties.
This allows us to closely compare our approach with Spec\#.
Since a case study composed by a single program is not very compelling, in the appendix we present many more case studies.


In this paper we argue that our protocol is not only more succinct than the pack/unpack approach, but is also easier and safer to use.
Moreover, our approach deals with more scenarios than most prior work: we allow sound catching of invariant failures and also carefully handle non-deterministic operations like I/O.
%In our case study we show that
%we can still encode most of the examples explored in ~\cite{???} (including for example mutable collections of immutable objects) whilst having a significantly lower annotation-burden.

%--I think we can avoid this to save space
%Section \ref{s:TMsAndOCs} explains the pre-existing \emph{type modifier} features we use for this work.
%Section \ref{s:protocol} explains the details of our invariant protocol, and Section \ref{s:formalism} formalises a language enforcing this protocol.
%Sections \ref{s:immutable} and \ref{s:encapsulated} explain and motivate how our protocol can handle invariants over immutable and encapsulated mutable data, respectively.
%Section \ref{s:case-study} presents our GUI case study and compares it against visible state semantics and Spec\#: they performed 5 orders of magnitude more invariant checks, and required 60\% more annotations, respectively.
%Sections \ref{s:related} and \ref{s:conclusion} provide related work and conclusions.

Appendix \ref{s:proof} provides a proof that our invariant protocol is sound. In
Appendix~\ref{s:hamster} we explore in the detail another case study, and we explains exactly why the 
Spec\# encoding of those examples is so verbose.
In Appendix~\ref{s:MoreCaseStudies}, we show even more case studies; one of those was designed a worst case scenario for our invariant protocol, where Spec\# 
performed four times less invariant checks,
 while D and Eiffel performed only twice as many.
We also compare with examples from others work on Spec\# ~\cite{DBLP:journals/jot/BarnettDFLS04,leino2004object,DBLP:conf/mpc/BarnettN04}; we show why we cannot encode some of their examples: namely when state that an object's invariant depends on can be directly modified by other objects.
At first glance, our approach may feel very restrictive; in Appendix~\ref{s:patterns}, we show
programming patterns demonstrating that these restrictions do not significantly hamper expressiveness, in particular we show how batch mutation operations can be performed with a single invariant check, and how the state of a `broken' object can be safely passed around.
% In Appendix \ref{s:runtime-verification}, we discuss more related work on runtime verification.


%you see this already later on... I wanted to avoid repating it
%to perform batch operations with a single invariant check, as well as how the state of `broken' objects can be passed around.}	
%http://www.cs.cmu.edu/~NatProg/papers/p496-coblenz-Glacier-ICSE-2017.pdf
