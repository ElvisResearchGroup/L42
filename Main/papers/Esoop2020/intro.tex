\section{Introduction}
\label{s:intro}
%\newpage
%\LINE
Representation invariants (sometimes called class invariants or object invariants) are
a useful concept when reasoning about software correctness in Object Oriented (OO) languages and are predicates on the state of an object and its ROG (reachable object graph).
They can be presented as documentation, checked as part of static verification, or, as we do in this paper, monitored for violations using runtime verification.
In our system, a class specifies its invariant by defining a boolean method called \Q@invariant()@.
We say that an object's invariant holds when its \Q@invariant()@ method would return \Q@true@. 
\footnote{Doing this (like Dafny~\cite{DBLP:conf/sigada/Leino12}) we minimise special treatment of invariants. Some other approaches treat invariants as a special annotation with its own syntax.}

Invariants are designed to hold in most moments, but 
is common to knowingly violate invariants for short periods before restoring them.
To support this behaviour the two main invariant protocols present in the literature allows invariants to be (temporarily) broken and observed broken.
They are the \emph{visible state semantics} \cite{Meyer:1988:OSC:534929} and the \emph{Boogie/Pack-Unpack methodology}~\cite{DBLP:journals/jot/BarnettDFLS04}.
In the visible state semantics, invariants can be broken when a method on the object is active (that is, one of the object methods is currently in execution somewhere in the stack trace).
Some interpretations of the visible state semantics are more permissive, requiring the invariants of receivers to hold only before and after every public method call, and after constructors. 
In pack-unpack, objects can be either packed or unpacked, and only the invariant of unpacked objects can be broken.

%------------
In this paper we propose a much stricter invariant protocol: in any moment the invariant of every object involved in execution must hold.
Thus, invariants can be broken only when the object is not involved in execution. We formally define \emph{involved in execution} later in the paper, but for example it requires that at the point of a method call, say \Q@a.foo(b)@, the invariant of all the objects in the ROG of the receiver and all the arguments (\Q@a@ and \Q@b@) must hold.

This stricter invariant protocol would clearly support easier reasoning; however 
at a first glance it may look too restrictive, preventing us to express useful program behaviour.
Consider the iconic example of a \Q@Range@ class, with a \Q@min@ and \Q@max@
value, where the invariant requires that \Q@min<=max@:
\begin{lstlisting}
class Range{ field min; field max;//assumed getters and setters
  method invariant(){return min<max;}
  method set(min, max){
    if(min>=max){throw new Error(/**/);}
    this.min(min);//setters for min/max
    this.max(max); }  }
\end{lstlisting}
In this first example we omit types to focus on the behaviour.
Under visible state, this code of \Q@set(min,max)@ is ok:
\Q@min(min)@ may temporarily break the invariant, that is fixed the moment
after by \Q@max(max)@. Visible state allows to break the invariant in that point, since
we are inside the method \Q@set(min,max)@ of \Q@Range@; thus there is an active method.
However, under our stricter approach, we consider this code to be intrinsically wrong. The moment
\Q@this.max(max)@ is called, the invariant of \Q@this@ may be broken, and
the invariant of an object involved in execution can never be broken. For example, if we was to inject a call
\Q@Do.stuff(this);@ between the two setters call,
arbitrary user code could observe a broken object.

Using the \emph{box pattern}\footnote{
This pattern is obvious, and we do not wish to claim it as a contribution of our work.
However we are unable to find it referenced with a specific name in the literature. Technically speaking, it is a simplification of the Decorator, but with a different goal.}, we can provide a modified
\Q@Range@ class with the desired client interface, while respecting the principles of our strict protocol:
\begin{lstlisting}
class BoxRange{//no invariant in BoxRange
  field min; field max;//assumed getters and setters
  BoxRange(min, max){this.min=min;this.max=max;}
  method Void set(min, max){
    if(min>=max){throw new Error(/**/);}
    this.min(min); this.max(max); }  }
class Range{ field box; //a BoxRange
  method invariant(){
    return this.box().min<this.box().max;}
  method set(min, max){
    return this.box().set(min,max); }  }
\end{lstlisting}
The code of \Q@Range.set(min,max)@ does not violate our invariant protocol:
  since \Q@this@ $\notin \mathit{ROG}($\Q@this.box()@$)$, the call
\Q@BoxRange.set@ works in an environment where the \Q@Range@ object is
not involved in execution, thus its invariant can be temporarily broken.
Indeed, using the box as an extra level of indirection, we restricted the set of objects involved in execution while the state of \Q@Range@ was modified.
\footnote{While in very specific situations the overhead of creating such additional box object may be unacceptable, 
we designed our work for environments where such fine performance differences are negligible.
Many VMs and compilers optimize away wrapper objects in many circumstances.\cite{help}}

With appropriate type annotations, the code of \Q@Range@ and \Q@BoxRange@ is accepted as correct by our system, and no matter how \Q@Range@ objects are used, a broken \Q@Range@ objects will never be involved in execution.

\subheading{Summary:}
In the reminder of this work, we discuss how to combine runtime checks and Reference and Object Capabilities to create a convenient language where our strict invariant protocol is soundly enforced; even in the presence of mutations, I/O, non-determinism and exceptions, all under the open world assumption, when
we only need to assume that all code is well typed.
We formalise our approach and (in the Appendix) prove that our protocol soundly enforce invariants.

We have fully implemented our protocol in L42\footnote{
Our implementation is implemented by checking that a given class conforms to our protocol, and injecting invariant checks in the appropriate places.
An anonymised version of L42, supporting the protocol described in this paper, together with the full code of our case studies, is available at \url{http://l42.is/InvariantArtifact.zip}. %We believe it would be easy to port our work on Pony and Gordon \etal's language.
}, we used this implementation to implement many case studies, showing that our protocol is more succinct than the pack/unpack approach, easier to use and much more efficient then the visible state semantic.
It is important to note that most prior work do not soundly handle catching of invariant failures and I/O.
%In our case study we show that
%we can still encode most of the examples explored in ~\cite{???} (including for example mutable collections of immutable objects) whilst having a significantly lower annotation-burden.
%--I think we can avoid this to save space
%Section \ref{s:TMsAndOCs} explains the pre-existing \emph{type modifier} features we use for this work.
%Section \ref{s:protocol} explains the details of our invariant protocol, and Section \ref{s:formalism} formalises a language enforcing this protocol.
%Sections \ref{s:immutable} and \ref{s:encapsulated} explain and motivate how our protocol can handle invariants over immutable and encapsulated mutable data, respectively.
%Section \ref{s:case-study} presents our GUI case study and compares it against visible state semantics and Spec\#: they performed 5 orders of magnitude more invariant checks, and required 60\% more annotations, respectively.
%Sections \ref{s:related} and \ref{s:conclusion} provide related work and conclusions.
We describe one of our case studies in the body of the paper; we explore more case studies in the Appendix.

At first glance, our approach may feel very restrictive; in Appendix~\ref{s:patterns}, we show
programming patterns demonstrating that these restrictions do not significantly hamper expressiveness, in particular we show how batch mutation operations can be performed with a single invariant check, and how the state of a `broken' object can be safely passed around.
% In Appendix \ref{s:runtime-verification}, we discuss more related work on runtime verification.


%you see this already later on... I wanted to avoid repating it
%to perform batch operations with a single invariant check, as well as how the state of `broken' objects can be passed around.}	
%http://www.cs.cmu.edu/~NatProg/papers/p496-coblenz-Glacier-ICSE-2017.pdf
