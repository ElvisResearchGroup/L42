\section{Formal Language Model}
\label{s:formalism}

%----------------------------------
IS THIS NEEDED?
For simplicity, in the reminder of the paper we require 
receivers to always be specified explicitly, and require that the receivers of field accesses and updates are always \Q!this!; that is, all fields are instance private.
We also do not allow explicit constructor definitions, instead we assume constructors are of the standard form \Q@$C$($T_1 x_1$,$\ldots$,$T_n x_n$) {this.$f_1$=$x_1$;$\ldots$;this.$f_n$=$x_n$;}@, where the fields of $C$ are $T_1 f_1;\ldots; T_n f_n;$. This ensures that partially uninitialised (and likely invalid) objects are not passed around or used. 
END OF IS THIS NEEDED?

In order to model our system, we need to formalise an imperative object-oriented language
with exceptions, object capabilities, and rich type system
support for TMs and strong exception safety.
Formal models of the runtime semantics of such languages are simple, but 
defining and proving the correctness of such a type system would require a paper
of its own, and indeed many such papers exist in literature%
~\cite{ServettoEtAl13a,ServettoZucca15,GordonEtAl12,clebsch2015deny,JOT:issue_2011_01/article1}.
Thus we are going to assume that we already have an expressive and sound type system enforcing the properties we need, and instead focus on invariant checking.
We clearly list in Appendix \ref{s:proof} the assumptions we make on such a type system, so that any language satisfying them, such as L42, can soundly support our invariant protocol.

To keep our small step semantics as conventional as possible, we follow Pierce~\cite{pierce2002types} and Featherweight Java~\cite{IgarashiEtAl01}, and assume:
\SSI\begin{itemize}
	\item An implicit program/class table; we use the notation $C.m$ to get the method declaration for $m$, within class $C$, similarly we use $C.f$ to get the declaration of field $f$, and $C.i$ to get the declaration of the $i$\textsuperscript{th} field.
	\item Memory, $\s : l\rightarrow C\{\Many{v}\}${, is} a finite map from locations, $l$, to annotated tuples, $C\{\Many{v}\}$, representing objects; where $C$ is the class name and $\Many{v}$ are the field values.
	We use the notation $\s[l.f=v]$ to update a field of $l$, $\s[l.f]$ to access one, and $\s \setminus l$ to delete $l$.
	\item A main expression that is reduced in the context of such a memory and program.
	\item A typing relation, $\S;\G;\E\vdash e:T$, where 
	the expression $e$ can contain locations and free variables. The types of locations are encoded in 
a memory environment, 
$\S : l\rightarrow C$,
	while the types of free variables are encoded in
a variable environment, $\G : x\rightarrow T$. \E encodes the location, relative to the top-level expression we are typing, where $e$ was found; this is needed so that $l$s can be typed with different type-modifiers when in different positions.
	\item We use \Ss to trivially extract the corresponding \S from a \s.
\end{itemize}

\noindent To encode object capabilities and I/O, we assume a special location  $c$ of class \Q@Cap@. This location would refer to an object with methods that behave non-deterministically, such methods would model operations such as file reading/writing. In order to simplify our proof, we assume that:
\SSI\begin{itemize}
	\item \Q@Cap@ has no fields,
	\item instances of \Q@Cap@ cannot be created with a \Q@new@ expression,
	\item \Q@Cap@'s \Q@invariant@ method is defined to have a body of `\Q!true!', and
	\item all other methods in the \Q@Cap@ class must require a \Q@mut@ receiver; such methods will have a non-deterministic body, i.e. calls to them may have multiple possible reductions.
\end{itemize}
For simplicity, we do not formalise actual exception objects, rather we have \error{}s, which correspond to expressions which are currently  `throwing' an exception; 
in this way there is no value associated with an \error.
Our L42 implementation instead allows arbitrary \Q!imm! values to be thrown as exceptions, formalising exceptions in this way would not cause any interesting variation of our proof.

%	\setlength{\belowcaptionskip}{0.06em}
\begin{figure}
	\begin{grammatica}
		\produzione{e}{\x\mid l\mid\Kw{true}\mid\Kw{false}\mid e\singleDot\m\oR\es\cR\mid e\singleDot\f 
			\mid e\singleDot\f\equals e 
			\mid\Kw{new}\ C\oR\es\cR
			\mid\Kw{try}\, \oC e_1\cC\, \Kw{catch}\, \oC e_2\cC
		}{expression}\\
		\seguitoProduzione{
			\mid \M{l}{e_1}{e_2}\mid\Kw{try}^{\s}\oC e_1\cC\ \Kw{catch}\ \oC e_2\cC
		}{runtime expr.}\\
		\produzione{v}{l}{value}\\
		\produzione{\EV}{\square
			\mid \EV\singleDot m\oR\es\cR
			\mid v\singleDot\m\oR\Many{v}_1,\EV,\es_2\cR
			%\mid \EV\singleDot\f 
			%\mid \EV\singleDot\f\equals\e
			\mid v\singleDot\f\equals\EV
		}{eval. context}\\
		\seguitoProduzione{
			\mid \Kw{new}\ C\oR\Many{v}_1,\EV,\es_2\cR
			\mid \M{l}{\EV}{e}
			\mid \M{l}{v}{\EV}
			\mid \Kw{try}^\s\oC\EV\cC\ \Kw{catch}\ \oC e\cC}{}\\
		
		\produzione{\E}{\square\mid\E\singleDot m\oR\es\cR\mid e\singleDot\m\oR\es_1,\E,\es_2\cR
			%\mid \E\singleDot\f 
			%\mid \E\singleDot\f\equals\e
			\mid e\singleDot\f\equals\E
			\mid \Kw{new}\ C\oR\es_1,\E,\es_2\cR
		}{full context}\\
		\seguitoProduzione{
			\mid
			\M{l}{\E}{e}\mid
			\M{l}{e}{\E}\mid
			\Kw{try}^{\s?}\oC\E\cC\ \Kw{catch}\ \oC e\cC\mid
			\Kw{try}^{\s?}\oC e\cC\ \Kw{catch}\ \oC\E\cC
			
		}{}\\
		
		
		%\produzione{M_l}{\E[M\oR l,e\cR]}{}\\
		%\produzione{\EG_l}{
		%  M_l\singleDot\m\oR\es_1,\E,\es_2\cR
		% |e\singleDot\m\oR\es_1, M_l, \es_2, \E, \es_3\cR
		% |M_l\singleDot\f\equals\E
		% |\Kw{new}\ C\oR\es_1,M_l,\es_2,\E,\es_3\cR
		% |\Kw{try}\oC\E\cC\ \Kw{catch}\ \oC e\cC
		% |\E[\EG_l]}{}\\
		\produzione{\mathit{CD}}{\Kw{class}\ C\ \Kw{implements}\ \Many{C}\oC\Many{F}\,\Many{M}\cC\mid 
			\Kw{interface}\ C\ \Kw{implements}\ \Many{C}\oC\Many{M}\cC
		}{class decl.}\\
		\produzione{F}{\T\ \f\semiColon}{field}\\
		\produzione{M}{\mdf\, \Kw{method}\, \T\ \m\oR\T_1\,\x_1,\ldots,\T_n\,\x_n\cR\ \Opt e}{method}\\
		\produzione{\mdf}{\Kw{mut}\mid\Kw{imm}\mid\Kw{capsule}\mid\Kw{read}}{type modifier}\\
		\produzione{\T}{\mdf\,C}{type}\\
		\produzione{r_l}{
			v\singleDot\m\oR\Many{v}\cR
			\mid v\singleDot\f
			\mid v_1\singleDot\f\equals v_2
			\mid \Kw{new}\,C\oR\Many{v}\cR
			,\quad\text{where }l\in \{v,v_1,v_2,\Many{v}\}
		}{redex with $l$}\\
		\produzione{\error}{
			\EV[\M{l}{v}{\Kw{false}}]
			,\quad\text{where }
			\EV \text{ not of form}\ \EV'[\Kw{try}^{\s?}\oC\EV''\cC\ \Kw{catch}\ \oC\_\cC]
		}{validation error}
	\end{grammatica}%\vspace{-1em}
\SS[1]\caption{Grammar}\label{f:grammar}\SS[1.5]
\end{figure}


\subheading{Grammar}
The detailed grammar is defined in Figure \ref{f:grammar}. 
Most of our expressions are standard.
\emph{Monitor expressions}
 are of the form \M{l}{e_1}{e_2}, they 
are run time expressions and thus are not present in method bodies, rather they are generated by our reduction rules inside the main expression. Here, $l$ refers to the object being monitored, $e_1$ is the expression which is being monitored, and $e_2$ denotes the evaluation of $l.\invariant$; $e_1$ will be evaluated to a value, and the $e_2$ will be further evaluated, if $e_2$ evaluated to \Q!false! or an \error, then $l$'s invariant failed to hold; such a monitor expression corresponds to the throwing of an unchecked exception.

In addition, our reduction rules will annotate \Q@try@ expressions with
the original state of memory. This is used in our type-system assumptions (see appendix \ref{s:proof}) to model the guarantee of strong exception safety, that is, the annotated memory will not be mutated by executing the body of the \Q@try@.

\subheading{Well-Formedness Criteria}
We additionally restrict the grammar with the following well-formedness criteria:
\SSI\begin{itemize}
	\item \Q@invariant@ methods and capsule mutators satisfy the restrictions in Section \ref{s:protocol}.
	\item Field accesses and updates in methods are of the form $\Kw{this}.f$ or $\Kw{this}.f\equals e$, respectively.
	\item Field accesses and updates in the main expression are of the form $l.f$ or $l.f\equals e$, respectively.
	\item Method bodies do not contain any $l$ or $\M{\_}{\_}{\_}$ expressions.
\end{itemize}
\newcommand{\rowSpace}{\\\vspace{2.5ex}}
\begin{figure}
	\!\!
	$\!\!\!\!\!\begin{array}{l}
	\inferrule[(update)]{{}_{}}{
		\s|l.f\equals{}v\rightarrow \s[l.f=v]|
		\M{l}{l}{l\singleDot\invariant}
	}{}
	\quad
	\inferrule[(new)]{{}_{}}{
		\s|\Kw{new}\ C\oR\vs\cR\rightarrow \s,l\mapsto C\{\vs\}|
		\M{l}{l}{l\singleDot\invariant}
	}{}
	\\
	\rowSpace
	\inferrule[(mcall)]{{}_{}}{
		\s|l\singleDot\m\oR v_1,\ldots,v_n\cR\rightarrow \s|
		e'[\Kw{this}\coloneqq l,\x_1\coloneqq v_1,\ldots,x_n\coloneqq v_n]
	}{
		\begin{array}{l}
		\s(l)=C\{\_\}\\
		C.m=\mdf\,\Kw{method}\,\T\,\m\oR\T_1\,\x_1\ldots\T_n\x_n\cR\,e\\
		
		
		\text{if } \mdf=\Kw{mut}\text{ and }\exists \f\text{ such that }
		\\*\quad C.f=\Kw{capsule}\,\_ \text{ and } e = \E[\Kw{this}\singleDot\f]\\*
		\text{then }e'=\M{l}{e}{l\singleDot\invariant}\\*
		\text{otherwise }e'= e
	\end{array}
}
\rowSpace
\inferrule[(monitor exit)]{{}_{}}{
	\s|\M{l}{v}{\Kw{true}}\rightarrow \s|v
}{}
\quad

\inferrule[(ctxv)]{\s_0|e_0\rightarrow\s_1|e_1}{
	\s_0|\EV[e_0]\rightarrow \s_1|\EV[e_1]
}{}

\quad
\inferrule[(try enter)]{{}_{}}{
	\s|\Kw{try}\ \oC e_1\cC\ \Kw{catch}\ \oC e_2\cC\rightarrow 
	\s|\Kw{try}^\s\oC e_1\cC\ \Kw{catch}\ \oC e_2\cC
}{}
\quad

\rowSpace

\inferrule[(try ok)]{{}_{}}{
	\s'|\Kw{try}^{\s}\oC v\cC\ \Kw{catch}\ \oC\_\cC\rightarrow \s'|v
}{}
\quad

\inferrule[(try error)]{{}_{}}{
	\s'|\Kw{try}^\s\oC \error\cC\ \Kw{catch}\ \oC e\cC\rightarrow \s'|e
}
\quad
\inferrule[(access)]{{}_{}}{
	\s|l.f\rightarrow \s|\s[l.f]
}{}
%\quad
\end{array}$
\SS[1.5]\caption{Reduction rules}\label{f:reductions}\SS[1.5]
\end{figure}

\subheading{Reduction rules}
Our reduction rules are defined in Figure \ref{f:reductions}.
They are pretty standard, except for our handling of monitor expressions. Monitor expressions are added after all field updates, \Q@new@ expressions, and calls to capsule mutators.
%Our formalism of monitor expressions are only a proof device, they need not be part of the language itself, for example L42 implements our invariant protocol by generating wrapper functions over primitive setters and factory methods.
%Monitor expressions are only a proof device, and an execution on a real hardware 
Monitor expressions are only a proof device, they need not be implemented directly as presented.
For example, in L42 we implement them by statically injecting calls to \Q!invariant! at the end of setters, factory methods and capsule mutators; this works as L42 follows the uniform access principle, so it does not have primitive expression forms for field updates and constructors, rather they are uniformly represented as method calls.
% do not need to represent them.  In L42 field updates are always performed throughout a setters, thus we can just inject calls to \Q@invariant@ on setters, at the end of constructor bodies and at the end of  capsule mutators.

The failure of a monitor expression, \M{l}{e_1}{e_2}, will be caught by our \textsc{try error} rule, as will any other uncaught monitor failure in $e_1$ or $e_2$.

\subheading{Statement of Soundness}
We define a deterministic reduction to mean that exactly one reduction is possible:\\*
\indent$\ \s_0|e_0\Rightarrow \s_1|e_1$ iff $\{\s_1|e_1\}=\{\s|e \text{, where } \s_0|e_0\rightarrow \s|e\}$

\noindent An object is \valid iff calling its \Q@invariant@ method would
deterministically produce \Q@true@ in a finite number of steps, i.e. it does not evaluate to \Q@false@, fail to terminate, or produce an \error.
We also require evaluating \Q@invariant@ to preserve existing memory (\s), however new objects ($\s'$) can be created and freely mutated.

\indent$\valid(\s,l)$ iff $\s | l.\invariant {\Rightarrow^+} \s,\s' | \Kw{true}$.%\loseSpace

\noindent 
To allow the invariant method to be called on an invalid object, and access fields on such object, we define the set of trusted execution steps as the the call to \Q@invariant@ itself, and any field accesses inside its evaluation. Note that this only applies to single small step reductions, and not the entire evaluation of \Q!invariant!.

%\loseSpace
\indent $\trusted(\EV,r_l)$ iff, either:
\begin{iitemize}
\item $r_l=l.\invariant$ and
$\EV=\EV'[\M{l}{v}{\square}]$, or
\item $r_l=l\singleDot f$ and
$\EV=\EV'[\M{l}{v}{\EV''}]$.
\end{iitemize}
%\loseSpace

\noindent We define a \VS as one that was obtained by any number of reductions from a well typed initial expression and memory, containing no monitors and with only the $c$ memory location available:\\
\indent $\VS(\s, e)$ iff $c\mapsto\Kw{Cap}\{\}|e_0\rightarrow^+ \s|e$, for some $e_0$ with:
\begin{iitemize}
\item[] $c:\Kw{Cap};\emptyset; \h \vdash e_0: 	T$, $\M{\_}{\_}{\_} \notin e_0$, and if $l \in e_0$ then $l = c$.
\end{iitemize}
%\loseSpace

\noindent Finally, we define what it means to soundly enforce our invariant protocol: every object referenced by any untrusted redex, within a \VS, is valid:%
\SS\begin{theorem}[Soundness]\rm
If $\VS(\s, \EV[r_l])$, then either $\valid(\s,l)$ or $\trusted(\EV,r_l)$.
\end{theorem}
%We believe this property captures very precisely our statements in section~\ref{s:protocol}.
