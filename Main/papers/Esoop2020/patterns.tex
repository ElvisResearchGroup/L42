\section{Patterns}
\label{s:patterns}
\lstset{morekeywords={invalid}}
In Section~\ref{s:case-study} and Appendix~\ref{s:MoreCaseStudies} we showed how the box pattern can be used to write invariants over cyclic mutable object graphs, the latter also shows how a complex mutation can be done in an `atomic' way, with a single invariant check. However the box pattern is much more powerful. Suppose we want to pass a temporarily `broken' object to other code as well as perform multiple field updates with a single invariant check. 
Instead of adding new features to the language, like an \Q!invalid! modifier (denoting an object whose invariant need not hold), and an \Q!expose! statement like Spec\#, we can use a `box' class and a capsule mutator to the same effect:
\begin{lstlisting}
interface Person {
  mut method Bool accept(read Account a, read Transaction t); }

interface Transaction { 
  // Here ImmList<T> represents a list of immutable Ts.
  mut method ImmList<Transfer> compute(); }

class Transfer { Int money;
  // An `AccountBox' is like an `invalid Account':
  //   `that' need not have income > expenses
  method Void execute(mut AccountBox that) {
    // Gain some money, or lose some money
    if (this.money > 0) { that.income += money; }
    else { that.expenses -= money; }}}

class AccountBox { UInt income = 0; UInt expenses = 0; }
class Account {
  capsule AccountBox box; mut Person holder;
  read method Bool invariant() {
    return this.box.income > this.box.expenses; }

  // `h' could be aliased elsewehere in the program    
  Account(mut Person h) { 
    this.holder = h; this.box = new AccountBox(); }

  mut method Void transfer(mut Transaction ts) {
    if (this.holder.accept(this, ts)) {
	  this.transferInner(ts.compute()); }}

  // capsule mutator, like an `expose(this)' statement
  private mut method Void transferInner(ImmList<Transfer> ts) {
     mut AccountBox b = this.box;
     for (Transfer t : ts) { t.execute(b); }
     // check the invariant here
}}
\end{lstlisting}
The idea here is that \Q!transfer(ts)! will first check to see if the account holder wishes to accept the transaction, it will then compute the full transaction (which could cache the result and/or do some I/O), and then execute each transfer in the transaction. We specifically want to allow an individual \Q!Transfer! to raise the \Q!expenses! field by more than the \Q!income!, however we don't want an entire \Q!Transaction! to do this. 
Our capsule mutator (\Q!transferInner!) allows this by behaving like a Spec\# \Q!expose! block: during its body (the \Q!for! loop) we don't know or care if \Q!this.invariant()! is \Q!true!, but at the end it will be checked. For this to make sense, we make \Q!Transfer.execute! take an \Q!AccountBox! instead of an \Q!Account!: it cannot assume that the invariant of \Q!Account! holds, and it is allowed to modify the fields of \Q!that! without needing to check it. As you can see, adding support for features like \Q!invalid! and \Q!expose! is unnecessary, and would likely require making the type system significantly more complicated as well as burdening the language with more core syntactic forms.

In particular, the above code demonstrates that our system can:
\SSI\begin{itemize}
\item Have useful objects that are not entirely encapsulated: the \Q!Person holder! is a \Q!mut! field; this is fine since it is not mentioned in the \Q!invariant! method.
\item Perform multiple state updates with only a single invariant check: the loop in \Q!transferInner! can perform multiple field updates of \Q!income! and \Q!expenses!, however the \Q!invariant! will only be checked at the end of the loop.
\item Temporarily break an invariant: it is fine if during the \Q!for! loop, \Q!expenses > income!, provided that this is fixed before the end of the loop.
\item Pass the state of an `invalid' object around, in a safe manner: an \Q!AccountBox! contains the state of \Q!Account!, but not its invariant: if you have an \Q!Account!, you can be sure that its \Q!income > expenses!, but not if you have an \Q!AccountBox!.
\item Wrap normal methods over capsule mutators: \Q!transfer! is not a capsule mutator, so it can use \Q!this! multiple times and take a \Q!mut! parameter.
\end{itemize}

\noindent Though capsule mutators can be used to perform batch operations like the above, they can only take immutable and capsule objects. This means that they can perform no non-deterministic I/O (due to our OC system), and other externally accessible objects (such as a \Q!mut Transaction!) cannot be mutated during such a batch operation.

\subheading{The Transform Pattern}
Recall the GUI case study in Section~\ref{s:case-study}, where we had a \Q!Widget! interface and a \Q!SafeMovable! (with an invariant) that implements \Q!Widget!.
% A capsule mutator method is essentially a mutation of a field, which is guaranteed to not see the \Q@this@ object.
% Thus, if \Q@this@ is made invalid during  the method's execution, we could not observe it until after the method completes.
Suppose we want to allow \Q@Widget@s to be scaled, we could add \Q@mut@ setters for \Q@width@, \Q@height@, \Q@left@, and \Q@top@ in the \Q@Widget@ interface. However, if we also wish to scale its children we have a problem, since \Q@Widget.children@ returns a \Q@read Widgets@, which does not allow mutation. We could of course add a \Q@mut@ method \Q@zoom@ to the \Q@Widget@ interface, however this does not scale if more operations are desired. If instead \Q@Widget.children@ returned a \Q@mut Widgets@, it would be difficult for \Q@Widget@ implementations, such as \Q@SafeMovable@, 
to mention their \Q!children! in their \Q!invariant!.

% In the above \Q@SafeMovable@ we only had one capsule mutator: \Q@dispatch@. However suppose a \Q@Widget@ wants to directly mutate it's descendents, however it can't do that since \Q@Widget.children@ returns a \Q@read Widgets@, if it returned a \Q@mut Widgets@ then \Q@SafeMovable@ could not be implement, as it's children are contained inside a capsule-field. 
% At first glance, it may seem that capsule mutators allow only very limited kinds %of mutation.
% This is however not the case. 
% Consider the following
% simple pattern to allow flexible use of encapsulated content: define a

A simple and practical solution would be to define a \Q@transform@ method in \Q@Widget@, and a \Q@Transformer@ interface 
like so:\footnote{A more general transformer could return a generic \Q@read R@.}
\begin{lstlisting}[escapechar=\%]
interface Transformer<T> { method Void apply(mut T elem); }
interface Widget { ...
  mut method Void top(Int that); // setter for immutable data
  // transformer for possibly encapsulated data
  mut method read Void transform(Transformer<Widgets> t);
}

class SafeMovable { ...
  // A well typed capsule mutator
  mut method Void transform(Transformer<Widgets> t) {
    t.apply(this.box.c); }}
\end{lstlisting}
% Note that the code above does not access a \Q!capsule! field but merely calls a method that does; thus  it is \emph{not} a capsule mutator method, so it is not constrained by the restrictions on them. Code like the above would also allow one to mutate multiple \Q!capsule! fields in one method.
%Our pattern cooperates with the language’s restrictions to ensure each mutation is completed as a separate operation, that is perceived by the rest of the system %as if it was atomic.%
%,  i.e. they can't see or update the other \Q!capsule! fields.
The \Q@transform@ method offers an expressive power similar to \Q@mut@ getters, but prevents \Q@Widgets@ from leaking out.  With a \Q@Transformer@, a \Q@zoom@ function could be simply written as:
\begin{lstlisting}
static method Void zoom(mut Widget w) {
  w.transform(ws -> { for (wi : ws) { zoom(wi,scale); }});
  w.width(w.width() / 2); ...; w.top(w.top() / 2); }
\end{lstlisting}

% One of the advantages of this approach is that a the \@zoom@ method can be written by anyone anywhere

% \begin{lstlisting}[escapechar=\%]
%// Lambda Expression that creates a new Transformer<...>
%this.transform(l -> l.add(new MyWidget(..)))
%\end{lstlisting}
%//`i' is captured by the closure.
%// `imm' and `capsule' varaibles can be captured.

%    %\Comment{}%this.items.add(i);
%    // Cant instead capture `this': it can't be typed %as `imm'
%    // since `ItemTransformer.transform()' is an %`imm' method
%  })
%}
%  // instead of:
%\Comment{}%this.exposeItems().add(i)

%Note that the code above does not access a \Q!capsule! field but merely calls a method that does; thus
%it is \emph{not} a capsule mutator method, so it is not constrained by the restrictions on them. Code like the above would also allow one to mutate multiple \Q!capsule! fields in one method.
%Our pattern cooperates with the language’s restrictions to ensure each mutation is completed as a separate operation, that is perceived by the rest of the system
%as if it was atomic.%
%,  i.e. they can't see or update the other \Q!capsule! fields.


\subheading{A general and flexible Graph class}

Here we rely on the box and the transform patterns to encode
a general \Q@Graph@
of \Q@Node@s, that can be defined as a library.
The user of the library can define personalized kinds of nodes,
with their personalized (sub-)invariant \Q@subInvariant@.
The library will ensure that no matter how it is used, all the 
user defined sub invariants of the inner nodes are preserved.
The \Q@Node@s, even if user defined, are guaranteed to be encapsulated by the \Q@Graph@.
The \Q@Graph@ can be arbitrarily modified by user defined transformations using the Transform Pattern.
\begin{lstlisting}
interface Transform<T>{method read T apply(mut Nodes nodes);}

interface Node{
  read method Bool subInvariant(read Nodes nodes)
  mut method mut Nodes directConnections()
  }
class Nodes{//just an ordered set of nodes 
  mut method Void add(mut Node n){..}
  read method Int indexOf(read Node n){..}
  mut method Void remove(read Node n){..}
  mut method mut Node get(Int index){..}
  }
class Graph{ 
  capsule Nodes nodes; //box pattern
  Graph(capsule Nodes nodes){..}
  read method read Nodes getNodes(){return this.nodes;}
  <T> mut method read T transform(Transform<T> t){
    mut Nodes ns=this.nodes;//capsule mutator single 'this' use
    return t.apply(ns);
    }
  read method Bool invariant(){
    for(read Node n: this.nodes){
      if(!n.subInvariant(this.nodes)){return false;}
      }
    return true;
    } }
\end{lstlisting}
Note how there is basically no logic in the above code.
All the logic can be freely defined by the user code, as shown below:

\begin{lstlisting}
class MyNode{
  mut Nodes directConnections;
  mut method mut Nodes directConnections(){
    return this.directConnections;}
  MyNode(mut Nodes directConnections){..}
  read method Bool subInvariant(read Nodes nodes){
    /*any condition on either this, nodes or the connections*/}  
  capsule method read MyNode addToGraph(mut Graph g){..}
  read method Void connectWith(read Node other,mut Graph g){..}
  }

mut Graph g=new Graph(new Nodes());
read MyNode n1=new MyNode(new Nodes())).addToGraph(g);
read MyNode n2=new MyNode(new Nodes())).addToGraph(g);
//lets connect our two nodes
n1.connectWith(n2,g);
\end{lstlisting}
Here we define a \Q@MyNode@ class, where the subinvariant can express any property over \Q@this@, or the direct connections of \Q@this@, or any other node in graph and their connections.

We can define methods in \Q@MyNode@ to add our nodes
to graphs and to connect with other nodes.
Note that the method \Q@addToGraph@ is \Q@capsule@; this ensures that the node is not in any other graph.
On the other side, the method \Q@connectWith@ takes a \Q@read this@, even if it clearly intend to modify the ROG of \Q@this@.
It works by recovering a \Q@mut@ reference to \Q@this@ from the \Q@mut Graph@.

We now examine the implementation of those two methods:
\begin{lstlisting}
  read method Void connectWith(read Node other,mut Graph g){
    Int i1=g.getNodes().indexOf(this);
    Int i2=g.getNodes().indexOf(other);
    if(i1==-1 || i2==-1){throw /*error nodes not in g*/;}
    g.transform(ns->{
      mut Node n1=ns.get(i1);
      mut Node n2=ns.get(i2);
      n1.directConnections().add(n2);
      });
    }
  capsule method read MyNode addToGraph(mut Graph g){
    return g.transform(ns->{
      mut MyNode n1=this;//single usage of capsule 'this'
      ns.add(n1);
      return n1;
      });
    }
\end{lstlisting}
As you can see, both methods relies on the Transform Pattern.

Those transformation operations are very general since they
can take in input the \Q@mut Nodes@ content of the graph and 
any \Q@capsule@ or \Q@imm@ data from outside.
Note how in \Q@connectWith@ we can not see the \Q@read this@ or the \Q@read other@
inside the lambda, but we need to get their (immutable) indexes 
and recover the concrete objects from the \Q@mut Nodes ns@ object.
In this way, we also obtain more useful \Q@mut@ references to those nodes.
On the other side, note how in \Q@addToGraph@ we can see the reference to \Q@capsule this@ inside the lambda.

Concluding, our approach imposes a very defensive programming style
that is not currently commonly used in programming practice.
But with a little bit of experience using our system, any data structure can be defined.
Any invariant relying only on immutable and encapsulated state can be enforced; and any transformation based on immutable and capsule input can be defined.
