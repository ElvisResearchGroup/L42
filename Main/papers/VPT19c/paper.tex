\PassOptionsToPackage{svgnames}{xcolor}
\documentclass[submission,copyright,creativecommons]{eptcs}
\providecommand{\event}{VPT 2019} % Name of the event you are submitting to
%\usepackage{breakurl}             % Not needed if you use pdflatex only.
\usepackage{underscore}           % Only needed if you use pdflatex.
\usepackage{listings}
\usepackage{xcolor}
\usepackage{letltxmacro}
\usepackage{mathtools}
\usepackage{mathpartir}
%\usepackage{stix}

\definecolor{darkRed}{RGB}{100,0,10}
\definecolor{darkBlue}{RGB}{10,0,100}
\newcommand*{\ttfamilywithbold}{\fontfamily{pcr}\selectfont}
%\newcommand*{\ttfamilywithbold}{\ttfamily}

%found on http://tex.stackexchange.com/questions/4198/emphasize-word-beginning-with-uppercase-letters-in-code-with-lstlisting-package
%\lstset{language=FortyTwo,identifierstyle=\idstyle}
%
\makeatletter
\newcommand*\idstyle{%
        \expandafter\id@style\the\lst@token\relax
}
\def\id@style#1#2\relax{%
        \ifcat#1\relax\else
                \ifnum`#1=\uccode`#1%
                        \ttfamilywithbold\bfseries
                \fi
        \fi
}
\makeatother

\lstset{language=Java,
  basicstyle=\upshape\ttfamily\footnotesize,%\small,%\scriptsize,
  keywordstyle=\upshape\bfseries\color{darkRed},
  showstringspaces=false,
  mathescape=true,
  xleftmargin=0pt,
  xrightmargin=0pt,
  breaklines=false,
  breakatwhitespace=false,
  breakautoindent=false,
 identifierstyle=\idstyle,
 morekeywords={method,Use,This,constructor,as,into,rename},
 deletekeywords={double},
 literate=
  {\%}{{\mbox{\textbf{\%}}}}1
  {~} {$\sim$}1
%  {<}{$\langle$}1
%  {>}{$\rangle$}1
}

\newcommand*{\SavedLstInline}{}
\LetLtxMacro\SavedLstInline\lstinline
\DeclareRobustCommand*{\lstinline}{%
	\ifmmode
	\let\SavedBGroup\bgroup
	\def\bgroup{%
		\let\bgroup\SavedBGroup
		\hbox\bgroup
	}%
	\fi
	\SavedLstInline
}

\newcommand\saveSpace{\vspace{-2pt}}

\newcommand\Rotated[1]{\begin{turn}{90}\begin{minipage}{12em}#1\end{minipage}\end{turn}}

\newcommand{\Q}{\lstinline}
\newenvironment{bnf}{$\begin{aligned}}{\end{aligned}$}
\newcommand{\production}[3]{\textit{#1}&\Coloneqq\textit{#2}&\text{#3}}
\newcommand{\prodNextLine}[2]{&\quad\quad\textit{#1}&\text{#2}}
\newenvironment{defye}{\\\indent$\begin{aligned}}{\end{aligned}$\\}
\newcommand{\defy}[2]{\!\!\!\!\!\!&&#1&\coloneqq#2\\}
%\newcommand{\defyc}[1]{&\phantom{\coloneqq}\ \ #1\\}
\newcommand{\defyc}[1]{\!\!\!\!\!\!\rlap{\quad \quad #1}&&\\}
\newcommand{\defya}[2]{#1&\!\!\!\!\!\!&\coloneqq#2\\}

%\newcommand{\prodFull}[3]{#1&::=&\mbox{#2}&\mbox{#3}}
\newcommand{\prodInline}[2]{#1\Coloneqq#2}
\newcommand{\terminal}[1]{\ensuremath{$\texttt{#1}$}}
%\newcommand{\metavariable}[1]{\ensuremath{\mathit{#1}}}

\newcommand{\Rulename}[1]{{\textsc{#1}}}
\newcommand{\ctx}[1]{\ensuremath{\mathcal{E}_#1}\!}
\newcommand{\libi}[2]{\Q@\{@\Q!interface!\ #1\Q{;} #2\Q@\}@}
\newcommand{\lib}[3]{\Q!interface!\ensuremath{?}\ \libc{#1}{#2}{#3}}
\newcommand{\libc}[3]{\,\Q@\{@\!#1\Q{;}\ #2 \Q{;}\ #3\Q@\}@\!\!}

\newcommand{\rp}[1]{\Q{(}\!#1\Q{)}}
\newcommand{\eq}[1]{\,\Q{=}#1}
\newcommand{\red}[3]{#1\,\Q{<}#2\eq#3\,\Q{>}}
\newcommand{\summ}[2]{#1\ \Q{<+}\ #2}
\newcommand{\from}[2]{#1\ensuremath{[}#2\ensuremath{]}}
\newcommand{\mmid}{{\ensuremath{{\mid}}}\!}
\newcommand{\hole}{\ensuremath{\square}}
\newcommand{\s}[1]{\ensuremath{\mathit{#1s}}}
\makeatletter
\newcommand{\This}[1]{\Q!This!#1\nextpath}
\newcommand{\Cs}[1]{#1\nextpath}
\newcommand{\nextpath}{\@ifnextchar\bgroup{\gobblenextpath}{}}
\newcommand{\gobblenextpath}[1]{\Q!.!#1\@ifnextchar\bgroup{\gobblenextpath}{}}
\makeatother



%--------------------------
\newcommand{\mynotes}[3]{{\color{#2} {\sc #1}: #3}}
\newcommand\isaac[1]{\mynotes{Isaac}{blue}{#1}}

\newcommand\IO[1]{\color{blue}{#1}}
\newcommand\marco[1]{\mynotes{Marco}{green}{#1}}


\lstset{
    numbers=left,xleftmargin=1.8em,
    stepnumber=1,
    showstringspaces=false,
    firstnumber=last
}

\providecommand*{\code}[1]{\Q`#1`}
\newcommand{\saveSpace}{\vspace{-3px}}
\newcommand{\loseSpace}{\vspace{1ex}}
\usepackage{verbatim}

\title{Iteratively Composing Statically Verified Traits}
\def\titlerunning{Iteratively Composing Statically Verified Traits}

%magic code from https://tex.stackexchange.com/questions/344794/centering-issues-with-multiple-authors-with-the-same-affiliation-eptcs-format
\RequirePackage{array}
\newenvironment{authors}[1]%
  {\begingroup
   \gdef\estyle{}%
   \renewcommand\institute[1]%
     {\\\multicolumn{#1}{@{}c@{}}{\scriptsize\begin{tabular}[t]{@{}>{\footnotesize}c@{}}##1\end{tabular}}}%
   \renewcommand\email[1]%
     {\gdef\estyle{\footnotesize\ttfamily}\\##1\gdef\estyle{}}
   \begin{tabular}[t]{@{}*{#1}{>{\estyle}c}@{}}
  }%
  {\end{tabular}%
   \endgroup
  }

\def\anauthor#1#2{%
	#1%
	\institute{}
	\email{#2}%
}
\def\vuw{\institute{School of Engineering and Computer Science\\%
			Victoria University of Wellington\\%
			Wellington, New Zealand}}
\author{
	\begin{authors}{4}
	Isaac Oscar Gariano & Marco Servetto & Alex Potanin & Hrshikesh Arora
	\vuw
	\email{isaac@ecs.vuw.ac.nz & marco.servetto@ecs.vuw.ac.nz & alex@ecs.vuw.ac.nz & arorahrsh@myvuw.ac.nz}
	\end{authors}
}

\def\authorrunning{Isaac O.\,G., M. Servetto, A. Potanin \& H. Arora}

% Allows one to write /hello/ instead of \Q@hello@.
% Use // to get a normal text slash
\chardef\Slash=`\/
\catcode\Slash=\active
\chardef\other=12 % char code for other characters
\def/#1/{%
	\ifx/#1/% #1 is empty
		\Slash% just print a slash
	\else%
		\lstinline/#1/%
	\fi%
}
\let\oldinput=\input
\def\input#1{\oldinput{\detokenize{#1}}} % Don't expand commands in input (needed since / is a command)
\newcommand{\sref}[1]{Section~\ref{s:#1}}


%\definecolor{blue}{HTML}{0000F0} %
%\definecolor{purple}{HTML}{700090}
%\definecolor{orange}{HTML}{F07000}
%\definecolor{teal}{HTML}{0090B0}
%\definecolor{brown}{HTML}{A00000}
%\definecolor{green}{HTML}{008000}
%\definecolor{pink}{HTML}{F000F0}

\makeatletter
\renewcommand*\idstyle{%
	\expandafter\id@style\the\lst@token\relax}
\def\id@style#1#2\relax{%
        \ifcat#1\relax\else
                \ifnum`#1=\uccode`#1%
                        \color{DarkGreen}%
                \fi%
        \fi%
}
\makeatother


\lstset{%
	language=Java, morekeywords={exists, forall, @requires, @ensures, result, rename, hide, with},
	tabsize=2,
	identifierstyle=\idstyle,
%	aboveskip=0pt
%	keywordstyle=\color{blue},
%	commentstyle=\color{green},
%	stringstyle=\color{brown}
%	literate=
%		{||}{{$\vee$}}1
%		{&&}{{$\wedge$}}1
%		{<=}{{$\leq$}}1
%		{>=}{{$\geq$}}1
%		{**}{{$^{**}$}}1
}
%citations
\newcommand{\REV}[3]{%
	\NoteColour{red}{#1\NoteText{\footnote{%
		\textcolor{red}{\textbf{REV#2{:} #3}}}}}}
	
\begin{document}
\maketitle
\begin{abstract}
Static verification relying on an automated theorem prover can be very slow and brittle: since static verification is undecidable, correct code may not pass a particular static verifier.
In this work we use metaprogramming to generate code that is correct by construction.
A theorem prover is used only to verify initial "traits": units of code that can be used to compose bigger programs.

In our work, meta-programming is done by trait composition, which starting from correct code, is guaranteed to produce correct code.
We do this be extending conventional traits with methods pre and post conditions; we also extend the  traditional trait composition (/+/) operator to check the compatibility of contracts. In this way, there is no need to re-verify the produced code.

We show how our approach can be applied to the standard ``power'' function example, where metaprogramming generates optimised, and correct, versions when the exponent is known in advance.
\end{abstract}

%\begin{abstract}
In this paper we use pre existing language support for type modifier and object capability to enable a system for sound runtime verification of invariants.
Our system guarantees that class invariants hold for all objects involved in execution.
Invariants are specified simply as methods whose execution is statically guaranteed to be deterministic and not access any externally mutable state.
We automatically call such invariant methods only when objects are created or the state they refer to may have been mutated.
Our design restricts the range of expressible invariants but improves upon the usability and performance of our system compared to prior work.
In addition, we soundly support mutation, dynamic dispatch, exceptions, and non determinism, while requiring only a modest amount of annotation.

We present a case study showing that our system requires a lower annotation burden compared to Spec\#, and  performs orders of magnitude less runtime invariant checks compared to the widely used `visible state semantics' protocols of D, Eiffel.
We also formalise our approach and prove that such pre existing type modifier and object capability support is sufficient to ensure its soundness.
\end{abstract}

%
\section{Introducing Quasi Quotation}

Lisp~\cite{pitman1980special}, MetaML~\cite{moggi1999idealized}, Template Haskell~\cite{sheard2002template} and many other approaches use Quasi Quotation (QQ).
This can be supported by two kinds of special parenthesis as a syntactic sugar to manipulate Abstract Syntax Trees (ASTs).
Lisp uses (\Q@`@) and (\Q@,@), while here we use
\Q@[|  |]@  and \Q@$\$$(  )@ (as Template Haskell) 
for better readability.

\noindent
The following example explains their meaning: 

\begin{lstlisting}
Int res0=x*x $\Comment{normal code}$
Expr<x:Int$\vdash$Int> res1=[| x*x |]
  $\Comment{new Mul(new Var("x"),new Var("x"))}$
Expr<x:Int$\vdash$Int> res2=[| x* $\$$(12+3) |]
  $\Comment{new Mul(new Var("x"),new Lit(15))}$
\end{lstlisting}

\noindent
Here \Q@res1@ is initialized using a ``quotation'' of code.
This is equivalent to generating the abstract syntax tree by hand, as shown in the comment.
\Q@res2@ is initialized using a ``quasi-quotation'' of code: a chunk of code with a hole, that is filled by executing an expression.

There are different ways to type QQ.
In an expression based language, 
the simplest way is to just have a primitive \Q@Expr@ type,
representing every type of code.
This ensures the result is syntactically well formed, but
it allows for the generation of ill-typed code.
Another option, for example used by MetaML,
is to have a parameterized type.
Here we use \Q@Expr<@$\Gamma\vdash T$\Q@>@; where
$\Gamma$ keeps track of the free variables and $T$ is the expected type
of the result.
This approach is restrictive (see~\cite{servetto2014meta})
 but ensures that all the resulting code is well typed.

Usually programming with QQ requires thinking about the desired method body,
 and often allows generating a more efficient body by generating code specialized for some input value.
A typical example is about generating a \Q@pow@ function, where the exponent is well known.
The ``inefficient'' version would be:

\begin{lstlisting}[language=ML]
fun power(x:Int,n:Int):Int 
  = if (n=0) then 1 else x*power(x,n-1);
power7_a=$\lambda$ x:Int. power x 7;
\end{lstlisting}

\noindent A more ``efficient'' version using QQ would be:

\begin{lstlisting}[language=ML]
fun powerAux(n:Int):Expr<x:Int$\vdash$Int> 
  = if (n=0) then [|1|] else [|x * $\$$(powerAux(n-1)) |];

fun powerGen(n:Int): Int->Int
  = compile([| $\lambda$ x. $\$$(powerAux(n)) |]);

power7_b=powerGen 7;
\end{lstlisting}

\noindent As you can see, by generating the abstract syntax tree, we can obtain exactly:

\begin{lstlisting}[language=ML]
power7_b=$\lambda$ x.x*x*x*x*x*x*x*1;
\end{lstlisting}

\noindent On most machines, \Q@power7_b@ runs faster than \Q@power7_a@.
Metaprogramming applications include more than just speed boosts, but we start with this example because it is very popular and simple.

The code generator above is quite compact, but it is actually \textbf{hiding} (not removing) the complexity of meta-programming.
A common approach to make the code more explicit is to extract
logical concepts as functions.
We can see that the code is proceeding in an inductive fashion:
we know the code for \Q@pow 0@, and given the code for
\Q@pow n@  we can create the code for \Q@pow (n+1)@.
Thus we define \Q@base@ and \Q@inductive@ functions, and we
use them inside \Q@powerAux@:

\begin{lstlisting}[language=ML]
fun base():Expr< x:Int$\vdash$Int > 
  = [| 1 |]
fun inductive(code:Expr< x:Int$\vdash$Int >):Expr< x:Int$\vdash$Int >
  = [| x * $\$$(code) |]

fun powerAux(n:Int):Expr< x:Int$\vdash$Int >
  = if (n=0) then base()
   else inductive ( powerAux(n-1) );
\end{lstlisting}

\noindent Then, we have to bind \Q@x:Int@ to a parameter in a function.
This is an important conceptual action and thus we make it a function:

\begin{lstlisting}[language=ML]
fun lambdaX(code:Expr< x:Int$\vdash$Int >):Expr<$\vdash$Int->Int > 
  = [| $\lambda$ x. $\$$( code ) |]

fun powerGen(n: Int):Int->Int
  = compile(lambdaX(powerAux(n) ))
\end{lstlisting}

The code we obtain is much larger, but is not logically more complex --- it is just showing the logical structure better.
Note how since QQ works near the code representation,
a function \Q@Int->Int@ is radically different from
code with a free variable \Q@x:Int@$\vdash$\Q@Int@, while they are 
logically similar concepts.


We propose Iterative Composition (IC):
while the unit of composition in QQ is the single AST node, 
IC enforces a higher level of abstraction and does not work directly on the AST.
The unit of composition in IC is a \emph{Library}:
a class body, containing methods and possibly nested classes.
Libraries are self contained in the sense that they contain no free variables.
This avoids all scope-extrusion related problems, and (as shown later) enforces local reasoning.

IC has already been presented in other work~\cite{servetto2014meta};
 IC's expressive power is shown by examples,
but is not compared with QQ; moreover such works suggested IC's expressive power is inferior to QQ.
The core idea of IC is to  \emph{rely on  operators of code composition inspired by normal
code reuse}, but lifted to the expression level.
As a concrete example, in Java operators \Q@+@ and \Q@*@ can be used in the expression \Q@1+2*3@,
but the operator \Q@extends@ can only be used in the specific context of class declaration, as in

\begin{lstlisting}[language=Java]
class A extends B{/*some code*/ int m(){return 1+super.m();}}
\end{lstlisting}

In our proposed approach, we lift \Q@extends@ and code literals to operator and constants
that can be used in conventional expressions.
Class declarations associate a class name with the result of an expression of type \Q@Library@. 
 We would write the former example as:

\begin{lstlisting}
A = Override[m()<-superM()]( 
  {/*code of B*/},
  { /*some code*/ int m(){return 1+this.superM();}}
  )
\end{lstlisting}

\noindent We support the conventional super call mechanism by annotating the operator with
the expected super call name: \Q@Override[m()<-superM()](...)@.


\noindent
We can rewrite our \Q@pow@ example 
in IC:

\begin{lstlisting}
Pow = {
  static method Library base()
   ={ method Num pow(Num x)= 1 }$\Comment{Code literal with 1 method "pow(x)"}$

  static method Library inductive()
   ={$\Comment{Code literal with 2 methods: "pow(x)", "superPow(x)"}$
    method Num pow(Num x)= x*this.superPow(x)
    method Num superPow(Num x)$\Comment{no body: it is an abstract method}$
    }
  static method Library inductive(Library code)
   = Override[pow(x)<-superPow(x)](code, this.inductive())
  
  static method Library generate(Num y)
   = if (y==0) then this.base();
     else this.inductive(generate(y-1))
  }
...
Pow7 = Pow.generate(7)
$\Comment{That would reduce into the desired code as follows:}$
Pow7 ={method Num pow(Num x)=x.x*x*x*x*x*x*x*1}
\end{lstlisting}

\noindent In more detail:
\begin{itemize}
\item
\Q@base()@ is a method with no parameter and a \Q@Library@ return type.
This is equivalent to a non-parameterized version of \Q@Expr@ in QQ.
However, our approach still guarantees that all the results are well typed.
\Q@base()@ returns a class with a single method \Q@pow(x)@,
returning 1.
\item
For the inductive case, the method \Q@pow(x)@ of \Q@inductive()@ is defined in terms of
another method (\Q@superPow(x)@), representing the delegation to
the former case in the inductive reasoning.
%Note that the declaration of \Q@superPow(x)@ is an abstract method: a method without body.
\item Method \Q@inductive(code)@ builds
the code for \Q@x+1@ from the code for \Q@x@.
Note how we use \Q@Override@ inside of a normal method body.
This \Q@Override@ will implement
\Q@superPow(x)@ using
the \Q@pow(x)@ body from the induction premise: the \Q@code@ parameter.
Then, \Q@superPow(x)@ is inlined.

\item Method \Q@generate(y)@ uses recursion to \textbf{iteratively compose} the result, using induction starting from
the base case.
Note how this method is logically identical to \Q@powerAux@. However,
since we always work on the actual self contained code neither \Q@lambdaX@ nor \Q@compile@ are needed.
\end{itemize}
Our approach builds on top of code composition operations like multiple inheritance and generics.
The literature offers \cite{barnett2004spec,burdy2005overview,muller2016viper} many successful efforts about proving the
\emph{semantic correctness} 
of code containing inheritance and generics.
On the other hand, static verification of code generated with meta-programming is an open research problem.
We speculate our approach may offer the opportunity to solve this problem,
and by construction generate statically verified code
by reusing techniques originally developed to verify normal object oriented code.
The contributions of our work are as follows:
\begin{itemize}
\item
In Section~\ref{s:verification},
by examples we demonstrate how to apply conventional object oriented verification techniques to IC.

\item
In Section~\ref{s:pattern1},
by examples we show that IC is as expressive as QQ, and that
generating code using a composition algebra
is a flexible and simple technique, if combined with
reasonable programming patterns.
\item In Section~\ref{s:study}, we discuss our experience implementing metaprogramming libraries for a language where IC is the only support for code reuse. 
\end{itemize}

In this paper we do not present a formal language semantic. This is partly due to space reasons
and partly because a similar semantic has been formalized in former work~\cite{servetto2014meta}.
Here we aim to show programming patterns that use this expressive power in surprising and novel ways.


%-------------------------------------------------------------------------------------
%-------------------------------------------------------------------------------------


%\section{Background}
\noindent Object oriented languages supporting static verification (SV) usually extend the syntax for method declarations
to support \emph{contracts} in the form of pre and post-conditions~\cite{Meyer:1988:OSC:534929}.
Correctness is defined only for code annotated with such contracts.

We say that a method is \emph{correct}, if whenever its precondition holds on entry, the precondition of every directly invoked method holds, and the postcondition of the method holds when the method returns. Automated SV typically works by asking an automated theorem prover to verify that each method is correct individually, by assuming the correctness of every other method~\cite{barnett2004spec}. This process can be very slow and can produce unexpected results: since SV is undecidable correct code may not pass SV.
Many SV approaches are not resilient to
\MSDel{some} standard refactoring techniques like 
method inlining. Sometimes SV even \IO{times out, making it harder to use such refactoring techniques.} \MSDel{terminate for a time-out, exacerbating the impact of transformations like method inlining.}

Metaprogramming is often used to programmatically generate faster specialised code when some parameters are known in advance, this is particularly useful where the specialisation mechanism is too complicated for a generic compiler to automatically derive~\cite{Ofenbeck:2017:SGP:3136040.3136060}
We could use metaprogramming to generate code together with contracts, and then once the metaprogramming has been run,
 \MSDel{ensure the correctness of} \IO{SV} the resulting code \MSDel{by applying SV.}. \IOComm{You don't `apply static verification', rather you `statically verify', you could also `use a static verifier'.}
However, the resulting code could be much larger than the input to the metaprogramming, and so it could take a long time to SV.
Moreover, one of the \MSDel{main} \IO{many} goal\IO{s} of metaprogramming is \IO{to} make it \IO{easier} \MSDel{easy} to generate \MSDel{many specialized} \IO{specialised} versions of the same \MSDel{functionality} \IO{code}. \IOComm{A `version of functionality' doesn't really make sense, a `version of code' does, we could also use `function//method'.}
%, Even if the generated code was produced by using straightforward transformations and compositions over the input code, a SV might not verify it's correctness.
The aim of our work is to \MSDel{apply} SV only \MSDel{to the code manually wrote by the programmer} \IO{code written directly, and not code produced by metaprogramming;}
\IO{instead, we}
\MSDel{and to} ensure that the result of metaprogramming is \MSDel{instead} correct by construction.

Here we use the disciplined form of metaprogramming introduced by Servetto \& Zucca \cite{servetto2014meta}, which is based on trait composition and adaptation~\cite{scharli2003traits}.
Here a /Trait/ is a unit of code: a set of method declarations.
\IO{Such methods can be abstract and be}
\MSDel{Those methods can be abstract, and they can}
be mutually recursive by using the implicit parameter /this/.

As in~\cite{servetto2014meta}
we require that all the traits are well-typed
before they are used.
\MSDel{Moreover, in our proposed approach we}
\IO{we extend this by allowing}
\MSDel{annotating} methods \IO{to be annotated} with pre//post-conditions, and 
\IO{ensuring that traits are correct in terms of such contracts}
\MSDel{we require that all traits are also correct}.
/Trait/s directly written in the source code are \IO{SVed} \MSDel{proven correct by SV}, while traits resulting from metaprogramming are \IO{ensured} correct by \IO{only providing trait operations that preserve correctness} \MSDel{construction}. \IOComm{SV proves correctness (that's the whole point), so no need to say 'proven correct by SV'. It was not clear how we `ensure correctness by construction', so I fixed that.}
\MSDel{Crucially}
\IO{In particular}, we extend the checking performed by \IO{the traditional trait composition (/+/)  operator, to also check the compatibility of contracts} \MSDel{composition and adaptation of /Trait/s to also check that contracts are composed correctly; thus ensuring the correctness of the result}. \IOComm{we only extend the /+/ operator here, so I've made that explicit.}
\IOComm{The above paragraph of changes are trying to make it more clear what our contribution is, but even then it's not good enough}.

%
%The result of composing and adapting /Trait/s is also correct and well-typed.
%
Our metaprogramming approach does not \IO{allow generating} \IODel{generate} code from scratch \IO{(such as by generating ASTs), rather the language provides a specific set of primitive composition and adapation operators which preserve correctness}.
\MSDel{; rather code is only generated by composing and adapting traits.
Each composition//adaptation step is guaranteed
to produce well typed and correct code; thus also the result of metaprogramming is well typed and correct.} \IOComm{I reworded it to make it sound like you are forced to use our safe operators, and can't subvert the system, since this makes our `guarantees` meaningful.}
Note that generated code may not be able to pass \IO{a particular} SVer, since theorem provers are not complete. \IOComm{In principle, since our code is correct, it should be able to pass some form of `static verification', however that dosn't mean it will base every `static verifier'.}


SV handles /extends/ and /implements/ by verifying that every 
time a method is implemented//overriden, 
the Liskov substitution principle~\cite{Liskov:1994:BNS:197320.197383} is satisfied
by checking that the \MSDel{new contract} \IO{contract of the override//implementation} implies the \IO{contract of the overriden//implemented method} \MSDel{overridden one}. 
\IOComm{In your version, it was not clear what contracts you were refering to.}
 In this way, there is no need to re-verify
inherited code in the context of the derived class.
This concept is easily adapted
to handle trait composition, which simply provides another way to implement an /abstract/ method.
When traits are composed,
it is sufficient
to match the contracts of the few composed methods
to ensure the whole result is correct.

In our examples we will use the notation /@requires($predicate$)/ 
to specify a precondition, and /@ensures($predicate$)/ 
to specify a postcondition; where $predicate$ is a boolean expression
in terms of the parameters of the method (including /this/), and for the /@ensures/ case, the /result/ of the method.
Suppose we want to implement an efficient exponentiation function, we could use recursion and the common technique of `repeated squaring':
\vspace{-1ex}
\begin{lstlisting}
@requires(exp > 0)
@ensures(result == x**exp) // Here x**y means x to the power of y
Int pow(Int x, Int exp) {
	if (exp == 1) return x;
	if (exp %2 == 0) return pow(x*x, exp/2); // exp is even
	return x*pow(x, exp-1); }  // exp is odd
\end{lstlisting}
If the exponent is known at compile time,
unfolding the recursion produces even more efficient code:
\vspace{-1ex}
\begin{lstlisting}
@ensures(result == x**7) Int pow7(Int x) { 
  Int x2 = x*x; // x**2
  Int x4 = x2*x2; // x**4
  return x*x2*x4; } // Since 7 = 1 + 2 + 4
\end{lstlisting}
\vspace{-1ex}


Now we show how the technique of \emph{Iterative Composition} (introduced in~\cite{servetto2014meta} and
enriched by \MSDel{our contract composition check} \IO{the contract compatibility check we propose performing in trait composition}) \IOComm{We never call `a contract composition check`, so it's not clear what you are talking about, since I already mentioned a `contract compatibility check above', I'm referencing it here}
can be used to write a metaprogram that given an exponent, produces code like the above.
Iterative Composition is a metacircular metaprogramming technique relying on \emph{compile-time execution} (as \IO{defined by}~\cite{sheard2002template}), \IOComm{I'm not sure what `as [10]' was meant to mean, correct me if my guess was wrong.} 
\MSDel{thus a metaprogram is just a function or a method wrote in the target programming language that is executed during compilation.}
\IO{, in our context this means that arbitrary expressions can be used as the RHS of a class declaration, during compilation such expressions will be evaluated to produce a /Trait/, which provides the body of the class. In this way metaprograms can be represented as otherwise normal functions//methods that return a /Trait/, without requiring the use of any additional `meta language'.} \IOComm{Major rewording, as your version didn't explain anything, in particular it was not clear what you meant by `during compilation'}.
 

\vspace{-1ex}
\begin{lstlisting}
Trait base=class {//induction base case: pow(x) == x**1
  @ensures(result>0) Int exp(){return 1;}  
  @ensures(result==x**exp()) Int pow(Int x){return x;}
  }
Trait even=class {//if _pow(x)== x**_exp(), pow(x) == x**(2*_exp())
  @ensures(result>0) Int $\_$exp();
  @ensures(result==2*$\_$exp()) Int exp(){return 2*$\_$exp();}
  @ensures(result==x**$\_$exp()) Int $\_$pow(Int x);
  @ensures(result==x**exp()) Int pow(Int x){return $\_$pow(x*x);}
}
Trait odd=class {//if _pow(x)== x**_exp(), pow(x) == x**(1+_exp())
  @ensures(result>0) Int $\_$exp();
  @ensures(result==1+$\_$exp()) Int exp(){return 1+$\_$exp();}
  @ensures(result==x**$\_$exp()) Int $\_$pow(Int x);
  @ensures(result==x**exp()) Int pow(Int x){return x*$\_$pow(x);}
}
//`compose' performs a step of iterative composition
Trait compose(Trait current, Trait next){
  current = current[rename exp->$\_$exp, pow->$\_$pow];
  return (current+next)[hide $\_$exp, $\_$pow];}
@requires(exp>0)//the entry point for our metaprogramming
Trait generate(Int exp) {
  if (exp==1) return base;
  if (exp%2==0) return compose(generate(exp/2),even);
  return compose(generate(exp-1),odd);
};
class Pow7: generate(7) //generate(7) is executed at compile time
//the body of class Pow7 is the result of generate(7)
/*example usage:*/new Pow7().pow(3)==2187//Compute 3**7
\end{lstlisting}
\vspace{-1ex}
\subsection{Trait Composition}\label{s:btc}
The idea behind \emph{trait composition} is to seperate the notions of \emph{subtyping} and \emph{inheritence}~\cite{useReuse}. Only \emph{traits} can be inherited, and only \emph{classes} can be used as types. A \emph{trait} is like an anonymous class, in particular it can contain methods, but because it is anonymous, one cannot directly use it as a type or create instances of it. Class are then declared from traits, effectively giving them a name, but such classes will be unrelated to other classes defined from the same trait. Traits can be `composed` to produce other traits, however since traits cannot be used as types, we are free to perform operations that would break the LSV. Consider the following example:
\begin{lstlisting}
Trait pair = class {
	String target();
	String hello() { return "Hello " + this.target(); }
};
// Defines a class called HelloWorld
HelloWorld: pair.extend(class {
	String target() { return "World"; }
	String hello() { this.super_print() + "!"; } // prints "Hello World!"
});
\end{lstlisting}
We use the type /Trait/ as the types of traits, and a class declaration (without a name) as a \emph{trait literal} expression. Here /trait1.extend(trait2)/ contains all the methods in /trait1/ and /trait2/. When a method declaration appears in both /trait1/ and /trait2/, their signatures will also be checked to ensure they are identical, this ensures that if both /trait1/ and /trait2/ are well-typed, the result will also be (this property is sometimes called \emph{meta level soudnes}~\cite{Servetto:2010:MMC:1869459.1869498}).

However if a method $m$ is also \emph{implemented} (i.e. non abstract) in both /trait1/ and /trait2/, the operation will proceed as if the version in /trait1/ was named /super_$m$/. Unlike conventional OO inheritance, this means that calls to $m$ in /trait1/ will call the implementation in /trait1/, and not the (overriding) implementation in /trait2/. In addition, the renamed /super_$m$/ method will be `private' to /trait2/, i.e. only code in /trait2/ itself can see it, not code added in later. Their are other approaches to dealing with this, such as only renaming the declaration of $m$ to /super_$m$/, so that calls in /trait1/ will call the method in /trait2/. However in order for this to be sound in our context of producing correct code, we would have to prevent the $m$ in /trait2/ from having an incompatible contract to the $m$ in /trait1/, thus breaking our examples in \sref{ex}.

In this paper, we will use the \emph{flattening} semantics of traits, in which trait operations produce a flattened result, equivalent to a trait literal, with no references to any of the input traits to the operation. Thus the above definition for /HelloWorld/ is identical to:
\begin{lstlisting}
HelloWorld: class {
	private String super_hello() { return "Hello " + this.target(); }
	String target() { return "World"; }
	String print() { this.super_print() + "!"; }
};
\end{lstlisting}

The main advantage of trait composition over conventional inheritance is that trait composition does \emph{not} induce subtyping:
\begin{itemize}
	\item Trait operations that would be unsound in a subtyping environment are allowed, such as removing methods or changing their types.
	\item The way code is reused in order to construct a class is not exposed to users of the class. This allows the code reuse mechanism to be changed without breaking any uses of a class.
\end{itemize}
In addition, this makes traits much simpler to reason about, making it easier to compose them in more complicated ways.
\subsection{Iterative Composition}
Iterative composition~\cite{?} takes the concept of trait composition one step forward, by treating traits as first class objects, and allowing arbitrary code to execute at meta-time. This allows for fully met-circular meta-programming. In this way, classes can be declared from arbitrary expressions (of type /Trait/), and methods can take and return /Trait/s. For example, suppose we want to generate specialised code for /pow/ based on the exponent\footnote{We use /static/ /class/ to mean a class with no constructors or instances, and where each member is implicitly /static/.}:
\begin{lstlisting}
Trait pow_generate(Nat exp) {
  if (exp == 0)
    return static class { 
      Int pow(Int x) { return 1; }
    };
  else if (exp == 1)
    return static class {
       Int pow(Int x) { return x; }
     };
 else if (exp % 2 == 0)
   return pow_generate(exp/2).extend(static class {
     Int super_pow(Int x);
     Int pow(Int x) { return super_pow(x*x); }
   });
 else 
   return pow_generate(exp - 1).extend(static class {
     Int super_pow(Int x);
     Int pow(Int x) { return x*super_pow(x); }
   });}
\end{lstlisting}
In the above code, we have a normal looking recursive /pow_generate/ method: the /pow_generate(...)/ calls (recursively) generate the bodies of /super_pow/. The calls to /super_pow/ (in the trait literals) correspond to the recursive calls in our original /pow/; . The above method can now be called to generate the body of a class:

\begin{lstlisting}
Pow7: pow_generate(7);

// Equivalent to:
Pow7: static class {
  Int pow(Int x) { return x*super1_pow(x); } // x**7

  private Int super1_pow(Int x) { return super2_pow(x*x); } // x**6
  private Int super2_pow(Int x) { return x*super3_pow(x); } // x**3
  private Int super3_pow(Int x) { return super4_pow(x*x); } // x**2
  private Int super4_pow(Int x) { return x; } // x**1
}

// Which could then be inlined/optimised to:
Pow7: static class {
  Int pow(Int x) { 
    Int x2 = x*x; // x**2
    Int x4 = x2*x2; // x**4
    return x*x2*x4; } // Since 7 = 1 + 2 + 4
}
\end{lstlisting}

\noindent Object oriented languages supporting static verification (SV) usually extend the syntax for method declarations
to support \emph{contracts} in the form of pre and post-conditions~\cite{Meyer:1988:OSC:534929}.
Correctness is defined only for code annotated with such contracts.

We say that a method is \emph{correct}, if whenever its precondition holds on entry, the precondition of every directly invoked method holds, and the postcondition of the method holds when the method returns. Automated SV typically works by asking an automated theorem prover to verify that each method is correct individually, by assuming the correctness of every other method~\cite{barnett2004spec}. This process can be very slow and can produce unexpected results: since SV is undecidable correct code may not pass SV.
Many SV approaches are not resilient to
\MSDel{some} standard refactoring techniques like 
method inlining. Sometimes SV even \IO{times out, making it harder to use such refactoring techniques.} \MSDel{terminate for a time-out, exacerbating the impact of transformations like method inlining.}

Metaprogramming is often used to programmatically generate faster specialised code when some parameters are known in advance, this is particularly useful where the specialisation mechanism is too complicated for a generic compiler to automatically derive~\cite{Ofenbeck:2017:SGP:3136040.3136060}
We could use metaprogramming to generate code together with contracts, and then once the metaprogramming has been run,
 \MSDel{ensure the correctness of} \IO{SV} the resulting code \MSDel{by applying SV.}. \IOComm{You don't `apply static verification', rather you `statically verify', you could also `use a static verifier'.}
However, the resulting code could be much larger than the input to the metaprogramming, and so it could take a long time to SV.
Moreover, one of the \MSDel{main} \IO{many} goal\IO{s} of metaprogramming is \IO{to} make it \IO{easier} \MSDel{easy} to generate \MSDel{many specialized} \IO{specialised} versions of the same \MSDel{functionality} \IO{code}. \IOComm{A `version of functionality' doesn't really make sense, a `version of code' does, we could also use `function//method'.}
%, Even if the generated code was produced by using straightforward transformations and compositions over the input code, a SV might not verify it's correctness.
The aim of our work is to \MSDel{apply} SV only \MSDel{to the code manually wrote by the programmer} \IO{code written directly, and not code produced by metaprogramming;}
\IO{instead, we}
\MSDel{and to} ensure that the result of metaprogramming is \MSDel{instead} correct by construction.

Here we use the disciplined form of metaprogramming introduced by Servetto \& Zucca \cite{servetto2014meta}, which is based on trait composition and adaptation~\cite{scharli2003traits}.
Here a /Trait/ is a unit of code: a set of method declarations.
\IO{Such methods can be abstract and be}
\MSDel{Those methods can be abstract, and they can}
be mutually recursive by using the implicit parameter /this/.

As in~\cite{servetto2014meta}
we require that all the traits are well-typed
before they are used.
\MSDel{Moreover, in our proposed approach we}
\IO{we extend this by allowing}
\MSDel{annotating} methods \IO{to be annotated} with pre//post-conditions, and 
\IO{ensuring that traits are correct in terms of such contracts}
\MSDel{we require that all traits are also correct}.
/Trait/s directly written in the source code are \IO{SVed} \MSDel{proven correct by SV}, while traits resulting from metaprogramming are \IO{ensured} correct by \IO{only providing trait operations that preserve correctness} \MSDel{construction}. \IOComm{SV proves correctness (that's the whole point), so no need to say 'proven correct by SV'. It was not clear how we `ensure correctness by construction', so I fixed that.}
\MSDel{Crucially}
\IO{In particular}, we extend the checking performed by \IO{the traditional trait composition (/+/)  operator, to also check the compatibility of contracts} \MSDel{composition and adaptation of /Trait/s to also check that contracts are composed correctly; thus ensuring the correctness of the result}. \IOComm{we only extend the /+/ operator here, so I've made that explicit.}
\IOComm{The above paragraph of changes are trying to make it more clear what our contribution is, but even then it's not good enough}.

%
%The result of composing and adapting /Trait/s is also correct and well-typed.
%
Our metaprogramming approach does not \IO{allow generating} \IODel{generate} code from scratch \IO{(such as by generating ASTs), rather the language provides a specific set of primitive composition and adapation operators which preserve correctness}.
\MSDel{; rather code is only generated by composing and adapting traits.
Each composition//adaptation step is guaranteed
to produce well typed and correct code; thus also the result of metaprogramming is well typed and correct.} \IOComm{I reworded it to make it sound like you are forced to use our safe operators, and can't subvert the system, since this makes our `guarantees` meaningful.}
Note that generated code may not be able to pass \IO{a particular} SVer, since theorem provers are not complete. \IOComm{In principle, since our code is correct, it should be able to pass some form of `static verification', however that dosn't mean it will base every `static verifier'.}


SV handles /extends/ and /implements/ by verifying that every 
time a method is implemented//overriden, 
the Liskov substitution principle~\cite{Liskov:1994:BNS:197320.197383} is satisfied
by checking that the \MSDel{new contract} \IO{contract of the override//implementation} implies the \IO{contract of the overriden//implemented method} \MSDel{overridden one}. 
\IOComm{In your version, it was not clear what contracts you were refering to.}
 In this way, there is no need to re-verify
inherited code in the context of the derived class.
This concept is easily adapted
to handle trait composition, which simply provides another way to implement an /abstract/ method.
When traits are composed,
it is sufficient
to match the contracts of the few composed methods
to ensure the whole result is correct.

In our examples we will use the notation /@requires($predicate$)/ 
to specify a precondition, and /@ensures($predicate$)/ 
to specify a postcondition; where $predicate$ is a boolean expression
in terms of the parameters of the method (including /this/), and for the /@ensures/ case, the /result/ of the method.
Suppose we want to implement an efficient exponentiation function, we could use recursion and the common technique of `repeated squaring':
\vspace{-1ex}
\begin{lstlisting}
@requires(exp > 0)
@ensures(result == x**exp) // Here x**y means x to the power of y
Int pow(Int x, Int exp) {
	if (exp == 1) return x;
	if (exp %2 == 0) return pow(x*x, exp/2); // exp is even
	return x*pow(x, exp-1); }  // exp is odd
\end{lstlisting}
If the exponent is known at compile time,
unfolding the recursion produces even more efficient code:
\vspace{-1ex}
\begin{lstlisting}
@ensures(result == x**7) Int pow7(Int x) { 
  Int x2 = x*x; // x**2
  Int x4 = x2*x2; // x**4
  return x*x2*x4; } // Since 7 = 1 + 2 + 4
\end{lstlisting}
\vspace{-1ex}


Now we show how the technique of \emph{Iterative Composition} (introduced in~\cite{servetto2014meta} and
enriched by \MSDel{our contract composition check} \IO{the contract compatibility check we propose performing in trait composition}) \IOComm{We never call `a contract composition check`, so it's not clear what you are talking about, since I already mentioned a `contract compatibility check above', I'm referencing it here}
can be used to write a metaprogram that given an exponent, produces code like the above.
Iterative Composition is a metacircular metaprogramming technique relying on \emph{compile-time execution} (as \IO{defined by}~\cite{sheard2002template}), \IOComm{I'm not sure what `as [10]' was meant to mean, correct me if my guess was wrong.} 
\MSDel{thus a metaprogram is just a function or a method wrote in the target programming language that is executed during compilation.}
\IO{, in our context this means that arbitrary expressions can be used as the RHS of a class declaration, during compilation such expressions will be evaluated to produce a /Trait/, which provides the body of the class. In this way metaprograms can be represented as otherwise normal functions//methods that return a /Trait/, without requiring the use of any additional `meta language'.} \IOComm{Major rewording, as your version didn't explain anything, in particular it was not clear what you meant by `during compilation'}.
 

\vspace{-1ex}
\begin{lstlisting}
Trait base=class {//induction base case: pow(x) == x**1
  @ensures(result>0) Int exp(){return 1;}  
  @ensures(result==x**exp()) Int pow(Int x){return x;}
  }
Trait even=class {//if _pow(x)== x**_exp(), pow(x) == x**(2*_exp())
  @ensures(result>0) Int $\_$exp();
  @ensures(result==2*$\_$exp()) Int exp(){return 2*$\_$exp();}
  @ensures(result==x**$\_$exp()) Int $\_$pow(Int x);
  @ensures(result==x**exp()) Int pow(Int x){return $\_$pow(x*x);}
}
Trait odd=class {//if _pow(x)== x**_exp(), pow(x) == x**(1+_exp())
  @ensures(result>0) Int $\_$exp();
  @ensures(result==1+$\_$exp()) Int exp(){return 1+$\_$exp();}
  @ensures(result==x**$\_$exp()) Int $\_$pow(Int x);
  @ensures(result==x**exp()) Int pow(Int x){return x*$\_$pow(x);}
}
//`compose' performs a step of iterative composition
Trait compose(Trait current, Trait next){
  current = current[rename exp->$\_$exp, pow->$\_$pow];
  return (current+next)[hide $\_$exp, $\_$pow];}
@requires(exp>0)//the entry point for our metaprogramming
Trait generate(Int exp) {
  if (exp==1) return base;
  if (exp%2==0) return compose(generate(exp/2),even);
  return compose(generate(exp-1),odd);
};
class Pow7: generate(7) //generate(7) is executed at compile time
//the body of class Pow7 is the result of generate(7)
/*example usage:*/new Pow7().pow(3)==2187//Compute 3**7
\end{lstlisting}
\vspace{-1ex}


%The /+/ operator is the main way to compose traits%
%~\cite{scharli2003traits,LagorioSZ09}.
%The result of /+/ will contain all the methods from both operands. 

%Crucially, it is possible to sum traits where a method is declared in both operands; in this case at least one of the two competing methods needs to be abstract, and the signatures of the two competing methods need to be \emph{compatible}.
Then, the operator /+/ is used to compose the code of the parameters.
Here we show how we ensure that the traditional /+/ operator also handles contracts: we require that the contract annotations of the two competing methods are \emph{compatible}.
In this paper, we just require them to be syntactically identical. Relaxing this constraint is an important future work.
Thanks to this constraint \textbf{the sum operator also preserves code correctness}. %\IO{There are many variations of the /+/ operator, in particular, we could easily extend our contract matching to work with an nary operator}.

The sum is executed when the method /compose/
%\IO{\footnote{\IO{a generic implementation of this method that renames and hides conflicting methods has been implemented L42~\cite{l42}}}}
runs: if the matched contracts are not identical an exception will be raised. A leaked exception during compile-time metaprogramming would become a compile-time error. 
Our approach is very similar to~\cite{servetto2014meta} and does not guarantee the success of the code generation process, rather it guarantees that if it succeeds, correct code is generated.

Executing /compose(base,even)/ or /compose(base,odd)/ will pass this test: since the contract of /base.pow()/
is the same of /even.$\_$pow()/ and /even.$\_$pow()/, and the same for /exp()/.

Finally the /$\_$pow(x)/ and /$\_$exp()/ method are hidden, so that the structural shape of the result is
the same as /base/'s.
Note that this structural equality includes the contracts of methods.

Note that /Trait/s are first class values and can be manipulated with a set of primitive operators that preserve code correctness and well-typedness.
In this way, by inductive reasoning, we can start from the /base/ case and then recursively compose /even/ and /odd/ until we get the desired code.
Note how the code of /generate(exp)/ follows the same scheme of the code of /pow(x,exp)/ in line 1.

To understand our example better, imagine executing the code of /generate(7)/ while keeping /compose/ in symbolic form. We would get the following (where /c/ is short for /compose/):
\vspace{-1ex}
\begin{lstlisting}[numbers=none]
generate(7) == c(generate(6),odd) == ...
 == c(c(c(c(base,even),odd),even),odd)
\end{lstlisting}
\vspace{-1ex}
As /base/ represents /pow1(x)/; /c(base,even)/ represents /pow2(x)/. Then \Q@c(/*pow2(x)*/,odd)@ represents \Q@pow3(x)@, \Q@c(/*pow3(x)*/,even)@ represents \Q@pow6(x)@, and finally,
\Q@c(/*pow6(x)*/,odd)@ represents \Q@pow7(x)@.
The code of each /$\_$pow(x)/ method is only executed once for each top-level /pow(x)/ call, so the /hide/ operator can inline them.
Thus, the result could be identical to the manually optimized code in line 7.
We can use our /generate(7)/ as follows:
\begin{lstlisting}
class Pow7: generate(7)//generate is executed at compile time
//the body of class Pow7 is the result of generate(7)
/*example usage:*/
new Pow7().pow(3)==2187//Compute 3**7
\end{lstlisting}

%\IO{We are investigation how an additional check can be performed to ensure the resulting code has specific contracts. However, our approach does guarantee that the result will be correct according to whatever contracts it contains.} 
\section{Future Work}
Our approach, as presented in this short paper, only guarantees that code resulting from metaprogramming follows its own contracts, it does
not statically ensure what those contracts may be. As future work, we are investigating how the resulting contracts can be ensured to have a particular meaning or form.
To do so, we need to allow assertions on the contracts of /Trait/s to be used within pre//post conditions.
For example we could allow post conditions like\\*
%\begin{lstlisting}[numbers=none]
/@ensures(result.$\mathit{methName}$.ensures ==\ $\mathit{predicate}$)/ \\*
%\end{lstlisting}
to mean that the resulting /Trait/ has
a method
called $\mathit{methName}$, whose /@ensures/ clause is syntactically identical to  /$predicate$/; whilst
\\*
/@ensures(result.$\mathit{methName}$.ensures ==>\ $\mathit{predicate}$)/
\\*
would use a static verifier to ensure that $\mathit{methName}$'s /@ensures/ clause logically implies $\mathit{predicate}$.
With these two features we could annotate the method /generate(exp)/ in line 32 above as:
\begin{lstlisting}
@requires(exp>0)
@ensures(result.exp().ensures ==> (result==exp))
@ensures(result.pow(x).ensures == (result==x**exp()))
Trait generate(Int exp) {...}
\end{lstlisting}

\vspace{-1ex}
In this way, we could statically verify the /generate(exp)/ method, however we fear such verification will be too complex or impractical. 
We could instead automatically check the above postconditions after each call to /generate(exp)/. If /generate(exp)/ is used to define a class (such as /Pow7/ above), we will guarantee that such class has the expected contracts, before it is used. Thus
there is no need to ensure the correctness of the metaprogram itself: such runtime checks are sufficient to ensure that after compilation, the code produced by metaprogramming has its expected behaviour.
%\IODel{In this case we could defer those difficult//novel predicates to run-time checks, without losing much safety:
%Iterative Composition execute metaprogramming code at
%compile time, thus even run-time verification of metaprograms would happen at compile time. This consideration could result in a crucial design decision: code performing metaprogramming does not need to be verified by SV to produce code annotated with the desired contracts; it may be sufficient to apply some type of runtime verification during compile-time execution.} \IOComm{I did a major rewording since we actually have multiple compile-times and run-times, so your version is confusing, hopefully my version makes the point more clear.}
%For example, the following code:
%\vspace{-1ex}
%\begin{lstlisting}[numbers=none]
%@ensures(new Pow7().exp()==7&&Pow7.pow.ensures=="result==x**exp()")
%class Pow7: generate(7)
%\end{lstlisting}
%\vspace{-1ex}
%may require the static verifier to check that the execution of
%/new Pow7().exp()/ will deterministically reduce to /7/, and that the /ensures/ clause of 
%/Pow7.pow/ is syntactically equivalent to 
%/result==x**exp()/. Note how this final step of static verification does not need to re-verify the body of
%/Pow7.pow/ and only needs to do a coarse grained 
%determinism check on the implementation of /Pow7.exp()/, before symbolically executing it.

\section{Conclusion}
By exploiting conventional OO static verification techniques, we have extended the Iterative Composition form of metaprogramming with a simple contract compatibility check, to statically ensure the correctness of code produced by such metaprogramming. In particular, our approach does not require static verification of the result of metaprogramming, but only requires verification of code present directly in source code.
Following general terminology in software verficiation, we say that a trait is \emph{correct} when its methods respect their contracts.
Thus our result is that starting from a set
of well typed and correct traits, 
any code resulting from arbitrary many steps of trait composition will also be correct and well typed.
In this way, the programmer need only provide correct bulding blocks using traits;
code generated by metaprogramming can be integrated with a correct program without needing to use expensive theorem provers or manual verification.
Our example is applied to code specialization of a mathematical function, but our experience suggests that Iterative composition can be used to synthesize arbitrary behaviour.

%\section{Combining Contracts and Trait Composition}\label{s:combining}
We can handle the composition of traits with contracts in a similar way to what is done for class-based inheritance: the methods in the result of /trait1.extend(trait2)/ contain the /@requires/ and /@ensures/ clauses of the corresponding method (if any) in /trait1/ and /trait2/. This ensures that any \emph{calls} to such methods in /trait1/ and /trait2/ are still correct, and any deductions made from the postconditions of such methods still hold. If a method is implemented in either /trait1/ or /trait2/, but abstract in the other, the implemented version must have at least all the /@requires/ and /@ensures/ clauses of the abstract one. This ensures that the implementation is still a correct implementation for the abstract method.

Note that renaming of methods, such as the /super_/ renaming described in \sref{btc}, is sound provided it is also performed within contracts.\footnote{This of course assumes that the language doesn't provide constructs that depend on method \emph{names} (such as reflection or stack trace operations).}

Consider the following examples illustrating how this works:

\vspace{-3\medskipamount}%
\noindent\begin{minipage}[t]{0pt}
\begin{lstlisting}
Trait trait: class {
	@requires($R_1$)
	@ensures($E_1$)
	Void foo();
}.extend(class {
	@requires($R_2$)
	@ensures($E_2$)
	Void foo();
});
// Identical to:
Trait trait: class {
	@requires($R_1$) @requires($R_2$)
	// or @requires($R_1$ || $R_2$)
	@ensures($E_1$) @ensures($E_2$)
	// or @ensures($E_1$ && $E_2$)
	Void foo();
};

Trait good: class {
	@requires($R_1$)
	@ensures($E_1$)
	Void foo();
}.extend(class {
	// precondition is weaker
	@requires($R_1$) @requires($R_2$)
	// postcondition is stronger
	@ensures($E_1$) @ensures($E_2$)
	Void foo() { ... }
});
\end{lstlisting}
\end{minipage}\hfill
\newlength{\listingWidth}
\settowidth{\listingWidth}{\texttt{a}}
\begin{minipage}[t]{33\listingWidth}
\begin{lstlisting}
// Error!
Trait error: class {
	@requires($R_1$) @requires($R_2$)
	@ensures($E_1$) @ensures($E_2$)
	Void foo();
}.extend(class {
	// stronger precondition
	@requires($R_1$)
	// weaker postcondition 
	@ensures($E_1$)
	Void foo() { ... }
});

Trait really_good: class {
	@requires($R_1$) @requires($R_2$)
	@ensures($E_1$) @ensures($E_2$)
	// will be renamed to super_foo
	Void foo() { ... }
}.extend(class {
	// stronger precondition
	@requires($R_1$)
	// weaker postcondition
	@ensures($E_1$)
	Void super_foo();

	// any contract is ok here
	@requires($R_3$)
	@ensures($E_3$)
	Void foo() { ... }
})
\end{lstlisting}%
\end{minipage}

\noindent These rules guarantee that if all methods in /trait1/ and /trait2/ are correct, then /trait1.extend(trait2)/ will fail with an error, or it will produce a correct result. This means that if all trait literals in the program are statically verified, we can be sure that any traits produced by meta-programming//composition (and hence all class declarations) will be correct by construction, without needing any further static verification.

A more general rule\footnote{We believe it is the most general sound rule that is independent of the method bodies in the traits, however we have not proven it.} would be that an abstract method with pre and post-conditions\footnote{Where the pre//post-conditions are the disjunction//conjunction of all the /@requires////@ensures/ clauses (respectively).} $R_1$ and $E_1$ (respectively) can only be implemented by a method with pre and post-conditions $R_2$ and $E_2$ if $R1 \Rightarrow R2$, and $R2 \wedge E2 \Rightarrow E1$. This would require using a theorem prover to verify that the these implications always hold. However, this could be significantly faster than the alternative of verifying that the implementation of the method is correct under the contract of the abstract version.
%\section{Examples of Our Technique}\label{s:ex}
We now show how the technique described above in \sref{combining} can be used with iterative composition to programmatically generate guaranteed-correct code.

\subsection{Recursive Composition Example}
We now extend our /pow_generate/ example from \sref{bic} to generate correct code, without needing to run a static verifier each time it is called:
\begin{lstlisting}
Trait pow_generate(Nat exp) {
	if (exp == 0)
		return static class { 
			@ensures(result == 0)
			Nat exp() { return 0; }

			@requires(x != 0)
			@ensures(result == x**exp())
			Int pow(Int x) { return 1; }
		};
	else if (exp == 1)
		return static class {
			... // similar to the above, but without the @requires
		};
	else if (exp % 2 == 0)
		return pow_generate(exp/2).extend(static class {
			Nat super_exp();

			@ensures(result == x**super_exp())
			Int super_pow(Int x);
			
			@ensures(result == 2*super_exp())
			Nat exp() { return 2*super_exp(); }

			@ensures(result == x**exp())
			Int pow(Int x) { return super_pow(x*x); }
		});
	else 
		return pow_generate(exp - 1).extend(static class {
			... // similar to the above
		});}
\end{lstlisting}

Each of the trait literals above (the /static class \{...\}/ expressions) can be statically verified as being correct. The function /pow_generate/ then combines these trait literals with our /extend/ operator, guranteeing that the result (if any) is also correct. The idea is that /exp/ and /pow/ represent the current exponent and power function we are generating, /super_exp/ and /super_pow/ correspond to the exponent and power function of the recursive call. Though our trait literals are more verbose than in our non verified version, they ensure that the contract matching performed by /extend/ will succeed. 

The idea is that /pow_generate($e$)/ (where $e > 0$) will return a literal of the following form:
\begin{lstlisting}
static class {
	@ensures(...)
	Nat exp() { ... } // Will return $e$ when called

	@ensures(result == x**exp())
	Int pow(Int x) { ... }
	
	... // and perhaps some private super_ methods
}
\end{lstlisting}
Then /pow_generate($e^\prime$)/ (where $e^\prime > e$) will /extend/ this trait with one of the form:
\begin{lstlisting}
static class {
	Nat super_exp();

	@ensures(result == x**super_exp())
	Int super_pow(Int x);
	
	@ensures(...)
	Nat exp() { ... } // Will return $e^\prime$ when called

	@ensures(result == x**exp())
	Int pow(Int x) { ... }
}
\end{lstlisting}

Because /exp/ and /pow/ are implemented in both /pow_generate($e$)/ and /pow_generate($e^\prime$)/, /extend/ will add a /super_/ prefix to the ones in /pow_generate($e$)/. Thus /pow_generate($e^\prime$)/ will return:
\begin{lstlisting}
static class {
	@ensures(...)
	Nat super_exp() { ... } // Will return $e$ when called

	@ensures(result == x**super_exp())
	Int super_exp(Int x) { ... }
	
	... // And some private methods (which will not clash with the above)
}.extend(static class {
	Nat super_exp();

	@ensures(result == x**super_exp())
	Int super_pow(Int x);
	
	@ensures(...)
	Nat exp() { ... } // Will return $e^\prime$ when called

	@ensures(result == x**exp())
	Int pow(Int x) { ... }
})
\end{lstlisting}

The contracts of both operands of the /extend/ are compatible, and so the above will succeed and produce something of the form:
\begin{lstlisting}
static class {
	@ensures(...)
	Nat exp() { ... } // Will return $e^\prime$ when called

	@ensures(result == x**exp())
	Int pow(Int x) { ... }

	@ensures(...)
	private Nat super_exp() { ... } // Will return $e$ when called

	@ensures(result == x**super_exp())
	private Int super_exp(Int x) { ... }
	
	... // maybe some more private methods
}
\end{lstlisting}

Note that /pow_generate/ never recursively calls /pow_generate(0)/, so the /@requires/ clause in the result of /pow_generate(0)/ will not cause any problems with our contract matching.

Though our system guarantees that the result of /pow_generate/ is `correct' (i.e. the methods in the return trait satisfy their contracts), it does not say what methods will be in the result or what their contracts will be. We could statically verify /pow_generate/ itself, however an easier option would be to check at runtime that the result of /pow_generate($e$)/ is of the form:
\begin{lstlisting}
static class {
	@requires(true)
	Nat exp() { ... }

	@requires(x != 0) // if $e$ == 0
	@requires(true) // otherwise
	@ensures(result == x**exp())
	Int pow(Int x) { ... }
}
\end{lstlisting}
This could be done with introspection on the AST of the resulting trait. We also need to check that calling the above /exp()/ method produces $e$. However, static methods in traits cannot be directly called, instead we could dynamically create a class from the trait and call /exp()/ on the result. Alternatively, we could manually `interpret' the method's AST.
\subsection{Iterative Composition Example}
We can use introspection to automatically generate code for an input trait. Consider for example the following interface\footnote{An abstract class with only abstract methods.}):
\begin{lstlisting}
Account: interface {
	Nat income();

	@ensures(result <= this.income())
	Nat expenses();
};
\end{lstlisting}

\noindent The idea being that an account cant spend more money than they it has.

Now Suppose we want to create a new type of account, a combined one with multiple sub accounts:
\begin{lstlisting}
BuisnessAccount: class implements Account {
	Account salary;
	Account sales;
	Account property;
	
	Nat income() {
		return this.salary.income() 
			+ this.sales.income() + this.property.income(); }
	
	@ensures(result <= this.income())
	Nat expenses() { 
		return this.salary.expenses() 
			+ this.sales.expenses() + this.property.expenses();}
		
	...
};
\end{lstlisting}

Now we can statically verify the above code, but what if we want to combine accounts like this frequently? This could take lots of time for both programmers and automated static verifiers, instead we can create a function /combine_accounts/ that does this for us:
\begin{lstlisting}
// Equivalent to the above
BuisnessAcount: combine_accounts(class { 
	Account salary;
	Account sales;
	Account property;
	... });
\end{lstlisting}

Were we define /combine_accounts/ like this:
\begin{lstlisting}
Trait combine_accounts(Trait input) {
	Trait res = input.extend(class implements Account {
		Nat income() { return 0; }

		@ensures(result <= this.income())
		Nat expenses() { return 0; }
	});
	for (FieldDeclaration fd : input.fields()) {
		res = res.extend(class implements Account {
			Nat super_income();
			
			@ensures(result <= this.super_income())
			Nat super_expenses();

			Account account; // Will be renamed to fd.name
			Nat income() { 
				return this.super_income() + this.account.income(); }

			@ensures(result <= this.income())
			Nat expenses() { 
				return this.super_expenses() + this.account.expenses(); }
		}.rename("account", fd.name)); }
	return res;
}
\end{lstlisting}

\noindent Where /trait.fields/ returns AST structures for each field in the given trait, and /trait.rename($a$, $b$)/ returns /trait/ but with the declaration named $a$ renamed to $b$. The code above is straightforward: it adds an implementation of /Account/ (suitable if /input/ has no fields) to the /input/ trait, it then iterates over every field in /input/ and adds its corresponding /income()/ and /expenses()/ to the implementations in /res/. We use contract matching on the  /super_/ methods in the same was as in /pow_generate/. As with /pow/, the above two trait literals can be statically verified, and so we guarantee that the result of /combine_accounts/ is correct.
%%Conclusions? future work?
%@StrongExceptionSafety is 
%a very strong property,
%and some languages may be unwilling to commit to always preserve it.
%In particular, depending on the details of a specific language
% releasing resources as in \Q@finally@ blocks may require
%some relaxation of @StrongExceptionSafety. Sound releasing of resources could be interesting
%future work.

\section{Conclusions and Future Work}
\label{s:conclusion}
Our approach follows the principles of \emph{offensive programming}
~\cite{stephens2015beginning}\IODel{,} wher\IO{e:}
no attempt to fix or recover an invalid object is performe\IO{d, a}nd
%	\begin{itemize}
%\item
 failures (unchecked exceptions)
		are raised close to their cause: at the end of constructors creating invalid objects and immediately after field updates and instance methods that invalidate their receivers.

%}{3}{[meaning] is not clear} (the operation creating an invalid object), i.e. we ``fail-fast''.    
%		\item
%	\end{itemize}


%The aim of our work is only to enforce object invariants, so we do not present complexities unnecessary for this purpose.
Our work builds on a specific form of TMs and OCs, whose
popularity is growing, and we expect future languages to support some variation of these.
Crucially, any language already designed with such TMs and OCs
can also support our invariant protocol with minimal added complexity.


We demonstrated the applicability and simplicity of our approach with a GUI example.
Our invariant protocol performs several orders of magnitude less checks than visible state semantics, and requires much less annotation 
than Spec\#, (the system with the most comparable goals). In Section~\ref{s:formalism} we formalised our invariant protocol and in Appendix~\ref{s:proof} we prove it sound.
%In appendix~\ref{s:formalism} we formalise our invariant protocol and prove it sound. 
To stay parametric over the various existing type systems which provably enforce the properties we require for our proof (and much more), we do not formalise any specific type system.


% a method could be declared as taking a class whose invariant corresponds to the method's pre-condition,  and returning a class whose invariant corresponds to the pos-condition.


% Such approach may be quite verbose, but would ensure that the precondition on the argument holds for the whole execution of the method, instead of just holding at the beginning.

%It could be worthwhile formalising the minimal type system required by validation.



%However the restrictions we make to ensures that \Q@validate@ is deterministic, namely those the type-system enforces due to its signature, seem quite flexible and reasonable;

%%%%%examples of things that future work may investigate allowing are deterministic I/O and multi-threading. 

The language we presented here restricts the forms of \Q@invariant@ and capsule mutator methods;
such strong restrictions allow for sound and efficient injection of invariant checks. 
\IOBlock{Merge this and the next paragraph \& compact them!}{These restrictions do not get in the way of writing invariants over immutable data, but the box pattern is required for verifying complex mutable data structures. We believe this pattern, although verbose, is simple and understandable. While it may be possible for a more complex and fragile type system to reduce the need for the pattern whilst still ensuring our desired semantics, we prioritize simplicity and generality. }

\IOBlock{Merge into the previous paragraph \& compact}{In order to obtain safety, simplicity, and efficiency we traded some expressive power:
the \Q@invariant@ method can only refer to immutable and encapsulated state.
This means that while we can easily verify that a doubly linked list of immutable elements
is correctly linked up,
we can not do the same for a doubly linked lists of mutable elements. Our approach does not prevent correctly implementing such data structures, but the \Q@invariant@ method would be unable to access the list's nodes, since they would contain \Q@mut@ references to shared objects.
In order to verify such data structures we could add a special kind of field which cannot be (transitively) accessed by invariants; such fields could freely refer to any object. We are however unsure if such complexity would be justified.} \IOComm{Mention flexible ownership types as a potential solution? (Assuming it is)}

% To verify those data-structures, in future work
% we may investigate a special kind of field that
% could be accessed only using a \Q@mut@ receiver.
% Such fields would be allowed to refer to not encapsulated state, 
% and they would be unreachable from the invariant code,that starts from a \Q@read this@.

% \LINE
% The language we presented here restricts the form of \Q@invariant@ and capsule mutator methods. 
% We have shown that such restrictions, albeit strong, allow sound and efficient injection of invariant checks. 
% While our restrictions do not hamper writing invariants over immutable data, invariants over complex mutable % data require the box pattern.  We believe the box pattern, although verbose, is simple and understandable, but % could be improved with syntax sugar.

% Our goals of simplicity and efficiency come at the cost of expressivity: we are unable to express invariants % over non-encapsulated mutable structures, 
% though a more complex and fragile type system may reduce such limitations. However we believe we have % demonstrated that our limitations are not too severe and that we have achieved our goals.
% \LINE

%, however such a language is unlikely to be easily understood by programmers;
%being able to predict whether code would be well typed allows programmers
%to better take advantage of the language.

For an implementation of our work to be sound, catching exceptions like stack overflows or out of memory
cannot be allowed in \Q@invariant@ methods, since they are not deterministically thrown.
%For an implementation of our work to be sound, non-deterministic exceptions like stack overflows or out of memory
%errors cannot be caught in invariants.
%this
%use exception catching as a non deterministic conditional choice, 
%allowing non deterministic behaviour.
Currently L42 never allows catching them, however we could also write a (native) capability method (which can't be used inside an invariant) that enables catching them. Another option worth exploring would be to make such exceptions deterministic, perhaps by giving invariants fixed stack and heap sizes.

Other directions that could be investigated to improve our work include the addition of syntax sugar to ease the burden of the box and the transform patterns; type modifier inference\IODel{, and support for flexible ownership types}.

%Our work, in comparison to previous RV techniques,
%aims to be efficient by limiting the number of validation calls, however we have \REVComm{no empirical evaluation of our approach's performance}{3}{\label{CONTRA1}contradictory to [see footnote \ref{CONTRA2}]}.
%To improve efficiency it could be worth investigating elision of unnecessary validation calls
%or even only validating parts of objects (by running the part of \Q@validate@ that could fail).

\begin{comment}
In literature, both static and runtime verification discuss
the correctness of common programming patterns in conventional languages.
Their struggle is proof of how hard it is to deal with the expressive power of unrestricted imperative object-oriented programming.
 Here instead we build on languages using TMs and OCs to tame the use of imperative features. In this way
we have a fresh start where static variables are disallowed, unchecked exceptions require care to be captured, and I/O is allowed only when an opportune capability object is reachable.
Following those restrictions allow simpler reasoning.
The philosophy of our approach is to be like an extended type system: 
It is the programmer's decision
to annotate a field with a certain type,
or the class with a certain \validate.
If the program is well-typed, they are not questioned in their intent.
During execution the system is solely responsible for soundly enforcing the invariant protocol.
This is in sharp contrast with most work in RV, that is often conceived more as a tool to ease debugging:
both deciding properties and enforcing them is controlled by the programmers.
This is also different from static verification,
%where the properties are ensured instead of enforced.
where the properties are ensured ahead of time instead of being enforced during execution.
%Static verification is very heavy weight, and often impractical/restrictive.

%Both static and runtime verification
%aim to monitor a wide range of properties; to this aim they accept a 
%great deal of complexity, and require the programmer to develop a deep understanding
%over the behaviour and the structure of code.
%For example, the specification of method’s pre and post-conditions
%encode a generalization of the program behaviour in the dedicated specification language.
%This means that, even in the best case scenario, 
%using pre/post-conditions the user is required to specify the program semantics twice:
%first in the specification language and then in the underlying programming language.
%In comparison, our approach aims to only verify conditions on immutable or well encapsulated state.
%This makes our approach \emph{ultra-lightweight}:
%the programmer specifies only the desired \Q@invariant@ method.

Moreover, our approach does not aim to replace static or run-time verification,
but is a building block they can rely upon.
\end{comment}

%\noindent\textit{Our approach:}


\catcode\Slash=12% turn of my awesome slash
\bibliographystyle{eptcs}
\bibliography{paper}
\clearpage
\appendix
\appendix
\section{Proof} 
\label{s:proof}

\begin{theorem}[Sound Validation]
	if $c:\Kw{Cap};\emptyset\vdash \e: \T$ and
	$c\mapsto\Kw{Cap}\{\_\}|\e\rightarrow^+ \sigma|\ctx_v[r_l]$, then
	either $valid(\sigma,l)$ or $\mathit{trusted}(\ctx_v,r_l)$.
\end{theorem}

We believe this property captures very precisely our statement in Section~\ref{s:validation}.

It is hard to prove Sound Validation directly,
so we first define a stronger property,
called \emph{Stronger Sound Validation} and
show that it is preserved during reduction by means of conventional 
Progress and Subject Reduction (Progress is one of our assumption,
while Subject Reduction relies heavily on SubjectReductionBase).
That is,
Progress+Subject Reduction $\Rightarrow$ Stronger Sound Validation,
\\*and Stronger Sound Validation $\Rightarrow$ Sound Validation.

\subsection{Stronger Sound Validation $\Rightarrow$ Sound Validation}

Stronger Sound Validation depends on 
$\mathit{wellEncapsulated}$, $\mathit{monitored}$
and $OK$:

\noindent\textbf{Define} $\mathit{wellEncapsulated}(\sigma,\e,l_0)$:\\*
\indent$\forall l \in \mathit{erog}(\sigma,l_0), \text{not}\ \mathit{mutatable}(l,\sigma,\e)$

\noindent The main idea is that an object is well encapsulated if its encapsulated state is safe from
modification. 

\noindent\textbf{Define} $\mathit{monitored}(\e,l)$:\\*
\indent$\e=\ctx_v[M(l;\e_1;\e_2)]$ and either $\e_1=l$ or $l$ is not inside $\e_1$.

\noindent An object is monitored if the execution
is currently inside of a monitor for that object, and
the monitored expression $\e_1$ does not
contains $l$ as a \emph{proper} subexpression.

A monitored object is associated with an expression that can not observe it, but may 
reference its internal representation directly.
In this way, we can safely modify its representation before checking for the invariant.

The idea is that at the start the object will be valid and $\e_1$ will contain $l$;
but during reduction, the $l$ reference will be used in order to
give access to the internal state of $l$; only after that moment, the object may become invalid.


\noindent\textbf{Define} $OK(\sigma,e)$:\\
\indent $\forall l\in\dom(\sigma)$
  either\\
\indent\indent 1. $\mathit{garbage}(l,\sigma,\e)$\\
\indent\indent 2. $\mathit{valid}(\sigma,l)$ and $\mathit{wellEncapsulated}(\sigma,\e,l)$\\
\indent\indent 3. $\mathit{monitored}(\e,l)$

Finally, the system is in a valid state with respect to validation
if for all the objects in the memory, one of these 3 cases apply:
%the class of the object has no invariant method;
the object is not (transitively) reachable from the expression (thus can be garbage collected);
the object is valid, and the object is encapsulated;
or the object is currently monitored.

\begin{theorem}[Stronger Sound Validation]
if $c:\Kw{Cap};\emptyset\vdash \e_0: \T_0$ and
$c\mapsto\Kw{Cap}\{\_\}|\e_0\rightarrow^+ \sigma|\e$, then
$OK(\sigma,\e)$
\end{theorem}
\noindent Starting from only the capability object,
any well typed expression $\e_0$ can be reduced for an arbitrary amount of steps,
and $OK$ will always hold.
\\
\begin{theorem} Stronger Sound Validation $\Rightarrow$ Sound Validation
\end{theorem}
\begin{proof}
\noindent By Stronger Sound Validation, each $l$ in the current redex must be $OK$:
\begin{enumerate}
	\item If $l$ is garbage, it cannot be in the current redex, a contradiction.
	\item If $\mathit{valid}(\sigma,l)$, then $l$ is valid, so thanks to Determinism
	no invalid object could be observed.
	\item Otherwise, if $\mathit{monitored}(\e,l)$ then either:
	\begin{itemize}
	 \item we are executing inside of $\e_1$ thus the current redex is inside of a sub-expression of the monitor that does not contain $l$, a contradiction.
	 \item or we are executing inside $\e_2$:
	 by our reduction rules, all monitor expressions start with 
	 $\e_2=l$\Q@.validate()@, thus the first execution step
	 of $\e_2$ is trusted. Following execution steps are also trusted, since by well formedness the body of invariant methods only use \Q@this@ (now translated to $l$) to access fields.
	\end{itemize}
\end{enumerate}
In any of the possible cases above, Sound Validation holds for $l$, and so it holds for all redexes.
\end{proof}

\subsection{Subject Reduction}

\noindent\textbf{Define} $\text{fieldGuarded}(\sigma,\e)$:\\*
\indent$\forall \ctx$ such that $\e=\ctx[l\singleDot\f] $
and $\Sigma^\sigma(l).f=\Kw{capsule}\,\_$, and $\f\mathrel{\mathit{inside}} \Sigma^\sigma(l).\mathit{validate}$\\*
\indent\indent either 
$\forall T, \forall C, \Sigma^\sigma;\x:\Kw{mut}\,C\,\not\vdash\ctx[\x]:T$, or\\*
\indent\indent $\ctx=\ctx'[$\Q@M(@$l$\Q@;@$\ctx''$\Q@;@$\e$\Q@)@$]$ and $l$ is contained exactly once in $\ctx''$

That is, all \emph{mutating} capsule field accesses are individually guarded by monitors.
Note how we use $C$ in $\x:\Kw{mut}\,C$ to guess the type of the accessed field,
and that we use the full context $\ctx$ instead of the evaluation context $\ctx_v$
to refer to field accesses everywhere in the expression $\e$.


\begin{theorem}[Subject Reduction]
if $\Sigma^{\sigma_0};\emptyset\vdash e_0: T_0$,
$\sigma_0|e_0\rightarrow \sigma_1|e_1$,
$OK(\sigma_0,\e_0)$
and
$\mathit{fieldGuarded}(\sigma_0,\e_0)$
then
$\Sigma^{\sigma_1};\emptyset\vdash e_1: T_1$,
$OK(\sigma_1,e_1)$ and
$\mathit{fieldGuarded}(\sigma_1,\e_1)$
\end{theorem}

\begin{theorem}
	Progress + Subject Reduction $\Rightarrow$ Stronger Sound Validation
\end{theorem}
\begin{proof}
This proof proceeds by induction in the usual manner.

\emph{Base Case}: At the start of the execution, the memory is going to only contain $c$: since $c$ is defined to be initially $\mathit{valid}$, and has only \Q@mut@ fields, and so it is trivially $\mathit{wellEncapsulated}$, thus $OK(c\mapsto\Kw{Cap},e)$.

\emph{Induction}: By Progress we always have another evaluation step to take, by Subject Reduction such a step will preserve $\mathit{OK}$, and so by induction $\mathit{OK}$ holds after any number of steps.

Note how for the proof garbage collection is important: 
when the \Q@validate()@ method evaluates to \Q@false@, 
execution can continue only if the offending object is classified as garbage.
\end{proof}

\subsection{Expose Instrumentation}
We first introduce a lemma derived from well formedness and the type system:
\begin{Lemma}[ExposerInstrumentation]
If $\sigma_0 | \e_0\rightarrow \sigma_1 |\e_1$ and
$\text{fieldGuarded}(\sigma_0,\e_0)$
\\*
then $\text{fieldGuarded}(\sigma_1,\e_1)$
\end{Lemma}
\begin{proof}
The only rule that can 
introduce a new field access is \textsc{mcall}.
In that case, ExposerInstrumentation holds
by well formedness (all field accesses in methods are of the form \Q@this.f@) 
and since \textsc{mcall} inserts a monitor while invoking capsule mutator methods, and not field accesses themselves. If however the method is not a \Q@mut@ method but still accesses a capsule field, by MutField such a field access expression cannot be typed as \Q@mut@ and so no monitor is needed.

Note that \textsc{monitor exit} is fine because monitors are removed only when
 $e_1$ is a value.
\end{proof}

\subsection{Proof of Subject Reduction}
Any reduction step can be obtained
by exactly one application of rule \textsc{ctx} and then one other rule. Thus the proof can simply proceed by cases on such other applied rule.

By SubjectReductionBase and ExposerInstrumentation, 
$\Sigma^{\sigma_1};\emptyset\vdash e_1: T_1$ and  $\mathit{fieldGuarded}(\sigma_1,\e_1)$. So we just need to proceed by cases on the reduction rule applied to verify that $OK(\sigma_1,\e_1)$ holds:


\begin{enumerate}
\item (\textsc{update}) $\sigma|l\singleDot f\equals v\rightarrow \sigma'|\e'$:
	\begin{itemize}
	  \item By \textsc{update} $\e'=\Kw{M}\oR l;l;l\singleDot\text{validate}\oR\cR\cR$, thus $\mathit{monitored}(\e,l)$.
	  \item Every $l_1$ such that $l\in \mathit{rog}(\sigma,l_1)$ will verify the same case as the former step:
	  \begin{itemize}
	  	\item If it was $\mathit{garbage}$, clearly it still is.
	  	\item If it was $\mathit{monitored}$, it also still is.
	    \item Otherwise it was $\mathit{valid}$ and $\mathit{wellEncapsulated}$:
			\begin{itemize}
				\item If $l\in \mathit{erog}(\sigma,l_1)$ we have a contradiction since $mutatable(l, \sigma, e)$, (by MutField)
		    	\item Otherwise, by our well-formedess criteria that \Q@.validate()@ only accesses \Q@imm@ and \Q@capsule@ fields, and by Determinism it is clearly the case that $\mathit{valid}$ still holds;
				By HeadNotCircular it cannot be the case that $l\in \mathit{erog}(\sigma',l_1)$ and so $l_1$ is still $\mathit{wellEncapsulated}$.
		  	\end{itemize}
	  \end{itemize}
	  \item Every other $l_0$ is not in the reachable object graph of $l$
	  thus it being $\mathit{OK}$ could not have been effected by this reduction step.
	\end{itemize}

\item (\textsc{access}) $\sigma|l\singleDot f \rightarrow \sigma|v$:
	\begin{itemize} 
		\item If $l$ was $valid$ and $wellEncapsulated$:
		\begin{itemize}
			\item If we have now broken $wellEncapsulated$ we must have made something in its $erog$  $mutatable$. As we can only type \Q@capsule@ fields as \Q@mut@ and not \Q@imm@ fields, by FieldMut we must have that $f$ is \Q@capsule@ and $l\singleDot f$ is being typed as \Q@mut@. By $\mathit{fieldGuarded}(\sigma_0,\e_0)$ the former step must have been inside a monitor \Q@M(@$l$\Q@;@$\ctx_v[l$\Q@.f@$]$\Q@;@$\e$\Q@)@
		    and the $l$ under reduction was the only occurrence of $l$.
		    Since $f$ is a capsule, we know that $l\notin \text{erog}(\sigma,l)$
		    by HeadNotCircular. Thus in our new step $l$ is not $inside$ $\ctx_v[v]$. Thus $l$ must be $monitored$ and hence it is $OK$.
		    
		    \item Otherwise, $l$ is still $OK$
    	\end{itemize}

		\item Nothing that was $\mathit{garbage}$ could have been made reachable by this expression, since the only value we produced was $v$ and it was reachable through $l$ (and so could not have been garbage), thus $garbage$ is still $OK$.
		
		\item As we don’t change any monitors here, nothing that was $monitored$ could have been made un-$monitored$, and so it is still $OK$.
		
		\item Suppose some $l_0$ was $wellEncapsulated$ and $valid$:
		\begin{itemize}
			\item If $l$ was in the $rog$ of $l_0$, by CapsulaeTree, if $l$ was in the $erog$ of $l$, then $v$ can only be reached from $l_0$ by passing through $l$, and so we could not have made $l_0$ non-$wellEncapsulated$. In addition, since only things in the $erog$ can be referenced by $\singleDot\Kw{validate}\oR\cR$, $l_0$’s validity can not depend on $l$, and by Determinism it is still the case that $l_0$ is $valid$. And so we can’t have effected $l_0$ being $OK$.
			\item Otherwise this reduction step could not have affected $l_0$ so $l_0$ is still $OK$.
		\end{itemize}
\end{itemize}

\item (\textsc{mcall}, \textsc{try enter} and \textsc{try ok}):

	These reduction steps do not modify memory, nor do they modify the memory-locations reachable inside of main-expression, nor do they modify any monitor expressions. Therefore it cannot have any effect on the $garbage$, $wellEncapsulated$, $valid$ (due to Determinism) or $monitored$ properties of any memory locations, thus $\mathit{OK}$ still holds.

\item (\textsc{new}) $\sigma|\Kw{new}\ C\oR\vs\cR\rightarrow \sigma,l\mapsto C\{\vs\}| \Kw{M}\oR l;l;l\singleDot\text{validate}\oR\cR\cR$:

	Clearly the newly created object ($l$) is monitored. As for \textsc{mcall}, other objects and properties are not disturbed, and so $\mathit{OK}$ still holds.


\item (\textsc{monitor exit}) $\sigma|\Kw{M}\oR l; v;\Kw{true}\cR\rightarrow \sigma|v$:
\begin{itemize}
	\item As monitor expressions are not present in the original source code, it must have been introduced by \textsc{update}, \textsc{mcall}, or \textsc{new}. In each case the 3\textsuperscript{rd} expression started of as $l\singleDot\Kw{validate}\oR\cR)$, and it has now (eventually) been reduced to $\Kw{true}$, thus by Determinism $l$ is $valid$.

	\item  If the monitor was introduced by \textsc{update}, then $v = l$. We must have had that $l$ was well encapsulated before \textsc{update} was executed (since it can’t have been garbage and $monitored$), as \textsc{update} itself preserves this property and we haven’t modified memory in anyway, we must still have that $l$ is $wellEncapsulated$. As $l$ is $valid$ and $wellEncapsulated$ it is $OK$.

	\item If the monitor was introduced by \textsc{mcall}. Then it was due to calling a capsule-mutator method that mutated a field $f$.
	\begin{itemize}
		\item A location that was $garbage$ obviously still is, and so is also $OK$.
		\item No location that was $valid$ could have been made non-valid since this reduction rule performs no mutation of memory. If a location was $wellEncapsulated$ before the only way it could be non-$wellEncapsulated$ is if we somehow leaked a \Q@mut@ reference to something, but by our well-formedness rules $v$ cannot be typed as \Q@mut@ and so we can’t have affected $wellEncapsulated$, hence such thing is still $OK$.
		\item The only location that could have been made un-$monitored$ is $l$ itself. By our well-formedness criteria $l$ was only used to modify $l.f$, and we have no parameters by which we could have made $l.f$ non-$wellEncapsulated$, since that would violate CapsuleTree. As nothing else in $l$ was modified, and it must have been $wellEncapsulated$ before the \textsc{mcall}, it still is, and since  $l$ is valid, it is $OK$.
	\end{itemize}
	\item Otherwise the monitor was introduced by \textsc{new}. Since we require that \Q@capsule@ fields and \Q@imm@ fields are only initialised to \Q@capsule@ and \Q@imm@ expressions, by CapsuleTree the resulting value, $l$, must be $wellEncapsulated$, since $l$ is also $valid$ we have that $l$ is $OK$.

\end{itemize}

\item (\textsc{try error}) $\sigma,\sigma_0|\Kw{try}^\sigma\oC \mathit{error}\cC\ \Kw{catch}\ \oC\e\cC\rightarrow \sigma|\e$:

	By StrongExceptionSafety we know that $\sigma_0$ is $\mathit{garbage}$ with respect to $\ctx_v[\e]$. By our well-formedness criteria no location inside $\sigma$ could have been $monitored$.

	Since we don’t modify memory, everything in $\sigma_0$ is $\mathit{garbage}$ and nothing inside $\sigma$ was previously monitored, it is still clearly the case that everything in $\sigma$ is $\mathit{OK}$
\end{enumerate}
\end{document}
