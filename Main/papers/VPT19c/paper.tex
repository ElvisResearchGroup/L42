\PassOptionsToPackage{svgnames}{xcolor}
\documentclass[submission,copyright,creativecommons]{eptcs}
\providecommand{\event}{VPT 2019} % Name of the event you are submitting to
%\usepackage{breakurl}             % Not needed if you use pdflatex only.
\usepackage{underscore}           % Only needed if you use pdflatex.
\usepackage{listings}
\usepackage{xcolor}
\usepackage{letltxmacro}
\usepackage{mathtools}
\usepackage{mathpartir}
%\usepackage{stix}

\definecolor{darkRed}{RGB}{100,0,10}
\definecolor{darkBlue}{RGB}{10,0,100}
\newcommand*{\ttfamilywithbold}{\fontfamily{pcr}\selectfont}
%\newcommand*{\ttfamilywithbold}{\ttfamily}

%found on http://tex.stackexchange.com/questions/4198/emphasize-word-beginning-with-uppercase-letters-in-code-with-lstlisting-package
%\lstset{language=FortyTwo,identifierstyle=\idstyle}
%
\makeatletter
\newcommand*\idstyle{%
        \expandafter\id@style\the\lst@token\relax
}
\def\id@style#1#2\relax{%
        \ifcat#1\relax\else
                \ifnum`#1=\uccode`#1%
                        \ttfamilywithbold\bfseries
                \fi
        \fi
}
\makeatother

\lstset{language=Java,
  basicstyle=\upshape\ttfamily\footnotesize,%\small,%\scriptsize,
  keywordstyle=\upshape\bfseries\color{darkRed},
  showstringspaces=false,
  mathescape=true,
  xleftmargin=0pt,
  xrightmargin=0pt,
  breaklines=false,
  breakatwhitespace=false,
  breakautoindent=false,
 identifierstyle=\idstyle,
 morekeywords={method,Use,This,constructor,as,into,rename},
 deletekeywords={double}
}

\newcommand*{\SavedLstInline}{}
\LetLtxMacro\SavedLstInline\lstinline
\DeclareRobustCommand*{\lstinline}{%
	\ifmmode
	\let\SavedBGroup\bgroup
	\def\bgroup{%
		\let\bgroup\SavedBGroup
		\hbox\bgroup
	}%
	\fi
	\SavedLstInline
}

\newcommand\saveSpace{\vspace{-2pt}}

\newcommand\Rotated[1]{\begin{turn}{90}\begin{minipage}{12em}#1\end{minipage}\end{turn}}

\newcommand{\Q}{\lstinline}
\newenvironment{bnf}{$\begin{array}{lcll}}{\end{array}$}
\newcommand{\production}[3]{%
	\text{\itshape #1}&%
	\!\!\!\!\!\Coloneqq\!\!\!\!\!&%
	\text{\itshape #2}&%
	\!\!\!\!\!\mbox{#3}}
%\newcommand{\prodFull}[3]{#1&::=&\mbox{#2}&\mbox{#3}}
\newcommand{\prodInline}[2]{#1\Coloneqq#2}
\newcommand{\prodNextLine}[2]{&&#1&\mbox{#2}}
\newcommand{\terminal}[1]{\ensuremath{$\texttt{#1}$}}
%\newcommand{\metavariable}[1]{\ensuremath{\mathit{#1}}}

\newcommand\Rulename[1]{{\textsc{#1}}}
\newcommand\ctx[1]{\ensuremath{\mathcal{E}_#1}\!}
\newcommand{\lib}[3]{\Q@\{@\!#1\Q{;}\ #2 \Q{;}\ #3\Q@\}@}
\newcommand{\rp}[1]{\Q{(}\!#1\Q{)}}
\newcommand{\red}[3]{#1\rp{#2\Q{=}#3}}
\newcommand{\summ}[2]{#1\ \Q{<+}\ #2}
\newcommand{\mmid}{\ensuremath{\mid}}
\newcommand{\hole}{\ensuremath{\square}}
%--------------------------
\newcommand{\mynotes}[3]{{\color{#2} {\sc #1}: #3}}
\newcommand\isaac[1]{\mynotes{Isaac}{red}{#1}}
\newcommand\marco[1]{\mynotes{Marco}{blue}{#1}}


\lstset{
    numbers=left,
    stepnumber=1,
    showstringspaces=false,
    firstnumber=last
}

\providecommand*{\code}[1]{\Q`#1`}
\newcommand{\saveSpace}{\vspace{-3px}}
\newcommand{\loseSpace}{\vspace{1ex}}
\usepackage{verbatim}

\title{Iteratively Composing Statically Verified Traits}
\def\titlerunning{Iteratively Composing Statically Verified Traits}

%magic code from https://tex.stackexchange.com/questions/344794/centering-issues-with-multiple-authors-with-the-same-affiliation-eptcs-format
\RequirePackage{array}
\newenvironment{authors}[1]%
  {\begingroup
   \gdef\estyle{}%
   \renewcommand\institute[1]%
     {\\\multicolumn{#1}{@{}c@{}}{\scriptsize\begin{tabular}[t]{@{}>{\footnotesize}c@{}}##1\end{tabular}}}%
   \renewcommand\email[1]%
     {\gdef\estyle{\footnotesize\ttfamily}\\##1\gdef\estyle{}}
   \begin{tabular}[t]{@{}*{#1}{>{\estyle}c}@{}}
  }%
  {\end{tabular}%
   \endgroup
  }

\def\anauthor#1#2{%
	#1%
	\institute{}
	\email{#2}%
}
\def\vuw{\institute{School of Engineering and Computer Science\\%
			Victoria University of Wellington\\%
			Wellington, New Zealand}}
\author{
	\begin{authors}{4}
	Isaac Oscar Gariano & Marco Servetto & Alex Potanin & Hrshikesh Arora
	\vuw
	\email{isaac@ecs.vuw.ac.nz & marco.servetto@ecs.vuw.ac.nz & alex@ecs.vuw.ac.nz & arorahrsh@myvuw.ac.nz}
	\end{authors}
}

\def\authorrunning{Isaac O.\,G., M. Servetto, A. Potanin \& H. Arora}

% Allows one to write /hello/ instead of \Q@hello@.
% Use // to get a normal text slash
\chardef\Slash=`\/
\catcode\Slash=\active
\chardef\other=12 % char code for other characters
\def/#1/{%
	\ifx/#1/% #1 is empty
		\Slash% just print a slash
	\else%
		\lstinline/#1/%
	\fi%
}
\let\oldinput=\input
\def\input#1{\oldinput{\detokenize{#1}}} % Don't expand commands in input (needed since / is a command)
\newcommand{\sref}[1]{Section~\ref{s:#1}}


%\definecolor{blue}{HTML}{0000F0} %
%\definecolor{purple}{HTML}{700090}
%\definecolor{orange}{HTML}{F07000}
%\definecolor{teal}{HTML}{0090B0}
%\definecolor{brown}{HTML}{A00000}
%\definecolor{green}{HTML}{008000}
%\definecolor{pink}{HTML}{F000F0}

\makeatletter
\renewcommand*\idstyle{%
	\expandafter\id@style\the\lst@token\relax}
\def\id@style#1#2\relax{%
        \ifcat#1\relax\else
                \ifnum`#1=\uccode`#1%
                        \color{DarkGreen}%
                \fi%
        \fi%
}
\makeatother


\lstset{%
	language=Java, morekeywords={exists, forall, @requires, @ensures, result, rename, hide, with},
	tabsize=2,
	identifierstyle=\idstyle,
%	aboveskip=0pt
%	keywordstyle=\color{blue},
%	commentstyle=\color{green},
%	stringstyle=\color{brown}
%	literate=
%		{||}{{$\vee$}}1
%		{&&}{{$\wedge$}}1
%		{<=}{{$\leq$}}1
%		{>=}{{$\geq$}}1
%		{**}{{$^{**}$}}1
}
%citations
\newcommand{\REV}[3]{%
	\NoteColour{red}{#1\NoteText{\footnote{%
		\textcolor{red}{\textbf{REV#2{:} #3}}}}}}
	
\begin{document}
\maketitle
\begin{abstract}
Static verification relying on an automated theorem prover can be very slow and brittle: since static verification is undecidable correct code may not pass a particular static verifier.
In this work we use metaprogramming to generate
code that is correct by construction.
A theorem prover is used only to verify the initial traits: little code unit that can be used to compose bigger programs.
In this way, trait composition is guaranteed to
start from correct code and to always produce correct code.
We extends conventional traits with methods pre and post conditions.
Crucially, the checking performed by the traditional trait composition (/+/)  operator need to also check the compatibility of contracts. In this way, there is no need to re-verify the produced code.

We show how to apply our approach to the
iconic example of the power function, where metaprogramming can generate fast (and correct by construction) specialized versions when the exponent known in advance.
\end{abstract}

%%% LyX 2.3.1 created this file.  For more info, see http://www.lyx.org/.
%% Do not edit unless you really know what you are doing.
\documentclass[10pt,twoside,british]{scrartcl}
\usepackage{amstext}
\usepackage{fontspec}
\usepackage{unicode-math}
\setmainfont[Mapping=tex-text]{TeX Gyre Schola}
\setsansfont[Scale=1.04,Mapping=tex-text]{Latin Modern Sans}
\setmonofont[Scale=1.09]{Latin Modern Mono}
\usepackage[a4paper]{geometry}
\geometry{verbose,tmargin=2.54cm,bmargin=2.54cm,lmargin=2.54cm,rmargin=2.54cm,headheight=0.7cm,headsep=0.7cm,footskip=1.05cm}
\setlength{\parskip}{\medskipamount}
\setlength{\parindent}{0pt}
\usepackage[xetex]{xcolor}
\usepackage{calc}
\usepackage{endnotes}

\makeatletter
%%%%%%%%%%%%%%%%%%%%%%%%%%%%%% Textclass specific LaTeX commands.
\let\footnote=\endnote

\@ifundefined{date}{}{\date{}}
%%%%%%%%%%%%%%%%%%%%%%%%%%%%%% User specified LaTeX commands.
\usepackage{xcolor}
\usepackage{scrlayer-scrpage}

\newcommand{\lfoot}[1]{\ifoot{\textnormal{#1}}}
\newcommand{\rfoot}[1]{\ofoot{\textnormal{#1}}}

\definecolor{AccentB80}{RGB}{197, 213, 255}
\definecolor{AccentB60}{RGB}{139, 171, 255}
\definecolor{AccentB40}{RGB}{81,  130, 255}
\definecolor{Accent}   {RGB}{0,   63,  221}
\definecolor{AccentD25}{RGB}{0,   46,  165}
\definecolor{AccentD50}{RGB}{0,   31,  110}

%\renewcommand\Huge{\@setfontsize\Huge{10pt}{26}} % Was 25?
%\renewcommand\Large{\@setfontsize\Large{10pt}{14}} % Was 14?
%\large	 10.5 % was 11?
\newcommand{\StyleTitle}[1]{{\Huge{\textcolor{Accent}{{#1}}}}} % Expanded +0.5pt
\newcommand{\StyleSubTitle}[1]{{\large{\textcolor{AccentB40}{{#1}}}\medskip}} % 25pt after , Expanded +0.5pt
\renewcommand{\maketitle}{\StyleTitle{\@title}\\\StyleSubTitle{\@author}}
\newcommand{\zwnbsp}{}
%\renewcommand{\ensuremath}[1]{#1}

\renewcommand{\thanks}[1]{\footnote{#1}}
\definecolor{blue}{HTML}{0000F0} % 
\definecolor{purple}{HTML}{700090}
\definecolor{orange}{HTML}{F07000}
\definecolor{teal}{HTML}{0090B0}
\definecolor{brown}{HTML}{A00000}
\definecolor{green}{HTML}{008000}
\definecolor{pink}{HTML}{F000F0}

\usepackage{enumitem}
\setlist{nolistsep}

\usepackage{fontspec} % To set xetex fonts
\setmainfont[Ligatures={Common, Discretionary}]{TeX Gyre Schola}
\setmonofont[Ligatures={Discretionary}, Scale=1.090909090909090909090909]{Latin Modern Mono} % 12/11
\setsansfont[Numbers={Monospaced},Ligatures={Common, Discretionary}, Scale=1.045454545454545]{Latin Modern Sans} %11.5/11

\unimathsetup{math-style=ISO}
\setmathfont{TeX Gyre Schola Math}

\newcommand{\parj}{\setlength{\parskip}{0pt}}
\newcommand{\pars}{\setlength{\parskip}{\medskipamount}}
\usepackage[para]{footmisc}
\usepackage{hyperref}
\hypersetup{colorlinks,urlcolor=[RGB]{0, 155, 240}}
\newcommand{\email}[1]{%
	\href{mailto:#1}{\texttt{#1}}
}
\raggedright

\makeatother

\usepackage{polyglossia}
\setdefaultlanguage[variant=british]{english}
\begin{document}
\title{Callability Control}
\author{By Isaac Oscar Gariano\textsuperscript{1} and Marco Servetto\textsuperscript{2}
(Victoria University of Wellington)}

\maketitle
\global\long\def\empty{\zwnbsp}%

\global\long\def\k#1{\textcolor{blue}{\texttt{#1}}}%
\global\long\def\t#1{\textcolor{teal}{#1}}%
\global\long\def\f#1{\textcolor{purple}{#1}}%
\global\long\def\l#1{\textcolor{brown}{#1}}%
\global\long\def\v#1{\textcolor{orange}{#1}}%
\global\long\def\m#1{\textcolor{green}{#1}}%

\global\long\def\c#1{\texttt{#1}}%
\global\long\def\ck#1{\c{\k{#1}}}%
\global\long\def\ct#1{\c{\t{#1}}}%
\global\long\def\cf#1{\c{\f{#1}}}%
\global\long\def\cv#1{\c{\v{#1}}}%
\global\long\def\cm#1{\c{\m{#1}}}%

\global\long\def\tab{\texttt{\ \ \ \ }}%

\global\long\def\calls#1{\ck{calls[}#1\ck ]}%

How does one know whether a called function will access arbitrary
files on your system? One common option is to say that if the called
function is not declared impure (e.g. with Haskell’s $\ct{IO}$
monad) it will not read any files (or do any other I/O). Another option
is to raise a runtime error or trap if it tries to do so (e.g. with
Java’s $\ct{SecurityManager}$). The first option is overly restrictive
and heavy weight (what if the function only needs access to standard
out?) whilst the second hampers static and modular reasoning. We call
the operations a function may call its \emph{callability}; we propose
a simple and flexible type-system feature that allows one to soundly
statically reason about functions callability on a fine-grained level.
Importantly our system only restricts \emph{calling} functions, it
does not restrict what \emph{objects} can be created, passed around,
or aliased.

In our systems, all functions are annotated with a base-callability
using the syntax $\calls{\f{\f x}_{1},\dots,\f x_{n}}$, where each
$\f x_{i}$ indicates the name of a function (or set of functions)
that can be called. Note how here a function declares what \emph{it
can access}, compare this with conventional accessibility, (like Scala’s
$\k{private[}\t C\k ]$ modifier) where a function declares what \emph{can
access it}. Our system has three core rules:\parj{}
\begin{itemize}
\item Any function can call call itself.
\item A function can call each function in its base-callability.
\item If a function $\f f$ can call everything in the base-callability
of a function $\f g$, then $\f f$ can call $\f g$.
\end{itemize}
\pars{}With these rules harmless language/library primitives/intrinsics
(such as integer addition) would be marked with $\calls{\empty}$,
allowing any function to call it. On the other hand, if a primitive
$\f f$ should be restricted (e.g. a function to exit the program),
one could annotate it with $\calls{\f f}$, thus allowing only specifically
marked functions to call it. Other systems can be envisioned, such
as having a dummy $\cf{io}$ function, and than marking all I/O primitives
with $\calls{\cf{io}}$.

We also support callability generics and dynamic dispatch, thus allowing
the system to become quite flexible; consider for example the following
(where $\cf{'a}$ denotes a generic callability parameter):\parj{}

\fbox{\begin{minipage}[t]{1\columnwidth - 2\fboxsep - 2\fboxrule}%
$\texttt{\ensuremath{\ck{static}} \ensuremath{\ck{extern}} \ensuremath{\ct{long}} \ensuremath{\cf{posix\_read(}\ct{int}} \ensuremath{\cv{file\_descriptor}}, \ensuremath{\ct{void*}} \ensuremath{\cv{buffer}}, \ensuremath{\ct{ulong}} \ensuremath{\cv{count}\cf )}}$\\
$\texttt{\ensuremath{\tab\calls{\cf{posix\_read}}}; \ensuremath{\cm{// system function defined by the OS}}}$\\
$\texttt{\ensuremath{\ck{static}} \ensuremath{\ct{long}} \ensuremath{\cf{stdin\_read(}\ct{void*}} \ensuremath{\cv{buffer}}, \ensuremath{\ct{ulong}} \ensuremath{\cv{count}\cf )} \ensuremath{\calls{\cf{posix\_read}}} \{}$\\
$\texttt{\ensuremath{\tab\ck{return}} \ensuremath{\cf{posix\_read(}\cv{STDIN\_FILENO}}, \ensuremath{\cv{buffer}}, \ensuremath{\cv{count}\cf )}; \}}$\\
\\
$\texttt{\ensuremath{\ck{interface}} \ensuremath{\ct{Input\_Stream<}\cf{'a}\ct >} \{}$\\
$\texttt{\ensuremath{\tab\ct{char}} \ensuremath{\cf{get\_char()}} \ensuremath{\calls{\cf{'a}}};}$\\
$\texttt{\ensuremath{\tab}… \} \ensuremath{\cm{// other usefull functions}}}$\\
$\texttt{\ensuremath{\ck{static}} \ensuremath{\ct{void}} \ensuremath{\cf{do\_stuff<'a>(}\ct{Input\_Stream<}\cf{'a}\ct >} \ensuremath{\cv{stream}\cf )} \ensuremath{\calls{\cf{'a}}} \{}$\\
$\tab\texttt{… \} \ensuremath{\cm{//}} \ensuremath{\cm{could have any well-typed code}}}$%
\end{minipage}}

\pars{}$\c{\t{Input\_Stream<}\f{'a}\t >}$ is conceptually similar
to a generic $\c{\t{Input\_Stream<T>}}$, where $\ct T$ denotes a
type-parameter.

For any base-callability $\f c$ one can then call $\cf{do\_stuff<}\f c\cf >$and
pass it an instance of a class that implements $\ct{\t{Input\_Stream<}}\f c\ct{\t >}$.
For example, a call $\cf{do\_stuff<[]>(}\v{a\_stream}\cf )$ cannot
(even indirectly) read from a file, since it doesn’t have the callability
$\cf{posix\_read}$ or any callability which could indirectly call
it. Another important benefit of our system is that we can even reason
over a call like $\cf{do\_stuff<[stdin\_read]>(}\v{a\_stream}\cf )$,
it can read-files, but only standard-input (the file identified by
$\cv{STDIN\_FILENO}$). As is usual in OO languages, for such a call
to type-check, $\v{a\_stream}$’s static-type must be a subtype
of $\ct{Input\_Stream<}\cf{[stdin\_read]}\ct >$ but one could always
perform a run-time cast, but risk getting an exception.

This system is designed to soundly enforce reasoning: by looking at
the declarations of functions $\f f$ and $\f g$, and all functions
transitively referenced in their base-capabilities, one can statically
determine whether $\f f$ can call $\f g$. In addition, it is also
designed to work in an environment with dynamic-code loading, if you
have reasoned that one function cannot call another, no matter what
additional code you load or dynamically invoke, that guarantee still
holds. We have also looked at adding other features to reduce the
burden of our annotations: such as wildcards and annotation inference.

\lfoot{\textsuperscript{1}\email{isaac@ecs.vuw.ac.nz} \textsuperscript{2}\email{marco.servetto@ecs.vuw.ac.nz} }

\rfoot{}
\end{document}

%
\section{Introducing Quasi Quotation}

Lisp~\cite{pitman1980special}, MetaML~\cite{moggi1999idealized}, Template Haskell~\cite{sheard2002template} and many other approaches use Quasi Quotation (QQ).
This can be supported by two kinds of special parenthesis as a syntactic sugar to manipulate Abstract Syntax Trees (ASTs).
Lisp uses (\Q@`@) and (\Q@,@), while here we use
\Q@[|  |]@  and \Q@$\$$(  )@ (as Template Haskell) 
for better readability.

\noindent
The following example explains their meaning: 

\begin{lstlisting}
Int res0=x*x $\Comment{normal code}$
Expr<x:Int$\vdash$Int> res1=[| x*x |]
  $\Comment{new Mul(new Var("x"),new Var("x"))}$
Expr<x:Int$\vdash$Int> res2=[| x* $\$$(12+3) |]
  $\Comment{new Mul(new Var("x"),new Lit(15))}$
\end{lstlisting}

\noindent
Here \Q@res1@ is initialized using a ``quotation'' of code.
This is equivalent to generating the abstract syntax tree by hand, as shown in the comment.
\Q@res2@ is initialized using a ``quasi-quotation'' of code: a chunk of code with a hole, that is filled by executing an expression.

There are different ways to type QQ.
In an expression based language, 
the simplest way is to just have a primitive \Q@Expr@ type,
representing every type of code.
This ensures the result is syntactically well formed, but
it allows for the generation of ill-typed code.
Another option, for example used by MetaML,
is to have a parametrized type.
Here we use \Q@Expr<@$\Gamma\vdash T$\Q@>@; where
$\Gamma$ keeps track of the free variables and $T$ is the expected type
of the result.
This approach is restrictive (see~\cite{servetto2014meta})
 but ensures that all the resulting code is well typed.

Usually programming with QQ requires thinking about the desired method body,
 and often allows generating a more efficient body by generating code specialized for some input value.
A typical example is about generating a \Q@pow@ function, where the exponent is well known.
The ``inefficient'' version would be:

\begin{lstlisting}[language=ML]
fun power(x:Int,n:Int):Int 
  = if (n=0) then 1 else x*power(x,n-1);
power7_a=$\lambda$ x:Int. power x 7;
\end{lstlisting}

\noindent A more ``efficient'' version using QQ would be:

\begin{lstlisting}[language=ML]
fun powerAux(n:Int):Expr<x:Int$\vdash$Int> 
  = if (n=0) then [|1|] else [|x * $\$$(powerAux(n-1)) |];

fun powerGen(n:Int): Int->Int
  = compile([| $\lambda$ x. $\$$(powerAux(n)) |]);

power7_b=powerGen 7;
\end{lstlisting}

\noindent As you can see, by generating the abstract syntax tree, we can obtain exactly:

\begin{lstlisting}[language=ML]
power7_b=$\lambda$ x.x*x*x*x*x*x*x*1;
\end{lstlisting}

\noindent On most machines, \Q@power7_b@ runs faster than \Q@power7_a@.
Metaprogramming applications include more than just speed boosts, but we start with this example because it is very popular and simple.

The code generator above is quite compact, but it is actually \textbf{hiding} (not removing) the complexity of meta-programming.
A common approach to make the code more explicit is to extract
logical concepts as functions.
We can see that the code is proceeding in an inductive fashion:
we know the code for \Q@pow 0@, and given the code for
\Q@pow n@  we can create the code for \Q@pow (n+1)@.
Thus we define \Q@base@ and \Q@inductive@ functions, and we
use them inside \Q@powerAux@:

\begin{lstlisting}[language=ML]
fun base():Expr< x:Int$\vdash$Int > 
  = [| 1 |]
fun inductive(code:Expr< x:Int$\vdash$Int >):Expr< x:Int$\vdash$Int >
  = [| x * $\$$(code) |]

fun powerAux(n:Int):Expr< x:Int$\vdash$Int >
  = if (n=0) then base()
   else inductive ( powerAux(n-1) );
\end{lstlisting}

\noindent Then, we have to bind \Q@x:Int@ to a parameter in a function.
This is an important conceptual action and thus we make it a function:

\begin{lstlisting}[language=ML]
fun lambdaX(code:Expr< x:Int$\vdash$Int >):Expr<$\vdash$Int->Int > 
  = [| $\lambda$ x. $\$$( code ) |]

fun powerGen(n: Int):Int->Int
  = compile(lambdaX(powerAux(n) ))
\end{lstlisting}

The code we obtain is much larger, but is not logically more complex --- it is just showing the logical structure better.
Note how since QQ works near the code representation,
a function \Q@Int->Int@ is radically different from
code with a free variable \Q@x:Int@$\vdash$\Q@Int@, while they are 
logically similar concepts.


We propose Iterative Composition (IC):
while the unit of composition in QQ is the single AST node, 
IC enforces a higher level of abstraction and does not work directly on the AST.
The unit of composition in IC is a \emph{Library}:
a class body, containing methods and possibly nested classes.
Libraries are self contained in the sense that they contain no free variables.
This avoids all scope-extrusion related problems, and (as shown later) enforces local reasoning.

IC has already been presented in other work~\cite{servetto2014meta};
 IC expressive power is shown by examples,
but is not compared with QQ; moreover such works suggested IC expressive power is inferior to QQ.
The core idea of IC is to  \emph{rely on  operators of code composition inspired by normal
code reuse}, but lifted to the expression level.
As a concrete example, in Java operators \Q@+@ and \Q@*@ can be used in the expression \Q@1+2*3@,
but the operator \Q@extends@ can only be used in the specific context of class declaration, as in

\begin{lstlisting}[language=Java]
class A extends B{/*some code*/ int m(){return 1+super.m();}}
\end{lstlisting}

In our proposed approach we lift \Q@extends@ and code literals to operator and constants
that can be used in conventional expressions.
Class declarations associate a class name with the result of an expression of type \Q@Library@. 
 We would write the former example as:

\begin{lstlisting}
A = Override[m<-superM]( 
  {/*code of B*/},
  { /*some code*/ int m(){return 1+this.superM();}}
  )
\end{lstlisting}

\noindent We support the conventional super call mechanism by annotating the operator with
the expected super call name: \Q@Override[m()<-superM()](...)@.


\noindent
We can rewrite our \Q@pow@ example 
in IC:

\begin{lstlisting}
Pow = {
  static method Library base()
   ={ method Num pow(Num x)= 1 }$\Comment{Code literal with 1 method "pow(x)"}$

  static method Library inductive()
   ={$\Comment{Code literal with 2 methods: "pow(x)", "superPow(x)"}$
    method Num pow(Num x)= x*this.superPow(x)
    method Num superPow(Num x)$\Comment{no body: it is an abstract method}$
    }
  static method Library inductive(Library code)
   = Override[pow(x)<-superPow(x)](code, this.inductive())
  
  static method Library generate(Num y)
   = if (y==0) then this.base();
     else this.inductive(generate(y-1))
  }
...
Pow7 = Pow.generate(7)
$\Comment{That would reduce into the desired code as follows:}$
Pow7 ={method Num pow(Num x)=x.x*x*x*x*x*x*x*1}
\end{lstlisting}

\noindent In more detail:
\begin{itemize}
\item
\Q@base()@ is a method with no parameter with \Q@Library@ return type.
This is equivalent to a non parametrised version of \Q@Expr@ in QQ.
However, our approach still guarantees that all the results are well typed.
\Q@base()@ returns a class with a single method \Q@pow(x)@,
returning 1.
\item
For the inductive case, the method \Q@pow(x)@ of \Q@inductive()@ is defined in terms of
another method (\Q@superPow(x)@), representing the delegation to
the former case in the inductive reasoning.
%Note that the declaration of \Q@superPow(x)@ is an abstract method: a method without body.
\item Method \Q@inductive(code)@ builds
the code for \Q@x+1@ from the code for \Q@x@.
Note how we use \Q@Override@ inside of a normal method body.
This \Q@Override@ will implements
\Q@superPow(x)@ using
the \Q@pow(x)@ body from the induction premise: the \Q@code@ parameter.
Then, \Q@superPow(x)@ is inlined.

\item Method \Q@generate(y)@ uses recursion to \textbf{iteratively compose} the result, using induction starting from
the base case.
Note how this method is logically identical to \Q@powerAux@. However,
since we always work on the actual self contained code neither \Q@lambdaX@ nor \Q@compile@ are needed.
\end{itemize}
Our approach builds on top of code composition operations like multiple inheritance and generics.
The literature offers \cite{barnett2004spec,burdy2005overview,muller2016viper} many successful efforts about proving the
\emph{semantic correctness} 
of code containing inheritance and generics.
On the other hand, static verification of code generated with meta programming is an open research problem.
We speculate our approach may offer the opportunity to solve this problem,
and by construction generate statically verified code,
by reusing techniques originally developed to verify normal object oriented code.
The contributions of our work are as follows:
\begin{itemize}
\item
In Section~\ref{s:verification}
by examples we demonstrate how to apply conventional object oriented verification techniques to IC.

\item
In Section~\ref{s:pattern1}
by examples we show that IC is as expressive as QQ, and that
generating code using a composition algebra
is a flexible and simple technique, if combined with
reasonable programming patterns.
\item In Section~\ref{s:study} we discuss our experience implementing metaprogramming libraries for a language where IC is the only support for code reuse. 
\end{itemize}

In this paper we do not present a formal language semantic. This is in part for space reasons
and in part because similar semantic has been formalized in former work~\cite{servetto2014meta};
here we aim to show programming patterns that use this expressive power in surprising and novel ways.


%-------------------------------------------------------------------------------------
%-------------------------------------------------------------------------------------


%\section{Background}
\subsection{Static Verification in OO}
There is a wealth of research in static verification, they usually follow the conventional notions of pre and post-conditions~\cite{?}: a method is annotated with a precondition which must hold when the method is called, and a postcondition which must hold when the method returns. In this paper, we will use the annotation /@requires($predicate$)/ to specify a precondition, and /@ensures($predicate$)/ to specify a postcondition; here $predicate$ is a boolean expression in terms of /this/, and the parameters of the method, and for the /@ensures/ case, the /result/ of the method call. For example one could write an implementation of exponentiation using repeated squaring like this:
\begin{lstlisting}
@requires(x != 0 || exp != 0) // since 0**0 is undefined.
@ensures(result == x**exp)
Int pow(Int x, Nat exp) {
	if (exp == 0)           return 1;
	else if (exp == 1)      return x;
	else if (exp % 2 == 0)  return pow(x * x, exp / 2); // even power
	else                    return x*pow(x, exp - 1);   // odd power
\end{lstlisting}
This says that for /pow/ to be `correct', whenever we don't have /x == 0/ and /exp == 0/, /pow(x, exp)/ must equal /x**exp/\footnote{We use the notation /x**y/ to mean `/x/ to the power of /y/'.}. For a \emph{call} /pow(x, exp)/ to be valid, either /x/ or /y/ must be non-zero.

The concepts of pre and post conditions can be extended to object oriented programming~\cite{?}, consider for example:
\begin{lstlisting}
interface List {
	@requires(this.contains(o))
	@ensures(this.get(result) == o)
	Nat find(Object o);
	...
}
\end{lstlisting}
This means that in order for a call /a_list.find(o)/ to be correct, /a_list/ must contain /o/. In order for an implementation of /List.find/ to be correct, /this.find(o)/ must return an index that corresponds to /o/, assuming that /this/ contains /o/. Note that \emph{abstract} methods (such as /List.find/), are always trivially correct, however the type system of the language should ensure that the methods of all \emph{objects} are implemented.

For simplicity, we only consider pre and post conditions in a purely functional OO language. Static verification can be extended to other cases, such as to verify class invariants~\cite{?}, termination~\cite{?}, function purity (where the language is otherwise imperative language) ~\cite{?}, aliasing~\cite{?} etc. However such properties are still typically encoded as pre and post-conditions.

The mechanism by which such programs are verified to be correct is also a large research area, one common approach is \emph{automated static verification}, where a tool tries to automatically verify code is correct during compilation, such tools typically encode the correctness of a program as a mathematical theorem, and use an SMT solver to verify it. Automated static verification is often easier to use and provides stronger guarantees compared to computer aided theorem proving~\cite{?} and runtime verification~\cite{?} (respectively). 

For simplicity, we will only consider automated static verification. The details as to how this is done is also out of scope of this paper. However, our approach should be easily extendible to other mechanisms of verification.

\subsection{Trait Composition}
Trait composition~\cite{?} uses \emph{traits} to represent the \emph{body} of a class, and allows composing traits together. Traits are similar to the conventional OO concept of classes, however they are not types. In this paper, we use the concept of \emph{flattening}~\cite{?} where trait operations flatten into a new AST. Their are multiple operators that can be applied to traits...

%TODO: Very bad example of inheritence! can easilly be acomplished with just pow1.pow.... come up with something better
\begin{lstlisting}
Trait pow1 = static class {
Int pow(Int x) {
return x; 
}
}

Trait pow2 = pow1.extend(static class {
Int pow(Int x) {
Int y = super_pow(x);
return y*y;
}
}
...
Pow2: pow2;
\end{lstlisting}

(a `static` class, )

\begin{lstlisting}
class Pow1 {
Int pow(Int x) {
return x; 
}
}

class Pow2 extends Pow1 {
Int pow(Int x) {
Int y = super.pow(x);
return y*y;
}
}
...
Pow2: pow2;
\end{lstlisting}

The main advantage of trait composition over conventional inheritance is that trait composition does \emph{not} induce sub-typing:
\begin{itemize}
	\item Trait operations that would be unsound in a sub-typing environment are allowed, such as removing methods or changing their types.
	\item The way the class is constructed is not exposed to users, this allows classes to be rewritten in a totally different way.
\end{itemize}
In addition, this makes traits much simpler to reason about, making it easier to compose them i'm more complicated ways~\cite{?}.

\subsection{Iterative Composition}
Iterative composition~\cite{?} takes the concept of trait composition one step forward, by treating traits as first class objects, and allowing arbitrary code to execute at meta-time. This allows for fully met-circular meta-programming. In this way, classes can be declared from arbitrary expressions (of type /Trait/), and methods can take and return /Trait/s. Consider again our /pow/ example:
\begin{lstlisting}
Pow: static class {
	Trait pow0 = static class {
		Int pow(Int x) { return 1; }
	}

	Trait pow1 = static class {
		Int pow(Int x) { return x; }
	}

	Trait powEven = static class {
		Int super_pow(Int x);
		Int pow(Int x) { return super_pow(x * x); }
	}
	Trait powOdd = static class {
		Int super_pow(Int x);
		Int pow(Int x) { return x*super_pow(x); }
	}

	Trait generate(Nat exp) {
		if (exp == 0)          { return pow0; }
		else if (exp == 1)     { return pow1; }
		else if (exp % 2 == 0) { return generate(exp / 2).extend(powEven);
		else                   { return generate(exp - 1).extend(powOdd); }}
}
\end{lstlisting}

In the above code, we have a normal looking recursive /generate/ method, however it returns a /trait/. The above method can now be called to generate the body of a class:

\begin{lstlisting}
Pow7: Pow.generate(7);
// Equivalent to:
Pow7: static class {
	Int pow(Int x) { return x*super1_pow(x); } // x ** 7

	private Int super1_pow(Int x) { return super2_pow(x * x); } // x ** 6
	private Int super2_pow(Int x) { return x*super3_pow(x); } // x ** 3
	private Int super3_pow(Int x) { return super4_pow(x * x); } // x ** 2
	private Int super4_pow(Int x) { return x; } // x ** 1
}
// Which could then be inlined to:
Pow7: static class {
	Int pow(Int x) { 
		Int x2 = x*x; // x**2
		Int x4 = x2*x2; // x**4
		return x*x2*x4; } // Since 7 = 1 + 2 + 4
}
\end{lstlisting}
\noindent Object oriented languages supporting static verification (SV) usually extend the syntax for method declarations
to support \emph{contracts} in the form of pre and post-conditions~\cite{Meyer:1988:OSC:534929}.
Correctness is defined only for code annotated with such contracts.

\IO{We say that a method is \emph{correct}, if whenever its precondition holds on entry, the precondition of every \MS{directly invoked method holds}, and the postcondition of the method holds \MS{when the method} returns.} Automated SV typically works by asking an automated theorem prover to verify that each method is correct individually, by assuming the correctness of every other method~\cite{barnett2004spec}. This process can be very slow \IO{and \MS{can} produce unexpected results}: \MS{since SV is undecidable correct code may not pass SV.
Many SV approaches are not resilient to
some standard refactoring techniques like 
method inlining. Sometimes SV even terminate for a time-out, exacerbating the impact of transformations like method inlining.}

Metaprogramming is often used to programmatically generate faster specialised code when some parameters are known in advance\IO{, this is particularly useful where the specialisation mechanism is too complicated for a generic compiler to automatically derive~\cite{Ofenbeck:2017:SGP:3136040.3136060}}
\IO{We could use metaprogramming to generate code together with contracts, and then once the metaprogramming has been run,} 
\MS{ensure the correctness of the resulting code
by applying SV.} However, the resulting code could be much larger than the input to the metaprogramming, and so it could take a long time to SV.
\MS{Moreover, one of the main goal of metaprogramming  is make it easy to generate many specialized versions of the same functionality}.
%, Even if the generated code was produced by using straightforward transformations and compositions over the input code, a SV might not verify it's correctness.
The aim of our work is to \MS{apply SV} only to the code manually wrote by the programmer, 
and to ensure that \MS{the} result of metaprogramming is \MS{instead} correct by construction.

\IO{Here we use the} disciplined form of metaprogramming \IO{introduced} by Servetto \& Zucca \cite{servetto2014meta}, which is based on trait composition and adaptation~\cite{scharli2003traits}.
Here a /Trait/ is a unit of code: a set of method declarations.
\MS{Those methods can be abstract, and they can
 be mutually recursive by using the implicit parameter /this/.}

\MS{As in~\cite{servetto2014meta}
we require that all the traits are well-typed
before they are used.
Moreover, in our proposed approach we 
annotating methods with pre//post-conditions, and 
we require that all traits are also correct.
/Trait/s directly written in the source code are proven correct by SV, while traits resulting from metaprogramming are correct by construction}.

\MS{Crucially, we improve the checking} performed by composition and adaptation of /Trait/s to also check that contracts \MS{are composed correctly}; thus ensuring the correctness of the result.
%
%The result of composing and adapting /Trait/s is also correct and well-typed.
%
\MS{Our metaprogramming approach} does not generate code \MS{from scratch}; \IO{rather} code is only generated by composing and adapting traits.
\MS{Each composition//adaptation step is guaranteed
to produce well typed and correct code}; \IO{thus also the result of metaprogramming is well typed and correct.}
Note that generated code may not be able to pass SV, since theorem provers are not complete.


SV handles /extends/ and /implements/ by verifying that every 
time a method is implemented//overriden, 
the Liskov substitution principle~\cite{Liskov:1994:BNS:197320.197383} is satisfied
by checking that the new contract implies the
overridden one. 
 In this way, there is no need to re-verify
inherited code in the context of the derived class.
This concept is easily adapted
to handle trait composition, which simply provides another way to implement an /abstract/ method.
When traits are composed,
it is sufficient
to match the contracts of the few composed methods
to ensure the whole result is correct.

In our examples we will use the notation /@requires($predicate$)/ 
to specify a precondition, and /@ensures($predicate$)/ 
to specify a postcondition; where $predicate$ is a boolean expression
in terms of the parameters of the method (including /this/), and for the /@ensures/ case, the /result/ of the method.
Suppose we want to implement an efficient exponentiation function, we could use recursion and the common technique of `repeated squaring':
\vspace{-1ex}
\begin{lstlisting}
@requires(exp > 0)
@ensures(result == x**exp) // Here x**y means x to the power of y
Int pow(Int x, Int exp) {
	if (exp == 1) return x;
	if (exp %2 == 0) return pow(x*x, exp/2); // exp is even
	return x*pow(x, exp-1); }  // exp is odd
\end{lstlisting}
If the exponent is known at compile time,
unfolding the recursion produces even more efficient code:
\vspace{-1ex}
\begin{lstlisting}[firstnumber=7]
@ensures(result == x**7) Int pow7(Int x) { 
  Int x2 = x*x; // x**2
  Int x4 = x2*x2; // x**4
  return x*x2*x4; } // Since 7 = 1 + 2 + 4
\end{lstlisting}
\vspace{-1ex}


Now we show how \MS{the technique of \emph{Iterative Composition} (introduced in~\cite{servetto2014meta} and
enriched by our contract composition check)
can be used to write a meta-\MS{program} that given an exponent, produces} code like the above.
\MS{Iterative Composition is a metacircular meta-programming technique relying on \emph{compile-time execution} (as ~\cite{sheard2002template}), thus a meta-program
 is just a function or a method wrote in the target programming language that is executed during compilation.}

\vspace{-1ex}
\begin{lstlisting}[firstnumber=11]
Trait base=class {//induction base case: pow(x) == x**1
  @ensures(result>0) Int exp(){return 1;}  
  @ensures(result==x**exp()) Int pow(Int x){return x;}
  }
Trait even=class {//if _pow(x)== x**_exp(), pow(x) == x**(2*_exp())
  @ensures(result>0) Int $\_$exp();
  @ensures(result==2*$\_$exp()) Int exp(){return 2*$\_$exp();}
  @ensures(result==x**$\_$exp()) Int $\_$pow(Int x);
  @ensures(result==x**exp()) Int pow(Int x){return $\_$pow(x*x);}
}
Trait odd=class {//if _pow(x)== x**_exp(), pow(x) == x**(1+_exp())
  @ensures(result>0) Int $\_$exp();
  @ensures(result==1+$\_$exp()) Int exp(){return 1+$\_$exp();}
  @ensures(result==x**$\_$exp()) Int $\_$pow(Int x);
  @ensures(result==x**exp()) Int pow(Int x){return x*$\_$pow(x);}
}
//`compose' performs a step of iterative composition
Trait compose(Trait current, Trait next){
  current = current[rename exp->$\_$exp, pow->$\_$pow];
  return (current+next)[hide $\_$exp, $\_$pow];}
@requires(exp>0)//the entry point for our metaprogramming
Trait generate(Int exp) {
  if (exp==1) return base;
  if (exp%2==0) return compose(generate(exp/2),even);
  return compose(generate(exp-1),odd);
};
class Pow7: generate(7) //generate(7) is executed at compile time
//the body of class Pow7 is the result of generate(7)
/*example usage:*/new Pow7().pow(3)==2187//Compute 3**7
\end{lstlisting}
\vspace{-1ex}

The traits /base/, /even/, and /odd/ are the basic building blocks we will use to compute our result. They will be compiled, typechecked and statically verified before the method /generate(exp)/ can run.
As you can see in line /37/, a class body can be an expression in the language itself.
At compile time such an expression will be run and the resulting /Trait/ will be used as the body of the class.
For example, we could write /class Pow1: base/; this would generate a class such that /new Pow1().pow(x)==x**1/.
The other two traits have abstract methods; implementations for /$\_$pow(x)/ and /$\_$exp()/ must be provided. However, given the contract of /pow(x)/,
and the fact that /even/ and /odd/ have both been statically verified,
if we supply method bodies respecting these contracts, we will get \emph{correct} code, without the need for further static verification.
Many works in literature allow adapting traits by renaming or hiding methods\cite{servetto2014meta,reppy2007metaprogramming,liquori2008feathertrait}. Hiding a method may also trigger inlining if the method body is simple enough or used only once.
Since all occurrences of names are consistently renamed, \textbf{renaming and hiding preserve code correctness}.

The /compose(current,next)/ method starts by renaming the /exp()/ and /pow(x)/ methods of /current/
so that they satisfy the contracts in /next/ (which will be 
/even/ or /odd/).
The /+/ operator is the main way to compose traits%
~\cite{scharli2003traits,LagorioSZ09}.
The result of /+/ will contain all the methods from both operands. 

Crucially, it is possible to sum traits where a method is declared in both operands; in this case at least one of the two competing methods needs to be abstract, and the signatures of the two competing methods need to be \emph{compatible}.
To make sure that the traditional /+/ operator also handles contracts, we need to require that the contract annotations of the two competing methods  are \emph{compatible}.
For the sake of our example, we can just require them to be syntactically identical. Relaxing this constraint is an important future work.
Thanks to this constraint \textbf{the sum operator also preserves code correctness}. %\IO{There are many variations of the /+/ operator, in particular, we could easily extend our contract matching to work with an nary operator}.

The sum is executed when the method /compose/
%\IO{\footnote{\IO{a generic implementation of this method that renames and hides conflicting methods has been implemented L42~\cite{l42}}}}
runs: if the matched contracts are not identical an exception will be raised. A leaked exception during compile-time metaprogramming would become a compile-time error. 
Our approach is very similar to~\cite{servetto2014meta}, and does not guarantee the success of the code generation process, rather it guarantees that if it succeeds, correct code is generated.

Finally the /$\_$pow(x)/ and /$\_$exp()/ method are hidden, so that the structural shape of the result is
the same as /base/'s.
As you can see, /Trait/s are first class values and can be manipulated with a set of primitive operators that preserve code correctness and well-typedness.
In this way, by inductive reasoning, we can start from the /base/ case and then recursively compose /even/ and /odd/ until we get the desired code.
Note how the code of /generate(exp)/ follows the same scheme of the code of /pow(x,exp)/ in line 1.

To understand our example better, imagine executing the code of /generate(7)/ while keeping /compose/ in symbolic form. We would get the following (where /c/ is short for /compose/):
\vspace{-1ex}
\begin{lstlisting}[numbers=none]
generate(7) == c(generate(6),odd) == ...
 == c(c(c(c(base,even),odd),even),odd)
\end{lstlisting}
\vspace{-1ex}
As /base/ represents /pow1(x)/; /c(base,even)/ represents /pow2(x)/. Then \Q@c(/*pow2(x)*/,odd)@ represents \Q@pow3(x)@, \Q@c(/*pow3(x)*/,even)@ represents \Q@pow6(x)@, and finally,
\Q@c(/*pow6(x)*/,odd)@ represents \Q@pow7(x)@.
The code of each /$\_$pow(x)/ method is only executed once for each top-level /pow(x)/ call, so the /hide/ operator can inline them.
Thus, the result could be identical to the manually optimized code in line 7.

%\IO{We are investigation how an additional check can be performed to ensure the resulting code has specific contracts. However, our approach does guarantee that the result will be correct according to whatever contracts it contains.} 

Our approach, as presented in this extended abstract, only guarantees that code resulting from metaprogramming follows its own contracts, it does
not statically ensure what those contracts may be. As future work, we are investigating how the resulting contracts can be ensured to have a particular meaning or form.
To do so, we need to allow assertions on the contracts of /Trait/s to be used within pre//post conditions.
For example we could allow post conditions like\\*
%\begin{lstlisting}[numbers=none]
/@ensures(result.$\mathit{methName}$.ensures ==\ $\mathit{predicate}$)/ \\*
%\end{lstlisting}
to mean that the resulting /Trait/ has
a method
called $\mathit{methName}$, whose /@ensures/ clause is syntactically identical to  /$predicate$/; whilst
\\*
/@ensures(result.$\mathit{methName}$.ensures ==>\ $\mathit{predicate}$)/
\\*
would use a static verifier to ensure that $\mathit{methName}$'s /@ensures/ clause logically implies $\mathit{predicate}$.
With these two features we could annotate the method /generate(exp)/ in line 32 above as:
\begin{lstlisting}
@requires(exp>0)
@ensures(result.exp().ensures ==> (result==exp))
@ensures(result.pow(x).ensures == (result==x**exp()))
Trait generate(Int exp) {...}
\end{lstlisting}

\vspace{-1ex}
In this way, we could statically verify the /generate(exp)/ method, however we fear such verification will be too complex or impractical. 
We could instead automatically check the above postconditions after each call to /generate(exp)/. If /generate(exp)/ is used to define a class (such as /Pow7/ above), we will guarantee that such class has the expected contracts, before it is used. Thus
there is no need to ensure the correctness of the metaprogram itself: such runtime checks are sufficient to ensure that after compilation, the code produced by metaprogramming has its expected behaviour.
%\IODel{In this case we could defer those difficult//novel predicates to run-time checks, without losing much safety:
%Iterative Composition execute metaprogramming code at
%compile time, thus even run-time verification of metaprograms would happen at compile time. This consideration could result in a crucial design decision: code performing metaprogramming does not need to be verified by SV to produce code annotated with the desired contracts; it may be sufficient to apply some type of runtime verification during compile-time execution.} \IOComm{I did a major rewording since we actually have multiple compile-times and run-times, so your version is confusing, hopefully my version makes the point more clear.}
%For example, the following code:
%\vspace{-1ex}
%\begin{lstlisting}[numbers=none]
%@ensures(new Pow7().exp()==7&&Pow7.pow.ensures=="result==x**exp()")
%class Pow7: generate(7)
%\end{lstlisting}
%\vspace{-1ex}
%may require the static verifier to check that the execution of
%/new Pow7().exp()/ will deterministically reduce to /7/, and that the /ensures/ clause of 
%/Pow7.pow/ is syntactically equivalent to 
%/result==x**exp()/. Note how this final step of static verification does not need to re-verify the body of
%/Pow7.pow/ and only needs to do a coarse grained 
%determinism check on the implementation of /Pow7.exp()/, before symbolically executing it.

In conclusion, by leveraging over conventional OO static verification techniques, we have extended the Iterative Composition form of metaprogramming with a simple contract compatibility check, to statically ensure the correctness of code produced by such metaprogramming. In particular, our approach does not require static verification of the result of metaprogramming, but only requires verification of code present directly in source code.
%\section{Combining Contracts and Trait Composition}
We can handle the composition of traits with contracts in a similar way to what we described for class-based inheritance: the methods in the result of /trait1.extend(trait2)/ contain the /@requires/ and /@ensures/ clauses of the corresponding method (if any) in /trait1/ and /trait2/. This ensures that any \emph{calls} to such methods in /trait1/ and /trait2/ are still correct, and any deductions made from the postconditions of such methods still hold. If however the method is abstract in either /trait1/ or /trait2/, but not both, the implementation of the method must have at least all the /@ensures/ clauses of the abstract method, and the abstract method must have at least all the /@requires/ clauses of the implemented method. This ensures that the implementation is still a correct implementation for the abstract method. For the purposes of this rule, if the method is implemented in both /trait1/ and /trait2/, the version in /trait1/ will be considered abstract, and their will be an implemented copy of it with a /super_/ prefix (and the same contracts) in /trait1/.
Consider the following examples illustrating how this works:
\begin{lstlisting}
Trait t1: class {
  @requires($R_1$)
  @ensures($E_1$)
  Void foo();
}.extend(class {
  @requires($R_2$)
  @ensures($E_2$)
  Void foo();
});

// Identical to:
Trait t1: class {
  @requires($R_1$)
  @requires($R_2$)
  // or @requires($R_1$ || $R_2$)
  @ensures($E_1$)
  @ensures($E_2$)
  // or @ensures($E_1$ && $E_2$)
  Void foo();
};

Trait good: class {
  @requires($R_1$)
  @requires($R_2$)
  @ensures($E_1$)
  Void foo();
}.extend(class {
  // Precondition is weaker, postcondition is stronger
  @requires($R_1$)
  @ensures($E_1$)
  @ensures($E_2$)
  Void foo() { ... }
});

// Error!
Trait error: class {
  @requires($R_1$)
  @ensures($E_1$)
  @ensures($E_2$)
  Void foo();
}.extend(class {
  // Error: precondition is stronger and postcondition is weaker
  @requires($R_1$)
  @requires($R_2$)
  @ensures($E_1$)
  Void foo() { ... }
});

Trait really_good: class {
  @requires($R_1$)
  @requires($R_2$)
  @ensures($E_1$)
  @ensures($E_2$)
  Void foo() { ... }
}.extend(class {
  // Stronger precondition, weaker postcondition
  @requires($R_1$)
  @requires($R_2$)
  @requires($R_3$)
  @ensures($E_1$)
  super_foo();

  // Weaker precondition, stronger postcondition
  @requires($R_1$)
  @ensures($E_1$)
  @ensures($E_2$)
  @ensures($E_3$)
  Void foo() { ... }
})
\end{lstlisting}


\subsection{Examples}
...

%\section{Examples of Our Technique}\label{s:ex}
We now show how the technique described above in \sref{combining} can be used with iterative composition to programmatically generate guaranteed-correct code.

\subsection{Recursive Composition Example}
We now extend our /pow_generate/ example from \sref{bic} to generate correct code, without needing to run a static verifier each time it is called:
\begin{lstlisting}
Trait pow_generate(Nat exp) {
	if (exp == 0)
		return static class { 
			@ensures(result == 0)
			Nat exp() { return 0; }

			@requires(x != 0)
			@ensures(result == x**exp())
			Int pow(Int x) { return 1; }
		};
	else if (exp == 1)
		return static class {
			... // similar to the above, but without the @requires
		};
	else if (exp % 2 == 0)
		return pow_generate(exp/2).extend(static class {
			Nat super_exp();

			@ensures(result == x**super_exp())
			Int super_pow(Int x);
			
			@ensures(result == 2*super_exp())
			Nat exp() { return 2*super_exp(); }

			@ensures(result == x**exp())
			Int pow(Int x) { return super_pow(x*x); }
		});
	else 
		return pow_generate(exp - 1).extend(static class {
			... // similar to the above
		});}
\end{lstlisting}

Each of the trait literals above (the /static class \{...\}/ expressions) can be statically verified as being correct. The function /pow_generate/ then combines these trait literals with our /extend/ operator, guranteeing that the result (if any) is also correct. The idea is that /exp/ and /pow/ represent the current exponent and power function we are generating, /super_exp/ and /super_pow/ correspond to the exponent and power function of the recursive call. Though our trait literals are more verbose than in our non verified version, they ensure that the contract matching performed by /extend/ will succeed. 

The idea is that /pow_generate($e$)/ (where $e > 0$) will return a literal of the following form:
\begin{lstlisting}
static class {
	@ensures(...)
	Nat exp() { ... } // Will return $e$ when called

	@ensures(result == x**exp())
	Int pow(Int x) { ... }
	
	... // and perhaps some private super_ methods
}
\end{lstlisting}
Then /pow_generate($e^\prime$)/ (where $e^\prime > e$) will /extend/ this trait with one of the form:
\begin{lstlisting}
static class {
	Nat super_exp();

	@ensures(result == x**super_exp())
	Int super_pow(Int x);
	
	@ensures(...)
	Nat exp() { ... } // Will return $e^\prime$ when called

	@ensures(result == x**exp())
	Int pow(Int x) { ... }
}
\end{lstlisting}

Because /exp/ and /pow/ are implemented in both /pow_generate($e$)/ and /pow_generate($e^\prime$)/, /extend/ will add a /super_/ prefix to the ones in /pow_generate($e$)/. Thus /pow_generate($e^\prime$)/ will return:
\begin{lstlisting}
static class {
	@ensures(...)
	Nat super_exp() { ... } // Will return $e$ when called

	@ensures(result == x**super_exp())
	Int super_exp(Int x) { ... }
	
	... // And some private methods (which will not clash with the above)
}.extend(static class {
	Nat super_exp();

	@ensures(result == x**super_exp())
	Int super_pow(Int x);
	
	@ensures(...)
	Nat exp() { ... } // Will return $e^\prime$ when called

	@ensures(result == x**exp())
	Int pow(Int x) { ... }
})
\end{lstlisting}

The contracts of both operands of the /extend/ are compatible, and so the above will succeed and produce something of the form:
\begin{lstlisting}
static class {
	@ensures(...)
	Nat exp() { ... } // Will return $e^\prime$ when called

	@ensures(result == x**exp())
	Int pow(Int x) { ... }

	@ensures(...)
	private Nat super_exp() { ... } // Will return $e$ when called

	@ensures(result == x**super_exp())
	private Int super_exp(Int x) { ... }
	
	... // maybe some more private methods
}
\end{lstlisting}

Note that /pow_generate/ never recursively calls /pow_generate(0)/, so the /@requires/ clause in the result of /pow_generate(0)/ will not cause any problems with our contract matching.

Though our system guarantees that the result of /pow_generate/ is `correct' (i.e. the methods in the return trait satisfy their contracts), it does not say what methods will be in the result or what their contracts will be. We could statically verify /pow_generate/ itself, however an easier option would be to check at runtime that the result of /pow_generate($e$)/ is of the form:
\begin{lstlisting}
static class {
	@requires(true)
	Nat exp() { ... }

	@requires(x != 0) // if $e$ == 0
	@requires(true) // otherwise
	@ensures(result == x**exp())
	Int pow(Int x) { ... }
}
\end{lstlisting}
This could be done with introspection on the AST of the resulting trait. We also need to check that calling the above /exp()/ method produces $e$. However, static methods in traits cannot be directly called, instead we could dynamically create a class from the trait and call /exp()/ on the result. Alternatively, we could manually `interpret' the method's AST.
\subsection{Iterative Composition Example}
We can use introspection to automatically generate code for an input trait. Consider for example the following interface\footnote{An abstract class with only abstract methods.}):
\begin{lstlisting}
Account: interface {
	Nat income();

	@ensures(result <= this.income())
	Nat expenses();
};
\end{lstlisting}

\noindent The idea being that an account cant spend more money than they it has.

Now Suppose we want to create a new type of account, a combined one with multiple sub accounts:
\begin{lstlisting}
BuisnessAccount: class implements Account {
	Account salary;
	Account sales;
	Account property;
	
	Nat income() {
		return this.salary.income() 
			+ this.sales.income() + this.property.income(); }
	
	@ensures(result <= this.income())
	Nat expenses() { 
		return this.salary.expenses() 
			+ this.sales.expenses() + this.property.expenses();}
		
	...
};
\end{lstlisting}

Now we can statically verify the above code, but what if we want to combine accounts like this frequently? This could take lots of time for both programmers and automated static verifiers, instead we can create a function /combine_accounts/ that does this for us:
\begin{lstlisting}
// Equivalent to the above
BuisnessAcount: combine_accounts(class { 
	Account salary;
	Account sales;
	Account property;
	... });
\end{lstlisting}

Were we define /combine_accounts/ like this:
\begin{lstlisting}
Trait combine_accounts(Trait input) {
	Trait res = input.extend(class implements Account {
		Nat income() { return 0; }

		@ensures(result <= this.income())
		Nat expenses() { return 0; }
	});
	for (FieldDeclaration fd : input.fields()) {
		res = res.extend(class implements Account {
			Nat super_income();
			
			@ensures(result <= this.super_income())
			Nat super_expenses();

			Account account; // Will be renamed to fd.name
			Nat income() { 
				return this.super_income() + this.account.income(); }

			@ensures(result <= this.income())
			Nat expenses() { 
				return this.super_expenses() + this.account.expenses(); }
		}.rename("account", fd.name)); }
	return res;
}
\end{lstlisting}

\noindent Where /trait.fields/ returns AST structures for each field in the given trait, and /trait.rename($a$, $b$)/ returns /trait/ but with the declaration named $a$ renamed to $b$. The code above is straightforward: it adds an implementation of /Account/ (suitable if /input/ has no fields) to the /input/ trait, it then iterates over every field in /input/ and adds its corresponding /income()/ and /expenses()/ to the implementations in /res/. We use contract matching on the  /super_/ methods in the same was as in /pow_generate/. As with /pow/, the above two trait literals can be statically verified, and so we guarantee that the result of /combine_accounts/ is correct.
%Related work

Class invariants are a fundamental part of the design by contract methodology. 
Many languages and tools support some form of invariant verification (e.g. Eiffel~\cite{Meyer:1992:EL:129093}, D~\cite{Alexandrescu:2010:DPL:1875434}, JML~\cite{Burdy2005}, Spec\#~\cite{Barnett:2004:SPS:2131546.2131549}).
%In order to be verified, the invariant needs to be expressed in some formal way.
Here we focus on multi-object invariants: the class invariant of a given object may depend upon the observable behaviour of any object referenced in its Reachable Object Graph (ROG).

\noindent\textit{Security and DMZ:}
Static verification let us reason about a complete program
and verify its correctness.
Traditional static verification is like a mathematical proof: is valid if it is \textbf{all correct},
but a single error invalidates all the claims.
Thus, it is hard to perform verification on large programs, or when independently maintained third party libraries
are involved.
To solve this issue, static verification systems are starting to consider a verified core
and a run-time verified boundary.

You can see our approach as an extremely modularized version of such system:
every class is its own demilitarized zone, and the rest of the code 
could have Byzantine behaviour.

Every class that compiles/type checks should be protected against breakage,
 independently of the code that uses this class or any other surrounding code.
 That is, our approach works both with open world assumption and in a library setting.


\saveSpace
\section{Conclusion}
\label{s:conclusion}
\saveSpace

Static verification requires great effort, but can ensures all invariants \textbf{always} holds, thus all objects are always coherent.

However, Static verification is very heavy weight, and often impractical.
In the context of a conventional OO language with imperative features,
we propose an \textbf{ultra-lightweight} verification approach,
where the programmer specifies \textbf{only} the desired class invariants as an 
\Q@invariant()@ method written in the language itself.
This is much more convenient with respect to requiring the specification of methods pre and post conditions,
since the number of classes is usually order of magnitude smaller then the number of methods,
and a fully annotated program requires to write down 
pre-post conditions for each methods, encoding a generalization of its behaviour
in the dedicated specification language.
This means that, even in the best case scenario, 
using pre-post conditions
the user is required to specify the program semantic twice:
first in the specification language and then in the underlying programming language.


With just invariants, our system will then 
\textbf{soundly ensure invariants of all objects involved in the execution}.
Our approach do not rely on assumption over the behaviour of methods/classes;
except for the language semantics and the type system guarantees.
Methods are just treated as black-boxes, producing a result or throwing an error.

Of course, there is a catch: this result is obtained by modifying/instrumenting the
semantic of the language, so that (as for type casts) \textbf{violations are detected at run-time}, and exceptions
are throw in order to stop the execution before involving any broken object.

\noindent\textit{SIC as extended type system:}
The philosophy of our approach is to be like an extended type system: 
\begin{itemize}
\item The programmer decides to annotate a field with a certain type, or the class with a certain invariant.
\item If that is a valid type, or a valid invariant, the user is not questioned in its intent.
\item The system enforce that field will only contain values or that type, or that instance of that class
will respect that invariant.
\end{itemize}
This is in sharp contrast with most work in RV, that is often conceived more as a tool to ease debugging:
both deciding the invariant and enforcing it is controlled by the programmers.


\catcode\Slash=12% turn of my awesome slash
\bibliographystyle{eptcs}
\bibliography{paper}
\clearpage
\appendix
\appendix
\section{Proof} 
\label{s:proof}

\begin{theorem}[Sound Validation]
	if $c:\Kw{Cap};\emptyset\vdash \e: \T$ and
	$c\mapsto\Kw{Cap}\{\_\}|\e\rightarrow^+ \sigma|\ctx_v[r_l]$, then
	either $valid(\sigma,l)$ or $\mathit{trusted}(\ctx_v,r_l)$.
\end{theorem}

We believe this property captures very precisely our statement in Section~\ref{s:validation}.

It is hard to prove Sound Validation directly,
so we first define a stronger property,
called \emph{Stronger Sound Validation} and
show that it is preserved during reduction by means of conventional 
Progress and Subject Reduction (Progress is one of our assumption,
while Subject Reduction relies heavily on SubjectReductionBase).
That is,
Progress+Subject Reduction $\Rightarrow$ Stronger Sound Validation,
\\*and Stronger Sound Validation $\Rightarrow$ Sound Validation.

\subsection{Stronger Sound Validation $\Rightarrow$ Sound Validation}

Stronger Sound Validation depends on 
$\mathit{wellEncapsulated}$, $\mathit{monitored}$
and $OK$:

\noindent\textbf{Define} $\mathit{wellEncapsulated}(\sigma,\e,l_0)$:\\*
\indent$\forall l \in \mathit{erog}(\sigma,l_0), \text{not}\ \mathit{mutatable}(l,\sigma,\e)$

\noindent The main idea is that an object is well encapsulated if its encapsulated state is safe from
modification. 

\noindent\textbf{Define} $\mathit{monitored}(\e,l)$:\\*
\indent$\e=\ctx_v[M(l,\e_1;\e_2)]$ and either $\e_1=l$ or $l$ is not inside $\e_1$.

\noindent An object is monitored if the execution
is currently inside of a monitor for that object, and
the monitored expression $\e_1$ does not
contains $l$ as a \emph{proper} subexpression.

A monitored object is associated with an expression that can not observe it, but may 
reference its internal representation directly.
In this way, we can safely modify its representation before checking for the invariant.

The idea is that at the start the object will be valid and $\e_1$ will contain $l$;
but during reduction, the $l$ reference will be used in order to
give access to the internal state of $l$; only after that moment, the object may become invalid.


\noindent\textbf{Define} $OK(\sigma,e)$:\\
\indent $\forall l\in\dom(\sigma)$
  either\\
\indent\indent 1. $\mathit{garbage}(l,\sigma,\e)$\\
\indent\indent 2. $\mathit{valid}(\sigma,l)$ and $\mathit{wellEncapsulated}(\sigma,\e,l)$\\
\indent\indent 3. $\mathit{monitored}(\e,l)$

Finally, the system is in a valid state with respect to validation
if for all the objects in the memory, one of these 3 cases apply:
%the class of the object has no invariant method;
the object is not (transitively) reachable from the expression (thus can be garbage collected);
the object is valid, and the object is encapsulated;
or the object is currently monitored.

\begin{theorem}[Stronger Sound Validation]
if $c:\Kw{Cap};\emptyset\vdash \e_0: \T_0$ and
$c\mapsto\Kw{Cap}\{\_\}|\e_0\rightarrow^+ \sigma|\e$, then
$OK(\sigma,\e)$
\end{theorem}
\noindent Starting from only the capability object,
any well typed expression $\e_0$ can be reduced for an arbitrary amount of steps,
and $IOK$ will always hold.
\\
\begin{theorem} Stronger Sound Validation $\Rightarrow$ Sound Validation
\end{theorem}
\begin{proof}
\noindent By Stronger Sound Validation, each $l$ in the current redex must be $OK$:
\begin{enumerate}
	\item If $l$ is garbage, it cannot be in the current redex, a contradiction.
	\item If $\mathit{valid}(\sigma,l)$, then $l$ is valid, so thanks to Determinism
	no invalid object could be observed.
	\item Otherwise, if $\mathit{monitored}(\e,l)$ then either:
	\begin{itemize}
	 \item we are executing inside of $\e_1$ thus the current redex is inside of a sub-expression of the monitor that does not contain $l$, a contradiction.
	 \item or we are executing inside $\e_2$:
	 by our reduction rules, all monitor expressions start with 
	 $\e_2=l$\Q@.validate()@, thus the first execution step
	 of $\e_2$ is trusted. Following execution steps are also trusted, since by well formedness the body of invariant methods only use \Q@this@ (now translated to $l$) to access fields.
	\end{itemize}
\end{enumerate}
In any of the possible cases above, Sound Validation holds for $l$, and so it holds for all redexes.
\end{proof}

\subsection{Subject Reduction}

\noindent\textbf{Define} $\text{fieldGuarded}(\sigma,\e)$:\\*
\indent$\forall \ctx$ such that $\e=\ctx[l\singleDot\f] $
and $\Sigma^\sigma(l).f=\Kw{capsule}\,\_$, and $\f\mathrel{\mathit{inside}} \Sigma^\sigma(l).\mathit{validate}$\\*
\indent\indent either 
$\forall T, \forall C, \Sigma^\sigma;\x:\Kw{mut}\,C\,\not\vdash\ctx[\x]:T$, or\\*
\indent\indent $\ctx=\ctx'[$\Q@M(@$l$\Q@;@$\ctx''$\Q@;@$\e$\Q@)@$]$ and $l$ is contained exactly once in $\ctx''$

That is, all \emph{mutating} capsule field accesses are individually guarded by monitors.
Note how we use $C$ in $\x:\Kw{mut}\,C$ to guess the type of the accessed field,
and that we use the full context $\ctx$ instead of the evaluation context $\ctx_v$
to refer to field accesses everywhere in the expression $\e$.


\begin{theorem}[Subject Reduction]
if $\Sigma^{\sigma_0};\emptyset\vdash e_0: T_0$,
$\sigma_0|e_0\rightarrow \sigma_1|e_1$,
$OK(\sigma_0,\e_0)$
and
$\mathit{fieldGuarded}(\sigma_0,\e_0)$
then
$\Sigma^{\sigma_1};\emptyset\vdash e_1: T_1$,
$OK(\sigma_1,e_1)$ and
$\mathit{fieldGuarded}(\sigma_1,\e_1)$
\end{theorem}

\begin{theorem}
	Progress + Subject Reduction $\Rightarrow$ Stronger Sound Validation
\end{theorem}
\begin{proof}
This proof proceeds by induction in the usual manner.

\emph{Base Case}: At the start of the execution, the memory is going to only contain $c$: since $c$ is defined to be initially $\mathit{valid}$, and has only \Q@mut@ fields, and so it is trivially $\mathit{wellEncapsulated}$, thus $OK(c\mapsto\Kw{Cap},e)$.

\emph{Induction}: By Progress we always have another evaluation step to take, by Subject Reduction such a step will preserve $\mathit{OK}$, and so by induction $\mathit{OK}$ holds after any number of steps.

Note how for the proof garbage collection is important: 
when the \Q@validate()@ method evaluates to \Q@false@, 
execution can continue only if the offending object is classified as garbage.
\end{proof}

\subsection{Proof of Subject Reduction}
We first introduce a lemma derived from well formedness and the type system:
\begin{Lemma}[ExposerInstrumentation]
If $\sigma_0 | \e_0\rightarrow \sigma_1 |\e_1$ and
$\text{fieldGuarded}(\sigma_0,\e_0)$
\\*
then $\text{fieldGuarded}(\sigma_1,\e_1)$
\end{Lemma}
\begin{proof}
The only rule that can 
introduce a new field access is \textsc{mcall}.
In that case, ExposerInstrumentation holds
by well formedness (all field accesses in methods are of the form \Q@this.f@) 
and since \textsc{m call} inserts a monitor while invoking capsule mutator methods, and not field accesses themselves. If however the method is not a \Q@mut@ method but still accesses a capsule field, by MutField such a field access expression cannot be typed as \Q@mut@ and so no monitor is needed.

Note that \textsc{monitor exit} is fine because monitors are removed only when
 $e_1$ is a value.
\end{proof}

\begin{theorem}
	Subject Reduction Base $\Rightarrow$ Subject Reduction
\end{theorem}
\begin{proof}
Any reduction step can be obtained
by exactly one application of rule \textsc{ctx} and then one other rule.



Thus the proof can simply proceed by cases on such other applied rule.

By SubjectReductionBase and ExposerInstrumentation, 
$\Sigma^{\sigma_1};\emptyset\vdash e_1: T_1$ and  $\mathit{fieldGuarded}(\sigma_1,\e_1)$. So we just need to proceed by cases on the reduction rule applied to verify that $OK(\sigma_1,\e_1)$:


\begin{enumerate}
\item \textsc{update:} $\sigma|l\singleDot f\equals v\rightarrow \sigma'|\e'$:
\begin{itemize}
  \item by \textsc{update} $\e'=\Kw{M}\oR l;l;l\singleDot\text{validate}\oR\cR\cR;$, thus $\mathit{monitored}(\e,l)$.
  \item Every $l_1$ such that $l\in \text{rog}(\sigma,l_1)$ will verify the same case
  as the former step:
  \begin{itemize}
  	\item If it was $\mathit{garbage}$, clearly it still is.
  	\item If it was $\mathit{monitored}$, it also still is.
  	\item If can't have been $\mathit{wellEncapsulated}$ since $mutatable(l, \sigma, e)$, (by MutField)
  \end{itemize}
  \item Every other $l_0$ is not reached by $l$ thus it being $\mathit{OK}$ could not have been effected by this reduction step.
\end{itemize}

\noindent\textbf{case field access} $l.f\rightarrow v$:

    If for $l$ $IOK$ holds by (2),  
    it is possible that the next step is not encapsulated.
    This would mean that the field $f$ is a capsule and that we are required
to type it as \Q@mut@ to type the expression for the next step.
By $\mathit{fieldGuarded}(\sigma_0,\e_0)$
    the former step was inside of a monitor \Q@M(@$l$\Q@;@$\ctx_v[l$\Q@.f@$]$\Q@;@$\e$\Q@)@
    and the $l$ under reduction was the only occurrence of $l$.
    since $f$ is a capsule, we know that $l\notin \text{erog}(\sigma,l)$
    by HeadNonCircular.
    Thus in the new step not $l\, \text{inside}\ \ctx_v[v]$.
    Thus for l (3)[monitored] holds.
    
We still need to show that properties $\mathit{monitored}$ and $\mathit{wellEncapsulated}$
 for other objects are
not disturbed. This is the point where our aliasing and mutability control are most crucial:
We know that mutable $v$ is (directly) reachable from
$l$ that have invariant.
Thanks to CapsuleTree we know that for all $l_0$ reaching $l$,
$v$ can be reached by $l_0$ only passing trough $l$.
Thus, we can conclude  $l_0$ is not encapsulated in the former step (containing mutable $l$).
Thus, $l_0$ is either without invariant, garbage or monitored.
None of those 3 cases can be disturbed by a field access.


\noindent\textbf{case meth call}:\\*
  This reduction step does not influence any object in the memory and does not
disturb the properties $\mathit{monitored}$ and $\mathit{wellEncapsulated}$.

\noindent\textbf{case new}:\\*
  If $C$ has invariant, then by @ConstructionInstrumentation the new object is monitored.
As for the method call, other objects and properties are not disturbed.


\noindent\textbf{case monitor exit} \Q@M(@$l;v;$\Q@true)@$\rightarrow v$ :
  \begin{itemize}
\item
    If it was a setter $v=l$, and 
    thanks to Determinism the execution of invariant is deterministic;
    thus for $l$ in the former step both case (2) and (3) holds.
    In the next step (2) will hold for $l$.
\item
    If it was a capsule mutator method, thanks to Determinism the execution
 of \Q@.validate()@ is deterministic;
    thus for $l$ in the former step both $H$ and case (3) holds.
    Thanks to ExposerInstrumentation $v$ is offered without mutation permissions, so
    In the next step $l$ is encapsulated and (2) will hold.
\item
    If it is was a constructor, 
    then $v$ is encapsulated and thanks to Determinism
    the execution of invariant is deterministic, thus in the next step (2) will hold.
\end{itemize}

\noindent\textbf{case try enter and try ok}
This case do not influence any object in the memory and does not
disturb the properties $\mathit{monitored}$ and $\mathit{wellEncapsulated}$.

\noindent\textbf{case try catch} $\sigma,\sigma_0|\Kw{try}^\sigma \oC\mathit{error}\cC\Kw{catch}\, \e\rightarrow \sigma|\e$:\\*
From the premise we know 
$IOK(\sigma,\sigma_0;\ctx_v[\Kw{try}^\sigma \oC\mathit{error}\cC\Kw{catch}\, \e])$;
thus we need to show
$IOK(\sigma;\ctx_v[\e])$.
By StrongExceptionSafety we know that $\sigma_0$ is garbage with respect to $\ctx_v[\e]$.

There could be many $l$ inside $\sigma,\sigma_0$ that are $\mathit{monitored}$
in the former step thanks to monitor expressions inside $\mathit{error}$.
However, all such $l$ are defined inside $\sigma_0$,
for the last well formedness condition.
\end{enumerate}
\end{proof}
\end{document}
