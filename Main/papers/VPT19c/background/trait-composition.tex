\subsection{Trait Composition}\label{s:btc}
The idea behind \emph{trait composition} is to seperate the notions of \emph{subtyping} and \emph{inheritence}~\cite{useReuse}. Only \emph{traits} can be inherited, and only \emph{classes} can be used as types. A \emph{trait} is like an anonymous class, in particular it can contain methods, but because it is anonymous, one cannot directly use it as a type or create instances of it. Class are then declared from traits, effectively giving them a name, but such classes will be unrelated to other classes defined from the same trait. Traits can be `composed` to produce other traits, however since traits cannot be used as types, we are free to perform operations that would break the LSV. Consider the following example:
\begin{lstlisting}
Trait pair = class {
	String target();
	String hello() { return "Hello " + this.target(); }
};
// Defines a class called HelloWorld
HelloWorld: pair.extend(class {
	String target() { return "World"; }
	String hello() { this.super_print() + "!"; } // prints "Hello World!"
});
\end{lstlisting}
We use the type /Trait/ as the types of traits, and a class declaration (without a name) as a \emph{trait literal} expression. Here /trait1.extend(trait2)/ contains all the methods in /trait1/ and /trait2/. When a method declaration appears in both /trait1/ and /trait2/, their signatures will also be checked to ensure they are identical, this ensures that if both /trait1/ and /trait2/ are well-typed, the result will also be (this property is sometimes called \emph{meta level soudnes}~\cite{Servetto:2010:MMC:1869459.1869498}).

However if a method $m$ is also \emph{implemented} (i.e. non abstract) in both /trait1/ and /trait2/, the operation will proceed as if the version in /trait1/ was named /super_$m$/. Unlike conventional OO inheritance, this means that calls to $m$ in /trait1/ will call the implementation in /trait1/, and not the (overriding) implementation in /trait2/. In addition, the renamed /super_$m$/ method will be `private' to /trait2/, i.e. only code in /trait2/ itself can see it, not code added in later. Their are other approaches to dealing with this, such as only renaming the declaration of $m$ to /super_$m$/, so that calls in /trait1/ will call the method in /trait2/. However in order for this to be sound in our context of producing correct code, we would have to prevent the $m$ in /trait2/ from having an incompatible contract to the $m$ in /trait1/, thus breaking our examples in \sref{ex}.

In this paper, we will use the \emph{flattening} semantics of traits, in which trait operations produce a flattened result, equivalent to a trait literal, with no references to any of the input traits to the operation. Thus the above definition for /HelloWorld/ is identical to:
\begin{lstlisting}
HelloWorld: class {
	private String super_hello() { return "Hello " + this.target(); }
	String target() { return "World"; }
	String print() { this.super_print() + "!"; }
};
\end{lstlisting}

The main advantage of trait composition over conventional inheritance is that trait composition does \emph{not} induce subtyping:
\begin{itemize}
	\item Trait operations that would be unsound in a subtyping environment are allowed, such as removing methods or changing their types.
	\item The way code is reused in order to construct a class is not exposed to users of the class. This allows the code reuse mechanism to be changed without breaking any uses of a class.
\end{itemize}
In addition, this makes traits much simpler to reason about, making it easier to compose them in more complicated ways.