\section{Introduction}
\MS{With this short paper we contribute
to research on safe metaprogramming, showing how combining 
pre//post conditions,
trait composition and metaprogramming
it is possible to create metaprograms that generat code which is correct by construction.
That is: only the original source code itself needs to be verified (for example by a theorem prover), and not the code produced by metaprogramming.
This is important since the original code is often orders of magnitude smaller then the generated code.
We start by providing some background on those three research areas:}

\MS{\textbf{Pre-post conditions}:}
object oriented languages supporting static verification usually extend the syntax for method declarations
to support \emph{contracts} in the form of pre and post-conditions~\cite{Meyer:1988:OSC:534929}.
Correctness is defined only for code annotated with such contracts.

We say that a method is \emph{correct}, if whenever its precondition holds on entry, the precondition of every directly invoked method holds, and the postcondition of the method holds when the method returns. Automated static verification typically works by asking an automated theorem prover to verify that each method is correct individually, by assuming the correctness of every other method~\cite{barnett2004spec}. This process can be very slow and can produce unexpected results: since static verification is undecidable, correct code may not pass a particular static verifier.
Many static verification approaches are not resilient to
standard refactoring techniques like 
method inlining. Sometimes static verification even times out, making the behaviour even more sensitive to such refactoring techniques.



\MS{\textbf{Traits}: originally introduced in smalltalk by Scharli~\cite{scharli2003traits}, traits are 
units of code reuse. They where created as a simpler way of performing multiple inheritance without the usual complexity.
Traits are just a set of method declarations.
Such methods can be abstract and
mutually recursive by using the implicit /this/ parameter.
Traits are different from Java-style abstract classes as
%\begin{itemize}
%\item
they are only for reuse: trait names do \emph{not} define types.}
%\item Traits contains only methods; thus they have no field or constructors.
%\item
Trait composition is seen as a form of flattening: after the composition, the resulting code contains copies of the adapted methods from the original traits, but do not reference the original traits directly. Since there is no trace of
the point of origin of the code, /super/ calls are not 
directly available and need to be somehow emulated.
%\item
Traits can be combined using many different composition operators, not just \emph{extends}.
%\end{itemize}
In this work we will rely on the traditional composition operators /+/(plus), /rename/ and /hide/.

The /+/ operator is the main way to compose traits
~\cite{scharli2003traits,LagorioSZ09}.
The result of /+/ will contain all the methods from both operands. 
Crucially, it is possible to sum traits where a method is declared in both operands; in this case at least one of the two competing methods needs to be abstract, and the signatures of the two competing methods need to be \emph{compatible}.
In this way, /+/ have an expressive power similar to multiple inheritance.

Rename and hide adapt a single trait by renaming a method or by making a method private.
Many works in literature allow adapting traits by renaming or hiding methods\cite{servetto2014meta,reppy2007metaprogramming,liquori2008feathertrait}. Hiding a method may also trigger inlining if the method body is simple enough or used only once.


Consider the following example code, where we use those 3 operators:
\begin{lstlisting}
Trait a=class{Int hello(){return 1;}}
Trait b=class{
  abstract Int hello();
  String world(){return "["+this.hello()+"]";}}
Trait c=(a+b)[hide hello()][rename world()->hello()]
\end{lstlisting}
After flattening we would get the following result; where /c/ now contains a single method called
/world()/, with the body of the method originally called /hello()/ declared in trait /b/. This body contains the
inlined version of method /a.world()/ that has been hidden.
Note how the order of operations is important.
\begin{lstlisting}
Trait a=/*as before*/
Trait b=/*as before*/
Trait c=class {String world(){return "["+1+"]";}}
\end{lstlisting}

\MS{\textbf{Metaprogramming}} is often used to programmatically generate faster specialised code when some parameters are known in advance, this is particularly useful where the specialisation mechanism is too complicated for a generic compiler to automatically derive~\cite{Ofenbeck:2017:SGP:3136040.3136060}.
A \emph{metaprogram} is a program, method or function 
that produces code. Depending on the kind of metaprogramming, such code can be directly executable or can be just an abstract representation of behaviour.
Metaprogramming is called \emph{metacircular} when the language used to write the metaprogram is the same language of the generate code.
Metaprogramming can happen at run time, or at compile time. In the latter case, the produced code can be needed to typecheck and compile the rest of the code in the program.
In this paper we will rely on 
Iterative Composition: 
a metacircular metaprogramming technique relying on \emph{compile-time execution} (a form of execution also used by~\cite{sheard2002template}).
This disciplined form of metaprogramming introduced by Servetto and Zucca \cite{servetto2014meta}, is based on the trait composition operators described before, but lifted at the expression level.
\MS{This means that arbitrary expressions can be used as} the right hand side of \MS{trait and class declarations}; during compilation such expressions will be evaluated to produce a /Trait/, which provides the body of the class. In this way metaprograms can be represented as otherwise normal functions//methods that return a /Trait/, without requiring the use of any additional `metalanguage'.

\section{Combining metaprogramming and static verification}

\MS{A na\"ive way to} combine metaprogramming and static verification could be to use metaprogramming to generate code together with contracts, and then once the metaprogramming has been run,
% \MSDel{ensure the correctness of} 
statically verify the resulting code. 
However, the resulting code could be much larger than the input to the metaprogramming, and so it could take a long time to statically verify.
Moreover, one of the many goals of metaprogramming is to make it easier to generate many specialised versions of the same  code.
%, Even if the generated code was produced by using straightforward transformations and compositions over the input code, a SV might not verify it's correctness.
The aim of our work is to statically verify only the original source code itself, and not the code produced by metaprogramming.
Instead, we
ensure that the result of metaprogramming is correct by construction.

%Here we use the disciplined form of metaprogramming introduced by Servetto and Zucca \cite{servetto2014meta}, which is based on trait %composition and adaptation~\cite{scharli2003traits}.
%Here a /Trait/ is a unit of code: a set of method declarations.
%Such methods can be abstract and can be
%mutually recursive by using the implicit parameter /this/.

We extend~\cite{servetto2014meta} by allowing
methods to be annotated with pre//post-conditions.
In addition to requiring that all the traits are well-typed before they are used (as in~\cite{servetto2014meta}) we also require that traits are correct in terms of their method contracts.
/Trait/s directly written in the source code are statically verified, while traits resulting from metaprogramming are ensured correct by only providing trait operations that preserve correctness. 
In particular, we \textbf{only} need to extend the checking performed by the traditional trait composition (/+/)  operator to also check the compatibility of contracts.

%
%The result of composing and adapting /Trait/s is also correct and well-typed.
%
Our metaprogramming approach does not allow generating code from scratch, such as by directly generating ASTs, rather the language provides a specific set of primitive composition and adaptation operators which preserve correctness.
Thus the result of metaprogramming is guaranteed to be well typed and correct.
%Note that generated code may not be able to pass a particular static verifier.


Static verification usually handles /extends/ and /implements/ by verifying that every 
time a method is implemented//overriden, 
the Liskov substitution principle~\cite{Liskov:1994:BNS:197320.197383} is satisfied
by checking that the contracts of the method in the derived class implies the contract of any corresponding methods in its base classes. 
 In this way, there is no need to re-verify
inherited code in the context of the derived class.
This concept is easily adapted
to handle trait composition, which simply provides another way to implement an /abstract/ method.
When traits are composed,
it is sufficient
to match the contracts of the few composed methods
to ensure the whole result is correct.

\section{Concrete example}

In our example we will use the notation /@requires($predicate$)/ 
to specify a precondition, and\\* /@ensures($predicate$)/ 
to specify a postcondition; where $predicate$ is a boolean expression
in terms of the parameters of the method (including /this/), and for the /@ensures/ case, the /result/ of the method.
Suppose we want to implement an efficient exponentiation function, we could use recursion and the common technique of `repeated squaring':
\vspace{-1ex}
\begin{lstlisting}
@requires(exp > 0)
@ensures(result == x**exp)//Here x**y means x to the power of y
Int pow(Int x, Int exp) {
	if (exp == 1) return x;
	if (exp %2 == 0) return pow(x*x, exp/2); // exp is even
	return x*pow(x, exp-1); }  // exp is odd
\end{lstlisting}
If the exponent is known at compile time,
unfolding the recursion produces even more efficient code:
\vspace{-1ex}
\begin{lstlisting}
@ensures(result == x**7) Int pow7(Int x) { 
  Int x2 = x*x; // x**2
  Int x4 = x2*x2; // x**4
  return x*x2*x4; } // Since 7 = 1 + 2 + 4
\end{lstlisting}
\vspace{-1ex}

\MS{We now show how \emph{Iterative Composition} %(introduced in~\cite{servetto2014meta} and
(enriched by the contract compatibility check we proposed) % performing in trait composition) 
can be used to write a metaprogram that given an exponent, produces code like the above.}
%Iterative Composition is a metacircular metaprogramming technique relying on \emph{compile-time execution} (a form of execution also used by~\cite{sheard2002template}),
%in our context this means that arbitrary expressions can be used as the right hand side of a class declaration; during compilation such expressions will be evaluated to produce a /Trait/, which provides the body of the class. In this way metaprograms can be represented as otherwise normal functions//methods that return a /Trait/, without requiring the use of any additional `metalanguage'.
 
%\vspace{-1ex}
\begin{lstlisting}
Trait base=class {//induction base case: pow(x)==x**1
  @ensures(result>0) Int exp(){return 1;}  
  @ensures(result==x**exp()) Int pow(Int x){return x;}
}
Trait even=class {//if _pow(x)==x**_exp(), pow(x)==x**(2*_exp())
  @ensures(result>0) Int $\_$exp();
  @ensures(result==2*$\_$exp()) Int exp(){return 2*$\_$exp();}
  @ensures(result==x**$\_$exp()) Int $\_$pow(Int x);
  @ensures(result==x**exp()) Int pow(Int x){return $\_$pow(x*x);}
}
Trait odd=class {//if _pow(x)==x**_exp(), pow(x)==x**(1+_exp())
  @ensures(result>0) Int $\_$exp();
  @ensures(result==1+$\_$exp()) Int exp(){return 1+$\_$exp();}
  @ensures(result==x**$\_$exp()) Int $\_$pow(Int x);
  @ensures(result==x**exp()) Int pow(Int x){return x*$\_$pow(x);}
}
\end{lstlisting}
First we will define tree traits: /base/, /even/ and /odd/.
They are the basic building blocks we will use to compute our result. They will be compiled, typechecked and statically verified before being use in any way.
Note that we could use /base/ directly:
we could write /class Pow1: base/; this would generate a class such that /new Pow1().pow(x)==x**1/.
The other two traits have abstract methods; implementations for /$\_$pow(x)/ and /$\_$exp()/ must be provided. However, given the contract of /pow(x)/,
and the fact that /even/ and /odd/ have both been statically verified,
if we supply method bodies respecting these contracts, we will get \emph{correct} code, without the need for further static verification.
Since all occurrences of names are consistently renamed, \textbf{renaming and hiding preserve code correctness}.
\MS{Method names starting with $\_$, like /$\_$pow(x)/ and /$\_$exp()/, are not special and are not treated in any special way by trait composition. We use the $\_$ naming convention to emulate /super/ when using trait composition: a call to /$\_$exp()/ is used like a /super.exp()/ call in a language with conventional class based inheritence like Java.}

\begin{lstlisting}
//`compose' performs a step of iterative composition
Trait compose(Trait current, Trait next){
  current = current[rename exp()->$\_$exp(), pow(x)->$\_$pow(x)];
  return (current+next)[hide $\_$exp(), $\_$pow(x)];}
@requires(exp>0)//the entry point for our metaprogramming
Trait generate(Int exp) {
  if (exp==1) return base;
  if (exp%2==0) return compose(generate(exp/2),even);
  return compose(generate(exp-1),odd);
};
\end{lstlisting}
%\vspace{-1ex}
