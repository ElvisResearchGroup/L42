
The /compose(current,next)/ method starts by renaming the /exp()/ and /pow(x)/ methods of\\* /current/
so that they satisfy the contracts in /next/ (which will be 
/even/ or /odd/).
%The /+/ operator is the main way to compose traits%
%~\cite{scharli2003traits,LagorioSZ09}.
%The result of /+/ will contain all the methods from both operands. 

%Crucially, it is possible to sum traits where a method is declared in both operands; in this case at least one of the two competing methods needs to be abstract, and the signatures of the two competing methods need to be \emph{compatible}.
Then, the operator /+/ is used to compose the code of the parameters.
Here we show how we ensure that the traditional /+/ operator also handles contracts: we require that the contract annotations of the two competing methods are \emph{compatible}.
\MS{In this paper, we just require them to be syntactically identical.} Relaxing this constraint is an important future work.
Thanks to this constraint \textbf{the sum operator also preserves code correctness}. %\IO{There are many variations of the /+/ operator, in particular, we could easily extend our contract matching to work with an nary operator}.

The sum is executed when the method /compose/
%\IO{\footnote{\IO{a generic implementation of this method that renames and hides conflicting methods has been implemented L42~\cite{l42}}}}
runs: if the matched contracts are not identical an exception will be raised. A leaked exception during compile-time metaprogramming would become a compile-time error. 
Our approach is very similar to~\cite{servetto2014meta}, and does not guarantee the success of the code generation process, rather it guarantees that if it succeeds, correct code is generated.

\MS{Executing /compose(base,even)/ or /compose(base,odd)/ will pass this test: since the contract of /base.pow()/
is the same of /even.$\_$pow()/ and /even.$\_$pow()/, and the same for /exp()/.}

Finally the /$\_$pow(x)/ and /$\_$exp()/ method are hidden, so that the structural shape of the result is
the same as /base/'s.
\MS{Note that this structural equality includes the contracts of methods}.

As you can see, /Trait/s are first class values and can be manipulated with a set of primitive operators that preserve code correctness and well-typedness.
In this way, by inductive reasoning, we can start from the /base/ case and then recursively compose /even/ and /odd/ until we get the desired code.
Note how the code of /generate(exp)/ follows the same scheme of the code of /pow(x,exp)/ in line 1.

To understand our example better, imagine executing the code of /generate(7)/ while keeping /compose/ in symbolic form. We would get the following (where /c/ is short for /compose/):
\vspace{-1ex}
\begin{lstlisting}[numbers=none]
generate(7) == c(generate(6),odd) == ...
 == c(c(c(c(base,even),odd),even),odd)
\end{lstlisting}
\vspace{-1ex}
As /base/ represents /pow1(x)/; /c(base,even)/ represents /pow2(x)/. Then \Q@c(/*pow2(x)*/,odd)@ represents \Q@pow3(x)@, \Q@c(/*pow3(x)*/,even)@ represents \Q@pow6(x)@, and finally,
\Q@c(/*pow6(x)*/,odd)@ represents \Q@pow7(x)@.
The code of each /$\_$pow(x)/ method is only executed once for each top-level /pow(x)/ call, so the /hide/ operator can inline them.
Thus, the result could be identical to the manually optimized code in line 7.
We can use our /generate(7)/ as following:
\begin{lstlisting}
class Pow7: generate(7) //generate(7) is executed at compile time
//the body of class Pow7 is the result of generate(7)
/*example usage:*/
new Pow7().pow(3)==2187//Compute 3**7
\end{lstlisting}

%\IO{We are investigation how an additional check can be performed to ensure the resulting code has specific contracts. However, our approach does guarantee that the result will be correct according to whatever contracts it contains.} 
\section{Future work}
Our approach, as presented in this short paper, only guarantees that code resulting from metaprogramming follows its own contracts, it does
not statically ensure what those contracts may be. As future work, we are investigating how the resulting contracts can be ensured to have a particular meaning or form.
To do so, we need to allow assertions on the contracts of /Trait/s to be used within pre//post conditions.
For example we could allow post conditions like\\*
%\begin{lstlisting}[numbers=none]
/@ensures(result.$\mathit{methName}$.ensures ==\ $\mathit{predicate}$)/ \\*
%\end{lstlisting}
to mean that the resulting /Trait/ has
a method
called $\mathit{methName}$, whose /@ensures/ clause is syntactically identical to  /$predicate$/; whilst
\\*
/@ensures(result.$\mathit{methName}$.ensures ==>\ $\mathit{predicate}$)/
\\*
would use a static verifier to ensure that $\mathit{methName}$'s /@ensures/ clause logically implies $\mathit{predicate}$.
With these two features we could annotate the method /generate(exp)/ in line 32 above as:
\begin{lstlisting}
@requires(exp>0)
@ensures(result.exp().ensures ==> (result==exp))
@ensures(result.pow(x).ensures == (result==x**exp()))
Trait generate(Int exp) {...}
\end{lstlisting}

\vspace{-1ex}
In this way, we could statically verify the /generate(exp)/ method, however we fear such verification will be too complex or impractical. 
We could instead automatically check the above postconditions after each call to /generate(exp)/. If /generate(exp)/ is used to define a class (such as /Pow7/ above), we will guarantee that such class has the expected contracts, before it is used. Thus
there is no need to ensure the correctness of the metaprogram itself: such runtime checks are sufficient to ensure that after compilation, the code produced by metaprogramming has its expected behaviour.
%\IODel{In this case we could defer those difficult//novel predicates to run-time checks, without losing much safety:
%Iterative Composition execute metaprogramming code at
%compile time, thus even run-time verification of metaprograms would happen at compile time. This consideration could result in a crucial design decision: code performing metaprogramming does not need to be verified by SV to produce code annotated with the desired contracts; it may be sufficient to apply some type of runtime verification during compile-time execution.} \IOComm{I did a major rewording since we actually have multiple compile-times and run-times, so your version is confusing, hopefully my version makes the point more clear.}
%For example, the following code:
%\vspace{-1ex}
%\begin{lstlisting}[numbers=none]
%@ensures(new Pow7().exp()==7&&Pow7.pow.ensures=="result==x**exp()")
%class Pow7: generate(7)
%\end{lstlisting}
%\vspace{-1ex}
%may require the static verifier to check that the execution of
%/new Pow7().exp()/ will deterministically reduce to /7/, and that the /ensures/ clause of 
%/Pow7.pow/ is syntactically equivalent to 
%/result==x**exp()/. Note how this final step of static verification does not need to re-verify the body of
%/Pow7.pow/ and only needs to do a coarse grained 
%determinism check on the implementation of /Pow7.exp()/, before symbolically executing it.

\section{Conclusion}
By levering over conventional OO static verification techniques, we have extended the Iterative Composition form of metaprogramming with a simple contract compatibility check, to statically ensure the correctness of code produced by such metaprogramming. In particular, our approach does not require static verification of the result of metaprogramming, but only requires verification of code present directly in source code.
\MS{Following general terminology in software verficiation, we say that a trait is \emph{correct} when its methods respect their contracts.}
\MS{Thus our result is that starting from a set
of well typed and correct traits, 
any code resulting from arbitrary many steps of trait composition will also be correct and well typed.
In this way, the programmer need only provide correct bulding blocks using traits;
code generated by metaprogramming can be integrated with a correct program without needing to use expensive theorem provers or manual verification.
Our example is applied to code specialization of a mathematical function, but our experience suggests that Iterative composition can be used to synthesize arbitrary behaviour.
}
