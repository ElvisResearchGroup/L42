The traits /base/, /even/, and /odd/ are the basic building blocks we will use to compute our result. They will be compiled, typechecked and SVed before the method /generate(exp)/ can run.
As you can see in line /37/, a class body can be an expression in the language itself.
At compile time such an expression will be run and the resulting /Trait/ will be used as the body of the class.
For example, we could write /class Pow1: base/; this would generate a class such that /new Pow1().pow(x)==x**1/.
The other two traits have abstract methods; implementations for /$\_$pow(x)/ and /$\_$exp()/ must be provided. However, given the contract of /pow(x)/,
and the fact that /even/ and /odd/ have both been SVed,
if we supply method bodies respecting these contracts, we will get \emph{correct} code, without the need for further SV.
Many works in literature allow adapting traits by renaming or hiding methods\cite{servetto2014meta,reppy2007metaprogramming,liquori2008feathertrait}. Hiding a method may also trigger inlining if the method body is simple enough or used only once.
Since all occurrences of names are consistently renamed, \textbf{renaming and hiding preserve code correctness}.

The /compose/ method starts by renaming the /exp/ and /pow/ methods of /current/
so that they satisfy the contracts in /next/ (which will be 
/even/ or /odd/).
\REV{The /+/ operator is the main way to compose traits%
~\cite{scharli2003traits,LagorioSZ09}.
The result of /+/ will contain all the methods from both operands.}{3}{Please take into account that the + operator is associative. Consequently, program specialization and program verification methods aiming at effective analysis and transformation of the corresponding expressions with the operator have to be used and developed.}

Crucially, it is possible to sum traits where a method is declared in both operands; in this case at least one of the two competing methods needs to be abstract, and the signatures of the two competing methods need to be \emph{compatible}.
To make sure that the traditional /+/ operator also handles contracts, we need to require that the contract annotations of the two competing methods  are \emph{compatible}.
For the sake of our example, we can just require them to be syntactically identical. Relaxing this constraint is an important future work.
Thanks to this constraint \textbf{the sum operator also preserves code correctness}.

The sum is executed when the method /compose/ runs, if the matched contracts are not identical an exception will be raised. A leaked exception during compile-time metaprogramming would become a compile-time error. 
Our approach is very similar to~\cite{servetto2014meta}, and does not guarantee the success of the code generation process, rather it guarantees that if it succeeds, correct code is generated.

Finally the /$\_$pow(x)/ and /$\_$exp()/ method are hidden, so that the structural shape of the result is
the same as /base/'s.
As you can see, /Trait/s are first class values and can be manipulated with a set of primitive operators that preserve code correctness and well-typedness.
In this way, by inductive reasoning, we can start from the /base/ case and then recursively compose /even/ and /odd/ until we get the desired code.
Note how the code of /generate(exp)/ follows the same scheme of the code of /pow(x,exp)/ in line 1.

To understand our example better, imagine executing the code of /generate(7)/ while keeping /compose/ in symbolic form. We would get the following (where /c/ is short for /compose/):
\vspace{-1ex}
\begin{lstlisting}[numbers=none]
generate(7) == c(generate(6),odd) == ...
 == c(c(c(c(base,even),odd),even),odd)
\end{lstlisting}
\vspace{-1ex}
As /base/ represents /pow1(x)/; /c(base,even)/ represents /pow2(x)/. Then \Q@c(/*pow2(x)*/,odd)@ represents \Q@pow3(x)@, \Q@c(/*pow3(x)*/,even)@ represents \Q@pow6(x)@, and finally,
\Q@c(/*pow6(x)*/,odd)@ represents \Q@pow7(x)@.
The code of each /$\_$pow/ method is only executed once for each top-level /pow/ call, so the /hide/ operator can inline them.
Thus, the result could be identical to the manually optimized code in line 7.

\REV{Note that while our approach guarantees that the resulting code follows its own contracts, it does not statically ensure what contracts it would have.}{2}{This paragraph casts doubt on the significance of what has been presented.}
We are investigating how
to perform an additional verification check
on the result of metaprogramming.
For example, the following code:
\vspace{-1ex}
\begin{lstlisting}[numbers=none]
@ensures(new Pow7().exp()==7&&Pow7.pow.ensures=="result==x**exp()")
class Pow7: generate(7)
\end{lstlisting}
\vspace{-1ex}
may require the static verifier to check that the execution of
/new Pow7().exp()/ will deterministically reduce to /7/, and that the /ensures/ clause of 
/Pow7.pow/ is syntactically equivalent to 
/result==x**exp()/. Note how this final step of static verification does not need to re-verify the body of
/Pow7.pow/ and only needs to do a coarse grained 
determinism check on the implementation of /Pow7.exp()/, before symbolically executing it.

In conclusion, static verification of metaprogramming is an exciting new area of research; we are attacking the problem by reusing conventional
object oriented static verification techniques coupled with trait composition, extended to also check contract compatibility. A crucial design decision is that code performing metaprogramming does not need to be SVed to produce code annotated with the desired contracts; it would be sufficient to apply some type of runtime verification during compile-time execution.