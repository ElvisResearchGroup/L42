\section{Extensions to our model}
  One of the main feature of our simple reuse/use model is that it can be
  easy extended. 
First we show two more composition operators, then we show 
a simple extension with nested classes.
Finally, we show how we can solve even the most
demanding version of the expression problem.

\paragraph{Rename}${}_{}$\\*
Rename allows to make some form of "compile time" re-factoring
There are a lots of different forms of rename in literature,
sometime allowing only to rename specific methods, sometime allowing to rename
nested classes into other nested classes either at the same or at a different nesting level.
%Renaming in the context of nested classes also means that when renaming a method of an interface, all the 
%nested classes implementing such interface inside of that code literal need to be adjusted.
Renaming need to rename not only the method headers, but all the method calls inside of method bodies.
At a first glimpse, this seems to be not always possible since we are considering to be able to apply those
operators also to non well typed code.
However, if the expression language is simple enough, it is possible to pre-process the code to
annotate the expected receiver type on all method calls by doing a purely syntactic analysis
on a single code literal in isolation. 
All the expression whose type is guessed to be out of the border of the literal can stay unannotated; they are not going to be renamed anyway.

\begin{lstlisting}
t:{ I:{interface method int mI() }
     A:{implements I  method int mI() 42}
     B:{ method int mB(I i, A a, C c) i.mI()+a.mI()+c.mI()}
     //mB would be annotated i[I].mI()+a[A].mI()+c.mI()
     }
D:t[rename A.mI kI]
\end{lstlisting}
 Notice how we are sure that \Q@C@ does not implements \Q@I@ since it is invisible from the outside: traits does not introduce nominal types!
 
 We expect the flattened version for \Q@D@ to be
\begin{lstlisting}
D:{ I:{interface method int kI() }
     A:{implements I  method int kI() 42}
     B:{ method int mB(I i, A a, C c) i.kI()+a.kI()+c.mI()}
     }
\end{lstlisting}

Hide can be seen as a variation of rename, where the method/class is renamed to a fresh unguessable name.

\paragraph{Redirect}${}_{}$\\*
Redirect allows to emulate generics; the main idea is that a (fully abstract) class can be redirect to another one external to the trait/code literal.
For example a linked list can be implement as
\begin{lstlisting}
list:{ Elem:{}
     Cell:{class method Cell of(Elem e,Cell c) 
       method Elem e()  method Cell c()
       }
   method Elem get(int x) ...
   ...more methods..
   }
ListString:list[redirect Elem to String]
\end{lstlisting}

%An expressive form of Redirect can be multiple, that is, can redirect may interdependent classes at the same time.
%We show a graph example, where also we can show how to propagate generics:
%For example
%\begin{lstlisting}
%t:{ method boolean reachable(Node start, Node end)/*implements reachability*/
%     Node:{method ListEdge nodes()}
%     Edge:{method Node in()  method Node out()}
%     ListEdge:list[redirect Elem to Edge]
%     }
%\end{lstlisting}




\paragraph{Nested classes}${}_{}$\\*
A nested class will be another kind of member in the code literal.
The general idea is that by composing code with \use, nested classes with the same name are recursively composed.
Note that while we have nested classes, we do not have nested traits: all traits are still
at top level.
Untypable/unresolved traits are also the only "dependency"
the type system keeps track of, this means that when a nested class at an arbirary
nested level is flattend, as in
\Q@C:{ D:{ E:Use t1,t2,L}}@\\*
\Q@t1@ and \Q@t2@ must be defined before \Q@C@ at top level; and they may require classes (and their
nested) defined before C. This means that the type system can still consider
the class table as a simple map from Types T to their definition.

This extension lets us challenge the expression problem~\cite{wadler1998expression},
with the requirements exposed in~\cite{scala}.
In the expression problem we have data variants and operations and we can
\textbf{extend our solution in both dimensions},
by adding new datavariants and adding new operations.
We also aim to \textbf{combine independently developed extensions} so
that they can be used jointly.

To be really modular, we want our extensions to
preserve \textbf{type safety}
and allow \textbf{separate compilation} (no re-type-checking),
while avoiding \textbf{duplication of source code}.

We can do this by creating a table of features, as in~\cite{Deepfjig}.
But in \name we can be even more compact as we can avoid a large amount of abstract declarations,
that clutters the solution in~\cite{Deepfjig}.


\begin{lstlisting}
//Definition of the datavariants state
exp:{//Exp declared once, reused everywhere
  Exp:{interface}
  T:{implements Exp}
  }
lit:Use exp[rename T in Lit],{//T renamed in Lit is 
  Lit:{ method Int inner() static method Lit of(Int inner)
    }//summed with the current Lit
  }
sum:Use exp[rename T in Sum],{//same for the other
  Sum:{ method Exp left() method Exp right()
    static method Sum(Exp left, Exp right) }
  }
uminus:Use exp[rename T in UMinus],{
  UMinus:{ ... }
  }

\end{lstlisting}
Here we define a trait for each datavariant.
Each trait will contain its version of \Q@Exp@
and a specific case of expression, with its state.
 To avoid repeating the declaration of \Q@Exp@ 
and \Q@implements Exp@ multiple times, we reuse
a \Q@exp@ trait.

\begin{lstlisting}
//Definition of toString for all the datavariants
expToS:Use exp, {//concept of toString declared once
  Exp:{interface method String toString()}
  }
litToS: Use lit, expToS[rename T in Lit], {
  Lit:{//just the implementation of the specific method
    method String toString(){return ""+this.inner();}
    } }
sumToS:Use sum, expToS[rename T in Sum], {
  Sum:{//just the implementation of the specific method
    method String toString(){
      return this.left().toString()+"+"+this.right().toString();
      }  }  }
uminusToS:...//implement toString for all the datavariants
\end{lstlisting}

Here we define a trait for each datavariant implementing the operation \Q@toString()@.
Again, each trait will contain its version of \Q@Exp@ with \Q@toString()@
and a specific case of expression, with the implementation for \Q@toString()@
for that specific case. Note how the state of such datavariant is 
\textbf{not repeated}, but imported by (for example) \Q@Use sum, ...@.

Again, \Q@expToS@ is a useful modularization point.
In the code below, the example of the other iconic operation: \Q@eval@.

\begin{lstlisting}
//Definition of other operations, for example eval
expEval:Use exp, {
  T:{interface method int eval()}
  }
//declare the next operation and implement it for all the datavariant
//example
sumEval:Use sum,expEval[rename T in Sum], {
  Sum:{//just the implementation of the
    method int eval(){//specific method
      return this.left().eval()+this.right().eval();
      }  }  }
\end{lstlisting}
In \name it is trivial also for the binary method equals.
Since every datavariant would have to copy the "cast" logic,
 we can modularize that into a \Q@expEquals@.

\begin{lstlisting}
expEquals:{
  Exp:{interface method Bool equals(Exp that)}
  T:{implements Exp
    method Bool exactEqual(T that)
    method Bool equals(Exp that){
      if(!(that instanceof T)){ return false;}
      return exactEqual( (T) that );
      }  }  }
sumEquals:Use sum,expEquals[rename T in Sum],{
  Sum:{
    method Bool exactEquals(Sum that){
      return this.left().equals(that.left()) 
        && this.right().equals(that.right());
      }  }  }
\end{lstlisting}
Note how the cast is safe since it is guarded by
an \Q@instanceof@ check. A sightly longer solution could avoid the
cast with double dispatch as done in [](show in appendix?).
%expEquals:{
%  Exp:{interface
%    method Bool equals(Exp that)
%    method Bool equalToT(T that)
%    }
%  T:{implements Exp
%    method Bool equals(Exp that){ that.equalToT(this); }
%    }  }
%sumEquals:Use sum,
%expEquals[rename T in Sum][rename Exp.equalToT in equalToSum],{
%  Sum:{
%    method Bool equalToSum(Sum that){
%      return this.left().equals(that.left()) 
%        && this.right().equals(that.right())
%      }  }  }

Now that you have nicely modularized the code, just compose all the traits you need.
\begin{lstlisting}
MySolution:Use sumToS,litToS
//sum,lit and exp traits are already included
\end{lstlisting}

The expression problem presented up to now is the traditional challenge proposed by~\cite{wadler1998expression};
this has been criticized to not really address the fundamental issues since it does not handle ....
Now we show how we can go beyond the traditional expression problem by encoding transformer methods:
For example, lets add 1 to all literals
\begin{lstlisting}
expAdd1:{Exp:{interface method Exp add1()}}
sumAdd1:Use sum,expAdd1,{Sum:{implements Exp
    method Exp add1()
      Sum.of(left.add1(), right.add1())
  }
litAdd1:Use lit,expAdd1,{Lit:{implements Exp
    method Exp add1()
      Lit.of(inner()+1);
    }

MySolutionAdd1:Use sumToS,litToS,sumAdd1,litAdd1
\end{lstlisting}

This nicely solves our problem. 
However, notice how if we wished to add many similar operations we would 
have to repeat the propagation code (as in \Q@sumAdd1@) many times
just changing the name of the operation.
Solutions to the expression problems involving the  Visitor Pattern 
allow defining a \Q@CloneVisitor@ in order
to reuse its propagation code.
You can see what we mean with the sketch of code below:
\begin{lstlisting}
class CloneVisitor{
  Exp visit(Lit l){return new Lit(l.inner);}
  Exp visit(Sum s){return new Sum(s.left.accept(this),s.right.accept(this);}
  }
class Add1 extends CloneVisitor{
  Exp visit(Lit l){return new Lit(l.inner+1);}
  }
\end{lstlisting}
In \name we can obtain the same kind of code reuse, without the need of introducing 
the concepts related to the Visitor Pattern.
With redirect, rename and restrict we can have the general operator propagator:
\begin{lstlisting}
operation:{//for sum and lit, easy to extends as before
  T:{}
  Exp:{interface method Exp op(T x)}}
  Sum:Use sum,{ extends Exp sum,expAdd1,{
    method Exp op(T x)
      Sum.of(left.op(x),right.op(x))  }
  Lit:Use lit,{
    method Exp op(T x)  this
  }
\end{lstlisting}

Now, to have \Q@addN@ we can do the following.

\begin{lstlisting}
opAddn: Use
  operation[redirect T to Int]
    [rename Exp.op(x) to addN(x)][restrict Lit.op(x)], {
  Lit:{method Exp addN(Int x) Lit.of(inner())+x}
  }
\end{lstlisting}  



%\paragraph{Full power of redirect}${}_{}$\\*
%An expressive form of Redirect can be multiple, that is, can redirect may interdependent classes at the same time.
%We show an example where a specific kind of \Q@Service@ can produce a \Q@Report@, and 
%\Q@Report@s can be combined together.
%The goal is to execute a list of such services and produce a collated report.
%This example also show how to propagate generics:
%
%\begin{lstlisting}
%Service:{interface method Void performService()}
%serviceCombinator:{
%  S:{implements Service method R report()  }
%  
%  R:{method R combine(R that)   class method R empty() }
%  
%  ListS:list[redirect Elem to S]
%  
%  class method R doAll(ListS ss){//here we use extended java like syntax
%    R r=R.empty()
%    for(S s in ss){
%      s.performService();
%      r=r.combine(s.report())
%      }
%    return r;
%  }
%}
%PaintingService:serviceCombinator[redirect S to PaintingService]
%PaintingService:{... method PaintingReport report()..}
%PaintingReport:{..}
%\end{lstlisting}
%
%The flattened version of \Q@PaintingService@ would look like:
%\begin{lstlisting}
%PaintingService:{
%  ListS:/*the expansion of list[redirect Elem to PaintingService]*/
%  
%  class method PaintingReport doAll(ListS ss){
%    PaintingReport r=PaintingReport.empty()
%    for(PaintingService s in ss){
%      s.performService();
%      r=r.combine(s.report())
%      }
%    return r;
%  }
%}
%\end{lstlisting}
%Where you can note how redirect figured out \Q@R=PaintingReport@ by comparing the structural shape of
%classes \Q@PaintingService@ and \Q@S@.
%
%To encode the former generic code in java you need to write
%the following headeche inducing interfaces for RService and Report.
%and require that the services you want to serve implement those.
%\begin{lstlisting}
%interface Service{ void performService();}
%interface Report<R extends Report<R>>{R combine(R that);}
%interface RService<R extends Report<R>> extends Service{ R report();}
%\end{lstlisting}
%Note how we still can not encode the method \Q@empty@.
