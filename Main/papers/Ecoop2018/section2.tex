\saveSpace
\section{The Design of \name: Separating Use and Reuse}\label{sec:separate}
\saveSpace\saveSpace
This section presents an overview of \name and illustrates the
key ideas of its design. In particular we illustrate how to separate code use and 
code reuse, and how \name solves the this-leaking problem.
% in \name. %, and how this allows improving both. 

%\subsection{The Design of \name: Decoupling Use from Reuse}
\saveSpace\saveSpace
\subsection{Classes in \name: A mechanism for code Use}
\saveSpace
%\name has a substantially different design from Java-like languages.
The concept of a class in \name provides a mechanism for code-use
only. This means that there is actually no subclassing:
classes are roughly equivalent to final classes in Java.  Thus,
compared to Java-like languages, the most noticeable difference is the
absence of the \Q@extends@ keyword in \name. 
Consider the example from Section~\ref{sec:intro}, rewritten in \name:
\saveSpace
\begin{lstlisting}
 Utils={ static method int m(A a){return ...} }
 A= { method int ma(){return Utils.m(this);} }
\end{lstlisting} 
\saveSpace
\noindent Classes in \name use a slightly different declaration style compared
to Java: there is no \lstinline{class} keyword, and an equal sign separates the class name (which must always start with
an uppercase letter) and the class implementation, which is used to specify the
definitions of the class. In our example, in the class declaration
for \lstinline{A}, the name of the class is \lstinline{A} and the code 
literal associated with the class is `\Q@{ method int ma(){return Utils.m(this);}}@' and it contains the member  \Q@ma()@.

%We will see next some 
%important differences to Java-like languages in the way 
%classes and code-literals are type-checked, as we shall see next. 
%Nevertheless, for this example, things still work in a similar way to Java. 

In the \name code above there is no way to add a class 
\Q@B@ reusing the code of \Q@A@: \Q@A@ is designed for code \emph{use} and not \emph{reuse}.
Thus, unlike the Java code, introducing a subclass
\lstinline{B} is not possible. This may seem like a severe restriction, but
\name has a different mechanism for \emph{code-reuse} that 
is more appropriate when \emph{code-reuse} is intended. 

\subsection{Traits in \name: A mechanism of code Reuse}

Unlike classes, traits in \name cannot be instantiated and do not introduce new
types. However they provide code reuse.
%So, lets try again encoding the code for the leaking problem, but this
%time aiming at code reuse. 
Trait declarations 
look very much like class declarations, but trait names 
start with a lowercase letter.

\saveSpace
\begin{lstlisting}
 Utils={ static method int m(A a){return ...} }
 ta={ method int ma(){return Utils.m(this);}}//type error
 A=Use ta
\end{lstlisting}
\saveSpace
\noindent Here \lstinline{ta} is a trait intended to replace the
original class \lstinline{A} so that the code of the method
\lstinline{ma} can be reused. Then the class \lstinline{A} 
is created by inheriting the code from the trait using the keyword 
{\bf \lstinline{Use}}. Note that \use\ cannot contain class names: only trait
names are allowed.
That is, \textbf{using a trait is the only way to induce code reuse}.

The crucial point is the call \Q@Utils.m(this)@ inside trait \Q@ta@:
the corresponding call in the Java code is coercing that code (even when reused) to be an \Q@A@.
In \name we do not commit the same mistake, and the former code is ill typed:
the type of \lstinline{this} in
\Q@ta@ has no relationship to the type \lstinline{A}.


Note that also that the following second attempt would not work:
\saveSpace
\begin{lstlisting}
 Utils={ static method int m(ta a){return ...}//syntax error
 ta={ method int ma(){return Utils.m(this);}}
 A=Use ta
\end{lstlisting}
\saveSpace
\Q@ta@ is not a type in the first place, since it is a (lowercase) trait name.
Indeed since the trait name is not a type, no code external to that trait can
refer to it, and that code can stay agnostic of its name. This is one of the key design decisions in \name:
traits can be reused in multiple places, and their code can be seen under multiple types.
\textbf{Interfaces are the only way to obtain subtyping}; thus
subtyping is the way to reason about commonalities between different types.

Thus, to model our example, we need an interface
capturing the commonalities between \Q@A@ and \Q@B@:
\saveSpace
\begin{lstlisting}
 IA={interface method int ma()}//interface with abstract method
 Utils={static method int m(IA a){return ...} }
 ta={implements IA //This line is the core of the solution
     method int ma(){return Utils.m(this);}}
 A=Use ta
\end{lstlisting}
\saveSpace
This code works: \Q@Utils@ relies on interface \Q@IA@ and the trait \Q@ta@
implements \Q@IA@.
\Q@ta@ is well typed: independently of what class name(s) will be
associated to its code, we know that such class(es) will implement
\Q@IA@. 
Therefore, while typechecking \Q@Utils.m(this)@ we can assume
\Q@this<:IA@.
 It is also possible to add a class \Q@B@ as follows:
\saveSpace
\begin{lstlisting}
  B=Use ta, { method int mb(){return this.ma();} }
\end{lstlisting}
\saveSpace
This also works.  \Q@B@ reuses the code of \Q@ta@, but has no knowledge of \Q@A@.
Since \Q@B@ reuses \Q@ta@, and \Q@ta@ implements \Q@IA@, also \Q@B@ implements \Q@IA@. 

Later, in Section \ref{sec:formal} we will provide the type
system. 
For now notice that in the former example the code is correct even if
no method called \Q@ma@ is explicitly declared.
DeepFJig and TR would require instead to explicitly declare an abstract \Q@ma@ method:
\saveSpace
\begin{lstlisting}
  B=Use ta, { method int ma()//not required by us
      method int mb(){return this.ma();} }
\end{lstlisting}\saveSpace
\noindent
The idea in \name is that such method is imported from trait \Q@ta@, exactly as in the Java equivalent
\saveSpace\begin{lstlisting}[language=Java]
  class B extends A{ int mb(){return this.ma();} }  
\end{lstlisting}
\saveSpace
where method \Q@ma@ is imported from \Q@A@.
This concept is natural for a Java programmer, but was not supported
in previous work \cite{BETTINI2013521,deep}. Those works require all
dependencies in code literals to be explicitly declared, so that the
code literal is completely self-contained. However, this results in
many redundant abstract method declarations.

\paragraph{Semantics of Use}
Albeit alternative semantic models for traits~\cite{Traits:ECOOP2003} have been proposed,
here we use the flattening model. This means that 
\saveSpace\begin{lstlisting}
A=Use ta
B=Use ta, { method int mb(){return this.ma();} }
\end{lstlisting}\saveSpace
 
\noindent is equivalent to the inlining the code of all used traits:
  
\saveSpace \begin{lstlisting}
A={implements IA method int ma(){return Utils.m(this);}}
B={implements IA
  method int ma(){return Utils.m(this);}
  method int mb(){return this.ma();} } 
 \end{lstlisting}
\saveSpace 
 This code is correct, and {\bf in the resulting code there is no mention of the trait
 \Q@ta@}. All the information about code-reuse/inheritance
  is just a private implementation detail of \Q@A@
 and \Q@B@; while subtyping is part of the class interface.


%To finish this section, Figure \ref{fig:compare} provides a summary of
%the differences between classes and traits. The comparison focus on
%the roles of traits and classes with respect to instantiation,
%reusability and whether the declarations also introduce new types or
%not.




