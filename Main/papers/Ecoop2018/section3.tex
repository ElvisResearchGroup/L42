\section{Improving Use}

To illustrate how 
\name improves the use of classes we model a simplified version of
Set and Bag collections first in Java, and then in \name.
The main benefit of the \name solution is that we can get reuse 
without introducing subtyping between Bags and Sets.
As shown below, this \textbf{improves the 
use of Bags} by eliminating logical errors arising from incorrect
subtyping relations that are allowed in the Java solution. 

\subsection{Sets and Bags in Java: the need of code reuse without subtyping}
An iconic example on why connecting inheritance/code reuse and
subtpying is problematic is provided by
LaLonde~\cite{LaLonde:1991:SSS:110673.110679}.  A reasonable
implementation for a \Q@Set@ is easy to extend into a \Q@Bag@ by
keeping track of how many times an element occurs.  We just need to
add some state and override a few methods.
For example in Java one could have:

\begin{lstlisting}
class Set {...//usual hashmap implementation
  private Elem[] hashMap;
  void put(Elem e) /*body*/
  boolean isIn(Elem e)/*body*/}
class Bag extends Set{...//for each element in the hash map,
  private int[] countMap;// keep track of how many times is there
  @Override void put(Elem e)/*body*/
  int howManyTimes(Elem e)/*body*/}
\end{lstlisting}

\noindent Coding \Q@Bag@ in this way avoids a lot of code
duplication, but \textbf{we induced unintended subtyping}! 
Since subclassing implies subtyping, our code breaks the Liskov substitution principle (LSP)~\cite{martin2000design}: not all bags are sets!
Indeed, the following is allowed:

\begin{lstlisting}
Set mySet=new Bag(); //OK for the type system but not for LSP
\end{lstlisting}
This encumbers the programmer:
to avoid conceptual errors that are not captured by the type system, 
they have to \textbf{use} \Q@Bag@ very carefully.


\paragraph{A (broken) attempt to fix the Problem in Java.}
One could \textbf{retroactively} fix this problem by introducing \Q@AbstractSetOrBag@
and making both \Q@Bag@ and \Q@Set@ inherit from it:
\begin{lstlisting}
abstract class AbstractSetOrBag {/*old set code goes here*/}
class Set extends AbstractSetOrBag {} //empty body
class Bag extends AbstractSetOrBag {/*old bag code goes here*/}
...
//AbstractSetOrBag type not designed to be used.
AbstractSetOrBag unexpected=new Bag(); 
\end{lstlisting}

This looks unnatural, since \Q@Set@ would extend \Q@AbstractSetOrBag@ without adding anything,
and we would be surprised to find a use of the type \Q@AbstractSetOrBag@.
Worst, if we are to constantly apply this mentally, we would introduce a very high number
of abstract classes that are not supposed to be used as types. Those classes would clutter the 
public interface of our classes and the code project as a whole.
A \textbf{use}able API should only provide the relevant information to the client.
In our example the information \Q@Set<:AbstractSetOrBag@ would be present in the public interface
of the class \Q@Set@, but such information is not needed to use the class properly!

Moreover, the original problem is not really solved, but only moved 
further away. For example, one day  we may need bags that can only store up to 5 copies of the same element.
We are now at the starting point again:
\begin{itemize}
\item either we insert \Q@class Bag5 extends Bag@ and we break the LSP; 
\item or we duplicate the code of the \Q@Bag@ implementation with minimal
  adjustments in \\* \Q@class Bag5 extends AbstractSetOrBag@;
\item or we introduce an
\Q@abstract class BagN extends AbstractSetOrBag@ and \\*\Q@class Bag5 extends BagN@
and we modify \Q@Bag@ so that  \Q@class Bag extends BagN@.
Note that this last solution is changing the public interface of the formerly released \Q@Bag@ class, and
this may even break retro-compatibility (if a client program was using
reflection, for example).
\end{itemize}

\subsection{Sets and Bags in \name}
Instead, in \name, if we were to originally declare
\saveSpace\begin{lstlisting}
Set={/*set implementation*/} 
\end{lstlisting}\saveSpace
Then our code would be impossible to reuse in the first place for any user of our library.
We consider this an advantage, since unintended code reuse runs into under-documented behaviour nearly all the time\footnote{See
``Design and document for inheritance or else prohibit
it''~\cite{Bloch08}: the
self use of public methods is rarely documented, thus is hard to understand the effects of overriding a library method.
}!
If the designer of the \Q@Set@ class wishes to make it reusable, they can do it explicitly by providing a set trait:
\saveSpace\begin{lstlisting}
set={/*set implementation*/} 
Set=Use set
\end{lstlisting}\saveSpace
Since \Q@set@ can never be used as a type, there is no reason to give it a {\bf fancy-future-aware} name like
\Q@AbstractSetOrBag@.
When bag will be added, the code will look either like
\saveSpace\begin{lstlisting}
set={/*set implementation*/} 
Set=Use set
Bag= Use set, {/*bag implementation*/}
\end{lstlisting}\saveSpace\saveSpace
or 
\saveSpace\saveSpace\begin{lstlisting}
set={/*set implementation*/} 
Set=Use set
bag=Use set, {/*bag implementation*/}
Bag=Use bag
\end{lstlisting}\saveSpace
Notice how, thanks to flattening, the resulting code for \Q@Bag@ is identical in both versions
and, as shown in Section 2, there is no trace of trait \Q@bag@. 
Thus if we are the developers of bags, we can temporarily go for the first version.
Then, when for example we need to add \Q@Bag5@ as discussed before,
we can introduce the \Q@bag@ trait without adding new undesired complexity for our old clients.


As a pleasurable accident, avoiding such code allows us to provide support for the
`\Q@This@' type and (in the extensions with nested classes seen later)
family polymorphism in a very simple way.

