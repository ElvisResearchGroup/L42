\saveSpace\saveSpace\section{Improving Reuse}\saveSpace

%\name allows reuse even when subtyping is impossible.
%\name traits do not induce a new (externally visible) type.
%However, locally in a trait, programmers can use the special self-type \Q@This@~\cite{bruce_1994,Saito:2009,ryu16ThisType} in order to denote the 
%type of \Q@this@.
%That is, a program is agnostic to what the \Q@This@ type is, so that it can
%be later assigned to any (or many) classes. 
%The idea is that during flattening, \Q@This@ will be replaced with the actual class name.
%In this way, \name allows reuse even when subtyping is
%impossible. For example for \emph{binary
%  methods}~\cite{bruce96binary} where the method parameter has type \Q@This@. 
%This type of situations is the primary motivator
%for previous work aiming at separating inheritance from subtyping~\cite{cook}.
%Leveraging on the \Q@This@ type, we can also provide self-instantiation (trait methods can create instances of the class using them) and smoothly integrate state and traits: a challenging problem that has limited the flexibility of traits and
%reuse in the past.

%\subsection{Managing State}

%\name improves reuse in many different ways,

To illustrate how \name improves reuse,
we will show a novel approach
to deal with \emph{state} in traits.
Moreover, we provide self-instantiation (trait methods can create instances of the class using them)
 and smoothly integrate state and traits: a challenging problem that has limited the flexibility of traits and
reuse in the past.

The idea of summing pieces of
code is elegant, and successful in module
composition languages~\cite{ancona2002calculus} and several trait
models~\cite{Traits:ECOOP2003,Bergel2007,BETTINI2013521,fjig}.  However the research
community is struggling to make it work with object state (constructors
and fields) while achieving the following goals:

\begin{itemize}
%complicated discussions on this point \item keep sum associative and commutative,
\item managing fields in a way that borrows the elegance of summing methods;
\item actually initializing objects, leaving no null fields;
\item making it easy to add new fields;
\item allowing a class to create instances of itself (self instantiation).
\end{itemize}

\subsection{State of the art}
In the related work we will show some alternative ways to handle
state.
We first present the state of the art solution, where 
traits have only methods but classes have also fields and constructors.
The idea is that the trait code just uses getter/setters/factories, while leaving
to classes the role to finally define the fields/constructors. That
is in this state of the art solution, classes have a syntax richer than traits, allowing
declaration for fields and constructors. 

\paragraph{Modelling Points} Consider, for example, two simple 
traits that deal with \emph{point} objects. That is, points
in the cartesian plane (with coordinates \lstinline{x} and
\lstinline{y}). The first trait provides a \emph{binary method} that 
sums the point object with another point to return a new point. 
The second trait provides a similar operation that does multiplication 
instead.
\saveSpace 
\begin{lstlisting}
  pointSum: { method int x()  method int y()//getters
    static method This of(int x,int y)//factory method
    method This sum(This that){
      return This.of(this.x()+that.x(),this.y()+that.y());//self instantiation
    }}
  pointMul: { method int x() method int y()//repeating getters
    static method This of(int x,int y)//repeating factory
    method This mul(This that){
      return This.of(this.x()*that.x(),this.y()*that.y());
    }}
\end{lstlisting}
\saveSpace
\noindent As we can see, all the state operations (the getters for the 
\lstinline{x} and \lstinline{y} coordinates) are represented as {\bf abstract} methods.
Notice the abstract \Q@static method This of(..)@ which acts as a constructor
for points. 
As for instance methods, they are late bound:  flattening can provide an implementation for them.
Abstract static methods are very similar to the original concept of member functions in the module composition setting~\cite{ancona2002calculus}.
Following the model of traits and classes common in literature \emph{in a traditional trait model}~\cite{Traits:ECOOP2003},
we can now compose the two traits, by \textbf{adding glue-code}
to implement methods \Q@x@,\Q@y@ and \Q@of@.
This approach is verbose but very
powerful as illustrated by Wang et al.~\cite{wang2016classless}.
\begin{lstlisting}
class PointAlgebra=Use pointSum,pointMul, {//not 42 code
    int x   int y//unsatisfactory state of the art solution
    constructor PointAlgebra(int x, int y){ this.x=x   this.y=y }
    method int x(){return x;}//repetitive code
    method int y(){return y;}// in traits terminology, this is all "glue code"
    static method This of(int x, int y){return new PointAlgebra(x,y);}
    }
\end{lstlisting}
%\bruno{We talk about withers later on. So I think we should consider
 % having withers in this code, so that readers can understand what 
%withers are!}
%\marco{with withers it will look more complicated}

\noindent 

With a slightly different syntax, this approach is available in both Scala and Rust, and they both require glue code.
It has some advantages, but also disadvantages: 

\begin{itemize}

\item {\bf Advantages:} This approach is associative and commutative, even self construction
  can be allowed if the trait requires a static method
  returning \Q@This@. The class will then implement the methods returning \Q@This@
  by forwarding a call to the constructor.
  
\item {\bf Disadvantages:}
   The semantic of \Q@Use@/code composition of a model with fields and constructors is necessarily
   more complex than a model with methods only.
 There is no way for a trait to specify a default value for a field.
The class needs to handle all the state, even state that is conceptually
   private of such trait. 
 Moreover, writing such obvious definitions to close
  the state/fixpoint in the class 
   with the constructors and fields and getter/setters and factories is tedious and error prone; previous work shows that such code can be automatically
   generated~\cite{wang2016classless}.
\end{itemize}

\subsection{Our proposed approach to State: Coherent Classes}

In \name there is no need to generate
code, or to explicitly write down constructors and fields. In fact in
\name there is not even syntax for those constructs!  The idea is that
any class that could be completed in the obvious way  is a
  complete \textbf{coherent} class.  In most other languages, a class is
abstract if it has abstract methods.  Instead, we call a class
abstract only when the set of abstract methods is not coherent. That
is, the unimplemented methods cannot not be automatically recognised
as factory, getters and setters. Methods recognised as factory, getters and setters are called
\emph{abstract state operations}.
  
A more detailed definition of coherent
classes is given next, and is formally defined in Section \ref{sec:formal}:
\begin{itemize}
\item a class with no abstract methods is coherent (just like Java
  \Q@Math@, for example). Such classes have no instances and are only useful for calling static methods.
\item a class with a single abstract \Q@static@ method returning \Q@This@
is coherent if all the other abstract methods can be seen as \emph{abstract state
operations} over one of its argument.
For example,
if there is a \Q@static method This of(int x, int y)@ as before,
then
\begin{itemize}
\item a method \Q@int x()@ is interpreted as an abstract state method: a \emph{getter} for \Q@x@.
\item a method \Q@void x(int that)@ is a \emph{setter} for x.
\end{itemize}
\end{itemize}
\noindent
While getters and setters are fundamental operations, it is possible to
support more operations. For example:
\begin{itemize}
\item \Q@method This withX(int that)@
may be a ``wither", doing a functional field update: it creates a new instance that is like \Q@this@ but where field \Q@x@ has now \Q@that@ value.
\item \Q@method void update(int x,int y)@
may do two field updates at a time.
\item\Q@method This clone()@ may do a shallow clone of the object.
\end{itemize}

We are not sure what is the best set of abstract state operations yet,
but we think this could become a very interesting area of research.
The work by Wang et al.~\cite{wang2016classless} explores a particular
set of such abstract state operations.

\paragraph{Points in \name:}
In \name and with our approach to handle the state, 
\lstinline{pointSum} and \lstinline{pointMul} can indeed be directly composed;
It works because resulting class is coherent.
\saveSpace
\begin{lstlisting}
  PointAlgebra:Use pointSum,pointMul //no glue code needed
\end{lstlisting}  
\saveSpace
\noindent
  Note how we can declare the methods independently and compose the result
  as we wish. 

  \paragraph{Improved solution} So far the current solution still
  repeats the abstract methods \Q@x@, \Q@y@ and \Q@of@.
  Moreover, in addition to \Q@sum@ and \Q@mul@ we may want many
  operations over points. It is possible to improve reuse
  and not repeat such declaration by abstracting the common
  declaration into a trait \Q@p@: 
\saveSpace
\begin{lstlisting}
  p: { method int x() method int y()
    static method This of(int x,int y)
    }
  pointSum:Use p, { 
    method This sum(This that){
      return This.of(this.x()+that.x(),this.y()+that.y());
    }}
  pointMul:Use p, { 
    method This mul(This that){
      return This.of(this.x()*that.x(),this.y()*that.y());
    }}
  pointDiv: ...
  PointAlgebra:Use pointSum,pointMul,pointDiv,...
\end{lstlisting}
\saveSpace      
Now the code is fully modularized, and each trait defines exactly one method.

\paragraph{Case Study}
In order to evaluate our approach,
we performed a case study:
we consider 4 different operations \Q@Sum@, \Q@Subtraction@, \Q@Multiplication@ and \Q@Division@.
These operations can be combined in 16 different ways.
We coded this example in four different styles:
(a) Java7 \emph{(= 115 lines)}%
\footnote{
Since we want to focus on the actual code, while counting line numbers we \textbf{omit} white lines and lines containing only open/closed parenthesis.
}%
, (b) Classless Java \emph{(= 82 lines)},
(c) Scala \emph{(= 81 lines)} and (d) \name \emph{(= 32 lines)}.
We chose Classless Java~\cite{wang2016classless} since it is a novel approach allowing to encode traits in Java levering on 
Java 8 default interface methods.
We then chose Java7, that lacks the features needed to encode traits, to show the impact of this feature.
Finally, the comparison with Scala is interesting 
since
it has good support for traits, and using abstract types, is possible to support `\Q@This@' type.
Rust is similar to Scala in this regard; we believe we would get similar results by comparing against either Scala or Rust.

We observed that in Java7 we had to duplicate\footnote{A duplicate body is repetition of identical code (may have different types in its scope/environment). The first occurrence is not counted. } 28 method bodies across the 16 classes.
Of these, 11 method bodies were duplicated because Java does not support multiple inheritance
 and the remaining 17 bodies had to be duplicated to ensure that the right type
 is returned by the method. Those could be avoided if Java supported
 the `\Q@This@' type.
 On the other hand, the solution in \name was much more compact since we could efficiently
reuse traits (this is why the number of toplevel concepts in \name was larger i.e. 21 due to the
 presence of traits in this solution).
In the detail, Java required 6 lines for the initial \Q@Point@ class,
5 lines for each of the 4 arithmetic operations, 7 lines for each of the 6 combinations
of two different operations, 9 lines for each of the 4 combinations of three different 
operations and finally 11 lines for the class with all the four operations.


 The solution in Classless Java was slightly smaller than Java7,
 but was still quite longer than the \name solution because it still had to redefine the
 sum/sub/etc operations in each of the classes because of limited support of the `\Q@This@' type, thus it also has 28 duplicated method bodies.

Finally, we compare it with a Scala solution.
%Scala has good support for traits, and using abstract types, is possible to support `\Q@This@' type.
There is no need for duplicate method bodies in Scala.
However, for `\Q@This@' instantiation we need to define abstract methods, that will be implemented in the concrete classes.
The Scala solution have the same exact advantages
of our proposed solution, and the declaration
of the trait is about the same size: 
5(point state)+3*4 (point operations).
However the glue code (the code needed to compose the traits into usable classes) is quite costly:
4 lines for each of the 16 cases.
In \name a single line for each case is sufficient.

This example is the best-case scenario for \name: this is an example where a maximum level of reuse
 is required since we considered the case where all the 16 permutations needed to be materialized in the code.
The results in each of the styles are presented below.
In all our case studies, to make a meaningful comparison, we formatted all code in a readable and consistent manner;
on the other hand for space limitations, the code snippets presented in the article
are formatted to be very compact.

\noindent\begin{tabular}{l|l|l|l}
Language       & Lines of code & Number of members & Number of top-level concepts (classes or traits)\\
\hline
Java7           &   115=6+5*4+7*6+9*4+11        & 50                &      16\\
Classless Java &   \ \ 82=3+3*4+5*6+7*4+9          & 34                &      16\\
Scala          &   \ \ 81=5+3*4+4*16  &  40                 &    21 = 16+4+1\\
\name          &   \ \ 32=4+3*4+1*16 & 7                 &      21 = 16+4+1\\
\end{tabular}

\subsection{State Extensibility}
Programmers may want to extend points with more state. For example 
they may want to add colors to the points. A first attempt at doing
this would be:
\saveSpace
\begin{lstlisting}
  colored:{ method Color color() }
  CPoint:Use pointSum,colored //Fails: class not coherent
\end{lstlisting}
\saveSpace
This first attempt does not work: the abstract color method
is not a getter for any of the parameters of 
\Q@ static method This of(int x,int y)@. 
A solution is to provide a richer factory:
\saveSpace
\begin{lstlisting}
  CPoint:Use pointSum,colored,{
    static method This of(int x,int y){ return This.of(x,y,Color.of(/*red*/));}
    static method This of(int x,int y,Color color)
    }
\end{lstlisting}
\saveSpace
\noindent 
where we assume support for overloading on different number of parameters.
This is a reasonable solution, however the method \Q@CPoint.sum@ resets
the color to red: we call the \Q@of(int,int)@ method, that now
delegates to \Q@of(int,int,Color)@ by passing red as the default field
value.  What should be the behaviour in this case?  If our abstract
state supports withers, we can use
\Q@this.withX(newX).withY(newY)@, instead of writing \Q@This.of(...)@, in order to preserve the color from
\Q@this@.  This solution is better but still not satisfactory since the color from \Q@that@ is ignored.

\paragraph{A better design}
If the designer of point (i.e. trait \Q@p@) is designing for reuse and extensibility, then 
a better design would be the following:  
\saveSpace\begin{lstlisting}
  p: { method int x() method int y() //getters
    method This withX(int that) method This withY(int that)//withers
    static method This of(int x,int y)
    method This merge(This that) //new method merge!
    }
  pointSum:Use p, { 
    method This sum(This that){
      return this.merge(that).withX(this.x()+that.x()).withY(this.y()+that.y());
    }}
  colored:{method Color color()
    method This withColor(Color that)
    method This merge(This that){ //how to merge color handled here
      return this.withColor(this.color().mix(that.color());
    }}
  CPoint:/*as before*/
\end{lstlisting}  \saveSpace
  \noindent This design allows merging colours, or any other kind of state we may want to add
  following this pattern.%\bruno{worried that withers are not explained enough.}

\paragraph{Independent Extensibility}
  Of course, quite frequently there can be multiple independent
  extensions~\cite{Zenger-Odersky2005} that need to be composed. Lets suppose that 
  we could have a notion of \Q@flavoured@ as well.   
  In order to compose \Q@colored@ with \Q@flavored@ we would
  need to compose the merge operation inside both of them; to this aim \use\ is not sufficient, since we need to combine the implementation of 2 different versions of \Q@merge@.
We introduce here an operator called \Q@super@, that
 makes a method abstract and
moves the implementation to another name. This is very useful to implement super calls
 and to compose conflicting implementations.
\noindent Consider the simple \Q@flavored@ trait:
\saveSpace\begin{lstlisting}
  flavored:{
      method Flavor flavor() //very similar to colored
      method This withFlavor(Flavor that)
      method This merge(This that) //merging flavors handled here
        this.withFlavor(that.flavor())}//inherits "that" flavor
\end{lstlisting}  \saveSpace\saveSpace

\noindent In order to merge \Q@colored@ and \Q@flavored@ we use  \Q@super@ to introduce method selectors \Q@_1merge@ and \Q@_2merge@
to refer to the version of \Q@merge@ as defined in the first/second element of \use.

\saveSpace\begin{lstlisting}
  FCPoint:Use
    colored[super merge as _1merge],
    flavoured[super merge as _2merge],
    pointSum,{
      static method This of(int x,int y){
        return This.of(x,y,Color.of(/*red*/),Flavor.none());}
      static method This of(int x, int y,Color color,Flavor flavor)
      method This merge(This that){//merge conflict is solved 
        return this._1merge(that)._2merge(that);}//by calling the two versions
      }
\end{lstlisting}  \saveSpace\saveSpace

Note how we are leveraging on the fact that the code literal
 does not need to be complete, 
thus we can just call \Q@_1merge@ and \Q@_2merge@ without
 declaring their abstract signature explicitly.


%, as in the following example:
%\begin{lstlisting}
%t:{method bool geq(This x) x.leq(this)   method bool leq(This x) x.qeq(this) }
%C:Use t[restrict geq],{method bool geq(This x){return /*actual geq impl*/}}
%\end{lstlisting}

\paragraph{Case Study}
To understand how easy it is to extend the state in this
way we compare the former code with an equivalent version in
Java.
For this example, in Java we encode \Q@Point@ with no operations,
\Q@PointSum@ reusing \Q@Point@ adding a functional \Q@sum@ operation,
a \Q@CPoint@ reusing \Q@PointSum@ with a \Q@Color@ field
and a \Q@FCPoint@ reusing \Q@CPoint@ with a \Q@Flavour@ field.
This second case study represents a \textbf{worst case scenario} for \name against Java because we model just a single chain of reuse,
easily supported in plain Java by single inheritance.
Like the previous experiment, we still found that the Java solution was longer (47 lines) than that in \name (33 lines). This is caused by the absence support for the `This' type, where the withers in each of the \Q@CPoint@/\Q@FCpoint@ classes had to be repeated to make sure that the returned type will be correct (the number of members in Java were 27 while 24 (less 3) in \name).

Complex patterns in Java~\cite{}\bruno{references} allow to support the `\Q@This@' type and the `\Q@This@' type instantiation but they require a lot of set-up code. We started experimenting using those patterns, but it soon became very clear that the resulting code of this approach would have been even larger; albeit without duplicated code.
Note how the Java code is less modular than the \name code, since \Q@Colored@ and \Q@Flavoured@ do not exist
as individual concepts.

We also compare with a solution in Scala, offering the same level of reuse and modularity of 
code, but again is more verbose than \name, and requires more members (31), an indication it may be logically heavier too.
We define the main \Q@tPoint@ trait (8 lines),
the \Q@tPointSum@ operation (3), the two 
\Q@tColored@ and \Q@tFlavoured@ traits (6*2)
and \Q@CPoint@ and \Q@CFPoint@ classes (12+18).
Note how the major benefit of \name is in the reduction
of the amount of glue-code needed to generate 
\Q@CPoint@ and \Q@CFPoint@ (4+9).

\noindent The results for the second experiment are presented below.

\noindent \begin{tabular}{l|l|l|l}
Language       & Lines of code & Number of members & Number of toplevel concepts (classes or traits)\\
\hline
Java           &  47= 10+9\ \ \ \ + \ \ \ \  13+15         &    27             &     6\\
Scala          &  53= \ \ 8+3+6*2+12+18        &    31             &         6\\
\name          &  33=\ \ \ 7+3+5*2+\ \ 4+9      &    24             &         6\\
\end{tabular}

