\subsection{Static Verification in OO}
The mechanism by which programs are verified to be correct is a large research area, one common approach is \emph{automated static verification}, where a tool tries to automatically verify that code is correct during compilation, such tools typically encode the correctness of a program as a mathematical theorem, and use an SMT solver to verify it~\cite{?}. Automated static verification is often easier to use and provides stronger guarantees compared to computer aided theorem proving~\cite{?} and runtime verification~\cite{?} (respectively). 

For simplicity, we will only consider automated static verification. The details as to how this is done is also out of scope of this paper. However, our approach should be easily extendible to other verification techniques.

\subsubsection{Contracts}
There are many notions as to what it means for a program to be correct, once can say that a program is correct if it conforms to an arbitrary mathematical \emph{specification}~\cite{?}. Rather than specifying a program as a whole, a simpler and more modular approach is to specify individual components by giving them \emph{contracts}~\cite{?}. For functions, contracts are typically just pre and post-conditions~\cite{?}. In this paper, we will use the annotation /@requires($predicate$)/ to specify a precondition, and /@ensures($predicate$)/ to specify a postcondition: where $predicate$ is a boolean expression in terms of /this/, the parameters of the method, and for the /@ensures/ case, the /result/ of the method call. We will allow specifying the annotations multiple times, so that /@requires($p_1$) @requires($p_2$)/ is equivalent to /@requires($p_1\ $ || $\ p_2$)/, and /@ensures($p_1$) @ensures($p_2$)/ is equivalent to /@ensures($p_1\ $ && $\ p_2$)/. Also, if either annotation was not specified, we will assume a $predicate$ of /true/. 

This means that the annotated \emph{function} is correct if whenever the precondition holds, executing the function will cause the postcondition to hold. A \emph{program} is then correct, if each of its functions is correct, and their precondition holds whenever they are called. For example one could write an implementation of exponentiation using repeated squaring like this:
\begin{lstlisting}
@requires(x != 0 || exp != 0) // since 0**0 is undefined.
@ensures(result == x**exp)
Int pow(Int x, Nat exp) {
  if (exp == 0)          return 1;
  else if (exp == 1)     return x;
  else if (exp % 2 == 0) return pow(x*x, exp/2); // even power
  else                   return x*pow(x, exp - 1);}  // odd power
\end{lstlisting}
This says that for the \emph{implementation} of /pow/ to be `correct', whenever we don't have /x == 0/ and /exp == 0/, /pow(x, exp)/ must equal /x**exp/\footnote{We use the notation /x**y/ to mean `/x/ to the power of /y/'.}. For a \emph{call} /pow(x, exp)/ to be correct, either /x/ or /y/ must be nonzero.

For simplicity, we will only consider pre and post conditions in a purely functional OO language. Contracts can be extended to other cases, such as to specify class invariants~\cite{?}, termination~\cite{?}, function purity (where the language is an otherwise imperative language)~\cite{?}, ownership~\cite{?}, etc. However such properties are still typically encoded as pre and post-conditions.

\subsection{OO and Static Verification}
One of the main things you can do with OO is create an abstract specification of behaviour (an `interface') and write code that will work with any concrete instance of this interface. Naturally, there is much research on verifying the implementations of such interfaces~\cite{?}. The common approach is to apply pre and post-conditions to methods, but only verify the post-conditions for concrete \emph{implementations} of them. Consider for example an OO style abstract class//interface:
\begin{lstlisting}
abstract class List {
  Nat length(); // abstract method
  @requires(index < this.length())
  Object get(Nat i);
  
  @requires(this.contains(o))
  @ensures(this.get(result) == o)
  Bool contains(Object o);
  
  @ensures(result == exists i : this.get(i) == o)
  Bool contains(Object o);
}
\end{lstlisting}
What this means is that any \emph{object} claiming to be a /List/ must have /get/, /contains/ and /find/ methods satisfying their declaration above, namely their type signatures and contracts. For static verification, this means that any \emph{call} to a method of /List/ must satisfy the precondition, and any \emph{implementation} of such a method must ensure that the postcondition holds (whenever the precondition holds on entry to the method). Because none of the methods in /List/ are implemented, they are trivially correct. 

One can use \emph{inheritance} to write implementations of these methods, for example consider the following two classes:
\begin{lstlisting}
abstract class ListFinder extends List {
  ensures(forall i : i < result ==> this.result(i) != o)
  //@requires(true) // implicit
  //@requires(this.contains(o)) // inherited
  //@ensures(this.get(result) == o) // inherited
  Nat find(o) {
    for (Nat i = 0; i < this.length(); i++) {
      if (this.get(i) == o) { return i; } }
    return this.length() + 1; }
}

abstract class ListContains extends List {
  //@ensures(result == exists i : this.get(i) == o) // inherited
  Bool contains(Object o) {
    for (nat i = 0; i < this.length(); i++) {
      if (this.get(i) == o) { return true; } }
    return false; }}
\end{lstlisting}

The above classes \emph{inherit} the methods of /List/, including their contracts, so their code can call these methods, and a static verifier can use these contracts. Importantly, the contract annotations of overriden methods will be inherited, as seen above, this ensures the Liskov Substitution Principle (LSV)~\cite{?}. Thus the precondition can be weakened, and the postcondition strengthened. An abstract method often overrides another abstract method, in order to refine contracts like this.

Finally, we can compose these two classes (using /extends/), and add the missing methods, to create a complete implementation of /List/:
\begin{lstlisting}
class LinkedList extends ListContains, ListFinder {
	... // implementations for get and length
}
\end{lstlisting}
\begin{comment}
Suppose that we want to write manual implementations of /pow/, specialised to a  specific exponent. Rather than fully inlining the recursion for each power, we can use \emph{inheritance} to reuse the implementations for previous exponents. Example:

\begin{lstlisting}
class Pow0 {
  @requires(x != 0)
  @ensures(result == x**0)
  Int pow(Int x) { return 0; }
}

...
class Pow4 extends Pow2 {
  @ensures(result == x**4)
  Int pow(Int x) { super.pow(x*x); }
}
...
\end{lstlisting}
Each class can be statically verified by relying on the previous one also being verified. Of course the above code dosn't look very usefully

\begin{comment}\subsection{Contracts in OO}
The concepts of pre and post conditions can be extended to object oriented programming~\cite{?}, consider for example:
\begin{lstlisting}
interface List {
  @requires(this.contains(o))
  @ensures(this.get(result) == o)
  Nat find(Object o);
	...
}
\end{lstlisting}
This means that in order for a call /a_list.find(o)/ to be correct, /a_list/ must contain /o/. In order for an implementation of /List.find/ to be correct, /this.find(o)/ must return an index that corresponds to /o/, assuming that /this/ contains /o/. Note that \emph{abstract} methods (such as /List.find/), are always trivially correct, however the type system of the language should ensure that the methods of all \emph{objects} are implemented.
\end{comment}


%This can be further extended to the conventional class-based multiple-inheritence, for example, consider the following (correct) implementation of the list interface:
% \begin{lstlisting}
% abstract class List {
%   Nat find(Object o)
%}
 
% abstract class FindFunction {
%  Nat get(result) {}
%  Nat find(Object o)
%}

%class AList extends ListBase, ListFind implements List { }
%\end{lstlisting}