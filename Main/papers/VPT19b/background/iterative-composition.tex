\subsection{Iterative Composition}
Iterative composition~\cite{?} takes the concept of trait composition one step forward, by treating traits as first class objects, and allowing arbitrary code to execute at meta-time. This allows for fully met-circular meta-programming. In this way, classes can be declared from arbitrary expressions (of type /Trait/), and methods can take and return /Trait/s. For example, suppose we want to generate specialised code for /pow/ based on the exponent\footnote{We use /static/ /class/ to mean a class with no constructors or instances, and where each member is implicitly /static/.}:
\begin{lstlisting}
Trait pow_generate(Nat exp) {
  if (exp == 0)
    return static class { 
      Int pow(Int x) { return 1; }
    };
  else if (exp == 1)
    return static class {
       Int pow(Int x) { return x; }
     };
 else if (exp % 2 == 0)
   return pow_generate(exp/2).extend(static class {
     Int super_pow(Int x);
     Int pow(Int x) { return super_pow(x*x); }
   });
 else 
   return pow_generate(exp - 1).extend(static class {
     Int super_pow(Int x);
     Int pow(Int x) { return x*super_pow(x); }
   });}
\end{lstlisting}
In the above code, we have a normal looking recursive /pow_generate/ method: the /pow_generate(...)/ calls (recursively) generate the bodies of /super_pow/. The calls to /super_pow/ (in the trait literals) correspond to the recursive calls in our original /pow/; . The above method can now be called to generate the body of a class:

\begin{lstlisting}
Pow7: pow_generate(7);

// Equivalent to:
Pow7: static class {
  Int pow(Int x) { return x*super1_pow(x); } // x**7

  private Int super1_pow(Int x) { return super2_pow(x*x); } // x**6
  private Int super2_pow(Int x) { return x*super3_pow(x); } // x**3
  private Int super3_pow(Int x) { return super4_pow(x*x); } // x**2
  private Int super4_pow(Int x) { return x; } // x**1
}

// Which could then be inlined/optimised to:
Pow7: static class {
  Int pow(Int x) { 
    Int x2 = x*x; // x**2
    Int x4 = x2*x2; // x**4
    return x*x2*x4; } // Since 7 = 1 + 2 + 4
}
\end{lstlisting}
