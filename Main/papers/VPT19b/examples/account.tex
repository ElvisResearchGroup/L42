\subsection{Iterative Composition Example}
We can use introspection to automatically generate code for an input trait. Consider for example the following interface\footnote{An abstract class with only abstract methods.}):
\begin{lstlisting}
Account: interface {
	Nat income();

	@ensures(result <= this.income())
	Nat expenses();
};
\end{lstlisting}

\noindent The idea being that an account cant spend more money than they it has.

Now Suppose we want to create a new type of account, a combined one with multiple sub accounts:
\begin{lstlisting}
BuisnessAccount: class implements Account {
	Account salary;
	Account sales;
	Account property;
	
	Nat income() {
		return this.salary.income() 
			+ this.sales.income() + this.property.income(); }
	
	@ensures(result <= this.income())
	Nat expenses() { 
		return this.salary.expenses() 
			+ this.sales.expenses() + this.property.expenses();}
		
	...
};
\end{lstlisting}

Now we can statically verify the above code, but what if we want to combine accounts like this frequently? This could take lots of time for both programmers and automated static verifiers, instead we can create a function /combine_accounts/ that does this for us:
\begin{lstlisting}
// Equivalent to the above
BuisnessAcount: combine_accounts(class { 
	Account salary;
	Account sales;
	Account property;
	... });
\end{lstlisting}

Were we define /combine_accounts/ like this:
\begin{lstlisting}
Trait combine_accounts(Trait input) {
	Trait res = input.extend(class implements Account {
		Nat income() { return 0; }

		@ensures(result <= this.income())
		Nat expenses() { return 0; }
	});
	for (FieldDeclaration fd : input.fields()) {
		res = res.extend(class implements Account {
			Nat super_income();
			
			@ensures(result <= this.super_income())
			Nat super_expenses();

			Account account; // Will be renamed to fd.name
			Nat income() { 
				return this.super_income() + this.account.income(); }

			@ensures(result <= this.income())
			Nat expenses() { 
				return this.super_expenses() + this.account.expenses(); }
		}.rename("account", fd.name)); }
	return res;
}
\end{lstlisting}

\noindent Where /trait.fields/ returns AST structures for each field in the given trait, and /trait.rename($a$, $b$)/ returns /trait/ but with the declaration named $a$ renamed to $b$. The code above is straightforward: it adds an implementation of /Account/ (suitable if /input/ has no fields) to the /input/ trait, it then iterates over every field in /input/ and adds its corresponding /income()/ and /expenses()/ to the implementations in /res/. We use contract matching on the  /super_/ methods in the same was as in /pow_generate/. As with /pow/, the above two trait literals can be statically verified, and so we guarantee that the result of /combine_accounts/ is correct.