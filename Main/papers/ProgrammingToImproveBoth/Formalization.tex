\saveSpace\section{Formalisation}\label{sec:formal}
\saveSpace

Here we show a simple formalization for the language we presented so far.
We also model nested classes, but in order to avoid uninteresting complexities, we assume that
all type names are fully qualified from top level, so the examples shown before should be
written like: \Q@This.Exp@, \Q@This.Sum@, etcetera.
In a real language, a simple pre-processor may take care of this step.

In most languages, when implementing an interface, the programmer may avoid repeating abstract methods
they do not wish to implement, however
to simplify our formalization, we consider source code always containing all the methods imported from interfaces. In a real language, a normalisation process
may hide this abstraction\footnote{
In the full 42 language scoping is indeed supported by an initial de-sugaring, and a normalisation phase takes care of importing methods from interfaces.
}.
We also consider a binary operator sum (\Q@+@) instead of the nary operator \Q@Use@.
Figure 1 contains the complete formalization for \name: syntax,
compilation process, typing, and finally reduction.




\begin{figure}
%NEW FORMALISATION below
% Syntax
% D::=TD|CD
% TE::=t:E Trait Decl Expr
% CE::=C:E Class Decl
% TD::=t:L
% CD::=C:L
% E::= L| t| E+E | E[rename T.m1->m2]|E[rename T1->T2]|E[redirect T1->T2]
% L::= {interface? implements Ts Ms}//all L are like LC in 42
% T::=C|C.T // .T is a shortcut for This.T
% M::= static? method T m(T1 x1..Tn xn) e? | CD
% e::= x| e.m(es) | T.m(es)

\begin{bnf}
\prodFull\mID{\mt\mid\mC}{class or trait name}\\
\prodFull\mDE{\mID~\terminalCode{=}~\mE}{Meta-declaration}\\
\prodFull\mD{\mID~\terminalCode{=}~\mL}{Declaration}\\
\prodFull\mE{\mL \mid \mt \mid \mE\,\terminalCode{+}\mE
\mid \ldots
%\mid \mE\terminalCode{[rename}\ \mT\terminalCode{.}\mm_1\ \terminalCode{to}\ \mm_2\terminalCode{]}
}{Code Expression}\\
%\prodNextLine{
% \mid
%\mE\terminalCode{[rename}\ \mT_1\ \terminalCode{in}\ \mT_2\terminalCode{]} \mid
%\mE\terminalCode{[redirect}\ \mT_1\ \terminalCode{to}\ \mT_2\terminalCode{]}}{Code Expression}\\
\prodFull\mL{
\oC \Opt{\terminalCode{interface}}\ \terminalCode{implements} \overline~\mT\ \overline\mM\ \cC}{Code Literal}\\
\prodFull\mT{\mC \mid \mC\terminalCode{.}\mT}{Type}\\
\prodFull\mM{\Opt{\terminalCode{static}}\ \terminalCode{method}\ \mT\ \mm\oR\overline{\mT\,\mx}\cR ~\Opt\me \mid \mC~\terminalCode{=}~\mL}{Member}\\

\prodFull\me{\mx \mid \me\terminalCode{.}\mm\oR\overline\me\cR \mid \mT\!\terminalCode{.}\mm\oR\overline\me\cR}{Expression}\\

\prodFull{v_{{\smallDs}}}{\mT\terminalCode{.}\mm\oR\overline{v_{{\smallDs}}}\cR
\text{,  where }\mm \text{ is abstract in }\overline\mD(\mT)
}{value}\\

\prodFull{\ctx_{\smallDs}}{[]\mid
\ctx_\smallDs\terminalCode{.}\mm\oR\overline\me\cR
\mid
v_{{\smallDs}}\terminalCode{.}\mm\oR\overline{v_{{\smallDs}}},\ctx_\smallDs,\overline\me\cR
\mid
\mT\terminalCode{.}\mm\oR\overline{v_{{\smallDs}}},\ctx_\smallDs,\overline\me\cR
}{evaluation context}\\

\prodFull{\ctx_c}{[]\mid\ctx_c\,\terminalCode+\mE \mid \mL\,\terminalCode+\ctx_c \mid\ldots}{compilation context}\\

\prodFull{\ctx}{[]\mid\ctx\,\terminalCode+\mE \mid \mE\,\terminalCode+\ctx \mid\ldots}{ctx}\\

\prodFull\mG{\mx_1{:}\mT_1,\ldots,\mx_n{:}\mT_n}{variable environment}
\end{bnf}\\

\newcommand{\pushLeft}{\!\!\!\!\!\!\!\!\!\!\!\!}
\newcommand{\rowSpace}{\\\vspace{2.5ex}}
$\begin{array}{l}

%       D.E -->^+_CDs L  CDs|-CD1:OK .. CDs|-CDn:OK       CDs=CD1..CDn
% (top)---------------------------------------------------------------    D.E not of form L
%      CD1..CDn CDs' D Ds -> CDs CDs' D[with E=L] Ds

\rowSpace
{\pushLeft\inferrule[(top)]{
  \mE_0 \xrightarrow[\smallDs]{} \mE_1
  \\
  \forall \mD\in\overline\mD,
  \overline\mD\vdash\mD:\text{OK}
  }{ 
    \overline\mD \ \overline{\mD'}\ \mID\terminalCode{=}\mE_0 \ \overline{\mDE}
    \rightarrow 
    \overline\mD\ \overline{\mD'}\ \mID\terminalCode{=}\mE_1\ \overline{\mDE}
  } %{\overline\mD=\mD_1..\mD_n }
\quad\quad
%
%     ------------------------
%      t -->_CDs CDs(t)

 \inferrule[(look-up)]{
    \ 
  }{ 
    \mt \xrightarrow[\smallDs]{}\ \overline\mD(\mt)
  }
\quad

\inferrule[(ctx-c)]{
    \mE_0 \xrightarrow[\smallDs]{}\ \mE_1
  }{ 
     {\ctx}_c[\mE_0] \xrightarrow[\smallDs]{}\ {\ctx}_c[\mE_1]
  }
\quad
%
%      --------------------------      L = L1+L2
%      L1+L2  -->_CDs L

\inferrule[(sum)]{
    \
  }{ 
     \mL_1\,\terminalCode{+}\mL_2 \xrightarrow[\smallDs]{}\ \mL
  }\mL = \mL_1+\mL_2
}\rowSpace

%  C;CDs,C=L |- L[This=C] :OK
% ----------------------------------------- coherent(L)
%  CDs|-C=L : OK

{\pushLeft\inferrule[(CD-OK)]{
    \mC;\overline\mD,\mC\terminalCode{=}\mL_1\vdash \mL_1\ :\text{OK}
  }{ 
     \overline\mD \vdash \mC\terminalCode{=}\mL_0\ :\text{OK}
  }
\begin{array}{l}
\mL_1=\mL_0[\terminalCode{This}=\mC]\\
\mathit{coherent}(\mC,\mL_1)
\end{array}
\quad\quad 

%    This;CDs,This=L |- L :OK
%----------------------------------------
%    CDs|-t=L : OK

\inferrule[(TD-OK)]{
    \terminalCode{This};\overline\mD,\terminalCode{This=}\mL\vdash \mL\ :\text{OK}
  }{ 
     \overline\mD \vdash \mt\terminalCode{=}\mL\ :\text{OK}
  }
}\rowSpace

%  forall i in 1..k T;CDs|-Mi:Ok
%--------------------------------------------------  L={interface? implements T1..Tn M1..Mk} 
%  T;CDs|-L:Ok                                         forall i in 1..n 	CDs(Ti).interface?=interface
%                                                             forall i in 1..n and m in 	dom(CDs(Ti)), m in dom(L)

{\pushLeft\inferrule[(L-OK)]{
    \forall\mM\in\overline\mM,\quad
  \mT;\overline\mD\vdash\mM:\text{OK}
  }{ 
     \mT;\overline\mD \vdash  \oC \Opt{\terminalCode{interface}}\ \terminalCode{implements} ~\overline\mT \ \overline\mM \cC \ :\text{OK}\\
  } 
%\begin{array}{l} 
%  \mL=\oC \Opt{\terminalCode{interface}}\ \terminalCode{implements} \overline\mT \ \overline\mM \cC \\
%  \forall \mT\in\overline\mT \text{and } m \in \dom(\mD(\mT)), \mm \in \dom(\mL)
%   \end{array}
\quad
\inferrule[(Nested-OK)]{
    \mT\terminalCode{.}\mC;\overline\mD\vdash \mL\ :\text{OK}
  }{ 
     \mT;\overline\mD \vdash \mC\terminalCode{=}\mL\ :\text{OK}
  }
}
\rowSpace

%  if e?=e then CDs; G|-e:T                         
%----------------------------------------------------------   forall T in CDs(C).Ts, if m in dom(CDs(Ti)) then
%   T;CDs|-static? T0 m(T1 x1..Tn xn) e?              static? T0 m(T1 x1..Tn xn) in CDs(Ti)
%                                                                        if static?=static then G=x1:T1 .. xn:Tn
%                                                                        else G=this:T,x1:T1 .. xn:Tn

{\pushLeft\inferrule[(Method-OK)]{
    \text{if}\ \Opt\me=\me\ \text{then}\ \overline\mD; \mG\vdash\me:\mT_0
  }{ 
     \mT;\overline\mD \vdash \Opt{\terminalCode{static}}\ \terminalCode{method}\ \mT_0\ \mm\oR\mT_1\,\mx_1\ldots\mT_n\,\mx_n\cR ~\Opt\me\ :\text{OK}
  } \begin{array}{l} 
  \text{if}\ \Opt{\terminalCode{static}}=\terminalCode{static}\\
  \quad \text{then}\ \mG=\mx_1:\mT_1\ .. \ \mx_n:\mT_n\ \\
  \quad\text{else}\ \mG=\terminalCode{this}:\mT,\mx_1:\mT_1\ ..\ \mx_n:\mT_n
  \\
%removed, now is well formedness
%  \forall \mT \in \text{implementsOf}(\overline\mD(\mC)),\ \text{if}\ \mm \in \dom(\overline\mD(\mT))\ \text{then} \\
%  \quad\Opt{\terminalCode{static}}\ \terminalCode{method}\ \mT_0\ \mm\oR\overline{\mT\,\mx}\cR \in \overline\mD(\mT) \\
   \end{array}
}\rowSpace



{\pushLeft\inferrule[(subsumption)]{
%  \begin{array}{l}
    \overline\mD; \mG\vdash\me: \mT_1  \\\\
    \overline\mD\vdash\mT_1 \leq \mT_2
%  \end{array}
  }{ 
     \overline\mD; \mG\vdash\me: \mT_2
  }
\quad \inferrule[(static-method-call)]{
    \overline\mD;\mG\vdash\me_1:\mT_1\ \ldots \ \overline\mD;\mG\vdash\me_n:\mT_n
  }{ 
    \overline\mD;\mG\vdash \mT_0.\mm\oR\me_1\ \ldots \ \me_n\cR:\mT
  } \terminalCode{static method}\ \mT\ \mm\oR\mT_1\,\mx_1\ldots\mT_n\,\mx_n\cR \text{\_} \in \overline\mD\oR\mT_0 \cR
}\rowSpace

%    CDs;G|-e0:T0 .. CDs;G|-en:Tn
%---------------------------------------------    static T m(T1 x1..Tn xn) _ in CDs(T0)
%  CDs;G|-e0.m(e1..en):T

{\pushLeft\inferrule[(x)]{
    \
  }{ 
    \overline\mD; \mG\vdash\mx: \mG\oR\mx\cR
  }
\quad
\inferrule[(method-call)]{
    \overline\mD;\mG\vdash\me_0:\mT_0\ \ldots \ \overline\mD;\mG\vdash\me_n:\mT_n
  }{ 
    \overline\mD;\mG\vdash \me_0.\mm\oR\me_1\ \ldots \ \me_n\cR:\mT
  } \terminalCode{method}\ \mT\ \mm\oR\mT_1\,\mx_1\ldots\mT_n\,\mx_n\cR \text{\_} \in \overline\mD\oR\mT_0 \cR
}
\rowSpace
{\pushLeft\inferrule[(ctxv)]{\me_0\xrightarrow[\smallDs]{}\me_1}{
 \ctx_{\smallDs}[\me_0]\xrightarrow[\smallDs]{} \ctx_{\smallDs}[\me_1]
 }

\quad
\inferrule[(s-m)]{{}_{}}{
 \mT\terminalCode{.}\mm\oR\overline\vds\cR\xrightarrow[\smallDs]{}
 \mathit{meth}(\overline\mD(\mT,\mm),\overline\vds)
}
\quad
\inferrule[(m)]{{}_{}}{
 \vds\terminalCode{.}\mm\oR\overline\vds\cR\xrightarrow[\smallDs]{}
 \mathit{meth}(\overline\mD(\mT,\mm),\vds\,\overline\vds)
}\vds=\mT\terminalCode{.}\mm'\oR\_\cR
}\\
\end{array}
$\\
\caption{Formalization}
\end{figure}
\saveSpace
\subsection{Syntax}
\saveSpace
%In the following section, we present a simplified grammar of \name. 
We use $\mt$ and $\mC$ to represent trait and class identifiers respectively.
A trait ($\mTD$) or a class ($\mCD$) declaration can use either a code literal $\mL$, or a trait
expression $\mE$. Note how in $\mE$\ you can refer to a trait by name.
In full 42, we support various operators including the ones presented before and much more,
 but here we only show the single sum operator: \Q@+@.
This operation is a generalization to the case of nested classes of the simplest and most elegant
trait composition operator~\cite{ducasse2006traits}.
Code literals \mL\ can be marked as interfaces. We use `?' to represent optional terms.
Note that the interface keyword is inside curly brackets,
so an uppercase name associated with an interface literal is a interface class, while a lowercase one is a interface trait.
Then we have a set of implemented interfaces and a set of member
declarations, which can be methods or nested classes.
The members of a code literal are a set, thus their order is immaterial.
If a code literal implements no interfaces, the concrete syntax omits the \Q@implements@ keyword.

Method declarations \mMD~can be instance methods or \Q@static@ methods. 
A static method in \name is similar to a \Q@static@ method in Java, but can be abstract.
This is very useful in the context of code composition.
To denote a method as abstract, instead of an explicit keyword we just omit the implementation \me.

Finally, expressions $\me$ are just variables, instance method calls or static method calls.
Having two different kinds of method calls is an artefact of our simplifications.
In the full 42 language, type names are a kind of expression whose type helps to model metaclasses.
Our values $v_{{\smallDs}}$ are
are just calls to abstract static methods:
thanks to abstract state, we have no \Q@new@ expressions, but just factory calls.
Thus values are parametric on the shape of the specific programs $\overline\mD$.
We then show the evaluation context, the compilation context and full
context.
\saveSpace
\subsection{Well-formedness}
\saveSpace
The whole program ($\overline\mDE$) is well formed if
all the traits and classes at top level have unique names. The special class name
\Q@This@ is not one of those,
and the subtype relations are consistent:
this means that the implementation of interfaces is not circular,
and that $\forall\ \mID\terminalCode{=}\ctx[\mL]\in\overline\mDE, \mathit{consistentSubtype}(\overline\mDE,\terminalCode{This=}\mL;\mL)$.
\quad That is, every literal declares
all the methods declared in its super interfaces.
The full 42 language allows covariant return types as in Java.
Here for simplicity we require them to have the same type declared in the super interface.


\noindent\textbf{Define }$\mathit{consistentSubtype}(\overline\mDE;\mL)$\\
$\begin{array}{l}
\!\!\!\bullet\ \mathit{consistentSubtype}(
  \overline\mDE,
  \oC
  \Opt{\terminalCode{interface}}
  \terminalCode{implements}\overline\mT\ 
  \overline\mM
  \cC
  )\quad\text{where}\\

\quad\quad
\forall \mT\in\overline\mT,\quad\overline\mDE(\mT)=\oC\terminalCode{interface}\,\_\cC
 \text{,\footnotemark}
\\
\quad\quad \forall\ \_\terminalCode{=}\mL\in  \overline\mM, \quad
\mathit{consistentSubtype}(\overline\mDE;\mL) 

\text{ and }
\\
\quad\quad 
\forall \mm, \mT\in\overline\mT,\quad
\text{if}~\terminalCode{method}\ \mT_0\ \mm\oR
\overline{\mT\,\mx}
%\mT'_1\,\mx'_1\ldots\mT'_k\,\mx'_k
\cR\in\overline\mDE(\mT)
\,\text{then}\,
\terminalCode{method}\ \mT_0 \mm\oR
%\mT_1\,\mx_1\ldots\mT_n\,\mx_n
\overline{\mT\,\mx}
\cR~\Opt\me
\in\overline\mM

%\mT_0=\mT'_0, \overline{\mT\,\mx}=\overline{\mT\,\mx}'
%\mT_0\ldots\mT_n=\mT'_0\ldots\mT'_k


\\
\end{array}$
\footnotetext{That is, in this simplified version 
in order to implement an interface nested in a different top level name, such interface can not be generated using a trait expression. This limitation is lifted in the full language.}
${}_{}$\\*
${}_{}$\\*
\noindent A code literal \mL\ is well formed iff:
\begin{itemize}
\item for all methods: parameters have unique names and no parameter is named \Q@this@,
\item all methods in a code literal have unique names,
\item all nested classes in a code literal have unique names, and no nested class is called \Q@This@,
\item all used variables are in scope, and
\item all methods in an interface are abstract, 
and they contain no static methods.
\end{itemize}

\saveSpace
\subsection{Compilation process}
\saveSpace
The compilation process is particularly interesting,
it includes the flattening process and how and when compilation errors may arise.
It is composed by rules \Rulename{top},\ \Rulename{look-up},\ \Rulename{ctx-c} and \Rulename{sum}.
To model more composition operators, they would each need their own rule.

Rule \Rulename{top}
compiles the leftmost top level (trait or class) declaration that needs to be compiled.
First it identifies the subset of the program $\overline\mD$ that can already be typed (second premise).
Then the expression is executed under the control of such compiled program (first premise).
All the traits inside the expression need to
be compiled (rule \Rulename{look-up}): $\forall\mt, \text{if}\, \mE=\ctx[\mt]\,\text{then}\, \mt\in\dom(\overline\mD)$.
If the required $\overline\mD$ cannot be typed, this would cause a compilation error
at this stage.
Rule \Rulename{look-up}
replaces a trait name $\mt$ with the corresponding literal $\mL$.
Since $\overline\mD$ is all well typed, $\mL$ is well typed too.
Rule \Rulename{ctx-c}
uses the compilation context to apply a deterministic left to right call by value\footnote{
In the flattening process, values are code literals $\mL$.} reduction;
thus the leftmost invalid sum that is performed will be the one providing the compilation error.

Keeping in mind the order of members in a literal is immaterial, rule \Rulename{sum}
applies the operator:

\noindent\textbf{Define }$\mL_1+\mL_2, \ \overline{\mM}+\overline{\mM},\ \mM+\mM$\\
$\begin{array}{l}
\!\!\!\bullet\ \mL_1+\mL_2 =\mL_3\quad\text{where}\\
\quad\quad \mL_1= \oC \Opt{\terminalCode{interface}}\ \terminalCode{implements}~ \overline\mT_1\ \overline\mM_1\ \overline\mM_0\cC\\
\quad\quad \mL_2= \oC \Opt{\terminalCode{interface}}\ \terminalCode{implements}~ \overline\mT_2\ \overline\mM_2\ \overline\mM_0'\cC\\
\quad\quad \mL_3= \oC \Opt{\terminalCode{interface}}\ \terminalCode{implements}~ \overline\mT_1,\overline\mT_2\ \overline\mM_1,\overline\mM_2\ (\overline\mM_0+\overline\mM_0')\cC\\
\quad\quad \dom(\overline\mM_1)
%\pitchfork
~\text{disjoint}~
 \dom(\overline\mM_2) \text{ and } \dom(\overline\mM_0)\ =\ \dom(\overline\mM_0')\\

\!\!\!\bullet\ (\mM_1\ldots\mM_n)+(\mM'_1\ldots\mM'_n)\ = \ (\mM_1+\mM'_1)\ldots(\mM_n+\mM'_n)\\

\!\!\!\bullet\ \mM_1+\mM_2=\mM_2+\mM_1\\

\!\!\!\bullet\ \mC\terminalCode{=}\mL_1+\mC\terminalCode{=}\mL_2\ = \ \mC\terminalCode{=}\mL_3\quad \text{if}~ \mL_1+\mL_2=\mL_3\\

\!\!\!\bullet\ \Opt{\terminalCode{static}}\ \terminalCode{method}\ \mT_0\ \mm\oR\overline{\mT\,\mx}\cR \ + \ \Opt{\terminalCode{static}}\ \terminalCode{method}\ \mT_0\ \mm\oR\overline{\mT\,\mx}\cR \Opt\me = \Opt{\terminalCode{static}}\ \terminalCode{method}\ \mT_0\ \mm\oR\overline{\mT\,\mx}\cR \Opt\me\\
\end{array}$

Sum composes the content of the arguments
by taking the union of their members and the union of their \Q@implements@.
Members with the same name are recursively composed.
There are three cases where the composition is impossible.
\begin{itemize}
\item \textit{Method-clash}: two methods with the same name are composed,
but either their headers have different types or they are both implemented.
\item \textit{Class-clash}: a class is composed with an interface.%
\footnote{
The full language relaxes this condition, for example an empty class can be seen as an empty interface during composition.
}
\item \textit{Implements-clash}:
the resulting code would not be well formed.
For example, in the following
\Q@t1+t2@ would result in a class \Q@B@ implementing \Q@A@ with method \Q@a()@,
but \Q@B@ does not have such method.%
\footnote{In \name it could be possible to try to patch class \Q@B@, for example by adding an
abstract method \Q@a()@;  we choose to instead give an error since in the full 42 language
such patch would 
be able to turn coherent private nested classes
into abstract (private) ones.}

\saveSpace\saveSpace
\begin{lstlisting}
t1={  A= {interface method Void a()}  }
t2={  A= {interface}     B= {implements A}  }
\end{lstlisting}\saveSpace\saveSpace

Implements-clash can happen only when composing nested interfaces. Note that while the first two kind of errors are obtained directly by the definition of 
$\mL_1+\mL_2$, Implements-clash is obtained from well-formedness, since injecting the resulting 
$\mL$ in to the program would make it ill-formed by 
$\mathit{consistentSubtype}(\overline\mDE,\mL)$.
\end{itemize}
\saveSpace
\subsection{Typing}
\saveSpace
Typing is composed by rules \Rulename{cd-ok}, \Rulename{td-ok},
\Rulename{l-ok},
\Rulename{nested-ok} and \Rulename{method-ok},
followed by expression typing rules
\Rulename{subsumption}, \Rulename{method-call}, \Rulename{x} and \Rulename{static-method-call}.

Rules \Rulename{cd-ok} and \Rulename{td-ok}
are interesting: a top level class is typed by replacing all occurrences of the name `\Q@This@' with the class name $C$,
and is required to be coherent.
On the other hand, a top level trait is typed by temporarily adding a mapping for
\Q@This@ to the typed program.

\noindent\textbf{Define }$\mathit{coherent}(\mT,\mL)$\\
$\begin{array}{l}
\!\!\!\bullet\ \mathit{coherent}(\mT,
\oC \Opt{\terminalCode{interface}}\ \terminalCode{implements} \overline\mT\ \overline\mM\cC
)\quad\text{holds where}\\

\quad\quad \forall \mC\terminalCode{=}\mL'\in\overline\mM \mathit{coherent}(\mT\terminalCode{.}\mC,\mL')\\
\quad\quad \text{and either }
\Opt{\terminalCode{interface}}=\terminalCode{interface}\\
\quad\quad\quad \text{or } 
\forall\ 
\terminalCode{method}\ \mT'\ \mm\oR\overline{\mT\,\mx}\cR \in\overline\mM,\ 
\text{state}(\text{factory}(\mT,\overline\mM), ~\terminalCode{method}\ \mT'\ \mm\oR\overline{\mT\,\mx}\cR)
\end{array}$

\noindent A Literal is \emph{coherent} if 
all the nested classes are coherent,
and either the Literal is an interface, there are no static methods, or all the static methods
are a valid \emph{state} method of the candidate \emph{factory}.
Note, by asking for
$\terminalCode{method}\ \mT'\ \mm\oR\overline{\mT\,\mx}\cR \in\overline\mM$
we select only abstract methods.

\noindent\textbf{Define }$\text{factory}(\mT,\overline\mM)$\\
$\begin{array}{l}

\!\!\!\bullet\ \text{factory}(\mT,\mM_1\ldots\mM_n)=\mM_i=\terminalCode{static method}\ \mT\, \mm
\oR
\_
\cR

\quad\text{where}\\
\quad\quad \forall j\neq i.\quad \mM_j 
\text{is not of the form}\ \terminalCode{static method}\ \_\, \_
\oR
\_
\cR
\end{array}$

\noindent The factory is the only static abstract  method, and
its return type is the nominal type of our class.

\noindent\textbf{Define }$\text{state}(\mM,\mM')$\\
$\begin{array}{l}


\!\!\!\bullet\ \text{state}(
\terminalCode{static}\ \terminalCode{method}\ \mT\ \mm\oR\mT_1\,\mx_1\ldots\mT_n\,\mx_n\cR,
~\terminalCode{method}\ \mT_i\ \mx_i\oR\cR
)\\

%\!\!\!\bullet\ \text{state}(
%\terminalCode{static}\ \terminalCode{method}\ \mT\ \mm\oR\mT_1\,\mx_1\ldots\mT_n\,\mx_n\cR,
%\terminalCode{method}\ \terminalCode{Void} \mx_i\oR\mT_i\,\terminalCode{that}\cR
%)\\

\!\!\!\bullet\ \text{state}(
\terminalCode{static}\ \terminalCode{method}\ \mT\ \mm\oR\mT_1\,\mx_1\ldots\mT_n\,\mx_n\cR,
~\terminalCode{method}\ \mT\ \terminalCode{with}\mx_i\oR\mT_i\,\terminalCode{that}\cR
)\\

\end{array}$

\noindent A non static method is part of the \emph{abstract state} if 
it is a valid getter or wither. In this simple formalism without imperative features we do not offer setters.


Rule \Rulename{Nested-OK} helps to accumulate the type of \Q@this@ so that rule \Rulename{Method-OK}
can use it.
Rule \Rulename{L-OK} is so simple since all the checks
related to correctly implementing interfaces are delegated to the well formedness criteria.
The expression typing rules are straightforward and standard.

\saveSpace
\subsection{Formal properties}
\saveSpace
In addition to conventional soundness of expression reduction,
\name ensures soundness of the compilation process itself.
A similar property was called meta-level-soundness in~\cite{servetto2014meta}; here we can obtain the same result in
a much simpler setting.
We denote $\mathit{wrong}(\overline\mD,\mE)$ to be the number of $\mL$s such that
$\mE=\ctx[\mL]\ \text{and not}\ \overline\mD\vdash\mL:\text{OK}$.

\newtheorem{theorem}{Theorem}[section]

\begin{theorem}[Compilation Soundness]

if $\mE_0 \xrightarrow[\smallDs]{} \mE_1$
then $\mathit{wrong}(\overline\mD,\mE_0)\geq\mathit{wrong}(\overline\mD,\mE_1)$.
\end{theorem}
\saveSpace\saveSpace\saveSpace
\noindent This can be proved by cases on the applied reduction rule:
\begin{itemize}
\item
\Rulename{look-up} preserves the number of wrong literals:
$t \in \overline\mD$ and $\overline\mD$ is well typed by \Rulename{top} preconditions.
\item \Rulename{sum} either preserves or reduces the number of
wrong literals:
the core of the proof is to show that the sum of two well typed literals produces a well typed one.
A code literal is well typed (\Rulename{l-ok}) if all its method bodies are correct.
This holds since those same method bodies
are well typed in a strictly weaker environment with respect to the one used to type the result.
This is because every member in the result of the sum
is structurally a subtype of
the corresponding members in the operands.
Note that by well formedness, if \Rulename{sum}
is applied the result still respect 
$\mathit{consistentSubtype}$.
\end{itemize}
\newpage
\noindent 
\textbf{Compilation Soundness has two important corollaries:}
\begin{itemize}
\item A class declared without literals
is well-typed after flattening; no need of further checking.
\item If a class is declared by using literals $\mL_1\ldots\mL_n$, and after successful flattening $\mC = \mL$ can not be type-checked,
then the issue was originally present in one of $\mL_1\ldots\mL_n$.
This also means that as an optimization strategy
 we may remember what method bodies come from traits and what method bodies come from code literals, and only type-check the latter.
If the result can not be type-checked, either it is intrinsically ill-typed or a 
referred type is declared \emph{after} the current class. 
As we see in the next section, we leverage on this 
to allow recursive types.
 \end{itemize}





\saveSpace
\subsection{Advantages of our compilation process}
\saveSpace

Our typing discipline is very simple from a formal perspective,  
and is what distinguishes our approach from a simple minded code composition macros~\cite{bawden1999quasiquotation}
or rigid module composition~\cite{ancona2002calculus}. 
It is built on two core ideas:

\paragraph{1: Traits are \emph{well-typed} before being reused.}
 For example:

\saveSpace\saveSpace\begin{lstlisting}
t={method int m(){return 2;} 
   method int n(){return this.m()+1;}}
\end{lstlisting}\saveSpace\saveSpace

\noindent \Q@t@ is well typed since \Q@m()@ is declared inside of \Q@t@, while the following would be ill-typed:

\saveSpace\saveSpace\begin{lstlisting}
t1={method int n(){return this.m()+1;}} //ill-typed
\end{lstlisting}\saveSpace\saveSpace


\paragraph{2: Code literals are \emph{not} required to be well-typed before flattening.}${}_{}$\\*
A literal $\mL$ in a declaration $\mD$
must be well formed and respect
$\mathit{consistentSubtype}$, but
it is not type-checked until flattening is complete:
only the result is required to be well-typed.
For example the following is correct since
the result of the flattening is well-typed:

\saveSpace\saveSpace\begin{lstlisting}
C= Use t, {method int k(){return this.n()+this.m();}}//correct code
\end{lstlisting}\saveSpace\saveSpace

\noindent The code literal
\Q@{method int k(){ return this.n()+this.m();}}@
is not well typed: \Q@n@, \Q@m@ are not locally defined.
This code would fail in many similar works in literature~\cite{deep,Bettini2015282,Bergel2007} where the
literals have to be \emph{self contained}. In this case we would have been forced to
declare abstract methods \Q@n@ and \Q@m@, even if \Q@t@ already 
provides such methods.

This relaxation allows multiple declarations to be flattened one at the time, without typing them individually, and only typing them all together.
In this way, we support recursive types%
\footnote{
OO languages leverage on recursive types most of the times:
for example \Q@String@ may offer a \Q@Int size()@
method, and \Q@Int@ may offer a \Q@String toString()@ method.
This means that typing classes 
\Q@String@ and \Q@Int@ in isolation one at a time is not possible.}
between multiple class declarations without
the need of predicting the resulting shape%
\footnote{This is needed in full 42: it is
impossible to predict the resulting shape since
arbitrary code can run at compile time.}.

As seen in \Rulename{top}, our compilation process
proceeds in a top-down fashion, flattening one declaration at a time,
a declaration needs
to be type-checked where their type is first needed,
that is, when they are required to type a trait used in a code expression.
That is, in \name typing and flattening are interleaved. We assume our compilation process stops as soon as 
an error arises. 
For example:
\saveSpace\saveSpace\begin{lstlisting}
ta={method int ma(){return 2;}}
tc={method int mc(A a, B b){return b.mb(a);}}
A= Use ta
B= {method int mb(A a){return a.ma()+1;}}
C= Use tc, {method int hello(){return 1;}}
\end{lstlisting}\saveSpace\saveSpace
In this scenario, since we compile top down, we first need to generate \Q@A@.
To generate \Q@A@, we need to use \Q@ta@ (but we do not need
\Q@tc@, in rule \Rulename{top}, $\overline\mD=\,$\Q@ta@ and $\overline\mD'=\,$\Q@tc@).
At this moment, \Q@tc@ cannot be compiled/checked alone:
information about \Q@A@ and \Q@B@ is needed.
To modularly ensure well-typedness,
we only require \Q@ta@ to be well typed at this stage; if it is not a type-error will be raised immediately. % This means that if \Q@ta@ was not well-typed%there would be a type error at this stage.
Now, we need to generate \Q@C@, and hence type-check \Q@tc@.
\Q@A@ is guaranteed to be already type-checked 
(since it is generated by an expression that does not contain any \mL),
and \Q@B@ can be typed. Finally \Q@tc@ can be typed and reused.
If the \Rulename{sum} rule could not be performed (for example if \Q@tc@ had a method \Q@hello@ too)
a composition error would be generated at this stage.
On the other hand, if \Q@B@ and \Q@C@ were swapped, as in:
\saveSpace\saveSpace\begin{lstlisting}
C= Use tc, {method int hello(){return 1;}}
B= {method int mb(A a){return a.ma()+1;}}
\end{lstlisting}\saveSpace\saveSpace
\noindent
we would be unable to type \Q@tc@, since we need to know the structure of \Q@A@ and \Q@B@.
A type error would be generated.%, on the lines of ``flattening of \Q@C@
%requires \Q@tc@, \Q@tc@ requires \Q@B@ that is defined later''.

%In this example, a more expressive compilation/precompilation process 
%could compute a dependency graph and, if possible, reorganize the list,

\paragraph{The cost: what expressive power we lose}${}_{}$\\*
We require declarations to be provided in the right dependency order, but sometimes no such order exists.
An example of a ``morally correct'' program where no right order exists is the following:
\saveSpace\saveSpace\begin{lstlisting}
t= { int mt(A a){return a.ma();}}
A= Use t, {int ma(){return 1;}}
\end{lstlisting}\saveSpace\saveSpace
Here the correctness of \Q@t@ depends on 
\Q@A@, that is in turn generated using \Q@t@.
We believe any typing allowing such programs would be fragile with respect to code evolution,
and could make human understanding of the code-reuse process much harder/involved.
%
%Rewriting our example in Java may help to show how involved it is.
%\saveSpace\begin{lstlisting}
%class T{ int mt(A a){return a.ma();}
%class A extends T {int ma() {return 1;}}
%\end{lstlisting}\saveSpace
%
In sharp contrast with others (TR, PT, DJ, but also Java, C\#, and Scala)
we chose to not support this kind of involved programs.

TR, PT, DJ, Java, C\#, and Scala
accept a great deal of complexity in order to predict the structural shape of the resulting code before doing the actual code reuse/adaptation.
Those approaches logically divide the program in groups of mutually dependent classes, where each group may depend on a number of other groups.
This forms a direct acyclic graph of groups.
To type a group, all depended groups are typed, then
the signature/structural shape of all
the classes of the group are extracted.
Finally, with the information of the dependent groups and the current group, it is possible to type-check the implementation of each class in the group.

%Following this model, it is reasonable to assume that flattening happens group by group, before extracting the class signatures.



%\paragraph{In \name, typechecking before compiling would be redundant}${}_{}$\\*
%In the world of strongly typed languages we are tempted to
%first check that all can go well, and then perform the flattening.
%This would however be overcomplicated without any observable difference:
%Indeed, in the \Q@A,B,C@ example above there is no difference
%between
%\begin{itemize}
%\item  (1) First check \Q@B@ and produce \Q@B@ code (that also contains \Q@B@ structural shape),
%  (2) then use \Q@B@ shape to check \Q@C@ and produce \Q@C@ code;\ 
%or a more involved
%\item  (1) First check \Q@B@ and discover just \Q@B@ structural shape as result of the checking,
%  (2) then use \Q@B@ shape to check \Q@C@.
%  (3) Finally produce both \Q@B@ and \Q@C@ code.
%\end{itemize}
%
%
%This may seems a dangerous relaxation at first, but also Java has the same behaviour:
%\saveSpace\begin{lstlisting}[language=Java]
%  class A{ int ma() {return 2;}  int n(){return this.ma()+1;} }
%  class B extends A{ int mb(){return this.ma();} }
%\end{lstlisting}\saveSpace
%\noindent in \Q@B@ we can call \lstinline{this.ma()} even if in the curly braces there is no declaration for \Q@ma()@.
%
%



\noindent 
In the world of strongly typed languages we are tempted to
first check that all will go well, and then perform the flattening. 
Such methodology would be redundant in our setting: we can only reuse code through trait names; but our point of relaxation is only the code literal: in no way can an error ``move around'' and be duplicated during the compilation process.
That is, our approach allows safe libraries of traits and classes to be typechecked once, and then deployed and reused by multiple clients: as Theorem A.1 states, in \name no type error will emerge from library code.
%However, we do not force the programmer to write self-contained code where all the abstract method definition are explicitly declared.


\saveSpace
\subsection{Expression reduction}
\saveSpace
Our reduction rules are incredibly simple and standard.
A great advantage of our compilation model is that expressions are executed on
a simple fully flattened program, 
where all the composition operators have been removed.
From the point of view of expression reduction, \name is a simple language of 
interfaces and final classes, where nested classes give structure to code but have no special semantics.
The reduction of expressions is defined by rules
\Rulename{ctx-v}, \Rulename{s-m}, and \Rulename{m}.
The only interesting point is the auxiliary function $\mathit{meth}$:


\noindent\textbf{Define }$\mathit{meth}(\mM,\overline\vds)$

$\begin{array}{l}

\!\!\!\bullet\,\mathit{meth}(\terminalCode{static method}\ \mT\ \mm\oR\mT_1\, \mx_1\ldots\mT_n\,\mx_n\cR\,\me,\vds_1\ldots\vds_n)=\me[\mx_1=\vds_1,\ldots,\me_n=\vds_n]
\\

\!\!\!\bullet\,\mathit{meth}(\terminalCode{method}\ \mT\ \mm\oR\mT_1\, \mx_1\ldots\mT_n\,\mx_n\cR\,\me,\vds_0,\ldots,\vds_n)=\me[\terminalCode{this}=\vds_0,\mx_1=\vds_1,\ldots,\me_n=\vds_n]
\\

\!\!\!\bullet\,\mathit{meth}(\terminalCode{method}\ \mT_i\ \mx_i\oR\cR,\,\mT\terminalCode{.}\mm\oR\vds_1\ldots\vds_n\cR)=\vds_i\\
\quad \quad\text{where}\ \ \overline\mD(\mT,\mm) =
\terminalCode{static method}
\ \mT\,\mm\oR\mT_1\,\mx_1\ldots\mT_n\,\mx_n\cR
\\

\!\!\!\bullet\,\mathit{meth}(\terminalCode{method}\ \mT\ \terminalCode{with}\mx_i\oR\mT_i\,\terminalCode{that}\cR,\mT\terminalCode{.}\mm\oR\vds_1\ldots\vds_n\cR\,
\vds
)=
\mT\terminalCode{.}\mm\oR
\vds_1\ldots\vds_{i-1},
\vds,
\vds_{i+1}\ldots\vds_n
\cR
\\
\quad \quad\text{where}\ \ \overline\mD(\mT,\mm) =
\terminalCode{static method}
\ \mT\,\mm\oR\mT_1\,\mx_1\ldots\mT_n\,\mx_n\cR
\end{array}$

\noindent 
Here we take care of reading method bodies and preparing for
execution.
The first case is for static methods
and the second is for instance methods.
The third and fourth cases are more interesting, since they take care of
the abstract state:
the third case reduce getters and the fourth reduces withers.
In our formalisation we are not modelling state mutation, so there is 
no case for setters.

%For space reasons,
We omit the proof of conventional soundness for the
reduction. It is unsurprising, since the flattened calculus is like a
simplified version of Featherweight Java~\cite{igarashi2001featherweight}.
