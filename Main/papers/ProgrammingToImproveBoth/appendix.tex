\newpage\appendix

\saveSpace\section{Formalisation}\label{sec:formal}
\saveSpace

Here we show a simple formalization for the language we presented so far.
We also model nested classes, but in order to avoid uninteresting complexities, we assume that
all type names are fully qualified from top level, so the examples shown before should be
written like: \Q@This.Exp@, \Q@This.Sum@, etcetera.
In a real language, a simple pre-processor may take care of this step.

In most languages, when implementing an interface, the programmer may avoid repeating abstract methods
they do not wish to implement, however
to simplify our formalization, we consider source code always containing all the methods imported from interfaces. In a real language, a normalisation process
may hide this abstraction\footnote{
In the full 42 language scoping is indeed supported by an initial de-sugaring, and a normalisation phase takes care of importing methods from interfaces.
}.
We also consider a binary operator sum (\Q@+@) instead of the nary operator \Q@Use@.
Figure 1 contains the complete formalization for \name: syntax,
compilation process, typing, and finally reduction.




\begin{figure}
%NEW FORMALISATION below
% Syntax
% D::=TD|CD
% TE::=t:E Trait Decl Expr
% CE::=C:E Class Decl
% TD::=t:L
% CD::=C:L
% E::= L| t| E+E | E[rename T.m1->m2]|E[rename T1->T2]|E[redirect T1->T2]
% L::= {interface? implements Ts Ms}//all L are like LC in 42
% T::=C|C.T // .T is a shortcut for This.T
% M::= static? method T m(T1 x1..Tn xn) e? | CD
% e::= x| e.m(es) | T.m(es)

\begin{bnf}
\prodFull\mID{\mt\mid\mC}{class or trait name}\\
\prodFull\mDE{\mID~\terminalCode{=}~\mE}{Meta-declaration}\\
\prodFull\mD{\mID~\terminalCode{=}~\mL}{Declaration}\\
\prodFull\mE{\mL \mid \mt \mid \mE\,\terminalCode{+}\mE
\mid \ldots
%\mid \mE\terminalCode{[rename}\ \mT\terminalCode{.}\mm_1\ \terminalCode{to}\ \mm_2\terminalCode{]}
}{Code Expression}\\
%\prodNextLine{
% \mid
%\mE\terminalCode{[rename}\ \mT_1\ \terminalCode{in}\ \mT_2\terminalCode{]} \mid
%\mE\terminalCode{[redirect}\ \mT_1\ \terminalCode{to}\ \mT_2\terminalCode{]}}{Code Expression}\\
\prodFull\mL{
\oC \Opt{\terminalCode{interface}}\ \terminalCode{implements} \overline~\mT\ \overline\mM\ \cC}{Code Literal}\\
\prodFull\mT{\mC \mid \mC\terminalCode{.}\mT}{Type}\\
\prodFull\mM{\Opt{\terminalCode{static}}\ \terminalCode{method}\ \mT\ \mm\oR\overline{\mT\,\mx}\cR ~\Opt\me \mid \mC~\terminalCode{=}~\mL}{Member}\\

\prodFull\me{\mx \mid \me\terminalCode{.}\mm\oR\overline\me\cR \mid \mT\!\terminalCode{.}\mm\oR\overline\me\cR}{Expression}\\

\prodFull{v_{{\smallDs}}}{\mT\terminalCode{.}\mm\oR\overline{v_{{\smallDs}}}\cR
\text{,  where }\mm \text{ is abstract in }\overline\mD(\mT)
}{value}\\

\prodFull{\ctx_{\smallDs}}{[]\mid
\ctx_\smallDs\terminalCode{.}\mm\oR\overline\me\cR
\mid
v_{{\smallDs}}\terminalCode{.}\mm\oR\overline{v_{{\smallDs}}},\ctx_\smallDs,\overline\me\cR
\mid
\mT\terminalCode{.}\mm\oR\overline{v_{{\smallDs}}},\ctx_\smallDs,\overline\me\cR
}{evaluation context}\\

\prodFull{\ctx_c}{[]\mid\ctx_c\,\terminalCode+\mE \mid \mL\,\terminalCode+\ctx_c \mid\ldots}{compilation context}\\

\prodFull{\ctx}{[]\mid\ctx\,\terminalCode+\mE \mid \mE\,\terminalCode+\ctx \mid\ldots}{ctx}\\

\prodFull\mG{\mx_1{:}\mT_1,\ldots,\mx_n{:}\mT_n}{variable environment}
\end{bnf}\\

\newcommand{\pushLeft}{\!\!\!\!\!\!\!\!\!\!\!\!}
\newcommand{\rowSpace}{\\\vspace{2.5ex}}
$\begin{array}{l}

%       D.E -->^+_CDs L  CDs|-CD1:OK .. CDs|-CDn:OK       CDs=CD1..CDn
% (top)---------------------------------------------------------------    D.E not of form L
%      CD1..CDn CDs' D Ds -> CDs CDs' D[with E=L] Ds

\rowSpace
{\pushLeft\inferrule[(top)]{
  \mE_0 \xrightarrow[\smallDs]{} \mE_1
  \\
  \forall \mD\in\overline\mD,
  \overline\mD\vdash\mD:\text{OK}
  }{ 
    \overline\mD \ \overline{\mD'}\ \mID\terminalCode{=}\mE_0 \ \overline{\mDE}
    \rightarrow 
    \overline\mD\ \overline{\mD'}\ \mID\terminalCode{=}\mE_1\ \overline{\mDE}
  } %{\overline\mD=\mD_1..\mD_n }
\quad\quad
%
%     ------------------------
%      t -->_CDs CDs(t)

 \inferrule[(look-up)]{
    \ 
  }{ 
    \mt \xrightarrow[\smallDs]{}\ \overline\mD(\mt)
  }
\quad

\inferrule[(ctx-c)]{
    \mE_0 \xrightarrow[\smallDs]{}\ \mE_1
  }{ 
     {\ctx}_c[\mE_0] \xrightarrow[\smallDs]{}\ {\ctx}_c[\mE_1]
  }
\quad
%
%      --------------------------      L = L1+L2
%      L1+L2  -->_CDs L

\inferrule[(sum)]{
    \
  }{ 
     \mL_1\,\terminalCode{+}\mL_2 \xrightarrow[\smallDs]{}\ \mL
  }\mL = \mL_1+\mL_2
}\rowSpace

%  C;CDs,C=L |- L[This=C] :OK
% ----------------------------------------- coherent(L)
%  CDs|-C=L : OK

{\pushLeft\inferrule[(CD-OK)]{
    \mC;\overline\mD,\mC\terminalCode{=}\mL_1\vdash \mL_1\ :\text{OK}
  }{ 
     \overline\mD \vdash \mC\terminalCode{=}\mL_0\ :\text{OK}
  }
\begin{array}{l}
\mL_1=\mL_0[\terminalCode{This}=\mC]\\
\mathit{coherent}(\mC,\mL_1)
\end{array}
\quad\quad 

%    This;CDs,This=L |- L :OK
%----------------------------------------
%    CDs|-t=L : OK

\inferrule[(TD-OK)]{
    \terminalCode{This};\overline\mD,\terminalCode{This=}\mL\vdash \mL\ :\text{OK}
  }{ 
     \overline\mD \vdash \mt\terminalCode{=}\mL\ :\text{OK}
  }
}\rowSpace

%  forall i in 1..k T;CDs|-Mi:Ok
%--------------------------------------------------  L={interface? implements T1..Tn M1..Mk} 
%  T;CDs|-L:Ok                                         forall i in 1..n 	CDs(Ti).interface?=interface
%                                                             forall i in 1..n and m in 	dom(CDs(Ti)), m in dom(L)

{\pushLeft\inferrule[(L-OK)]{
    \forall\mM\in\overline\mM,\quad
  \mT;\overline\mD\vdash\mM:\text{OK}
  }{ 
     \mT;\overline\mD \vdash  \oC \Opt{\terminalCode{interface}}\ \terminalCode{implements} ~\overline\mT \ \overline\mM \cC \ :\text{OK}\\
  } 
%\begin{array}{l} 
%  \mL=\oC \Opt{\terminalCode{interface}}\ \terminalCode{implements} \overline\mT \ \overline\mM \cC \\
%  \forall \mT\in\overline\mT \text{and } m \in \dom(\mD(\mT)), \mm \in \dom(\mL)
%   \end{array}
\quad
\inferrule[(Nested-OK)]{
    \mT\terminalCode{.}\mC;\overline\mD\vdash \mL\ :\text{OK}
  }{ 
     \mT;\overline\mD \vdash \mC\terminalCode{=}\mL\ :\text{OK}
  }
}
\rowSpace

%  if e?=e then CDs; G|-e:T                         
%----------------------------------------------------------   forall T in CDs(C).Ts, if m in dom(CDs(Ti)) then
%   T;CDs|-static? T0 m(T1 x1..Tn xn) e?              static? T0 m(T1 x1..Tn xn) in CDs(Ti)
%                                                                        if static?=static then G=x1:T1 .. xn:Tn
%                                                                        else G=this:T,x1:T1 .. xn:Tn

{\pushLeft\inferrule[(Method-OK)]{
    \text{if}\ \Opt\me=\me\ \text{then}\ \overline\mD; \mG\vdash\me:\mT_0
  }{ 
     \mT;\overline\mD \vdash \Opt{\terminalCode{static}}\ \terminalCode{method}\ \mT_0\ \mm\oR\mT_1\,\mx_1\ldots\mT_n\,\mx_n\cR ~\Opt\me\ :\text{OK}
  } \begin{array}{l} 
  \text{if}\ \Opt{\terminalCode{static}}=\terminalCode{static}\\
  \quad \text{then}\ \mG=\mx_1:\mT_1\ .. \ \mx_n:\mT_n\ \\
  \quad\text{else}\ \mG=\terminalCode{this}:\mT,\mx_1:\mT_1\ ..\ \mx_n:\mT_n
  \\
%removed, now is well formedness
%  \forall \mT \in \text{implementsOf}(\overline\mD(\mC)),\ \text{if}\ \mm \in \dom(\overline\mD(\mT))\ \text{then} \\
%  \quad\Opt{\terminalCode{static}}\ \terminalCode{method}\ \mT_0\ \mm\oR\overline{\mT\,\mx}\cR \in \overline\mD(\mT) \\
   \end{array}
}\rowSpace



{\pushLeft\inferrule[(subsumption)]{
%  \begin{array}{l}
    \overline\mD; \mG\vdash\me: \mT_1  \\\\
    \overline\mD\vdash\mT_1 \leq \mT_2
%  \end{array}
  }{ 
     \overline\mD; \mG\vdash\me: \mT_2
  }
\quad \inferrule[(static-method-call)]{
    \overline\mD;\mG\vdash\me_1:\mT_1\ \ldots \ \overline\mD;\mG\vdash\me_n:\mT_n
  }{ 
    \overline\mD;\mG\vdash \mT_0.\mm\oR\me_1\ \ldots \ \me_n\cR:\mT
  } \terminalCode{static method}\ \mT\ \mm\oR\mT_1\,\mx_1\ldots\mT_n\,\mx_n\cR \text{\_} \in \overline\mD\oR\mT_0 \cR
}\rowSpace

%    CDs;G|-e0:T0 .. CDs;G|-en:Tn
%---------------------------------------------    static T m(T1 x1..Tn xn) _ in CDs(T0)
%  CDs;G|-e0.m(e1..en):T

{\pushLeft\inferrule[(x)]{
    \
  }{ 
    \overline\mD; \mG\vdash\mx: \mG\oR\mx\cR
  }
\quad
\inferrule[(method-call)]{
    \overline\mD;\mG\vdash\me_0:\mT_0\ \ldots \ \overline\mD;\mG\vdash\me_n:\mT_n
  }{ 
    \overline\mD;\mG\vdash \me_0.\mm\oR\me_1\ \ldots \ \me_n\cR:\mT
  } \terminalCode{method}\ \mT\ \mm\oR\mT_1\,\mx_1\ldots\mT_n\,\mx_n\cR \text{\_} \in \overline\mD\oR\mT_0 \cR
}
\rowSpace
{\pushLeft\inferrule[(ctxv)]{\me_0\xrightarrow[\smallDs]{}\me_1}{
 \ctx_{\smallDs}[\me_0]\xrightarrow[\smallDs]{} \ctx_{\smallDs}[\me_1]
 }

\quad
\inferrule[(s-m)]{{}_{}}{
 \mT\terminalCode{.}\mm\oR\overline\vds\cR\xrightarrow[\smallDs]{}
 \mathit{meth}(\overline\mD(\mT,\mm),\overline\vds)
}
\quad
\inferrule[(m)]{{}_{}}{
 \vds\terminalCode{.}\mm\oR\overline\vds\cR\xrightarrow[\smallDs]{}
 \mathit{meth}(\overline\mD(\mT,\mm),\vds\,\overline\vds)
}\vds=\mT\terminalCode{.}\mm'\oR\_\cR
}\\
\end{array}
$\\
\caption{Formalization}
\end{figure}
\saveSpace
\subsection{Syntax}
\saveSpace
%In the following section, we present a simplified grammar of \name. 
We use $\mt$ and $\mC$ to represent trait and class identifiers respectively.
A trait ($\mTD$) or a class ($\mCD$) declaration can use either a code literal $\mL$, or a trait
expression $\mE$. Note how in $\mE$\ you can refer to a trait by name.
In full 42, we support various operators including the ones presented before and much more,
 but here we only show the single sum operator: \Q@+@.
This operation is a generalization to the case of nested classes of the simplest and most elegant
trait composition operator~\cite{ducasse2006traits}.
Code literals \mL\ can be marked as interfaces. We use `?' to represent optional terms.
Note that the interface keyword is inside curly brackets,
so an uppercase name associated with an interface literal is a interface class, while a lowercase one is a interface trait.
Then we have a set of implemented interfaces and a set of member
declarations, which can be methods or nested classes.
The members of a code literal are a set, thus their order is immaterial.
If a code literal implements no interfaces, the concrete syntax omits the \Q@implements@ keyword.

Method declarations \mMD~can be instance methods or \Q@static@ methods. 
A static method in \name is similar to a \Q@static@ method in Java, but can be abstract.
This is very useful in the context of code composition.
To denote a method as abstract, instead of an explicit keyword we just omit the implementation \me.

Finally, expressions $\me$ are just variables, instance method calls or static method calls.
Having two different kinds of method calls is an artefact of our simplifications.
In the full 42 language, type names are a kind of expression whose type helps to model metaclasses.
Our values $v_{{\smallDs}}$ are
are just calls to abstract static methods:
thanks to abstract state, we have no \Q@new@ expressions, but just factory calls.
Thus values are parametric on the shape of the specific programs $\overline\mD$.
We then show the evaluation context, the compilation context and full
context.
\saveSpace
\subsection{Well-formedness}
\saveSpace
The whole program ($\overline\mDE$) is well formed if
all the traits and classes at top level have unique names. The special class name
\Q@This@ is not one of those,
and the subtype relations are consistent:
this means that the implementation of interfaces is not circular,
and that $\forall\ \mID\terminalCode{=}\ctx[\mL]\in\overline\mDE, \mathit{consistentSubtype}(\overline\mDE,\terminalCode{This=}\mL;\mL)$.
\quad That is, every literal declares
all the methods declared in its super interfaces.
The full 42 language allows covariant return types as in Java.
Here for simplicity we require them to have the same type declared in the super interface.


\noindent\textbf{Define }$\mathit{consistentSubtype}(\overline\mDE;\mL)$\\
$\begin{array}{l}
\!\!\!\bullet\ \mathit{consistentSubtype}(
  \overline\mDE,
  \oC
  \Opt{\terminalCode{interface}}
  \terminalCode{implements}\overline\mT\ 
  \overline\mM
  \cC
  )\quad\text{where}\\

\quad\quad
\forall \mT\in\overline\mT,\quad\overline\mDE(\mT)=\oC\terminalCode{interface}\,\_\cC
 \text{,\footnotemark}
\\
\quad\quad \forall\ \_\terminalCode{=}\mL\in  \overline\mM, \quad
\mathit{consistentSubtype}(\overline\mDE;\mL) 

\text{ and }
\\
\quad\quad 
\forall \mm, \mT\in\overline\mT,\quad
\text{if}~\terminalCode{method}\ \mT_0\ \mm\oR
\overline{\mT\,\mx}
%\mT'_1\,\mx'_1\ldots\mT'_k\,\mx'_k
\cR\in\overline\mDE(\mT)
\,\text{then}\,
\terminalCode{method}\ \mT_0 \mm\oR
%\mT_1\,\mx_1\ldots\mT_n\,\mx_n
\overline{\mT\,\mx}
\cR~\Opt\me
\in\overline\mM

%\mT_0=\mT'_0, \overline{\mT\,\mx}=\overline{\mT\,\mx}'
%\mT_0\ldots\mT_n=\mT'_0\ldots\mT'_k


\\
\end{array}$
\footnotetext{That is, in this simplified version 
in order to implement an interface nested in a different top level name, such interface can not be generated using a trait expression. This limitation is lifted in the full language.}
${}_{}$\\*
${}_{}$\\*
\noindent A code literal \mL\ is well formed iff:
\begin{itemize}
\item for all methods: parameters have unique names and no parameter is named \Q@this@,
\item all methods in a code literal have unique names,
\item all nested classes in a code literal have unique names, and no nested class is called \Q@This@,
\item all used variables are in scope, and
\item all methods in an interface are abstract, 
and they contain no static methods.
\end{itemize}

\saveSpace
\subsection{Compilation process}
\saveSpace
The compilation process is particularly interesting,
it includes the flattening process and how and when compilation errors may arise.
It is composed by rules \Rulename{top},\ \Rulename{look-up},\ \Rulename{ctx-c} and \Rulename{sum}.
To model more composition operators, they would each need their own rule.

Rule \Rulename{top}
compiles the leftmost top level (trait or class) declaration that needs to be compiled.
First it identifies the subset of the program $\overline\mD$ that can already be typed (second premise).
Then the expression is executed under the control of such compiled program (first premise).
All the traits inside the expression need to
be compiled (rule \Rulename{look-up}): $\forall\mt, \text{if}\, \mE=\ctx[\mt]\,\text{then}\, \mt\in\dom(\overline\mD)$.
If the required $\overline\mD$ cannot be typed, this would cause a compilation error
at this stage.
Rule \Rulename{look-up}
replaces a trait name $\mt$ with the corresponding literal $\mL$.
Since $\overline\mD$ is all well typed, $\mL$ is well typed too.
Rule \Rulename{ctx-c}
uses the compilation context to apply a deterministic left to right call by value\footnote{
In the flattening process, values are code literals $\mL$.} reduction;
thus the leftmost invalid sum that is performed will be the one providing the compilation error.

Keeping in mind the order of members in a literal is immaterial, rule \Rulename{sum}
applies the operator:

\noindent\textbf{Define }$\mL_1+\mL_2, \ \overline{\mM}+\overline{\mM},\ \mM+\mM$\\
$\begin{array}{l}
\!\!\!\bullet\ \mL_1+\mL_2 =\mL_3\quad\text{where}\\
\quad\quad \mL_1= \oC \Opt{\terminalCode{interface}}\ \terminalCode{implements}~ \overline\mT_1\ \overline\mM_1\ \overline\mM_0\cC\\
\quad\quad \mL_2= \oC \Opt{\terminalCode{interface}}\ \terminalCode{implements}~ \overline\mT_2\ \overline\mM_2\ \overline\mM_0'\cC\\
\quad\quad \mL_3= \oC \Opt{\terminalCode{interface}}\ \terminalCode{implements}~ \overline\mT_1,\overline\mT_2\ \overline\mM_1,\overline\mM_2\ (\overline\mM_0+\overline\mM_0')\cC\\
\quad\quad \dom(\overline\mM_1)
%\pitchfork
~\text{disjoint}~
 \dom(\overline\mM_2) \text{ and } \dom(\overline\mM_0)\ =\ \dom(\overline\mM_0')\\

\!\!\!\bullet\ (\mM_1\ldots\mM_n)+(\mM'_1\ldots\mM'_n)\ = \ (\mM_1+\mM'_1)\ldots(\mM_n+\mM'_n)\\

\!\!\!\bullet\ \mM_1+\mM_2=\mM_2+\mM_1\\

\!\!\!\bullet\ \mC\terminalCode{=}\mL_1+\mC\terminalCode{=}\mL_2\ = \ \mC\terminalCode{=}\mL_3\quad \text{if}~ \mL_1+\mL_2=\mL_3\\

\!\!\!\bullet\ \Opt{\terminalCode{static}}\ \terminalCode{method}\ \mT_0\ \mm\oR\overline{\mT\,\mx}\cR \ + \ \Opt{\terminalCode{static}}\ \terminalCode{method}\ \mT_0\ \mm\oR\overline{\mT\,\mx}\cR \Opt\me = \Opt{\terminalCode{static}}\ \terminalCode{method}\ \mT_0\ \mm\oR\overline{\mT\,\mx}\cR \Opt\me\\
\end{array}$

Sum composes the content of the arguments
by taking the union of their members and the union of their \Q@implements@.
Members with the same name are recursively composed.
There are three cases where the composition is impossible.
\begin{itemize}
\item \textit{Method-clash}: two methods with the same name are composed,
but either their headers have different types or they are both implemented.
\item \textit{Class-clash}: a class is composed with an interface.%
\footnote{
The full language relaxes this condition, for example an empty class can be seen as an empty interface during composition.
}
\item \textit{Implements-clash}:
the resulting code would not be well formed.
For example, in the following
\Q@t1+t2@ would result in a class \Q@B@ implementing \Q@A@ with method \Q@a()@,
but \Q@B@ does not have such method.%
\footnote{In \name it could be possible to try to patch class \Q@B@, for example by adding an
abstract method \Q@a()@;  we choose to instead give an error since in the full 42 language
such patch would 
be able to turn coherent private nested classes
into abstract (private) ones.}

\saveSpace\saveSpace
\begin{lstlisting}
t1={  A= {interface method Void a()}  }
t2={  A= {interface}     B= {implements A}  }
\end{lstlisting}\saveSpace\saveSpace

Implements-clash can happen only when composing nested interfaces. Note that while the first two kind of errors are obtained directly by the definition of 
$\mL_1+\mL_2$, Implements-clash is obtained from well-formedness, since injecting the resulting 
$\mL$ in to the program would make it ill-formed by 
$\mathit{consistentSubtype}(\overline\mDE,\mL)$.
\end{itemize}
\saveSpace
\subsection{Typing}
\saveSpace
Typing is composed by rules \Rulename{cd-ok}, \Rulename{td-ok},
\Rulename{l-ok},
\Rulename{nested-ok} and \Rulename{method-ok},
followed by expression typing rules
\Rulename{subsumption}, \Rulename{method-call}, \Rulename{x} and \Rulename{static-method-call}.

Rules \Rulename{cd-ok} and \Rulename{td-ok}
are interesting: a top level class is typed by replacing all occurrences of the name `\Q@This@' with the class name $C$,
and is required to be coherent.
On the other hand, a top level trait is typed by temporarily adding a mapping for
\Q@This@ to the typed program.

\noindent\textbf{Define }$\mathit{coherent}(\mT,\mL)$\\
$\begin{array}{l}
\!\!\!\bullet\ \mathit{coherent}(\mT,
\oC \Opt{\terminalCode{interface}}\ \terminalCode{implements} \overline\mT\ \overline\mM\cC
)\quad\text{holds where}\\

\quad\quad \forall \mC\terminalCode{=}\mL'\in\overline\mM \mathit{coherent}(\mT\terminalCode{.}\mC,\mL')\\
\quad\quad \text{and either }
\Opt{\terminalCode{interface}}=\terminalCode{interface}\\
\quad\quad\quad \text{or } 
\forall\ 
\terminalCode{method}\ \mT'\ \mm\oR\overline{\mT\,\mx}\cR \in\overline\mM,\ 
\text{state}(\text{factory}(\mT,\overline\mM), ~\terminalCode{method}\ \mT'\ \mm\oR\overline{\mT\,\mx}\cR)
\end{array}$

\noindent A Literal is \emph{coherent} if 
all the nested classes are coherent,
and either the Literal is an interface, there are no static methods, or all the static methods
are a valid \emph{state} method of the candidate \emph{factory}.
Note, by asking for
$\terminalCode{method}\ \mT'\ \mm\oR\overline{\mT\,\mx}\cR \in\overline\mM$
we select only abstract methods.

\noindent\textbf{Define }$\text{factory}(\mT,\overline\mM)$\\
$\begin{array}{l}

\!\!\!\bullet\ \text{factory}(\mT,\mM_1\ldots\mM_n)=\mM_i=\terminalCode{static method}\ \mT\, \mm
\oR
\_
\cR

\quad\text{where}\\
\quad\quad \forall j\neq i.\quad \mM_j 
\text{is not of the form}\ \terminalCode{static method}\ \_\, \_
\oR
\_
\cR
\end{array}$

\noindent The factory is the only static abstract  method, and
its return type is the nominal type of our class.

\noindent\textbf{Define }$\text{state}(\mM,\mM')$\\
$\begin{array}{l}


\!\!\!\bullet\ \text{state}(
\terminalCode{static}\ \terminalCode{method}\ \mT\ \mm\oR\mT_1\,\mx_1\ldots\mT_n\,\mx_n\cR,
~\terminalCode{method}\ \mT_i\ \mx_i\oR\cR
)\\

%\!\!\!\bullet\ \text{state}(
%\terminalCode{static}\ \terminalCode{method}\ \mT\ \mm\oR\mT_1\,\mx_1\ldots\mT_n\,\mx_n\cR,
%\terminalCode{method}\ \terminalCode{Void} \mx_i\oR\mT_i\,\terminalCode{that}\cR
%)\\

\!\!\!\bullet\ \text{state}(
\terminalCode{static}\ \terminalCode{method}\ \mT\ \mm\oR\mT_1\,\mx_1\ldots\mT_n\,\mx_n\cR,
~\terminalCode{method}\ \mT\ \terminalCode{with}\mx_i\oR\mT_i\,\terminalCode{that}\cR
)\\

\end{array}$

\noindent A non static method is part of the \emph{abstract state} if 
it is a valid getter or wither. In this simple formalism without imperative features we do not offer setters.


Rule \Rulename{Nested-OK} helps to accumulate the type of \Q@this@ so that rule \Rulename{Method-OK}
can use it.
Rule \Rulename{L-OK} is so simple since all the checks
related to correctly implementing interfaces are delegated to the well formedness criteria.
The expression typing rules are straightforward and standard.

\saveSpace
\subsection{Formal properties}
\saveSpace
\newtheorem{theorem}{Theorem}[section]
\MS{As can be expected, \name ensures conventional soundness of expression reduction. This property is expressed on a completely flattened program (a program where all $E$ are of form $L$):}
\begin{theorem}[Main Soundness]
if $\vdash\overline{D}:\text{OK}$,
e not of form $v_{{\smallDs}}$
and $\overline{D}\vdash e:T$
then
$e\leftarrow_{\overline{D}} \_$
\end{theorem}
\saveSpace\saveSpace\saveSpace
\noindent The proof is standard since the flattened language is just a minor variation over FJ.

In addition to conventional soundness of expression reduction,
\name ensures soundness of the compilation process itself.
A similar property was called meta-level-soundness in~\cite{servetto2014meta}; here we can obtain the same result in
a much simpler setting.
We denote $\mathit{wrong}(\overline\mD,\mE)$ to be the number of $\mL$s such that
$\mE=\ctx[\mL]\ \text{and not}\ \overline\mD\vdash\mL:\text{OK}$.

\begin{theorem}[Compilation Soundness]

if $\mE_0 \xrightarrow[\smallDs]{} \mE_1$
then $\mathit{wrong}(\overline\mD,\mE_0)\geq\mathit{wrong}(\overline\mD,\mE_1)$.
\end{theorem}
\saveSpace\saveSpace\saveSpace
\noindent This can be proved by cases on the applied reduction rule:
\begin{itemize}
\item
\Rulename{look-up} preserves the number of wrong literals:
$t \in \overline\mD$ and $\overline\mD$ is well typed by \Rulename{top} preconditions.
\item \Rulename{sum} either preserves or reduces the number of
wrong literals:
the core of the proof is to show that the sum of two well typed literals produces a well typed one.
A code literal is well typed (\Rulename{l-ok}) if all its method bodies are correct.
This holds since those same method bodies
are well typed in a strictly weaker environment with respect to the one used to type the result.
This is because every member in the result of the sum
is structurally a subtype of
the corresponding members in the operands.
Note that by well formedness, if \Rulename{sum}
is applied the result still respect 
$\mathit{consistentSubtype}$.
\end{itemize}
\noindent 
\textbf{Compilation Soundness has two important corollaries:}
\begin{itemize}
\item A class declared without literals
is well-typed after flattening; no need of further checking.
\item If a class is declared by using literals $\mL_1\ldots\mL_n$, and after successful flattening $\mC = \mL$ can not be type-checked,
then the issue was originally present in one of $\mL_1\ldots\mL_n$.
This also means that as an optimization strategy
 we may remember what method bodies come from traits and what method bodies come from code literals, and only type-check the latter.
If the result can not be type-checked, either it is intrinsically ill-typed or a 
referred type is declared \emph{after} the current class. 
As we see in the next section, we leverage on this 
to allow recursive types.
 \end{itemize}





\saveSpace
\subsection{Advantages of our compilation process}
\saveSpace

Our typing discipline is very simple from a formal perspective,  
and is what distinguishes our approach from a simple minded code composition macros~\cite{bawden1999quasiquotation}
or rigid module composition~\cite{ancona2002calculus}. 
It is built on two core ideas:

\paragraph{1: Traits are \emph{well-typed} before being reused.}
 For example:

\saveSpace\saveSpace\begin{lstlisting}
t={method int m(){return 2;} 
   method int n(){return this.m()+1;}}
\end{lstlisting}\saveSpace\saveSpace

\noindent \Q@t@ is well typed since \Q@m()@ is declared inside of \Q@t@, while the following would be ill-typed:

\saveSpace\saveSpace\begin{lstlisting}
t1={method int n(){return this.m()+1;}} //ill-typed
\end{lstlisting}\saveSpace\saveSpace


\paragraph{2: Code literals are \emph{not} required to be well-typed before flattening.}${}_{}$\\*
A literal $\mL$ in a declaration $\mD$
must be well formed and respect
$\mathit{consistentSubtype}$, but
it is not type-checked until flattening is complete:
only the result is required to be well-typed.
For example the following is correct since
the result of the flattening is well-typed:

\saveSpace\saveSpace\begin{lstlisting}
C= Use t, {method int k(){return this.n()+this.m();}}//correct code
\end{lstlisting}\saveSpace\saveSpace

\noindent The code literal
\Q@{method int k(){ return this.n()+this.m();}}@
is not well typed: \Q@n@, \Q@m@ are not locally defined.
This code would fail in many similar works in literature~\cite{deep,Bettini2015282,Bergel2007} where the
literals have to be \emph{self contained}. In this case we would have been forced to
declare abstract methods \Q@n@ and \Q@m@, even if \Q@t@ already 
provides such methods.

This relaxation allows multiple declarations to be flattened one at the time, without typing them individually, and only typing them all together.
In this way, we support recursive types%
\footnote{
OO languages leverage on recursive types most of the times:
for example \Q@String@ may offer a \Q@Int size()@
method, and \Q@Int@ may offer a \Q@String toString()@ method.
This means that typing classes 
\Q@String@ and \Q@Int@ in isolation one at a time is not possible.}
between multiple class declarations without
the need of predicting the resulting shape%
\footnote{This is needed in full 42: it is
impossible to predict the resulting shape since
arbitrary code can run at compile time.}.

As seen in \Rulename{top}, our compilation process
proceeds in a top-down fashion, flattening one declaration at a time,
a declaration needs
to be type-checked where their type is first needed,
that is, when they are required to type a trait used in a code expression.
That is, in \name typing and flattening are interleaved. We assume our compilation process stops as soon as 
an error arises. 
For example:
\saveSpace\saveSpace\begin{lstlisting}
ta={method int ma(){return 2;}}
tc={method int mc(A a, B b){return b.mb(a);}}
A= Use ta
B= {method int mb(A a){return a.ma()+1;}}
C= Use tc, {method int hello(){return 1;}}
\end{lstlisting}\saveSpace\saveSpace
In this scenario, since we compile top down, we first need to generate \Q@A@.
To generate \Q@A@, we need to use \Q@ta@ (but we do not need
\Q@tc@, in rule \Rulename{top}, $\overline\mD=\,$\Q@ta@ and $\overline\mD'=\,$\Q@tc@).
At this moment, \Q@tc@ cannot be compiled/checked alone:
information about \Q@A@ and \Q@B@ is needed.
To modularly ensure well-typedness,
we only require \Q@ta@ to be well typed at this stage; if it is not a type-error will be raised immediately. % This means that if \Q@ta@ was not well-typed%there would be a type error at this stage.
Now, we need to generate \Q@C@, and hence type-check \Q@tc@.
\Q@A@ is guaranteed to be already type-checked 
(since it is generated by an expression that does not contain any \mL),
and \Q@B@ can be typed. Finally \Q@tc@ can be typed and reused.
If the \Rulename{sum} rule could not be performed (for example if \Q@tc@ had a method \Q@hello@ too)
a composition error would be generated at this stage.
On the other hand, if \Q@B@ and \Q@C@ were swapped, as in:
\saveSpace\saveSpace\begin{lstlisting}
C= Use tc, {method int hello(){return 1;}}
B= {method int mb(A a){return a.ma()+1;}}
\end{lstlisting}\saveSpace\saveSpace
\noindent
we would be unable to type \Q@tc@, since we need to know the structure of \Q@A@ and \Q@B@.
A type error would be generated.%, on the lines of ``flattening of \Q@C@
%requires \Q@tc@, \Q@tc@ requires \Q@B@ that is defined later''.

%In this example, a more expressive compilation/precompilation process 
%could compute a dependency graph and, if possible, reorganize the list,

\paragraph{The cost: what expressive power we lose}${}_{}$\\*
We require declarations to be provided in the right dependency order, but sometimes no such order exists.
An example of a ``morally correct'' program where no right order exists is the following:
\saveSpace\saveSpace\begin{lstlisting}
t= { int mt(A a){return a.ma();}}
A= Use t, {int ma(){return 1;}}
\end{lstlisting}\saveSpace\saveSpace
Here the correctness of \Q@t@ depends on 
\Q@A@, that is in turn generated using \Q@t@.
We believe any typing allowing such programs would be fragile with respect to code evolution,
and could make human understanding of the code-reuse process much harder.
%
%Rewriting our example in Java may help to show how involved it is.
%\saveSpace\begin{lstlisting}
%class T{ int mt(A a){return a.ma();}
%class A extends T {int ma() {return 1;}}
%\end{lstlisting}\saveSpace
%
In sharp contrast with others (TR, PT, DJ, but also Java, C\#, and Scala)
we chose to not support this kind of involved programs.

TR, PT, DJ, Java, C\#, and Scala
accept a great deal of complexity in order to predict the structural shape of the resulting code before doing the actual code reuse/adaptation.
Those approaches logically divide the program in groups of mutually dependent classes, where each group may depend on a number of other groups.
This forms a direct acyclic graph of groups.
To type a group, all depended groups are typed, then
the signature/structural shape of all
the classes of the group are extracted.
Finally, with the information of the dependent groups and the current group, it is possible to type-check the implementation of each class in the group.

%Following this model, it is reasonable to assume that flattening happens group by group, before extracting the class signatures.



%\paragraph{In \name, typechecking before compiling would be redundant}${}_{}$\\*
%In the world of strongly typed languages we are tempted to
%first check that all can go well, and then perform the flattening.
%This would however be overcomplicated without any observable difference:
%Indeed, in the \Q@A,B,C@ example above there is no difference
%between
%\begin{itemize}
%\item  (1) First check \Q@B@ and produce \Q@B@ code (that also contains \Q@B@ structural shape),
%  (2) then use \Q@B@ shape to check \Q@C@ and produce \Q@C@ code;\ 
%or a more involved
%\item  (1) First check \Q@B@ and discover just \Q@B@ structural shape as result of the checking,
%  (2) then use \Q@B@ shape to check \Q@C@.
%  (3) Finally produce both \Q@B@ and \Q@C@ code.
%\end{itemize}
%
%
%This may seems a dangerous relaxation at first, but also Java has the same behaviour:
%\saveSpace\begin{lstlisting}[language=Java]
%  class A{ int ma() {return 2;}  int n(){return this.ma()+1;} }
%  class B extends A{ int mb(){return this.ma();} }
%\end{lstlisting}\saveSpace
%\noindent in \Q@B@ we can call \lstinline{this.ma()} even if in the curly braces there is no declaration for \Q@ma()@.
%
%



\noindent 
In the world of strongly typed languages we are tempted to
first check that all will go well, and then perform the flattening. 
Such methodology would be redundant in our setting: we can only reuse code through trait names; but our point of relaxation is only the code literal: in no way can an error ``move around'' and be duplicated during the compilation process.
That is, our approach allows safe libraries of traits and classes to be typechecked once, and then deployed and reused by multiple clients: as Theorem A.2 states, in \name no type error will emerge from library code.
%However, we do not force the programmer to write self-contained code where all the abstract method definition are explicitly declared.


\saveSpace
\subsection{Expression reduction}
\saveSpace
Our reduction rules are incredibly simple and standard.
A great advantage of our compilation model is that expressions are executed on
a simple fully flattened program, 
where all the composition operators have been removed.
From the point of view of expression reduction, \name is a simple language of 
interfaces and final classes, where nested classes give structure to code but have no special semantics.
The reduction of expressions is defined by rules
\Rulename{ctx-v}, \Rulename{s-m}, and \Rulename{m}.
The only interesting point is the auxiliary function $\mathit{meth}$:


\noindent\textbf{Define }$\mathit{meth}(\mM,\overline\vds)$

$\begin{array}{l}

\!\!\!\bullet\,\mathit{meth}(\terminalCode{static method}\ \mT\ \mm\oR\mT_1\, \mx_1\ldots\mT_n\,\mx_n\cR\,\me,\vds_1\ldots\vds_n)=\me[\mx_1=\vds_1,\ldots,\me_n=\vds_n]
\\

\!\!\!\bullet\,\mathit{meth}(\terminalCode{method}\ \mT\ \mm\oR\mT_1\, \mx_1\ldots\mT_n\,\mx_n\cR\,\me,\vds_0,\ldots,\vds_n)=\me[\terminalCode{this}=\vds_0,\mx_1=\vds_1,\ldots,\me_n=\vds_n]
\\

\!\!\!\bullet\,\mathit{meth}(\terminalCode{method}\ \mT_i\ \mx_i\oR\cR,\,\mT\terminalCode{.}\mm\oR\vds_1\ldots\vds_n\cR)=\vds_i\\
\quad \quad\text{where}\ \ \overline\mD(\mT,\mm) =
\terminalCode{static method}
\ \mT\,\mm\oR\mT_1\,\mx_1\ldots\mT_n\,\mx_n\cR
\\

\!\!\!\bullet\,\mathit{meth}(\terminalCode{method}\ \mT\ \terminalCode{with}\mx_i\oR\mT_i\,\terminalCode{that}\cR,\mT\terminalCode{.}\mm\oR\vds_1\ldots\vds_n\cR\,
\vds
)=
\mT\terminalCode{.}\mm\oR
\vds_1\ldots\vds_{i-1},
\vds,
\vds_{i+1}\ldots\vds_n
\cR
\\
\quad \quad\text{where}\ \ \overline\mD(\mT,\mm) =
\terminalCode{static method}
\ \mT\,\mm\oR\mT_1\,\mx_1\ldots\mT_n\,\mx_n\cR
\end{array}$

\noindent 
Here we take care of reading method bodies and preparing for
execution.
The first case is for static methods
and the second is for instance methods.
The third and fourth cases are more interesting, since they take care of
the abstract state:
the third case reduce getters and the fourth reduces withers.
In our formalisation we are not modelling state mutation, so there is 
no case for setters.

%For space reasons,
We omit the proof of conventional soundness for the
reduction. It is unsurprising, since the flattened calculus is like a
simplified version of Featherweight Java~\cite{igarashi2001featherweight}.


\section{Benefits without heavy costs}
In this section we discuss why the design of \name
is not intrinsically harder to use than Java.


\subsection{Introducing more names}
At first it may seem that our approach requires
introducing and keeping more names in mind, as for the \Q@set@ and \Q@Set@ example;
however this is not the case. 
The \textbf{user} of \Q@Set@ needs to keep in mind
only the class \Q@Set@.
The \textbf{re-user} of \Q@set@ needs to keep in mind only
 the trait \Q@set@.
Those are quite different roles, so they may very well
be handled by different programmers.
As shown in the \Q@set/Set@ example, defining a reusable class
takes only an extra line with respect to defining a final one.
In some sense, in Java the default is ``non final'', and
the programmer can write \Q@final@ to prevent code reuse, while
in 42 the default is reverted.

It could be possible to define a syntactic sugar, like
\begin{lstlisting}
reusable Set={...}
//expanded into
set={...}
Set=set
\end{lstlisting}
While this may sound appealing to a Java programmers, it only saves one line, and it may obscure the actual behaviour of the code.
This kind of behaviour is offered in Cecil~\cite{chambers1995typechecking}, where it makes sense, since in 
Cecil classes are not types.
Cecil offers syntactic sugar to declare a class and an
interface with the same structural type in a single declaration.

\subsection{Two ways to separate subclassing and subtyping}
There are indeed two different ways to separate subclassing and subtyping:

\begin{itemize}
\item \textbf{(1)Classes are not types}
In this solution (used by Cecil, Toil/PolyToil and most structurally typed languages) classes and types are different concepts; thus subclassing is not subtyping.
This also means that every method call is dynamically dispatched. Some authors consider this to be a security problem, and it is also out of those concerns that Java designers decided to support \Q@final@ classes.\footnote{
For example, a Java programmer may assume the call \Q@myPoint.getX()@ to behave as specified in the \Q@Point@ class. However, dynamic class loading may be used in a malicious way to provide a \Q@Point@ instance with a
dangerous behaviour. This can be easily prevented by declaring \Q@Point@ final. To insist on all calls to be dynamically dispatched requires to disable dynamic class loading or to verify all the dynamically loaded code.}
\item \textbf{(2)Classes are not extensible}
This is our solution: classes are types, but can not be reused. This requires to introduce another concept for code reuse: Traits.
We think our solution is much less radical than the former one.
A key advantage we have is that all class types are exact types, and method calls on class types are statically dispatched; supporting easier reasoning and higher security.
On the other side, all interface types obviously are not exact, since interfaces can not be instantiated,
and method calls on interface types obviously are dynamically dispatched, since interfaces have no implementation.
This builds toward a simpler mental model, and prevent bugs where programmers may forget about dynamic dispatch and reasoning over method calls using the specific implementation found in the receiver static type.
\end{itemize}

Both ways enforce about the same solution for the \emph{this-leaking} problem:
\begin{lstlisting}[language=Java]
  class A{ int ma(){return Utils.m(this);} }
  class Utils{static int m(A a){..}}
  class B extends A{ int mb(){return this.ma();} }  
\end{lstlisting}
In (1) the error is using \Q@A@ as a type in \Q@Utils.m@,
while in (2) the error is using \Q@extends@.

The solution proposed by (1) would look like
\begin{lstlisting}[language=Java, morekeywords={type, method}]
 type IA{interface method int ma()}//type = interface
 class Utils{static method int m(IA a){return ...} }
 class A{implements IA 
   method int ma(){return Utils.m(this);}}
 class B extends A{ int mb(){return this.ma();} }  
\end{lstlisting}
while our solution (2) is just one line longer:
\begin{lstlisting}
 IA={interface method int ma()}//interface with abstract method
 Utils={static method int m(IA a){return ...} }
 ta={implements IA
     method int ma(){return Utils.m(this);}}
 A=Use ta
 B=Use ta,{ int mb(){return this.ma();} }  
\end{lstlisting}


\subsection{Class hierarchies}
By having all the classes final, we prevent class hierarchies.
However,
we allow both trait hierarchies and interface hierarchies.
We believe having both these hierarchies,
one for code reuse and one for subtyping can improve code
maintainability, since those two hierarchies will be able to evolve independently. Furthermore, having two hierarchies does not makes code comprehension harder since, as for before, they have different roles and thus are relevant only one at the time.

Often, class hierarchies are used as a hack in order to obtain code reuse (and in those cases, the LSP is often violated).
In \name, trait hierarchies will just serve the same role better.


\subsection{Constructors and initialization}
Java allows constructors to do arbitrary operations.
The same behaviour can be obtained in \name by using a factory method; that is, the following:
\begin{lstlisting}[language=Java]
class A{
  public int x;
  public A(String s){
    this.x=/*any computation here*/;
    }
  }
\end{lstlisting}
could be expressed in \name as 
\begin{lstlisting}
A={
  method int x();
  static method This of(int x)
  static method This of(String s){
    return This.of(/*any computation here*/);
    }
  }
\end{lstlisting}

For space reasons we have not yet explained how to allow traits to have their own state with a default initialization.
This could be solved by adding to the abstract state operations (getters, setters, withers) a default initializer;
for example the trait
\begin{lstlisting}
t={
  method int x();
  method void x(int x);
  method int defaultX(){return 0;}//new concept here!
}
\end{lstlisting}
would loosely correspond to a Java abstract class
\begin{lstlisting}[language=Java]
abstract class T{
  private int x=0;
  public int x(){return x;}
  public void x(int that){return this.x=that;}
  }
\end{lstlisting}
and could be used to obtain concrete classes where the abstract factory does not need to initialize the field \Q@x@.
For example classes \Q@A@ and \Q@B@ below would be coherent classes
\begin{lstlisting}
A= Use t,{ static method This of()}
B= Use t,{ method String s(); static method This of(String s)}
...
A a=A.of();
a.x()==0;//assert a.x()==0
a.x(42);
a.x()==42;//assert a.x()==42

\end{lstlisting}

Finally, when self construction is not needed, declaring the abstract factory can be redundant, and it can indeed be omitted in the full 42 language.

Moreover, when code reuse is involved, \name is much simpler to use then Java,C++ or C\#: a class extending another class needs to define a constructor calling a super constructor.
   This typically has as many parameters as the full set of fields. Those parameters are then used at least one time in the body of the constructor.
In \name, to compose traits without adding fields, there is no need to repeat the abstract factory.
To add a field, you need to define a new abstract factory and implement the "super" factory method by calling the new one.
Thus, when adding a field the amount of code between 42 and Java is about the same, but is much easier when no fields are added.

How class invariants and state encapsulation is established and preserved without explicit super constructors is a different issue, and a paper on such topic is being prepared right now.

\subsection{Self instantiation is very useful}
In many cases the absence of self instantiation causes bugs: often programmers ``think'' that \Q@new C()@ inside of class \Q@C@ would create an instance of the ``current class'', while of course \Q@class D extends C@ still creates instances of \Q@C@.
The absence of self instantiation also prevents to use functional programming patterns in OO languages.
Just a look at how involved the code is to implement the functional \Q@Point@ in Java is sufficient to dissuade most programmers.

\subsection{Comprehensibility and usability}
Of course it could take a while to adapt from the Java model to the \name one. However, when it comes to model complex reuse scenario, we believe \name is much more explicit and clear then Scala and other options. We invite the curios reader to look to the complete code of the case studies (below) and compare our solution with the cryptic version of Scala, where behaviour traits are nested inside other traits.

In particular, with respect to Java/Scala we encourage more explicit code since:
\begin{itemize}
\item sub-typing need to be explicitly declared.
\item when sub typing is needed, an opportune interface needs to be declared. Thus, when using a class type, it is always clear what implementation is referred to.
\item \name requires to be explicit when subtyping and subclassing are both supposed to happen (by declaring an extra interface).
\item \Q@this.@ is required to access fields/methods (so they can not be confused with local variables)
\item our object model is equivalent to an untyped language.
\end{itemize}
The last point is quite important: languages where inheritance is not subtyping proposed in the past tend to have more complex object models, instead 
our flattened language is just a language of interfaces and final classes. Novice programmers may start by learning that. They could learn to use traits later, and to define their own traits even later.

\section{Case studies complete code} 

Here, for reference, you can find the complete
of the Case Studies.
It is quite a lot of code, about 18 pages. The interested reader
can navigate it in order to gain a perfect understanding on
how we encoded the various solutions in the various styles and languages,
but is not needed to understand the overall value of our paper.

While evaluating our approach, we consider not only lines of code but also number of methods and classes.
Our particular examples are focused on code reuse, thus method implementations are all trivial.
We believe that in this particular context the number of methods/classes is a good indicator.


\subsection{Point algebra}
Here the code of Java7 for the first case study, the \Q@Point@ algebra:
\begin{lstlisting}[basicstyle=\tiny]
//0
class Point {//6
  final int x;
  final int y;

  public Point(int x, int y) {
    this.x = x;
    this.y = y;
  }
}

// 1
class PointSum extends Point {//5*4
  public PointSum(int x, int y) {
    super(x, y);
  }

  public PointSum sum(Point that) {
    return new PointSum(this.x + that.x, this.y + that.y);
  }
}

// 2
class PointSub extends Point {
  public PointSub(int x, int y) {
    super(x, y);
  }

  public PointSub sub(Point that) {
    return new PointSub(this.x - that.x, this.y - that.y);
  }
}

// 3
class PointMul extends Point {
  public PointMul(int x, int y) {
    super(x, y);
  }

  public PointMul mul(Point that) {
    return new PointMul(this.x * that.x, this.y * that.y);
  }
}

// 4
class PointDiv extends Point {
  public PointDiv(int x, int y) {
    super(x, y);
  }

  public PointDiv div(Point that) {
    return new PointDiv(this.x / that.x, this.y / that.y);
  }
}

// 5
class PointSumSub extends PointSum {//7*6
  public PointSumSub(int x, int y) {
    super(x, y);
  }

  public PointSumSub sum(Point that) {
    return new PointSumSub(this.x + that.x, this.y + that.y);
  }

  public PointSumSub sub(Point that) {
    return new PointSumSub(this.x - that.x, this.y - that.y);
  }
}

// 6
class PointSumMul extends PointSum {
  public PointSumMul(int x, int y) {
    super(x, y);
  }

  public PointSumMul sum(Point that) {
    return new PointSumMul(this.x + that.x, this.y + that.y);
  }

  public PointSumMul mul(Point that) {
    return new PointSumMul(this.x * that.x, this.y * that.y);
  }
}

// 7
class PointSumDiv extends PointSum {
  public PointSumDiv(int x, int y) {
    super(x, y);
  }

  public PointSumDiv sum(Point that) {
    return new PointSumDiv(this.x + that.x, this.y + that.y);
  }

  public PointSumDiv div(Point that) {
    return new PointSumDiv(this.x / that.x, this.y / that.y);
  }
}

// 8
class PointSubMul extends PointSub {
  public PointSubMul(int x, int y) {
    super(x, y);
  }

  public PointSubMul sub(Point that) {
    return new PointSubMul(this.x - that.x, this.y - that.y);
  }

  public PointSubMul mul(Point that) {
    return new PointSubMul(this.x * that.x, this.y * that.y);
  }
}

// 9
class PointSubDiv extends PointSub {
  public PointSubDiv(int x, int y) {
    super(x, y);
  }

  public PointSubDiv sub(Point that) {
    return new PointSubDiv(this.x - that.x, this.y - that.y);
  }

  public PointSubDiv div(Point that) {
    return new PointSubDiv(this.x / that.x, this.y / that.y);
  }
}

// 10
class PointMulDiv extends PointMul {
  public PointMulDiv(int x, int y) {
    super(x, y);
  }

  public PointMulDiv mul(Point that) {
    return new PointMulDiv(this.x * that.x, this.y * that.y);
  }

  public PointMulDiv div(Point that) {
    return new PointMulDiv(this.x / that.x, this.y / that.y);
  }
}

// 11
class PointSumSubDiv extends PointSumSub {//9*4
  public PointSumSubDiv(int x, int y) {
    super(x, y);
  }

  public PointSumSubDiv sum(Point that) {
    return new PointSumSubDiv(this.x + that.x, this.y + that.y);
  }

  public PointSumSubDiv sub(Point that) {
    return new PointSumSubDiv(this.x - that.x, this.y - that.y);
  }

  public PointSumSubDiv div(Point that) {
    return new PointSumSubDiv(this.x / that.x, this.y / that.y);
  }
}

// 12
class PointSumSubMul extends PointSumSub {
  public PointSumSubMul(int x, int y) {
    super(x, y);
  }

  public PointSumSubMul sum(Point that) {
    return new PointSumSubMul(this.x + that.x, this.y + that.y);
  }

  public PointSumSubMul sub(Point that) {
    return new PointSumSubMul(this.x - that.x, this.y - that.y);
  }

  public PointSumSubMul mul(Point that) {
    return new PointSumSubMul(this.x * that.x, this.y * that.y);
  }
}

// 13
class PointSumMulDiv extends PointMulDiv {
  public PointSumMulDiv(int x, int y) {
    super(x, y);
  }

  public PointSumMulDiv sum(Point that) {
    return new PointSumMulDiv(this.x + that.x, this.y + that.y);
  }

  public PointSumMulDiv mul(Point that) {
    return new PointSumMulDiv(this.x * that.x, this.y * that.y);
  }

  public PointSumMulDiv div(Point that) {
    return new PointSumMulDiv(this.x / that.x, this.y / that.y);
  }
}

// 14
class PointSubMulDiv extends PointMulDiv {
  public PointSubMulDiv(int x, int y) {
    super(x, y);
  }

  public PointSubMulDiv sub(Point that) {
    return new PointSubMulDiv(this.x - that.x, this.y - that.y);
  }

  public PointSubMulDiv mul(Point that) {
    return new PointSubMulDiv(this.x * that.x, this.y * that.y);
  }

  public PointSubMulDiv div(Point that) {
    return new PointSubMulDiv(this.x / that.x, this.y / that.y);
  }
}

// 15
class PointSumSubMulDiv extends PointSumSubMul {//11
  public PointSumSubMulDiv(int x, int y) {
    super(x, y);
  }

  public PointSumSubMulDiv sum(Point that) {
    return new PointSumSubMulDiv(this.x + that.x, this.y + that.y);
  }

  public PointSumSubMulDiv sub(Point that) {
    return new PointSumSubMulDiv(this.x - that.x, this.y - that.y);
  }

  public PointSumSubMulDiv mul(Point that) {
    return new PointSumSubMulDiv(this.x * that.x, this.y * that.y);
  }

  public PointSumSubMulDiv div(Point that) {
    return new PointSumSubMulDiv(this.x / that.x, this.y / that.y);
  }
}
\end{lstlisting}

Here you can find the ClassLessJava style code for the Point Algebra:
\begin{lstlisting}[basicstyle=\tiny]
// 0
@Obj // 3 lines
interface Point {
  static Point of(int x, int y) {// methods "of" generated by @Obj thus not counted in the line numbers
    return new Point() {
      public int x() {
        return x;
      }

      public int y() {
        return y;
      }
    };
  }

  int x();

  int y();
}

// 1
@Obj // 3*4 lines
interface PointSum extends Point {
  static PointSum of(int x, int y) {
    return new PointSum() {
      public int x() {
        return x;
      }

      public int y() {
        return y;
      }
    };
  }

  default PointSum sum(Point that) {
    return of(this.x() + that.x(), this.y() + that.y());
  }
}

// 2
@Obj
interface PointSub extends Point {
  static PointSub of(int x, int y) {
    return new PointSub() {
      public int x() {
        return x;
      }

      public int y() {
        return y;
      }
    };
  }

  default PointSub sub(Point that) {
    return of(this.x() - that.x(), this.y() - that.y());
  }
}

// 3
@Obj
interface PointMul extends Point {
  static PointMul of(int x, int y) {
    return new PointMul() {
      public int x() {
        return x;
      }

      public int y() {
        return y;
      }
    };
  }

  default PointMul mul(Point that) {
    return of(this.x() * that.x(), this.y() * that.y());
  }
}

// 4
@Obj
interface PointDiv extends Point {
  static PointDiv of(int x, int y) {
    return new PointDiv() {
      public int x() {
        return x;
      }

      public int y() {
        return y;
      }
    };
  }

  default PointDiv div(Point that) {
    return of(this.x() / that.x(), this.y() / that.y());
  }
}

// 5
@Obj // 5*6 lines
interface PointSumMul extends PointSum, PointMul {
  static PointSumMul of(int x, int y) {
    return new PointSumMul() {
      public int x() {
        return x;
      }

      public int y() {
        return y;
      }
    };
  }

  default PointSumMul sum(Point that) {// we have to rewrite the method calling the new "of"
    return of(this.x() + that.x(), this.y() + that.y());// in order to produce an instance of PointSumMul
  }

  default PointSumMul mul(Point that) {
    return of(this.x() * that.x(), this.y() * that.y());
  }
}

// 6
@Obj
interface PointSumSub extends PointSum, PointSub {
  static PointSumSub of(int x, int y) {
    return new PointSumSub() {
      public int x() {
        return x;
      }

      public int y() {
        return y;
      }
    };
  }

  default PointSumSub sum(Point that) {// we have to rewrite the method calling the new "of"
    return of(this.x() + that.x(), this.y() + that.y());// in order to produce an instance of PointSumMul
  }

  default PointSumSub sub(Point that) {
    return of(this.x() - that.x(), this.y() - that.y());
  }
}

// 7
@Obj
interface PointSumDiv extends PointSum, PointDiv {
  static PointSumDiv of(int x, int y) {
    return new PointSumDiv() {
      public int x() {
        return x;
      }

      public int y() {
        return y;
      }
    };
  }

  default PointSumDiv sum(Point that) {// we have to rewrite the method calling the new "of"
    return of(this.x() + that.x(), this.y() + that.y());// in order to produce an instance of PointSumMul
  }

  default PointSumDiv div(Point that) {
    return of(this.x() / that.x(), this.y() / that.y());
  }
}

// 8
@Obj
interface PointSubMul extends PointMul, PointSub {
  static PointSubMul of(int x, int y) {
    return new PointSubMul() {
      public int x() {
        return x;
      }

      public int y() {
        return y;
      }
    };
  }

  default PointSubMul mul(Point that) {// we have to rewrite the method calling the new "of"
    return of(this.x() * that.x(), this.y() * that.y());// in order to produce an instance of PointSumMul
  }

  default PointSubMul sub(Point that) {
    return of(this.x() - that.x(), this.y() - that.y());
  }
}

// 9
@Obj
interface PointSubDiv extends PointSub, PointDiv {
  static PointSubDiv of(int x, int y) {
    return new PointSubDiv() {
      public int x() {
        return x;
      }

      public int y() {
        return y;
      }
    };
  }

  default PointSubDiv sub(Point that) {// we have to rewrite the method calling the new "of"
    return of(this.x() - that.x(), this.y() - that.y());// in order to produce an instance of PointSumMul
  }

  default PointSubDiv div(Point that) {
    return of(this.x() / that.x(), this.y() / that.y());
  }
}

// 10
@Obj
interface PointMulDiv extends PointMul, PointDiv {
  static PointMulDiv of(int x, int y) {
    return new PointMulDiv() {
      public int x() {
        return x;
      }

      public int y() {
        return y;
      }
    };
  }

  default PointMulDiv mul(Point that) {// we have to rewrite the method calling the new "of"
    return of(this.x() * that.x(), this.y() * that.y());// in order to produce an instance of PointSumMul
  }

  default PointMulDiv div(Point that) {
    return of(this.x() / that.x(), this.y() / that.y());
  }
}

// 11
@Obj // 7*4 lines
interface PointSumSubDiv extends PointSumSub, PointSumDiv, PointSubDiv {
  static PointSumSubDiv of(int x, int y) {
    return new PointSumSubDiv() {
      public int x() {
        return x;
      }

      public int y() {
        return y;
      }
    };
  }

  default PointSumSubDiv sum(Point that) {
    return of(this.x() + that.x(), this.y() + that.y());
  }

  default PointSumSubDiv sub(Point that) {
    return of(this.x() - that.x(), this.y() - that.y());
  }

  default PointSumSubDiv div(Point that) {
    return of(this.x() / that.x(), this.y() / that.y());
  }
}

// 12
@Obj
interface PointSumSubMul extends PointSumSub, PointSumMul, PointSubMul {
  static PointSumSubMul of(int x, int y) {
    return new PointSumSubMul() {
      public int x() {
        return x;
      }

      public int y() {
        return y;
      }
    };
  }

  default PointSumSubMul sum(Point that) {
    return of(this.x() + that.x(), this.y() + that.y());
  }

  default PointSumSubMul sub(Point that) {
    return of(this.x() - that.x(), this.y() - that.y());
  }

  default PointSumSubMul mul(Point that) {
    return of(this.x() * that.x(), this.y() * that.y());
  }
}

// 13
@Obj
interface PointSumMulDiv extends PointSumMul, PointMulDiv, PointSumDiv {
  static PointSumMulDiv of(int x, int y) {
    return new PointSumMulDiv() {
      public int x() {
        return x;
      }

      public int y() {
        return y;
      }
    };
  }

  default PointSumMulDiv sum(Point that) {
    return of(this.x() + that.x(), this.y() + that.y());
  }

  default PointSumMulDiv mul(Point that) {// we have to rewrite the method calling the new "of"
    return of(this.x() * that.x(), this.y() * that.y());// in order to produce an instance of PointSumMul
  }

  default PointSumMulDiv div(Point that) {
    return of(this.x() / that.x(), this.y() / that.y());
  }
}

// 14
@Obj
interface PointSubMulDiv extends PointSubMul, PointMulDiv, PointSubDiv {
  static PointSubMulDiv of(int x, int y) {
    return new PointSubMulDiv() {
      public int x() {
        return x;
      }

      public int y() {
        return y;
      }
    };
  }

  default PointSubMulDiv sub(Point that) {
    return of(this.x() - that.x(), this.y() - that.y());
  }

  default PointSubMulDiv mul(Point that) {// we have to rewrite the method calling the new "of"
    return of(this.x() * that.x(), this.y() * that.y());// in order to produce an instance of PointSumMul
  }

  default PointSubMulDiv div(Point that) {
    return of(this.x() / that.x(), this.y() / that.y());
  }
}

// 15
@Obj//9*1 lines
interface PointSumSubMulDiv extends PointSumSubMul, PointSumMulDiv, PointSumSubDiv, PointSubMulDiv {
  static PointSumSubMulDiv of(int x, int y) {
    return new PointSumSubMulDiv() {
      public int x() {
        return x;
      }

      public int y() {
        return y;
      }
    };
  }

  default PointSumSubMulDiv sum(Point that) {
    return of(this.x() + that.x(), this.y() + that.y());
  }

  default PointSumSubMulDiv sub(Point that) {
    return of(this.x() - that.x(), this.y() - that.y());
  }

  default PointSumSubMulDiv mul(Point that) {// we have to rewrite the method calling the new "of"
    return of(this.x() * that.x(), this.y() * that.y());// in order to produce an instance of PointSumMul
  }

  default PointSumSubMulDiv div(Point that) {
    return of(this.x() / that.x(), this.y() / that.y());
  }
}
\end{lstlisting}

Here you can find the Scala code for the Point Algebra:
\begin{lstlisting}[basicstyle=\tiny]
trait tPointState {//5
  type p <: tPointState
  def x: Int
  def y: Int
  def of(x:Int,y:Int):p
}

trait tPointSum extends tPointState {//3*4
  def sum(that:p)=
    this.of(this.x+that.x,this.y+that.y)  
}

trait tPointSub extends tPointState {
  def sub(that:p)=
    this.of(this.x-that.x,this.y-that.y)  
}

trait tPointMul extends tPointState {
  def mul(that:p)=
    this.of(this.x*that.x,this.y*that.y)  
}

trait tPointDiv extends tPointState {
  def div(that:p)=
    this.of(this.x/that.x,this.y/that.y)  
}
//glue code from now on
class Point0(val x:Int, val y:Int) extends tPointState {//4*16
 override type p = Point0
 override def of(x:Int,y:Int)=
   new Point0(x,y)//3 duplication of "this name"
}

class PointSum(val x:Int, val y:Int) extends tPointSum {
 override type p = PointSum
 override def of(x:Int,y:Int)=
   new PointSum(x,y)
}

class PointSub(val x:Int, val y:Int) extends tPointSub {
 override type p = PointSub
 override def of(x:Int,y:Int)=
   new PointSub(x,y)
}

class PointMul(val x:Int, val y:Int) extends tPointMul {
 override type p = PointMul
 override def of(x:Int,y:Int)=
   new PointMul(x,y)
}

class PointDiv(val x:Int, val y:Int) extends tPointDiv {
 override type p = PointDiv
 override def of(x:Int,y:Int)=
   new PointDiv(x,y)
}

class PointSumSub(val x:Int, val y:Int) extends tPointSum with tPointSub {
 override type p = PointSumSub
 override def of(x:Int,y:Int)=
   new PointSumSub(x,y)
}

class PointSumMul(val x:Int, val y:Int) extends tPointSum with tPointMul {
 override type p = PointSumMul
 override def of(x:Int,y:Int)=
   new PointSumMul(x,y)
}

class PointSumDiv(val x:Int, val y:Int) extends tPointSum with tPointDiv {
 override type p = PointSumDiv
 override def of(x:Int,y:Int)=
   new PointSumDiv(x,y)
}

class PointSubMul(val x:Int, val y:Int) extends tPointSub with tPointMul {
 override type p = PointSubMul
 override def of(x:Int,y:Int)=
   new PointSubMul(x,y)
}

class PointSubDiv(val x:Int, val y:Int) extends tPointSub with tPointDiv {
 override type p = PointSubDiv
 override def of(x:Int,y:Int)=
   new PointSubDiv(x,y)
}

class PointMulDiv(val x:Int, val y:Int) extends tPointMul with tPointDiv {
 override type p = PointMulDiv
 override def of(x:Int,y:Int)=
   new PointMulDiv(x,y)
}

class PointSumSubDiv(val x:Int, val y:Int) extends tPointSum with tPointSub with tPointDiv {
 override type p = PointSumSubDiv
 override def of(x:Int,y:Int)=
   new PointSumSubDiv(x,y)
}

class PointSumSubMul(val x:Int, val y:Int) extends tPointSum with tPointSub with tPointMul {
 override type p = PointSumSubMul
 override def of(x:Int,y:Int)=
   new PointSumSubMul(x,y)
}

class PointSumMulDiv(val x:Int, val y:Int) extends tPointSum with tPointMul with tPointDiv {
 override type p = PointSumMulDiv
 override def of(x:Int,y:Int)=
   new PointSumMulDiv(x,y)
}

class PointSubMulDiv(val x:Int, val y:Int) extends tPointSub with tPointMul with tPointDiv {
 override type p = PointSubMulDiv
 override def of(x:Int,y:Int)=
   new PointSubMulDiv(x,y)
}

class PointSumSubMulDiv(val x:Int, val y:Int) extends tPointSum with tPointSub with tPointMul with tPointDiv {
 override type p = PointSumSubMulDiv
 override def of(x:Int,y:Int)=
   new PointSumSubMulDiv(x,y)
}
\end{lstlisting}

Here you can find the full 42 code for the Point Algebra.
The main differences with respect to \name as presented in
the paper, is that we use \Q@Resource<><@
to declare a trait,
all trait names start with \texttt{\$}
and are referred with `\Q@()@'.
Method bodies can be just simple expressions, without the need
of `\Q@return@'.
Note that in full 42 there is no primitive type \Q@int@ and
the class representing numbers is called \Q@Num@.

\begin{lstlisting}[basicstyle=\tiny, mathescape=false]
$p: Resource <>< {//4
  method Num x()
  method Num y()
  class method This of(Num x,Num y)
  }

$pointSum: Resource <>< Use[$p()] <>< {//3*4
  method This sum(This that)
    This.of(x: this.x()+that.x(), y: this.y()+that.y())
  }

$pointSub: Resource <>< Use[$p()] <>< {
  method This sub(This that)
    This.of(x: this.x()-that.x(), y: this.y()-that.y())
  }

$pointMul: Resource <>< Use[$p()] <>< {
  method This mul(This that)
    This.of(x: this.x()*that.x(), y: this.y()*that.y())
  }

$pointDiv: Resource <>< Use[$p()] <>< {
  method This div(This that)
    This.of(x: this.x()/that.x(), y: this.y()/that.y())
  }

Point: Use[$p()] <>< {}//1*16

PointSum: Use[$pointSum()] <>< {}

PointSub: Use[$pointSub()] <>< {}

PointMul: Use[$pointMul()] <>< {}

PointDiv: Use[$pointDiv()] <>< {}

PointSumSub: Use[$pointSum();$pointSub()] <>< {}

PointSumMul: Use[$pointSum();$pointMul()] <>< {}

PointSumDiv: Use[$pointSum();$pointDiv()] <>< {}

PointSubMul: Use[$pointSub();$pointMul()] <>< {}

PointSubDiv: Use[$pointSub();$pointDiv()] <>< {}

PointMulDiv: Use[$pointMul();$pointDiv()] <>< {}

PointSumSubDiv: Use[$pointSum();$pointSub();$pointDiv()] <>< {}

PointSumSubMul: Use[$pointSum();$pointSub();$pointMul()] <>< {}

PointSumMulDiv: Use[$pointSum();$pointMul();$pointDiv()] <>< {}

PointSubMulDiv: Use[$pointSub();$pointMul();$pointDiv()] <>< {}

PointSumSubMulDiv: Use[$pointSum();$pointSub();$pointMul();$pointDiv()] <>< {}
\end{lstlisting}


\subsection{FCPoint}
Here is the code of Java7 for the second case study;
flavoured and colored points:
\begin{lstlisting}[basicstyle=\tiny]
enum Flavor {
  NONE, SOUR, SWEET, SALTY, SPLICY;
}

class Color {
  final int r;
  final int g;
  final int b;

  public Color(int r, int g, int b) {
    this.r = r;
    this.g = g;
    this.b = b;
  }

  public Color mix(Color that) {
    return new Color((this.r + that.r) / 2, (this.g + that.g) / 2, (this.b + that.b) / 2);
  }
}

class Point {//10 lines
  final int x;
  final int y;

  public Point(int x, int y) {
    this.x = x;
    this.y = y;
  }

  public Point withX(int that) {
    return new Point(that, this.y);
  }

  public Point withY(int that) {
    return new Point(this.x, that);
  }
}

class PointSum extends Point {//9 lines
  public PointSum(int x, int y) {
    super(x, y);
  }

  public PointSum withX(int that) {
    return new PointSum(that, this.y);
  }

  public PointSum withY(int that) {
    return new PointSum(this.x, that);
  }

  public PointSum sum(Point that) {
    return this.withX(this.x + that.x).withY(this.y + that.y);
  }
}

class CPoint extends PointSum {//13 lines
  final Color color;

  public CPoint(int x, int y, Color color) {
    super(x, y);
    this.color = color;
  }

  public CPoint withX(int that) {
    return new CPoint(that, this.y, this.color);
  }

  public CPoint withY(int that) {
    return new CPoint(this.x, that, this.color);
  }

  public CPoint withColor(Color that) {
    return new CPoint(this.x, this.y, that);
  }

  public CPoint merge(CPoint that) {
    return this.withColor(this.color.mix(that.color));
  }
}

class FCPoint extends CPoint {//15 lines
  final Flavor flavor;

  public FCPoint(int x, int y, Color color, Flavor flavor) {
    super(x, y, color);
    this.flavor = flavor;
  }

  public FCPoint withX(int that) {
    return new FCPoint(that, this.y, this.color, this.flavor);
  }

  public FCPoint withY(int that) {
    return new FCPoint(this.x, that, this.color, this.flavor);
  }

  public FCPoint withColor(Color that) {
    return new FCPoint(this.x, this.y, that, this.flavor);
  }

  public FCPoint withFlavor(Flavor that) {
    return new FCPoint(this.x, this.y, this.color, that);
  }

  public FCPoint merge(FCPoint that) {
    return this.withColor(that.color).withFlavor(that.flavor);
  }
}
\end{lstlisting}
Here is the code in Scala for flavoured and colored points:
\begin{lstlisting}[basicstyle=\tiny]
trait tPointState {//8
  type p <: tPointState
  def x: Int
  def y: Int
  def withX(that:Int):p
  def withY(that:Int):p
  def of(x:Int,y:Int):p
  def merge(that:p):p
}

trait tPointSum extends tPointState {//3
  def sum(that:p)=
    this.merge(that).withX(this.x+that.x).withY(this.y+that.y)  
}

trait tColored {//6*2
  type p <: tColored
  def color:Int
  def withColor(that:Int):p
  def merge(that:p)=
    this.withColor(this.color+that.color)
  }

trait tFlavored{
  type p <: tFlavored
  def flavor:Int
  def withFlavor(that:Int):p
  def merge(that:p)=
    this.withFlavor(that.flavor)
  }
//glue code from now on //12 lines
class CPoint(val x:Int, val y:Int, val color:Int) extends tPointSum with tColored {
  override type p = CPoint
  def of(x:Int,y:Int)=
    this.of(x,y,0)
  def of(x:Int,y:Int,color:Int)=
    new CPoint(x,y,color)
  def withX(that:Int)=
    of(that,y,color)
  def withY(that:Int)=
    of(x,that,color)
  def withColor(that:Int)=
    of(x,y,that)
  }
//18 lines
class FCPoint(val x:Int, val y:Int, val color:Int, val flavor:Int) extends tPointSum with tColored with tFlavored{
  override type p = FCPoint
  def of(x:Int,y:Int)=
    this.of(x,y,0,0)
  def of(x:Int,y:Int,color:Int,flavor:Int)=
    new FCPoint(x,y,color,flavor)
  def withX(that:Int)=
    of(that,y,color,flavor)
  def withY(that:Int)=
    of(x,that,color,flavor)
  def withColor(that:Int)=
    of(x,y,that,flavor)
  def withFlavor(that:Int)=
    of(x,y,color,that)
  def superMergeFlavoured(that:FCPoint)=
    super[tFlavored].merge(that)
  override def merge(that:FCPoint)=
      super[tColored].merge(that).superMergeFlavoured(that)
  }
\end{lstlisting}

Here you can find the 42 code for flavored and colored points..

\begin{lstlisting}[basicstyle=\tiny, mathescape=false]
Flavor: Enumeration"sour, sweet, salty, spicy"

Color:Data<><{Num r Num g Num b
  method This mix(This that)
    Color( r:(this.r()+that.r())/2Num, g:(this.g()+that.g())/2Num, b:(this.b()+that.b())/2Num)
  }

$p: Resource <>< {// 7 lines
  method Num x()
  method Num y() //getters
  method This withX(Num that)
  method This withY(Num that)//withers
  class method This of(Num x,Num y)
  method This merge(This that) //new method merge!
  }

$pointSum: Resource <>< Use[$p()] <>< {// 3 lines
  method This sum(This that)
    this.merge(that).withX(this.x()+that.x()).withY(this.y()+that.y())
  }

$colored: Resource <>< {// 5 *2 lines
  method Color color()
  method This withColor(Color that)
  method This merge(This that) //how to merge color handled here
    this.withColor(this.color().mix(that.color()))
  }

$flavored: Resource <>< {
  method Flavor flavor() //very similar to colored
  method This withFlavor(Flavor that)
  method This merge(This that) //merging flavors handled here
    this.withFlavor(that.flavor())//inherits "that" flavor
  }

CPoint: Use[$pointSum();$colored()] <>< {// 4 lines
  class method This of(Num x,Num y)
    This.of(x:x,y:y,color:Color(r:100Num,g:0Num,b:0Num))
  class method This of(Num x, Num y,Color color)
  }

FCPoint: Use[//9 lines
  Refactor2.toAbstract(selector:\"merge(that)", into:\"_1merge(that)")<><$colored();
  Refactor2.toAbstract(selector:\"merge(that)", into:\"_2merge(that)")<><$flavored();
  $pointSum()] <>< {
    class method This of(Num x,Num y)
      This.of(x:x,y:y,color:Color(r:100Num,g:0Num,b:0Num),flavor:Flavor.none())
    class method This of(Num x, Num y,Color color,Flavor flavor)
    method This merge(This that)
      this._1merge(that)._2merge(that)
    }
\end{lstlisting}

\subsection{Expression problem}
Here is the original code of Scala solving the expression problem.
As you can see, sometimes multiple operations/cases are declared together in
the same trait.
\begin{lstlisting}[basicstyle=\tiny]
trait Base {
  type exp <: Exp

  trait Exp {
    def eval: Int
  }

  class Num(val value: Int) extends Exp {
    def eval: Int = value
  }

  type BaseNum = Num
}

trait BasePlus extends Base {
  class Plus(val left: exp, val right: exp) extends Exp {
    def eval: Int = left.eval + right.eval
  }

  type BasePlus = Plus
}

trait BaseNeg extends Base {
  class Neg(val term: exp) extends Exp {
    def eval = -term.eval
  }

  type BaseNeg = Neg
}
trait BasePlusNeg extends BasePlus with BaseNeg
trait Show extends Base {
  type exp <: Exp

  trait Exp extends super.Exp {
    def show: String
  }

  trait NumBehavior extends Exp {
    self: BaseNum =>
    override def show: String = value.toString
  }

  final class Num(v: Int) extends BaseNum(v) with NumBehavior with Exp
}

trait ShowPlusNeg extends BasePlusNeg with Show {
  trait PlusBehavior {
    self: BasePlus =>
    def show = left.show + "+" + right.show;
  }

  final class Plus(l: exp, r: exp) extends BasePlus(l, r) with PlusBehavior with Exp

  trait NegBehavior {
    self: BaseNeg =>
    def show = "-(" + term.show + ")";
  }
  
  class Neg(t: exp) extends BaseNeg(t) with NegBehavior with Exp
}

trait DblePlusNeg extends BasePlusNeg {
  type exp <: Exp

  trait Exp extends super.Exp {
    def dble: exp
  }

  def Num(v: Int): exp

  def Plus(l: exp, r: exp): exp

  def Neg(t: exp): exp

  trait NumBehavior {
    self: BaseNum =>
    def dble = Num(value * 2)
  }

  final class Num(v: Int) extends BaseNum(v) with NumBehavior with Exp

  trait PlusBehavior {
    self: BasePlus =>
    def dble = Plus(left.dble, right.dble)
  }

  class Plus(l: exp, r: exp) extends BasePlus(l, r) with PlusBehavior with Exp

  trait NegBehavior {
    self: BaseNeg =>
    def dble = Neg(term.dble)
  }

  class Neg(t: exp) extends super.Neg(t) with NegBehavior with Exp
}
//-- 52 lines up to here, not counting new lines and '}'
trait Equals extends Base {
  type exp <: Exp;

  trait Exp extends super.Exp {
    def eql(other: exp): Boolean;

    def isNum(v: Int): Boolean = false;
  }

  trait NumBehavior extends Exp {
    self: BaseNum =>
    def eql(other: exp): Boolean = other.isNum(value);

    override def isNum(v: Int) = v == value;
  }

  final class Num(v: Int) extends BaseNum(v) with NumBehavior with Exp
}

trait EqualsPlusNeg extends BasePlusNeg with Equals {
  type exp <: Exp;

  trait Exp extends super[BasePlusNeg].Exp
    with super[Equals].Exp {
    def isPlus(l: exp, r: exp): Boolean = false;

    def isNeg(t: exp): Boolean = false;
  }

  final class Num(v: Int) extends BaseNum(v)
    with NumBehavior // effectively super[Equals].NumBehavior
    with Exp

  trait PlusBehavior extends Exp {
    self: BasePlus =>
    def eql(other: exp): Boolean = other.isPlus(left, right);

    override def isPlus(l: exp, r: exp) = (left eql l) && (right eql r)
  }

  final class Plus(l: exp, r: exp) extends BasePlus(l, r) with PlusBehavior with Exp

  trait NegBehavior extends Exp {
    self: BaseNeg =>
    def eql(other: exp): Boolean = other.isNeg(term);

    override def isNeg(t: exp) = term eql t
  }

  final class Neg(t: exp) extends BaseNeg(t) with NegBehavior with Exp
}

trait EqualsShowPlusNeg extends EqualsPlusNeg with ShowPlusNeg {
  type exp <: Exp

  trait Exp extends super[EqualsPlusNeg].Exp
    with super[ShowPlusNeg].Exp

  trait NumBehavior extends super[EqualsPlusNeg].NumBehavior with super[ShowPlusNeg].NumBehavior {
    self: BaseNum =>
  }

  class Num(v: Int) extends BaseNum(v) with NumBehavior with Exp

  trait PlusBehavior extends super[EqualsPlusNeg].PlusBehavior with super[ShowPlusNeg].PlusBehavior {
    self: BasePlus =>
  }

  class Plus(l: exp, r: exp) extends BasePlus(l, r) with PlusBehavior with Exp

  trait NegBehavior extends super[EqualsPlusNeg].NegBehavior with super[ShowPlusNeg].NegBehavior {
    self: BaseNeg =>
  }

  class Neg(term: exp) extends BaseNeg(term) with NegBehavior with Exp
}
//--- 40 lines for equals
\end{lstlisting}


In the following, the fully modularized Scala code,
every trait defines exactly one operation for each datavariant:
\begin{lstlisting}[basicstyle=\tiny]
trait Base {
  type exp <: Exp

  trait Exp {  }
}

trait BaseNum extends Base {
  class Num(val value: Int) extends Exp {  }

  type BaseNum = Num
}

trait BasePlus extends Base {
  class Plus(val left: exp, val right: exp) extends Exp {  }

  type BasePlus = Plus
}

trait BaseNeg extends Base {
  class Neg(val term: exp) extends Exp {  }

  type BaseNeg = Neg
}
//----------------------EVAL
trait Eval extends Base {
  type exp <: Exp

  trait Exp extends super.Exp {
    def eval: Int
  }
}

//----------------------EVALNUM
trait EvalNum extends BaseNum with Eval {
  trait NumBehavior {
    self: BaseNum =>
    def eval: Int = value
  }

  class Num(v: Int) extends BaseNum(v) with NumBehavior with Exp
}

//----------------------EVALPLUS
trait EvalPlus extends BasePlus with Eval {
  trait PlusBehavior {
    self: BasePlus =>
    def eval = left.eval + right.eval;
  }

  class Plus(l: exp, r: exp) extends BasePlus(l, r) with PlusBehavior with Exp
}
//----------------------EVALNEG
trait EvalNeg extends BaseNeg with Eval {
  trait NegBehavior {
    self: BaseNeg =>
    def eval = - term.eval;
  }
  
  class Neg(t: exp) extends BaseNeg(t) with NegBehavior with Exp
}
//----------------------SHOW
trait Show extends Base {
  type exp <: Exp

  trait Exp extends super.Exp {
    def show: String
  }
}
//----------------------SHOWNUM
trait ShowNum extends BaseNum with Show {
  trait NumBehavior {
    self: BaseNum =>
    def show: String = value.toString
  }

  class Num(v: Int) extends BaseNum(v) with NumBehavior with Exp
}

//----------------------SHOWPLUS
trait ShowPlus extends BasePlus with Show {
  trait PlusBehavior {
    self: BasePlus =>
    def show = left.show + "+" + right.show;
  }

  class Plus(l: exp, r: exp) extends BasePlus(l, r) with PlusBehavior with Exp
}
//----------------------SHOWNEG
trait ShowNeg extends BaseNeg with Show {
  trait NegBehavior {
    self: BaseNeg =>
    def show = "-(" + term.show + ")";
  }

  class Neg(t: exp) extends BaseNeg(t) with NegBehavior with Exp
}
//----------------------DBLE
trait Dble extends Base {
  type exp <: Exp

  trait Exp extends super.Exp {
    def dble: exp
  }
}
//----------------------DBLENUM
trait DbleNum extends BaseNum with Dble {
  type exp <: Exp

  trait Exp extends super[BaseNum].Exp with super[Dble].Exp 
  
  trait NumBehavior {
    self: BaseNum =>
    def dble = Num(value * 2)
  }
  
  def Num(v: Int): exp

  class Num(v: Int) extends super.Num(v) with NumBehavior
}
//----------------------DBLEPLUS
trait DblePlus extends BasePlus with Dble {
  type exp <: Exp
  
  trait Exp extends super[BasePlus].Exp with super[Dble].Exp
  
  trait PlusBehavior {
    self: BasePlus =>
    def dble = Plus(left.dble, right.dble)
  }

  def Plus(l: exp, r: exp): exp
  
  class Plus(l: exp, r: exp) extends super.Plus(l, r) with PlusBehavior with Exp
}
//----------------------DBLENEG
trait DbleNeg extends BaseNeg with Dble {
  type exp <: Exp
  
  trait Exp extends super[BaseNeg].Exp with super[Dble].Exp

  trait NegBehavior {
    self: BaseNeg =>
    def dble = Neg(term.dble)
  }

  def Neg(t: exp): exp

  class Neg(t: exp) extends super.Neg(t) with NegBehavior with Exp
}//78 lines up to here, not couting new lines and '}'


//glue code: 27 lines
object All0 extends 
Eval with Show with Dble 
with EvalNum with EvalPlus with EvalNeg 
with ShowNum with ShowPlus with ShowNeg 
with DbleNum with DblePlus with DbleNeg 
    {
  override type exp = Exp
  
  trait Exp extends 
  super[Eval].Exp with super[Show].Exp with super[Dble].Exp
  with super[DbleNum].Exp with super[DblePlus].Exp with super[DbleNeg].Exp 

  trait NumBehavior extends
  super[EvalNum].NumBehavior with super[ShowNum].NumBehavior
  with super[DbleNum].NumBehavior{
    self: BaseNum =>
  }

  class Num(v: Int) extends BaseNum(v) with NumBehavior with Exp

  trait PlusBehavior extends
  super[EvalPlus].PlusBehavior with super[ShowPlus].PlusBehavior
  with super[DblePlus].PlusBehavior {
    self: BasePlus =>
  }

   class Plus(l: exp, r: exp) extends BasePlus(l, r) with PlusBehavior with Exp

  trait NegBehavior extends
      super[EvalNeg].NegBehavior with super[ShowNeg].NegBehavior
      with super[DbleNeg].NegBehavior {
    self: BaseNeg =>
  }

  class Neg(t: exp) extends BaseNeg(t) with NegBehavior with Exp
  
  def Num(v: Int) = new Num(v)
  
  def Plus(l: exp, r: exp) =new Plus(l,r)

  def Neg(t: exp) = new Neg(t)
}
//----------------------EQUALS
trait Equals extends Base {
  type exp <: Exp;

  trait Exp extends super.Exp {
    def eql(other: exp): Boolean;
  }
}
//----------------------EQUALSNUM
trait EqualsNum extends BaseNum with Equals {
  type exp <: Exp;

  trait Exp extends super.Exp {
    def isNum(v: Int): Boolean = false;
  }

  trait NumBehavior extends Exp {
    self: BaseNum =>
    def eql(other: exp): Boolean = other.isNum(value);

    override def isNum(v: Int) = v == value;
  }

  class Num(v: Int) extends BaseNum(v) with NumBehavior with Exp
}
//----------------------EQUALSPLUS
trait EqualsPlus extends BasePlus with Equals {
  type exp <: Exp;

  trait Exp extends super[BasePlus].Exp with super[Equals].Exp {
    def isPlus(l: exp, r: exp): Boolean = false;
  }

  trait PlusBehavior extends Exp {
    self: BasePlus =>
    def eql(other: exp): Boolean = other.isPlus(left, right);

    override def isPlus(l: exp, r: exp) = (left eql l) && (right eql r)
  }

  class Plus(l: exp, r: exp) extends BasePlus(l, r) with PlusBehavior with Exp
}
//----------------------EQUALSNEG
trait EqualsNeg extends BaseNeg with Equals {
  type exp <: Exp;

  trait Exp extends super[BaseNeg].Exp with super[Equals].Exp {
    def isNeg(t: exp): Boolean = false;
  }

  trait NegBehavior extends Exp {
    self: BaseNeg =>
    def eql(other: exp): Boolean = other.isNeg(term);

    override def isNeg(t: exp) = term eql t
  }

  class Neg(t: exp) extends BaseNeg(t) with NegBehavior with Exp
}
//31 lines for equals, 29 glue
object All extends 
Eval with Show with Dble with Equals 
with EvalNum with EvalPlus with EvalNeg 
with ShowNum with ShowPlus with ShowNeg 
with DbleNum with DblePlus with DbleNeg 
with EqualsNum with EqualsPlus with EqualsNeg 
    {
  override type exp = Exp
  
  trait Exp extends 
  super[Eval].Exp with super[Show].Exp with super[Dble].Exp with super[Equals].Exp
  with super[DbleNum].Exp with super[DblePlus].Exp with super[DbleNeg].Exp
  with super[EqualsNum].Exp with super[EqualsPlus].Exp with super[EqualsNeg].Exp

  trait NumBehavior extends
  super[EvalNum].NumBehavior with super[ShowNum].NumBehavior
  with super[DbleNum].NumBehavior with super[EqualsNum].NumBehavior {
    self: BaseNum =>
  }

  class Num(v: Int) extends BaseNum(v) with NumBehavior with Exp

  trait PlusBehavior extends
  super[EvalPlus].PlusBehavior with super[ShowPlus].PlusBehavior
  with super[DblePlus].PlusBehavior with super[EqualsPlus].PlusBehavior {
    self: BasePlus =>
  }

   class Plus(l: exp, r: exp) extends BasePlus(l, r) with PlusBehavior with Exp

  trait NegBehavior extends
      super[EvalNeg].NegBehavior with super[ShowNeg].NegBehavior
      with super[DbleNeg].NegBehavior with super[EqualsNeg].NegBehavior {
    self: BaseNeg =>
  }

  class Neg(t: exp) extends BaseNeg(t) with NegBehavior with Exp
  
  def Num(v: Int) = new Num(v)
  
  def Plus(l: exp, r: exp) =new Plus(l,r)

  def Neg(t: exp) = new Neg(t)
}
\end{lstlisting}


Finally, here is the 42 code solving the expression problem.
The classes \Q@RenNum@,\Q@RenPlus@ and \Q@RenNeg@
can be understood as declarations of `short-cuts'
in order to make the full 42 language more in line with the
(more compact) syntax of \name.
In 42, these `short-cuts' can be very expressive 
but here we use them only to better align full 42 and \name.
\begin{lstlisting}[basicstyle=\tiny, mathescape=false]]
RenNum:{class method Library<>< (Library that) exception Guard
  Refactor2.rename(path:\"T" into:\"Num")<><that
  }

RenPlus:{class method Library<>< (Library that) exception Guard
  Refactor2.rename(path:\"T" into:\"Plus")<><that
  }

RenNeg:{class method Library<>< (Library that) exception Guard
  Refactor2.rename(path:\"T" into:\"Neg")<><that
  }
//above, header not counted

$exp:Resource<><{ //3 lines
  Exp:{interface}

  T:{implements Exp}
  }

$num:Resource<><Use[RenNum<><$exp()]<><{ //4 lines
  Num:{
    method Size value()
    class method Num of(Size value)
    }
  }

$plus:Resource<><Use[RenPlus<><$exp()]<><{  //6 lines
  Exp:{interface}

  Plus:{
    method Exp left()
    method Exp right()
    class method Plus of(Exp left, Exp right)
    }
  }

$neg:Resource<><Use[RenNeg<><$exp()]<><{  //5 lines
  Exp:{interface}

  Neg:{
    method Exp term()
    class method Neg of(Exp term)
  }
}

$eval:Resource<>< {  //4 lines
  Exp:{interface
    method Size eval()
    }

  T:{implements Exp}
  }

$evalNum:Resource<><Use[$num(); RenNum<><$eval()]<>< {  // 2*3 lines
  Num:{ method Size eval() this.value() }
  }

$evalPlus:Resource<><Use[$plus(); RenPlus<><$eval()]<>< {
  Plus:{ method Size eval() this.left().eval()+this.right().eval() }
  }

$evalNeg:Resource<><Use[$neg(); RenNeg<><$eval()]<>< {
  Neg:{ method Size eval() Size"-1" * this.term().eval() }
  }

$show:Resource<><Use[$exp()]<><{  //4 lines
  Exp:{interface
    method S show()
    }

  T:{implements Exp}
  }

$showNum:Resource<><Use[$num(); RenNum<><$show()]<><{  //2*3 lines
  Num:{ method S show() this.value().toS() }
  }

$showPlus:Resource<><Use[$plus(); RenPlus<><$show()]<><{
  Plus:{ method S show() this.left().show()++S" + "++this.right().show() }
  }

$showNeg:Resource<><Use[$neg(); RenNeg<><$show()]<><{
  Neg:{ method S show() S"-("++this.term().show()++S")" }
  }
//----------------
$double:Resource<><{  //4 lines
  Exp:{interface
    method Exp double()
    }

  T:{implements Exp}
  }

$doubleNum:Resource<><Use[$num(); RenNum<><$double()]<><{ //2*3 lines
  Exp:{interface}//not counted, needed in the full language only to guide scope resolution in the desugaring

  Num:{ method Exp double() Num.of(value: this.value()*2Size) }
  }

$doublePlus:Resource<><Use[$plus(); RenPlus<><$double()]<><{
  Exp:{interface}

  Plus:{ method Exp double() Plus.of(left: this.left().double(), right: this.right().double()) }
  }

$doubleNeg:Resource<><Use[$neg(); RenNeg<><$double()]<><{
  Exp:{interface}

  Neg:{ method Exp double() Neg.of(term: this.term().double()) }
  }
//---------
$equals:Resource<><Use[$eval()]<><{  //6 lines
  Exp:{interface
    method Bool eql(Exp that)
    method Bool equalToT(T that)
    }

  T:{implements Exp
    method eql(that) that.equalToT(this)
    }
  }

$equalsNum:Resource<><Use[$num();  //5*3 lines
  Refactor2.Method[rename:\"equalToT(that)" of:\"Exp" into:\"equalToNum(that)"]
  <><RenNum<>< $equals()]<><{
  Num:{
    method Bool equalToNum(Num that) this.value()==that.value()
    }
  }

$equalsPlus:Resource<><Use[$plus();
  Refactor2.Method[rename:\"equalToT(that)" of:\"Exp" into:\"equalToPlus(that)"]
  <><RenPlus<><$equals()]<><{
  Plus:{
    method Bool equalToPlus(Plus that)
      this.left().eql(that.left()) & this.right().eql(that.right())
    }
  }

$equalsNeg:Resource<><Use[$neg();
  Refactor2.Method[rename:\"equalToT(that)" of:\"Exp" into:\"equalToNeg(that)"]
  <><RenNeg<><$equals()]<><{
  Neg:{
    method Bool equalToNeg(Neg that) this.term().eql(that.term())
    }
  }

 //16+6 lines glue code
$equalsAll:Resource<><Use[$equalsNeg();$equalsPlus();$equalsNum()]<><{
  Exp:{interface
    method Bool eql(Exp that)
    method Bool equalToNum(Num that)
    method Bool equalToPlus(Plus that)
    method Bool equalToNeg(Neg that)
    method Size eval()
    }

  Num:{implements Exp
    method equalToPlus(that) Bool.false()
    method equalToNeg(that) Bool.false()
    }

  Plus:{implements Exp
    method equalToNum(that) Bool.false()
    method equalToNeg(that) Bool.false()
    }

  Neg:{implements Exp
    method equalToNum(that) Bool.false()
    method equalToPlus(that) Bool.false()
    }
  }

$evalAll:Resource<><Use[$evalPlus();$evalNeg()]<><$evalNum()

$showAll:Resource<><Use[$showNum();$showPlus()]<><$showNeg()

$doubleAll:Resource<><Use[$doubleNum();$doublePlus()]<><$doubleNeg()

ESDAll0:Use[Use[$evalPlus();$evalNeg()]<><$evalNum();
  Use[$showNum();$showPlus()]<><$showNeg();
  $doubleNum();$doublePlus()]<><$doubleNeg()


ESDAll:Use[$evalAll();$showAll()]<><$doubleAll()

ESDEAll:Use[$evalAll();$showAll();$doubleAll()]<><$equalsAll()

$equals2:Resource<><{ //7 lines
  Exp:{interface method Bool equals(Exp that)}
  T:{implements Exp
    method Bool exactEquals(T that)
    method equals(that){
      with that (on T  return this.exactEquals(that) )
      return Bool.false()
      }
    }
  }

$equalsNum2:Resource<><Use[$num();RenNum<><$equals2()]<><{  //2*3 lines
  Num:{ method Bool exactEquals(Num that) this.value().equals(that.value()) }
  }

$equalsPlus2:Resource<><Use[$plus();RenPlus<><$equals2()]<><{
  Plus:{ method Bool exactEquals(Plus that) this.left().equals(that.left()) & this.right().equals(that.right()) }
  }

$equalsNeg2:Resource<><Use[$neg();RenNeg<><$equals2()]<><{
  Neg:{method Bool exactEquals(Neg that) this.term().equals(that.term()) }
  }

$equalsAll2:Resource<><Use[$equalsNum2();$equalsPlus2()]<><$equalsNeg2()

ESDEAll2:Use[$evalAll();$showAll();$doubleAll()]<><$equalsAll2()

\end{lstlisting}
