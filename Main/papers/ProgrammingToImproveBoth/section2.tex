
\section{The Design of \name: Separating Use and Reuse}\label{sec:separate}
%
%This section presents an overview of \name and illustrates the
%key ideas of its design. In particular we illustrate %how to separate code use and 
%code reuse, and how \name solves the this-leaking problem.

%\subsection{The Design of \name: Decoupling Use from Reuse}

\subsection{Classes in \name: a mechanism for code use}

%\name has a substantially different design from Java-like languages.
Consider the example of%
%Section~\ref{sec:intro}
%~\hyperref[sec:intro]{Section~\ref*{sec:intro}}
~\autoref{sec:intro}
rewritten in \name, introducing classes \Q@Utils@ and \Q@A@:
\begin{lstlisting}
 A={ method int ma(){ return Utils.m(this); } }
 Utils={ static method int m(A a){ return ..; } }
\end{lstlisting} 
Classes in \name use a different declaration style compared
to Java: there is no \lstinline{class} keyword, and an equals sign separates the class name (which must always start with
an uppercase letter) and the class implementation, which is used to specify the
definitions of the class. In our example, in the class declaration
for \lstinline{A}, the name of the class is \lstinline{A} and the code 
literal associated with the class is `\Q@{ method int ma(){return Utils.m(this);}}@' and it contains the method \Q@ma()@.
%We will see next some 
%important differences to Java-like languages in the way 
%classes and code-literals are type-checked, as we shall see next. 
%Nevertheless, for this example, things still work in a similar way to Java. 
In the \name code above, there is no way to add a class 
\Q@B@ reusing the code of \Q@A@: class \Q@A@ (uppercase) is designed for code \emph{use} and not \emph{reuse}.
Indeed, a noticeable difference with Java is the
absence of the \Q@extends@ keyword.
\name classes are roughly equivalent to final classes in Java. This means that there is actually no subclassing.
Thus, unlike the Java code, introducing a subclass
\lstinline{B} is not possible. This may seem like a severe restriction, but
\name has a different mechanism for \emph{code-reuse} that 
is more appropriate when \emph{code-reuse} is intended. 


\subsection{Traits in \name: a mechanism for code reuse}

Traits in \name cannot be instantiated and do not introduce new
types. However they provide code reuse.
%So, lets try again encoding the code for the leaking problem, but this
%time aiming at code reuse. 
Trait declarations 
look very much like class declarations, but trait names 
start with a lowercase letter (even syntactically they can not be used as types).
\begin{lstlisting}
 Utils={ static method int m(A a){return ...} }
 ta={ method int ma(){return Utils.m(this);}} //type error
 A=Use ta
\end{lstlisting}
Here \lstinline{ta} is a trait intended to replace the
original class \lstinline{A} so that the code of the method
\lstinline{ma} can be reused. Then the class \lstinline{A} 
is created by reusing the code from the trait \Q@ta@, introduced by the keyword 
\lstinline{Use}. Note that \use\ expressions cannot contain class names: only trait
names are allowed.
\emph{Referring to a trait is the only way to induce code reuse}.

The crucial point is the call \Q@Utils.m(this)@ inside trait \Q@ta@:
the corresponding call in the Java code is correct since Java guarantees that such occurrence of \Q@this@ will be a subtype of \Q@A@ everywhere it is reused.
In \name 
the type of \lstinline{this} in
\Q@ta@ has no relationship to the type \lstinline{A};
thus the code \Q@Utils.m(this)@ is illtyped.

The following second attempt would not work either:
\begin{lstlisting}
 Utils={ static method int m(ta a){return ...}} //syntax error
 ta={ method int ma(){return Utils.m(this);}}
 A=Use ta
\end{lstlisting}
\Q@ta@ is not a type in the first place, since it is a (lowercase) trait name.
Indeed, trait names can only be used in \use\ expressions, and thus they can not appear in method bodies or type annotations.
In this way, the code of a trait can stay agnostic of its name. This is one of the key design decisions in \name:
traits can be reused in multiple places, and their code can be seen under multiple types.
In \name, \emph{interfaces are the only way to obtain subtyping}. As shown in the code below, interfaces are special kinds of code literals, where all the methods are abstract.
%In \name
%subtyping is the way to reason about commonalities between different types.
Thus, to model the original Java example, we need an interface
capturing the commonalities between \Q@A@ and \Q@B@:
\begin{lstlisting}
 IA={interface method int ma()} //interface with abstract method
 Utils={static method int m(IA a){return ...} }
 ta={implements IA //This line is the core of the solution
     method int ma(){return Utils.m(this);}}
 A=Use ta
\end{lstlisting}
This code works: \Q@Utils@ relies on interface \Q@IA@ and the trait \Q@ta@
implements \Q@IA@.
%\Q@ta@ is well typed:
Any class reusing \Q@ta@ will contain the code of \Q@ta@,
including the \Q@implements IA@ subtyping declaration; thus any class reusing \Q@ta@ will be a subtype of \Q@IA@. 
Therefore, while typechecking \Q@Utils.m(this)@ we can assume
\Q@this<:IA@.
 It is also possible to add a class \Q@B@ as follows:
\begin{lstlisting}
  B=Use ta, { method int mb(){return this.ma();} }
\end{lstlisting}
This also works.  \Q@B@ reuses the code of \Q@ta@, but has no knowledge of \Q@A@.
Since \Q@B@ reuses \Q@ta@, and \Q@ta@ implements \Q@IA@, \Q@B@ also implements \Q@IA@. 

Later, in \autoref{sec:formal} we will provide the type
system. 
Here notice that the former declaration of \Q@B@ is correct even if
no method called \Q@ma@ is explicitly declared.
DJ and TR would instead require explicitly declaring an abstract \Q@ma@ method:
\begin{lstlisting}
  B=Use ta, { method int ma() //not required by us
      method int mb(){return this.ma();} }
\end{lstlisting}
In \name, methods are directly accessible from \Q@ta@, exactly as in the Java equivalent
\begin{lstlisting}[language=Java]
  class B extends A{ int mb(){return this.ma();} }  
\end{lstlisting}
where method \Q@ma@ is imported from \Q@A@.
This concept is natural for a Java programmer, but was not supported
in previous work \cite{BETTINI2013521,deep}. Those works require all
dependencies in code literals to be explicitly declared, so that the
code literal is self-contained;
in this way it can be typed in isolation before flattening.
However, this results in
many redundant abstract method declarations.

\paragraph{Semantics of Use:}
The semantics of traits is defined with flattening, which is 
simple to formalize and understand.
However, if implemented naively, flattening may cause a lot of
bytecode duplication.
The `delegation semantics'%
~\cite{lagorio2009featherweight},
is a proposed alternative semantic model for traits
that is observationally equivalent to flattening but does not require
bytecode duplication. The formalism presented here will rely on simple
flattening, but we expect the techniques of
\cite{lagorio2009featherweight} would be useful to produce an
efficient implementation in term of bytecode space. 
With the flattening semantics
 \Q@A@ and \Q@B@ are equivalent to the inlined code of all used traits{.}%
\begin{lstlisting}
A=Use ta
B=Use ta, { method int mb(){return this.ma();} }
//equivalent to
A={implements IA method int ma(){return Utils.m(this);}}
B={implements IA
  method int ma(){return Utils.m(this);}
  method int mb(){return this.ma();} } 
 \end{lstlisting}
\emph{In the resulting code, there is no mention of the trait
 \Q@ta@}. Information about code-reuse/inheritance
  is a private implementation detail of \Q@A@
 and \Q@B@; while subtyping is part of the class interface.%
This position has been defended by 
Bracha~\cite{bracha1992programming}:
the choice of inheriting behaviour %from some source 
should be in the hands of the programmer; if a method implementation is not appropriate, such method can be
overridden. If too many methods do not provide an appropriate behaviour, inheriting code from another location or implementing the behaviour from scratch may also be considered.
This should not impact the interface exposed to the user, otherwise the programmer may be unable to change their implementation decisions at a later time.
%
In summary, to leak \Q@this@ in \name, either code reuse is disallowed, or an appropriate interface (\Q@IA@ in this case) must be implemented.
We believe the code with \Q@IA@ better transmits programmer intention. Some
readers may instead see requiring \Q@IA@ as a cost of our approach.
Even from this point of view, such cost is counter balanced by 
the very natural and simple support for code reuse, `\Q@This@' type and (in the extensions with nested classes seen later)
family polymorphism.
The syntactic cost of introducing new names can be reduced
with some syntactic sugar. %%, as explained in Appendix B.1.


%To finish this section, \autoref{fig:compare} provides a summary of
%the differences between classes and traits. The comparison focus on
%the roles of traits and classes with respect to instantiation,
%reusability and whether the declarations also introduce new types or
%not.




