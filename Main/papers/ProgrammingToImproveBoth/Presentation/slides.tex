
\begin{frame}[fragile]
\frametitle{Subtyping and Subclassing}

\begin{itemize}
\item Subtyping without subclassing;
easy: Java interfaces

\item Subclassing without subtyping;
hard:
\begin{lstlisting}[language=Java]
  class A{ int ma(){ return Utils.m(this); } }
  class Utils{ static int m(A a){..} }
  class B extends A{ int mb(){return this.ma();} }
  ..
  new B().mb();
\end{lstlisting}
\end{itemize}


This-leaking problem:
\Q@this@ written in class \Q@A@ is of type \Q@A@.
when \Q@B@ inherits the code of \Q@A@,
code \Q@Uitls.m(this)@ would pass a \Q@this@
as an \Q@A@. To be sound, 
\Q@B@ $\leq$ \Q@A@ must hold. 
\end{frame}



\begin{frame}[fragile]
\frametitle{Reusing code without inducing subtyping}

Three different approaches separating subtyping and subclassing:
\begin{itemize}
\item DeepFJig
\item TraitRecordJ
\item PackageTemplate
\end{itemize}
Widely different approaches,
similar core ideas that we synthesize and improve:
\emph{separating code from use and code for reuse}

\end{frame}


\begin{frame}[fragile]
\frametitle{\name: our approach:}
We synthesize the best from these 3 approaches.

Traits! (inspired from the original approach, not Scala/Rust traits).

\begin{itemize}
\item \textbf{Traits} for Code Reuse: traits can be reused to produce more code.
\item \textbf{Classes} for Code Use: classes can be instantiated.
\item \textbf{Types}: classes and interface names can be used as a types.
\item \textbf{Exact types}: all classes are final; inheritance by reusing traits.
\item \textbf{Sub types}: interfaces are the only way to induce subtyping.
\item \textbf{Not a type}: trait names can not be used as types.
\item[] \Q@This@ can be used to refer to the eventual class of the code.
\item \textbf{State}: No explicit constructors/fields; state is implicitly defined.
\item \textbf{Composition}: traits code can be composed to obtain more code.
\end{itemize}


\end{frame}


\begin{frame}[fragile]
\frametitle{This leaking in our approach}

\begin{lstlisting}
IA={interface
  method int ma()
  }
Utils={
  static method int m(IA a){return ...} 
  }
ta={implements IA
  method int ma(){return Utils.m(this);}
  }
A=Use ta
B=Use ta
\end{lstlisting}

\Q@this@ written in trait \Q@ta@ is of type \Q@This@ and \Q@IA@.
when \Q@B@ inherits the code of \Q@ta@,
code \Q@Uitls.m(this)@ would pass a \Q@this@
as an \Q@IA@. In this way, even if
\Q@B@ and \Q@A@ share the same code, there
are not in a subtype relation.
(Note that \Q@ta@ is not a valid typename)
\end{frame}

\begin{frame}[fragile]
\frametitle{Trait composition}
The composed code contains all members from the used traits.
Methods with same name and type signature are summed into a single one.
At most one of those summed methods
can have a body, which will be propagated into the result.

\begin{lstlisting}
t1={
  method String hello();
  method String helloWorld(){return hello()+" World";}
  }

t2={ method String hello(){return "Hi";} }

t3= Use t1,t2
//-- flatten to ---------------------------
t3= {
  method String hello(){return "Hi";}
  method String helloWorld(){return hello()+" World";}
  }
\end{lstlisting}
\end{frame}

\begin{frame}[fragile]
\frametitle{Handling State}
How to make this process work with constructors and fields
while achieving the following goals:
\begin{itemize}
\item managing fields in a way that borrows the elegance of summing methods;
\item actually initializing objects, leaving no \Q@null@ fields;
\item making it easy to add new fields;
\item allowing self instantiation: a trait method can instantiate the class using it.
\end{itemize}

\end{frame}

\begin{frame}[fragile]
\frametitle{Abstract State Operations}
Idea: use abstract methods as getters, setters and factory methods.

Conventionally, an abstract class is a class with some abstract method.

Here, a class whose abstract methods can be seen as state operations is a
concrete coherent class.

For example
Java
\begin{lstlisting}
class A{
  int x;
  A(int x){this.x=x;}
  int x(){return this.x;}
  void x(int that){this.x=that;}
  }
\end{lstlisting}

\name
\begin{lstlisting}
A={
  static method This of(Int x)
  method Int x()
  method Void x(Int that)
  }
\end{lstlisting}
\end{frame}

\begin{frame}[fragile]
\frametitle{Coherent class}
\begin{itemize}
\item A class with no abstract methods is coherent (just like Java
  \Q@Math@, for example). Such classes have no instances and are only useful for calling static methods.
\item A class with a single abstract \Q@static@ method 
returning \Q@This@ and with parameters $T_1\ x_1, \ldots, T_n\ x_n$
is coherent if all the other abstract methods can be seen as \emph{abstract state
operations} over one of $x_1, \ldots, x_n$.
That is:
\begin{itemize}
\item A method $T_i\ x_i$\Q@()@ is interpreted as an abstract state method: a \emph{getter} for $x_i$.
\item A method \Q@void @$x_i$\Q@(@$T_i\ $\Q@ that)@ is a \emph{setter} for $x_i$.
\end{itemize}
\end{itemize}
\end{frame}


\begin{frame}[fragile]
\frametitle{Example: \Q@Point@s with algebraic operations}

static factory method:

\quad\quad \Q@Point p=Point.of(3,4)@

getters:

\quad\quad \Q@p.x()==3@,\ \ \ \   \Q@p.y()==4@

summing points:

\quad\quad \Q@p.sum(Point.of(10,20))@ equivalent to \Q@Point.of(13,24)@
\\*${}_{}$\\*
NOTE: sum operation creates a new point
\\*${}_{}$\\*
If no code reuse is desired, it is easy to use Java to encode such \Q@Point@ class.
However, we would like to define operations \Q@sum@, \Q@mul@, \Q@div@ and \Q@sub@
independently and compose them to create classes with the operations we want,
so that it is easy to have points with \Q@sum@ and \Q@mul@, points with
just \Q@sum@ or points with just \Q@mul@.

For example, we expect \Q@PointSumMul.sum(PointSumMul) : PointSumMul@.

This breaks subtyping: \Q@PointSumMul@ $\not\leq$ \Q@PointSum@.


\end{frame}

\begin{frame}[fragile]
\frametitle{Abstract state}
\begin{lstlisting}
p= { //point trait with abstract state
  method int x() //getter for field x
  method int y() //getter for field y
  static method This of(int x,int y) //factory method
  }

Point= Use p //this is a concrete coherent class
...
.. Point.of(3,7).x() ..//valid code
\end{lstlisting}
\end{frame}


\begin{frame}[fragile]
\frametitle{Abstract state}
\begin{lstlisting}
p={
  method int x()
  method int y()
  static method This of(int x,int y)
  }

pointSum= Use p, {
  method This sum(This that){
    return This.of(this.x()+that.x(),this.y()+that.y());
    }}

pointMul= Use p, {
  method This mul(This that){
    return This.of(this.x()*that.x(),this.y()*that.y());
    }}

pointDiv= ...

MyPoint= Use pointSum,pointMul,pointDiv,...

\end{lstlisting}
\end{frame}


\begin{frame}[fragile]
\frametitle{Case study 1: \Q@Point@s with algebraic operations}
Encoding the point example above, with the 4 arithmetic operations,
and instantiating all the 16 possible permutations as separate classes:
\begin{center}
\begin{tabular}{@{} l l l l @{}}
%\toprule
Language       & Lines of code & members & classes/traits\\
%\rule%\midrule
Java7           &  $115$        & $50$                &      $16$\\
Classless Java &   $82$          & $34$                &      $16$\\
Scala          &   $81$  &  $40$                 &    $21$\\
\name          &   $32$ & $7$                 &      $21$\\
%\bottomrule
\end{tabular}
\end{center}

\end{frame}

\begin{frame}[fragile]
\frametitle{State extensibility}
\vspace{-2ex}
%flavored= { method Flavor flavor()}
\begin{lstlisting}
p={
  method int x()
  method int y()
  static method This of(int x,int y)
  }
pointSum= Use p, {
  method This sum(This that){
    return This.of(this.x()+that.x(),this.y()+that.y());
    }}

colored= { method Color color() }

CPoint= Use pointSum,colored,{
  static method This of(int x,int y){
    return This.of(x,y,Color.of(/*red*/));
    }
  static method This of(int x,int y,Color color) 
  }
\end{lstlisting}

Done in this way, \Q@CPoint.sum@ would return red points.
\end{frame}

\begin{frame}[fragile]
\frametitle{State extensibility, with withers}
\vspace{-1ex}
A wither is like a field setter, but creates a new copy of the object
\vspace{-1ex}
 \begin{lstlisting}
p= { 
  method int x()
  method int y()
  method This withX(int that)
  method This withY(int that)
  method This merge(This that)
  }
pointSum= Use p, { 
  method This sum(This that){
    return this.merge(that)
      .withX(this.x()+that.x()).withY(this.y()+that.y());
  }}
colored= {
  method Color color()
  method This withColor(Color that)
  method This merge(This that){
    return this.withColor(this.color().mix(that.color());
  }}
CPoint= Use pointSum,colored,{
  static method This of(int x, int y, Color color)}
\end{lstlisting}  
\end{frame}





\begin{frame}[fragile]
\frametitle{Independent extensibility}

 \begin{lstlisting}
flavored= {
  method Flavor flavor()
  method This withFlavor(Flavor that)
  method This merge(This that){
    return this.withFlavor(that.flavor());
  }}
FCPoint= Use //aliasing conflicting implementations
  colored[super merge as m1],
  flavored[super merge as m2],
  pointSum,{
    method This merge(This that){
      return this.m1(that).m2(that);
    }
    static method This of(
      int x, int y, Color color, Flavor flavor)
    }
\end{lstlisting} 
\end{frame}

\begin{frame}[fragile]
\frametitle{More in the paper}

\begin{itemize}
\item Transparent handling of nested classes
\item \Q@This@ type generalized for family polymorphism
\item A very natural encoding of the Expression Problem 
\item Simple formalization of our language
\end{itemize}

\end{frame}

\begin{frame}[fragile]
\frametitle{Thanks}
${}_{}$\\*
${}_{}$\\*
${}_{}$\\*
${}_{}$\\*
Questions?
\end{frame}

%\begin{frame}[fragile]
%\frametitle{This leaking in our approach}
%\end{frame}

%\begin{frame}[fragile]
%\frametitle{This leaking in our approach}
%\end{frame}
