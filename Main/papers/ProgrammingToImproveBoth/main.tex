\documentclass[english,submission,code=tt]{programming}
\makeatletter
\DeclareOldFontCommand{\rm}{\normalfont\rmfamily}{\mathrm}
\DeclareOldFontCommand{\sf}{\normalfont\sffamily}{\mathsf}
\DeclareOldFontCommand{\tt}{\normalfont\ttfamily}{\mathtt}
\DeclareOldFontCommand{\bf}{\normalfont\bfseries}{\mathbf}
\DeclareOldFontCommand{\it}{\normalfont\itshape}{\mathit}
\DeclareOldFontCommand{\sl}{\normalfont\slshape}{\@nomath\sl}
\DeclareOldFontCommand{\sc}{\normalfont\scshape}{\@nomath\sc}
\makeatother
\usepackage[backend=biber]{biblatex} % Use Biblatex
\addbibresource{main.bib}
\addbibresource{main.bib}
\usepackage{mathpartir}
\usepackage{amsmath}
\usepackage{amsthm}
\theoremstyle{plain}
\newcounter{definition}
\newtheorem{Definition}[definition]{Definition}
\newcounter{assumption}
\newtheorem{Assumption}[assumption]{Assumption}
\newcounter{lemma}
\newtheorem{Lemma}[lemma]{Lemma}
%\usepackage{listings}
\usepackage{xcolor}
\usepackage{letltxmacro}
\usepackage{mathtools}
\usepackage{mathpartir}
%\usepackage{stix}

\definecolor{darkRed}{RGB}{100,0,10}
\definecolor{darkBlue}{RGB}{10,0,100}
\newcommand*{\ttfamilywithbold}{\fontfamily{pcr}\selectfont}
%\newcommand*{\ttfamilywithbold}{\ttfamily}

%found on http://tex.stackexchange.com/questions/4198/emphasize-word-beginning-with-uppercase-letters-in-code-with-lstlisting-package
%\lstset{language=FortyTwo,identifierstyle=\idstyle}
%
\makeatletter
\newcommand*\idstyle{%
        \expandafter\id@style\the\lst@token\relax
}
\def\id@style#1#2\relax{%
        \ifcat#1\relax\else
                \ifnum`#1=\uccode`#1%
                        \ttfamilywithbold\bfseries
                \fi
        \fi
}
\makeatother

\lstset{language=Java,
  basicstyle=\upshape\ttfamily\footnotesize,%\small,%\scriptsize,
  keywordstyle=\upshape\bfseries\color{darkRed},
  showstringspaces=false,
  mathescape=true,
  xleftmargin=0pt,
  xrightmargin=0pt,
  breaklines=false,
  breakatwhitespace=false,
  breakautoindent=false,
 identifierstyle=\idstyle,
 morekeywords={method,Use,This,constructor,as,into,rename},
 deletekeywords={double}
}

\newcommand*{\SavedLstInline}{}
\LetLtxMacro\SavedLstInline\lstinline
\DeclareRobustCommand*{\lstinline}{%
	\ifmmode
	\let\SavedBGroup\bgroup
	\def\bgroup{%
		\let\bgroup\SavedBGroup
		\hbox\bgroup
	}%
	\fi
	\SavedLstInline
}

\newcommand\saveSpace{\vspace{-2pt}}

\newcommand\Rotated[1]{\begin{turn}{90}\begin{minipage}{12em}#1\end{minipage}\end{turn}}

\newcommand{\Q}{\lstinline}
\newenvironment{bnf}{$\begin{array}{lcll}}{\end{array}$}
\newcommand{\production}[3]{%
	\text{\itshape #1}&%
	\!\!\!\!\!\Coloneqq\!\!\!\!\!&%
	\text{\itshape #2}&%
	\!\!\!\!\!\mbox{#3}}
%\newcommand{\prodFull}[3]{#1&::=&\mbox{#2}&\mbox{#3}}
\newcommand{\prodInline}[2]{#1\Coloneqq#2}
\newcommand{\prodNextLine}[2]{&&#1&\mbox{#2}}
\newcommand{\terminal}[1]{\ensuremath{$\texttt{#1}$}}
%\newcommand{\metavariable}[1]{\ensuremath{\mathit{#1}}}

\newcommand\Rulename[1]{{\textsc{#1}}}
\newcommand\ctx[1]{\ensuremath{\mathcal{E}_#1}\!}
\newcommand{\lib}[3]{\Q@\{@\!#1\Q{;}\ #2 \Q{;}\ #3\Q@\}@}
\newcommand{\rp}[1]{\Q{(}\!#1\Q{)}}
\newcommand{\red}[3]{#1\rp{#2\Q{=}#3}}
\newcommand{\summ}[2]{#1\ \Q{<+}\ #2}
\newcommand{\mmid}{\ensuremath{\mid}}
\newcommand{\hole}{\ensuremath{\square}}
%--------------------------
\newcommand{\mynotes}[3]{{\color{#2} {\sc #1}: #3}}
\newcommand\isaac[1]{\mynotes{Isaac}{red}{#1}}
\newcommand\marco[1]{\mynotes{Marco}{blue}{#1}}



%\newenvironment{listing}{\vspace{-3pt}\begin{lstlisting}}{\end{lstlisting}\vspace{-3pt}}
%
\lstset{language=Java,
  basicstyle=\ttfamily\footnotesize,%\small,%\scriptsize,
%  keywordstyle=\bfseries,%\color{darkRed},
%  showstringspaces=false,
  mathescape=true,
%  xleftmargin=0pt,
%  xrightmargin=0pt,
%  breaklines=false,
%  breakatwhitespace=false,
%  breakautoindent=false,
%  linewidth=1.005\textwidth,% should be enough
%  identifierstyle=\idstyle
 morekeywords={method,Use,This,constructor,as,into,rename},
 deletekeywords={double}
}
\newcommand\Q\lstinline

\newcommand\Rotated[1]{\begin{turn}{90}\begin{minipage}{9em}#1\end{minipage}\end{turn}}

\newcommand\Opt[1]{#1 ?}

\newenvironment{bnf}{$\begin{array}{lcll}}{\end{array}$}
\newcommand{\prodFull}[3]{#1&::=&#2&\mbox{#3}}
%\newcommand{\prodFull}[3]{#1&::=&\mbox{#2}&\mbox{#3}}
\newcommand{\prodInline}[2]{#1::=#2}
\newcommand{\prodNextLine}[2]{&&#1&\mbox{#2}}
\newcommand{\terminal}[1]{%
\ensuremath{$\texttt{#1}$}%
}

\newcommand{\terminalCode}[1]{\mbox{\lstinline{#1}}}
\newcommand{\metavariable}[1]{%
\ensuremath{\mathit{#1}}%
}

%-------------------------------- comms--------
\newif\ifsubmit
%\submitfalse
\submittrue
\ifsubmit
\newcommand{\BO}[1]{#1} 
\newcommand{\BOComm}[1]{} 
\newcommand{\MS}[1]{#1} 
\newcommand{\MSComm}[1]{} 
\newcommand{\HA}[1]{#1} 
\newcommand{\HAComm}[1]{} 
\else
\newcommand{\BO}[1]{\textcolor{blue}{#1}} 
\newcommand{\BOComm}[1]{{\scriptsize \textcolor{blue}{[Elena{:} #1]}}}
\newcommand{\MS}[1]{\textcolor{green}{#1}} 
\newcommand{\MSComm}[1]{{\scriptsize \textcolor{green}{[Marco{:} #1]}}}
\newcommand{\HA}[1]{\textcolor{red}{#1}}
\newcommand{\HAComm}[1]{{\scriptsize \textcolor{red}{[Arora{:} #1]}}}
\fi

%----------------------
\newcommand\dom{\text{dom}}
\newcommand*{\Scale}[2][4]{\scalebox{#1}{$#2$}}%
\newcommand\smallDs{{\Scale[0.5]{\overline{\mD}}}}
\newcommand\vds{{v_{\smallDs}}}
\newcommand\Rulename[1]{{\textsc{#1}}}
\newcommand\ctx{{\mathcal{E}}}
\newcommand\mID{\metavariable{ID}}
\newcommand\mL{\metavariable{L}}
\newcommand\mE{\metavariable{E}}
\newcommand\mC{\metavariable{C}}
\newcommand\mT{\metavariable{T}}
\newcommand\mV{\metavariable{V}} %code value t or L
\newcommand\mM{\metavariable{M}}
\newcommand\mG{\Gamma}
\newcommand\mDE{\metavariable{DE}}
\newcommand\mTE{\metavariable{TE}}
\newcommand\mCE{\metavariable{CE}}
\newcommand\mMD{\metavariable{MD}}
\newcommand\mTD{\metavariable{TD}}
\newcommand\mCD{\metavariable{CD}}
\newcommand\mD{\metavariable{D}}
\newcommand\me{\metavariable{e}}
\newcommand\mx{\metavariable{x}}
\newcommand\mm{\metavariable{m}}
\newcommand\mt{\metavariable{t}}
\newcommand\use{\terminalCode{Use}}
\newcommand\oC{\mbox{\lstinline@\{@}}
\newcommand\cC{\mbox{\lstinline@\}@}}
\newcommand\oR{\mbox{\lstinline@(@}}
\newcommand\cR{\mbox{\lstinline@)@}}

\newcommand{\mynotes}[3]{{\color{#2} {\sc #1}: #3}}
\newcommand\bruno[1]{\mynotes{Bruno}{red}{#1}}
\newcommand\marco[1]{\mynotes{Marco}{blue}{#1}}

\newcommand{\syndef}{$::=$}

%\newcommand\name{\textbf{$42_{\mu}$}\xspace}
\newcommand\name{\texorpdfstring{42\textsubscript{\mu}}{42µ}\xspace}


%\lstset{language=FortyTwo, morekeywords={imm,new,class,this,assert}}
\newcommand\saveSpace{\vspace{-3px}}
\newcommand\loseSpace{\vspace{1ex}}
\begin{document}
\paperdetails{
perspective=theoretical,
area={Type systems}
}
%\title{Validation}
\title{Separating Use and Reuse to Improve Both}
%\author{Authors omitted for double-blind review.}

\author{Hrshikesh Arora}
\author{Marco Servetto}
%\affiliation{Victoria University of Wellington}
%\email{marco.servetto@ecs.vuw.ac.nz}
\affiliation{Victoria University of Wellington}
%\email{alex@ecs.vuw.ac.nz}
\author{Bruno C. d. S. Oliveira}
\affiliation{The University of Hong Kong}
\keywords{xxx,xxx,xxx}

\begin{CCSXML}
<ccs2012>
	<concept>
		<concept_id>10003752.10010124.10010138.10010139</concept_id>
		<concept_desc>Theory of computation~Invariants</concept_desc>
		<concept_significance>500</concept_significance>
	</concept>
	<concept>
		<concept_id>10003752.10010124.10010138.10010142</concept_id>
		<concept_desc>Theory of computation~Program verification</concept_desc>
		<concept_significance>500</concept_significance>
	</concept>
	<concept>
		<concept_id>10011007.10011006.10011008.10011009.10011011</concept_id>
		<concept_desc>Software and its engineering~Object oriented languages</concept_desc>
		<concept_significance>500</concept_significance>
	</concept>
	<concept>
		<concept_id>10011007.10010940.10010992.10010998.10011001</concept_id>
		<concept_desc>Software and its engineering~Dynamic analysis</concept_desc>
		<concept_significance>300</concept_significance>
	</concept>
	<concept>
		<concept_id>10011007.10011006.10011008.10011024.10011032</concept_id>
		<concept_desc>Software and its engineering~Constraints</concept_desc>
		<concept_significance>300</concept_significance>
	</concept>
</ccs2012>
\end{CCSXML}

\ccsdesc[500]{Theory of computation~Invariants}
\ccsdesc[500]{Theory of computation~Program verification}
\ccsdesc[500]{Software and its engineering~Object oriented languages}
\ccsdesc[300]{Software and its engineering~Dynamic analysis}
\ccsdesc[300]{Software and its engineering~Constraints}

\maketitle

\begin{abstract}
${}_{}$


\noindent\textit{Context:} % What is the broad context of the work? What is the importance of the general research area?

\loseSpace
\noindent\textit{Inquiry:} %What problem or question does the paper address? How has this problem or question been addressed by others (if at all)?


\loseSpace
\noindent\textit{Approach:} %What was done that unveiled new knowledge?

\loseSpace
\noindent\textit{Knowledge:} %What new facts were uncovered? If the research was not results oriented, what new capabilities are enabled by the work?

\loseSpace
\noindent\textit{Grounding:} %What argument, feasibility proof, artifacts, or results and evaluation support this work?


\loseSpace
\noindent\textit{Importance:} %Why does this work matter?

	
\end{abstract}

\section{Introduction}
\begin{lstlisting}
hi method class Foo{ bar}
\end{lstlisting}

In literature, in addition to conventional Java style F-bound polimorphism, there
Another way to obtain generics is to use associated types (to specify generic paramaters) and inheritence (to instantiate the paramaters).
However, when parametrizing multiple types, the user to specify the full mapping.
For example in Java
    interface A<B>{ B m(); }
    inteface B{String f();}
    class G<TA extends A<TB>, TB>{//TA and TB explicitly listed
      String g(TA a TB b){return a.m().f();}
    }
    class MyA implements A<MyB>{..}
    class MyB implements B {..}
    G<MyA,MyB>//instantiation
Also scala offers genercs, and could encode the example in the same way, but Scala
also offers associated types, allowing to write instead....

Rust also offers generics and associated types, but also support calling static methods
over generic and associated types.

We provide here a fundational model for genericty that subsume the power
of F-bound polimorphims and  associated types.
Moreover, it allows for large sets of generic parameter instantiations to be inferred starting from a much smaller mapping.
For example, in our system we could just write
    g={
      A={ method B m()}
      B={ method String f()}
      method String g(A a B b)=a.m().f()
    }
    MyA={ method MyB m()= new MyB(); ..}
    MyB={ method String f()="Hello"; ..}
    g<A=MyA>//instantiation. The mapping A=MyA,B=MyB

We model a minimal calculus with interfaces and final classes, where implementing an interface is the only way to induce subtyping.
We will show how supporting subtyping constitute the core technical difficulty in our work, inducing ambiguity in the mappings.
As you can see, we base our generic matches the structor of the type instead of respecting a subtype requirement as in F-bound polymorphis.
We can easily encode subtype requirements by using implements:
Print=interface{ method String print();}
g={
  A:{implements Print}
  method A printMe(A a1,A a2){ if(a1.print().size()>a2.print.size()){return a1;} return a2;}
  }
MyPrint={implements Print ..}
g<A=MyPrint> //instantiation
g<A=Print> //works too


--------------
example showing ordering need to strictly improve
EI1: {interface}
EA1: {implements EI1}

EI2: {interface}
EA2: {implements EI2}

EB: {EA1 a1 EA1 a1}

{
A1: {}
A2: {}
B: {A1 a1 A2 a2}
}[B = EB] // A1 -> EI1, A2 -> EA2 a
          // A1 -> EA1, A2 -> EI2 b
          // A1 -> EA1, A2 -> EA2 c

a <=b
b <=a
c<= a,b
a <= c

\Q@hi@
\Q@Hi@
\Q@class@

$ aa \Q@hi@
\Q@Hi@
\Q@class@  qaq$
\begin{bnf}
	\production{a}{b}{c}\\
	\production{a}{b}{c}\\
	\production{a}{b}{c}\\
\end{bnf}

$\Q@}}][()]@$

$\begin{array}{l}
\inferrule[(top)]{
	a \xrightarrow[b]{} c\quad
	\forall i<3 a\vdash b:\text{OK}\\\\
	\forall i<3 a\vdash b:\text{OK}
}{
	1+2
	\rightarrow
	3
}\begin{array}{l}
a\\b\\c
\end{array}
\end{array}$
\saveSpace
\section{The Design of \name: Separating Use and Reuse}\label{sec:separate}
\saveSpace\saveSpace
This section presents an overview of \name and illustrates the
key ideas of its design. In particular we illustrate how to separate code use and 
code reuse, and how \name solves the this-leaking problem.
% in \name. %, and how this allows improving both. 

%\subsection{The Design of \name: Decoupling Use from Reuse}
\saveSpace\saveSpace
\subsection{Classes in \name: A mechanism for code Use}
\saveSpace
%\name has a substantially different design from Java-like languages.
The concept of a class in \name provides a mechanism for code-use
only. This means that there is actually no subclassing:
classes are roughly equivalent to final classes in Java.  Thus,
compared to Java-like languages, the most noticeable difference is the
absence of the \Q@extends@ keyword in \name. 
Consider the example from Section~\ref{sec:intro}, rewritten in \name:
\saveSpace
\begin{lstlisting}
 Utils={ static method int m(A a){return ...} }
 A= { method int ma(){return Utils.m(this);} }
\end{lstlisting} 
\saveSpace
\noindent Classes in \name use a slightly different declaration style compared
to Java: there is no \lstinline{class} keyword, and an equal sign separates the class name (which must always start with
an uppercase letter) and the class implementation, which is used to specify the
definitions of the class. In our example, in the class declaration
for \lstinline{A}, the name of the class is \lstinline{A} and the code 
literal associated with the class is `\Q@{ method int ma(){return Utils.m(this);}}@' and it contains the member  \Q@ma()@.

%We will see next some 
%important differences to Java-like languages in the way 
%classes and code-literals are type-checked, as we shall see next. 
%Nevertheless, for this example, things still work in a similar way to Java. 

In the \name code above there is no way to add a class 
\Q@B@ reusing the code of \Q@A@: class \Q@A@ (uppercase) is designed for code \emph{use} and not \emph{reuse}.
Thus, unlike the Java code, introducing a subclass
\lstinline{B} is not possible. This may seem like a severe restriction, but
\name has a different mechanism for \emph{code-reuse} that 
is more appropriate when \emph{code-reuse} is intended. 

\saveSpace
\subsection{Traits in \name: A mechanism of code Reuse}
\saveSpace
Unlike classes, traits in \name cannot be instantiated and do not introduce new
types. However they provide code reuse.
%So, lets try again encoding the code for the leaking problem, but this
%time aiming at code reuse. 
Trait declarations 
look very much like class declarations, but trait names 
start with a lowercase letter.

\saveSpace
\begin{lstlisting}
 Utils={ static method int m(A a){return ...} }
 ta={ method int ma(){return Utils.m(this);}}//type error
 A=Use ta
\end{lstlisting}
\saveSpace
\noindent Here \lstinline{ta} is a trait intended to replace the
original class \lstinline{A} so that the code of the method
\lstinline{ma} can be reused. Then the class \lstinline{A} 
is created by inheriting the code from the trait using the keyword 
{\bf \lstinline{Use}}. Note that \use\ cannot contain class names: only trait
names are allowed.
That is, \textbf{using a trait is the only way to induce code reuse}.

The crucial point is the call \Q@Utils.m(this)@ inside trait \Q@ta@:
the corresponding call in the Java code is coercing that code (even when reused) to be an \Q@A@.
In \name we do not commit the same mistake, and the former code is ill typed:
the type of \lstinline{this} in
\Q@ta@ has no relationship to the type \lstinline{A}.


Also that the following second attempt would not work:
\saveSpace
\begin{lstlisting}
 Utils={ static method int m(ta a){return ...}//syntax error
 ta={ method int ma(){return Utils.m(this);}}
 A=Use ta
\end{lstlisting}
\saveSpace
\Q@ta@ is not a type in the first place, since it is a (lowercase) trait name.
Indeed since the trait name is not a type, no code external to that trait can
refer to it, and that code can stay agnostic of its name. This is one of the key design decisions in \name:
traits can be reused in multiple places, and their code can be seen under multiple types.
\textbf{Interfaces are the only way to obtain subtyping}; thus
subtyping is the way to reason about commonalities between different types.

Thus, to model our example, we need an interface
capturing the commonalities between \Q@A@ and \Q@B@:
\saveSpace
\begin{lstlisting}
 IA={interface method int ma()}//interface with abstract method
 Utils={static method int m(IA a){return ...} }
 ta={implements IA //This line is the core of the solution
     method int ma(){return Utils.m(this);}}
 A=Use ta
\end{lstlisting}
\saveSpace
This code works: \Q@Utils@ relies on interface \Q@IA@ and the trait \Q@ta@
implements \Q@IA@.
\Q@ta@ is well typed: independently of what class name(s) will be
associated to its code, we know that such class(es) will implement
\Q@IA@. 
Therefore, while typechecking \Q@Utils.m(this)@ we can assume
\Q@this<:IA@.
 It is also possible to add a class \Q@B@ as follows:
\saveSpace
\begin{lstlisting}
  B=Use ta, { method int mb(){return this.ma();} }
\end{lstlisting}
\saveSpace
This also works.  \Q@B@ reuses the code of \Q@ta@, but has no knowledge of \Q@A@.
Since \Q@B@ reuses \Q@ta@, and \Q@ta@ implements \Q@IA@, also \Q@B@ implements \Q@IA@. 

Later, in Section \ref{sec:formal} we will provide the type
system. 
For now notice that in the former example the code is correct even if
no method called \Q@ma@ is explicitly declared.
DJ and TR would require instead to explicitly declare an abstract \Q@ma@ method:
\saveSpace
\begin{lstlisting}
  B=Use ta, { method int ma()//not required by us
      method int mb(){return this.ma();} }
\end{lstlisting}\saveSpace
\noindent
The idea in \name is that such method is imported from trait \Q@ta@, exactly as in the Java equivalent
\saveSpace\begin{lstlisting}[language=Java]
  class B extends A{ int mb(){return this.ma();} }  
\end{lstlisting}
\saveSpace
where method \Q@ma@ is imported from \Q@A@.
This concept is natural for a Java programmer, but was not supported
in previous work \cite{BETTINI2013521,deep}. Those works require all
dependencies in code literals to be explicitly declared, so that the
code literal is completely self-contained. However, this results in
many redundant abstract method declarations.

\paragraph{Semantics of Use}
Albeit alternative semantic models for traits~\cite{lagorio2009featherweight} have been proposed,
here we use the flattening model. This means that 
\saveSpace\begin{lstlisting}
A=Use ta
B=Use ta, { method int mb(){return this.ma();} }
\end{lstlisting}\saveSpace 
\noindent is equivalent to the inlining the code of all used traits:
\saveSpace \begin{lstlisting}
A={implements IA method int ma(){return Utils.m(this);}}
B={implements IA
  method int ma(){return Utils.m(this);}
  method int mb(){return this.ma();} } 
 \end{lstlisting}
\saveSpace 
 This code is correct, and {\bf in the resulting code there is no mention of the trait
 \Q@ta@}. All the information about code-reuse/inheritance
  is just a private implementation detail of \Q@A@
 and \Q@B@; while subtyping is part of the class interface.


Summarizing, \Q@IA@ is required by our example.
We found the code with \Q@IA@ to better transmits the intention of the programmer; but some
reader may see requiring \Q@IA@ as a cost of our approach.
Even from this point of view, such cost is counter balanced by 
very natural and simple support for code reuse, `\Q@This@' type and (in the extensions with nested classes seen later)
family polymorphism.


%To finish this section, Figure \ref{fig:compare} provides a summary of
%the differences between classes and traits. The comparison focus on
%the roles of traits and classes with respect to instantiation,
%reusability and whether the declarations also introduce new types or
%not.





\section{Improving Use}

To illustrate how 
\name improves the use of classes we model a simplified version of
Set and Bag collections first in Java, and then in \name.
The main benefit of the \name solution is that we can get reuse 
without introducing subtyping between Bags and Sets.
As shown below, this \textbf{improves the 
use of Bags} by eliminating logical errors arising from incorrect
subtyping relations that are allowed in the Java solution. 

\subsection{Sets and Bags in Java: the need of code reuse without subtyping}
An iconic example on why connecting inheritance/code reuse and
subtpying is problematic is provided by
LaLonde~\cite{LaLonde:1991:SSS:110673.110679}.  A reasonable
implementation for a \Q@Set@ is easy to extend into a \Q@Bag@ by
keeping track of how many times an element occurs.  We just need to
add some state and override a few methods.
For example in Java one could have:

\begin{lstlisting}
class Set {...//usual hashmap implementation
  private Elem[] hashMap;
  void put(Elem e) /*body*/
  boolean isIn(Elem e)/*body*/}
class Bag extends Set{...//for each element in the hash map,
  private int[] countMap;// keep track of how many times is there
  @Override void put(Elem e)/*body*/
  int howManyTimes(Elem e)/*body*/}
\end{lstlisting}

\noindent Coding \Q@Bag@ in this way avoids a lot of code
duplication, but \textbf{we induced unintended subtyping}! 
Since subclassing implies subtyping, our code breaks the Liskov substitution principle (LSP)~\cite{martin2000design}: not all bags are sets!
Indeed, the following is allowed:

\begin{lstlisting}
Set mySet=new Bag(); //OK for the type system but not for LSP
\end{lstlisting}
This encumbers the programmer:
to avoid conceptual errors that are not captured by the type system, 
they have to \textbf{use} \Q@Bag@ very carefully.


\paragraph{A (broken) attempt to fix the Problem in Java.}
One could \textbf{retroactively} fix this problem by introducing \Q@AbstractSetOrBag@
and making both \Q@Bag@ and \Q@Set@ inherit from it:
\begin{lstlisting}
abstract class AbstractSetOrBag {/*old set code goes here*/}
class Set extends AbstractSetOrBag {} //empty body
class Bag extends AbstractSetOrBag {/*old bag code goes here*/}
...
//AbstractSetOrBag type not designed to be used.
AbstractSetOrBag unexpected=new Bag(); 
\end{lstlisting}

This looks unnatural, since \Q@Set@ would extend \Q@AbstractSetOrBag@ without adding anything,
and we would be surprised to find a use of the type \Q@AbstractSetOrBag@.
Worst, if we are to constantly apply this mentally, we would introduce a very high number
of abstract classes that are not supposed to be used as types. Those classes would clutter the 
public interface of our classes and the code project as a whole.
A \textbf{use}able API should only provide the relevant information to the client.
In our example the information \Q@Set<:AbstractSetOrBag@ would be present in the public interface
of the class \Q@Set@, but such information is not needed to use the class properly!

Moreover, the original problem is not really solved, but only moved 
further away. For example, one day  we may need bags that can only store up to 5 copies of the same element.
We are now at the starting point again:
\begin{itemize}
\item either we insert \Q@class Bag5 extends Bag@ and we break the LSP; 
\item or we duplicate the code of the \Q@Bag@ implementation with minimal
  adjustments in \\* \Q@class Bag5 extends AbstractSetOrBag@;
\item or we introduce an
\Q@abstract class BagN extends AbstractSetOrBag@ and \\*\Q@class Bag5 extends BagN@
and we modify \Q@Bag@ so that  \Q@class Bag extends BagN@.
Note that this last solution is changing the public interface of the formerly released \Q@Bag@ class, and
this may even break retro-compatibility (if a client program was using
reflection, for example).
\end{itemize}

\subsection{Sets and Bags in \name}
Instead, in \name, if we were to originally declare
\saveSpace\begin{lstlisting}
Set={/*set implementation*/} 
\end{lstlisting}\saveSpace
Then our code would be impossible to reuse in the first place for any user of our library.
We consider this an advantage, since unintended code reuse runs into under-documented behaviour nearly all the time\footnote{See
``Design and document for inheritance or else prohibit
it''~\cite{Bloch08}: the
self use of public methods is rarely documented, thus is hard to understand the effects of overriding a library method.
}!
If the designer of the \Q@Set@ class wishes to make it reusable, they can do it explicitly by providing a set trait:
\saveSpace\begin{lstlisting}
set={/*set implementation*/} 
Set=Use set
\end{lstlisting}\saveSpace
Since \Q@set@ can never be used as a type, there is no reason to give it a {\bf fancy-future-aware} name like
\Q@AbstractSetOrBag@.
When bag will be added, the code will look either like
\saveSpace\begin{lstlisting}
set={/*set implementation*/} 
Set=Use set
Bag= Use set, {/*bag implementation*/}
\end{lstlisting}\saveSpace\saveSpace
or 
\saveSpace\saveSpace\begin{lstlisting}
set={/*set implementation*/} 
Set=Use set
bag=Use set, {/*bag implementation*/}
Bag=Use bag
\end{lstlisting}\saveSpace
Notice how, thanks to flattening, the resulting code for \Q@Bag@ is identical in both versions
and, as shown in Section 2, there is no trace of trait \Q@bag@. 
Thus if we are the developers of bags, we can temporarily go for the first version.
Then, when for example we need to add \Q@Bag5@ as discussed before,
we can introduce the \Q@bag@ trait without adding new undesired complexity for our old clients.


As a pleasurable accident, avoiding such code allows us to provide support for the
`\Q@This@' type and (in the extensions with nested classes seen later)
family polymorphism in a very simple way.


\saveSpace\saveSpace\section{Improving Reuse}\saveSpace

%\name allows reuse even when subtyping is impossible.
%\name traits do not induce a new (externally visible) type.
%However, locally in a trait, programmers can use the special self-type \Q@This@~\cite{bruce_1994,Saito:2009,ryu16ThisType} in order to denote the 
%type of \Q@this@.
%That is, a program is agnostic to what the \Q@This@ type is, so that it can
%be later assigned to any (or many) classes. 
%The idea is that during flattening, \Q@This@ will be replaced with the actual class name.
%In this way, \name allows reuse even when subtyping is
%impossible. For example for \emph{binary
%  methods}~\cite{bruce96binary} where the method parameter has type \Q@This@. 
%This type of situations is the primary motivator
%for previous work aiming at separating inheritance from subtyping~\cite{cook}.
%Leveraging on the \Q@This@ type, we can also provide self-instantiation (trait methods can create instances of the class using them) and smoothly integrate state and traits: a challenging problem that has limited the flexibility of traits and
%reuse in the past.

%\subsection{Managing State}

\name improves reuse in many different ways,

To illustrate how \name improves reuse,
we will show a novel approach
to deal with \emph{state} in traits.
Moreover, we provide self-instantiation (trait methods can create instances of the class using them)
 and smoothly integrate state and traits: a challenging problem that has limited the flexibility of traits and
reuse in the past.

The idea of summing pieces of
code is very elegant, and has proven very successful in module
composition languages~\cite{ancona2002calculus} and several trait
models~\cite{Traits:ECOOP2003,Bergel2007,BETTINI2013521,fjig}.  However the research
community is struggling to make it work with object state (constructors
and fields) while achieving the following goals:

\begin{itemize}
%complicated discussions on this point \item keep sum associative and commutative,
\item managing fields in a way that borrows the elegance of summing methods;
\item actually initialize objects, leaving no null fields;
\item making it easy to add new fields;
\item allowing a class to create instances of itself.
\end{itemize}

\subsection{State of the art}
In the related work we will show some alternative ways to handle
state.  However the purest solution requires methods only. The idea is
that the trait code just uses getter/setters/factories, while leaving
to classes the role to finally define the fields/constructors. That
is, classes have syntax richer than traits, allowing
declaration for fields and constructors.  This approach is very
powerful as illustrated by Wang et al.~\cite{wang2016classless}.

\paragraph{Modelling Points} Consider, for example, two simple 
traits that deal with \emph{point} objects. That is, points
in the cartesian plane (with coordinates \lstinline{x} and
\lstinline{y}). The first trait provides a \emph{binary method} that 
sums the point object with another point to return a new point. 
The second trait provides a similar operation that does multiplication 
instead.
\saveSpace 
\begin{lstlisting}
  pointSum: { method int x()  method int y()//getters
    class method This of(int x,int y)//factory method
    method This sum(This that)
      This.of(this.x()+that.x(),this.y()+that.y())//self instantiation
    }
  pointMul: { method int x() method int y()//repeating getters
    class method This of(int x,int y)//repeating factory
    method This mul(This that)
      This.of(this.x()*that.x(),this.y()*that.y())
    }
\end{lstlisting}
\saveSpace
\noindent As we can see, all the state operations (the getters for the 
\lstinline{x} and \lstinline{y} coordinates) are represented as {\bf abstract} methods.
Notice the abstract \Q@class method This of(..)@ which acts as a constructor
for points:
a class method is similar to a \Q@static@ method in Java but can be abstract. 
As for instance methods, they are late bound:  flattening can provide an implementation for them.
Abstract class methods are very similar to the original concept of member functions in the module composition setting~\cite{ancona2002calculus}.

Following the general model of traits and classes common in literature \emph{in a traditional trait model}~\cite{Traits:ECOOP2003},
we can now compose the two traits, by adding glue-code
to implement methods \Q@x@,\Q@y@ and \Q@of@.
\begin{lstlisting}
  CPoint:Use pointSum,pointMul, {//not our suggested solution
    int x   int y
    method int x() x       
    method int y() y
    class method This of(int x, int y) new Point(x,y)
    constructor Point(int x, int y){ this.x=x   this.y=y }
    }
\end{lstlisting}
%\bruno{We talk about withers later on. So I think we should consider
 % having withers in this code, so that readers can understand what 
%withers are!}
%\marco{with withers it will look more complicated}

\noindent 
With slightly different syntax, this approach is available in both Scala and Rust.
This verbose approach works, and it as some advantages, but also
some disadvantages: 

\begin{itemize}

\item {\bf Advantages:} This approach is associative and commutative, even self construction
  can be allowed if the trait requires a static/class method
  returning \Q@This@. The class will then implement the methods returning \Q@This@
  by forwarding a call to the constructor.
  
\item {\bf Disadvantages:} Writing such obvious definitions to close
  the state/fixpoint in the class 
   with the constructors and fields and getter/setters and factories is tedious.
   Moreover, there is no way for a trait to specify a default value for a field,
   the class need to handle all the state, even state that is conceptually
   "private" of such trait. 
   Previous work shows that such code can be automatically
   generated~\cite{wang2016classless}.
   Also the semantic of \Q@Use@/code composition of a model with fields and constructors is necessarily
   more complex than a model with methods only.
\end{itemize}

\subsection{Our proposed approach to State: Coherent Classes}

In \name there is no need to generate
code, or to explicitly write down constructors and fields. In fact in
\name there is not even syntax for those constructs!  The idea is that
any class that ``can" be completed in the obvious way  is \emph{a
  complete ``coherent" class}.  In most other languages, a class is
abstract if it has abstract methods.  Instead, we call a class
abstract only when the set of abstract methods is not coherent. That
is, the unimplemented methods cannot not be automatically recognised
as factory, getters and setters. Methods recognised as factory, getters and setters are called
\emph{abstract state operations}.
  
\paragraph{Coherent classes} A more detailed definition of coherent
classes is given next:
\begin{itemize}
\item a class with no abstract methods is coherent (just like Java
  \Q@Math@, for example). Such classes are useful for calling class/static methods.
\item a class with a single abstract \Q@class@ method returning \Q@This@
is coherent if all the other abstract methods can be seen as \emph{abstract state
operations} over one of its argument.
For example,
if there is a \Q@class method This of(int x, int y)@ as before,
then
\begin{itemize}
\item a method \Q@int x()@ is interpreted as an abstract state method: a \emph{getter} for \Q@x@.
\item a method \Q@Void x(int that)@ is a \emph{setter} for x.
\end{itemize}
\end{itemize}
\noindent
While getters and setters are fundamental operations, it is possible to
support more operations. For example:
\begin{itemize}
\item \Q@method This withX(int that)@
may be a ``wither", doing a functional field update: it creates a new instance that is like \Q@this@ but where field \Q@x@ has now \Q@that@ value.
\item \Q@method Void update(int x,int y)@
may do two field updates at a time.
\item\Q@method This clone()@ may do a shallow clone of the object.
\end{itemize}

We are not sure what is the best set of abstract state operations yet,
but we think this could become a very interesting area of research.
The work by Wang et al.~\cite{wang2016classless} explores a particular
set of such abstract state operations.

\paragraph{Points in \name:}
In \name and with our approach to handle the state, 
\lstinline{pointSum} and \lstinline{pointMul} can indeed be directly composed:
\saveSpace
\begin{lstlisting}
  //Same code as before! Works because resulting class is coherent.
  PointAlgebra:Use pointSum,pointMul 
\end{lstlisting}  
\saveSpace
\noindent
  Note how we can declare the methods independently and compose the result
  as we wish. 

  \paragraph{Improved solution} So far the current solution still
  repeats the abstract methods \Q@x@, \Q@y@ and \Q@of@.
  Moreover, in addition to \Q@sum@ and \Q@mul@ we may want many
  operations over points. It is possible to improve reuse
  and not repeat such abstract definitions by abstracting the common
  abstract definitions into a trait \Q@p@: 
\saveSpace
\begin{lstlisting}
  p: { method int x() method int y()
    class method This of(int x,int y)
    }
  pointSum:Use p, { method This sum(This that)
      This.of(this.x()+that.x(),this.y()+that.y())
    }
  pointMul:Use p, { method This mul(This that)
      This.of(this.x()*that.x(),this.y()*that.y())
    }
  pointDiv: ...
  PointAlgebra:Use pointSum,pointMul,pointDiv,...
\end{lstlisting}
\saveSpace      
Now the code is fully modularized, and each trait handles exactly one method.

\subsection{State Extensibility}
Programmers may want to extend points with more state. For example 
they may want to add colors to the points. A first attempt at doing
this would be:
\saveSpace
\begin{lstlisting}
  colored:{ method Color color() }
  Point:Use pointSum,colored //Fails: class not coherent
\end{lstlisting}
\saveSpace
This first attempt does not work: the abstract color method
is not a getter for any of the parameters of 
\Q@ class method This of(int x,int y)@. 
A solution is to provide a richer factory:
\saveSpace
\begin{lstlisting}
  CPoint:Use pointSum,colored,{
    class method This of(int x,int y) This.of(x,y,Color.of(/*red*/))
    class method This of(int x, int y,Color color)
    }
\end{lstlisting}
\saveSpace
\noindent 
where we assume support for overloading on different number of parameters.
This is a reasonable solution, however the method \Q@CPoint.sum@ resets
the color to red: we call the \Q@of(int,int)@ method, that now
delegates to \Q@of(int,int,Color)@ by passing red as the default field
value.  What should be the behaviour in this case?  If our abstract
state supports withers, we can use
\Q@this.withX(newX).withY(newY)@, instead of writing \Q@This.of(...)@, in order to preserve the color from
\Q@this@.  This solution is still not satisfactory: this design ignores
the color from \Q@that@.

\paragraph{A better design}
If the point designer is designing for reuse and extensibility, then 
a better design would be the following:  
\saveSpace\begin{lstlisting}
  p: { method int x() method int y() //getters
    method This withX(int that) method This withY(int that)//withers
    class method This of(int x,int y)
    method This merge(This that) //new method merge!
    }
  pointSum:Use p, { method This sum(This that)
      this.merge(that).withX(this.x()+that.x()).withY(this.y()+that.y())
    }
  colored:{method Color color()
    method This withColor(Color that)
    method This merge(This that) //how to merge color handled here
      this.withColor(this.color().mix(that.color())
    }
  CPoint:/*as before*/
\end{lstlisting}  \saveSpace
  \noindent This design allows merging colours, or any other kind of state we may want to add
  following this pattern.%\bruno{worried that withers are not explained enough.}

\paragraph{Independent Extensibility}
  Of course, quite frequently there can be multiple independent
  extensions~\cite{Zenger-Odersky2005} that need to be composed. Lets suppose that 
  we could have a notion of flavoured points as well.   
  In order to compose, let say \Q@colored@ with \Q@flavored@ we would
  need to compose the merge operation inside of both of them.

Just \use\ is not sufficient, since we need to combine the implementation of 2 different version of \Q@merge@.
We introduce here an operator \Q@restrict@.
Restrict makes a method abstract and
move the implementation to another name. This is very useful to implement \Q@super@
 and to compose conflicting implementations.

The following code shows how to mix colours and flavours. 
Note how we use \Q@restrict@ to introduce method selectors \Q@_2merge@ and \Q@_3merge@
to refer to the version of \Q@merge@ as defined in the second/third element of \use.
\saveSpace\begin{lstlisting}
  p: {/*as before*/ }
  pointSum:Use p, {/*as before*/ }
  colored:{/*as before*/}
  flavored:{
      method Flavor flavor() //very similar to colored
      method This withFlavor(Flavor that)
      method This merge(This that) //merging flavors handled here
        this.withFlavor(that.flavor())}//inherits ``that'' flavor
  FCPoint:Use
    colored[restrict merge as _1merge],
    flavoured[restrict merge as _2merge],
    pointSum,{
      class method This of(int x,int y)
        This.of(x,y,Color.of(/*red*/),Flavor.none())
      class method This of(int x, int y,Color color,Flavor flavor)
      //resolves the conflict about two implementations for merge
      //by proving our own implementation here
      method This merge(This that) this._1merge(that)._2merge(that)
      }
\end{lstlisting}  \saveSpace\saveSpace

Note how we are levering on the fact that the code literal
 does not need to be complete, 
thus we can just call \Q@_1merge@ and \Q@_2merge@ without
 declaring their abstract signature explicitly.

In case someone wish to discard the implementation of a \Q@restricted@ method,
they can just omit the new name, as in the following example:
\begin{lstlisting}
t:{method bool geq(This x) x.leq(this)   method bool leq(This x) x.qeq(this) }
C:Use t[restrict geq],{method bool geq(This x){return /*actual geq impl*/}}
\end{lstlisting}

\section{Extensions to our model}
  One of the main feature of our simple reuse/use model is that it can be
  easy extended. 
First we show two more composition operators, then we show 
a simple extension with nested classes.
Finally we show how we can solve even the most
demanding version of the expression problem.

\paragraph{Rename}${}_{}$\\*
Rename allows to make some form of ``compile time'' re-factoring
There are a lots of different forms of rename in literature,
sometime allowing only to rename specific methods, sometime allowing to rename
nested classes into other nested classes either at the same or at a different nesting level.
%Renaming in the context of nested classes also means that when renaming a method of an interface, all the 
%nested classes implementing such interface inside of that code literal need to be adjusted.
Renaming need to rename not only the method headers, but all the method calls inside of method bodies.
At first glimps, this seams to be not always possible since we are considering to be able to apply those
operators also to non well typed code.
However, if the expression language is simple enough, it is possible to pre process the code to
annotate the expected receiver type on all method calls by doing a purelly sintactic analysis
on a single code literal in isolation. 
All the expression whose type is guessed to be out of the border of the literal can stay unannotated; they are not going to be renamed anyway.

\begin{lstlisting}
t:{ I:{interface method int mI() }
     A:{implements I  method int mI() 42}
     B:{ method int mB(I i, A a, C c) i.mI()+a.mI()+c.mI()}
     //mB would be annotated i[I].mI()+a[A].mI()+c.mI()
     }
D:t[rename A.mI kI]
\end{lstlisting}
 Notice how we are sure that c does not implements I since it is invisible from the outside: traits does not introduce nominal types!
 
 We expect the flattened version for \Q@D@ to be
\begin{lstlisting}
D:{ I:{interface method int kI() }
     A:{implements I  method int kI() 42}
     B:{ method int mB(I i, A a, C c) i.kI()+a.kI()+c.mI()}
     }
\end{lstlisting}

Hide can be seen as a variation of rename, where the method/class is renamed to a fresh unguessable name.

\paragraph{Redirect}${}_{}$\\*
Redirect allows to emulate generics; the main idea is that a (fully abstract) class can be redirect to another one external to the trait/code literal.
For example a linked list can be implement as
\begin{lstlisting}
list:{ Elem:{}
     Cell:{class method Cell of(Elem e,Cell c) 
       method Elem e()  method Cell c()
       }
   method Elem get(int x) ...
   ...more methods..
   }
ListString:list[redirect Elem to String]
\end{lstlisting}

%An expressive form of Redirect can be multiple, that is, can redirect may interdependent classes at the same time.
%We show a graph example, where also we can show how to propagate generics:
%For example
%\begin{lstlisting}
%t:{ method boolean reachable(Node start, Node end)/*implements reachability*/
%     Node:{method ListEdge nodes()}
%     Edge:{method Node in()  method Node out()}
%     ListEdge:list[redirect Elem to Edge]
%     }
%\end{lstlisting}




\paragraph{Nested classes}${}_{}$\\*

A nested class will be another kind of member
in the code Literal.
The general idea is that by composing code with \use, nested classes with the same name are recursivelly composed.
Note that while we have nested classes, we do not have nested traits: all traits are still
at top level.
Untypable/unresolved Traits are also the only``dependency"
the type system keeps track of, this means that when a nested class at an arbirary
nested level is flattend, as in
\Q@C:{ D:{ E:Use t1,t2,L}}@\\*
\Q@t1@ and \Q@t2@ must be defined before \Q@C@ at top level; and they may require classes (and their
nested) defined before C. This means that the type system can still consider
the class table as a simple map from Types T to their definition.

This extension lets us challenge the expression problem~\cite{wadler1998expression},
with the requirements exposed in~\cite{scala}.
In the expression problem we have data variants and operations and we can
\textbf{extend our solution in both dimensions},
by adding new datavariants and adding new operations.
We also aim to \textbf{combine independently developed extensions} so
that they can be used jointly.

To be really modular, we want our extensions to
preserve \textbf{type safety}
and allow \textbf{separate compilation} (no re-type-checking),
while avoiding \textbf{duplication of source code}.

We can do this by creating a table of features, as in~\cite{Deepfjig}.
But in \name we can be even more compact, and we can avoid a large amount of abstract declarations,
that clutters the solution in~\cite{Deepfjig}.


\begin{lstlisting}
//Definition of the datavariants state
exp:{//Exp declared once, reused everywhere
  Exp:{interface}
  T:{implements Exp}
  }
lit:Use exp[rename T in Lit],{//T renamed in Lit is 
  Lit:{ method Int inner() class method Lit of(Int inner)
    }//summed with the current Lit
  }
sum:Use exp[rename T in Sum],{//same for the other
  Sum:{ method Exp left() method Exp right()
    class method Sum(Exp left, Exp right) }
  }
uminus:Use exp[rename T in UMinus],{
  UMinus:{ ... }
  }

\end{lstlisting}
Here we define a trait for each datavariant.
Each trait will contain its version of \Q@Exp@
and a specific case of expression, with its state.
 To avoid repeating the declaration of \Q@Exp@ 
and \Q@implements Exp@ multiple times, we reuse
a \Q@exp@ trait.

\begin{lstlisting}
//Definition of toString for all the datavariants
expToS:Use exp, {//concept of toString declared once
  Exp:{interface method String toString()}
  }
litToS: Use Lit, expToS[rename T in Sum], {
  Lit:{//just the implementation of the specific method
    method String toString(){return ""+this.inner();}
    } }
sumToS:Use sum, expToS[rename T in Sum], {
  Sum:{//just the implementation of the specific method
    method String toString(){
      return this.left().toString()+"+"+this.right().toString();}
      }  }  }
uminusToS:...//implement toString for all the datavariants
\end{lstlisting}

Here we define a trait for each datavariant implementing the operation \Q@toString()@.
Again, Each trait will contain its version of \Q@Exp@ with \Q@toString()@.
and a specific case of expression, with the implementation for \Q@toString()@
for that specific case. Note how the state of such datavariant is 
\textbf{not repeated}, but imported by (for example) \Q@Use sum, ...@.

Again, \Q@expToS@ is a useful modularization point.
In the code below, the example of the other iconic operation: \Q@eval@.

\begin{lstlisting}
//Definition of other operations, for example eval
expEval:Use exp, {
  Exp:{interface method int eval()}
  }
//declare the next operation and implement it for all the datavariant
//example
sumEval:Use sum,expEval[rename T in Sum], {
  Sum:{//just the implementation of the
    method int eval(){//specific method
      return this.left().eval()+this.right().eval();}

      }  }  }
\end{lstlisting}
In \name it is trivial also for the binary method equals.
Since every datavariant would have to copy the "cast" logic,
 we can modularize that into a \Q@expEquals@.

\begin{lstlisting}
expEquals:{
  Exp:{interface method Bool equals(Exp that)}
  T:{implements Exp
    method Bool exactEqual(T that)
    method Bool equals(Exp that){
      if(!(that instanceof T)){ return false;}
      return exactEqual( (T) that );
      }  }  }
sumEquals:Use sum,expEquals[rename T in Sum],{
  Sum:{
    method Bool exactEquals(Sum that){
      return this.left().equals(that.left()) 
        && this.right().equals(that.right())
      }  }  }
\end{lstlisting}
Note how the cast is safe since there is guarded by
and \Q@instanceof@. A sightly longer solution could avoid the
cast with double dispatch as done in [](show in appendix?)
%expEquals:{
%  Exp:{interface
%    method Bool equals(Exp that)
%    method Bool equalToT(T that)
%    }
%  T:{implements Exp
%    method Bool equals(Exp that){ that.equalToT(this); }
%    }  }
%sumEquals:Use sum,
%expEquals[rename T in Sum][rename Exp.equalToT in equalToSum],{
%  Sum:{
%    method Bool equalToSum(Sum that){
%      return this.left().equals(that.left()) 
%        && this.right().equals(that.right())
%      }  }  }

Now that you have nicely modularized the code, just compose all the traits you need.
\begin{lstlisting}
MySolution:Use sumToS,litToS
//sum,lit and exp traits are already included
\end{lstlisting}

The expression problem presented up to now is the traditional challenge proposed by~\cite{wadler1998expression};
this has been criticized to not really address the fundamental issues since it does not handle ....
Now we show how we can go beyond the traditional expression problem by encoding transformer methods:
For example, lets add 1 to all literals
\begin{lstlisting}
expAdd1:{Exp:{interface method Exp add1()}}
sumAdd1:Use sum,expAdd1,{Sum:{implements Exp
    method Exp add1()
      Sum.of(left.add1,right.add1())
  }
litAdd1:Use lit,expAdd1,{Lit:{implements Exp
    method Exp add1()
      Lit.of(inner()+1);
    }

MySolutionAdd1:Use sumToS,litToS,sumAdd1,litAdd1
\end{lstlisting}

This nicely solve our problem. 
However, notice how if we wished to add many similar operations we would 
have to repeat the propagation code (as in \Q@sumAdd1@) many times
just changing the name of the operation.
Solutions to the expression problems involving the  Visitor Pattern 
allow defining a \Q@CloneVisitor@ in order
to reuse its propagation code.
You can see what we mean with the sketch of code below:
\begin{lstlisting}
class CloneVisitor{
  Exp visit(Lit l){return new Lit(l.inner);}
  Exp visit(Sum s){return new Sum(s.left.accept(this),s.right.accept(this);}
  }
class Add1 extends CloneVisitor{
  Exp visit(Lit l){return new Lit(l.inner+1);}
  }
\end{lstlisting}
In \name we can obtain the same kind of code reuse, without the need of introducing 
the concepts related to the Visitor Pattern.
With redirect, rename and restrict we can have the general operator propagator:
\begin{lstlisting}
operation:{//for sum and lit, easy to extends as before
  T:{}
  Exp:{interface method Exp op(T x)}}
  Sum:Use sum,{ extends Exp sum,expAdd1,{
    method Exp op(T x)
      Sum.of(left.op(x),right.op(x))  }
  Lit:Use lit,{
    method Exp op(T x)  this
  }
\end{lstlisting}

Now, to have my \Q@addN@ we can

\begin{lstlisting}
opAddn: Use
  operation[redirect T to Int]
    [rename Exp.op(x) to addN(x)][restrict Lit.op(x)], {
  Lit:{method Exp addN(Int x) Lit.of(inner())+x}
  }
\end{lstlisting}  



%\paragraph{Full power of redirect}${}_{}$\\*
%An expressive form of Redirect can be multiple, that is, can redirect may interdependent classes at the same time.
%We show an example where a specific kind of \Q@Service@ can produce a \Q@Report@, and 
%\Q@Report@s can be combined together.
%The goal is to execute a list of such services and produce a collated report.
%This example also show how to propagate generics:
%
%\begin{lstlisting}
%Service:{interface method Void performService()}
%serviceCombinator:{
%  S:{implements Service method R report()  }
%  
%  R:{method R combine(R that)   class method R empty() }
%  
%  ListS:list[redirect Elem to S]
%  
%  class method R doAll(ListS ss){//here we use extended java like syntax
%    R r=R.empty()
%    for(S s in ss){
%      s.performService();
%      r=r.combine(s.report())
%      }
%    return r;
%  }
%}
%PaintingService:serviceCombinator[redirect S to PaintingService]
%PaintingService:{... method PaintingReport report()..}
%PaintingReport:{..}
%\end{lstlisting}
%
%The flattened version of \Q@PaintingService@ would look like:
%\begin{lstlisting}
%PaintingService:{
%  ListS:/*the expansion of list[redirect Elem to PaintingService]*/
%  
%  class method PaintingReport doAll(ListS ss){
%    PaintingReport r=PaintingReport.empty()
%    for(PaintingService s in ss){
%      s.performService();
%      r=r.combine(s.report())
%      }
%    return r;
%  }
%}
%\end{lstlisting}
%Where you can note how redirect figured out \Q@R=PaintingReport@ by comparing the structural shape of
%classes \Q@PaintingService@ and \Q@S@.
%
%To encode the former generic code in java you need to write
%the following headeche inducing interfaces for RService and Report.
%and require that the services you want to serve implement those.
%\begin{lstlisting}
%interface Service{ void performService();}
%interface Report<R extends Report<R>>{R combine(R that);}
%interface RService<R extends Report<R>> extends Service{ R report();}
%\end{lstlisting}
%Note how we still can not encode the method \Q@empty@.

\saveSpace\saveSpace\section{Intuitions on formalization}\label{sec:formal}
\saveSpace\saveSpace

%NEW FORMALISATION below



% Syntax
% D::=TD|CD
% TE::=t:E Trait Decl Expr
% CE::=C:E Class Decl
% TD::=t:L
% CD::=C:L
% E::= L| t| E+E | E[rename T.m1->m2]|E[rename T1->T2]|E[redirect T1->T2]
% L::= {interface? implements Ts Ms}//all L are like LC in 42
% T::=C|C.T // .T is a shortcut for This.T
% M::= static? method T m(T1 x1..Tn xn) e? | CD
% e::= x| e.m(es) | T.m(es)

\begin{bnf}
\prodFull\mD{\mTD\mid\mCD}{Declaration}\\
\prodFull\mTE{\mt\terminalCode{:}\mE }{Trait Decl Expr}\\
\prodFull\mCE{\mC\terminalCode{:}\mE}{Class Decl Expr}\\
\prodFull\mTD{\mt\terminalCode{:}\mL}{Trait Decl}\\
\prodFull\mCD{\mC\terminalCode{:}\mL}{Class Decl}\\
\prodFull\mE{\mL \mid \mt \mid \mE\terminalCode{+}\mE \mid
\mE\terminalCode{[rename}\ \mT\terminalCode{.}\mm_1\ \terminalCode{to}\ \mm_2\terminalCode{]} \mid
\mE\terminalCode{[rename}\ \mT_1\ \terminalCode{in}\ \mT_2\terminalCode{]} \mid
\mE\terminalCode{[redirect}\ \mT_1\ \terminalCode{to}\ \mT_2\terminalCode{]}}{Code Expression}\\

\prodFull\mL{
\oC \Opt{\terminalCode{interface}}\ \terminalCode{implements} \overline\mT\ \overline\mM\ \cC}{Code Literal}\\
\prodFull\mT{\mC \mid \mC\terminalCode{.}\mT}{Type}\\
\prodFull\mM{\Opt{\terminalCode{static}}\ \terminalCode{method}\ \mT\ \mm\oR\overline{\mT\,\mx}\cR \Opt\me \mid \mCD}{Member}\\

\prodFull\me{\mx \mid \me\terminalCode{.}\mm\oR\overline\me\cR \mid \mT\terminalCode{.}\mm\oR\overline\me\cR}{Expression}\\
\end{bnf}\\
\\

\noindent Notations: we use the . to extract sub parts of the tree, as in the 42 documents, to explain\\

$
\!\!\!\begin{array}{l}

%       D.E -->^+_CDs L  CDs|-CD1:OK .. CDs|-CDn:OK       CDs=CD1..CDn
% (top)---------------------------------------------------------------    D.E not of form L
%      CD1..CDn CDs' D Ds -> CDs CDs' D[with E=L] Ds

 \inferrule[(top)]{
    \mD.\mE \xrightarrow[\overline{\mCD}]{}^+ \mL \quad\overline{\mCD}\vdash\mCD_1:\text{OK} \  .. \  \overline{\mCD}\vdash\mCD_n:\text{OK} 
  }{ 
    \mCD_1..\mCD_n \ \overline{\mCD'}\ \mD \ \overline{D} \rightarrow \overline{\mCD}\ \overline{\mCD'}\ \mD\text{[with \mE=\mL]}\ \overline{\mD}
  } \begin{array}{l}\overline{\mCD}\text{ = }\mCD_1..\mCD_n \\ \mD.\mE\ \text{not of form}\ \mL \end{array}
\\[4ex]

%
%     ------------------------
%      t -->_CDs CDs(t)

 \inferrule[]{
    \ 
  }{ 
    \mt \xrightarrow[\overline{\mCD}]{}\ \overline{\mCD}\oR\mt\cR
  }
\\[5ex] 

%      E0-->_CDsE1
%    ------------------------
%    E0+E2  -->_CDs E1+E2

\inferrule[]{
    \mE_0 \xrightarrow[\overline{\mCD}]{}\ \mE_1
  }{ 
     \mE_0+\mE_2 \xrightarrow[\overline{\mCD}]{}\ \mE_1+\mE_2
  }
\\[5ex] 

%      E0-->_CDsE1
%    ------------------------
%    L+E0  -->_CDs L+E1

\inferrule[]{
    \mE_0 \xrightarrow[\overline{\mCD}]{}\ \mE_1
  }{ 
     \mL+\mE_2 \xrightarrow[\overline{\mCD}]{}\ \mL+\mE_2
  }
\\[5ex] 

%
%      --------------------------      L = L1+L2
%      L1+L2  -->_CDs L

\inferrule[]{
    \
  }{ 
     \mL_1+\mL_2 \xrightarrow[\overline{\mCD}]{}\ \mL
  }\mL = \mL_1+\mL_2
\end{array}
$\\
\\

%   Define L1+L2, Ms+Ms,  M+M
%   L1+L2  =  L3
%   L1={ interface? implements Ts1 Ms1 Ms0}
%   L2={ interface? implements Ts2 Ms2 Ms0'}
%   L3={ interface? implements Ts1,Ts2 Ms1,Ms2 (Ms0+Ms0')}
%   dom(Ms1) disj dom(Ms2)
%   dom(Ms0) = dom(Ms0')
%    
%   M1..Mn+M'1..M'n  =  M1+M'1..Mn+M'n
%   
%   C=L1+C=L2  =  C=L3  if L1+L2
%   
%   M1+M2  =  M2+M1
%   
%   static? method T0 m(T1 x1..Tn xn) + static? method T0 m(T1 x1..Tn xn) e?
%     =  static? method T0 m(T1 x1..Tn xn) e?

\noindent\textbf{Define }$\mL_1+\mL_2, \ \overline{\mM}+\overline{\mM},\ \mM+\mM$\\
$\begin{array}{l}
\!\!\!\bullet\ \mL_1+\mL_2 =\mL_3\\
\!\!\!\bullet\ \mL_1= \oC \Opt{\terminalCode{interface}}\ \terminalCode{implements} \overline\mT_1\ \overline\mM_1\ \overline\mM_0\cC\\
\!\!\!\bullet\ \mL_2= \oC \Opt{\terminalCode{interface}}\ \terminalCode{implements} \overline\mT_2\ \overline\mM_2\ \overline\mM_0\cC\\
\!\!\!\bullet\ \mL_3= \oC \Opt{\terminalCode{interface}}\ \terminalCode{implements} \overline\mT_1,\overline\mT_2\ \overline\mM_1,\overline\mM_2\ \oR\overline\mM_0+\overline\mM_0'\cR\cC\\
\!\!\!\bullet\ dom\oR\overline\mM_1\cR\ \text{disj}\ dom\oR\overline\mM_2\cR\\
\!\!\!\bullet\ dom\oR\overline\mM_0\cR\ =\ dom\oR\overline\mM_0'\cR\\\\

\!\!\!\bullet\ \mM_1..\mM_n+\mM'_1+\mM'_n\ = \ \mM_1+\mM'_1..\mM_n+\mM'_n\\

\!\!\!\bullet\ \mC=\mL_1+\mC=\mL_2\ = \ \mC=\mL_3\quad if \mL_1+\mL_2\\

\!\!\!\bullet\ \mM_1+\mM_2=\mM_2+\mM_1\\

\!\!\!\bullet\ \Opt{\terminalCode{static}}\ \terminalCode{method}\ \mT_0\ \mm\oR\overline{\mT\,\mx}\cR \ + \ \Opt{\terminalCode{static}}\ \terminalCode{method}\ \mT_0\ \mm\oR\overline{\mT\,\mx}\cR \Opt\me = \Opt{\terminalCode{static}}\ \terminalCode{method}\ \mT_0\ \mm\oR\overline{\mT\,\mx}\cR \Opt\me\\
\end{array}$

$
\!\!\!\begin{array}{l}

%  C;CDs,C=L |- L[This=C] :OK
% ----------------------------------------- coherent(L)
%  CDs|-C=L : OK

\inferrule[]{
    \mC;\overline\mCD,\mC=\mL\vdash \mL[This=\mC]\ :\text{OK}
  }{ 
     \overline\mCD \vdash \mC=\mL\ :\text{OK}
  } \terminalCode{coherent}\oR\mL\cR
\\[5ex] 

%    This;CDs,This=L |- L :OK
%----------------------------------------
%    CDs|-t=L : OK

\inferrule[]{
    This;\overline\mCD,This=\mL\vdash \mL\ :\text{OK}
  }{ 
     \overline\mCD \vdash \mt=\mL\ :\text{OK}
  }
\\[5ex] 

%  forall i in 1..k T;CDs|-Mi:Ok
%--------------------------------------------------  L={interface? implements T1..Tn M1..Mk} 
%  T;CDs|-L:Ok                                         forall i in 1..n 	CDs(Ti).interface?=interface
%                                                             forall i in 1..n and m in 	dom(CDs(Ti)), m in dom(L)

\inferrule[]{
    \forall i \in 1..k\ \mT;\overline\mCD\vdash\mM_1:\text{OK}
  }{ 
     \mT;\overline\mCD \vdash \mL\ :\text{OK}
  } \begin{array}{l} 
  \mL=\oC \Opt{\terminalCode{interface}}\ \terminalCode{implements} \mT_1 .. \mT_n \ \mM_1\ .. \mM_k \cC \\
  \forall i \in 1..n\ \overline\mCD\oR\mT_i \cR.\Opt{\terminalCode{interface}}=\terminalCode{interface} \\
  \forall i \in 1..n \ \text{and } m \in \text{dom}\oR\mCD\oR\mT_i\cR\cR. \mm \in \text{dom}\oR\mL\cR
   \end{array}

\\[5ex] 

%  T.C; CDs|-L :Ok
% ------------------
%  T;CDs|-C=L : Ok 

\inferrule[]{
    \mT;\overline\mC,\overline\mCD\vdash \mL\ :\text{OK}
  }{ 
     \mT;\overline\mCD \vdash \mC=\mL\ :\text{OK}
  }
\\[5ex] 

%  if e?=e then CDs; G|-e:T                         
%----------------------------------------------------------   forall T in CDs(C).Ts, if m in dom(CDs(Ti)) then
%   T;CDs|-static? T0 m(T1 x1..Tn xn) e?              static? T0 m(T1 x1..Tn xn) in CDs(Ti)
%                                                                        if static?=static then G=x1:T1 .. xn:Tn
%                                                                        else G=this:T,x1:T1 .. xn:Tn

\inferrule[]{
    \terminalCode{if}\ \Opt\me=\me\ \terminalCode{then}\ \overline\mCD; \mG\vdash\me:\mT
  }{ 
     \mT;\overline\mCD \vdash \Opt{\terminalCode{static}}\ \terminalCode{method}\ \mT_0\ \mm\oR\overline{\mT\,\mx}\cR \Opt\me
  } \begin{array}{l} 
  \forall \mT \in \overline\mCD\oR\mC\cR.\overline\mT,\ \terminalCode{if}\ \mm \in \text{dom}\oR \overline\mCD\oR\mT_i\cR\cR\ \terminalCode{then} \\
  \quad\Opt{\terminalCode{static}}\ \terminalCode{method}\ \mT_0\ \mm\oR\overline{\mT\,\mx}\cR \in \overline\mCD\oR\mT_i\cR \\
  \terminalCode{if}\ \Opt{\terminalCode{static}}=\terminalCode{static}\ \terminalCode{then}\ \mG=\mx_1:\mT_1\ .. \ \mx_n:\mT_n\ \\
  \quad\terminalCode{else}\ \mG=this:\mT,\mx_1:\mT_1\ ..\ \mx_n:\mT_n
   \end{array}
\\[5ex] 

%   CDs;G|-e:T1
%   CDs|-T1<=T2
%--------------------
%   CDs;G|-e:T2

\inferrule[]{
%  \begin{array}{l}
    \overline\mCD; \mG\vdash\me: \mT_1  \\\\
    \overline\mCD\vdash\mT_1 \Leftarrow \mT_2
%  \end{array}
  }{ 
     \overline\mCD; \mG\vdash\me: \mT_2
  }
\\[5ex] 

%   
%--------------------
%   CDs;G|-x:G(x)

\inferrule[]{
    \
  }{ 
    \overline\mCD; \mG\vdash\mx: \mG\oR\mx\cR
  }
\\[5ex] 

%  CDs;G|-e1:T1 .. CDs;G|-en:Tn
%------------------------------------------    static T m(T1 x1..Tn xn) _ in CDs(T0)
%  CDs;G|-T0.m(e1..en):T

\inferrule[]{
    \mCD;\mG\vdash\me_1:\mT_1\ .. \ \mCD;\mG\vdash\me_n:\mT_n
  }{ 
    \mCD;\mG\vdash \mT_0.\mm\oR\me_1\ .. \ \me_n\cR:\mT
  } \terminalCode{static}\ \mT\ \mm\oR\overline{\mT\,\mx}\cR \text{\_} \in \overline\mCD\oR\mT_0 \cR
\\[5ex] 

%    CDs;G|-e0:T0 .. CDs;G|-en:Tn
%---------------------------------------------    static T m(T1 x1..Tn xn) _ in CDs(T0)
%  CDs;G|-e0.m(e1..en):T

\inferrule[]{
    \mCD;\mG\vdash\me_0:\mT_0\ .. \ \mCD;\mG\vdash\me_n:\mT_n
  }{ 
    \mCD;\mG\vdash \me_0.\mm\oR\me_1\ .. \ \me_n\cR:\mT
  } \terminalCode{static}\ \mT\ \mm\oR\overline{\mT\,\mx}\cR \text{\_} \in \overline\mCD\oR\mT_0 \cR


\end{array}
$\\

\textbf {>> FORMALISATION ENDS HERE} \\
---------------------------------------------------------------------------------------

%new added FORMALISATION ENDS here


This section sketches some of the key ideas in \name more
formally. A similarly interleaving approach can be seen in~\cite{servetto2014meta} fully formalized.

To simplify the formalism, 
 we have some difference in the concrete syntax.
We use binary \Q@+@ instead of \use
and when referring to a nested type, we require the fully
qualified typename. For a nested class of a trait,
is going to look like \Q@This.Exp@.

%In this article we dedicate more space to examples and informal presentation and motivations;
%so we do not have space to provide a full formalizations.
%We will provide here some hints on how the formalization works.

\subsection{Syntax}

In the following, we present a simplified grammar of \name:
%\begin{bnf}
%\prodFull{aa}{bb}{Declaration}\\
%\end{bnf}

%\begin{comment}
\begin{bnf}
\prodFull\mTD{\mt\terminalCode{:}\mL \mid \mt\terminalCode{:} \use\ \overline\mV}{Trait Decl}\\
\prodFull\mCD{\mC\terminalCode{:}\mL \mid \mC\terminalCode{:} \use\ \overline\mV}{Class Decl}\\
\prodFull\mV{\mt \mid \mL}{Code Value}\\
\prodFull\mL{
\oC
\Opt{\terminalCode{interface}}\ \terminalCode{implements} \overline\mT\ \overline\mMD
\cC
}{Code Literal}\\
\prodFull\mT{\mC}{types are class names}\\
\prodFull\mMD{\Opt{\terminalCode{static}}\ \terminalCode{method}\ \mT\ \mm\oR\overline{\mT\,\mx}\cR \Opt\me}{Method Decl}\\

\prodFull\me{\mx\mid\mT\mid\me\terminalCode{.}\mm\oR\overline\me\cR}{expressions}\\
\prodFull\mD{\mCD\mid\mTD}{Declaration}\\
\end{bnf}
%\end{comment}

\noindent\textbf{Definition:}$a+b=c$\\
$\begin{array}{l}
\!\!\!\bullet\ a+b=c\quad  \text{where}\\
\quad\quad a=\oC\ldots \cC\\
\!\!\!\bullet\ a+(b b)=a b b\\
\!\!\!\bullet\ \text{otherwise} a+ b =a\text{where}\\
\quad\quad a<b\\
\end{array}$

\noindent To declare a trait \mTD\ or a class \mCD, we can use either a code literal \mL\ or a trait
expression.  In full 42 we support various operators (restrict, hide,
alias), but in \name we focus on the single operator 
$\use$. $\use$ takes a set
of code values ($\overline\mV$): that is trait names \mt\ or code literals \mL\ and composes them.  
This operation, sometimes called \emph{sum}, is the simplest and most elegant
trait composition operator~\cite{ducasse2006traits}.

$\use\ \overline\mV$ composes the content of $\overline\mV$
by taking the union of the methods and the union of the
implementations.
\use\ cannot be applied if multiple versions of the same method are
present in different traits.  An exception is done for abstract methods:
methods where the implementation \me\ is missing. In this case (if the
headers are compatible) the implemented version is selected.  In a sum
of two abstract methods with compatible headers, the one with the more
specific type is selected.

Code literals \mL\ can be marked as interfaces. 
That is, the interface keyword is inside the curly brackets, so an upper case name associated with an interface literal is a class-interface, while a lowercase one is a trait-interface.
In our simple model, we consider an error trying to merge an interface with a non-interface.
 Then we have a set of implemented interfaces and a set of member
  declarations. In this simple language, the only members are methods.
If there are no implemented interfaces, in the concrete syntax we will omit the \Q@implements@ keyword.

Methods \mMD~can be instance methods or \Q@static@ methods. A static method is similar to a \Q@static@ method in Java but can be abstract. This is very useful in the context of code composition.
To denote a method as abstract, instead of an optional keyword we just omit the implementation \me.

A version of this language where there are no traits can be seen 
as a restriction/variation of FJ~\cite{igarashi2001featherweight}.

\noindent\textbf{Nested classes:}
A nested class will be another kind of member in the literal, so  
the grammar could be updated as following:

\begin{bnf}
\prodFull\mMD{
\Opt{\terminalCode{static}}\ \terminalCode{method}\ \mT\ \mm\oR\overline{\mT\,\mx}\cR \Opt\me
\mid \mCD
}{Member Decl}\\
\prodFull\mT{\mC\mid\mC\terminalCode{.}\mT}{types are now paths}
\end{bnf}\\


\subsection{Well-formedness}
In \name the basic well formedness rules apply:
\begin{itemize}
\item all method parameters have unique names and the special parameter name \Q@this@ is not declared
 in the parameter list,
\item all methods in a code literal have unique names,
\item all used variables are in scope,
\item all methods in an interface are abstract, and there are no interface static methods.
\end{itemize}
Those rules can be applied on any given \mL~individually and in full isolation.
We expect the type system to enforce: 
\begin{itemize}
\item all the traits and classes have unique names in a program $\overline\mD$, and the special class name
\Q@This@ is reserved,
\item all used types correspond to class declarations in the program, or are \Q@This@, 
\item subtyping between interfaces and classes,
\item method call typechecking,
\item no circular implementation of interfaces,
\item type signature of methods from interfaces can be refined following the well known variant-contravariant rules,
\item only interfaces can be implemented.
\end{itemize}
While classes are typed assuming \lstinline{this} is of the nominal type of the
class, trait declarations, do not introduce any nominal type.  \lstinline{this}
in a trait is typed with a special type \lstinline{This} that is visible only
inside such trait. Syntactically, \Q@This@ is just a special, reserved, class name $\mC$.
A Literal can use the \lstinline{This} type,
and when flattening completes creating a class definition, \Q@This@ will be replaced with such class name.

For the sake of simplicity, method bodies are just simple expressions
\me: they can be just variables, types and method calls. We need types as part of expressions in order to use them as receivers for static methods.

\subsection{Remarks on Typing}
 Our typing discipline is 
what distinguishes our approach from a simple minded code composition macro~\cite{bawden1999quasiquotation}
or a rigid module composition~\cite{ancona2002calculus}.
There are two core ideas:

\paragraph{1: traits are \emph{well-typed} before being reused.}
 For example in

\saveSpace\begin{lstlisting}
t:{method int m() 2 
   method int n() this.m()+1}
\end{lstlisting}\saveSpace

\noindent \Q@t@ is well typed since \Q@m()@ is declared inside of \Q@t@, while

\saveSpace\begin{lstlisting}
t1:{method int n() this.m()+1} 
\end{lstlisting}\saveSpace
\noindent would be ill typed.

\paragraph{2: code literals are not required to be \emph{well-typed} before flattening.}${}_{}$\\*
In class expressions  $\use\ \overline\mV$,
an \mL\ in $\overline\mV$ is not typechecked before flattening, and only the result is expected to be well-typed.
While this seems a dangerous approach at first, consider that also Java has the same behaviour:
for example in
\saveSpace\begin{lstlisting}[language=Java]
  class A{ int ma() {return 2;}  int n(){return this.ma()+1;} }
  class B extends A{ int mb(){return this.ma();} }
\end{lstlisting}\saveSpace
\noindent in \Q@B@ we can call \lstinline{this.ma()} even if in the curly braces there is no declaration for \Q@ma()@.
In our example, using the trait \Q@t@

\saveSpace\begin{lstlisting}
C: Use t {method int k() this.n()+this.m()}
\end{lstlisting}\saveSpace
\noindent would be correct. In the code literal
\Q@{method int k() this.n()+this.m()}@, 
 even if \Q@n@, \Q@m@ are not locally defined, in 
\name the result of the flattening is well typed.
This is not the case in many similar works in literature~\cite{deep,Bettini2015282,Bergel2007} where the
literals have to be \emph{self contained}. In this case we would have been forced to
declare abstract methods \Q@n@ and \Q@m@, even if \Q@t@ already 
provides such methods.

Our typing strategy has two important properties:
\begin{itemize}
\item If a class is declared by using $\mC : \use\ \overline\mt$, that is without literals,
and the flattening is successful then \mC\ is well typed: there is no need of further checking.
\item On the other side, if a class is declared by $\mC : \use\ \overline\mV$, with
$\mL_1\ldots\mL_n \in \overline\mV$, and after successful flattening $\mC : \mL$ can not be typechecked,
then the issue was originally present in one of $\mL_1\ldots\mL_n$.
It may be that the result is intrinsically ill-typed, if one of the methods in $\mL_1\ldots\mL_n$ is not well typed,
but it may also happen that a type referred from one of those methods
is declared \emph{after} the current class. As we will see later, this
is how our relaxation allows us to support (indirectly) recursive types.

This also means that as an optimization strategy
 we may remember what method bodies come from traits and what method bodies come from code literals, in order to typecheck only the latter.
 \end{itemize}

 \subsection{Recursive types}

OO languages leverage on recursive types most of the times.
In a pure OO language, \Q@String@ may offer a \Q@Int size()@
method, and \Q@Int@ may offer a \Q@String toString()@ method.
This means that typing classes 
\Q@String@ and \Q@Int@ in isolation is not possible.

The most expressive compilation process may divide the program in groups of mutually 
dependent classes.
Each group may also depend on a number of other groups.
This would form a \emph{direct acyclic graph} of groups.
To type a group, we first need to type all depended groups, then
we can extract the structure/signature/structural type of all
the classes of the group.
Now, with the information of the depended groups and the one extracted
from the current group, it is possible to typecheck the implementation
 of each class in the group.
In this model, it is reasonable to assume that flattening happens group by group, before
extracting the class signatures.

Here we go for a much simpler top down execution/interpretation for flattening, where flattening
happens one at the time, and classes are typechecked where their type is first needed.
That is, in \name typing and flattening are interleaved. We assume our compilation process to stop as soon as 
an error arises. There are two main kinds of errors: \emph{type errors} (like method not found) or \emph{composition errors} (like summing two conflicting implementation for the same method).
For example
\saveSpace\begin{lstlisting}
A:{method int ma(B b) b.mb()+1}
tb:{method int mb() 2}
tc:{method int mc(A a,B b) a.ma(b)}
B: Use tb
C: Use tc, {method int hello() 1}
\end{lstlisting}\saveSpace
In this scenario, since we go top down, we first need to generate \Q@B@.
To generate \Q@B@, we need to use \Q@tb@.
In order to modularly ensure well typedness,
we require \Q@tb@ to be well typed at this stage. If \Q@tb@ was not well typed
a type error could be generated at this stage.
At this moment, \Q@A@ cannot be compiled/checked alone:
information about \Q@B@ is needed, but \Q@A@ is not used in \Q@tb@,
thus we do not need to type \Q@A@ and we can type \Q@tb@ with
 the available informations and proceed to generate \Q@B@.
Now, we need to generate \Q@C@, and we need to ensure well typedness of \Q@tc@.
Now \Q@B@ is already well typed (since generated by \use\ \Q@tb@, with no \mL),
and \Q@A@ can be typed. Finally \Q@tc@ can be typed and used.
If \use\ could not be performed (for example it \Q@tc@ had a method \Q@hello@ too)
a composition error could be generated at this stage.
On the opposite side, if \Q@B@ and \Q@C@ were swapped, as in
\saveSpace\begin{lstlisting}
C: Use tc, {method int hello() 1}  
B: Use tb
\end{lstlisting}\saveSpace
\noindent
now the first task would be to generate \Q@C@, but 
to type \Q@tc@ we need to know the type of \Q@A@ and \Q@B@.
However they are both unavailable: \Q@B@ is still not computed and 
\Q@A@ cannot be compiled/checked without information about \Q@B@.
A type error would be generated, on the line of ``flattening of \Q@C@
requires \Q@tc@, \Q@tc@ requires \Q@A@,\Q@B@, but \Q@B@ is still in need of flattening".

In this example, a more expressive compilation/precompilation process 
could compute a dependency graph and, if possible, reorganize the list,
but for simplicity lets consider to always provide the declarations
in the right order, if one exists.

\paragraph{Criticism: existence of an order is restrictive.}${}_{}$\\*
Some may find the requirement of the existence of an order restrictive;
An example of a ``morally correct" program where no right order exists is the following:
\saveSpace\begin{lstlisting}
t:{ int mt(A a) a.ma()}
A:Use t {int ma() 1}
\end{lstlisting}\saveSpace

In a system without inference for method types,
if the result of composition operators depends only on the
structural shape of their input (as for \use)
is indeed possible to optimistically compute the resulting structural shape of the classes
and use it to type involved examples like the former.
We stick to our simple approach, since we believe such typing discipline would be fragile,
and could make human understanding the code-reuse process much harder/involved.
Indeed we just wrote an involved program where the correctness of trait \Q@t@ depends of 
\Q@A@, that is in turn generated using trait \Q@t@.

\paragraph{Criticism: it would be better to typecheck before flattening.}${}_{}$\\*
In the world of strongly typed languages we are tempted to
first check that all can go well, and then perform the flattening.
This would however be overcomplicated without any observable difference:
Indeed, in the \Q@A,B,C@ example above there is no difference
between
\begin{itemize}
\item  (1)First check \Q@B@ and produce \Q@B@ code (that also contains \Q@B@ structural shape),
  (2) then use \Q@B@ shape to check \Q@C@ and produce \Q@C@ code;\ 
or a more involved
\item  (1)First check \Q@B@ and discover just \Q@B@ structural shape as result of the checking,
  (2)then use \Q@B@ shape to check \Q@C@.
  (3) Finally produce both \Q@B@ and \Q@C@ code.
\end{itemize}

Note that we can reuse code only by naming traits; but our only point of relaxation is {\bf only} the code literal: there is no way an error can ``move around'' and be duplicated during the compilation process.
In particular, our approach allows for safe libraries of traits and classes to be fully typechecked, deployed and reused by multiple clients: no type error will emerge from library code.
On the other side, we do not enforce the programmer to always write self-contained code where all the abstract method definition are explicitly declared.

\saveSpace\saveSpace
\section{Related Work}
\saveSpace\saveSpace
Literature on code reuse is too vast to let us do justice of it in a few pages.
Our work is inspired by traits~\cite{ducasse2006traits}, which in turn
are inspired by module composition languages~\cite{ancona2002calculus}.

We claim that our presented solution to the expression problem is the most natural in literature to date.
While a similar syntax can be achieved with the scandinavian style~\cite{ernst2004expression}, their dependent type system makes reasoning quite complex, and indeed more recent solutions have accepted a more involved syntax in order to have a much simpler type system~\cite{igarashi2005lightweight}.
Challenging the expression problem, our close contented is DeepFJig~\cite{deep}: all our gain over their model is based on our relaxation over abstract signatures.


\saveSpace
\subsection{Separating Inheritance and Subtyping}
We are aware of at least 3 independently designed research languages 
that address the this-leaking problem: TraitRecordJ~(TR)\cite{Bettini:2010:ISP:1774088.1774530,BETTINI2013521,Bettini2015282}, Package Templates~(PT)\cite{KrogdahlMS09,DBLP:journals/taosd/AxelsenSKM12,DBLP:conf/gpce/AxelsenK12}, DeepFJig~\cite{deep,servetto2014meta,fjig}.
Levering on \emph{traits}, in this work we aim to synthesize
the best ideas of those very different designs, while at the same time 
coming up with a simpler and improved design for separating
subclassing from subtyping, which also addresses various limitations of those
3 particular language designs.
The following compares 
various aspects of the language designs;
we underline 3 properties where one approach shines the most, and 3 properties where one approach is more lacking.
\begin{itemize}
\item {\bf A simple uniform syntax for code literals}
DeepFJig is best in this sense, since TR has separate syntax for class literals, trait literals and record literals.
PT on the other hand is built on top of full Java, thus has a very
involved syntax.
\name leverages on DeepFJig's approach but,
\emph{thanks to our novel representation of state}, \name also offers a much simpler and uniform syntax than
all other approaches: everything is just a method.
\item 
{\bf Reusable code cannot be ``used'', that is instantiated or used as a type.}
This happens in TR and in PT, but not in DeepFJig. To allow reusable code to be directly 
usable, in DeepFJig
classes introduce nominal types in an unnatural way: the type of
\Q@this@ is only \Q@This@ (sometimes called \Q@<>@) and not the
nominal type of its class. 
That is in DeepFJig 
`\Q@A:{ method A m()this}@' is not well typed. This is because
`\Q@B: Use A@' flattens to `\Q@B:{ method A m()this}@', which is clearly not well typed.
Looking to this example is clear why we need reusable code to be agnostic of its name.
Then, either reusable code has no name (as in TR, PT and \name)
or all code is reusable and usable, and all code needs to be awkwardly agnostic of its name, as in DeepFJig.

\item 
{\bf Requiring abstract signatures is a left over of module composition mindset.}
TR and DeepFJig comes from a tradition of functional module composition, where 
modules are typed in isolation under an environment, and then the composition is performed.
As we show in this work, this ends up requiring verbose repetition of abstract signatures,
which (for highly modularized code) may end up constituting most of the program.
Simple Java (and thus PT, since it is a Java extension) shows us a better way:
the meaning of names can be understood from the reuse context.
The typing strategy of PT offers the same advantages of our typing model, 
but is more involved and indirect. This may be caused by the
heavy task of integrating with full Java.
Recent work based on TR is trying to address this issue too~\cite{damiani2017unified}.
\item {\bf Composition algebra.}
The idea of using composition operators over atomic values as in an arithmetic expression is very powerful,
and makes it easy to extend languages with more operators. DeepFJig and TR embrace this idea, while PT takes the traditional Java/C++ approach of using enhanced class/package declaration syntax.
The typing strategy of PT also seems to be connected with this
decision, so it would be hard to move their approach in a composition
algebra setting.
\item {\bf Complete ontological separation between use and reuse}
While all 3 works allow separating inheritance and subtyping only TR properly enforces 
separation between use (classes and interfaces) and reuse (traits).
This is because in DeepFJig all classes are both units of use and reuse (however, subtyping is not induced).
PT imports all the complexity of Java, so although is possible to separate use and reuse, the model have powerful but non-obvious implications where (conventional Java) \Q@extends@ and PT are used together.
\item {\bf Naming the self type, even if there is none yet.}
Both DeepFJig and PT allow a class to refer to its name, albeit this is
less obvious in PT since both a package and a class have to be introduced to express it.
This allows encoding binary methods, expressing patterns like withers or fluent setters and to instantiate instances of the (future) class(es)  using the reused code.

\end{itemize}

\subsection{State and traits}
The original trait model has no self construction 
and purposely avoided any connection between state and traits.
It was applied to a dynamically typed language, so
it was not clear if the author intended of modelling `\Q@This@' type.


The idea of abstract state operations emerged from Classless
Java~\cite{wang2016classless}. This approach offers a clean solution to handle state
in a trait composition setting.
Note how abstract state operations are different from just hiding fields under getter and setters: 
in our model the programmer simply never has to declare what is the state of the class, not even what information is stored in fields.
The state is computed by the system as an overall result of the whole code composition process.

In the literature there have been many attempts to add state in traits/module composition languages:
\begin{itemize}  
\item An early approach is to have {\bf no constructors}: all the fields start with {\bf null} or a default specified value.
  Fields are just like another kind of (abstract) member, and two fields
  with identical types can be merged by sum/use; \Q@new C()@ can be used for all classes, and \Q@init@ methods may be called later, as in
  \Q@Point p=new Point(); p.init(10,30)@.
  
  To its credit, this simple approach is commutative and associative and does not disrupt elegance of summing methods.
  However, objects are created "broken" and the user is trusted with fixing them.
  While it is easy to add fields, the load of initializing them is on the user; moreover
    all the objects are intrinsically mutable, so this model is unfriendly
    to a functional programming style.
\item {\bf Constructors compose fields}:
In this approach (used by \cite{fjig}) the fields are declared but not initialized, and
a canonical constructor (as in FJ) taking a value for each field and just initializing such field
is automatically generated in the resulting class.
It is easy to add fields, however this model is associative but not commutative: composition order influences field order, and thus the constructor signature.
Self construction is not possible 
since the signature of the constructors changes during composition.

\item {\bf Constructors can be composed if they offer the same exact parameters}:
In this approach (used by DeepFJig) traits declare fields and constructors.
The constructor initializes the fields but can do any other computation.
Traits whose constructors have the same signature can be composed.
The composed constructor will execute both constructor bodies in order.
This approach is designed to allow self construction.
It is also associative and mostly commutative: composition order only influences execution order of side effects during construction.
However constructor composition requires identical constructor signatures: this
hampers reuse, and if a field is added, its initial value needs to be
somehow synthesized from the constructor parameters.

\end{itemize}

\subsection{Tablular comparision of many approaches}
\begin{minipage}[t]{0.30\textwidth}
In this table we show if some constructs support certain features:
Direct instantiation (as in \Q@new C()@),
Self instantiation (as in \Q@new This()@),
Is this construct a `Unit of use'?, a `Unit of reuse'?,
Does using this construct introduce a type? and is the induced type the type of \Q@this@?,
support for binary methods,
does inheritance of this construct induce subtype?,
is the code of this construct required to be well-typed before being inherited /imported in a new context?
is it required to be well-typed before being composed with other code?
\end{minipage}
%second column
\begin{minipage}[t]{0.6\textwidth}
\newcommand{\YY}{\textbf{Y}}
\begin{center}
\begin{tabular}{c|c|c|c|c|c|c|c|c|c|c}
&\Rotated{direct instantation}
&\Rotated{self instantiation}
&\Rotated{unit of use}
&\Rotated{unit of reuse}
&\Rotated{introduce type}
&\Rotated{induced type is this type}
&\Rotated{binary methods}
&\Rotated{{${}_{}$\!inheritance induce subtype\!\!\!}}
&\Rotated{{${}_{}$\!well-typed before imported\!\!\!}}
&\Rotated{{${}_{}$\!well-typed before composed\!\!\!}} 
\\
\hline
java/scala class&\YY &X&\YY &\YY &\YY &\YY &X&\YY &\YY &X\\
java8 interface &X&X&X&\YY &\YY &\YY       &X&\YY &\YY &X\\
scala trait        &X&X&X&\YY &\YY &\YY    &-&\YY &\YY&X\\
original trait     &X&X&X&\YY &-&-         &-&X&-&-\\
TR  &X&X&X&\YY &X&-                        &X&X&\YY &\YY \\
\name trait        &X&\YY &X&\YY &X&-      &\YY &X&\YY &X\\
\name class        &\YY &\YY &\YY &X&\YY   &\YY &\YY &-&\YY &-\\
module composition
                      &-&-&\YY &\YY &-&-   &-&-&\YY &\YY \\
deepFJig class &\YY &\YY &\YY &\YY &\YY &X &\YY &X&\YY &\YY \\
package template
                      &X&\YY &X&\YY &X&-   &-&X&\YY &X\\
${}_{}$\\
\end{tabular}
\end{center}
\end{minipage}

\noindent \textbf{Y} and X means yes and no, and we use ``-'' where the question is not really applicable to the current approach. For example the original trait model was untyped, so typing questions makes no sense here.

\subsection{ThisType with Subclassing implying Subtyping}
With the exception of those mentioned 3 lines of work, to the best of our understading
other famous work in literature, like~\cite{odersky2008programming,nystrom2006j}
do not completely break the relation between inheritance and subtyping, but only prevent subtyping where 
it would be unsound.
Recent work on {\bf ThisType} \cite{Saito:2009,ryu16ThisType}
also continues on this line.
In those works, ``subtyping by subclassing'' is preserved, which means
that those designs are more suitable to retain the programming model
of mainstream OOP languages and backwards compatibility. The design 
of \name (and 42) is a more radical departure of mainstream OOP, with
the hope to improve both the mechanisms for use and reuse in OOP.



\section{Related work} 

The operator redirect fist emerged with DeepFJig[],
and was then explored more in MetaFJig[];
however, in this line of research, it was only
able to redirect one nested class at a time, thus it was
impossible to redirect mutually recursive classes.

We apply redirect over a trait composition language[sharly].
We are particularly inspired by Ferruccio, MetaFJig
(and UseReuse).
The main difference in the compilation model with respect to DeepFJig is an improved
treatment of $\This{n}$: in MetaFJig $\This{n}$ wrote in
class \Q@A@ is not equivalent to $\This{(n+1)}{\Q@A@}$.
Moreover, the formalism is simpler and more compact.

Redirect supports a similar expressive power 
with respect to abstract types refinement,
and in some sense here is proposed
as an alternative to them.
A good formal discussion about Abstract types
can be found in [
Foundations of Path-Dependent Types].

\section{Conclusions} 

\section{Appendix?}

PUT LATER?However, he type system of the language is more restrictive when 
it comes to refine an interface method, allowing only return type refinement. This is not just to align our calculus with existing languages like Java/C\# and C++, but is required to make reasoning about parameter types influential while expanding redirect mappings.END PUT LATER

\begin{bnf}
\production{%
\ctx{V}}{\hole \mmid{}  \summ{\ctx{V}}{E} %
                \mmid{}  \summ{LV}{\ctx{V}} \mmid{} \red{\ctx{V}}{Cs}{T}}  {context of library-evaluation}\\\production{%
%LV}     {\libi{Tz}{amtz}{}\ \ \mmid{}\ \ \libc{Tz}{MVs}{K$?$}}       {literal value}\\\production{%
%MV}     {C\eq{}LV \mmid{} mt}                                                    {}\\\production{%
\ctx{v}}{\hole \mmid{}  \ctx{v}\Q{.}m\rp{es} \mmid{}  v\Q{.}m\rp{vs \ctx{v} es} %
	\mmid{} T\Q{.}m\rp{vs \ctx{v} es}  }           {}%\\\production{%
%DL}     {id\eq{}L}                                                         {partially-evaluated-declaration}\\\production{%
%DV}     {id\eq{}LV}                                                       {evaluated-declaration}\\\production{%
%Mid}    {C \mmid{} m}                                                      {member-id}\\\production{%
%p}      {DLs\Q{;} DVs}                                                     {program}
\end{bnf}



\section{Type System}

The type system is split into two parts: type checking programs and class literals, and the typechecking of expressions. The latter part is mostly convential, it involves typing judgments of the form $p; Txs \vdash e : T$, with the usual program $p$ and variable environement $Txs$ (often called $\Gamma$ in the literature). rule ($Ds ok$) type checks a sequence of top-level declarations by simply push each declaration onto a program and typecheck the resulting program.
Rule $p ok$ typechecks a program by check the topmost class literal: we type check each of it’s members (including all nested classes), check that it properly implements each interface it claims to, does something weird, and finanly check check that it’s constructor only referenced existing types,

\begin{verbatim}


Define p |- Ok
===========================================================

D1; Ds |- Ok ... Dn; Ds|- Ok
(Ds ok) ------------------------------ Ds = D1 ... Dn
Ds |- Ok

p |- M1 : Ok .... p |- Mn : Ok
p |- P1 : Implemented .... p |- Pn : Implemented
p |- implements(Pz; Ms) /*WTF?*/                   if K? = K: p.exists(K.Txs.Ts)
(p ok) ------------------------------------------- p.top() = interface? {P1...Pn; M1, ..., Mn; K?}
p |- Ok

p.minimize(Pz) subseteq p.minimize(p.top().Pz)
amt1 _ in p.top().Ms ... amtn _ in p.top().Ms
(P implemented) ----------------------------------------------- p[P] = interface {Pz; amt1 ... amtn;}
p |- P : Implemented

(amt-ok) ------------------- p.exists(T, Txs.Ts)
p |- T m(Tcs) : Ok

p; This0 this, Txs |- e : T
(mt-ok) ------------------------------ p.exists(T, Txs.Ts)
p |- T m(Tcs) e : Ok

C = L, p |- Ok
(cd-Ok) -------------------
p |- C = L : OK

\end{verbatim}

Rule $(P implemented)$ checks that an interface is properly implemented by the program-top, we simply check that it declares that it implements every one of the interfaces super-interfaces and methods.
Rules $(amt-ok)$ and $(mt-ok)$ are straightforward, they both check that types mensioned in the method signature exist, and ofcourse for the latter case, that the body respects this signature.

To typecheck a nested class declaration, we simply push it onto the program and typecheck the top-of the program as before.


The expression typesystem is mostly straightforward and similar to feartherwieght Java, notable we we use $p[T]$ to look up information about types, as it properly ‘from’s paths, and use a classes constructor definitions to determine the types of fields.

\begin{verbatim}
Define p; Txs |- e : T
=====================================
(var)
----------------------- T x in Txs
p;  Txs |- x : T

(call)
p; Txs |- e0 : T0
...
p; Txs |- en : Tn
-----------------------------------  T' m(T1 x1 ... Tn xn) _ in p[T0].Ms
p; Txs |- e0.m(e1 ... en) : T'

(field)
p; Txs |- e : T
---------------------------------------  p[T].K = constructor(_ T' x _)
p; Txs |- e.x : T'


(new)
p; Txs |- e1 : T1 ... p; Txs |- en : Tn
------------------------------------------- p[T].K = constructor(T1 x1 ... Tn xn)
p; Txs |- new T(e1 ... en)


(sub)
p; Txs |- e : T
-----------------------------------  T' in p[T].Pz
p; Txs |- e : T'


(equiv)
p; Txs |- e : T
-----------------------------------  T =p T'
p; Txs |- e : T'
\end{verbatim}





FROM and minimize that will go in the appendix:

To fetch a trait form a program, we will use notation $p(t)=LV$, to 
fetch a class we will use $p(T)$.

To look up the definition of a class in the program we will use the notation
%$p(t)=LV$ and
$p(T)=\textit{LV}$, which is defined by the following:% but only if the RHS denotes an $LV$:

We will use members $Mz$ as a function containing both method names $m$
and class names $C$ in its domain; thus we will assume
notation $\dom{Mz}$, $Mz(m)$, $Mz(C)$ with the usual meaning.
Under here, we define useful auxiliary notations to
access literals $L$ with functional notation with the intent of accessing their members. We define notations $L[Cs=E]=L'$ and $Mz[C=E]=Mz'$ serving the role of function update.
We use those notations to define $p(T)=LV$ accessing a program $p$ as function. We also define operations on programs: $p\op{push}{D}=p'$, allowing to work with programs as if they was stacks, and
$p\op{min}{T}=T'$, denoting the shortest type $T'$ referring to the same 
nested class of $T$.
We define $\from{T}{T',j}$ and $\from{L}{T,j}$; we omit all the trivial propagation cases of form $\from{M}{T,j}$, $\from{K}{T,j}$ and $\from{e}{T,j}$.
Finally, we we combine those to notation for the
most common task of getting the value of a literal, in a way that can be understand from the current location: $p[t]$ and $p[T]$:


\noindent
\begin{minipage}{0.48\textwidth}
\noindent\!\!\!$\begin{array}{l}
\hline
(\DLs\Q@;@ \DVs)\op{push}{id\eq{L}} =
\id\eq{L},\DLs\Q@;@ \DVs\\
\hline
(; \_, C\eq{L}, \_)(\This{0}{C}{Cs})=\mathit{L(Cs)}\\
p\op{push}{\_\eq{L}}(\This{0}{Cs})=L(\mathit{Cs})\\
p\op{push}{\_}(\This{n+1}{Cs})=p(\This{n}{Cs})\\
\hline
\fop{members}{\lib{\_}{Mz}{\_}}=Mz\\\hline
L(m)=\fop{members}{L}(m)\\\hline
L(C)=\fop{members}{L}(C)\\\hline
\dom{L}=\dom{\fop{members}{L}}\\\hline
\mdom{L}=\{m \in \dom{L}\}\\
\end{array}$
\end{minipage}%
\begin{minipage}{0.5\textwidth}
$\begin{array}{l}
\hline
(Mz,\Q@private@? C\eq{\_})[C=E]=Mz,\Q@private@? C\eq{E}\\\hline
LV({\emptyset})=LV\\
L(C\Q@.@Cs)=L(C)(Cs)\\
\hline
L[\Empty=E] = E\\
\lib{Tz}{Mz}{K?}[C\Q@.@Cs=E]=\\\quad
\lib{Tz}{Mz[C=Mz(C)[Cs=E]]}{K?}
\\\hline
p\op{min}{\This{n+1}{\id_n}{Cs}} = p\op{min}{\This{n}{Cs}}\\\quad
  \text{where }p = \id_0 \eq{L_0}\ldots\id_n\eq{L_n} \_\Q@;@ Ds\\
\text{otherwise } p\op{min}{T} = T\\
\end{array}$
\end{minipage}

\noindent\!\!\!\!
$\begin{array}{l}
\hline
\from{\This{n}{Cs}}{T,j}
=
\This{n}{Cs} \quad \textit{with }n<j
\\
\from{\This{n+j}{Cs}}{\This{m}{C_1\ldots C_k},j}
=
\This{m+j}{C_1\ldots C_{k-n}} \quad \textit{with }n\leq k
\\
\from{\This{n+j}{Cs}}{\This{m}{C_1\ldots C_k},j}
=
\This{m+j+n-k}{C_1\ldots C_{k-n}{Cs}} \quad \textit{with }n> k
\\
\from{
\libc{\Q@interface@? Tz}{Mz}{K}
}{T,j-1}
=
\libc{\Q@interface@? \from{Tz}{T,j}}{\from{Mz}{T,j}}{\from{K}{T,j}}
\\\hline
(DL_1\ldots DL_n; \_, t\eq{LV})[t]
=p\op{min}{\from{LV}{\This{n},0}}\\\hline
p[T]=p\op{min}{
\lib{\from{Tz}{T,0}}{\from{Mz}{T,0}}{}
}
\quad\text{where }p(T)=\lib{Tz}{Mz}{K?}
\\\hline
\end{array}$


sdgsd

\noindent\begin{defye}%
%\defy{(\_; \_, t\eq{}LV, \_)(t)}{\mathit{LV}}%
\defy{(\DLs\Q{;} \DVs)\op{push}{id\eq{L}}}{
id\eq{L},\DLs\Q{;} \DVs}%
\defy{(; \_, C\eq{}L, \_)(\This{0}{C}{Cs})}{\mathit{L(Cs)}}%
%\defy{(id\eq{}L, p)(\This{0}{Cs})}{L(\mathit{Cs})}%
\defy{p\op{push}{\_\eq{L}}(\This{0}{Cs})}{L(\mathit{Cs})}%
%p' id=L,A;B = id=L,p
%\defy{(id\eq{}L, p)(\This{n+1}{Cs})}{p(\This{n}{Cs})}%shorter version, not wrong but confusing
\defy{p\op{push}{\_}(\This{n+1}{Cs})}{p(\This{n}{Cs})}%
\defy{LV({\emptyset})}{LV}%
\defy{\lib{\_}{\_, \Q@private@?\,C\eq{L_0}, \_}{\_}(\Cs{C}{\s{C}})}{L_0(Cs) }%\text{ where } L = \lib{\_}{\_, C\eq{L_0}, \_}{\_}}%
	\defyc{\text{where } L = a}
\end{defye}


This notation just fetch the referred $LV$ without any modification.
To adapt the paths we define $\from{T_0}{T_1,j}$, $\from{L}{T,j}$ and $p\op{minimize}{T}$ as following:
\begin{defye}%
  \defy{(DL_1\ldots DL_n; \_, t\eq{LV},\_)[t]}{\from{LV}{\This{n}}}
	\defy{p[T]}{p\op{minimize}{\from{p(T)}{T}}}%
\end{defye}
\\${}_{}$\\






--
towel1:.. //{Map:{} }
towel2:.. //{Map:{} }
lib:{
  T:towel1
  f1
  ...
  fn
  }

MyProgram:{
  T:towel2
  Lib:lib[.T=This0.T]
  ...
  }
-- 
\section{extra}

Features:
Structural based generics embedded in a nominal type system.
Code is Nominal, Reuse is Structural.
Static methods support for generics, so generics are not just a trik to make the type system happy but actually
change the behaviour
Subsume associate types.
After the fact generics; redirect is like mixins for generics
Mapping is inferred-> very large maps are possible -> application to libraries


In literature, in addition to conventional Java style F-bound polymorphism, there is
another way to obtain generics: to use associated types (to specify generic paramaters) and inheritence (to instantiate the paramaters).
However, when parametrizing multiple types, the user to specify the full mapping.
For example in Java
    interface A<B>{ B m(); }
    inteface B{String f();}
    class G<TA extends A<TB>, TB>{//TA and TB explicitly listed
      String g(TA a TB b){return a.m().f();}
    }
    class MyA implements A<MyB>{..}
    class MyB implements B {..}
    G<MyA,MyB>//instantiation
Also scala offers genercs, and could encode the example in the same way, but Scala
also offers associated types, allowing to write instead....

Rust also offers generics and associated types, but also support calling static methods
over generic and associated types.

We provide here a fundational model for genericty that subsume the power
of F-bound polimorphims and  associated types.
Moreover, it allows for large sets of generic parameter instantiations to be inferred starting from a much smaller mapping.
For example, in our system we could just write
    g={
      A={ method B m()}
      B={ method String f()}
      method String g(A a B b)=a.m().f()
    }
    MyA={ method MyB m()= new MyB(); ..}
    MyB={ method String f()="Hello"; ..}
    g<A=MyA>//instantiation. The mapping A=MyA,B=MyB

We model a minimal calculus with interfaces and final classes, where implementing an interface is the only way to induce subtyping.
We will show how supporting subtyping constitute the core technical difficulty in our work, inducing ambiguity in the mappings.
As you can see, we base our generic matches the structor of the type instead of respecting a subtype requirement as in F-bound polymorphis.
We can easily encode subtype requirements by using implements:
Print=interface{ method String print();}
g={
  A:{implements Print}
  method A printMe(A a1,A a2){ if(a1.print().size()>a2.print.size()){return a1;} return a2;}
  }
MyPrint={implements Print ..}
g<A=MyPrint> //instantiation
g<A=Print> //works too


--------------
example showing ordering need to strictly improve
EI1: {interface}
EA1: {implements EI1}

EI2: {interface}
EA2: {implements EI2}

EB: {EA1 a1 EA1 a1}

{
A1: {}
A2: {}
B: {A1 a1 A2 a2}
}[B = EB] // A1 -> EI1, A2 -> EA2 a
          // A1 -> EA1, A2 -> EI2 b
          // A1 -> EA1, A2 -> EA2 c

a <=b
b <=a
c<= a,b
a <= c

\Q@hi@
\Q@Hi@
\Q@class@

$ aa \Q@hi@
\Q@Hi@
\Q@class@  qaq$
\begin{bnf}
	\production{a}{b}{c}\\
	\production{a}{b}{c}\\
	\production{a}{b}{c}\\
\end{bnf}

$\Q@}}][()]@$

$\begin{array}{l}
\inferrule[(top)]{
	a \xrightarrow[b]{} c\quad
	\forall i<3 a\vdash b:\text{OK}\\\\
	\forall i<3 a\vdash b:\text{OK}
}{
	1+2
	\rightarrow
	3
}\begin{array}{l}
a\\b\\c
\end{array}
\end{array}$

%\appendix
\section{Proof} 
\label{s:proof}

\begin{theorem}[Sound Validation]
	if $c:\Kw{Cap};\emptyset\vdash \e: \T$ and
	$c\mapsto\Kw{Cap}\{\_\}|\e\rightarrow^+ \sigma|\ctx_v[r_l]$, then
	either $valid(\sigma,l)$ or $\mathit{trusted}(\ctx_v,r_l)$.
\end{theorem}

We believe this property captures very precisely our statement in Section~\ref{s:validation}.

It is hard to prove Sound Validation directly,
so we first define a stronger property,
called \emph{Stronger Sound Validation} and
show that it is preserved during reduction by means of conventional 
Progress and Subject Reduction (Progress is one of our assumption,
while Subject Reduction relies heavily on SubjectReductionBase).
That is,
Progress+Subject Reduction $\Rightarrow$ Stronger Sound Validation,
\\*and Stronger Sound Validation $\Rightarrow$ Sound Validation.

\subsection{Stronger Sound Validation $\Rightarrow$ Sound Validation}

Stronger Sound Validation depends on 
$\mathit{wellEncapsulated}$, $\mathit{monitored}$
and $OK$:

\noindent\textbf{Define} $\mathit{wellEncapsulated}(\sigma,\e,l_0)$:\\*
\indent$\forall l \in \mathit{erog}(\sigma,l_0), \text{not}\ \mathit{mutatable}(l,\sigma,\e)$

\noindent The main idea is that an object is well encapsulated if its encapsulated state is safe from
modification. 

\noindent\textbf{Define} $\mathit{monitored}(\e,l)$:\\*
\indent$\e=\ctx_v[M(l,\e_1;\e_2)]$ and either $\e_1=l$ or $l$ is not inside $\e_1$.

\noindent An object is monitored if the execution
is currently inside of a monitor for that object, and
the monitored expression $\e_1$ does not
contains $l$ as a \emph{proper} subexpression.

A monitored object is associated with an expression that can not observe it, but may 
reference its internal representation directly.
In this way, we can safely modify its representation before checking for the invariant.

The idea is that at the start the object will be valid and $\e_1$ will contain $l$;
but during reduction, the $l$ reference will be used in order to
give access to the internal state of $l$; only after that moment, the object may become invalid.


\noindent\textbf{Define} $OK(\sigma,e)$:\\
\indent $\forall l\in\dom(\sigma)$
  either\\
\indent\indent 1. $\mathit{garbage}(l,\sigma,\e)$\\
\indent\indent 2. $\mathit{valid}(\sigma,l)$ and $\mathit{wellEncapsulated}(\sigma,\e,l)$\\
\indent\indent 3. $\mathit{monitored}(\e,l)$

Finally, the system is in a valid state with respect to validation
if for all the objects in the memory, one of these 3 cases apply:
%the class of the object has no invariant method;
the object is not (transitively) reachable from the expression (thus can be garbage collected);
the object is valid, and the object is encapsulated;
or the object is currently monitored.

\begin{theorem}[Stronger Sound Validation]
if $c:\Kw{Cap};\emptyset\vdash \e_0: \T_0$ and
$c\mapsto\Kw{Cap}\{\_\}|\e_0\rightarrow^+ \sigma|\e$, then
$OK(\sigma,\e)$
\end{theorem}
\noindent Starting from only the capability object,
any well typed expression $\e_0$ can be reduced for an arbitrary amount of steps,
and $IOK$ will always hold.
\\
\begin{theorem} Stronger Sound Validation $\Rightarrow$ Sound Validation
\end{theorem}
\begin{proof}
\noindent By Stronger Sound Validation, each $l$ in the current redex must be $OK$:
\begin{enumerate}
	\item If $l$ is garbage, it cannot be in the current redex, a contradiction.
	\item If $\mathit{valid}(\sigma,l)$, then $l$ is valid, so thanks to Determinism
	no invalid object could be observed.
	\item Otherwise, if $\mathit{monitored}(\e,l)$ then either:
	\begin{itemize}
	 \item we are executing inside of $\e_1$ thus the current redex is inside of a sub-expression of the monitor that does not contain $l$, a contradiction.
	 \item or we are executing inside $\e_2$:
	 by our reduction rules, all monitor expressions start with 
	 $\e_2=l$\Q@.validate()@, thus the first execution step
	 of $\e_2$ is trusted. Following execution steps are also trusted, since by well formedness the body of invariant methods only use \Q@this@ (now translated to $l$) to access fields.
	\end{itemize}
\end{enumerate}
In any of the possible cases above, Sound Validation holds for $l$, and so it holds for all redexes.
\end{proof}

\subsection{Subject Reduction}

\noindent\textbf{Define} $\text{fieldGuarded}(\sigma,\e)$:\\*
\indent$\forall \ctx$ such that $\e=\ctx[l\singleDot\f] $
and $\Sigma^\sigma(l).f=\Kw{capsule}\,\_$, and $\f\mathrel{\mathit{inside}} \Sigma^\sigma(l).\mathit{validate}$\\*
\indent\indent either 
$\forall T, \forall C, \Sigma^\sigma;\x:\Kw{mut}\,C\,\not\vdash\ctx[\x]:T$, or\\*
\indent\indent $\ctx=\ctx'[$\Q@M(@$l$\Q@;@$\ctx''$\Q@;@$\e$\Q@)@$]$ and $l$ is contained exactly once in $\ctx''$

That is, all \emph{mutating} capsule field accesses are individually guarded by monitors.
Note how we use $C$ in $\x:\Kw{mut}\,C$ to guess the type of the accessed field,
and that we use the full context $\ctx$ instead of the evaluation context $\ctx_v$
to refer to field accesses everywhere in the expression $\e$.


\begin{theorem}[Subject Reduction]
if $\Sigma^{\sigma_0};\emptyset\vdash e_0: T_0$,
$\sigma_0|e_0\rightarrow \sigma_1|e_1$,
$OK(\sigma_0,\e_0)$
and
$\mathit{fieldGuarded}(\sigma_0,\e_0)$
then
$\Sigma^{\sigma_1};\emptyset\vdash e_1: T_1$,
$OK(\sigma_1,e_1)$ and
$\mathit{fieldGuarded}(\sigma_1,\e_1)$
\end{theorem}

\begin{theorem}
	Progress + Subject Reduction $\Rightarrow$ Stronger Sound Validation
\end{theorem}
\begin{proof}
This proof proceeds by induction in the usual manner.

\emph{Base Case}: At the start of the execution, the memory is going to only contain $c$: since $c$ is defined to be initially $\mathit{valid}$, and has only \Q@mut@ fields, and so it is trivially $\mathit{wellEncapsulated}$, thus $OK(c\mapsto\Kw{Cap},e)$.

\emph{Induction}: By Progress we always have another evaluation step to take, by Subject Reduction such a step will preserve $\mathit{OK}$, and so by induction $\mathit{OK}$ holds after any number of steps.

Note how for the proof garbage collection is important: 
when the \Q@validate()@ method evaluates to \Q@false@, 
execution can continue only if the offending object is classified as garbage.
\end{proof}

\subsection{Proof of Subject Reduction}
We first introduce a lemma derived from well formedness and the type system:
\begin{Lemma}[ExposerInstrumentation]
If $\sigma_0 | \e_0\rightarrow \sigma_1 |\e_1$ and
$\text{fieldGuarded}(\sigma_0,\e_0)$
\\*
then $\text{fieldGuarded}(\sigma_1,\e_1)$
\end{Lemma}
\begin{proof}
The only rule that can 
introduce a new field access is \textsc{mcall}.
In that case, ExposerInstrumentation holds
by well formedness (all field accesses in methods are of the form \Q@this.f@) 
and since \textsc{m call} inserts a monitor while invoking capsule mutator methods, and not field accesses themselves. If however the method is not a \Q@mut@ method but still accesses a capsule field, by MutField such a field access expression cannot be typed as \Q@mut@ and so no monitor is needed.

Note that \textsc{monitor exit} is fine because monitors are removed only when
 $e_1$ is a value.
\end{proof}

\begin{theorem}
	Subject Reduction Base $\Rightarrow$ Subject Reduction
\end{theorem}
\begin{proof}
Any reduction step can be obtained
by exactly one application of rule \textsc{ctx} and then one other rule.



Thus the proof can simply proceed by cases on such other applied rule.

By SubjectReductionBase and ExposerInstrumentation, 
$\Sigma^{\sigma_1};\emptyset\vdash e_1: T_1$ and  $\mathit{fieldGuarded}(\sigma_1,\e_1)$. So we just need to proceed by cases on the reduction rule applied to verify that $OK(\sigma_1,\e_1)$:


\begin{enumerate}
\item \textsc{update:} $\sigma|l\singleDot f\equals v\rightarrow \sigma'|\e'$:
\begin{itemize}
  \item by \textsc{update} $\e'=\Kw{M}\oR l;l;l\singleDot\text{validate}\oR\cR\cR;$, thus $\mathit{monitored}(\e,l)$.
  \item Every $l_1$ such that $l\in \text{rog}(\sigma,l_1)$ will verify the same case
  as the former step:
  \begin{itemize}
  	\item If it was $\mathit{garbage}$, clearly it still is.
  	\item If it was $\mathit{monitored}$, it also still is.
  	\item If can't have been $\mathit{wellEncapsulated}$ since $mutatable(l, \sigma, e)$, (by MutField)
  \end{itemize}
  \item Every other $l_0$ is not reached by $l$ thus it being $\mathit{OK}$ could not have been effected by this reduction step.
\end{itemize}

\noindent\textbf{case field access} $l.f\rightarrow v$:

    If for $l$ $IOK$ holds by (2),  
    it is possible that the next step is not encapsulated.
    This would mean that the field $f$ is a capsule and that we are required
to type it as \Q@mut@ to type the expression for the next step.
By $\mathit{fieldGuarded}(\sigma_0,\e_0)$
    the former step was inside of a monitor \Q@M(@$l$\Q@;@$\ctx_v[l$\Q@.f@$]$\Q@;@$\e$\Q@)@
    and the $l$ under reduction was the only occurrence of $l$.
    since $f$ is a capsule, we know that $l\notin \text{erog}(\sigma,l)$
    by HeadNonCircular.
    Thus in the new step not $l\, \text{inside}\ \ctx_v[v]$.
    Thus for l (3)[monitored] holds.
    
We still need to show that properties $\mathit{monitored}$ and $\mathit{wellEncapsulated}$
 for other objects are
not disturbed. This is the point where our aliasing and mutability control are most crucial:
We know that mutable $v$ is (directly) reachable from
$l$ that have invariant.
Thanks to CapsuleTree we know that for all $l_0$ reaching $l$,
$v$ can be reached by $l_0$ only passing trough $l$.
Thus, we can conclude  $l_0$ is not encapsulated in the former step (containing mutable $l$).
Thus, $l_0$ is either without invariant, garbage or monitored.
None of those 3 cases can be disturbed by a field access.


\noindent\textbf{case meth call}:\\*
  This reduction step does not influence any object in the memory and does not
disturb the properties $\mathit{monitored}$ and $\mathit{wellEncapsulated}$.

\noindent\textbf{case new}:\\*
  If $C$ has invariant, then by @ConstructionInstrumentation the new object is monitored.
As for the method call, other objects and properties are not disturbed.


\noindent\textbf{case monitor exit} \Q@M(@$l;v;$\Q@true)@$\rightarrow v$ :
  \begin{itemize}
\item
    If it was a setter $v=l$, and 
    thanks to Determinism the execution of invariant is deterministic;
    thus for $l$ in the former step both case (2) and (3) holds.
    In the next step (2) will hold for $l$.
\item
    If it was a capsule mutator method, thanks to Determinism the execution
 of \Q@.validate()@ is deterministic;
    thus for $l$ in the former step both $H$ and case (3) holds.
    Thanks to ExposerInstrumentation $v$ is offered without mutation permissions, so
    In the next step $l$ is encapsulated and (2) will hold.
\item
    If it is was a constructor, 
    then $v$ is encapsulated and thanks to Determinism
    the execution of invariant is deterministic, thus in the next step (2) will hold.
\end{itemize}

\noindent\textbf{case try enter and try ok}
This case do not influence any object in the memory and does not
disturb the properties $\mathit{monitored}$ and $\mathit{wellEncapsulated}$.

\noindent\textbf{case try catch} $\sigma,\sigma_0|\Kw{try}^\sigma \oC\mathit{error}\cC\Kw{catch}\, \e\rightarrow \sigma|\e$:\\*
From the premise we know 
$IOK(\sigma,\sigma_0;\ctx_v[\Kw{try}^\sigma \oC\mathit{error}\cC\Kw{catch}\, \e])$;
thus we need to show
$IOK(\sigma;\ctx_v[\e])$.
By StrongExceptionSafety we know that $\sigma_0$ is garbage with respect to $\ctx_v[\e]$.

There could be many $l$ inside $\sigma,\sigma_0$ that are $\mathit{monitored}$
in the former step thanks to monitor expressions inside $\mathit{error}$.
However, all such $l$ are defined inside $\sigma_0$,
for the last well formedness condition.
\end{enumerate}
\end{proof}

%\printbibliography
%\appendix
\section{Proof} 
\label{s:proof}

\begin{theorem}[Sound Validation]
	if $c:\Kw{Cap};\emptyset\vdash \e: \T$ and
	$c\mapsto\Kw{Cap}\{\_\}|\e\rightarrow^+ \sigma|\ctx_v[r_l]$, then
	either $valid(\sigma,l)$ or $\mathit{trusted}(\ctx_v,r_l)$.
\end{theorem}

We believe this property captures very precisely our statement in Section~\ref{s:validation}.

It is hard to prove Sound Validation directly,
so we first define a stronger property,
called \emph{Stronger Sound Validation} and
show that it is preserved during reduction by means of conventional 
Progress and Subject Reduction (Progress is one of our assumption,
while Subject Reduction relies heavily on SubjectReductionBase).
That is,
Progress+Subject Reduction $\Rightarrow$ Stronger Sound Validation,
\\*and Stronger Sound Validation $\Rightarrow$ Sound Validation.

\subsection{Stronger Sound Validation $\Rightarrow$ Sound Validation}

Stronger Sound Validation depends on 
$\mathit{wellEncapsulated}$, $\mathit{monitored}$
and $OK$:

\noindent\textbf{Define} $\mathit{wellEncapsulated}(\sigma,\e,l_0)$:\\*
\indent$\forall l \in \mathit{erog}(\sigma,l_0), \text{not}\ \mathit{mutatable}(l,\sigma,\e)$

\noindent The main idea is that an object is well encapsulated if its encapsulated state is safe from
modification. 

\noindent\textbf{Define} $\mathit{monitored}(\e,l)$:\\*
\indent$\e=\ctx_v[M(l,\e_1;\e_2)]$ and either $\e_1=l$ or $l$ is not inside $\e_1$.

\noindent An object is monitored if the execution
is currently inside of a monitor for that object, and
the monitored expression $\e_1$ does not
contains $l$ as a \emph{proper} subexpression.

A monitored object is associated with an expression that can not observe it, but may 
reference its internal representation directly.
In this way, we can safely modify its representation before checking for the invariant.

The idea is that at the start the object will be valid and $\e_1$ will contain $l$;
but during reduction, the $l$ reference will be used in order to
give access to the internal state of $l$; only after that moment, the object may become invalid.


\noindent\textbf{Define} $OK(\sigma,e)$:\\
\indent $\forall l\in\dom(\sigma)$
  either\\
\indent\indent 1. $\mathit{garbage}(l,\sigma,\e)$\\
\indent\indent 2. $\mathit{valid}(\sigma,l)$ and $\mathit{wellEncapsulated}(\sigma,\e,l)$\\
\indent\indent 3. $\mathit{monitored}(\e,l)$

Finally, the system is in a valid state with respect to validation
if for all the objects in the memory, one of these 3 cases apply:
%the class of the object has no invariant method;
the object is not (transitively) reachable from the expression (thus can be garbage collected);
the object is valid, and the object is encapsulated;
or the object is currently monitored.

\begin{theorem}[Stronger Sound Validation]
if $c:\Kw{Cap};\emptyset\vdash \e_0: \T_0$ and
$c\mapsto\Kw{Cap}\{\_\}|\e_0\rightarrow^+ \sigma|\e$, then
$OK(\sigma,\e)$
\end{theorem}
\noindent Starting from only the capability object,
any well typed expression $\e_0$ can be reduced for an arbitrary amount of steps,
and $IOK$ will always hold.
\\
\begin{theorem} Stronger Sound Validation $\Rightarrow$ Sound Validation
\end{theorem}
\begin{proof}
\noindent By Stronger Sound Validation, each $l$ in the current redex must be $OK$:
\begin{enumerate}
	\item If $l$ is garbage, it cannot be in the current redex, a contradiction.
	\item If $\mathit{valid}(\sigma,l)$, then $l$ is valid, so thanks to Determinism
	no invalid object could be observed.
	\item Otherwise, if $\mathit{monitored}(\e,l)$ then either:
	\begin{itemize}
	 \item we are executing inside of $\e_1$ thus the current redex is inside of a sub-expression of the monitor that does not contain $l$, a contradiction.
	 \item or we are executing inside $\e_2$:
	 by our reduction rules, all monitor expressions start with 
	 $\e_2=l$\Q@.validate()@, thus the first execution step
	 of $\e_2$ is trusted. Following execution steps are also trusted, since by well formedness the body of invariant methods only use \Q@this@ (now translated to $l$) to access fields.
	\end{itemize}
\end{enumerate}
In any of the possible cases above, Sound Validation holds for $l$, and so it holds for all redexes.
\end{proof}

\subsection{Subject Reduction}

\noindent\textbf{Define} $\text{fieldGuarded}(\sigma,\e)$:\\*
\indent$\forall \ctx$ such that $\e=\ctx[l\singleDot\f] $
and $\Sigma^\sigma(l).f=\Kw{capsule}\,\_$, and $\f\mathrel{\mathit{inside}} \Sigma^\sigma(l).\mathit{validate}$\\*
\indent\indent either 
$\forall T, \forall C, \Sigma^\sigma;\x:\Kw{mut}\,C\,\not\vdash\ctx[\x]:T$, or\\*
\indent\indent $\ctx=\ctx'[$\Q@M(@$l$\Q@;@$\ctx''$\Q@;@$\e$\Q@)@$]$ and $l$ is contained exactly once in $\ctx''$

That is, all \emph{mutating} capsule field accesses are individually guarded by monitors.
Note how we use $C$ in $\x:\Kw{mut}\,C$ to guess the type of the accessed field,
and that we use the full context $\ctx$ instead of the evaluation context $\ctx_v$
to refer to field accesses everywhere in the expression $\e$.


\begin{theorem}[Subject Reduction]
if $\Sigma^{\sigma_0};\emptyset\vdash e_0: T_0$,
$\sigma_0|e_0\rightarrow \sigma_1|e_1$,
$OK(\sigma_0,\e_0)$
and
$\mathit{fieldGuarded}(\sigma_0,\e_0)$
then
$\Sigma^{\sigma_1};\emptyset\vdash e_1: T_1$,
$OK(\sigma_1,e_1)$ and
$\mathit{fieldGuarded}(\sigma_1,\e_1)$
\end{theorem}

\begin{theorem}
	Progress + Subject Reduction $\Rightarrow$ Stronger Sound Validation
\end{theorem}
\begin{proof}
This proof proceeds by induction in the usual manner.

\emph{Base Case}: At the start of the execution, the memory is going to only contain $c$: since $c$ is defined to be initially $\mathit{valid}$, and has only \Q@mut@ fields, and so it is trivially $\mathit{wellEncapsulated}$, thus $OK(c\mapsto\Kw{Cap},e)$.

\emph{Induction}: By Progress we always have another evaluation step to take, by Subject Reduction such a step will preserve $\mathit{OK}$, and so by induction $\mathit{OK}$ holds after any number of steps.

Note how for the proof garbage collection is important: 
when the \Q@validate()@ method evaluates to \Q@false@, 
execution can continue only if the offending object is classified as garbage.
\end{proof}

\subsection{Proof of Subject Reduction}
We first introduce a lemma derived from well formedness and the type system:
\begin{Lemma}[ExposerInstrumentation]
If $\sigma_0 | \e_0\rightarrow \sigma_1 |\e_1$ and
$\text{fieldGuarded}(\sigma_0,\e_0)$
\\*
then $\text{fieldGuarded}(\sigma_1,\e_1)$
\end{Lemma}
\begin{proof}
The only rule that can 
introduce a new field access is \textsc{mcall}.
In that case, ExposerInstrumentation holds
by well formedness (all field accesses in methods are of the form \Q@this.f@) 
and since \textsc{m call} inserts a monitor while invoking capsule mutator methods, and not field accesses themselves. If however the method is not a \Q@mut@ method but still accesses a capsule field, by MutField such a field access expression cannot be typed as \Q@mut@ and so no monitor is needed.

Note that \textsc{monitor exit} is fine because monitors are removed only when
 $e_1$ is a value.
\end{proof}

\begin{theorem}
	Subject Reduction Base $\Rightarrow$ Subject Reduction
\end{theorem}
\begin{proof}
Any reduction step can be obtained
by exactly one application of rule \textsc{ctx} and then one other rule.



Thus the proof can simply proceed by cases on such other applied rule.

By SubjectReductionBase and ExposerInstrumentation, 
$\Sigma^{\sigma_1};\emptyset\vdash e_1: T_1$ and  $\mathit{fieldGuarded}(\sigma_1,\e_1)$. So we just need to proceed by cases on the reduction rule applied to verify that $OK(\sigma_1,\e_1)$:


\begin{enumerate}
\item \textsc{update:} $\sigma|l\singleDot f\equals v\rightarrow \sigma'|\e'$:
\begin{itemize}
  \item by \textsc{update} $\e'=\Kw{M}\oR l;l;l\singleDot\text{validate}\oR\cR\cR;$, thus $\mathit{monitored}(\e,l)$.
  \item Every $l_1$ such that $l\in \text{rog}(\sigma,l_1)$ will verify the same case
  as the former step:
  \begin{itemize}
  	\item If it was $\mathit{garbage}$, clearly it still is.
  	\item If it was $\mathit{monitored}$, it also still is.
  	\item If can't have been $\mathit{wellEncapsulated}$ since $mutatable(l, \sigma, e)$, (by MutField)
  \end{itemize}
  \item Every other $l_0$ is not reached by $l$ thus it being $\mathit{OK}$ could not have been effected by this reduction step.
\end{itemize}

\noindent\textbf{case field access} $l.f\rightarrow v$:

    If for $l$ $IOK$ holds by (2),  
    it is possible that the next step is not encapsulated.
    This would mean that the field $f$ is a capsule and that we are required
to type it as \Q@mut@ to type the expression for the next step.
By $\mathit{fieldGuarded}(\sigma_0,\e_0)$
    the former step was inside of a monitor \Q@M(@$l$\Q@;@$\ctx_v[l$\Q@.f@$]$\Q@;@$\e$\Q@)@
    and the $l$ under reduction was the only occurrence of $l$.
    since $f$ is a capsule, we know that $l\notin \text{erog}(\sigma,l)$
    by HeadNonCircular.
    Thus in the new step not $l\, \text{inside}\ \ctx_v[v]$.
    Thus for l (3)[monitored] holds.
    
We still need to show that properties $\mathit{monitored}$ and $\mathit{wellEncapsulated}$
 for other objects are
not disturbed. This is the point where our aliasing and mutability control are most crucial:
We know that mutable $v$ is (directly) reachable from
$l$ that have invariant.
Thanks to CapsuleTree we know that for all $l_0$ reaching $l$,
$v$ can be reached by $l_0$ only passing trough $l$.
Thus, we can conclude  $l_0$ is not encapsulated in the former step (containing mutable $l$).
Thus, $l_0$ is either without invariant, garbage or monitored.
None of those 3 cases can be disturbed by a field access.


\noindent\textbf{case meth call}:\\*
  This reduction step does not influence any object in the memory and does not
disturb the properties $\mathit{monitored}$ and $\mathit{wellEncapsulated}$.

\noindent\textbf{case new}:\\*
  If $C$ has invariant, then by @ConstructionInstrumentation the new object is monitored.
As for the method call, other objects and properties are not disturbed.


\noindent\textbf{case monitor exit} \Q@M(@$l;v;$\Q@true)@$\rightarrow v$ :
  \begin{itemize}
\item
    If it was a setter $v=l$, and 
    thanks to Determinism the execution of invariant is deterministic;
    thus for $l$ in the former step both case (2) and (3) holds.
    In the next step (2) will hold for $l$.
\item
    If it was a capsule mutator method, thanks to Determinism the execution
 of \Q@.validate()@ is deterministic;
    thus for $l$ in the former step both $H$ and case (3) holds.
    Thanks to ExposerInstrumentation $v$ is offered without mutation permissions, so
    In the next step $l$ is encapsulated and (2) will hold.
\item
    If it is was a constructor, 
    then $v$ is encapsulated and thanks to Determinism
    the execution of invariant is deterministic, thus in the next step (2) will hold.
\end{itemize}

\noindent\textbf{case try enter and try ok}
This case do not influence any object in the memory and does not
disturb the properties $\mathit{monitored}$ and $\mathit{wellEncapsulated}$.

\noindent\textbf{case try catch} $\sigma,\sigma_0|\Kw{try}^\sigma \oC\mathit{error}\cC\Kw{catch}\, \e\rightarrow \sigma|\e$:\\*
From the premise we know 
$IOK(\sigma,\sigma_0;\ctx_v[\Kw{try}^\sigma \oC\mathit{error}\cC\Kw{catch}\, \e])$;
thus we need to show
$IOK(\sigma;\ctx_v[\e])$.
By StrongExceptionSafety we know that $\sigma_0$ is garbage with respect to $\ctx_v[\e]$.

There could be many $l$ inside $\sigma,\sigma_0$ that are $\mathit{monitored}$
in the former step thanks to monitor expressions inside $\mathit{error}$.
However, all such $l$ are defined inside $\sigma_0$,
for the last well formedness condition.
\end{enumerate}
\end{proof}
\end{document}