\saveSpace
\section{Introduction}\label{sec:intro}
\saveSpace
In Java, C++, Scala and C\#, subclassing 
implies subtyping. A Java subclass declaration, such as 
\Q@class A extends B {}@
\noindent does two things at the same time:
it {\bf inherits} code from \lstinline{B}; and it creates
a subtype of \lstinline{B}. Therefore a subclass must \emph{always} be
a subtype of the extended class.
Such design choice where subclassing implies subtyping is not
universally accepted.
Historically, there has been a lot of focus on
separating subtyping from subclassing~\cite{cook}.  This separation is claimed to be
good for code-reuse, design and reasoning. There are at
least two distinct situations where the separation of subtyping and 
subclassing is helpful.

\begin{itemize}

\item {\bf Allowing inheritance/reuse even when subtyping is impossible:} 
Situations where inheritance is desirable are prevented
by the enforced subtyping relation. A well-known example are the so-called \emph{binary methods}~\cite{cook,bruce96binary}.
For example, consider a class \Q@Point@ with a method 
\Q@Point sum(Point o){return new Point(x+o.x,y+o.y);}@.
Can we reuse the \Q@Point@ code so that \Q@ColorPoint.sum@
would take and return a \Q@ColorPoint@?
In Java/C\# declaring \Q@class ColorPoint extends Point{..}@ would result
in \Q@sum@ still taking a \Q@Point@ and returning a \Q@Point@.
Moreover, manually redeclaring a \Q@ColorPoint sum(ColorPoint that)@
would just induce overloading, not overriding.
In this case we would like to have inheritance, but we cannot
have (safe) subtyping.
%
%In some situations a subclass contains methods whose signatures 
%are incompatible with the superclass, yet inheritance is still
%desirable. Classes with \emph{binary methods}~\cite{cook,bruce96binary} are a typical example.

\item {\bf Preventing unintended subtyping:} For certain classes we
  would like to inherit code without creating a subtype even if, from
  the typing point of view, subtyping is still sound. A typical
  example~\cite{LaLonde:1991:SSS:110673.110679} is \emph{Sets} and
  \emph{Bags}. Bag implementations can often inherit 
  from Set implementations, and the interfaces of the two collection types are
  similar and type compatible. 
  However, from the logical point-of-view a Bag is \emph{not a
    subtype} of a Set. 

\end{itemize}

\noindent Structural typing~\cite{cook} may deal with the first
situation, but not the second. Since structural subtyping
accounts for the types of the methods only, a Bag would be a subtype
of a Set if the two interfaces are type compatible. For dealing with
the second situation, nominal subtyping is preferable: an explicit subtyping relation must be signalled by the programmer. Thus if subtyping is not desired, the
programmer can simply {\bf not} declare a subtyping relationship.

While there is no problem in subtyping without subclassing, in most nominal OO languages subclassing fundamentally implies subtyping. 
This is because of what we call the
\emph{this-leaking problem}, illustrated by the following
(Java) code, where
method \lstinline{A.ma} passes \lstinline{this} as \lstinline{A} to \Q@Utils.m@.
This code is correct, and there is no subtyping/subclassing{.}\saveSpace\saveSpace
\begin{lstlisting}[language=Java]
  class A{ int ma(){ return Utils.m(this); } }
  class Utils{ static int m(A a){..} }
\end{lstlisting}
\saveSpace\saveSpace
  Now, lets add a class \Q@B@
\saveSpace\saveSpace\saveSpace
\begin{lstlisting}[language=Java]
  class B extends A{ int mb(){return this.ma();} }  
\end{lstlisting}
\saveSpace\saveSpace
%%Class \lstinline{B} does two things at the same time: 
%%1) it {\bf inherits} the method \lstinline{ma} from
%%\lstinline{A}; and 2) it creates a {\bf subtype} of \lstinline{A}.
\noindent We can see an invocation of \lstinline{A.ma} inside
\lstinline{B.mb}, where the self-reference \lstinline{this} is of type \lstinline{B}. 
The execution will eventually call \lstinline{Utils.m} with an
instance of \lstinline{B}. However, \emph{this can be correct only if \lstinline{B} is a subtype of
\lstinline{A}}. 

%If Java was to support a mechanism to allow reuse/inheritance 
%without introducing subtyping, such as:
%
%\begin{lstlisting}[language=Java]
%  class B inherits A{ int mb(){return this.ma();} }
%\end{lstlisting}
%
%\noindent Then an invocation of 
%\lstinline{mb} would be type-unsafe (i.e. it would 
%result in a run-time type error). 
%Note that here the intention of using the imaginary keyword {\bf
%  inherits} is to allow the code from \lstinline{A} to be inherited 
%without \lstinline{B} becoming a subtype of \lstinline{A}. 
%However this breaks type-safety. The problem is that the
%self-reference \lstinline{this} in class \lstinline{B} has 
%type \lstinline{B}. Thus, when \lstinline{this} is passed as an argument to 
%the method \lstinline{Utils.m} as a result of the invocation of
%\lstinline{mb}, it will have a type that is incompatible with the
%expected argument of type \lstinline{A}.  


Suppose Java code-reuse (the {\bf extends} keyword) did not introduce subtyping\footnote{
C++ allows "extending privately"; this is not
what we mean by not introducing subtyping: in C++ it is a limitation over
  subtyping visibility, not over subtyping itself.  Indeed, the
  former example would be \emph{accepted} even if \lstinline{B} \MS{were to}
  "privately extends" \lstinline{A}}: then an invocation of 
\lstinline{B.mb} would result in a run-time type error.
The problem is that the
self-reference \lstinline{this} in class \lstinline{B} has 
type \lstinline{B}. Thus, when \lstinline{this} is passed as an argument to 
the method \lstinline{Utils.m} (as a result of the invocation of
\lstinline{B.mb}), it will have a type that is incompatible with the
expected argument of type \lstinline{A}.  
Therefore, every OO language with the minimal features exposed in the example (using \lstinline{this},
{\bf extends} and method calls) is forced to accept that subclassing implies
subtyping.
  

What the \emph{this-leaking problem} shows is that adopting a more flexible
nominally typed OO model where subclassing does not imply subtyping is
not trivial: a more substantial change in the language design is
necessary.  In essence we believe that in languages like Java, classes do too many
things at once. In particular they act both as units of \emph{use} and
\emph{reuse}: classes can be \emph{use}d as types and can be instantiated;
classes can also be subclassed to provide \emph{reuse} of code.
We are aware of at least $3$ independently designed research
languages that address the \emph{this-leaking problem}:
\begin{itemize}
\item In {\bf TraitRecordJ (TR)}~\cite{Bettini:2010:ISP:1774088.1774530,BETTINI2013521,Bettini2015282}
each construct has a single responsibility: classes instantiate objects,
interfaces induce types, records express state, and traits are reuse units.
\item {\bf Package Templates (PT)}~\cite{KrogdahlMS09,DBLP:journals/taosd/AxelsenSKM12,DBLP:conf/gpce/AxelsenK12}:
an extension of (full) Java where new packages can be ``synthesized'' by mixing
and integrating code templates. 
Such ``synthesized'' packages can be used for code reuse without inducing subtyping.
%As an extension of Java, PT allows but does not require
%separation of inheritance and subtyping.
\item {\bf
    DeepFJig(DJ)}~\cite{deep,servetto2014meta,fjig} is
a module composition language where nested classes with the same name are recursively composed.
\end{itemize}
%Those $3$ languages also shown that family polymorphism 
%is easy to support when inheritance and subtyping are separated.

While there are various good ideas in those designs, those designs are
still quite involved.
This paper shows a simple language design, called \name,
addressing the \emph{this-leaking problem} and decoupling subtyping from inheritance.
We build on traits to distinguish code designed for
\emph{use} from code designed for \emph{reuse}.
We synthesize and simplify the best ideas from those $3$ very
different designs, and couple them with an elegant novel approach to
state and self construction in traits that avoids the complexities and redundancies introduced by fields and their initialisation.
%-------

In \name, there are two separate concepts: classes
and traits~\cite{ducasse2006traits}. Classes are meant for code \emph{use}, and cannot be inherited/extended. Classes in \name are like final classes in
Java, and can be used as types and as object factories. Traits are meant for code \emph{reuse} only: multiple traits can be
composed to form a class. However, traits 
cannot be instantiated or used as types.
This allows fine-grained control of subtyping while handling examples like \Q@Set/Bag@.

In \name, as in many module composition languages~\cite{ancona2002calculus},
all methods can be abstract, including static ones.
Moreover, module composition can be used to make
an already implemented method abstract.
Thus, as for dynamic dispatch, the behaviour of a method call is never set in stone.
%The behaviour is similar to Python class-methods.
We will show how in \name, state is induced by an implicit fixpoint operation over abstract methods,
where an abstract static method can perform the role of a constructor.
This allows handling examples like \Q@Point/ColorPoint@ in a natural way, without requiring code duplication.

Our design brings several benefits. In particular, 
Family Polymorphism~\cite{ernst2004expression} is
radically simpler to support soundly.
This is already clear in the $3$ lines of
research above, and is even more outstanding in the clean \name model.

We first focus on an example-driven presentation to illustrate how to
improve use and reuse. 
In Appendix~\ref{sec:formal}, we then provide a compact formalization.
Most of the hard technical aspects of the
semantics have been studied in previous 
work~\cite{Bettini:2010:ISP:1774088.1774530,BETTINI2013521,Bettini2015282,KrogdahlMS09,DBLP:journals/taosd/AxelsenSKM12,DBLP:conf/gpce/AxelsenK12,deep,servetto2014meta,fjig},
and the design of \name synthesizes some of those concepts.
The language design ideas have been implemented in the full 42 language, which supports all
the examples we show in the paper, and is available at: \url{http://l42.is}.
Work on 42 is now slowly reaching maturity after about $5$ years of
intense research and development. The current implementation 
is now robust enough to create realistic medium sized programs running 
on the JVM, and the standard library consists of over $10000$ lines of
42 code. 
%more then a thousand GIT commits and about a thousand tests.

\noindent In summary, our contributions are:

%\marco{We need to talk of unanticipated extensions?}
%\bruno{Talk more about typing aspects here. Summarise the important 
%aspects of the design of \name.}

%\bruno{Need to say something about state?}

%The language design of \name is adopted by the 42 programming 
%language, All the examples shown in this paper can be run in 42.
%The compiler for 42 is available at: 

%\url{http://l42.is}



\begin{itemize}
\item We identify the this-leaking problem, that makes separating inheritance and subtyping difficult.

\item We synthesize the key ideas of previous designs that solve the
  this-leaking problem into a novel and
  minimalistic language design. This language is the core logic of the language 42, and 
  all the examples in this paper can be encoded as valid 42 programs.
This design improves both code use and code reuse.

\item We propose a clean and elegant approach to the handling of state in a trait based language.

\item We illustrate how \name, extended with nested classes, enables a
  powerful (but at the same time simple) form of family polymorphism. % the expression problem better than former work.
\item We show the simplicity of our approach by providing a compact $1$ page formalization (in Appendix A).
\item We perform $3$ case studies, comparing our work with other
  approaches, and we collect clear data showing that we can express the same examples in a cleaner and more modular manner.
\end{itemize}
\saveSpace
\saveSpace

