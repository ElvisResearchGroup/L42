\newpage\appendix

\section{Benefits without heavy costs}
In this section we discuss why the design of \name
is not intrinsically harder to use than Java.

\subsection{Introducing more names}
At first it may seams our approach requires to
introduce and keep in mind more names, as for \Q@set@ and \Q@Set@;
however this is not the case. 
The \textbf{user} of \Q@Set@ needs to keep in mind
only the class \Q@Set@.
The \textbf{reuser} of \Q@set@ needs to keep in mind only
 the trait \Q@set@.
Those are quite different roles, so they may very well
being handled by different programmers.
As shown in the \Q@set/Set@ example, defining a reusable class
takes only an extra line with respect to defining a final one.
In some sense, in Java the default is ``non final'', and
the programmer can write \Q@final@ to prevent code reuse, while
in 42 the default is reverted.

It could be possible to define a syntactic sugar, like
\begin{lstlisting}
reusable Set={...}
//expanded into
set={...}
Set=set
\end{lstlisting}
While this may sound appealing to a Java programmers, it only saves one line, and it may obscure the actual behaviour of the code.
On the same line, Cecil~\cite{chambers1995typechecking} offers 
syntactic sugar to declare a class and an interface with the same structural type in a single declaration.

\subsection{Class hierarchies}
By having all the classes final, we prevent class hierarchies.
However,
we allow both trait hierarchies and interface hierarchies.
We believe having those two trees,
one for code reuse and one for subtyping can improve code
maintainability, since those two trees will be able to evolve independently. Having two trees does not makes code comprehension harder since, as for before, they have different roles and thus are relevant only one at the time.

Often, class hierarchies are used as a hack in order to obtain code reuse (and in those cases, the LSP is often violated).
In \name, trait hierarchies will just serve the same role better.


\subsection{Constructors and initialization}
Java allows constructors to do arbitrary operations.
The same behaviour can be obtained in \name by using a factory method; that is, the following:
\begin{lstlisting}[language=Java]
class A{
  public int x;
  public A(String s){
    this.x=/*any computation here*/;
    }
  }
\end{lstlisting}
could be expressed in \name as 
\begin{lstlisting}
A={
  method Num x();
  static method This of(Num x)
  static method This of(String s){
    return This.of(/*any computation here*/);
    }
  }
\end{lstlisting}

For space reasons we did not explain yet how to allow for traits to have their own state with a default initialization.
This could be solved by adding to the abstract state operations (getters, setters, withers) a default initializer;
for example the trait
\begin{lstlisting}
t={ method Num x();
  method defaultX(){return 0;}}//new concept here!
\end{lstlisting}
would loosely correspond to a Java abstract class
\begin{lstlisting}[language=Java]
class T{ int x=0;  }
\end{lstlisting}
and could be used to obtain concrete classes where the abstract factory do not need to initialize the field \Q@x@.
For example classes \Q@A@ and \Q@B@ below would be coherent classes
\begin{lstlisting}
A= Use t,{ static method This of()}
B= Use t,{ method String s(); static method This of(String s)}
...
A.of().x()==0
\end{lstlisting}

Finally, when self construction is not needed, declaring the abstract factory can be redundant, and it can indeed be omitted in the full 42 language.

Moreover, when code reuse is involved, \name is much simpler to use then Java,C++ or C\#: a class extending another class needs to define a constructor calling a super constructor.
   This typically has as many parameters as the full set of fields. Those parameters are then used at least one time in the body of the constructor.
In \name, to compose traits without adding fields, there is no need to repeat the abstract factory.
To add a field, you need to define a new abstract factory and implement the "super" factory method by calling the new one.
Thus, when adding a field the amount of code between 42 and Java is about the same, but is much easier when no fields are added.

How class invariants and state encapsulation is established and preserved without explicit super constructors is a different issue, and a paper on such topic is being prepared right now.

\subsection{Self instantiation is very useful}
In many cases the absence of self instantiation cause bugs: often programmers ``think'' that \Q@new C()@ inside of class \Q@C@ would create an instance of the ``current class'', while of course \Q@class D extends C@ still creates instances of \Q@C@.
The absence of self instantiation also prevents to use functional programming patterns in OO languages.
Just a look to how involved is the code to implement the functional \Q@Point@ in Java is sufficient to dissuade most programmers.

\subsection{Comprehensibility and usability}
Of course it could take a while to adapt from the Java model to the \name one. However, when it comes to model complex reuse scenario, we believe \name is much more explicit and clear then Scala and other options. We invite the curios reader to look to the complete code of the case studies (below) and compare our solution with the cryptic version of Scala, where behaviour traits are nested inside other traits.

In particular, with respect to Java/Scala we encourage more explicit code since:
\begin{itemize}
\item sub-typing need to be explicitly declared.
\item when sub typing is needed, an opportune interface need to be declared. Thus when using a class type, it is always clear what implementation is referred.
\item \name requires to be explicit when subtyping and subclassing are both supposed to happen (by declaring an extra interface)
\item \Q@this.@ is required to access fields/methods (so they can not be confused with local variables)
\item our object model is equivalent to an untyped language
\end{itemize}
The last point is quite important: languages where inheritance is not subtyping proposed in the past tend to have more complex object models, instead our 
our flattened language is just a language of interfaces and final classes. Novice programmers may start by learning that. They could learn to use traits later, and to define their own traits even later.

\section{Case studies complete code} 

Here, for reference, you can find the complete
of the Case Studies.
It is quite a lot of code, about 18 pages. The interested reader
can navigate it in order to gain a perfect understanding on
how we encoded the various solutions in the various styles and languages,
but is not needed to understand the overall value of our paper.

While evaluating our approach, we consider not only lines of code but also number of methods and classes.
Our particular examples are focused on code reuse, thus method implementations are all trivial.
We believe that in this particular context the number of methods/classes is a good indicator.


\subsection{Point algebra}
Here the code of Java7 for the first case study, the \Q@Point@ algebra:
\begin{lstlisting}[basicstyle=\tiny]
//0
class Point {//6
  final int x;
  final int y;

  public Point(int x, int y) {
    this.x = x;
    this.y = y;
  }
}

// 1
class PointSum extends Point {//5*4
  public PointSum(int x, int y) {
    super(x, y);
  }

  public PointSum sum(Point that) {
    return new PointSum(this.x + that.x, this.y + that.y);
  }
}

// 2
class PointSub extends Point {
  public PointSub(int x, int y) {
    super(x, y);
  }

  public PointSub sub(Point that) {
    return new PointSub(this.x - that.x, this.y - that.y);
  }
}

// 3
class PointMul extends Point {
  public PointMul(int x, int y) {
    super(x, y);
  }

  public PointMul mul(Point that) {
    return new PointMul(this.x * that.x, this.y * that.y);
  }
}

// 4
class PointDiv extends Point {
  public PointDiv(int x, int y) {
    super(x, y);
  }

  public PointDiv div(Point that) {
    return new PointDiv(this.x / that.x, this.y / that.y);
  }
}

// 5
class PointSumSub extends PointSum {//7*6
  public PointSumSub(int x, int y) {
    super(x, y);
  }

  public PointSumSub sum(Point that) {
    return new PointSumSub(this.x + that.x, this.y + that.y);
  }

  public PointSumSub sub(Point that) {
    return new PointSumSub(this.x - that.x, this.y - that.y);
  }
}

// 6
class PointSumMul extends PointSum {
  public PointSumMul(int x, int y) {
    super(x, y);
  }

  public PointSumMul sum(Point that) {
    return new PointSumMul(this.x + that.x, this.y + that.y);
  }

  public PointSumMul mul(Point that) {
    return new PointSumMul(this.x * that.x, this.y * that.y);
  }
}

// 7
class PointSumDiv extends PointSum {
  public PointSumDiv(int x, int y) {
    super(x, y);
  }

  public PointSumDiv sum(Point that) {
    return new PointSumDiv(this.x + that.x, this.y + that.y);
  }

  public PointSumDiv div(Point that) {
    return new PointSumDiv(this.x / that.x, this.y / that.y);
  }
}

// 8
class PointSubMul extends PointSub {
  public PointSubMul(int x, int y) {
    super(x, y);
  }

  public PointSubMul sub(Point that) {
    return new PointSubMul(this.x - that.x, this.y - that.y);
  }

  public PointSubMul mul(Point that) {
    return new PointSubMul(this.x * that.x, this.y * that.y);
  }
}

// 9
class PointSubDiv extends PointSub {
  public PointSubDiv(int x, int y) {
    super(x, y);
  }

  public PointSubDiv sub(Point that) {
    return new PointSubDiv(this.x - that.x, this.y - that.y);
  }

  public PointSubDiv div(Point that) {
    return new PointSubDiv(this.x / that.x, this.y / that.y);
  }
}

// 10
class PointMulDiv extends PointMul {
  public PointMulDiv(int x, int y) {
    super(x, y);
  }

  public PointMulDiv mul(Point that) {
    return new PointMulDiv(this.x * that.x, this.y * that.y);
  }

  public PointMulDiv div(Point that) {
    return new PointMulDiv(this.x / that.x, this.y / that.y);
  }
}

// 11
class PointSumSubDiv extends PointSumSub {//9*4
  public PointSumSubDiv(int x, int y) {
    super(x, y);
  }

  public PointSumSubDiv sum(Point that) {
    return new PointSumSubDiv(this.x + that.x, this.y + that.y);
  }

  public PointSumSubDiv sub(Point that) {
    return new PointSumSubDiv(this.x - that.x, this.y - that.y);
  }

  public PointSumSubDiv div(Point that) {
    return new PointSumSubDiv(this.x / that.x, this.y / that.y);
  }
}

// 12
class PointSumSubMul extends PointSumSub {
  public PointSumSubMul(int x, int y) {
    super(x, y);
  }

  public PointSumSubMul sum(Point that) {
    return new PointSumSubMul(this.x + that.x, this.y + that.y);
  }

  public PointSumSubMul sub(Point that) {
    return new PointSumSubMul(this.x - that.x, this.y - that.y);
  }

  public PointSumSubMul mul(Point that) {
    return new PointSumSubMul(this.x * that.x, this.y * that.y);
  }
}

// 13
class PointSumMulDiv extends PointMulDiv {
  public PointSumMulDiv(int x, int y) {
    super(x, y);
  }

  public PointSumMulDiv sum(Point that) {
    return new PointSumMulDiv(this.x + that.x, this.y + that.y);
  }

  public PointSumMulDiv mul(Point that) {
    return new PointSumMulDiv(this.x * that.x, this.y * that.y);
  }

  public PointSumMulDiv div(Point that) {
    return new PointSumMulDiv(this.x / that.x, this.y / that.y);
  }
}

// 14
class PointSubMulDiv extends PointMulDiv {
  public PointSubMulDiv(int x, int y) {
    super(x, y);
  }

  public PointSubMulDiv sub(Point that) {
    return new PointSubMulDiv(this.x - that.x, this.y - that.y);
  }

  public PointSubMulDiv mul(Point that) {
    return new PointSubMulDiv(this.x * that.x, this.y * that.y);
  }

  public PointSubMulDiv div(Point that) {
    return new PointSubMulDiv(this.x / that.x, this.y / that.y);
  }
}

// 15
class PointSumSubMulDiv extends PointSumSubMul {//11
  public PointSumSubMulDiv(int x, int y) {
    super(x, y);
  }

  public PointSumSubMulDiv sum(Point that) {
    return new PointSumSubMulDiv(this.x + that.x, this.y + that.y);
  }

  public PointSumSubMulDiv sub(Point that) {
    return new PointSumSubMulDiv(this.x - that.x, this.y - that.y);
  }

  public PointSumSubMulDiv mul(Point that) {
    return new PointSumSubMulDiv(this.x * that.x, this.y * that.y);
  }

  public PointSumSubMulDiv div(Point that) {
    return new PointSumSubMulDiv(this.x / that.x, this.y / that.y);
  }
}
\end{lstlisting}

Here you can find the ClassLessJava style code for the Point Algebra:
\begin{lstlisting}[basicstyle=\tiny]
// 0
@Obj // 3 lines
interface Point {
  static Point of(int x, int y) {// methods "of" generated by @Obj thus not counted in the line numbers
    return new Point() {
      public int x() {
        return x;
      }

      public int y() {
        return y;
      }
    };
  }

  int x();

  int y();
}

// 1
@Obj // 3*4 lines
interface PointSum extends Point {
  static PointSum of(int x, int y) {
    return new PointSum() {
      public int x() {
        return x;
      }

      public int y() {
        return y;
      }
    };
  }

  default PointSum sum(Point that) {
    return of(this.x() + that.x(), this.y() + that.y());
  }
}

// 2
@Obj
interface PointSub extends Point {
  static PointSub of(int x, int y) {
    return new PointSub() {
      public int x() {
        return x;
      }

      public int y() {
        return y;
      }
    };
  }

  default PointSub sub(Point that) {
    return of(this.x() - that.x(), this.y() - that.y());
  }
}

// 3
@Obj
interface PointMul extends Point {
  static PointMul of(int x, int y) {
    return new PointMul() {
      public int x() {
        return x;
      }

      public int y() {
        return y;
      }
    };
  }

  default PointMul mul(Point that) {
    return of(this.x() * that.x(), this.y() * that.y());
  }
}

// 4
@Obj
interface PointDiv extends Point {
  static PointDiv of(int x, int y) {
    return new PointDiv() {
      public int x() {
        return x;
      }

      public int y() {
        return y;
      }
    };
  }

  default PointDiv div(Point that) {
    return of(this.x() / that.x(), this.y() / that.y());
  }
}

// 5
@Obj // 5*6 lines
interface PointSumMul extends PointSum, PointMul {
  static PointSumMul of(int x, int y) {
    return new PointSumMul() {
      public int x() {
        return x;
      }

      public int y() {
        return y;
      }
    };
  }

  default PointSumMul sum(Point that) {// we have to rewrite the method calling the new "of"
    return of(this.x() + that.x(), this.y() + that.y());// in order to produce an instance of PointSumMul
  }

  default PointSumMul mul(Point that) {
    return of(this.x() * that.x(), this.y() * that.y());
  }
}

// 6
@Obj
interface PointSumSub extends PointSum, PointSub {
  static PointSumSub of(int x, int y) {
    return new PointSumSub() {
      public int x() {
        return x;
      }

      public int y() {
        return y;
      }
    };
  }

  default PointSumSub sum(Point that) {// we have to rewrite the method calling the new "of"
    return of(this.x() + that.x(), this.y() + that.y());// in order to produce an instance of PointSumMul
  }

  default PointSumSub sub(Point that) {
    return of(this.x() - that.x(), this.y() - that.y());
  }
}

// 7
@Obj
interface PointSumDiv extends PointSum, PointDiv {
  static PointSumDiv of(int x, int y) {
    return new PointSumDiv() {
      public int x() {
        return x;
      }

      public int y() {
        return y;
      }
    };
  }

  default PointSumDiv sum(Point that) {// we have to rewrite the method calling the new "of"
    return of(this.x() + that.x(), this.y() + that.y());// in order to produce an instance of PointSumMul
  }

  default PointSumDiv div(Point that) {
    return of(this.x() / that.x(), this.y() / that.y());
  }
}

// 8
@Obj
interface PointSubMul extends PointMul, PointSub {
  static PointSubMul of(int x, int y) {
    return new PointSubMul() {
      public int x() {
        return x;
      }

      public int y() {
        return y;
      }
    };
  }

  default PointSubMul mul(Point that) {// we have to rewrite the method calling the new "of"
    return of(this.x() * that.x(), this.y() * that.y());// in order to produce an instance of PointSumMul
  }

  default PointSubMul sub(Point that) {
    return of(this.x() - that.x(), this.y() - that.y());
  }
}

// 9
@Obj
interface PointSubDiv extends PointSub, PointDiv {
  static PointSubDiv of(int x, int y) {
    return new PointSubDiv() {
      public int x() {
        return x;
      }

      public int y() {
        return y;
      }
    };
  }

  default PointSubDiv sub(Point that) {// we have to rewrite the method calling the new "of"
    return of(this.x() - that.x(), this.y() - that.y());// in order to produce an instance of PointSumMul
  }

  default PointSubDiv div(Point that) {
    return of(this.x() / that.x(), this.y() / that.y());
  }
}

// 10
@Obj
interface PointMulDiv extends PointMul, PointDiv {
  static PointMulDiv of(int x, int y) {
    return new PointMulDiv() {
      public int x() {
        return x;
      }

      public int y() {
        return y;
      }
    };
  }

  default PointMulDiv mul(Point that) {// we have to rewrite the method calling the new "of"
    return of(this.x() * that.x(), this.y() * that.y());// in order to produce an instance of PointSumMul
  }

  default PointMulDiv div(Point that) {
    return of(this.x() / that.x(), this.y() / that.y());
  }
}

// 11
@Obj // 7*4 lines
interface PointSumSubDiv extends PointSumSub, PointSumDiv, PointSubDiv {
  static PointSumSubDiv of(int x, int y) {
    return new PointSumSubDiv() {
      public int x() {
        return x;
      }

      public int y() {
        return y;
      }
    };
  }

  default PointSumSubDiv sum(Point that) {
    return of(this.x() + that.x(), this.y() + that.y());
  }

  default PointSumSubDiv sub(Point that) {
    return of(this.x() - that.x(), this.y() - that.y());
  }

  default PointSumSubDiv div(Point that) {
    return of(this.x() / that.x(), this.y() / that.y());
  }
}

// 12
@Obj
interface PointSumSubMul extends PointSumSub, PointSumMul, PointSubMul {
  static PointSumSubMul of(int x, int y) {
    return new PointSumSubMul() {
      public int x() {
        return x;
      }

      public int y() {
        return y;
      }
    };
  }

  default PointSumSubMul sum(Point that) {
    return of(this.x() + that.x(), this.y() + that.y());
  }

  default PointSumSubMul sub(Point that) {
    return of(this.x() - that.x(), this.y() - that.y());
  }

  default PointSumSubMul mul(Point that) {
    return of(this.x() * that.x(), this.y() * that.y());
  }
}

// 13
@Obj
interface PointSumMulDiv extends PointSumMul, PointMulDiv, PointSumDiv {
  static PointSumMulDiv of(int x, int y) {
    return new PointSumMulDiv() {
      public int x() {
        return x;
      }

      public int y() {
        return y;
      }
    };
  }

  default PointSumMulDiv sum(Point that) {
    return of(this.x() + that.x(), this.y() + that.y());
  }

  default PointSumMulDiv mul(Point that) {// we have to rewrite the method calling the new "of"
    return of(this.x() * that.x(), this.y() * that.y());// in order to produce an instance of PointSumMul
  }

  default PointSumMulDiv div(Point that) {
    return of(this.x() / that.x(), this.y() / that.y());
  }
}

// 14
@Obj
interface PointSubMulDiv extends PointSubMul, PointMulDiv, PointSubDiv {
  static PointSubMulDiv of(int x, int y) {
    return new PointSubMulDiv() {
      public int x() {
        return x;
      }

      public int y() {
        return y;
      }
    };
  }

  default PointSubMulDiv sub(Point that) {
    return of(this.x() - that.x(), this.y() - that.y());
  }

  default PointSubMulDiv mul(Point that) {// we have to rewrite the method calling the new "of"
    return of(this.x() * that.x(), this.y() * that.y());// in order to produce an instance of PointSumMul
  }

  default PointSubMulDiv div(Point that) {
    return of(this.x() / that.x(), this.y() / that.y());
  }
}

// 15
@Obj//9*1 lines
interface PointSumSubMulDiv extends PointSumSubMul, PointSumMulDiv, PointSumSubDiv, PointSubMulDiv {
  static PointSumSubMulDiv of(int x, int y) {
    return new PointSumSubMulDiv() {
      public int x() {
        return x;
      }

      public int y() {
        return y;
      }
    };
  }

  default PointSumSubMulDiv sum(Point that) {
    return of(this.x() + that.x(), this.y() + that.y());
  }

  default PointSumSubMulDiv sub(Point that) {
    return of(this.x() - that.x(), this.y() - that.y());
  }

  default PointSumSubMulDiv mul(Point that) {// we have to rewrite the method calling the new "of"
    return of(this.x() * that.x(), this.y() * that.y());// in order to produce an instance of PointSumMul
  }

  default PointSumSubMulDiv div(Point that) {
    return of(this.x() / that.x(), this.y() / that.y());
  }
}
\end{lstlisting}

Here you can find the Scala code for the Point Algebra:
\begin{lstlisting}[basicstyle=\tiny]
trait tPointState {//5
  type p <: tPointState
  def x: Int
  def y: Int
  def of(x:Int,y:Int):p
}

trait tPointSum extends tPointState {//3*4
  def sum(that:p)=
    this.of(this.x+that.x,this.y+that.y)  
}

trait tPointSub extends tPointState {
  def sub(that:p)=
    this.of(this.x-that.x,this.y-that.y)  
}

trait tPointMul extends tPointState {
  def mul(that:p)=
    this.of(this.x*that.x,this.y*that.y)  
}

trait tPointDiv extends tPointState {
  def div(that:p)=
    this.of(this.x/that.x,this.y/that.y)  
}
//glue code from now on
class Point0(val x:Int, val y:Int) extends tPointState {//4*16
 override type p = Point0
 override def of(x:Int,y:Int)=
   new Point0(x,y)//3 duplication of "this name"
}

class PointSum(val x:Int, val y:Int) extends tPointSum {
 override type p = PointSum
 override def of(x:Int,y:Int)=
   new PointSum(x,y)
}

class PointSub(val x:Int, val y:Int) extends tPointSub {
 override type p = PointSub
 override def of(x:Int,y:Int)=
   new PointSub(x,y)
}

class PointMul(val x:Int, val y:Int) extends tPointMul {
 override type p = PointMul
 override def of(x:Int,y:Int)=
   new PointMul(x,y)
}

class PointDiv(val x:Int, val y:Int) extends tPointDiv {
 override type p = PointDiv
 override def of(x:Int,y:Int)=
   new PointDiv(x,y)
}

class PointSumSub(val x:Int, val y:Int) extends tPointSum with tPointSub {
 override type p = PointSumSub
 override def of(x:Int,y:Int)=
   new PointSumSub(x,y)
}

class PointSumMul(val x:Int, val y:Int) extends tPointSum with tPointMul {
 override type p = PointSumMul
 override def of(x:Int,y:Int)=
   new PointSumMul(x,y)
}

class PointSumDiv(val x:Int, val y:Int) extends tPointSum with tPointDiv {
 override type p = PointSumDiv
 override def of(x:Int,y:Int)=
   new PointSumDiv(x,y)
}

class PointSubMul(val x:Int, val y:Int) extends tPointSub with tPointMul {
 override type p = PointSubMul
 override def of(x:Int,y:Int)=
   new PointSubMul(x,y)
}

class PointSubDiv(val x:Int, val y:Int) extends tPointSub with tPointDiv {
 override type p = PointSubDiv
 override def of(x:Int,y:Int)=
   new PointSubDiv(x,y)
}

class PointMulDiv(val x:Int, val y:Int) extends tPointMul with tPointDiv {
 override type p = PointMulDiv
 override def of(x:Int,y:Int)=
   new PointMulDiv(x,y)
}

class PointSumSubDiv(val x:Int, val y:Int) extends tPointSum with tPointSub with tPointDiv {
 override type p = PointSumSubDiv
 override def of(x:Int,y:Int)=
   new PointSumSubDiv(x,y)
}

class PointSumSubMul(val x:Int, val y:Int) extends tPointSum with tPointSub with tPointMul {
 override type p = PointSumSubMul
 override def of(x:Int,y:Int)=
   new PointSumSubMul(x,y)
}

class PointSumMulDiv(val x:Int, val y:Int) extends tPointSum with tPointMul with tPointDiv {
 override type p = PointSumMulDiv
 override def of(x:Int,y:Int)=
   new PointSumMulDiv(x,y)
}

class PointSubMulDiv(val x:Int, val y:Int) extends tPointSub with tPointMul with tPointDiv {
 override type p = PointSubMulDiv
 override def of(x:Int,y:Int)=
   new PointSubMulDiv(x,y)
}

class PointSumSubMulDiv(val x:Int, val y:Int) extends tPointSum with tPointSub with tPointMul with tPointDiv {
 override type p = PointSumSubMulDiv
 override def of(x:Int,y:Int)=
   new PointSumSubMulDiv(x,y)
}
\end{lstlisting}

Here you can find the 42 code for the Point Algebra.
The main differences with respect to \name as presented in
the paper, is that we use \Q@Resource<><@
to declare a trait,
all trait names start with \texttt{\$}
and are referred with `\Q@()@'.
Method bodies can be just simple expressions, without the need
or `\Q@return@'.

\begin{lstlisting}[basicstyle=\tiny, mathescape=false]
$p: Resource <>< {//4
  method Num x()
  method Num y()
  class method This of(Num x,Num y)
  }

$pointSum: Resource <>< Use[$p()] <>< {//3*4
  method This sum(This that)
    This.of(x: this.x()+that.x(), y: this.y()+that.y())
  }

$pointSub: Resource <>< Use[$p()] <>< {
  method This sub(This that)
    This.of(x: this.x()-that.x(), y: this.y()-that.y())
  }

$pointMul: Resource <>< Use[$p()] <>< {
  method This mul(This that)
    This.of(x: this.x()*that.x(), y: this.y()*that.y())
  }

$pointDiv: Resource <>< Use[$p()] <>< {
  method This div(This that)
    This.of(x: this.x()/that.x(), y: this.y()/that.y())
  }

Point: Use[$p()] <>< {}//1*16

PointSum: Use[$pointSum()] <>< {}

PointSub: Use[$pointSub()] <>< {}

PointMul: Use[$pointMul()] <>< {}

PointDiv: Use[$pointDiv()] <>< {}

PointSumSub: Use[$pointSum();$pointSub()] <>< {}

PointSumMul: Use[$pointSum();$pointMul()] <>< {}

PointSumDiv: Use[$pointSum();$pointDiv()] <>< {}

PointSubMul: Use[$pointSub();$pointMul()] <>< {}

PointSubDiv: Use[$pointSub();$pointDiv()] <>< {}

PointMulDiv: Use[$pointMul();$pointDiv()] <>< {}

PointSumSubDiv: Use[$pointSum();$pointSub();$pointDiv()] <>< {}

PointSumSubMul: Use[$pointSum();$pointSub();$pointMul()] <>< {}

PointSumMulDiv: Use[$pointSum();$pointMul();$pointDiv()] <>< {}

PointSubMulDiv: Use[$pointSub();$pointMul();$pointDiv()] <>< {}

PointSumSubMulDiv: Use[$pointSum();$pointSub();$pointMul();$pointDiv()] <>< {}
\end{lstlisting}


\subsection{FCPoint}
Here the code of Java7 for the second case study;
flavoured and colored points:
\begin{lstlisting}[basicstyle=\tiny]
enum Flavor {
  NONE, SOUR, SWEET, SALTY, SPLICY;
}

class Color {
  final int r;
  final int g;
  final int b;

  public Color(int r, int g, int b) {
    this.r = r;
    this.g = g;
    this.b = b;
  }

  public Color mix(Color that) {
    return new Color((this.r + that.r) / 2, (this.g + that.g) / 2, (this.b + that.b) / 2);
  }
}

class Point {//10 lines
  final int x;
  final int y;

  public Point(int x, int y) {
    this.x = x;
    this.y = y;
  }

  public Point withX(int that) {
    return new Point(that, this.y);
  }

  public Point withY(int that) {
    return new Point(this.x, that);
  }
}

class PointSum extends Point {//9 lines
  public PointSum(int x, int y) {
    super(x, y);
  }

  public PointSum withX(int that) {
    return new PointSum(that, this.y);
  }

  public PointSum withY(int that) {
    return new PointSum(this.x, that);
  }

  public PointSum sum(Point that) {
    return this.withX(this.x + that.x).withY(this.y + that.y);
  }
}

class CPoint extends PointSum {//13 lines
  final Color color;

  public CPoint(int x, int y, Color color) {
    super(x, y);
    this.color = color;
  }

  public CPoint withX(int that) {
    return new CPoint(that, this.y, this.color);
  }

  public CPoint withY(int that) {
    return new CPoint(this.x, that, this.color);
  }

  public CPoint withColor(Color that) {
    return new CPoint(this.x, this.y, that);
  }

  public CPoint merge(CPoint that) {
    return this.withColor(this.color.mix(that.color));
  }
}

class FCPoint extends CPoint {//15 lines
  final Flavor flavor;

  public FCPoint(int x, int y, Color color, Flavor flavor) {
    super(x, y, color);
    this.flavor = flavor;
  }

  public FCPoint withX(int that) {
    return new FCPoint(that, this.y, this.color, this.flavor);
  }

  public FCPoint withY(int that) {
    return new FCPoint(this.x, that, this.color, this.flavor);
  }

  public FCPoint withColor(Color that) {
    return new FCPoint(this.x, this.y, that, this.flavor);
  }

  public FCPoint withFlavor(Flavor that) {
    return new FCPoint(this.x, this.y, this.color, that);
  }

  public FCPoint merge(FCPoint that) {
    return this.withColor(that.color).withFlavor(that.flavor);
  }
}
\end{lstlisting}
Here the code of Scala for flavoured and colored points:
\begin{lstlisting}[basicstyle=\tiny]
trait tPointState {//8
  type p <: tPointState
  def x: Int
  def y: Int
  def withX(that:Int):p
  def withY(that:Int):p
  def of(x:Int,y:Int):p
  def merge(that:p):p
}

trait tPointSum extends tPointState {//3
  def sum(that:p)=
    this.merge(that).withX(this.x+that.x).withY(this.y+that.y)  
}

trait tColored {//6*2
  type p <: tColored
  def color:Int
  def withColor(that:Int):p
  def merge(that:p)=
    this.withColor(this.color+that.color)
  }

trait tFlavored{
  type p <: tFlavored
  def flavor:Int
  def withFlavor(that:Int):p
  def merge(that:p)=
    this.withFlavor(that.flavor)
  }
//glue code from now on //12 lines
class CPoint(val x:Int, val y:Int, val color:Int) extends tPointSum with tColored {
  override type p = CPoint
  def of(x:Int,y:Int)=
    this.of(x,y,0)
  def of(x:Int,y:Int,color:Int)=
    new CPoint(x,y,color)
  def withX(that:Int)=
    of(that,y,color)
  def withY(that:Int)=
    of(x,that,color)
  def withColor(that:Int)=
    of(x,y,that)
  }
//18 lines
class FCPoint(val x:Int, val y:Int, val color:Int, val flavor:Int) extends tPointSum with tColored with tFlavored{
  override type p = FCPoint
  def of(x:Int,y:Int)=
    this.of(x,y,0,0)
  def of(x:Int,y:Int,color:Int,flavor:Int)=
    new FCPoint(x,y,color,flavor)
  def withX(that:Int)=
    of(that,y,color,flavor)
  def withY(that:Int)=
    of(x,that,color,flavor)
  def withColor(that:Int)=
    of(x,y,that,flavor)
  def withFlavor(that:Int)=
    of(x,y,color,that)
  def superMergeFlavoured(that:FCPoint)=
    super[tFlavored].merge(that)
  override def merge(that:FCPoint)=
      super[tColored].merge(that).superMergeFlavoured(that)
  }
\end{lstlisting}

Here you can find the 42 code for flavored and colored points..

\begin{lstlisting}[basicstyle=\tiny, mathescape=false]
Flavor: Enumeration"sour, sweet, salty, spicy"

Color:Data<><{Num r Num g Num b
  method This mix(This that)
    Color( r:(this.r()+that.r())/2Num, g:(this.g()+that.g())/2Num, b:(this.b()+that.b())/2Num)
  }

$p: Resource <>< {// 7 lines
  method Num x()
  method Num y() //getters
  method This withX(Num that)
  method This withY(Num that)//withers
  class method This of(Num x,Num y)
  method This merge(This that) //new method merge!
  }

$pointSum: Resource <>< Use[$p()] <>< {// 3 lines
  method This sum(This that)
    this.merge(that).withX(this.x()+that.x()).withY(this.y()+that.y())
  }

$colored: Resource <>< {// 5 *2 lines
  method Color color()
  method This withColor(Color that)
  method This merge(This that) //how to merge color handled here
    this.withColor(this.color().mix(that.color()))
  }

$flavored: Resource <>< {
  method Flavor flavor() //very similar to colored
  method This withFlavor(Flavor that)
  method This merge(This that) //merging flavors handled here
    this.withFlavor(that.flavor())//inherits "that" flavor
  }

CPoint: Use[$pointSum();$colored()] <>< {// 4 lines
  class method This of(Num x,Num y)
    This.of(x:x,y:y,color:Color(r:100Num,g:0Num,b:0Num))
  class method This of(Num x, Num y,Color color)
  }

FCPoint: Use[//9 lines
  Refactor2.toAbstract(selector:\"merge(that)", into:\"_1merge(that)")<><$colored();
  Refactor2.toAbstract(selector:\"merge(that)", into:\"_2merge(that)")<><$flavored();
  $pointSum()] <>< {
    class method This of(Num x,Num y)
      This.of(x:x,y:y,color:Color(r:100Num,g:0Num,b:0Num),flavor:Flavor.none())
    class method This of(Num x, Num y,Color color,Flavor flavor)
    method This merge(This that)
      this._1merge(that)._2merge(that)
    }
\end{lstlisting}

\subsection{Expression problem}
Here the original code of Scala solving the expression problem.
As you can see, sometimes multiple operations/cases are declared together in
the same trait.
\begin{lstlisting}[basicstyle=\tiny]
trait Base {
  type exp <: Exp

  trait Exp {
    def eval: Int
  }

  class Num(val value: Int) extends Exp {
    def eval: Int = value
  }

  type BaseNum = Num
}

trait BasePlus extends Base {
  class Plus(val left: exp, val right: exp) extends Exp {
    def eval: Int = left.eval + right.eval
  }

  type BasePlus = Plus
}

trait BaseNeg extends Base {
  class Neg(val term: exp) extends Exp {
    def eval = -term.eval
  }

  type BaseNeg = Neg
}
trait BasePlusNeg extends BasePlus with BaseNeg
trait Show extends Base {
  type exp <: Exp

  trait Exp extends super.Exp {
    def show: String
  }

  trait NumBehavior extends Exp {
    self: BaseNum =>
    override def show: String = value.toString
  }

  final class Num(v: Int) extends BaseNum(v) with NumBehavior with Exp
}

trait ShowPlusNeg extends BasePlusNeg with Show {
  trait PlusBehavior {
    self: BasePlus =>
    def show = left.show + "+" + right.show;
  }

  final class Plus(l: exp, r: exp) extends BasePlus(l, r) with PlusBehavior with Exp

  trait NegBehavior {
    self: BaseNeg =>
    def show = "-(" + term.show + ")";
  }
  
  class Neg(t: exp) extends BaseNeg(t) with NegBehavior with Exp
}

trait DblePlusNeg extends BasePlusNeg {
  type exp <: Exp

  trait Exp extends super.Exp {
    def dble: exp
  }

  def Num(v: Int): exp

  def Plus(l: exp, r: exp): exp

  def Neg(t: exp): exp

  trait NumBehavior {
    self: BaseNum =>
    def dble = Num(value * 2)
  }

  final class Num(v: Int) extends BaseNum(v) with NumBehavior with Exp

  trait PlusBehavior {
    self: BasePlus =>
    def dble = Plus(left.dble, right.dble)
  }

  class Plus(l: exp, r: exp) extends BasePlus(l, r) with PlusBehavior with Exp

  trait NegBehavior {
    self: BaseNeg =>
    def dble = Neg(term.dble)
  }

  class Neg(t: exp) extends super.Neg(t) with NegBehavior with Exp
}
//-- 52 lines up to here, not counting new lines and '}'
trait Equals extends Base {
  type exp <: Exp;

  trait Exp extends super.Exp {
    def eql(other: exp): Boolean;

    def isNum(v: Int): Boolean = false;
  }

  trait NumBehavior extends Exp {
    self: BaseNum =>
    def eql(other: exp): Boolean = other.isNum(value);

    override def isNum(v: Int) = v == value;
  }

  final class Num(v: Int) extends BaseNum(v) with NumBehavior with Exp
}

trait EqualsPlusNeg extends BasePlusNeg with Equals {
  type exp <: Exp;

  trait Exp extends super[BasePlusNeg].Exp
    with super[Equals].Exp {
    def isPlus(l: exp, r: exp): Boolean = false;

    def isNeg(t: exp): Boolean = false;
  }

  final class Num(v: Int) extends BaseNum(v)
    with NumBehavior // effectively super[Equals].NumBehavior
    with Exp

  trait PlusBehavior extends Exp {
    self: BasePlus =>
    def eql(other: exp): Boolean = other.isPlus(left, right);

    override def isPlus(l: exp, r: exp) = (left eql l) && (right eql r)
  }

  final class Plus(l: exp, r: exp) extends BasePlus(l, r) with PlusBehavior with Exp

  trait NegBehavior extends Exp {
    self: BaseNeg =>
    def eql(other: exp): Boolean = other.isNeg(term);

    override def isNeg(t: exp) = term eql t
  }

  final class Neg(t: exp) extends BaseNeg(t) with NegBehavior with Exp
}

trait EqualsShowPlusNeg extends EqualsPlusNeg with ShowPlusNeg {
  type exp <: Exp

  trait Exp extends super[EqualsPlusNeg].Exp
    with super[ShowPlusNeg].Exp

  trait NumBehavior extends super[EqualsPlusNeg].NumBehavior with super[ShowPlusNeg].NumBehavior {
    self: BaseNum =>
  }

  class Num(v: Int) extends BaseNum(v) with NumBehavior with Exp

  trait PlusBehavior extends super[EqualsPlusNeg].PlusBehavior with super[ShowPlusNeg].PlusBehavior {
    self: BasePlus =>
  }

  class Plus(l: exp, r: exp) extends BasePlus(l, r) with PlusBehavior with Exp

  trait NegBehavior extends super[EqualsPlusNeg].NegBehavior with super[ShowPlusNeg].NegBehavior {
    self: BaseNeg =>
  }

  class Neg(term: exp) extends BaseNeg(term) with NegBehavior with Exp
}
//--- 40 lines for equals
\end{lstlisting}


In the following, the fully modularized Scala code, so that
every trait define exactly one operation for one datavariant:
\begin{lstlisting}[basicstyle=\tiny]
trait Base {
  type exp <: Exp

  trait Exp {  }
}

trait BaseNum extends Base {
  class Num(val value: Int) extends Exp {  }

  type BaseNum = Num
}

trait BasePlus extends Base {
  class Plus(val left: exp, val right: exp) extends Exp {  }

  type BasePlus = Plus
}

trait BaseNeg extends Base {
  class Neg(val term: exp) extends Exp {  }

  type BaseNeg = Neg
}
//----------------------EVAL
trait Eval extends Base {
  type exp <: Exp

  trait Exp extends super.Exp {
    def eval: Int
  }
}

//----------------------EVALNUM
trait EvalNum extends BaseNum with Eval {
  trait NumBehavior {
    self: BaseNum =>
    def eval: Int = value
  }

  class Num(v: Int) extends BaseNum(v) with NumBehavior with Exp
}

//----------------------EVALPLUS
trait EvalPlus extends BasePlus with Eval {
  trait PlusBehavior {
    self: BasePlus =>
    def eval = left.eval + right.eval;
  }

  class Plus(l: exp, r: exp) extends BasePlus(l, r) with PlusBehavior with Exp
}
//----------------------EVALNEG
trait EvalNeg extends BaseNeg with Eval {
  trait NegBehavior {
    self: BaseNeg =>
    def eval = - term.eval;
  }
  
  class Neg(t: exp) extends BaseNeg(t) with NegBehavior with Exp
}
//----------------------SHOW
trait Show extends Base {
  type exp <: Exp

  trait Exp extends super.Exp {
    def show: String
  }
}
//----------------------SHOWNUM
trait ShowNum extends BaseNum with Show {
  trait NumBehavior {
    self: BaseNum =>
    def show: String = value.toString
  }

  class Num(v: Int) extends BaseNum(v) with NumBehavior with Exp
}

//----------------------SHOWPLUS
trait ShowPlus extends BasePlus with Show {
  trait PlusBehavior {
    self: BasePlus =>
    def show = left.show + "+" + right.show;
  }

  class Plus(l: exp, r: exp) extends BasePlus(l, r) with PlusBehavior with Exp
}
//----------------------SHOWNEG
trait ShowNeg extends BaseNeg with Show {
  trait NegBehavior {
    self: BaseNeg =>
    def show = "-(" + term.show + ")";
  }

  class Neg(t: exp) extends BaseNeg(t) with NegBehavior with Exp
}
//----------------------DBLE
trait Dble extends Base {
  type exp <: Exp

  trait Exp extends super.Exp {
    def dble: exp
  }
}
//----------------------DBLENUM
trait DbleNum extends BaseNum with Dble {
  type exp <: Exp

  trait Exp extends super[BaseNum].Exp with super[Dble].Exp 
  
  trait NumBehavior {
    self: BaseNum =>
    def dble = Num(value * 2)
  }
  
  def Num(v: Int): exp

  class Num(v: Int) extends super.Num(v) with NumBehavior
}
//----------------------DBLEPLUS
trait DblePlus extends BasePlus with Dble {
  type exp <: Exp
  
  trait Exp extends super[BasePlus].Exp with super[Dble].Exp
  
  trait PlusBehavior {
    self: BasePlus =>
    def dble = Plus(left.dble, right.dble)
  }

  def Plus(l: exp, r: exp): exp
  
  class Plus(l: exp, r: exp) extends super.Plus(l, r) with PlusBehavior with Exp
}
//----------------------DBLENEG
trait DbleNeg extends BaseNeg with Dble {
  type exp <: Exp
  
  trait Exp extends super[BaseNeg].Exp with super[Dble].Exp

  trait NegBehavior {
    self: BaseNeg =>
    def dble = Neg(term.dble)
  }

  def Neg(t: exp): exp

  class Neg(t: exp) extends super.Neg(t) with NegBehavior with Exp
}//78 lines up to here, not couting new lines and '}'


//glue code: 27 lines
object All0 extends 
Eval with Show with Dble 
with EvalNum with EvalPlus with EvalNeg 
with ShowNum with ShowPlus with ShowNeg 
with DbleNum with DblePlus with DbleNeg 
    {
  override type exp = Exp
  
  trait Exp extends 
  super[Eval].Exp with super[Show].Exp with super[Dble].Exp
  with super[DbleNum].Exp with super[DblePlus].Exp with super[DbleNeg].Exp 

  trait NumBehavior extends
  super[EvalNum].NumBehavior with super[ShowNum].NumBehavior
  with super[DbleNum].NumBehavior{
    self: BaseNum =>
  }

  class Num(v: Int) extends BaseNum(v) with NumBehavior with Exp

  trait PlusBehavior extends
  super[EvalPlus].PlusBehavior with super[ShowPlus].PlusBehavior
  with super[DblePlus].PlusBehavior {
    self: BasePlus =>
  }

   class Plus(l: exp, r: exp) extends BasePlus(l, r) with PlusBehavior with Exp

  trait NegBehavior extends
      super[EvalNeg].NegBehavior with super[ShowNeg].NegBehavior
      with super[DbleNeg].NegBehavior {
    self: BaseNeg =>
  }

  class Neg(t: exp) extends BaseNeg(t) with NegBehavior with Exp
  
  def Num(v: Int) = new Num(v)
  
  def Plus(l: exp, r: exp) =new Plus(l,r)

  def Neg(t: exp) = new Neg(t)
}
//----------------------EQUALS
trait Equals extends Base {
  type exp <: Exp;

  trait Exp extends super.Exp {
    def eql(other: exp): Boolean;
  }
}
//----------------------EQUALSNUM
trait EqualsNum extends BaseNum with Equals {
  type exp <: Exp;

  trait Exp extends super.Exp {
    def isNum(v: Int): Boolean = false;
  }

  trait NumBehavior extends Exp {
    self: BaseNum =>
    def eql(other: exp): Boolean = other.isNum(value);

    override def isNum(v: Int) = v == value;
  }

  class Num(v: Int) extends BaseNum(v) with NumBehavior with Exp
}
//----------------------EQUALSPLUS
trait EqualsPlus extends BasePlus with Equals {
  type exp <: Exp;

  trait Exp extends super[BasePlus].Exp with super[Equals].Exp {
    def isPlus(l: exp, r: exp): Boolean = false;
  }

  trait PlusBehavior extends Exp {
    self: BasePlus =>
    def eql(other: exp): Boolean = other.isPlus(left, right);

    override def isPlus(l: exp, r: exp) = (left eql l) && (right eql r)
  }

  class Plus(l: exp, r: exp) extends BasePlus(l, r) with PlusBehavior with Exp
}
//----------------------EQUALSNEG
trait EqualsNeg extends BaseNeg with Equals {
  type exp <: Exp;

  trait Exp extends super[BaseNeg].Exp with super[Equals].Exp {
    def isNeg(t: exp): Boolean = false;
  }

  trait NegBehavior extends Exp {
    self: BaseNeg =>
    def eql(other: exp): Boolean = other.isNeg(term);

    override def isNeg(t: exp) = term eql t
  }

  class Neg(t: exp) extends BaseNeg(t) with NegBehavior with Exp
}
//31 lines for equals, 29 glue
object All extends 
Eval with Show with Dble with Equals 
with EvalNum with EvalPlus with EvalNeg 
with ShowNum with ShowPlus with ShowNeg 
with DbleNum with DblePlus with DbleNeg 
with EqualsNum with EqualsPlus with EqualsNeg 
    {
  override type exp = Exp
  
  trait Exp extends 
  super[Eval].Exp with super[Show].Exp with super[Dble].Exp with super[Equals].Exp
  with super[DbleNum].Exp with super[DblePlus].Exp with super[DbleNeg].Exp
  with super[EqualsNum].Exp with super[EqualsPlus].Exp with super[EqualsNeg].Exp

  trait NumBehavior extends
  super[EvalNum].NumBehavior with super[ShowNum].NumBehavior
  with super[DbleNum].NumBehavior with super[EqualsNum].NumBehavior {
    self: BaseNum =>
  }

  class Num(v: Int) extends BaseNum(v) with NumBehavior with Exp

  trait PlusBehavior extends
  super[EvalPlus].PlusBehavior with super[ShowPlus].PlusBehavior
  with super[DblePlus].PlusBehavior with super[EqualsPlus].PlusBehavior {
    self: BasePlus =>
  }

   class Plus(l: exp, r: exp) extends BasePlus(l, r) with PlusBehavior with Exp

  trait NegBehavior extends
      super[EvalNeg].NegBehavior with super[ShowNeg].NegBehavior
      with super[DbleNeg].NegBehavior with super[EqualsNeg].NegBehavior {
    self: BaseNeg =>
  }

  class Neg(t: exp) extends BaseNeg(t) with NegBehavior with Exp
  
  def Num(v: Int) = new Num(v)
  
  def Plus(l: exp, r: exp) =new Plus(l,r)

  def Neg(t: exp) = new Neg(t)
}
\end{lstlisting}


Finally, here is the 42 code solving the expression problem.
The classes \Q@RenNum@,\Q@RenPlus@ and \Q@RenNeg@
can be understood as declarations of `short-cuts'
In order to make the full 42 language more in line with the
(more compact) syntax of \name.
In 42 those `short-cuts' can be very expressive 
but here we use them only to better align 42 and \name.
\begin{lstlisting}[basicstyle=\tiny, mathescape=false]]
RenNum:{class method Library<>< (Library that) exception Guard
  Refactor2.rename(path:\"T" into:\"Num")<><that
  }

RenPlus:{class method Library<>< (Library that) exception Guard
  Refactor2.rename(path:\"T" into:\"Plus")<><that
  }

RenNeg:{class method Library<>< (Library that) exception Guard
  Refactor2.rename(path:\"T" into:\"Neg")<><that
  }
//above, header not counted

$exp:Resource<><{ //3 lines
  Exp:{interface}

  T:{implements Exp}
  }

$num:Resource<><Use[RenNum<><$exp()]<><{ //4 lines
  Num:{
    method Size value()
    class method Num of(Size value)
    }
  }

$plus:Resource<><Use[RenPlus<><$exp()]<><{  //6 lines
  Exp:{interface}

  Plus:{
    method Exp left()
    method Exp right()
    class method Plus of(Exp left, Exp right)
    }
  }

$neg:Resource<><Use[RenNeg<><$exp()]<><{  //5 lines
  Exp:{interface}

  Neg:{
    method Exp term()
    class method Neg of(Exp term)
  }
}

$eval:Resource<>< {  //4 lines
  Exp:{interface
    method Size eval()
    }

  T:{implements Exp}
  }

$evalNum:Resource<><Use[$num(); RenNum<><$eval()]<>< {  // 2*3 lines
  Num:{ method Size eval() this.value() }
  }

$evalPlus:Resource<><Use[$plus(); RenPlus<><$eval()]<>< {
  Plus:{ method Size eval() this.left().eval()+this.right().eval() }
  }

$evalNeg:Resource<><Use[$neg(); RenNeg<><$eval()]<>< {
  Neg:{ method Size eval() Size"-1" * this.term().eval() }
  }

$show:Resource<><Use[$exp()]<><{  //4 lines
  Exp:{interface
    method S show()
    }

  T:{implements Exp}
  }

$showNum:Resource<><Use[$num(); RenNum<><$show()]<><{  //2*3 lines
  Num:{ method S show() this.value().toS() }
  }

$showPlus:Resource<><Use[$plus(); RenPlus<><$show()]<><{
  Plus:{ method S show() this.left().show()++S" + "++this.right().show() }
  }

$showNeg:Resource<><Use[$neg(); RenNeg<><$show()]<><{
  Neg:{ method S show() S"-("++this.term().show()++S")" }
  }
//----------------
$double:Resource<><{  //4 lines
  Exp:{interface
    method Exp double()
    }

  T:{implements Exp}
  }

$doubleNum:Resource<><Use[$num(); RenNum<><$double()]<><{ //2*3 lines
  Exp:{interface}//not counted, needed in the full language only to guide scope resolution in the desugaring

  Num:{ method Exp double() Num.of(value: this.value()*2Size) }
  }

$doublePlus:Resource<><Use[$plus(); RenPlus<><$double()]<><{
  Exp:{interface}

  Plus:{ method Exp double() Plus.of(left: this.left().double(), right: this.right().double()) }
  }

$doubleNeg:Resource<><Use[$neg(); RenNeg<><$double()]<><{
  Exp:{interface}

  Neg:{ method Exp double() Neg.of(term: this.term().double()) }
  }
//---------
$equals:Resource<><Use[$eval()]<><{  //6 lines
  Exp:{interface
    method Bool eql(Exp that)
    method Bool equalToT(T that)
    }

  T:{implements Exp
    method eql(that) that.equalToT(this)
    }
  }

$equalsNum:Resource<><Use[$num();  //5*3 lines
  Refactor2.Method[rename:\"equalToT(that)" of:\"Exp" into:\"equalToNum(that)"]
  <><RenNum<>< $equals()]<><{
  Num:{
    method Bool equalToNum(Num that) this.value()==that.value()
    }
  }

$equalsPlus:Resource<><Use[$plus();
  Refactor2.Method[rename:\"equalToT(that)" of:\"Exp" into:\"equalToPlus(that)"]
  <><RenPlus<><$equals()]<><{
  Plus:{
    method Bool equalToPlus(Plus that)
      this.left().eql(that.left()) & this.right().eql(that.right())
    }
  }

$equalsNeg:Resource<><Use[$neg();
  Refactor2.Method[rename:\"equalToT(that)" of:\"Exp" into:\"equalToNeg(that)"]
  <><RenNeg<><$equals()]<><{
  Neg:{
    method Bool equalToNeg(Neg that) this.term().eql(that.term())
    }
  }

 //16+6 lines glue code
$equalsAll:Resource<><Use[$equalsNeg();$equalsPlus();$equalsNum()]<><{
  Exp:{interface
    method Bool eql(Exp that)
    method Bool equalToNum(Num that)
    method Bool equalToPlus(Plus that)
    method Bool equalToNeg(Neg that)
    method Size eval()
    }

  Num:{implements Exp
    method equalToPlus(that) Bool.false()
    method equalToNeg(that) Bool.false()
    }

  Plus:{implements Exp
    method equalToNum(that) Bool.false()
    method equalToNeg(that) Bool.false()
    }

  Neg:{implements Exp
    method equalToNum(that) Bool.false()
    method equalToPlus(that) Bool.false()
    }
  }

$evalAll:Resource<><Use[$evalPlus();$evalNeg()]<><$evalNum()

$showAll:Resource<><Use[$showNum();$showPlus()]<><$showNeg()

$doubleAll:Resource<><Use[$doubleNum();$doublePlus()]<><$doubleNeg()

ESDAll0:Use[Use[$evalPlus();$evalNeg()]<><$evalNum();
  Use[$showNum();$showPlus()]<><$showNeg();
  $doubleNum();$doublePlus()]<><$doubleNeg()


ESDAll:Use[$evalAll();$showAll()]<><$doubleAll()

ESDEAll:Use[$evalAll();$showAll();$doubleAll()]<><$equalsAll()

$equals2:Resource<><{ //7 lines
  Exp:{interface method Bool equals(Exp that)}
  T:{implements Exp
    method Bool exactEquals(T that)
    method equals(that){
      with that (on T  return this.exactEquals(that) )
      return Bool.false()
      }
    }
  }

$equalsNum2:Resource<><Use[$num();RenNum<><$equals2()]<><{  //2*3 lines
  Num:{ method Bool exactEquals(Num that) this.value().equals(that.value()) }
  }

$equalsPlus2:Resource<><Use[$plus();RenPlus<><$equals2()]<><{
  Plus:{ method Bool exactEquals(Plus that) this.left().equals(that.left()) & this.right().equals(that.right()) }
  }

$equalsNeg2:Resource<><Use[$neg();RenNeg<><$equals2()]<><{
  Neg:{method Bool exactEquals(Neg that) this.term().equals(that.term()) }
  }

$equalsAll2:Resource<><Use[$equalsNum2();$equalsPlus2()]<><$equalsNeg2()

ESDEAll2:Use[$evalAll();$showAll();$doubleAll()]<><$equalsAll2()

\end{lstlisting}
