%% For double-blind review submission, w/o CCS and ACM Reference (max submission space)
%\documentclass[acmsmall,review,anonymous]{acmart}\settopmatter{printfolios=true,printccs=false,printacmref=false}
\documentclass[acmsmall]{acmart}\settopmatter{printfolios=true,printccs=false,printacmref=false}
%% For double-blind review submission, w/ CCS and ACM Reference
%\documentclass[acmsmall,review,anonymous]{acmart}\settopmatter{printfolios=true}
%% For single-blind review submission, w/o CCS and ACM Reference (max submission space)
%\documentclass[acmsmall,review]{acmart}\settopmatter{printfolios=true,printccs=false,printacmref=false}
%% For single-blind review submission, w/ CCS and ACM Reference
%\documentclass[acmsmall,review]{acmart}\settopmatter{printfolios=true}
%% For final camera-ready submission, w/ required CCS and ACM Reference
%\documentclass[acmsmall]{acmart}\settopmatter{}


%% Journal information
%% Supplied to authors by publisher for camera-ready submission;
%% use defaults for review submission.
\acmJournal{PACMPL}
\acmVolume{1}
\acmNumber{CONF} % CONF = POPL or ICFP or OOPSLA
\acmArticle{1}
\acmYear{2018}
\acmMonth{1}
\acmDOI{} % \acmDOI{10.1145/nnnnnnn.nnnnnnn}
\startPage{1}

%% Copyright information
%% Supplied to authors (based on authors' rights management selection;
%% see authors.acm.org) by publisher for camera-ready submission;
%% use 'none' for review submission.
\setcopyright{none}
%\setcopyright{acmcopyright}
%\setcopyright{acmlicensed}
%\setcopyright{rightsretained}
%\copyrightyear{2018}           %% If different from \acmYear

%% Bibliography style
\bibliographystyle{plain}
%% Citation style
%% Note: author/year citations are required for papers published as an
%% issue of PACMPL.
%\citestyle{acmauthoryear}   %% For author/year citations


%%%%%%%%%%%%%%%%%%%%%%%%%%%%%%%%%%%%%%%%%%%%%%%%%%%%%%%%%%%%%%%%%%%%%%
%% Note: Authors migrating a paper from PACMPL format to traditional
%% SIGPLAN proceedings format must update the '\documentclass' and
%% topmatter commands above; see 'acmart-sigplanproc-template.tex'.
%%%%%%%%%%%%%%%%%%%%%%%%%%%%%%%%%%%%%%%%%%%%%%%%%%%%%%%%%%%%%%%%%%%%%%


%% Some recommended packages.
\usepackage{url}
\usepackage{mathpartir}
\usepackage{mathtools}
\usepackage{listings}
\usepackage{color}
\usepackage{xspace}
%\usepackage{times}%is this making all the font into times?
\usepackage{comment}
\usepackage{rotating} 
\usepackage[T1]{fontenc}
\usepackage{enumitem}
\setlist{nosep}
\setlist{leftmargin=1ex}
\setlist[itemize]{noitemsep,nolistsep}
\setlist[enumerate]{noitemsep,nolistsep}
%\let\oldparagraph\paragraph
%\renewcommand{\paragraph}[1]{\vspace{-5pt}\oldparagraph{#1}}
\renewcommand{\paragraph}[1]{\noindent{\bf #1}}

\let\oldsubsection\subsection
\renewcommand{\subsection}[1]{\vspace{-4pt}\oldsubsection{#1}\vspace{-4pt}}

%\newenvironment{listing}{\vspace{-3pt}\begin{lstlisting}}{\end{lstlisting}\vspace{-3pt}}
%
\lstset{language=Java,
  basicstyle=\ttfamily\footnotesize,%\small,%\scriptsize,
  keywordstyle=\bfseries,%\color{darkRed},
  showstringspaces=false,
  mathescape=true,
  xleftmargin=0pt,
  xrightmargin=0pt,
  breaklines=false,
  breakatwhitespace=false,
  breakautoindent=false,
  %linewidth=4\textwidth,% should be enough
%  identifierstyle=\idstyle
 morekeywords={method,Use,This,constructor,as,into,rename},
 deletekeywords={double}
}
%
\newcommand\saveSpace{}
%\newcommand\saveSpace{\vspace{-2pt}}

\newcommand\Rotated[1]{\begin{turn}{90}\begin{minipage}{10em}#1\end{minipage}\end{turn}}
%
%marco
\newcommand{\Q}{\lstinline}
\newcommand\Opt[1]{#1 ?}
%bnf
\newenvironment{bnf}{$\begin{array}{lcll}}{\end{array}$}
\newcommand{\prodFull}[3]{#1&::=&#2&\mbox{#3}}
%\newcommand{\prodFull}[3]{#1&::=&\mbox{#2}&\mbox{#3}}
\newcommand{\prodInline}[2]{#1::=#2}
\newcommand{\prodNextLine}[2]{&&#1&\mbox{#2}}
\newcommand{\terminal}[1]{%
\ensuremath{$\texttt{#1}$}%
}
\newcommand{\terminalCode}[1]{\mbox{\lstinline{#1}}}
\newcommand{\metavariable}[1]{%
\ensuremath{\mathit{#1}}%
}
%----------------------
\newcommand\dom{\text{dom}}
\newcommand*{\Scale}[2][4]{\scalebox{#1}{$#2$}}%
\newcommand\smallDs{{\Scale[0.5]{\overline{\mD}}}}
\newcommand\vds{{v_{\smallDs}}}
\newcommand\Rulename[1]{{\textsc{#1}}}
\newcommand\ctx{{\mathcal{E}}}
\newcommand\mID{\metavariable{ID}}
\newcommand\mL{\metavariable{L}}
\newcommand\mE{\metavariable{E}}
\newcommand\mC{\metavariable{C}}
\newcommand\mT{\metavariable{T}}
\newcommand\mV{\metavariable{V}} %code value t or L
\newcommand\mM{\metavariable{M}}
\newcommand\mG{\Gamma}
\newcommand\mDE{\metavariable{DE}}
\newcommand\mTE{\metavariable{TE}}
\newcommand\mCE{\metavariable{CE}}
\newcommand\mMD{\metavariable{MD}}
\newcommand\mTD{\metavariable{TD}}
\newcommand\mCD{\metavariable{CD}}
\newcommand\mD{\metavariable{D}}
\newcommand\me{\metavariable{e}}
\newcommand\mx{\metavariable{x}}
\newcommand\mm{\metavariable{m}}
\newcommand\mt{\metavariable{t}}
\newcommand\use{\terminalCode{Use}}
\newcommand\oC{\mbox{\lstinline@\{@}}
\newcommand\cC{\mbox{\lstinline@\}@}}
\newcommand\oR{\mbox{\lstinline@(@}}
\newcommand\cR{\mbox{\lstinline@)@}}
%--------------------------
\newcommand{\mynotes}[3]{{\color{#2} {\sc #1}: #3}}
\newcommand\bruno[1]{\mynotes{Bruno}{red}{#1}}
\newcommand\marco[1]{\mynotes{Marco}{blue}{#1}}

\newcommand{\syndef}{$::=$}

\newcommand\name{{\bf $42_{\mu}$}\xspace}

\begin{document}

%% Title information
\title{Separating Use and Reuse to Improve Both}         %% [Short Title] is optional;
                                        %% when present, will be used in
                                        %% header instead of Full Title.
%\titlenote{Separating Use and Reuse to Improve Both}             %% \titlenote is optional;
                                        %% can be repeated if necessary;
                                        %% contents suppressed with 'anonymous'
%\subtitle{Subtitle}                     %% \subtitle is optional
%\subtitlenote{with subtitle note}       %% \subtitlenote is optional;
                                        %% can be repeated if necessary;
                                        %% contents suppressed with 'anonymous'


%% Author information
%% Contents and number of authors suppressed with 'anonymous'.
%% Each author should be introduced by \author, followed by
%% \authornote (optional), \orcid (optional), \affiliation, and
%% \email.
%% An author may have multiple affiliations and/or emails; repeat the
%% appropriate command.
%% Many elements are not rendered, but should be provided for metadata
%% extraction tools.

%% Author with single affiliation.
\author{Marco Servetto}
%\authornote{with author1 note}          %% \authornote is optional;
                                        %% can be repeated if necessary
%\orcid{nnnn-nnnn-nnnn-nnnn}             %% \orcid is optional
\affiliation{
 % \position{Position1}
%  \department{Department1}              %% \department is recommended
  \institution{Victoria University of Wellington}            %% \institution is required
%  \streetaddress{Street1 Address1}
 % \city{City1}
 % \state{State1}
 % \postcode{Post-Code1}
  \country{New Zealand}                    %% \country is recommended
}
\email{marco.servetto@ecs.vuw.ac.nz}          %% \email is recommended

%Arora
\author{Hrshikesh Arora}
\affiliation{
  \institution{Victoria University of Wellington}      
  \country{New Zealand} 
}
\email{arorahrsh@ecs.vuw.ac.nz}

%Bruno
\author{Bruno C. d. S. Oliveira}
\affiliation{
  \institution{The University of Hong Kong}      
  \country{Hong Kong} 
}
\email{bruno@cs.hku.hk}  



%% Abstract
%% Note: \begin{abstract}...\end{abstract} environment must come
%% before \maketitle command
\begin{abstract}
\saveSpace\saveSpace\saveSpace
In most OO languages subclassing/inheritance implies
subtyping. This is considered by many a design error, but it
seems required for technical reasons, due to what we call the
\emph{this-leaking problem}, showing that separating
inheritance from subtyping is non-trivial and requires a significant
departure from the OO models in existing mainstream OO languages.
We are aware of at least 3 independently designed research languages 
addressing this limitation: \emph{TraitRecordJ}, \emph{Package Templates} and \emph{DeepFJig}.
The goal of this paper is to synthesize and improve on
the main ideas of those very different designs into a nominally typed
minimalist language, called \name.
By making our type system \textbf{distinguish between code-use and code-reuse}
we can separate inheritance and subtyping, while avoiding 
redundant abstract declarations required in TraitRecordJ and
DeepFJig. At the same time \emph{self construction},
\emph{binary methods} and \emph{recursive types} are also supported.
Moreover, we provide a novel and elegant solution to \textbf{uniformly
handling behaviour and state} within trait composition and without the need of glue-code.
These ideas have been implemented in the full 42 language, 
supporting all the examples shown in this paper.
%EVALUATION
We depict the logical simplicity of our model through an elegant and compact (one page) formal calculus.
We also evaluate our design over three case studies comparing 42 with languages where use and reuse are connected, like Java and Scala.
\saveSpace\saveSpace

\keywords{Code Reuse, Object-Oriented Programming}
\saveSpace\saveSpace
\end{abstract}


%% 2012 ACM Computing Classification System (CSS) concepts
%% Generate at 'http://dl.acm.org/ccs/ccs.cfm'.
%\begin{CCSXML}
%<ccs2012>
%<concept>
%<concept_id>10011007.10011006.10011008</concept_id>
%<concept_desc>Software and its engineering~General programming languages</concept_desc>
%<concept_significance>500</concept_significance>
%</concept>
%<concept>
%<concept_id>10003456.10003457.10003521.10003525</concept_id>
%<concept_desc>Social and professional topics~History of programming languages</concept_desc>
%<concept_significance>300</concept_significance>
%</concept>
%</ccs2012>
%\end{CCSXML}

%\ccsdesc[500]{Software and its engineering~General programming languages}
%\ccsdesc[300]{Social and professional topics~History of programming languages}
%% End of generated code


%% Keywords
%% comma separated list
% \keywords{keyword1, keyword2, keyword3}  %% \keywords are mandatory in final camera-ready submission


%% \maketitle
%% Note: \maketitle command must come after title commands, author
%% commands, abstract environment, Computing Classification System
%% environment and commands, and keywords command.
\maketitle

\section{Introduction}\label{sec:intro}
In mainstream OO languages like Java, C++ or C\# subclassing 
implies subtyping. For example, a Java subclass definition, such as 
\Q@class A extends B {}@
\noindent does two things at the same time:
it {\bf inherits} code from \lstinline{B}; and it {\bf creates
a subtype} of \lstinline{B}. Therefore in a language like Java, 
a subclass is \emph{always} a subtype of the extended class.

Historically, there has been a lot of focus on
separating subtyping from subclassing~\cite{cook}.  This is claimed to be
good for code-reuse, design and reasoning. There are at
least two distinct situations where the separation of subtyping and 
subclassing is helpful.

\begin{itemize}

\item {\bf Allowing inheritance/reuse even when subtyping is impossible:} 
In some situations a subclass contains methods whose signatures 
are incompatible with the superclass, yet inheritance is still
possible. A typical example, which was illustrated by Cook et al.~\cite{cook}, are 
classes with \emph{binary methods}~\cite{bruce96binary}.

\item {\bf Preventing unintended subtyping:} For certain classes we
  would like to inherit code without creating a subtype even if, from
  the typing point of view, subtyping is still possible. A typical
  example~\cite{LaLonde:1991:SSS:110673.110679} of this are methods for collection classes such as \emph{Sets} and
  \emph{Bags}. Bag implementations can often inherit 
  from Set implementations, and the interfaces of the two collection types are
  similar and type compatible. 
  However, from the logical point-of-view a Bag is \emph{not a
    subtype} of a Set. 

\end{itemize}

Type systems based on structural typing~\cite{cook} can deal with the first
situation well, but not the second. Since structural subtyping
accounts for the types of the methods only, a Bag would be a subtype
of a Set if the two interfaces are type compatible. For dealing with
the second situation nominal subtyping is preferable. With nominal
subtyping an explicit subtyping relation must be signalled by the
programmer. Thus if subtyping is not desired, the idea is that 
programmer can simply {\bf not} declare a subtyping relationship.

While there is no problem in subtyping without subclassing, in the design
of most nominal OO languages subclassing implies subtyping in a
fundamental way. This is because of what we call the
\emph{this-leaking problem}, illustrated by the following
(Java) code:

\begin{lstlisting}[language=Java]
  class A{ int ma(){return Utils.m(this);} }
  class Utils{static int m(A a){..}}
\end{lstlisting}

Method \lstinline{A.ma} passes \lstinline{this} as \lstinline{A} to \Q@Util.m@.
This code is correct, and there is no subtyping/subclassing.  Now, lets add a class \Q@B@

\begin{lstlisting}[language=Java]
  class B extends A{ int mb(){return this.ma();} }  
\end{lstlisting}

%%Class \lstinline{B} does two things at the same time: 
%%1) it {\bf inherits} the method \lstinline{ma} from
%%\lstinline{A}; and 2) it creates a {\bf subtype} of \lstinline{A}.
\noindent We can see an invocation of \lstinline{A.ma} inside
\lstinline{B.mb}, where the self-reference \lstinline{this} is of type \lstinline{B}. 
The execution will eventually call \lstinline{Utils.m} with an
instance of \lstinline{B}. However, \emph{this can be correct only if \lstinline{B} is a subtype of
\lstinline{A}}. 

%If Java was to support a mechanism to allow reuse/inheritance 
%without introducing subtyping, such as:
%
%\begin{lstlisting}[language=Java]
%  class B inherits A{ int mb(){return this.ma();} }
%\end{lstlisting}
%
%\noindent Then an invocation of 
%\lstinline{mb} would be type-unsafe (i.e. it would 
%result in a run-time type error). 
%Note that here the intention of using the imaginary keyword {\bf
%  inherits} is to allow the code from \lstinline{A} to be inherited 
%without \lstinline{B} becoming a subtype of \lstinline{A}. 
%However this breaks type-safety. The problem is that the
%self-reference \lstinline{this} in class \lstinline{B} has 
%type \lstinline{B}. Thus, when \lstinline{this} is passed as an argument to 
%the method \lstinline{Utils.m} as a result of the invocation of
%\lstinline{mb}, it will have a type that is incompatible with the
%expected argument of type \lstinline{A}.  


As a thought experiment, imagine that Java code-reuse (the {\bf extends} keyword) was not introducing subtyping: then an invocation of 
\lstinline{B.mb} would result in a run-time type error.
The problem is that the
self-reference \lstinline{this} in class \lstinline{B} has 
type \lstinline{B}. Thus, when \lstinline{this} is passed as an argument to 
the method \lstinline{Utils.m} as a result of the invocation of
\lstinline{mb}, it will have a type that is incompatible with the
expected argument of type \lstinline{A}.  
Therefore, every OO language with the minimal features exposed in the example (using \lstinline{this},
{\bf extends} and method calls) is forced to accept that subclassing implies
subtyping\footnote{C++ allows to "extends privately", but it is a limitation over
  subtyping visibility, not over subtyping itself.  The
  former example would be accepted even if \lstinline{B} was to
  "privately extends" \lstinline{A}}.
  

What the \emph{this-leaking problem} shows is that adopting a more flexible
nominally typed OO model where subclassing does not imply subtyping is
not trivial: a more substantial change in the language design is
necessary.  In essence we believe that, in languages like Java, classes do too many
things at once. In particular they act both as units of \emph{use} and
\emph{reuse}: classes can be \emph{use}d as types and can be instantiated;
classes can also be subclassed to provide \emph{reuse} of code.
We are aware of at least 3 independently designed research
languages that address the \emph{this-leaking problem}:
\begin{itemize}
\item In {\bf TraitRecordJ (TR)}~\cite{Bettini:2010:ISP:1774088.1774530,BETTINI2013521,Bettini2015282}
each construct has a single responsibility: classes instantiate objects,
interfaces induce types, records express state and traits are reuse units.
\item {\bf Package Templates (PT)}~\cite{KrogdahlMS09,DBLP:journals/taosd/AxelsenSKM12,DBLP:conf/gpce/AxelsenK12}
are an extension of (full) Java where new packages can be ``synthesized'' by mixing
and integrating code templates. As an extension of Java, PT allows but does not require
separation of inheritance and subtyping.
\item {\bf
    DeepFJig}~\cite{deep,servetto2014meta,fjig} is
a module composition language where the main idea is that
nested classes with the same name are recursively composed.
\end{itemize}
This paper aims at showing a simple language design, called \name,
which addresses the \emph{this-leaking problem} and decouples subtyping from inheritance.
Leveraging on traits, in this work we aim to synthesize the best ideas
of those 3 very different designs, while at the same time coming up with a simpler and
improved design for separating subclassing from subtyping.
The keys ideas in \name are to divide between code designed for
\textbf{use} and code designed for \textbf{reuse}, and a novel
approach to state in traits that avoids the complexities introduced by
fields and their initialisation in previous approaches.
In \name there are two separate concepts: classes
and traits~\cite{Traits:ECOOP2003}. Classes are meant for code use, and cannot be used
for reuse. In some sense classes in \name are like final classes in
Java. Traits are meant for code reuse and multiple traits can be
composed to form a class which can then be instantiated. Traits 
cannot be instantiated (or used) directly. Such design allows
subtyping and code reuse to be treated separately, which in turn
brings several benefits in terms of flexibility and code reuse.

We first focus on an example-driven presentation to illustrate how to
improve use and reuse. 
in Section~\ref{sec:formal} we then provide a compact 1 page formalization.
Most of the hard technical aspects of the
semantics have been studied in previous 
work~\cite{Bettini:2010:ISP:1774088.1774530,BETTINI2013521,Bettini2015282,KrogdahlMS09,DBLP:journals/taosd/AxelsenSKM12,DBLP:conf/gpce/AxelsenK12,deep,servetto2014meta,fjig},
and the design of \name synthesises some of these previously studied
concepts.
The language design ideas have been implemented in the {\bf 42 language}, which supports all
the examples we show in the paper, and is available at:\url{http://l42.is} (this site can be accessed without breaking double blind review).

In summary, our contributions are:

%\marco{We need to talk of unanticipated extensions?}
%\bruno{Talk more about typing aspects here. Summarise the important 
%aspects of the design of \name.}

%\bruno{Need to say something about state?}

%The language design of \name is adopted by the 42 programming 
%language, All the examples shown in this paper can be run in 42.
%The compiler for 42 is available at: 

%\url{http://l42.is}



\begin{itemize}
\item We identify the {\bf this-leaking problem} and argue why it
  makes the separation of subclassing and subtyping difficult.
\item We synthesize the key ideas of previous designs that solve the
  this-leaking problem into {\bf a novel and
  simple language design}. This language is the logic core of the language 42, and 
  all the examples in the paper can be encoded as valid 42 programs. 

\item We illustrate how the new design {\bf improves both code use and code
  reuse}.
\item We propose {\bf a clean and elegant approach to handle of state} in a trait based language.

\item We show how to extend our system with nested classes, and how this make us challenge the expression problem.
\item we show the simplicity of our approach by providing a compact 1 page formalization
\item we perform 3 case studies, comparing our work with other approaches, showing that we can express the same examples in a cleaner and more modular manner.
\end{itemize}
\saveSpace
\saveSpace


\saveSpace
\section{The Design of \name: Separating Use and Reuse}\label{sec:separate}
%\saveSpace\saveSpace
%This section presents an overview of \name and illustrates the
%key ideas of its design. In particular we illustrate %how to separate code use and 
%code reuse, and how \name solves the this-leaking problem.

%\subsection{The Design of \name: Decoupling Use from Reuse}
%\saveSpace\saveSpace
\subsection{Classes in \name: a mechanism for code use}
\saveSpace
%\name has a substantially different design from Java-like languages.
Consider the example from Section~\ref{sec:intro} rewritten in \name, introducing classes \Q@Utils@ and \Q@A@:
\saveSpace\saveSpace
\begin{lstlisting}
 A={ method int ma(){ return Utils.m(this); } }
 Utils={ static method int m(A a){ return ..; } }
\end{lstlisting} 
\saveSpace\saveSpace
\noindent Classes in \name use a different declaration style compared
to Java: there is no \lstinline{class} keyword, and an equals sign separates the class name (which must always start with
an uppercase letter) and the class implementation, which is used to specify the
definitions of the class. In our example, in the class declaration
for \lstinline{A}, the name of the class is \lstinline{A} and the code 
literal associated with the class is `\Q@{ method int ma(){return Utils.m(this);}}@' and it contains the method \Q@ma()@.
%We will see next some 
%important differences to Java-like languages in the way 
%classes and code-literals are type-checked, as we shall see next. 
%Nevertheless, for this example, things still work in a similar way to Java. 
In the \name code above, there is no way to add a class 
\Q@B@ reusing the code of \Q@A@: class \Q@A@ (uppercase) is designed for code \emph{use} and not \emph{reuse}.
Indeed, a noticeable difference with Java is the
absence of the \Q@extends@ keyword.
\name classes are roughly equivalent to final classes in Java. This means that there is actually no subclassing.
Thus, unlike the Java code, introducing a subclass
\lstinline{B} is not possible. This may seem like a severe restriction, but
\name has a different mechanism for \emph{code-reuse} that 
is more appropriate when \emph{code-reuse} is intended. 

\saveSpace
\subsection{Traits in \name: a mechanism for code reuse}
\saveSpace
Traits in \name cannot be instantiated and do not introduce new
types. However they provide code reuse.
%So, lets try again encoding the code for the leaking problem, but this
%time aiming at code reuse. 
Trait declarations 
look very much like class declarations, but trait names 
start with a lowercase letter: \MS{even syntactically they can not be used as types.}
\saveSpace\saveSpace
\begin{lstlisting}
 Utils={ static method int m(A a){return ...} }
 ta={ method int ma(){return Utils.m(this);}} //type error
 A=Use ta
\end{lstlisting}
\saveSpace\saveSpace
\noindent Here \lstinline{ta} is a trait intended to replace the
original class \lstinline{A} so that the code of the method
\lstinline{ma} can be reused. Then the class \lstinline{A} 
is created by reusing the code from the trait \Q@ta@, introduced by the keyword 
{\bf \lstinline{Use}}. Note that \use\ expressions cannot contain class names: only trait
names are allowed.
\emph{Referring to a trait is the only way to induce code reuse}.

The crucial point is the call \Q@Utils.m(this)@ inside trait \Q@ta@:
the corresponding call in the Java code is correct since Java guarantees that such occurrence of \Q@this@ will be a subtype of \Q@A@ everywhere it is reused.
In \name we do not commit the same mistake, and the code \Q@Utils.m(this)@ is ill-typed:
the type of \lstinline{this} in
\Q@ta@ has no relationship to the type \lstinline{A}.
The following second attempt would not work either:
\saveSpace\saveSpace
\begin{lstlisting}
 Utils={ static method int m(ta a){return ...}} //syntax error
 ta={ method int ma(){return Utils.m(this);}}
 A=Use ta
\end{lstlisting}
\saveSpace\saveSpace
\Q@ta@ is not a type in the first place, since it is a (lowercase) trait name.
Indeed, trait names can only be used in \use\ expressions, and thus they can not appear in method bodies or type annotations.
In this way, the code of a trait can stay agnostic of its name. This is one of the key design decisions in \name:
traits can be reused in multiple places, and their code can be seen under multiple types.
In \name, \emph{interfaces are the only way to obtain subtyping}. As shown in the code below, interfaces are special kinds of code literals, where all the methods are abstract.
%In \name
%subtyping is the way to reason about commonalities between different types.
Thus, to model the original Java example, we need an interface
capturing the commonalities between \Q@A@ and \Q@B@:
\saveSpace\saveSpace
\begin{lstlisting}
 IA={interface method int ma()} //interface with abstract method
 Utils={static method int m(IA a){return ...} }
 ta={implements IA //This line is the core of the solution
     method int ma(){return Utils.m(this);}}
 A=Use ta
\end{lstlisting}
\saveSpace\saveSpace
This code works: \Q@Utils@ relies on interface \Q@IA@ and the trait \Q@ta@
implements \Q@IA@.
%\Q@ta@ is well typed:
Any class reusing \Q@ta@ will contain the code of \Q@ta@,
including the \Q@implements IA@ subtyping declaration; thus any class reusing \Q@ta@ will be a subtype of \Q@IA@. 
Therefore, while typechecking \Q@Utils.m(this)@ we can assume
\Q@this<:IA@.
 It is also possible to add a class \Q@B@ as follows:
\saveSpace\saveSpace
\begin{lstlisting}
  B=Use ta, { method int mb(){return this.ma();} }
\end{lstlisting}
\saveSpace\saveSpace
This also works.  \Q@B@ reuses the code of \Q@ta@, but has no knowledge of \Q@A@.
Since \Q@B@ reuses \Q@ta@, and \Q@ta@ implements \Q@IA@, \Q@B@ \MS{also} implements \Q@IA@. 

Later, in Section \ref{sec:formal} we will provide the type
system. 
For now notice that the former declaration of \Q@B@ is correct even if
no method called \Q@ma@ is explicitly declared.
DJ and TR would instead require explicitly declaring an abstract \Q@ma@ method:
\saveSpace\saveSpace
\begin{lstlisting}
  B=Use ta, { method int ma() //not required by us
      method int mb(){return this.ma();} }
\end{lstlisting}
\saveSpace\saveSpace
\noindent
In \name, methods are directly accessible from \Q@ta@, exactly as in the Java equivalent
\saveSpace\saveSpace\begin{lstlisting}[language=Java]
  class B extends A{ int mb(){return this.ma();} }  
\end{lstlisting}
\saveSpace\saveSpace
where method \Q@ma@ is imported from \Q@A@.
This concept is natural for a Java programmer, but was not supported
in previous work \cite{BETTINI2013521,deep}. Those works require all
dependencies in code literals to be explicitly declared, so that the
code literal is completely self-contained. However, this results in
many redundant abstract method declarations.

\paragraph{Semantics of Use:}
Albeit alternative semantic models for traits~\cite{lagorio2009featherweight} have been proposed,
we use flattening: \Q@A@ and \Q@B@ are equivalent to the inlined code of all used traits{.}%
\saveSpace\saveSpace\begin{lstlisting}
A=Use ta
B=Use ta, { method int mb(){return this.ma();} }
//equivalent to
A={implements IA method int ma(){return Utils.m(this);}}
B={implements IA
  method int ma(){return Utils.m(this);}
  method int mb(){return this.ma();} } 
 \end{lstlisting}
\saveSpace\saveSpace
\emph{In the resulting code, there is no mention of the trait
 \Q@ta@}. Information about code-reuse/inheritance
  is a private implementation detail of \Q@A@
 and \Q@B@; while subtyping is part of the class interface.
In summary, to leak \Q@this@ in \name, either code reuse is disallowed, or an appropriate interface (\Q@IA@ in this case) \MS{must be} implemented.
We believe the code with \Q@IA@ better transmits programmer intention. Some
readers may instead see requiring \Q@IA@ as a cost of our approach.
Even from this point of view, such cost is counter balanced by 
the very natural and simple support for code reuse, `\Q@This@' type and (in the extensions with nested classes seen later)
family polymorphism.


%To finish this section, Figure \ref{fig:compare} provides a summary of
%the differences between classes and traits. The comparison focus on
%the roles of traits and classes with respect to instantiation,
%reusability and whether the declarations also introduce new types or
%not.





\saveSpace
\section{Improving Use}
\saveSpace
To illustrate how 
\name improves the use of classes, we model a simplified version of
Set and Bag collections first in Java, and then in \name.
The benefit of \name is that we get reuse 
without introducing subtyping between Bags and Sets.
As shown below, this improves the 
use of Bags by eliminating logical errors arising from incorrect
subtyping relations that are allowed in the Java solution. 
\saveSpace
\subsection{Sets and Bags in Java: the need for code reuse without 
subtyping}
\saveSpace
An iconic example on why connecting inheritance/code reuse and
subtpying is problematic is provided by
LaLonde~\cite{LaLonde:1991:SSS:110673.110679}.  A reasonable
implementation for a \Q@Set@ is easy to extend into a \Q@Bag@ by
keeping track of how many times an element occurs.  We just add some state and override a few methods.
For example in Java one could have:
\saveSpace\saveSpace
\begin{lstlisting}
class Set {..//usual hashmap implementation
  private Elem[] hashMap;
  void put(Elem e){..}
  boolean isIn(Elem e){..}}
class Bag extends Set{ ..//for each element in the hash map,
  private int[] countMap;// keep track of how many occurrences are in the bag
  @Override void put(Elem e){..}
  int howManyTimes(Elem e){..}}
\end{lstlisting}
\saveSpace\saveSpace\saveSpace
\noindent Coding \Q@Bag@ in this way avoids a lot of code
duplication, but we induced unintended subtyping! 
Since subclassing implies subtyping, our code breaks the Liskov substitution principle (LSP)~\cite{martin2000design}: not all bags are sets!\footnote{The LSP is often broken in real programs because of the need of inheritance: the LSP allows only refinement not extension. Traits provide extension without breaking the LSP.}
Indeed, the following is allowed{:}%
\saveSpace\saveSpace
\begin{lstlisting}
Set mySet=new Bag(); //OK for the type system but not for LSP
\end{lstlisting}
\saveSpace\saveSpace
This encumbers the programmer:
to avoid conceptual errors that are not captured by the type system, 
they have to \emph{use} \Q@Bag@ very carefully.

\saveSpace\saveSpace\saveSpace\saveSpace
\paragraph{A (broken) attempt to fix the Problem in Java:}
One could \emph{retroactively} fix this problem by introducing \Q@AbstractSetOrBag@
and making both \Q@Bag@ and \Q@Set@ inherit from it:
\saveSpace\saveSpace
\begin{lstlisting}
abstract class AbstractSetOrBag {/*old set code goes here*/}
class Set extends AbstractSetOrBag {} //empty body
class Bag extends AbstractSetOrBag {/*old bag code goes here*/}
...
//AbstractSetOrBag type not designed to be used.
AbstractSetOrBag unexpected=new Bag(); 
\end{lstlisting}
\saveSpace\saveSpace
This looks unnatural, since \Q@Set@ would extend \Q@AbstractSetOrBag@ without adding anything,
and we would be surprised to find a use of the type \Q@AbstractSetOrBag@.
Worst, if we are to constantly apply this mentally, we would introduce a very high number
of abstract classes that are not supposed to be used as types. Those classes would clutter the 
public interface of our classes and the project as a whole.
A \emph{use}able API should provide only the information relevant to the client.
In our example, the information \Q@Set<:AbstractSetOrBag@ would be present in the public interface
of the class \Q@Set@, but such information is not needed to use the class properly!
Moreover, the original problem is not really solved, but only moved 
further away. For example, one day  we may need bags that can only store up to $5$ copies of the same element.
We are now at the starting point again:
\begin{itemize}
\item either we insert \Q@class Bag5 extends Bag@ and we break the LSP; 
\item or we duplicate the code of the \Q@Bag@ implementation with minimal
  adjustments in \\* \Q@class Bag5 extends AbstractSetOrBag@;
\item or we introduce an
\Q@abstract class BagN extends AbstractSetOrBag@ and \\*\Q@class Bag5 extends BagN@
and we modify \Q@Bag@ so that  \Q@class Bag extends BagN@.
Note that this last solution is changing the public interface of the formerly released \Q@Bag@ class, and
this may even break backwards-compatibility (if a client program was using
reflection, for example).
\end{itemize}
\saveSpace\saveSpace
\subsection{Sets and Bags in \name}
\saveSpace\saveSpace\saveSpace\saveSpace
Instead, in \name, if we were to originally declare
\saveSpace\saveSpace\begin{lstlisting}
Set={/*set implementation*/} 
\end{lstlisting}\saveSpace\saveSpace
Then our code would be impossible to reuse in the first place for any user of our library.
We consider this an advantage, since unintended code reuse runs into under-documented behaviour nearly all the time!\footnote{See
``Design and document for inheritance or else prohibit
it''~\cite{Bloch08}: the
self use of public methods is rarely documented, thus is hard to understand the effects of overriding a library method.
}
If the designer of the \Q@Set@ class wishes to make it reusable, they can do it explicitly by providing a set trait{:}%
\saveSpace\saveSpace\begin{lstlisting}
set={/*set implementation*/} 
Set=Use set
\end{lstlisting}\saveSpace\saveSpace
Since \Q@set@ can never be used as a type, there is no reason to give it a fancy-future-aware name like
\Q@AbstractSetOrBag@.
When bag is added, the code will look like%
\saveSpace\saveSpace\begin{lstlisting}
set={/*set implementation*/} 
Set=Use set
Bag= Use set, {/*bag implementation*/}
\end{lstlisting}\saveSpace\saveSpace
or %
\saveSpace\saveSpace\begin{lstlisting}
set={/*set implementation*/} 
Set=Use set
bag=Use set, {/*bag implementation*/}
Bag=Use bag
\end{lstlisting}\saveSpace\saveSpace
Notice how, thanks to flattening, the resulting code for \Q@Bag@ is identical in both versions
and, as shown in Section 2, there is no trace of trait \Q@bag@ at run time. 
Thus if we are the developers of bags, we can temporarily go for the first version.
Then, when for example we need to add \Q@Bag5@ as discussed before,
we can introduce the \Q@bag@ trait without adding new undesired complexity for our old clients.


\saveSpace\saveSpace\section{Improving Reuse}\saveSpace

%\name allows reuse even when subtyping is impossible.
%\name traits do not induce a new (externally visible) type.
%However, locally in a trait, programmers can use the special self-type \Q@This@~\cite{bruce_1994,Saito:2009,ryu16ThisType} in order to denote the 
%type of \Q@this@.
%That is, a program is agnostic to what the \Q@This@ type is, so that it can
%be later assigned to any (or many) classes. 
%The idea is that during flattening, \Q@This@ will be replaced with the actual class name.
%In this way, \name allows reuse even when subtyping is
%impossible. For example for \emph{binary
%  methods}~\cite{bruce96binary} where the method parameter has type \Q@This@. 
%This type of situations is the primary motivator
%for previous work aiming at separating inheritance from subtyping~\cite{cook}.
%Leveraging on the \Q@This@ type, we can also provide self-instantiation (trait methods can create instances of the class using them) and smoothly integrate state and traits: a challenging problem that has limited the flexibility of traits and
%reuse in the past.

%\subsection{Managing State}

%\name improves reuse in many different ways,

To illustrate how \name improves reuse,
we show a novel approach
to smoothly integrating state and traits: a challenging problem that has limited the flexibility of traits and
reuse in the past.
The idea of \MS{flattening} is elegant, and successful in module
composition languages~\cite{ancona2002calculus} and several trait
models~\cite{ducasse2006traits,Bergel2007,BETTINI2013521,fjig}. 
\MS{Flattening is elegant in those two settings
since traits (or modules) only have one kind of member: methods (or functions). In this way 
flattening is defined as simply collecting 
all members from all the used traits (or composed modules), where methods with same name and type signature are summed into a single one.
At most one of those summed methods can have a body, and such body would be propagated in the result.}
 However the research
community is struggling to make it work with object state (constructors
and fields) while achieving the following goals:

\begin{itemize}
%complicated discussions on this point \item keep sum associative and commutative,
\item managing fields in a way that borrows the elegance of summing methods;
\item actually initializing objects, leaving no null fields;
\item making it easy to add new fields;
\item allowing self instantiation: trait methods can create instances of the classes using them.
\end{itemize}
\MS{An in depth discussion on how such goals are 
difficult to achieve and on how they have been challenged
in the existing literature is available in Section 7.3.}
\subsection{State of the art}
We first present the state of the art solution: 
traits have only methods but classes also have fields and constructors.
The idea is that the trait code just uses getter/setters/factories, while leaving
classes to finally define the fields/constructors. That
is, in this state of the art solution, classes have a richer syntax than traits, allowing
declaration for fields and constructors. 

\paragraph{Modelling Points} Consider two simple 
traits dealing with \emph{point} objects with coordinates \lstinline{x} and
\lstinline{y}.
\saveSpace\saveSpace
\begin{lstlisting}
//idealized state of the art trait language, not 42
pointSum= { method int x()  method int y()//getters
  static method This of(int x,int y)//factory method
  method This sum(This that){//sum code
    return This.of(this.x()+that.x(),this.y()+that.y());//self instantiation
  }}
pointMul= { method int x() method int y()//repeating getters
  static method This of(int x,int y)//repeating factory
  method This mul(This that){//multiplication code
    return This.of(this.x()*that.x(),this.y()*that.y());
  }}
\end{lstlisting}
\saveSpace\saveSpace
The first trait provides a \emph{binary method} that 
adds the point object to another point to return a new point. 
The second trait provides multiplication.
\noindent In this code all the operations dealing with state are represented as \emph{abstract methods}.
Notice the abstract \Q@static method This of(..)@ which acts as a factory/constructor
for points. 
As for instance methods, static methods are late bound:  flattening can provide an implementation for them.
Thus, in \name\ they can be abstract, and abstract static methods are similar to the original concept of member functions in the module composition setting~\cite{ancona2002calculus}.
Following the traditional model of traits and classes common in literature~\cite{ducasse2006traits},
we can compose the two traits, by \emph{adding glue-code}
to implement methods \Q@x@, \Q@y@ and \Q@of@.
This approach is verbose but very
powerful, as illustrated by Wang et al.~\cite{wang2016classless}.
\saveSpace\saveSpace
\begin{lstlisting}
//idealized state of the art trait language, not 42
class PointAlgebra=Use pointSum,pointMul, {//not 42 code
    int x   int y//unsatisfactory state of the art solution
    constructor PointAlgebra(int x, int y){ this.x=x   this.y=y }
    method int x(){return x;}//repetitive code
    method int y(){return y;}// in traits terminology, this is all "glue code"
    static method This of(int x, int y){return new PointAlgebra(x,y);}
    }
\end{lstlisting}
\saveSpace\saveSpace
%\bruno{We talk about withers later on. So I think we should consider
 % having withers in this code, so that readers can understand what 
%withers are!}
%\marco{with withers it will look more complicated}

\noindent 

With a slightly different syntax, this approach is available in both Scala and Rust, and they both require glue code.
It has some advantages, but also disadvantages: 

\begin{itemize}

\item {\bf Advantages:} This approach is associative and commutative, even self \MS{instantiation}
  can be allowed if the trait requires a static method
  returning \Q@This@. The class will then implement the methods returning \Q@This@
  by forwarding a call to the constructor.
  
\item {\bf Disadvantages:}
   %The semantics of \Q@Use@/code composition of a model with fields and constructors is necessarily
%   more complex than a model with methods only.
The class needs to handle all the state, even state conceptually
   private to a trait. 
\MSComm{REMOVED: There is no way for a trait to specify a default value for a field.}
 Moreover, writing such obvious code to close
  the state/fixpoint in the class 
   with the constructors and fields and getter/setters and factories is tedious and error prone; such code could be automatically
   generated~\cite{wang2016classless}.
\end{itemize}

\subsection{Our proposed approach to State: Coherent Classes}

In \name there is no need to write down constructors and fields. In fact, in
\name there is not even syntax for those constructs!  The \MS{intuition} is that
any class \MS{whose all abstract methods could be seen as field getters, setters or factories%
%\footnote{\MS{An all-args-constructor is a constructor taking in input as many parameter as the fields and using those parameter values just to initialize the corresponding field.}}
}
%
is a
complete \emph{coherent} class.  In most other languages, a class is
abstract if it has abstract methods.  Instead, we call a class
abstract only when the set of abstract methods are not coherent. That
is, the abstract methods cannot be automatically recognised
as factory, getters or setters. Methods recognised as factory, getters and setters are called
\emph{abstract state operations}.
  
\noindent A definition of coherent
classes is given next, and is formally modelled in Appendix \ref{sec:formal}:
\begin{itemize}
\item A class with no abstract methods is coherent (just like Java
  \Q@Math@, for example). Such classes have no instances and are only useful for calling static methods.
\item A class with a single abstract \Q@static@ method 
returning \Q@This@ and with parameters $T_1\ x_1, \ldots, T_n\ x_n$
is coherent if all the other abstract methods can be seen as \emph{abstract state
operations} over one of $x_1, \ldots, x_n$.
That is:
\begin{itemize}
\item A method $T_i\ x_i$\Q@()@ is interpreted as an abstract state method: a \emph{getter} for $x_i$.
\item A method \Q@void @$x_i$\Q@(@$T_i\ $\Q@ that)@ is a \emph{setter} for $x_i$.
\end{itemize}

Note how the single, abstract static
method acts as a \emph{factory method}.
The signature of the factory method plays an important role, since
abstract state operations are identified by using the names of the
factory method arguments.
\MS{The idea of creating object in a single atomic step by providing a value for all the fields is well explored (for example by primary constructors in Scala) and does not limit the freedom of the programmer to specify personalized initialization strategy.
A static method can freely compute the concrete field  before creating the object. Appendix B.4 discusses usability implications of this pattern.}
\end{itemize}
\noindent
While getters and setters are fundamental operations, it is possible to
support more operations. For example:
\begin{itemize}
\item \Q@method This withX(int that)@
may create a new instance that is like \Q@this@ \MS{except that} field \Q@x@ has now \Q@that@ value.
\MS{Those kinds of methods 
performs functional field updates
and are called \emph{withers}}.
\item \Q@method void update(int x,int y)@
may do two field updates at a time.
\item\Q@method This clone()@ may do a shallow clone of the object.
\end{itemize}

\MS{The concept of `abstract state operations'
is novel, and we think it}
could become a very interesting area of research.
The work by Wang et al.~\cite{wang2016classless} explores a particular
set of such abstract state operations,
\MS{ but we suspect there are more unexplored possible options that could be even more beneficial.}

\paragraph{Points in \name:}
In \name and with our approach to handle the state, 
\lstinline{pointSum} and \lstinline{pointMul} can indeed be directly composed.
This works because the resulting class is coherent.
\saveSpace\saveSpace
\begin{lstlisting}
PointAlgebra= Use pointSum,pointMul //no glue code needed
\end{lstlisting}  
\saveSpace\saveSpace
\noindent
  Note: we declare the methods independently and compose the result
  as we wish. 

  \paragraph{Improved solution} So far the current solution still
  repeats the abstract methods \Q@x@, \Q@y@ and \Q@of@.
  Moreover, in addition to \Q@sum@ and \Q@mul@ we may want many
  operations over points. It is possible to improve reuse
  and not repeat such declaration by abstracting the common
  declaration into a trait \Q@p@: 
\saveSpace\saveSpace
\begin{lstlisting}
p= { method int x() method int y()
  static method This of(int x,int y)
  }
pointSum= Use p, { 
  method This sum(This that){
    return This.of(this.x()+that.x(),this.y()+that.y());
  }}
pointMul= Use p, { 
  method This mul(This that){
    return This.of(this.x()*that.x(),this.y()*that.y());
  }}
pointDiv= ...
PointAlgebra= Use pointSum,pointMul,pointDiv,...
\end{lstlisting}
\saveSpace\saveSpace      
Now the code is fully modularized, that is: each trait defines exactly one method \MS{and contains its abstract dependencies. In this way it can be modularly composed with any code requiring such behaviour and offering or requiring such dependencies.}

\paragraph{Case Study 1:}
In order to evaluate our approach
we performed a case study:
we consider $4$ different operations \Q@Sum@, \Q@Subtraction@, \Q@Multiplication@ and \Q@Division@.
These operations can be combined in $16$ different ways.
We wrote this example in four different styles:
(a) Java7 \emph{($115$ lines)},
(b) Classless Java \emph{($82$ lines)},
(c) Scala \emph{($81$ lines)} and (d) \name \emph{($32$ lines)}.%
\footnote{
Since we want to focus on the actual code, while counting line numbers we \emph{omit} empty lines and lines containing only open/closed
parenthesis/braces.
}
We chose Classless Java~\cite{wang2016classless} since it is a novel approach allowing
Java8 default interface methods to encode traits in Java.
We then chose Java7, that lacks the features needed to encode traits, to show the impact of this feature.
Finally, the comparison with Scala is interesting 
since
it has good support for traits, and using abstract types, it is possible to support the `\Q@This@' type.
Rust is similar to Scala in this regard; we believe we would get similar results by comparing against either Scala or Rust.

\noindent\begin{tabular}{l l l l}
Language       & Lines of code & N. of members & N. classes/traits\\
\hline
\!Java7           &\!\!\!\!   $115=6+5*4+7*6+9*4+11$        & $50$                &      $16$\\
\!Classless Java\!\! &\!   $82=3+3*4+5*6+7*4+9$          & $34$                &      $16$\\
\!Scala          &\!   $81=5+3*4+4*16$  &  $40$                 &    $21 = 16+4+1$\\
\!\name          &\!   $32=4+3*4+1*16$ & $7$                 &      $21 = 16+4+1$\\
\hline
\end{tabular}

\noindent We observed that in Java7 we had to duplicate\footnote{A duplicate body is repetition of identical code (may have different types in its scope/environment). The first occurrence is not counted. } $28$ method bodies across the $16$ classes.
Of these, $11$ method bodies were duplicated because Java does not support multiple inheritance
 and the remaining $17$ bodies had to be duplicated to ensure that the right type
 is returned by the method. Those could be avoided if Java supported
 the `\Q@This@' type.
 On the other hand, the solution in \name was much more compact since we could efficiently
reuse traits (this is why the number of top-level concepts in \name was larger i.e. $21$ due to the
 presence of traits in this solution).
In the detail, Java required $6$ lines for the initial \Q@Point@ class,
$5$ lines for each of the $4$ arithmetic operations, $7$ lines for each of the $6$ combinations
of two different operations, $9$ lines for each of the $4$ combinations of three different 
operations and finally $11$ lines for the class with all four operations.
 The solution in Classless Java was slightly smaller than Java7,
 but was still longer than the \name solution: it still had to redefine the
 sum, sub and other operations in each of the classes. Here the limited
 support for the `\Q@This@' type is to blame, thus Classless Java also has $28$ duplicated method bodies.

Finally, we compare it with a Scala solution.
%Scala has good support for traits, and using abstract types, is possible to support `\Q@This@' type.
There is no need for duplicate method bodies in Scala.
However, for `\Q@This@' instantiation we need to define abstract methods, that will be implemented in the concrete classes.
The Scala solution has the same exact advantages
of our proposed solution, and the declaration
of the trait is about the same size: 
$5$ (point state) $+3*4$ (point operations).
However the glue code (the code needed to compose the traits into usable classes) is quite costly:
$4$ lines for each of the $16$ cases.
In \name a single line for each case is sufficient.

This example is the best-case scenario for \name: where a maximum level of reuse
 is required since we considered the case where all the $16$ permutations needed to be materialized in the code.
%The results in each of the styles are presented below.
In all our case studies, to make a meaningful comparison, we formatted all code in a readable and consistent manner;
on the other hand for space limitations, the code snippets presented in the article
are formatted for compactness.


\subsection{State Extensibility}
Programmers may want to extend points with more state. For example 
they may want to add colors to the points. A first attempt at doing
this would be:
\saveSpace\saveSpace
\begin{lstlisting}
colored= { method Color color() }
CPoint= Use pointSum,colored //Fails: class not coherent
\end{lstlisting}
\saveSpace\saveSpace 
This first attempt does not work: the abstract color method
is not a getter for any of the parameters of 
\Q@ static method This of(int x,int y)@. 

\noindent
A solution is to provide a richer factory:
\saveSpace\saveSpace 
\begin{lstlisting}
CPoint= Use pointSum,colored,{
  static method This of(int x,int y){return This.of(x,y,Color.of(/*red*/));}
  static method This of(int x,int y,Color color) }
\end{lstlisting}
\saveSpace\saveSpace 
\noindent 
where we assume support for overloading based on different numbers of parameters.
This is a reasonable solution, however the method \Q@CPoint.sum@ resets
the color to red: we call the \Q@of(int, int)@ method, that now
delegates to \Q@of(int, int, Color)@ by passing red as the default field
value.  What should be the behaviour in this case?  If our abstract
state supports withers, we can use
\Q@this.withX(newX).withY(newY)@, instead of writing \Q@This.of(...)@, in order to preserve the color from
\Q@this@.  This solution is better but still not satisfactory since the color from \Q@that@ is ignored.

\paragraph{A better design:}
We can design trait \Q@p@ for reuse and extensibility
by adding an abstract \Q@merge(This)@ method as an extensibility hook;
\Q@colored@ can now define color merging.
Using withers we can merge \Q@color@s, or any other kind of state 
following this pattern.%\bruno{worried that withers are not explained enough.}

\saveSpace\saveSpace \begin{lstlisting}
p= { method int x() method int y() //getters
  method This withX(int that) method This withY(int that)//withers
  static method This of(int x,int y)
  method This merge(This that) //new method merge!
  }
pointSum= Use p, { 
  method This sum(This that){
    return this.merge(that).withX(this.x()+that.x()).withY(this.y()+that.y());
  }}
colored= {method Color color()
  method This withColor(Color that)
  method This merge(This that){ //how to merge color handled here
    return this.withColor(this.color().mix(that.color());
  }}
CPoint= /*as before*/
\end{lstlisting} \saveSpace\saveSpace 

\paragraph{Independent Extensibility}
  Of course, quite frequently there can be multiple independent
  extensions~\cite{Zenger-Odersky2005} that need to be composed. Lets suppose that 
  we could have a notion of \Q@flavored@ as well.   
  In order to compose \Q@colored@ with \Q@flavored@ we would
  need to compose their respective merge operations. To this aim \use\ is not sufficient. To combine the implementation of two different implementation of methods, we introduce an operator called \Q@super@, that
 makes a method abstract and
moves the implementation to another name. This is very useful to implement super calls
 and to compose conflicting implementations.
\noindent Consider the simple \Q@flavored@ trait:
\saveSpace\saveSpace \begin{lstlisting}
flavored= {
  method Flavor flavor() //very similar to colored
  method This withFlavor(Flavor that)
  method This merge(This that) //merging flavors handled here
  this.withFlavor(that.flavor())}//inherits "that" flavor
\end{lstlisting}  \saveSpace\saveSpace

\noindent In order to merge \Q@colored@ and \Q@flavored@ we use  \Q@super@ to introduce method selectors \Q@_1merge@ and \Q@_2merge@
to refer to the version of \Q@merge@ as defined in the first/second element of \use.

\saveSpace\saveSpace \begin{lstlisting}
FCPoint= Use
  colored[super merge as _1merge], //this leaves merge as an abstract method, and
  flavored[super merge as _2merge],//copies the bodies into _1merge and _2merge
  pointSum,{
    static method This of(int x,int y){
      return This.of(x,y,Color.of(/*red*/),Flavor.none());}
    static method This of(int x, int y,Color color,Flavor flavor)
    method This merge(This that){//merge conflict is solved 
      return this._1merge(that)._2merge(that);}//by calling the two versions
    }
\end{lstlisting}  \saveSpace\saveSpace

Note how we are relying on the fact that the code literal
 does not need to be complete, 
thus we can just call \Q@_1merge@ and \Q@_2merge@ without
 declaring their abstract signature explicitly.

\MS{
In this last example, when we tried to obtain state extensibility, we refactored the code to introduce  the \Q@merge(This)@ method.
This suggest that we had to
anticipate the need for state extensibility
in order to design our original code.
As illustrated by the following example, we can instead rely on the \Q@super@ operator to inject the \Q@merge(This)@ method when needed.}

\saveSpace\saveSpace \begin{lstlisting}
p=/*as originally designed: no merge*/
pointSum=/*as originally designed: no merge*/
merge={method This merge(This that)}
pointSumMerge=Use merge, pointSum[super sum as _1sum], { 
  method This sum(This that){return this.merge(that)._1sum(that);}}
colored=/*as before, with merge implementation*/
CPoint= /*as before, but using pointSumMerge*/
\end{lstlisting} \saveSpace\saveSpace 


%, as in the following example:
%\begin{lstlisting}
%t:{method bool geq(This x) x.leq(this)   method bool leq(This x) x.qeq(this) }
%C:Use t[restrict geq],{method bool geq(This x){return /*actual geq impl*/}}
%\end{lstlisting}

\paragraph{Case Study 2:}
To understand how easy it is to extend the state in this
way we compare the former code with an equivalent version in
Java.
For this example, in Java we encode \Q@Point@ with the fields but no operations,
\Q@PointSum@ reuses \Q@Point@ adding a functional \Q@sum@ operation,
\Q@CPoint@ reuses \Q@PointSum@ with a \Q@Color@ field
and \Q@FCPoint@ reuses \Q@CPoint@ with a \Q@Flavour@ field.
This second case study represents a \emph{worst case scenario} for \name against Java because we model just a single chain of reuse,
easily supported in plain Java by single inheritance.
Like the previous experiment, we still found that the Java solution was longer ($47$ lines) than that
in \name ($33$ lines). This is caused by the absence of support for the `\Q@This@' type,
where the withers in each of the \Q@CPoint@/\Q@FCpoint@ classes had to be repeated
to make sure that the returned type will be correct (the number of members in Java were $27$ while $24$ ($3$ less)
in \name).

Complex patterns in Java%
\footnote{Combining the ones used in those works~\cite{saito2008essence,torgersen2004expression},
with abstract methods
to allow self instantiation as in~\cite{Zenger-Odersky2005}.}
 allow supporting the `\Q@This@' type and `\Q@This@' type instantiation but they require a lot of set-up code. We experimented with those patterns, but it soon became very clear that the resulting code of this approach would have been even larger; albeit without duplicated code.
Note how the Java code is less modular than the \name code, since \Q@Colored@ and \Q@Flavored@ do not exist
as individual concepts.

We also compare with a solution in Scala, offering the same level of reuse and code modularity of 
the \name solution, but again it is more verbose and requires more members (31): an indication 
that it may be logically heavier too.
We define the main \Q@tPoint@ trait ($8$ lines),
the \Q@tPointSum@ operation ($3$), the two 
\Q@tColored@ and \Q@tFlavored@ traits ($6*2$)
and the \Q@CPoint@ and \Q@CFPoint@ classes ($12+18$).
The major benefit of \name is the reduction
of the amount of glue-code needed to generate 
\Q@CPoint@ and \Q@CFPoint@ ($4+9$).

\noindent The results for the second experiment are presented below.

\noindent \begin{tabular}{l l l l}
\hline
Language       & Lines of code & N. of members & N. of classes or traits\\
\hline
Java           &  $47= 10+9$\ \ \ \ $+$ \ \ \ \  $13+15$         &    $27$             &     $6$\\
Scala          &  $53=$ \ \ $8+3+6*2+12+18$        &    $31$             &         $6$\\
\name          &  $33=$\ \ \ $7+3+5*2+$\ \ $4+9$      &    $24$             &         $6$\\
\hline
\end{tabular}


\section{Family
Polymorphism by Disconnecting Use and Reuse}

A nested class is just another kind of member in a code literal.
In Java and Scala if a subclass declares a nested class with the same
name of a nested of the superclass, the parent declaration is simply hidden.
Approaches supporting FP~\cite{igarashi2005lightweight,IgarashiEtAl08,nystrom2006j,Ernst06,BruceEtAl98,IgarashiViroli07,deep}
consider such definition a form of \emph{overriding},
called \emph{further extension}.
That is, the following Java code is ill typed:
\saveSpace\saveSpace\begin{lstlisting}[language=Java]
abstract class A{static class B{..}   abstract B m();}
class AA extends A{static class B{..}   B m(){..}}
\end{lstlisting}\saveSpace\saveSpace
Invalid overriding: method \Q@AA.m()@ return type is \Q@AA.B@, that is unrelated to \Q@A.B@.
We extend \name with nested classes, so that that
by composing code with \use,
nested classes with the same name are recursively composed.
The corresponding code in \name instead would work, and behave like a FP further extension.
\saveSpace\saveSpace\begin{lstlisting}
a={B={..}   method B m()}
AA= Use a, {B={..}   method B m(){..}}
\end{lstlisting}\saveSpace\saveSpace

For simplicity, we discuss nested classes but not nested traits:
and all traits and code composition expressions are still at top level.
In this way all dependencies are about top level names, allowing the type system 
to consider the class table as a simple map from (nested) type names
(such as \Q@A@ and \Q@A.B.C@)
to their definition.


%\paragraph{Redirect}${}_{}$\\*
%Redirect is another composition operator that
%leverage on nested classes to emulate generics;
%the main idea is that a (fully abstract) nested class can be redirect to
%another one external to the trait/code literal.
%For example a linked list can be implement as
%\begin{lstlisting}
%list:{ Elem:{}
%     Cell:{static method Cell of(Elem e,Cell c) 
%       method Elem e()  method Cell c()
%       }
%   method Elem get(int x) ...
%   ...more methods..
%   }
%ListString:list[redirect Elem to String]
%\end{lstlisting}
%Note how redirect only influence its input, not the rest of the program.

%An expressive form of Redirect can be multiple, that is, can redirect may interdependent classes at the same time.
%We show a graph example, where also we can show how to propagate generics:
%For example
%\begin{lstlisting}
%t:{ method boolean reachable(Node start, Node end)/*implements reachability*/
%     Node:{method ListEdge nodes()}
%     Edge:{method Node in()  method Node out()}
%     ListEdge:list[redirect Elem to Edge]
%     }
%\end{lstlisting}



There are a lots of different forms of rename in
literature~\cite{deep,ancona2002calculus,bracha1992programming}.
Here we introduce a simple variant to rename nested types to other nested types.
For example:
\begin{lstlisting}
t={ method B m() B={ method B mb()} }
D= t[rename B into C]
\end{lstlisting}
\saveSpace
would flatten to:
\saveSpace
\begin{lstlisting}
t={ method B m() B={ method B mb()} }
D={ method C m() C={ method C mb()} }
\end{lstlisting}\saveSpace
The rename only influences its argument.
Since traits do not induce nominal types, we can
consistently change their
internally used names without breaking any code.
The whole L42 offers many other kinds of renames, but we do not need them to show our next example.

%Here we present two variants: one to rename nested types in other nested types
%and one to rename methods; this also allows to rename an (nested) interface method
%and having all the implementations renamed as well.
%Thanks to our division of use and reuse, those renames need to work only inside
%of their input and global rewriting of the whole program is never needed.
%
%Rename modify the method headers, and also all the method calls inside of the input.
%At a first glimpse, this seems to be not always possible since we are considering to be able to apply those
%operators also to non well typed code.
%However, if the expression language is simple enough, it is possible to pre-process the code to
%annotate the expected receiver type on all method calls by doing a purely syntactic analysis
%on a single code literal in isolation. 
%All the expression whose type is guessed to be out of the border of the literal can stay unannotated; they are not going to be renamed anyway.
%
%\begin{lstlisting}
%t:{ I:{interface method int mI() }
%     A:{implements I  method int mI() 42}
%     B:{ method int mB(I i, A a, C c) i.mI()+a.mI()+c.mI()}
%     //mB would be annotated i[I].mI()+a[A].mI()+c.mI()
%     }
%D:t[rename A.mI kI]
%\end{lstlisting}
% Notice how we are sure that \Q@C@ does not implements \Q@I@ since it is invisible from the outside:
% traits does not introduce nominal types!
% 
% We expect the flattened version for \Q@D@ to be
%\begin{lstlisting}
%D:{ I:{interface method int kI() }
%     A:{implements I  method int kI() 42}
%     B:{ method int mB(I i, A a, C c) i.kI()+a.kI()+c.mI()}
%     }
%\end{lstlisting}
%
%Hide can be seen as a variation of rename, where the method/class is renamed to a fresh unguessable name.



\paragraph{Application to the expression problem. Case Study 3:}${}_{}$\\*
The above extensions lets us challenge the expression problem~\cite{wadler1998expression},
with the requirements exposed in~\cite{Zenger-Odersky2005}.
In the expression problem we have data-variants and operations and we can
\emph{extend our solution in both dimensions},
by adding new data-variants and operations.
We aim to \emph{combine independently developed extensions} so
that they can be used jointly.
To be modular, extensions will preserve \emph{type safety}
and allow \emph{separate compilation} (no re-type-checking),
while avoiding \emph{duplication of source code}.

Following closely
the example of Zenger and Odersky~\cite{Zenger-Odersky2005},
we consider a language where the
expressions \Q@Exp@ can
be \Q@Num@ (for number literal),
\Q@Plus@ (for binary plus operator)
and \Q@Neg@ (for unary minus).
\MS{We then proceed to define operations
\Q@show@ to convert them into strings,
 \Q@eval@ to compute their numeric values and 
\Q@double@ to double their containing \Q@Num@s.}
We thus have $3$ classes, $1$ interface,
the definition of the state, and $3$ operations.
We model this
as a table of features, as in~\cite{deep}:
a ($3$ classes + $1$ interface)*($1$ state + $3$ operations)
table composed by $16$ traits.
The features are atomic: they exactly 
declare the state of a class
or define a single operation for a single class.
\name avoids the large amount of abstract declarations
that clutters the solution in~\cite{deep}.
Intuitively, we would like our traits to look like the following:
\begin{lstlisting}
evalPlus= Use plus, {//eval operation for Plus data-variant
  Exp= {interface
    method int eval()}
  Plus= {implements Exp
    method int eval(){
      return this.left().eval()+this.right().eval();}}}
\end{lstlisting}
\Q@evalPlus@ uses the trait \Q@plus@ to import the state (the \Q@left()@ and \Q@right()@ methods)
and defines the \Q@eval()@ method from interface \Q@Exp@.
But, if we were to declare those
explicitly, we would repeat \Q@Exp@, the abstract
declaration of \Q@eval()@ and `\Q@implements Exp@'
for all data-variants.
To avoid this duplication,  we write 
the trait \Q@eval @with a placeholder \Q@T@ nested class, that can then be renamed
into the corresponding data-variant.
Thus, our source code is as follows;
First we declare the $4$ traits to represent the state:
\newcommand\multiCode{\vspace{-5pt}}
\saveSpace
\begin{lstlisting}
exp= { $\ \ \ \ $Exp= {interface} $\ \ \ \ $T= {implements Exp}}
\end{lstlisting}
\multiCode
\begin{lstlisting}
num= Use exp[rename T into Num],{//T is renamed to Num and summed with
  Num= {method int value()  static method Num of(int value)}} // this Num
\end{lstlisting}
\multiCode
\begin{lstlisting}
plus= Use exp[rename T into Plus], {
  Plus= {method Exp left()  method Exp right()
    static method Plus of(Exp left,Exp right)}}
\end{lstlisting}
\multiCode
\begin{lstlisting}
neg= Use exp[rename T into Neg],{
  Neg= {method Exp term()  static method Neg of(Exp term)}}
\end{lstlisting}
Here we define a trait for each data-variant.
Each trait will contain its version of \Q@Exp@
and a specific kind of expression, with its state.
Next, we define the operation \Q@eval@ for all the data-variants.
The former solutions in~\cite{deep}
required repeating the state declaration of the 
data-variant in each operation, while we can just import it.

\begin{lstlisting}
eval= {Exp= {interface $\ \ \ \ $         method int eval()} $\ \ \ \ $     T= {implements Exp}}
\end{lstlisting}
\multiCode
\begin{lstlisting}
evalNum= Use num, eval[rename T into Num],{         //just the implementation
  Num= { method int eval(){return this.value();} }}//of the specific method
\end{lstlisting}
\multiCode
\begin{lstlisting}
evalPlus= Use plus, eval[rename T into Plus], {
  Plus= { method int eval(){ return this.left().eval()+this.right().eval();} }}
\end{lstlisting}
\multiCode
\begin{lstlisting}
evalNeg= Use neg, eval[rename T into Neg], {Neg={ method int eval(){..}}}
\end{lstlisting}

The \Q@show@ operation can be trivially defined
following exactly the same pattern (omitted here for space reasons).
The operation \Q@double@ is is a challenge for some proposed solutions
to the expression problem, as explained by Zhang and Oliveira~\cite{zhang2017evf}.
The \Q@double@ operation is called a \emph{transformation}: an operation from \Q@Exp@ to \Q@Exp@.
Thanks to \name's separation between use and reuse,
together with support
for self-instantiation of nested classes
\Q@double@ does not need any special attention
and can be written just like \Q@eval@ and \Q@show@.
\begin{lstlisting}
double= { $\ \ \ \ $Exp= {interface $\ \ \ \ $ method Exp double()}$\ \ \ \ $ T= {implements Exp} }
\end{lstlisting}
\multiCode
\begin{lstlisting}
doubleNum= Use num, double[rename T in Num],{
  Num= { method Exp double(){return Num.of(this.value()*2);} }}
\end{lstlisting}
\multiCode
\begin{lstlisting}
doublePlus= Use plus,double[rename T in Plus],{
  Plus= { method Exp double(){
    return Plus.of(this.left().double(),this.right().double());}  }}
\end{lstlisting}
\multiCode
\begin{lstlisting}
doubleNeg=....
\end{lstlisting}
Here we define a trait for each data-variant implementing the operation \Q@double()@.
Again, each trait will contain its version of \Q@Exp@ with \Q@double()@
and a specific kind of expression, with the implementation for \Q@double()@
for that specific kind.
%We can express \Q@double@ with this level of simplicity
%thanks to the separation between use and reuse:
%every trait have its own \Q@Exp@,
%and there is no subtyping between those \Q@Exp@, they can not even
%see each other.

Our third case study compares with the results presented
in Scala~\cite{Zenger-Odersky2005}.
The proposed solution is not fully modularized as a table,
so in order to make a more close comparison, we provide an alternative
version where we isolate all the units of behaviour as is done in \name.

\noindent\begin{minipage}{0.60\textwidth}
\begin{tabular}{l |l |l}
&                              lines  &   methods\\
\hline
Original Scala          & $52$     &  $15=12+3$\\
Scala  isolated units   & $78$    &  $15=12+3$\\
Scala  glue-code        & $27$   &     $3$\\
42 traits               & $48$
%=$3+4+6+5 + (4+2*3)*3$
 &    $19=4\times3+7$\\
42 classes              &   $3$    &     $0$\\
\end{tabular}
\end{minipage}
\begin{minipage}{0.40\textwidth}
Scala uses $12=4\times3$ methods plus $3$ extra factory methods (for \Q@double@).
We use $12=4\times3$ methods plus our abstract state: 
$4$ getters and $3$ factories.
\end{minipage}

\noindent As we can see, encoding atomic units in Scala is
more verbose,
but more importantly,
in \name we can just define a class supporting any subset of operations
and data-variants by listing the desired traits:
for example, a solution for \Q@Num@ and \Q@Plus@ (but not \Q@Neg@)
with \Q@eval@ and \Q@double@ would look like this:
%\saveSpace\saveSpace
%\begin{lstlisting}
\Q@Example= Use evalNum,evalPlus,doubleNum,doublePlus@.
%\end{lstlisting}\saveSpace\saveSpace
The composition of all our traits would just requiring listing all
of the relevant behaviour;
reasonably formatted, it could take up to $3$ lines.
On the other hand, the presented solution in Scala requires
$27$ lines of glue code to put the traits together.
This means that a full Scala solution \emph{requiring a single instantiation with all the traits} would be $78+27=105$ lines.
If we were to require more instantiations with a different subset of traits, the glue code would dominate the line count,
and the Scala solution would end up being up to $9$ times heavier than the
\name one (if all $64$ permutations were required).

The line count for \name is very predictable: after defining \Q@exp@ ($3$) and the state traits ($4+6+5$)
for each of the three operations (\Q@eval,show,double@) 
we just needed $4$ lines to declare the operation 
in the interface, and $2$ lines for each of the $3$ data-variants.

Following~\cite{Zenger-Odersky2005}, after \Q@double@ we present an implementation of \Q@equals@.
Their solution involved double dispatch to avoid casting.
To show understandable code, we show a simpler solution 
with a guarded cast (sometime called a typecase).%
\footnote{
The interested reader can find a \name implementation of \Q@equals@ with double dispatch
in the appendix.
}
The idea is that since every data-variant contains
 the same "cast" logic, 
 we can modularize it into an \Q@equals@ trait;
\Q@equals@ in~\cite{Zenger-Odersky2005} is more involved and
 and requires glue code.
\saveSpace\saveSpace
\begin{lstlisting}
equals= {
  Exp= {interface $\ \ \ \ $ method Bool equals(Exp that)}
  T= {implements Exp
    method Bool exactEquals(T that)
    method equals(that){
      if(T instanceof This){return this.exactEquals(that);}else{ return false;}}}
\end{lstlisting}
\multiCode
\begin{lstlisting}
equalsNum= Use num, equals[rename T into Num],{
  Num= {method Bool exactEquals(Num that){
    return this.value().equals(that.value());}  }}
\end{lstlisting}
\multiCode
\begin{lstlisting}
equalsPlus= Use plus, equals[rename T into Plus],{
  Plus= {method Bool exactEquals(Plus that){
    return this.left().equals(that.left()) && this.right().equals(that.right());
  }  }}
\end{lstlisting}
\multiCode
\begin{lstlisting}
equalsNeg= Use neg, equals[rename T into Neg],{
  Neg= {method Bool exactEquals(Neg that){
    return this.term().equals(that.term());}  }}
\end{lstlisting}
\saveSpace\saveSpace

\noindent\!\!
\begin{minipage}{0.46\textwidth}
\begin{tabular}{l |l |l}
&                              lines  &   methods\\
\hline
Original Scala equals       &    $40$   &   $10$\\
Isolated Scala equals       &   $31$    &   $10$\\
Scala equals instance       &    $29$   &    $3$\\
42 trait eq d-dispatch      &   $21$
%=6+5*3
   &    $6$\\
42 class dd instance        &    $22$   &   $11$\\
42 traits eq Cast            &   $13$
%=7+2*3
   &    $6$\\
42 class cast instance      &     $3$   &    $0$\\
\end{tabular}
\end{minipage}\quad\ \ \ \ 
\begin{minipage}{0.5\textwidth}
The Scala code here can be made fully isolated with little
extra syntactic cost. The original Scala eq is $40$ lines and
contains a part of the glue code mixed inside.
The isolated version is $31$ lines and to merge all the operations together in Scala, it
takes $29$ lines of glue code. Note that this
is mostly the same glue code from before ($27$ lines), that
needs to be manually adapted.
\end{minipage}

In \name we are
more compact than Scala both when using the double dispatch ($21+22$ vs. $31+29$)
or the guarded cast ($13+3$ vs. $31+29$).
To instantiate the double dispatch 
version in \name we need $22$ lines of glue code.
We could remove such glue code using 
features from the full 42 language, but here we stick to only the features presented in this paper.
The interesting point is that the nature of our needed glue code 
is different with respect to the Scala glue code:
Scala requires lots of trait multiple inheritance declarations to explicitly merge
nested traits with the same name, while in \name we mostly need 
to add the negative cases for the double dispatch (such as
\Q@Sum={method Bool equalToNum(Num that){return false;}}@).


%\paragraph{Transformations (and queries)}${}_{}$\\*
%
%The expression problem presented up to now is the traditional
% challenge proposed by~\cite{wadler1998expression};
%this has been criticized to not really address the fundamental issues
%since if many transformations have to be defined,
%it is hard to share common code between them.
%For example, we shown the \Q@double@ transformation.
%An equivalent transformation \Q@triple@ (tripling the
%values of the literals) should be easy to define by
%reusing all the traversal code and just redefining the
%case of \Q@Num@.
%When modularity is not a concern, this
%can be obtained in Java using the visitor pattern,
%defining a \Q@CloneVisitor@  and then overriding 
%only the relevant cases.
%Consider the Java code below:
%\begin{lstlisting}
%class CloneVisitor{
%  Exp visit(Num n){return new Num(n.value);}
%  Exp visit(Plus p){return new Plus(p.left.accept(this),p.right.accept(this);}
%  }
%class Double extends CloneVisitor{
%  Exp visit(Lit l){return new Lit(l.inner+1);}
%  }
%\end{lstlisting}
%In \name we can obtain the same kind of code reuse, without the need of introducing 
%the concepts related to the Visitor Pattern.
%With redirect, rename and restrict we can have the general operator propagator:
%\begin{lstlisting}
%operation:Use lit, sum, {//for sum and lit, easy to extends as before
%  Arg:{}
%  Exp:{interface method Exp op(Arg x)}
%  Lit:{
%    method Exp op(Arg x){return this}
%    }
%  Sum:{
%    method Exp op(Arg x){
%      return Sum.of(this.left().op(x),this.right().op(x))
%      }
%    }
%\end{lstlisting}
%
%Now, to have \Q@addN@ we can do the following.
%
%\begin{lstlisting}
%opAddn:Use operation
%  [redirect Arg to Int]
%  [rename Exp.op(x) to addN(x)]
%  [restrict Lit.op(x)], {
%  Lit:{
%    method Exp addN(Int x){ return Lit.of(inner())+x}
%    }
%  }
%\end{lstlisting}  



%\paragraph{Full power of redirect}${}_{}$\\*
%An expressive form of Redirect can be multiple, that is, can redirect may interdependent classes at the same time.
%We show an example where a specific kind of \Q@Service@ can produce a \Q@Report@, and 
%\Q@Report@s can be combined together.
%The goal is to execute a list of such services and produce a collated report.
%This example also show how to propagate generics:
%
%\begin{lstlisting}
%Service:{interface method Void performService()}
%serviceCombinator:{
%  S:{implements Service method R report()  }
%  
%  R:{method R combine(R that)   class method R empty() }
%  
%  ListS:list[redirect Elem to S]
%  
%  class method R doAll(ListS ss){//here we use extended java like syntax
%    R r=R.empty()
%    for(S s in ss){
%      s.performService();
%      r=r.combine(s.report())
%      }
%    return r;
%  }
%}
%PaintingService:serviceCombinator[redirect S to PaintingService]
%PaintingService:{... method PaintingReport report()..}
%PaintingReport:{..}
%\end{lstlisting}
%
%The flattened version of \Q@PaintingService@ would look like:
%\begin{lstlisting}
%PaintingService:{
%  ListS:/*the expansion of list[redirect Elem to PaintingService]*/
%  
%  class method PaintingReport doAll(ListS ss){
%    PaintingReport r=PaintingReport.empty()
%    for(PaintingService s in ss){
%      s.performService();
%      r=r.combine(s.report())
%      }
%    return r;
%  }
%}
%\end{lstlisting}
%Where you can note how redirect figured out \Q@R=PaintingReport@ by comparing the structural shape of
%classes \Q@PaintingService@ and \Q@S@.
%
%To encode the former generic code in java you need to write
%the following headeche inducing interfaces for RService and Report.
%and require that the services you want to serve implement those.
%\begin{lstlisting}
%interface Service{ void performService();}
%interface Report<R extends Report<R>>{R combine(R that);}
%interface RService<R extends Report<R>> extends Service{ R report();}
%\end{lstlisting}
%Note how we still can not encode the method \Q@empty@.

\saveSpace\saveSpace\section{Formalization}\label{sec:formal}
\saveSpace\saveSpace

Here we show a simple formalization for the language we presented so far.
We also model nested classes, but in order to avoid uninteresting complexities, we assume that
all the type names are fully qualified from the top level, so the examples shown before should be
written like \Q@This.Exp@, \Q@This.Sum@ and so on.
In a real language, a simple pre-processor may take care of this step.

In most languages, when implementing an interface, the programmer may avoid repeating abstract methods
they do not wish to implement.
In that same spirit, in our simplified formalization we consider source code containing
all the methods imported from interfaces. In a real language, a normalization process
may hide this abstraction\footnote{
In the full 42 language scoping is indeed supported by an initial de-sugaring, and a normalization phase takes care of importing methods from interfaces.
}.
We also consider a binary operator sum \Q@+@ instead of the variable arity operator \Q@use@.

Figure 1 contains the complete formalization for the 
compilation process and the type system for \name.
It starts with the syntax, then
we show the compilation process and the typing rules.




\begin{figure}
%NEW FORMALISATION below
% Syntax
% D::=TD|CD
% TE::=t:E Trait Decl Expr
% CE::=C:E Class Decl
% TD::=t:L
% CD::=C:L
% E::= L| t| E+E | E[rename T.m1->m2]|E[rename T1->T2]|E[redirect T1->T2]
% L::= {interface? implements Ts Ms}//all L are like LC in 42
% T::=C|C.T // .T is a shortcut for This.T
% M::= static? method T m(T1 x1..Tn xn) e? | CD
% e::= x| e.m(es) | T.m(es)

\begin{bnf}
\prodFull\mID{\mt\mid\mC}{class or trait name}\\
\prodFull\mDE{\mID\terminalCode{=}\mE}{Meta-declaration}\\
\prodFull\mD{\mID\terminalCode{=}\mL}{Declaration}\\
\prodFull\mE{\mL \mid \mt \mid \mE\,\terminalCode{+}\mE
\mid \ldots
%\mid \mE\terminalCode{[rename}\ \mT\terminalCode{.}\mm_1\ \terminalCode{to}\ \mm_2\terminalCode{]}
}{Code Expression}\\
%\prodNextLine{
% \mid
%\mE\terminalCode{[rename}\ \mT_1\ \terminalCode{in}\ \mT_2\terminalCode{]} \mid
%\mE\terminalCode{[redirect}\ \mT_1\ \terminalCode{to}\ \mT_2\terminalCode{]}}{Code Expression}\\
\prodFull\mL{
\oC \Opt{\terminalCode{interface}}\ \terminalCode{implements} \overline\mT\ \overline\mM\ \cC}{Code Literal}\\
\prodFull\mT{\mC \mid \mC\terminalCode{.}\mT}{Type}\\
\prodFull\mM{\Opt{\terminalCode{static}}\ \terminalCode{method}\ \mT\ \mm\oR\overline{\mT\,\mx}\cR \Opt\me \mid \mC\terminalCode{=}\mL}{Member}\\

\prodFull\me{\mx \mid \me\terminalCode{.}\mm\oR\overline\me\cR \mid \mT\terminalCode{.}\mm\oR\overline\me\cR}{Expression}\\

\prodFull{v_{{\smallDs}}}{\mT\terminalCode{.}\mm\oR\overline{v_{{\smallDs}}}\cR
\text{  where }\mm \text{ is abstract in }\overline\mD(\mT)
}{value}\\
\prodFull{\ctx_{\smallDs}}{[]\mid
\ctx\terminalCode{.}\mm\oR\overline\me\cR
\mid
v_{{\smallDs}}\terminalCode{.}\mm\oR\overline{v_{{\smallDs}}},\ctx,\overline\me\cR
\mid
\mT\terminalCode{.}\mm\oR\overline{v_{{\smallDs}}},\ctx,\overline\me\cR
}{evaluation ctx}\\
\prodFull{\ctx_c}{[]\mid\ctx\,\terminalCode+\mE | \mL\,\terminalCode+\ctx |\ldots}{compilation ctx}\\
\prodFull{\ctx}{[]\mid\ctx\,\terminalCode+\mE | \mE\,\terminalCode+\ctx |\ldots}{ctx}\\
\end{bnf}\\
\\
\newcommand{\pushLeft}{\!\!\!\!\!\!\!\!\!\!\!\!}
$\begin{array}{l}

%       D.E -->^+_CDs L  CDs|-CD1:OK .. CDs|-CDn:OK       CDs=CD1..CDn
% (top)---------------------------------------------------------------    D.E not of form L
%      CD1..CDn CDs' D Ds -> CDs CDs' D[with E=L] Ds

\pushLeft\inferrule[(top)]{
  \mE_0 \xrightarrow[\smallDs]{} \mE_1
  \\
  \forall \mD\in\overline\mD,
  \overline\mD\vdash\mD:\text{OK}
  }{ 
    \overline\mD \ \overline{\mD'}\ \mID\terminalCode{=}\mE_0 \ \overline{\mDE}
    \rightarrow 
    \overline\mD\ \overline{\mD'}\ \mID\terminalCode{=}\mE_1\ \overline{\mDE}
  } %{\overline\mD=\mD_1..\mD_n }
\quad\quad
%
%     ------------------------
%      t -->_CDs CDs(t)

 \inferrule[(look-up)]{
    \ 
  }{ 
    \mt \xrightarrow[\smallDs]{}\ \overline\mD(\mt)
  }
\quad

\inferrule[(ctx-c)]{
    \mE_0 \xrightarrow[\smallDs]{}\ \mE_1
  }{ 
     {\ctx}_c[\mE_0] \xrightarrow[\smallDs]{}\ {\ctx}_c[\mE_1]
  }
\quad
%
%      --------------------------      L = L1+L2
%      L1+L2  -->_CDs L

\inferrule[(sum)]{
    \
  }{ 
     \mL_1\,\terminalCode{+}\mL_2 \xrightarrow[\smallDs]{}\ \mL
  }\mL = \mL_1+\mL_2
\\[5ex]
%  C;CDs,C=L |- L[This=C] :OK
% ----------------------------------------- coherent(L)
%  CDs|-C=L : OK

\pushLeft\inferrule[(CD-OK)]{
    \mC;\overline\mD,\mC\terminalCode{=}\mL_1\vdash \mL_1\ :\text{OK}
  }{ 
     \overline\mD \vdash \mC\terminalCode{=}\mL_0\ :\text{OK}
  }
\begin{array}{l}
\mL_1=\mL_0[\terminalCode{This}=\mC]\\
\text{coherent}(\mC,\mL_1)
\end{array}
\quad\quad 

%    This;CDs,This=L |- L :OK
%----------------------------------------
%    CDs|-t=L : OK

\inferrule[(TD-OK)]{
    \terminalCode{This};\overline\mD,\terminalCode{This=}\mL\vdash \mL\ :\text{OK}
  }{ 
     \overline\mD \vdash \mt\terminalCode{=}\mL\ :\text{OK}
  }
\\[5ex] 

%  forall i in 1..k T;CDs|-Mi:Ok
%--------------------------------------------------  L={interface? implements T1..Tn M1..Mk} 
%  T;CDs|-L:Ok                                         forall i in 1..n 	CDs(Ti).interface?=interface
%                                                             forall i in 1..n and m in 	dom(CDs(Ti)), m in dom(L)

\pushLeft\inferrule[(L-OK)]{
    \forall\mM\in\overline\mM,
  \mT;\overline\mD\vdash\mM:\text{OK}
  }{ 
     \mT;\overline\mD \vdash  \oC \Opt{\terminalCode{interface}}\ \terminalCode{implements} \overline\mT \ \overline\mM \cC \ :\text{OK}\\
  } 
%\begin{array}{l} 
%  \mL=\oC \Opt{\terminalCode{interface}}\ \terminalCode{implements} \overline\mT \ \overline\mM \cC \\
%  \forall \mT\in\overline\mT \text{and } m \in \dom(\mD(\mT)), \mm \in \dom(\mL)
%   \end{array}
\quad
\inferrule[(Nested-OK)]{
    \mT\terminalCode{.}\mC;\overline\mD\vdash \mL\ :\text{OK}
  }{ 
     \mT;\overline\mD \vdash \mC\terminalCode{=}\mL\ :\text{OK}
  }

\\[5ex] 

%  if e?=e then CDs; G|-e:T                         
%----------------------------------------------------------   forall T in CDs(C).Ts, if m in dom(CDs(Ti)) then
%   T;CDs|-static? T0 m(T1 x1..Tn xn) e?              static? T0 m(T1 x1..Tn xn) in CDs(Ti)
%                                                                        if static?=static then G=x1:T1 .. xn:Tn
%                                                                        else G=this:T,x1:T1 .. xn:Tn

\pushLeft\inferrule[(Method-OK)]{
    \text{if}\ \Opt\me=\me\ \text{then}\ \overline\mD; \mG\vdash\me:\mT
  }{ 
     \mT;\overline\mD \vdash \Opt{\terminalCode{static}}\ \terminalCode{method}\ \mT_0\ \mm\oR\mT_1\,\mx_1\ldots\mT_n\,\mx_n\cR \Opt\me
  } \begin{array}{l} 
  \text{if}\ \Opt{\terminalCode{static}}=\terminalCode{static}\\
  \quad \text{then}\ \mG=\mx_1:\mT_1\ .. \ \mx_n:\mT_n\ \\
  \quad\text{else}\ \mG=\terminalCode{this}:\mT,\mx_1:\mT_1\ ..\ \mx_n:\mT_n
  \\
%removed, now is well formedness
%  \forall \mT \in \text{implementsOf}(\overline\mD(\mC)),\ \text{if}\ \mm \in \dom(\overline\mD(\mT))\ \text{then} \\
%  \quad\Opt{\terminalCode{static}}\ \terminalCode{method}\ \mT_0\ \mm\oR\overline{\mT\,\mx}\cR \in \overline\mD(\mT) \\
   \end{array}
\\[5ex] 



\pushLeft\inferrule[(subsumption)]{
%  \begin{array}{l}
    \overline\mD; \mG\vdash\me: \mT_1  \\\\
    \overline\mD\vdash\mT_1 \leq \mT_2
%  \end{array}
  }{ 
     \overline\mD; \mG\vdash\me: \mT_2
  }
\quad \inferrule[(static-method-call)]{
    \overline\mD;\mG\vdash\me_1:\mT_1\ \ldots \ \overline\mD;\mG\vdash\me_n:\mT_n
  }{ 
    \overline\mD;\mG\vdash \mT_0.\mm\oR\me_1\ \ldots \ \me_n\cR:\mT
  } \terminalCode{static method}\ \mT\ \mm\oR\mT_1\,\mx_1\ldots\mT_n\,\mx_n\cR \text{\_} \in \overline\mD\oR\mT_0 \cR
\\[5ex] 

%    CDs;G|-e0:T0 .. CDs;G|-en:Tn
%---------------------------------------------    static T m(T1 x1..Tn xn) _ in CDs(T0)
%  CDs;G|-e0.m(e1..en):T

\pushLeft\inferrule[(x)]{
    \
  }{ 
    \overline\mD; \mG\vdash\mx: \mG\oR\mx\cR
  }
\quad
\inferrule[(method-call)]{
    \overline\mD;\mG\vdash\me_0:\mT_0\ \ldots \ \overline\mD;\mG\vdash\me_n:\mT_n
  }{ 
    \overline\mD;\mG\vdash \me_0.\mm\oR\me_1\ \ldots \ \me_n\cR:\mT
  } \terminalCode{method}\ \mT\ \mm\oR\mT_1\,\mx_1\ldots\mT_n\,\mx_n\cR \text{\_} \in \overline\mD\oR\mT_0 \cR

\\[5ex] 
\pushLeft\inferrule[(ctxv)]{\me_0\xrightarrow[\smallDs]{}\me_1}{
 \ctx_{\smallDs}[\me_0]\xrightarrow[\smallDs]{} \ctx_{\smallDs}[\me_1]
 }

\quad
\inferrule[(s-m)]{{}_{}}{
 \mT\terminalCode{.}\mm\oR\overline\vds\cR\xrightarrow[\smallDs]{}
 \text{meth}(\overline\mD(\mT,\mm),\overline\vds)
}
\quad
\inferrule[(m)]{{}_{}}{
 \vds\terminalCode{.}\mm\oR\overline\vds\cR\xrightarrow[\smallDs]{}
 \text{meth}(\overline\mD(\mT,\mm),\vds\,\overline\vds)
}\vds=\mT\terminalCode{.}\mm'\oR\_\cR\\
\end{array}
$\\
\caption{Formalization}
\end{figure}

\subsection{Syntax}

%In the following section, we present a simplified grammar of \name. 
We use $\mt$ and $\mC$ to represent lower case trait and upper case class identifiers respectively.
To declare a trait \mTD\ or a class \mCD, we can use either a code literal \mL\ or a trait
expression $\mE$. Note how in $\mE$\ you can refer to a trait by name.
In full 42, we support various operators including the ones presented before,
 but here we only show the single sum operator \Q@+@.
This operation is a generalization to the case of nested classes of the simplest and most elegant
trait composition operator~\cite{ducasse2006traits}.
Code literals \mL\ can be marked as interfaces. We use `?' to represent optional terms.
Note that the interface keyword is inside the curly brackets,
so an upper case name associated with an interface literal is a class-interface, while a lowercase one is a trait-interface.
Then we have a set of implemented interfaces and a set of member
declarations, they can be methods or nested classes.
The members of a code literal are a set, thus their order is immaterial.
If a code literal implements zero interfaces, the concrete syntax omits the \Q@implements@ keyword.

Methods \mMD~can be instance methods or \Q@static@ methods. 
A static method in \name is similar to a \Q@static@ method in Java but can be abstract.
This is very useful in the context of code composition.
To denote a method as abstract, instead of an optional keyword we just omit the implementation \me.

Finally, expressions $\me$ are just variables, method calls or static method calls.
The ugliness of having two different kinds of method calls is an artefact of our simplifications.
In the full 42 language, type names are a kind of expression whose type helps to model metaclasses.
Our concept of abstract state implies we have no \Q@new@ expressions, and
the values are just calls to abstract static methods.
Thus values are parametric on the shape of the specific programs $\overline\mD$.

We then show the evaluation context, the compilation context and full
context.

\subsection{Well-formedness}

The whole program, that is $\overline\mDE$, is well formed if
all the traits and classes at top level have unique names. The special class name
\Q@This@ is not one of those,
and the subtype relations are consistent:
this means that the implementation of interfaces is not circular,
and that $\forall\ \_\terminalCode{=}\ctx[\mL]\in\overline\mDE, \mathit{consistentSubtype}(\overline\mDE,\terminalCode{This=}\mL;\mL)$

\noindent That is, every literal declares
all the methods that are declared in its super interfaces, and declare them with the same exact type.\footnote{The full 42 language allows for covariant return types, like in Java.}


\noindent\textbf{Define }$\mathit{consistentSubtype}(\overline\mDE;\mL)$\\
$\begin{array}{l}
\!\!\!\bullet\ \mathit{consistentSubtype}(
  \overline\mDE,
  \oC
  \Opt{\terminalCode{interface}}
  \terminalCode{implements}\overline\mT\ 
  \overline\mM
  \cC
  )\quad\text{where}\\

\quad\quad
\forall \mT\in\overline\mT,\overline\mDE(\mT)=\oC\terminalCode{interface}\,\_\cC
 \text{,\footnotemark}
\\
\quad\quad \forall\ \_\terminalCode{=}\mL\in  \overline\mM, 
\mathit{consistentSubtype}(\overline\mDE;\mL) 

\text{ and }
\\
\quad\quad 
\forall \mm, \mT\in\overline\mT,
\text{if}\,\overline\mDE(\mT,\mm)=\terminalCode{method}\ \mT_0 \mm\oR
\overline{\mT\,\mx}
%\mT'_1\,\mx'_1\ldots\mT'_k\,\mx'_k
\cR
\,\text{then}\,
\terminalCode{method}\ \mT_0 \mm\oR
%\mT_1\,\mx_1\ldots\mT_n\,\mx_n
\overline{\mT\,\mx}
\cR\Opt\me
\in\overline\mM

%\mT_0=\mT'_0, \overline{\mT\,\mx}=\overline{\mT\,\mx}'
%\mT_0\ldots\mT_n=\mT'_0\ldots\mT'_k


\\
\end{array}$
\footnotetext{That is, in this simplified version 
in order to implements an interface nested in a different top level name, such interface can not be generated using a trait expression. This limitation is lifted in the full language.}
${}_{}$\\*
${}_{}$\\*
\noindent A code literal \mL\ is well formed if:
\begin{itemize}
\item all method parameters have unique names and the special parameter name \Q@this@ is not declared
 in the parameter list,
\item all methods in a code literal have unique names,
\item all nested classes have unique names, and no nested class is called \Q@This@,
\item all used variables are in scope, and
\item all methods in an interface are abstract, and there are no interface static methods.
\end{itemize}

\saveSpace
\subsection{Compilation process}
\saveSpace
Usually the compilation process is not modelled, but here it is the \textbf{most interesting part}.
Here we explain in the detail the flattening process and how and when compilation errors may arise.
It is composed by rules \Rulename{top},\ \Rulename{look-up},\ \Rulename{ctx-c} and \Rulename{sum}.
If we were to formally model more composition operators, each operator would have its own  rule.

Rule \Rulename{top}
compiles the leftmost top level (trait or class) declaration that needs to be compiled.
In order to do so,
it identifies the subset of the program $\overline\mD$ that can already be typed (second premise).
Then the expression is executed under the control of such compiled program (first premise).
According to rule \Rulename{look-up}, all the traits inside the expression need to
be compiled, that is $\forall\mt. \mE=\ctx[\mt], \mt\in\dom(\overline\mD)$.
If a large enough $\overline\mD$ cannot be typed, this would cause a compilation error
at this stage.
Rule \Rulename{look-up}
replaces a trait name $\mt$ with the corresponding literal $\mL$.
Thanks to the fact that $\overline\mD$ is all well typed, we know that $\mL$ is well typed too.
Rule \Rulename{ctx-c}
uses the compilation context to decide what step to apply next.
It enforces a deterministic left-right call by value\footnote{
In the flattening process, values are code literals $\mL$.} reduction;
thus the leftmost invalid sum that is performed will be the one providing the compilation error.

Keeping in mind the order of members in a literal is immaterial, rule \Rulename{sum}
applies the operator:

\noindent\textbf{Define }$\mL_1+\mL_2, \ \overline{\mM}+\overline{\mM},\ \mM+\mM$\\
$\begin{array}{l}
\!\!\!\bullet\ \mL_1+\mL_2 =\mL_3\quad\text{where}\\
\quad\quad \mL_1= \oC \Opt{\terminalCode{interface}}\ \terminalCode{implements} \overline\mT_1\ \overline\mM_1\ \overline\mM_0\cC\\
\quad\quad \mL_2= \oC \Opt{\terminalCode{interface}}\ \terminalCode{implements} \overline\mT_2\ \overline\mM_2\ \overline\mM_0\cC\\
\quad\quad \mL_3= \oC \Opt{\terminalCode{interface}}\ \terminalCode{implements} \overline\mT_1,\overline\mT_2\ \overline\mM_1,\overline\mM_2\ (\overline\mM_0+\overline\mM_0')\cC\\
\quad\quad \dom(\overline\mM_1)
%\pitchfork
\,\text{disjoint}\,
 \dom(\overline\mM_2) \text{ and } \dom(\overline\mM_0)\ =\ \dom(\overline\mM_0')\\

\!\!\!\bullet\ \mM_1..\mM_n+\mM'_1+\mM'_n\ = \ \mM_1+\mM'_1..\mM_n+\mM'_n\\

\!\!\!\bullet\ \mC\terminalCode{=}\mL_1+\mC\terminalCode{=}\mL_2\ = \ \mC\terminalCode{=}\mL_3\quad if \mL_1+\mL_2\\

\!\!\!\bullet\ \mM_1+\mM_2=\mM_2+\mM_1\\

\!\!\!\bullet\ \Opt{\terminalCode{static}}\ \terminalCode{method}\ \mT_0\ \mm\oR\overline{\mT\,\mx}\cR \ + \ \Opt{\terminalCode{static}}\ \terminalCode{method}\ \mT_0\ \mm\oR\overline{\mT\,\mx}\cR \Opt\me = \Opt{\terminalCode{static}}\ \terminalCode{method}\ \mT_0\ \mm\oR\overline{\mT\,\mx}\cR \Opt\me\\
\end{array}$

Sum composes the content of the arguments
by taking the union of their members and the union of their \Q@implements@.
Members with the same name are recursively composed.
There are three cases where the composition is impossible.
\begin{itemize}
\item MethodClash: two methods with the same name are composed,
but either their headers have different types or they are both implemented.
\item ClassClash: a class is composed with an interface.%
\footnote{
The full language offers some relaxation here, so that for example an empty class can be seen as an empty interface during composition.
}
\item ImplementsClash:
The result code would not be well formed.
For example
\begin{lstlisting}
t1={
  A:{interface method Void a()}
  B:{}
   }
t2={
  A:{interface}
  B:{implements A}
  }
\end{lstlisting}
Naively, \Q@t1+t2@ should result in a class \Q@B@ implementing \Q@A@ with method \Q@a()@,
but \Q@B@ would not offer such method \Q@a()@.%
\footnote{While in \name it could be possible to try to patch class \Q@B@, for example adding a
abstract method \Q@a()@, we chose to give an error in this case, since in the full 42 language
such patch would 
be able to turn private nested classes
into abstract (private) ones.}

ImplementsClash can happen only when composing nested interfaces. Note that while the first two kind of errors are obtained directly by the definition of 
$\mL_1+\mL_2$, ImplementsClash is obtained since injecting the resulting 
$\mL$ in the program would make it ill-formed by 
$\mathit{consistentSubtype}(\overline\mDE,\mL)$.
\end{itemize}

\subsection{Typing}
Typing is composed by rules \Rulename{cd-ok}, \Rulename{td-ok},
\Rulename{l-ok},
\Rulename{nested-ok} and \Rulename{method-ok},
followed by expression typing rules
\Rulename{subsumption}, \Rulename{method-call}, \Rulename{x} and \Rulename{static-method-call}.

Rules \Rulename{cd-ok} and \Rulename{td-ok}
are interesting: a top level class is typed by replacing all occurrences of the name `\Q@This@' with the class name $C$,
and is required to be coherent.
On the other side, a top level trait is typed by temporary adding to the typed program a mapping for
\Q@This@.

\noindent\textbf{Define }$\text{coherent}(\mT,\mL)$\\
$\begin{array}{l}
\!\!\!\bullet\ \text{coherent}(\mT,
\oC \Opt{\terminalCode{interface}}\ \terminalCode{implements} \overline\mT\ \overline\mM\cC
)\quad\text{where}\\

\quad\quad \forall \mC\terminalCode{=}\mL'\in\overline\mM \text{coherent}(\mT\terminalCode{.}\mC,\mL')\\
\quad\quad \text{either }
\Opt{\terminalCode{interface}}=\terminalCode{interface}\\
\quad\quad\quad \text{or } 
\forall\ 
\terminalCode{method}\ \mT'\ \mm\oR\overline{\mT\,\mx}\cR \in\overline\mM,\ 
\text{state}(\text{factory}(\mT,\overline\mM),\terminalCode{method}\ \mT'\ \mm\oR\overline{\mT\,\mx}\cR)
\end{array}$

\noindent A Library is \emph{coherent} if 
all the nested classes are coherent,
and either the Library is an interface,
there are no static methods, or all the static methods
are a valid \emph{state} method of the candidate \emph{factory}.
Note, by asking for
$\terminalCode{method}\ \mT'\ \mm\oR\overline{\mT\,\mx}\cR \in\overline\mM$
we select only the abstract methods.

\noindent\textbf{Define }$\text{factory}(\mT,\overline\mM)$\\
$\begin{array}{l}

\!\!\!\bullet\ \text{factory}(\mT,\mM_1\ldots\mM_n)=\mM_i=\terminalCode{static method}\ \mT\, \mm
\oR
\_
\cR

\quad\text{where}\\
\quad\quad \forall j\neq i.\ \mM_j=
\text{not of the form}\ \terminalCode{static method}\ \_\, \_
\oR
\_
\cR
\end{array}$

\noindent The factory is the only static abstract  method, and
its return type is the nominal type of our class.

\noindent\textbf{Define }$\text{state}(\mM,\mM')$\\
$\begin{array}{l}


\!\!\!\bullet\ \text{state}(
\terminalCode{static}\ \terminalCode{method}\ \mT\ \mm\oR\mT_1\,\mx_1\ldots\mT_n\,\mx_n\cR,
\terminalCode{method}\ \mT_i\ \mx_i\oR\cR
)\\

%\!\!\!\bullet\ \text{state}(
%\terminalCode{static}\ \terminalCode{method}\ \mT\ \mm\oR\mT_1\,\mx_1\ldots\mT_n\,\mx_n\cR,
%\terminalCode{method}\ \terminalCode{Void} \mx_i\oR\mT_i\,\terminalCode{that}\cR
%)\\

\!\!\!\bullet\ \text{state}(
\terminalCode{static}\ \terminalCode{method}\ \mT\ \mm\oR\mT_1\,\mx_1\ldots\mT_n\,\mx_n\cR,
\terminalCode{method}\ \mT\ \terminalCode{with}\mx_i\oR\mT_i\,\terminalCode{that}\cR
)\\

\end{array}$

\noindent A non static method is part of the \emph{abstract state} if 
it is a valid getter or wither. This simple formalism without imperative features do not offer setters.


Rule \Rulename{Nested-OK} helps to accumulate the type of \Q@this@ so that rule \Rulename{Method-OK}
can use it.
Rule \Rulename{L-OK} is so simple since all the checks
related to correctly implementing interfaces are delegated to the well formedness criteria.
The other rules are straightforward and standard.

\subsection{Formal properties}
In addition to conventional soundness of the expression reduction,
\name ensures soundness of the compilation process itself.
A similar property was called meta-level-soundness in~\cite{servetto2014meta}; here we can obtain the same result in
a much simpler setting.
We denote $\mathit{wrong}(\overline\mD,\mE)$ to be the count of $\mL$ such that
$\mE=\ctx[\mL]\ \text{and not}\ \overline\mD\vdash\mL:\text{OK}$.

\begin{Theorem}[Compilation Soundness]

if $\mE_0 \xrightarrow[\smallDs]{} \mE_1$
then $\mathit{wrong}(\overline\mD,\mE_0)\geq\mathit{wrong}(\overline\mD,\mE_1)$.
\end{Theorem}
This can be proved by cases on the applied reduction rule:
\begin{itemize}
\item
\Rulename{look-up} preserve the number of wrong literals,
since $t \in \overline\mD$ and $\overline\mD$ are well typed by \Rulename{top} second preconditions.
\item \Rulename{sum} either preserves or reduces the number of
wrong literals:
the core of the proof is to show the sum of two well typed literals produces a well typed one.
A code literal is well typed (\Rulename{l-ok}) if all the method bodies are correct.
This holds since those same method bodies
are well typed in a strictly poorer environment with respect to the one used to type the result.
This is because every member in the result
is structurally a subtype of
the corresponding member in the argument.
Note that by well formedness, if \Rulename{sum}
is applied, the result still respects 
$\mathit{consistentSubtype}$.
\end{itemize}
\noindent 
Compilation Soundness has two important corollaries:
\begin{itemize}
\item If a class is declared without literals,
and the flattening is successful then \mC\ is well-typed: there is no need of further checking.
\item On the other side, if a class is declared by using literals $\mL_1\ldots\mL_n$, and after successful flattening $\mC = \mL$ can not be type-checked,
then the issue was originally present in one of $\mL_1\ldots\mL_n$.
This also means that as an optimization strategy
 we may remember what method bodies come from traits and what method bodies come from code literals, in order to type-check only the latter.

If the result can not be type-checked, does not means
that is intrinsically ill-typed: it may happen that a 
referred type is declared \emph{after} the current class. 
As we see in the next section, we leverage on this 
to allow recursive types.
 \end{itemize}






\subsection{Advantages of our compilation process}


Our typing discipline is very simple from a formal perspective,  
and is what distinguishes our approach from a simple minded code composition macro~\cite{bawden1999quasiquotation}
or a rigid module composition~\cite{ancona2002calculus}. 
It is built on two core ideas:

\paragraph{1: traits are \textbf{well-typed} before being reused.}
 For example in

\saveSpace\begin{lstlisting}
t={method int m() 2 
   method int n() this.m()+1}
\end{lstlisting}\saveSpace

\noindent \Q@t@ is well typed since \Q@m()@ is declared inside of \Q@t@, while

\saveSpace\begin{lstlisting}
t1={method int n() this.m()+1} 
\end{lstlisting}\saveSpace
\noindent would be ill-typed.

\paragraph{2: code literals are \textbf{not required to be well-typed} before flattening.}${}_{}$\\*
A code literal $\mL$ in a declaration $\mD$
is must be well formed and respect
$\mathit{consistentSubtype}$. However 
it is not type-checked before flattening,
and only the result is expected to be well-typed.
This example using the trait \Q@t@ is correct:

\saveSpace\begin{lstlisting}
C= Use t, {method int k() this.n()+this.m()}
\end{lstlisting}\saveSpace

\noindent The code literal
\Q@{method int k() this.n()+this.m()}@, 
is not well typed,
since \Q@n@, \Q@m@ are not locally defined. However in 
\name the result of the flattening is well-typed.
This is not the case in many similar works in literature~\cite{deep,Bettini2015282,Bergel2007} where the
literals have to be \emph{self contained}. In this case we would have been forced to
declare abstract methods \Q@n@ and \Q@m@, even if \Q@t@ already 
provides such methods.

This relaxation allows multiple declarations to be flattened one at the time, without typing them individually, and then to type them all together.
In this way, we support recursive types%
\footnote{
OO languages leverage on recursive types most of the times:
for example \Q@String@ may offer a \Q@Int size()@
method, and \Q@Int@ may offer a \Q@String toString()@ method.
This means that typing classes 
\Q@String@ and \Q@Int@ in isolation one at a time is not possible.}
between multiple $\mC$\Q@=@$\mE$ \textbf{without
the need of predicting the resulting shape}%
\footnote{This is fundamental in the full language where arbitrary code can be run at compile time, making impossible to predict the resulting shape.}.

As seen in \Rulename{top}, our compilation process
proceeds in a top-down fashion, flattening one declaration at a time,
and declarations need type-checking
only where their type is first needed,
that is, when they are required to type a trait $\mt$ used in an expression $\mE$.
That is, in \name typing and flattening are interleaved. We assume our compilation process to stop as soon as 
an error arises. 
For example
\saveSpace\begin{lstlisting}
ta:{method int ma() 2}
tc:{method int mc(A a,B b) b.mb(a)}
A: Use ta
B:{method int mb(A a) a.ma()+1}
C: Use tc, {method int hello() 1}
\end{lstlisting}\saveSpace
In this scenario, since we go top down, we first need to generate \Q@A@.
To generate \Q@A@, we need to use \Q@ta@ (but we do not need
\Q@tc@, in rule \Rulename{top} $\overline\mD=$\Q@ta@ and $\overline\mD'=$\Q@tc@).
At this moment, \Q@tc@ cannot be compiled/checked alone:
information about \Q@A@ and \Q@B@ is needed.
In order to modularly ensure well-typedness,
we require only \Q@ta@ to be well typed at this stage. If \Q@ta@ was not well-typed
a type error could be generated at this stage.

Now, we need to generate \Q@C@, and we need to ensure well-typedness of \Q@tc@.
Now \Q@A@ is already well typed (since generated by \use\ \Q@ta@, with no \mL),
and \Q@B@ can be typed. Finally \Q@tc@ can be typed and used.
If \Rulename{sum} could not be performed (for example it \Q@tc@ had a method \Q@hello@ too)
a composition error could be generated at this stage.
On the opposite side, if \Q@B@ and \Q@C@ were swapped, as in
\saveSpace\begin{lstlisting}
C: Use tc, {method int hello() 1}
B:{method int mb(A a) a.ma()+1}
\end{lstlisting}\saveSpace
\noindent
we would be unable to type \Q@tc@, since we need to know the type of \Q@A@ and \Q@B@.
A type error would be generated, on the lines of ``flattening of \Q@C@
requires \Q@tc@, \Q@tc@ requires \Q@B@ that is defined later''.

%In this example, a more expressive compilation/precompilation process 
%could compute a dependency graph and, if possible, reorganize the list,

\paragraph{The cost: what expressive power we lose}${}_{}$\\*
For simplicity, that declarations are always provided in the right
order, if such order exists.
An example of a ``morally correct'' program where no right order exists is the following:
\saveSpace\begin{lstlisting}
t={ int mt(A a) a.ma()}
A=Use t {int ma() 1}
\end{lstlisting}\saveSpace

We just wrote an involved program where the correctness of trait \Q@t@ depends of 
\Q@A@, that is in turn generated using trait \Q@t@.
We believe any typing allowing those programs would be fragile with respect to code evolution,
and could make human understanding the code-reuse process much harder/involved.

%Rewriting our example in Java may help to show how involved it is.
%\saveSpace\begin{lstlisting}
%class T{ int mt(A a){return a.ma();}
%class A extends T {int ma() {return 1;}}
%\end{lstlisting}\saveSpace

In sharp contrast with
many other approaches (TR, PT, DeepFJig and in some sense even Java, C\# and Scala)
we chose to not support this kind of involved programs.
In a system without inference for method types,
if the result of composition operators depends only on the
structural shape of their input (as for \use)
it is indeed possible to optimistically compute the resulting structural shape of the classes
and use it to type involved examples like the former.

TR, PT, DeepFJig, Java, C\# and Scala
accept a great complexity in order to \textbf{predict the structural shape} of the resulting code before doing the actual code reuse/adaptation.
Those approaches logically divide the program in groups of mutually dependent classes, where each group may depend on a number of other groups.
This form a direct acyclic graph of groups.
To type a group, all depended groups are typed, then
the signature/structural shape of all
the classes of the group is extracted.
Finally, with the information of the depended groups and the one extracted
from the current group, it is possible to type-check the implementation of each class in the group.
%Following this model, it is reasonable to assume that flattening happens group by group, before extracting the class signatures.



%\paragraph{In \name, typechecking before compiling would be redundant}${}_{}$\\*
%In the world of strongly typed languages we are tempted to
%first check that all can go well, and then perform the flattening.
%This would however be overcomplicated without any observable difference:
%Indeed, in the \Q@A,B,C@ example above there is no difference
%between
%\begin{itemize}
%\item  (1) First check \Q@B@ and produce \Q@B@ code (that also contains \Q@B@ structural shape),
%  (2) then use \Q@B@ shape to check \Q@C@ and produce \Q@C@ code;\ 
%or a more involved
%\item  (1) First check \Q@B@ and discover just \Q@B@ structural shape as result of the checking,
%  (2) then use \Q@B@ shape to check \Q@C@.
%  (3) Finally produce both \Q@B@ and \Q@C@ code.
%\end{itemize}
%
%
%This may seems a dangerous relaxation at first, but also Java has the same behaviour:
%\saveSpace\begin{lstlisting}[language=Java]
%  class A{ int ma() {return 2;}  int n(){return this.ma()+1;} }
%  class B extends A{ int mb(){return this.ma();} }
%\end{lstlisting}\saveSpace
%\noindent in \Q@B@ we can call \lstinline{this.ma()} even if in the curly braces there is no declaration for \Q@ma()@.
%
%



\noindent 
In the world of strongly typed languages we are tempted to
first check that all can go well, and then perform the flattening. However, that we can reuse code only by naming traits; but our point of relaxation is {\bf only} the code literal: in no way an error can ``move around'' and be duplicated during the compilation process.
Our approach allows for safe libraries of traits and classes to be typechecked once and then deployed and reused by multiple clients: no type error will emerge from library code.
%However, we do not force the programmer to write self-contained code where all the abstract method definition are explicitly declared.


\saveSpace
\subsection{Expression reduction}
\saveSpace
Reduction rules are incredibly simple and standard.
A great advantage of our compilation model is that expressions are executed on
a simple fully flattened program, 
where all the composition operators have been removed.
From the point of view of expression reduction, \name is a simple language of 
interfaces and final classes, where nested classes gives structure to the code but have no special semantics.
The reduction of expressions is composed by rules
\Rulename{ctx-v},\Rulename{s-m} and \Rulename{m}.
The only interesting point is the auxiliary function meth:


\noindent\textbf{Define }$\text{meth}(\mM,\overline\vds)$

$\begin{array}{l}

\!\!\!\bullet\text{meth}(\terminalCode{static method}\ \mT\ \mm\oR\mT_1\, \mx_1\ldots\mT_n\,\mx_n\cR\me,\vds_1\ldots\vds_n)=\me[\mx_1=\vds_1\ldots\me_n=\vds_n]
\\

\!\!\!\bullet\text{meth}(\terminalCode{method}\ \mT\ \mm\oR\mT_1\, \mx_1\ldots\mT_n\,\mx_n\cR\me,\vds_0\ldots\vds_n)=\me[\terminalCode{this}=\vds_0,\mx_1=\vds_1\ldots\me_n=\vds_n]
\\

\!\!\!\bullet\text{meth}(\terminalCode{method}\ \mT_i\ \mx_i\oR\cR,\mT\terminalCode{.}\mm\oR\vds_1\ldots\vds_n\cR)=\vds_i\\
\quad \quad\text{where}\ \ \overline\mD(\mT,\mm) =
\terminalCode{static method}
\ \mT\,\mm\oR\mT_1\,\mx_1\ldots\mT_n\,\mx_n\cR
\\

\!\!\!\bullet\text{meth}(\terminalCode{method}\ \mT\ \terminalCode{with}\mx_i\oR\mT_i\,\terminalCode{that}\cR,\mT\terminalCode{.}\mm\oR\vds_1\ldots\vds_i\ldots\vds_n\cR,
\vds
)=
\mT\terminalCode{.}\mm\oR\vds_1\ldots\vds\ldots\vds_n\cR
\\
\quad \quad\text{where}\ \ \overline\mD(\mT,\mm) =
\terminalCode{static method}
\ \mT\,\mm\oR\mT_1\,\mx_1\ldots\mT_n\,\mx_n\cR
\end{array}$

\noindent 
Here we take care of reading bodies and preparing for
execution.
The first case is about static methods,
the second is about instance methods.
The third and fourth cases are more interesting, since they take care of
the abstract state:
the third case takes care of getters and the fourth takes care of withers.
In our formalization we are not modelling state mutation, so there is 
no case for setters.

For space reasons, we omit the proof of conventional soundness for the
reduction. It is unsurprising, since the flattened calculus is like a
simplified version of Featherweight Java~\cite{igarashi2001featherweight}.
\saveSpace\saveSpace
\section{Related Work}
\saveSpace\saveSpace
Literature on code reuse is too vast to let us do justice of it in a few pages.
Our work is inspired by traits~\cite{ducasse2006traits}, which in turn
are inspired by module composition languages~\cite{ancona2002calculus}.

We claim that our presented solution to the expression problem is the most natural in literature to date.
While a similar syntax can be achieved with the scandinavian style~\cite{ernst2004expression}, their dependent type system makes reasoning quite complex, and indeed more recent solutions have accepted a more involved syntax in order to have a much simpler type system~\cite{igarashi2005lightweight}.
Challenging the expression problem, our close contented is DeepFJig~\cite{deep}: all our gain over their model is based on our relaxation over abstract signatures.


\saveSpace
\subsection{Separating Inheritance and Subtyping}
We are aware of at least 3 independently designed research languages 
that address the this-leaking problem: TraitRecordJ~(TR)\cite{Bettini:2010:ISP:1774088.1774530,BETTINI2013521,Bettini2015282}, Package Templates~(PT)\cite{KrogdahlMS09,DBLP:journals/taosd/AxelsenSKM12,DBLP:conf/gpce/AxelsenK12}, DeepFJig~\cite{deep,servetto2014meta,fjig}.
Levering on \emph{traits}, in this work we aim to synthesize
the best ideas of those very different designs, while at the same time 
coming up with a simpler and improved design for separating
subclassing from subtyping, which also addresses various limitations of those
3 particular language designs.
The following compares 
various aspects of the language designs;
we underline 3 properties where one approach shines the most, and 3 properties where one approach is more lacking.
\begin{itemize}
\item {\bf A simple uniform syntax for code literals}
DeepFJig is best in this sense, since TR has separate syntax for class literals, trait literals and record literals.
PT on the other hand is built on top of full Java, thus has a very
involved syntax.
\name leverages on DeepFJig's approach but,
\emph{thanks to our novel representation of state}, \name also offers a much simpler and uniform syntax than
all other approaches: everything is just a method.
\item 
{\bf Reusable code cannot be ``used'', that is instantiated or used as a type.}
This happens in TR and in PT, but not in DeepFJig. To allow reusable code to be directly 
usable, in DeepFJig
classes introduce nominal types in an unnatural way: the type of
\Q@this@ is only \Q@This@ (sometimes called \Q@<>@) and not the
nominal type of its class. 
That is in DeepFJig 
`\Q@A:{ method A m()this}@' is not well typed. This is because
`\Q@B: Use A@' flattens to `\Q@B:{ method A m()this}@', which is clearly not well typed.
Looking to this example is clear why we need reusable code to be agnostic of its name.
Then, either reusable code has no name (as in TR, PT and \name)
or all code is reusable and usable, and all code needs to be awkwardly agnostic of its name, as in DeepFJig.

\item 
{\bf Requiring abstract signatures is a left over of module composition mindset.}
TR and DeepFJig comes from a tradition of functional module composition, where 
modules are typed in isolation under an environment, and then the composition is performed.
As we show in this work, this ends up requiring verbose repetition of abstract signatures,
which (for highly modularized code) may end up constituting most of the program.
Simple Java (and thus PT, since it is a Java extension) shows us a better way:
the meaning of names can be understood from the reuse context.
The typing strategy of PT offers the same advantages of our typing model, 
but is more involved and indirect. This may be caused by the
heavy task of integrating with full Java.
Recent work based on TR is trying to address this issue too~\cite{damiani2017unified}.
\item {\bf Composition algebra.}
The idea of using composition operators over atomic values as in an arithmetic expression is very powerful,
and makes it easy to extend languages with more operators. DeepFJig and TR embrace this idea, while PT takes the traditional Java/C++ approach of using enhanced class/package declaration syntax.
The typing strategy of PT also seems to be connected with this
decision, so it would be hard to move their approach in a composition
algebra setting.
\item {\bf Complete ontological separation between use and reuse}
While all 3 works allow separating inheritance and subtyping only TR properly enforces 
separation between use (classes and interfaces) and reuse (traits).
This is because in DeepFJig all classes are both units of use and reuse (however, subtyping is not induced).
PT imports all the complexity of Java, so although is possible to separate use and reuse, the model have powerful but non-obvious implications where (conventional Java) \Q@extends@ and PT are used together.
\item {\bf Naming the self type, even if there is none yet.}
Both DeepFJig and PT allow a class to refer to its name, albeit this is
less obvious in PT since both a package and a class have to be introduced to express it.
This allows encoding binary methods, expressing patterns like withers or fluent setters and to instantiate instances of the (future) class(es)  using the reused code.
\end{itemize}

\subsection{Family
Polymorphism by disconnecting Use and Reuse}

Researchers with a strong grasp on FP: Family
Polymorphism know that support for FP strictly includes
support for `\Q@This@' type and self instantiation, since the
class is a member of its own family.
~\cite{deep} contains an in depth comparison between various FP approaches.
The main point is synthesized by the following example, allowed when FP 
is obtained while keeping Use and Reuse connected:
\begin{lstlisting}
A = { B = {...}                Int m(B b){...} }
A2 = Use A, { B = {... Int y;} Int m(B b){ ... b.y ... }}
A a=new A2();//this line is considered sound by most FP
a.m(new A.B());//this line is unsound, but is hard to prevent
\end{lstlisting}
In those approaches, \Q@A2@ is a subtype of \Q@A@,
so code like \Q@A a=new A2();@ is accepted.
However, this implies that innocent looking code like
\Q@a.m(new A.B());@ is now unsound: method \Q@A2.m@ will
try to access field \Q@A2.B.y@
that is not present in class \Q@A.B@.
To track those calls, and enforce the \emph{family} of the receiver and the argument is the same, complex dependent type systems are often used~\cite{ernst2004expression,Zenger-Odersky2005}.

Our approach logically avoid all this complexity: by disconnecting use and reuse we outlaw \Q@A a=new A2()@.
While in \name this is also reducing the expressing power a little,
in the full 42 language, as well as in DeepFJig, the operator \textbf{redirect} allows to write code that is parametric on families of data types.

\subsection{State and traits}
The original trait model has no self construction 
and purposely avoided any connection between state and traits.
It was applied to a dynamically typed language, so
it was not clear if the author intended of modelling `\Q@This@' type.


The idea of abstract state operations emerged from Classless
Java~\cite{wang2016classless}. This approach offers a clean solution to handle state
in a trait composition setting.
Note how abstract state operations are different from just hiding fields under getter and setters: 
in our model the programmer simply never has to declare what is the state of the class, not even what information is stored in fields.
The state is computed by the system as an overall result of the whole code composition process.

In the literature there have been many attempts to add state in traits/module composition languages:
\begin{itemize}  
\item An early approach is to have {\bf no constructors}: all the fields start with {\bf null} or a default specified value.
  Fields are just like another kind of (abstract) member, and two fields
  with identical types can be merged by sum/use; \Q@new C()@ can be used for all classes, and \Q@init@ methods may be called later, as in
  \Q@Point p=new Point(); p.init(10,30)@.
  
  To its credit, this simple approach is commutative and associative and does not disrupt elegance of summing methods.
  However, objects are created "broken" and the user is trusted with fixing them.
  While it is easy to add fields, the load of initializing them is on the user; moreover
    all the objects are intrinsically mutable, so this model is unfriendly
    to a functional programming style.
\item {\bf Constructors compose fields}:
In this approach (used by \cite{fjig}) the fields are declared but not initialized, and
a canonical constructor (as in FJ) taking a value for each field and just initializing such field
is automatically generated in the resulting class.
It is easy to add fields, however this model is associative but not commutative: composition order influences field order, and thus the constructor signature.
Self construction is not possible 
since the signature of the constructors changes during composition.

\item {\bf Constructors can be composed if they offer the same exact parameters}:
In this approach (used by DeepFJig) traits declare fields and constructors.
The constructor initializes the fields but can do any other computation.
Traits whose constructors have the same signature can be composed.
The composed constructor will execute both constructor bodies in order.
This approach is designed to allow self construction.
It is also associative and mostly commutative: composition order only influences execution order of side effects during construction.
However constructor composition requires identical constructor signatures: this
hampers reuse, and if a field is added, its initial value needs to be
somehow synthesized from the constructor parameters.

\end{itemize}

\subsection{Tablular comparision of many approaches}
\begin{minipage}[t]{0.30\textwidth}
In this table we show if some constructs support certain features:
Direct instantiation (as in \Q@new C()@),
Self instantiation (as in \Q@new This()@),
Is this construct a `Unit of use'?, a `Unit of reuse'?,
Does using this construct introduce a type? and is the induced type the type of \Q@this@?,
support for binary methods,
does inheritance of this construct induce subtype?,
is the code of this construct required to be well-typed before being inherited /imported in a new context?
is it required to be well-typed before being composed with other code?
\end{minipage}
%second column
\begin{minipage}[t]{0.6\textwidth}
\newcommand{\YY}{\textbf{Y}}
\begin{center}
\begin{tabular}{c|c|c|c|c|c|c|c|c|c|c}
&\Rotated{direct instantation}
&\Rotated{self instantiation}
&\Rotated{unit of use}
&\Rotated{unit of reuse}
&\Rotated{introduce type}
&\Rotated{induced type is this type}
&\Rotated{binary methods}
&\Rotated{{${}_{}$\!inheritance induce subtype\!\!\!}}
&\Rotated{{${}_{}$\!well-typed before imported\!\!\!}}
&\Rotated{{${}_{}$\!well-typed before composed\!\!\!}} 
\\
\hline
java/scala class&\YY &X&\YY &\YY &\YY &\YY &X&\YY &\YY &X\\
java8 interface &X&X&X&\YY &\YY &\YY       &X&\YY &\YY &X\\
scala trait        &X&X&X&\YY &\YY &\YY    &-&\YY &\YY&X\\
original trait     &X&X&X&\YY &-&-         &-&X&-&-\\
TR  &X&X&X&\YY &X&-                        &X&X&\YY &\YY \\
\name trait        &X&\YY &X&\YY &X&-      &\YY &X&\YY &X\\
\name class        &\YY &\YY &\YY &X&\YY   &\YY &\YY &-&\YY &-\\
module composition
                      &-&-&\YY &\YY &-&-   &-&-&\YY &\YY \\
deepFJig class &\YY &\YY &\YY &\YY &\YY &X &\YY &X&\YY &\YY \\
package template
                      &X&\YY &X&\YY &X&-   &-&X&\YY &X\\
${}_{}$\\
\end{tabular}
\end{center}
\end{minipage}

\noindent \textbf{Y} and X means yes and no, and we use ``-'' where the question is not really applicable to the current approach. For example the original trait model was untyped, so typing questions makes no sense here.

\subsection{ThisType with Subclassing implying Subtyping}
With the exception of those mentioned 3 lines of work, to the best of our understading
other famous work in literature, like~\cite{odersky2008programming,nystrom2006j}
do not completely break the relation between inheritance and subtyping, but only prevent subtyping where 
it would be unsound.
Recent work on {\bf ThisType} \cite{Saito:2009,ryu16ThisType}
also continues on this line.
In those works, ``subtyping by subclassing'' is preserved, which means
that those designs are more suitable to retain the programming model
of mainstream OOP languages and backwards compatibility. The design 
of \name (and 42) is a more radical departure of mainstream OOP, with
the hope to improve both the mechanisms for use and reuse in OOP.



\section{Conclusions, extensions and practical applications}

In this paper we explained a simple model to 
radically decouple inheritance/code reuse and subtyping.
One important point is that our decoupling does not
makes the language more complex:
% since
%interfaces (subtyping without subclassing)
%exists in both Java and C\#.
we \textbf{replace the concept} of abstract classes with
the concept of traits, while keeping the concepts of
interfaces and final classes.
Concrete non final classes are simply not needed in our model.

The model presented here is easy to extend.
More composition operators can be added in addition to \use.
In particular variants of the sophisticated operators of DJ are
included in the full 42 language.
 Indeed we can add any operator respecting following criteria:

\begin{itemize}
\item The operator does not need to be total, but if it fails it needs to provide an error that will be reported to the programmer.
\item When the operator takes in input only traits (they are going to be well typed code), if a result is produced,
 such result is also well typed.
\item When the operator takes in input also code literals, if a non-well typed result is produced,
the type error must be traced back to code in one of those not-yet typed code literals.
 \end{itemize}
 

 
 Our simplified model represents the conceptual core of  42: a novel full blown programming language,
using the ideas presented in this paper to obtain reliable and understandable metaprogramming.
Formalization (in progress) for full 42 can be found at
\url{http://}\footnote{Omitted for anonymous review}. 
%\verb@urlOmittedForDoubleBlindReview@.
%\verb@github.com/ElvisResearchGroup/L42/tree/master/Main/formal@.
42 extends our model allowing
flattening to execute arbitrary computations.
In such model we do not need an explicit notion of traits: they are encoded as methods returning a code literal.
42 also has features less related to code composition, like
  a strong type system supporting aliasing mutability and circularity control,
   checked exceptions, and errors (unchecked exceptions) with strong-exception-safety.

\begin{comment}
42 do not have a finite set of composition operators; they can be
added using the built in support for native method calls. They can
be dynamically checked to verify that they are well behaved
according to our predicate, or they can be trusted to achieve
efficiency.
\end{comment}



%% Acknowledgments
%\begin{acks}                            %% acks environment is optional
%                                        %% contents suppressed with 'anonymous'
%  %% Commands \grantsponsor{<sponsorID>}{<name>}{<url>} and
%  %% \grantnum[<url>]{<sponsorID>}{<number>} should be used to
%  %% acknowledge financial support and will be used by metadata
%  %% extraction tools.
%  This material is based upon work supported by the
%  \grantsponsor{GS100000001}{National Science
%    Foundation}{http://dx.doi.org/10.13039/100000001} under Grant
%  No.~\grantnum{GS100000001}{nnnnnnn} and Grant
%  No.~\grantnum{GS100000001}{mmmmmmm}.  Any opinions, findings, and
%  conclusions or recommendations expressed in this material are those
%  of the author and do not necessarily reflect the views of the
%  National Science Foundation.
%\end{acks}


%% Bibliography
%\bibliography{bibfile}
\bibliography{main}

%% Appendix
\appendix
\appendix
\section{Proof} 
\label{s:proof}

\begin{theorem}[Sound Validation]
	if $c:\Kw{Cap};\emptyset\vdash \e: \T$ and
	$c\mapsto\Kw{Cap}\{\_\}|\e\rightarrow^+ \sigma|\ctx_v[r_l]$, then
	either $valid(\sigma,l)$ or $\mathit{trusted}(\ctx_v,r_l)$.
\end{theorem}

We believe this property captures very precisely our statement in Section~\ref{s:validation}.

It is hard to prove Sound Validation directly,
so we first define a stronger property,
called \emph{Stronger Sound Validation} and
show that it is preserved during reduction by means of conventional 
Progress and Subject Reduction (Progress is one of our assumption,
while Subject Reduction relies heavily on SubjectReductionBase).
That is,
Progress+Subject Reduction $\Rightarrow$ Stronger Sound Validation,
\\*and Stronger Sound Validation $\Rightarrow$ Sound Validation.

\subsection{Stronger Sound Validation $\Rightarrow$ Sound Validation}

Stronger Sound Validation depends on 
$\mathit{wellEncapsulated}$, $\mathit{monitored}$
and $OK$:

\noindent\textbf{Define} $\mathit{wellEncapsulated}(\sigma,\e,l_0)$:\\*
\indent$\forall l \in \mathit{erog}(\sigma,l_0), \text{not}\ \mathit{mutatable}(l,\sigma,\e)$

\noindent The main idea is that an object is well encapsulated if its encapsulated state is safe from
modification. 

\noindent\textbf{Define} $\mathit{monitored}(\e,l)$:\\*
\indent$\e=\ctx_v[M(l;\e_1;\e_2)]$ and either $\e_1=l$ or $l$ is not inside $\e_1$.

\noindent An object is monitored if the execution
is currently inside of a monitor for that object, and
the monitored expression $\e_1$ does not
contains $l$ as a \emph{proper} subexpression.

A monitored object is associated with an expression that can not observe it, but may 
reference its internal representation directly.
In this way, we can safely modify its representation before checking for the invariant.

The idea is that at the start the object will be valid and $\e_1$ will contain $l$;
but during reduction, the $l$ reference will be used in order to
give access to the internal state of $l$; only after that moment, the object may become invalid.


\noindent\textbf{Define} $OK(\sigma,e)$:\\
\indent $\forall l\in\dom(\sigma)$
  either\\
\indent\indent 1. $\mathit{garbage}(l,\sigma,\e)$\\
\indent\indent 2. $\mathit{valid}(\sigma,l)$ and $\mathit{wellEncapsulated}(\sigma,\e,l)$\\
\indent\indent 3. $\mathit{monitored}(\e,l)$

Finally, the system is in a valid state with respect to validation
if for all the objects in the memory, one of these 3 cases apply:
%the class of the object has no invariant method;
the object is not (transitively) reachable from the expression (thus can be garbage collected);
the object is valid, and the object is encapsulated;
or the object is currently monitored.

\begin{theorem}[Stronger Sound Validation]
if $c:\Kw{Cap};\emptyset\vdash \e_0: \T_0$ and
$c\mapsto\Kw{Cap}\{\_\}|\e_0\rightarrow^+ \sigma|\e$, then
$OK(\sigma,\e)$
\end{theorem}
\noindent Starting from only the capability object,
any well typed expression $\e_0$ can be reduced for an arbitrary amount of steps,
and $OK$ will always hold.
\\
\begin{theorem} Stronger Sound Validation $\Rightarrow$ Sound Validation
\end{theorem}
\begin{proof}
\noindent By Stronger Sound Validation, each $l$ in the current redex must be $OK$:
\begin{enumerate}
	\item If $l$ is garbage, it cannot be in the current redex, a contradiction.
	\item If $\mathit{valid}(\sigma,l)$, then $l$ is valid, so thanks to Determinism
	no invalid object could be observed.
	\item Otherwise, if $\mathit{monitored}(\e,l)$ then either:
	\begin{itemize}
	 \item we are executing inside of $\e_1$ thus the current redex is inside of a sub-expression of the monitor that does not contain $l$, a contradiction.
	 \item or we are executing inside $\e_2$:
	 by our reduction rules, all monitor expressions start with 
	 $\e_2=l$\Q@.validate()@, thus the first execution step
	 of $\e_2$ is trusted. Following execution steps are also trusted, since by well formedness the body of invariant methods only use \Q@this@ (now translated to $l$) to access fields.
	\end{itemize}
\end{enumerate}
In any of the possible cases above, Sound Validation holds for $l$, and so it holds for all redexes.
\end{proof}

\subsection{Subject Reduction}

\noindent\textbf{Define} $\text{fieldGuarded}(\sigma,\e)$:\\*
\indent$\forall \ctx$ such that $\e=\ctx[l\singleDot\f] $
and $\Sigma^\sigma(l).f=\Kw{capsule}\,\_$, and $\f\mathrel{\mathit{inside}} \Sigma^\sigma(l).\mathit{validate}$\\*
\indent\indent either 
$\forall T, \forall C, \Sigma^\sigma;\x:\Kw{mut}\,C\,\not\vdash\ctx[\x]:T$, or\\*
\indent\indent $\ctx=\ctx'[$\Q@M(@$l$\Q@;@$\ctx''$\Q@;@$\e$\Q@)@$]$ and $l$ is contained exactly once in $\ctx''$

That is, all \emph{mutating} capsule field accesses are individually guarded by monitors.
Note how we use $C$ in $\x:\Kw{mut}\,C$ to guess the type of the accessed field,
and that we use the full context $\ctx$ instead of the evaluation context $\ctx_v$
to refer to field accesses everywhere in the expression $\e$.


\begin{theorem}[Subject Reduction]
if $\Sigma^{\sigma_0};\emptyset\vdash e_0: T_0$,
$\sigma_0|e_0\rightarrow \sigma_1|e_1$,
$OK(\sigma_0,\e_0)$
and
$\mathit{fieldGuarded}(\sigma_0,\e_0)$
then
$\Sigma^{\sigma_1};\emptyset\vdash e_1: T_1$,
$OK(\sigma_1,e_1)$ and
$\mathit{fieldGuarded}(\sigma_1,\e_1)$
\end{theorem}

\begin{theorem}
	Progress + Subject Reduction $\Rightarrow$ Stronger Sound Validation
\end{theorem}
\begin{proof}
This proof proceeds by induction in the usual manner.

\emph{Base Case}: At the start of the execution, the memory is going to only contain $c$: since $c$ is defined to be initially $\mathit{valid}$, and has only \Q@mut@ fields, and so it is trivially $\mathit{wellEncapsulated}$, thus $OK(c\mapsto\Kw{Cap},e)$.

\emph{Induction}: By Progress we always have another evaluation step to take, by Subject Reduction such a step will preserve $\mathit{OK}$, and so by induction $\mathit{OK}$ holds after any number of steps.

Note how for the proof garbage collection is important: 
when the \Q@validate()@ method evaluates to \Q@false@, 
execution can continue only if the offending object is classified as garbage.
\end{proof}

\subsection{Expose Instrumentation}
We first introduce a lemma derived from well formedness and the type system:
\begin{Lemma}[ExposerInstrumentation]
If $\sigma_0 | \e_0\rightarrow \sigma_1 |\e_1$ and
$\text{fieldGuarded}(\sigma_0,\e_0)$
\\*
then $\text{fieldGuarded}(\sigma_1,\e_1)$
\end{Lemma}
\begin{proof}
The only rule that can 
introduce a new field access is \textsc{mcall}.
In that case, ExposerInstrumentation holds
by well formedness (all field accesses in methods are of the form \Q@this.f@) 
and since \textsc{mcall} inserts a monitor while invoking capsule mutator methods, and not field accesses themselves. If however the method is not a \Q@mut@ method but still accesses a capsule field, by MutField such a field access expression cannot be typed as \Q@mut@ and so no monitor is needed.

Note that \textsc{monitor exit} is fine because monitors are removed only when
 $e_1$ is a value.
\end{proof}

\subsection{Proof of Subject Reduction}
Any reduction step can be obtained
by exactly one application of rule \textsc{ctx} and then one other rule. Thus the proof can simply proceed by cases on such other applied rule.

By SubjectReductionBase and ExposerInstrumentation, 
$\Sigma^{\sigma_1};\emptyset\vdash e_1: T_1$ and  $\mathit{fieldGuarded}(\sigma_1,\e_1)$. So we just need to proceed by cases on the reduction rule applied to verify that $OK(\sigma_1,\e_1)$ holds:


\begin{enumerate}
\item (\textsc{update}) $\sigma|l\singleDot f\equals v\rightarrow \sigma'|\e'$:
	\begin{itemize}
	  \item By \textsc{update} $\e'=\Kw{M}\oR l;l;l\singleDot\text{validate}\oR\cR\cR$, thus $\mathit{monitored}(\e,l)$.
	  \item Every $l_1$ such that $l\in \mathit{rog}(\sigma,l_1)$ will verify the same case as the former step:
	  \begin{itemize}
	  	\item If it was $\mathit{garbage}$, clearly it still is.
	  	\item If it was $\mathit{monitored}$, it also still is.
	    \item Otherwise it was $\mathit{valid}$ and $\mathit{wellEncapsulated}$:
			\begin{itemize}
				\item If $l\in \mathit{erog}(\sigma,l_1)$ we have a contradiction since $mutatable(l, \sigma, e)$, (by MutField)
		    	\item Otherwise, by our well-formedess criteria that \Q@.validate()@ only accesses \Q@imm@ and \Q@capsule@ fields, and by Determinism it is clearly the case that $\mathit{valid}$ still holds;
				By HeadNotCircular it cannot be the case that $l\in \mathit{erog}(\sigma',l_1)$ and so $l_1$ is still $\mathit{wellEncapsulated}$.
		  	\end{itemize}
	  \end{itemize}
	  \item Every other $l_0$ is not in the reachable object graph of $l$
	  thus it being $\mathit{OK}$ could not have been effected by this reduction step.
	\end{itemize}

\item (\textsc{access}) $\sigma|l\singleDot f \rightarrow \sigma|v$:
	\begin{itemize} 
		\item If $l$ was $valid$ and $wellEncapsulated$:
		\begin{itemize}
			\item If we have now broken $wellEncapsulated$ we must have made something in its $erog$  $mutatable$. As we can only type \Q@capsule@ fields as \Q@mut@ and not \Q@imm@ fields, by FieldMut we must have that $f$ is \Q@capsule@ and $l\singleDot f$ is being typed as \Q@mut@. By $\mathit{fieldGuarded}(\sigma_0,\e_0)$ the former step must have been inside a monitor \Q@M(@$l$\Q@;@$\ctx_v[l$\Q@.f@$]$\Q@;@$\e$\Q@)@
		    and the $l$ under reduction was the only occurrence of $l$.
		    Since $f$ is a capsule, we know that $l\notin \text{erog}(\sigma,l)$
		    by HeadNotCircular. Thus in our new step $l$ is not $inside$ $\ctx_v[v]$. Thus $l$ must be $monitored$ and hence it is $OK$.
		    
		    \item Otherwise, $l$ is still $OK$
    	\end{itemize}

		\item Nothing that was $\mathit{garbage}$ could have been made reachable by this expression, since the only value we produced was $v$ and it was reachable through $l$ (and so could not have been garbage), thus $garbage$ is still $OK$.
		
		\item As we don’t change any monitors here, nothing that was $monitored$ could have been made un-$monitored$, and so it is still $OK$.
		
		\item Suppose some $l_0$ was $wellEncapsulated$ and $valid$:
		\begin{itemize}
			\item If $l$ was in the $rog$ of $l_0$, by CapsulaeTree, if $l$ was in the $erog$ of $l$, then $v$ can only be reached from $l_0$ by passing through $l$, and so we could not have made $l_0$ non-$wellEncapsulated$. In addition, since only things in the $erog$ can be referenced by $\singleDot\Kw{validate}\oR\cR$, $l_0$’s validity can not depend on $l$, and by Determinism it is still the case that $l_0$ is $valid$. And so we can’t have effected $l_0$ being $OK$.
			\item Otherwise this reduction step could not have affected $l_0$ so $l_0$ is still $OK$.
		\end{itemize}
\end{itemize}

\item (\textsc{mcall}, \textsc{try enter} and \textsc{try ok}):

	These reduction steps do not modify memory, nor do they modify the memory-locations reachable inside of main-expression, nor do they modify any monitor expressions. Therefore it cannot have any effect on the $garbage$, $wellEncapsulated$, $valid$ (due to Determinism) or $monitored$ properties of any memory locations, thus $\mathit{OK}$ still holds.

\item (\textsc{new}) $\sigma|\Kw{new}\ C\oR\vs\cR\rightarrow \sigma,l\mapsto C\{\vs\}| \Kw{M}\oR l;l;l\singleDot\text{validate}\oR\cR\cR$:

	Clearly the newly created object ($l$) is monitored. As for \textsc{mcall}, other objects and properties are not disturbed, and so $\mathit{OK}$ still holds.


\item (\textsc{monitor exit}) $\sigma|\Kw{M}\oR l; v;\Kw{true}\cR\rightarrow \sigma|v$:
\begin{itemize}
	\item As monitor expressions are not present in the original source code, it must have been introduced by \textsc{update}, \textsc{mcall}, or \textsc{new}. In each case the 3\textsuperscript{rd} expression started of as $l\singleDot\Kw{validate}\oR\cR)$, and it has now (eventually) been reduced to $\Kw{true}$, thus by Determinism $l$ is $valid$.

	\item  If the monitor was introduced by \textsc{update}, then $v = l$. We must have had that $l$ was well encapsulated before \textsc{update} was executed (since it can’t have been garbage and $monitored$), as \textsc{update} itself preserves this property and we haven’t modified memory in anyway, we must still have that $l$ is $wellEncapsulated$. As $l$ is $valid$ and $wellEncapsulated$ it is $OK$.

	\item If the monitor was introduced by \textsc{mcall}. Then it was due to calling a capsule-mutator method that mutated a field $f$.
	\begin{itemize}
		\item A location that was $garbage$ obviously still is, and so is also $OK$.
		\item No location that was $valid$ could have been made non-valid since this reduction rule performs no mutation of memory. If a location was $wellEncapsulated$ before the only way it could be non-$wellEncapsulated$ is if we somehow leaked a \Q@mut@ reference to something, but by our well-formedness rules $v$ cannot be typed as \Q@mut@ and so we can’t have affected $wellEncapsulated$, hence such thing is still $OK$.
		\item The only location that could have been made un-$monitored$ is $l$ itself. By our well-formedness criteria $l$ was only used to modify $l.f$, and we have no parameters by which we could have made $l.f$ non-$wellEncapsulated$, since that would violate CapsuleTree. As nothing else in $l$ was modified, and it must have been $wellEncapsulated$ before the \textsc{mcall}, it still is, and since  $l$ is valid, it is $OK$.
	\end{itemize}
	\item Otherwise the monitor was introduced by \textsc{new}. Since we require that \Q@capsule@ fields and \Q@imm@ fields are only initialised to \Q@capsule@ and \Q@imm@ expressions, by CapsuleTree the resulting value, $l$, must be $wellEncapsulated$, since $l$ is also $valid$ we have that $l$ is $OK$.

\end{itemize}

\item (\textsc{try error}) $\sigma,\sigma_0|\Kw{try}^\sigma\oC \mathit{error}\cC\ \Kw{catch}\ \oC\e\cC\rightarrow \sigma|\e$:

	By StrongExceptionSafety we know that $\sigma_0$ is $\mathit{garbage}$ with respect to $\ctx_v[\e]$. By our well-formedness criteria no location inside $\sigma$ could have been $monitored$.

	Since we don’t modify memory, everything in $\sigma_0$ is $\mathit{garbage}$ and nothing inside $\sigma$ was previously monitored, it is still clearly the case that everything in $\sigma$ is $\mathit{OK}$
\end{enumerate}

\end{document}
