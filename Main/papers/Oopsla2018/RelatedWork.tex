\saveSpace
\section{Related Work}
\saveSpace
Literature on code reuse is too vast to let us do justice of it in a few pages.
In particular, we where unable to discuss all the variations on Family polymorphism.
Our work is inspired by traits~\cite{ducasse2006traits}, which in turn
are inspired by module composition languages~\cite{ancona2002calculus}.

\subsection{Separating Inheritance and Subtyping}
In languages like Cecil~\cite{chambers1995typechecking} and polyToil~\cite{bruce1995polytoil}
classes are not types. This is a more radical solution to Inheritance implies Subtyping, and equivalent to a restricted version of \name where only interface names can be used as types.
This complicates typing of \Q@this@,
and may prevent any usefully application of \Q@This@ type (polyToil uses polymorphism to support it).
For example, in those approaches, the following code does not work since \Q@A@ is not a type:
\saveSpace\saveSpace\begin{lstlisting}[language=Java]
  class A{ int ma(){return Utils.m(this);} }  class Utils{static int m(A a){..}}
\end{lstlisting}\saveSpace\saveSpace
To overcome this limit Cecil  rely on some syntactic sugar, advantages and pitfalls of this design are discussed in the appendix.

\name is directyl inspired by the 3 independently designed research languages already mentioned:
 TraitRecordJ~(TR)\cite{Bettini:2010:ISP:1774088.1774530}, Package Templates~(PT)\cite{KrogdahlMS09}, DeepFJig(DJ)~\cite{deep}.
We synthesize
the best ideas of those very different designs, while at the same time 
coming up with a simpler and improved design for separating
subclassing from subtyping, which also addresses various limitations of those
3 particular language designs.
The following compares 
various aspects of the language designs;
we \underline{underline} 3 properties where one approach shines the most, and 3 properties where one approach is more lacking.
\begin{itemize}
\item {\bf A simple uniform syntax for code literals}
\underline{DJ} is best in this sense: TR has separate syntax for class literals, trait literals and record literals.
PT is built on top of full Java, thus has a very
involved syntax.
\name relies on DJ's approach but,
\emph{thanks to our novel representation of state}, \name also offers a much simpler and uniform syntax than
all other approaches: everything is just a method.
\item 
{\bf Reusable code cannot be ``used'' (that is be instantiated or used as a type).}
This happens in TR and in PT, but not in \underline{DJ}. To allow reusable code to be directly 
usable, in DJ
classes introduce nominal types in an unnatural way: the type of
\Q@this@ is only \Q@This@ (sometimes called \Q@<>@) and not the
nominal type of its class. 
That is in DJ 
`\Q@A={method A m(){return this;}}@' is not well typed. This is because
`\Q@B= Use A@' flattens to `\Q@B={method A m(){return this;}}@', which is clearly not well typed.
Looking to this example is clear why we need reusable code to be agnostic of its name.
Then, either reusable code has no name (as in TR, PT and \name)
or all code is reusable and usable, and all code needs to be awkwardly agnostic of its name, as in DJ.

\item 
{\bf Requiring abstract signatures is a left over of module composition mindset.}
TR and DJ comes from a tradition of functional module composition, where 
modules are typed in isolation under an environment, and then the composition is performed.
As we show in this work, this ends up requiring verbose repetition of abstract signatures,
which (for highly modularized code) may end up constituting most of the program.
Java (and thus \underline{PT}, as a Java extension) shows us a better way:
names are understood from the reuse context.
The typing of PT offers the same advantages of our typing model, 
but is more involved and indirect. This may be caused by the
heavy task of integrating with full Java.
Recent work based on TR is trying to address this issue too~\cite{damiani2017unified}.
\item {\bf Composition algebra.}
The idea of using composition operators over atomic values as in an arithmetic expression is very powerful,
and makes it easy to extend languages with more operators. DJ and TR embrace this idea, while \underline{PT} takes the traditional Java/C++ approach of using enhanced class/package declaration syntax.
The typing strategy of PT also seems to be connected with this
decision, so it would be hard to move their approach in a composition
algebra setting.
\item {\bf Complete ontological separation between use and reuse}
While all 3 works allow separating inheritance and subtyping only \underline{TR} properly enforces 
separation between use (classes and interfaces) and reuse (traits).
In DJ all classes are both units of use and reuse (however, subtyping is not induced).
PT imports all the complexity of Java: is possible to separate use and reuse, the model have powerful but non-obvious implications where Java \Q@extends@ and PT are used together.
\item {\bf Naming the self type, even if there is none yet.}
\underline{TR} is lacking here, while DJ and PT both allow a class to refer to its name; albeit this is
less obvious in PT since both a package and a class have to be introduced to express it.
This allows encoding binary methods, expressing patterns like withers or fluent setters and to instantiate instances of the (future) class(es)  using the reused code.
\end{itemize}

\subsection{Family
Polymorphism by disconnecting Use and Reuse}

Our \Q@Use@ operator is similar to deep mixin composition~\cite{Ernst99a,Zenger-Odersky2005, Hutchins06}
and family polymorphism~\cite{Ernst06, igarashi2005lightweight, IgarashiViroli07, IgarashiEtAl08}, but is symmetric, and with the operator
\Q@super@ offers a flexible  explicit conflict resolution.
Our presented solution to the expression problem is
natural and improves over existing solutions in the literature, where our close contender is DJ~\cite{deep}: all our gain over their model is based on our relaxation over abstract signatures.
A similar syntax can be achieved with the Scandinavian style~\cite{ernst2004expression}, or with the work of 
Nystrom (Jx~\cite{NystromEtAl04} and
J\&~\cite{nystrom2006j}), where the composition behaves similarly to our sum operator.
Both Jx,J\& and the virtual classes of Ernst~\cite{Ernst06}
make uses of dependent types.
As in \Q@.@FJ and \Q@^@FJ~\cite{igarashi2005lightweight,IgarashiEtAl08,IgarashiViroli07,saito2008essence},
 instead, we do not need sophisticated types. 
The work on DJ~\cite{deep} contains an in-depth comparison between various FP approaches, including an example written in \Q@.@FJ syntax synthesizing the difficulty to support FP while keeping Use and Reuse connected:
\saveSpace\saveSpace
\begin{lstlisting}
class A{static class B{int f1;}    int k(.B x){ return x.f1;}}
class AA extends A{static class B{int f2;}   int k(.B x){return x.f2+new .B().f2;}}
\end{lstlisting}\saveSpace\saveSpace
The syntax \lstinline{.B} denotes a relative path, that is, the 
class \lstinline{B} in scope.
In Java and Scala declaring a subclass of a nested class with the same
name of a nested of the superclass only has the effect of hiding parent's declaration.
Approaches supporting FP~\cite{igarashi2005lightweight,IgarashiEtAl08,nystrom2006j,Ernst06,BruceEtAl98,IgarashiViroli07,deep}
consider such definition a form of \emph{overriding},
called \emph{further extension}:
\lstinline{AA.B} further extends \lstinline{A.B}: it is implicitly considered a subclass of \lstinline{A.B}, adding the field \Q@f2@.
Consider now the following code:
\saveSpace\saveSpace\begin{lstlisting}
new AA().k(new AA.B())//well-typed
new A().k(new A.B())//well-typed
A a=new AA(); //well-typed assuming AA subtype of A
a.k(new A.B())//runtime error: A.B.f2 does not exist
\end{lstlisting}\saveSpace\saveSpace
The sound \Q@.@FJ type system consider the last method invocation ill-typed even though \lstinline{AA.B} is a subtype of \lstinline{A.B}.
With minor changes, others~\cite{nystrom2006j,Ernst06,BruceEtAl98,IgarashiViroli07}
supports this example in the same way.
Inheritance implies Subtyping is broken only in a controlled way, and it is allowed whenever does not lead to unsoundness.
Recent work on {\bf ThisType} \cite{Saito:2009,ryu16ThisType}
also continues on this line.
In those works, ``subtyping by subclassing'' is preserved: those designs aim to retain the programming model of mainstream OOP languages and backwards compatibility. 42 is instead a radical departure of mainstream OOP, hoping to improve the mechanisms for use and reuse in OOP and unlock new ways to design software.

From a different perspective, we can say that traditional
implementations of family polymorphism are still heavily influenced by
the ``Inheritance implies Subtyping'' model.
We believe that this is a major source of complexity in the type
systems of those approaches:
They need to track those calls, and enforce the \emph{family} of the receiver and the argument is the same.
Because we separate inheritance from subtyping we liberate ourselves
from tricky issues that arise in such type systems, and we can
provide a simpler model of family polymorphism, soundly supported by 
a straightforward nominal type system:
by disconnecting use and reuse we outlaw \Q@A a=new AA()@.
In \name this is also reducing the expressing power a little, but
in the full 42 language, as well as in DJ, the operator \textbf{redirect} allows to write code that is parametric on families of data types.
To the same aim, .FJ relies on generics.

Researchers with a strong grasp on Family
Polymorphism know that support for FP strictly includes
support for `\Q@This@' type and self instantiation, since the
class is a member of its own family.
Scala allows to \textbf{encode} further extension/deep mixin composition,
but it requires to do it explicitly, growing the amount of required glue-code.


%\begin{lstlisting}
%A = { B = {...}                Int m(B b){...} }
%A2 = Use A, { B = {... Int y;} Int m(B b){ ... b.y ... }}
%A a=new A2();//this line is considered sound by most FP
%a.m(new A.B());//this line is unsound, but is hard to prevent
%\end{lstlisting}

%In those approaches, \Q@A2@ is a subtype of \Q@A@,
%so code like \Q@A a=new A2();@ is accepted.
%However, this implies that innocent looking code like
%\Q@a.m(new A.B());@ is now unsound: method \Q@A2.m@ will
%try to access field \Q@A2.B.y@
%that is not present in class \Q@A.B@.


\saveSpace
\subsection{State and traits}
\saveSpace
The original trait model has no self construction 
and purposely avoided any connection between state and traits.
Since it was applied to a dynamic language, the relation with `\Q@This@' type is unclear.

The idea of abstract state operations emerged from Classless
Java~\cite{wang2016classless}. This approach offers a clean solution to handle state
in a trait composition setting.
Note how abstract state operations are different from just hiding fields under getter and setters: 
in our model the programmer simply never has to declare what is the state of the class, not even what information is stored in fields.
The state is computed by the system as an overall result of the whole code composition process.

In the literature there have been many attempts to add state in traits/module composition languages:
\begin{itemize}  
\item The simplest solution have {\bf no constructors}, and fields start with {\bf null} (or zero/false).
In this setting fields are another kind of (abstract) member, and two fields with identical types can be merged by sum/use; \Q@new C()@ can be used for all classes, and \Q@init@ methods may be called later, as in
  \Q@Point p=new Point(); p.init(10,30)@.
  
This approach is commutative and associative.
  However, objects are created "broken" and the user is trusted with fixing them.
  While it is easy to add fields, the load of initializing them is on the user; moreover
    all the objects are intrinsically mutable, preventing a functional programming style.
\item {\bf Constructors compose fields}:
Here a canonical constructor (as in FJ) taking a parameter for each field and just initializing the fields is assumed to be present.
It is easy to add fields, however this model (used by \cite{fjig}) is associative but not commutative: composition order influences field order, and thus the constructor signature.
Self construction is also not possible 
since the signature of the constructors changes during composition.


\item {\bf Constructors can be composed if they offer the same exact parameters}:
Traits declare fields and constructors, they initializes the fields using any kind of computation.
Traits whose constructors have the same signature can be composed.
The composed constructor will execute both constructor bodies in order.
This approach is designed in DJ to allow self construction.
It is associative and mostly commutative: composition order only influences execution order of side effects during construction.
However trait composition requires identical constructor signatures: this
hampers reuse, and if a field is added, its initial value needs to be
synthesized from the other parameters.
\end{itemize}
\saveSpace\saveSpace
\subsection{Tabular comparison of many approaches}
\saveSpace
\begin{minipage}[t]{0.30\textwidth}
In this table we show if some constructs support certain features:
Direct instantiation (as in \Q@new C()@),
Self instantiation (as in \Q@new This()@),
Is this construct a `Unit of use'?, a `Unit of reuse'?,
Does using this construct introduce a type? and is the induced type the type of \Q@this@?,
support for binary methods,
does inheritance of this construct induce subtype?,
is the code of this construct required to be well-typed before being inherited /imported in a new context?
is it required to be well-typed before being composed with other code?
\end{minipage}
%second column
\begin{minipage}[t]{0.6\textwidth}
\newcommand{\YY}{\textbf{Y}}
\begin{center}
\begin{tabular}{c|c|c|c|c|c|c|c|c|c|c}
&\Rotated{direct instantiation}
&\Rotated{self instantiation}
&\Rotated{unit of use}
&\Rotated{unit of reuse}
&\Rotated{introduce type}
&\Rotated{induced type is this type}
&\Rotated{binary methods}
&\Rotated{{${}_{}$\!inheritance induce subtype\!\!\!}}
&\Rotated{{${}_{}$\!typed before imported\!\!\!}}
&\Rotated{{${}_{}$\!typed before composed\!\!\!}} 
\\
\hline
java/scala class&\YY &X&\YY &\YY &\YY &\YY &X&\YY &\YY &X\\
java8 interface &X&X&X&\YY &\YY &\YY       &X&\YY &\YY &X\\
scala trait        &X&X&X&\YY &\YY &\YY    &-&\YY &\YY&X\\
original trait     &X&X&X&\YY &-&-         &-&X&-&-\\
TR  &X&X&X&\YY &X&-                        &X&X&\YY &\YY \\
\name trait        &X&\YY &X&\YY &X&-      &\YY &X&\YY &X\\
\name class        &\YY &\YY &\YY &X&\YY   &\YY &\YY &-&\YY &-\\
module composition
                      &-&-&\YY &\YY &-&-   &-&-&\YY &\YY \\
deepFJig class &\YY &\YY &\YY &\YY &\YY &X &\YY &X&\YY &\YY \\
package template
                      &X&\YY &X&\YY &X&-   &-&X&\YY &X\\
${}_{}$\\
\end{tabular}
\end{center}
\end{minipage}

\noindent \textbf{Y} and X means yes and no; we use ``-'' where the question is not applicable to the current approach. For example the original trait model was untyped, so typing questions makes no sense here.


