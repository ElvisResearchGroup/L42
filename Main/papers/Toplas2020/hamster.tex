\section{Hamster Case Study}
\label{s:hamster}

 
Suppose we have a \Q@Cage@ class which contains a \Q@Hamster@; the \Q@Cage@ will move its \Q@Hamster@ along a path. We would like to ensure that the \Q@Hamster@ does not deviate from the path. We can express this as the invariant of \Q@Cage@: the position of the \Q@Cage@'s \Q@Hamster@ must be within the path (stored as a field of \Q@Cage@).
This example is interesting since it relies on \Q@List@s and \Q@Point@s that are not designed with \Q@Hamster/Cage@s in mind.

%We show a detailed analysis on our Spec\# implementation, showing all the effort we put in trying to find better solutions.

%
% While Spec\# requires specialised \Q@Point@, \Q@Hamster@, and \Q@Cage@ declarations to be able to enforce the invariant, our version manages to capture the required information in just a few annotations on \Q@Cage@ and leaves \Q@Point@ and \Q@Hamster@ unmodified.
%	if(that==null || !(that instanceof Point)){return false;}
% 	return ((Point)that).x==this.x && ((Point)that).y==this.y; 
%  }
\begin{lstlisting}
class Point { Double x; Double y; Point(Double x, Double y) {..}
  @Override read method Bool equals(read Object that) {
    if (!(that instanceof Point)) { return false; }
    Point p = (Point)that;
    return this.x == p.x && this.y == p.y;
  }
}
class Hamster { Point pos; Hamster(Point pos) {..} }//pos is imm by default
class Cage {
  capsule Hamster h;
  List<Point> path; //path is imm by default
  Cage(capsule Hamster h, List<Point> path) {..}
  read method Bool invariant() { return this.path.contains(this.h.pos); }
  mut method Void move() {
    Int index = 1 + this.path.indexOf(this.pos()));
    this.moveTo(this.path.get(index % this.path.size())); }
  read method Point pos() { return this.h.pos; }
  mut method Void moveTo(Point p) { this.h.pos = p; }
}
\end{lstlisting}

The \Q@invariant()@ method on \Q@Cage@ simply verifies that the \Q@pos@ of \Q@this.h@ is within the \Q@this.path@ list. This is accepted by our invariant protocol since \Q@path@ is an \Q@imm@ field (hence deeply immutable) and \Q@h@ is a \Q@capsule@ field (hence fully encapsulated). The \Q@path.contains@ call is accepted by our type system as it only needs \Q@read@ access: it merely needs to be able to access each element of the list and call \Q@Point@'s equal method, which takes a \Q@read@ receiver and parameter.
The \Q@move@ method actually moves the hamster along the path, but to ensure that our restrictions on \Q@capsule@ fields are respected we forwarded some of the behaviour to separate methods: \Q@pos()@ which returns the position of \Q@h@ and \Q@moveTo(p)@ which updates the position of \Q@h@.
The \Q@pos@ method is needed since \Q@move()@ is a \Q@mut@ method, and so any direct \Q@this.h@ access would cause it to be a capsule mutator, which would make the program erroneous as \Q@move()@ uses \Q@this@ multiple times.
Similarly, we need the \Q@moveTo(p)@ method to modify the ROG of the \Q@h@ field, this must be done within a capsule mutator that uses \Q@this@ only once.

As our \Q@path@ and \Q@h@ fields are never themselves updated, the only point where the ROG of our \Q@Cage@ can mutate is in the \Q@moveTo(p)@ capsule mutator, thus 
our invariant protocol will insert runtime invariant checks only here and at the end of the constructor.

\begin{comment}
We use the \Q@read@ annotation on the \Q@equals(that)@ method to express that it does not modify either its
receiver or its parameter. In \Q@Cage@ we use 
the \Q@capsule@ annotation to ensure
that the modification of the \Q@Hamster@'s ROG is fully under the control
of the containing \Q@Cage@. 
We annotated the \Q@move()@
and \Q@moveTo(p)@ methods with \Q@mut@, since they modify
their receivers' ROG. The default annotation is \Q@imm@, thus \Q@Cage@'s \Q@path@ field is a deeply immutable list of \Q@Point@s.
% Note how we just use \Q@List.contains()@ and \Q@List.indexOf()@
% to check if the hamster position is inside the list.
% The conventional syntax correctly instantiates a \Q@Cage@:
% \Q@new Cage(new Hamster(new Point(..)), List.of(new Point(...))@.
Our system performs runtime checks for the invariant
at the end of \Q@Cage@'s constructor, \Q@moveTo(p)@ method, and after any update to one of its fields.
The \Q@moveTo(p)@ method is the only one that may (directly) break the \Q@Cage@'s invariant. However, there is only a single occurrence of \Q@this@ and it is used to read the \Q@h@ field. We use the guarantees of RCs to ensure that no alias to \Q@this@ could be reachable from either \Q@h@ or the immutable \Q@Point@ parameter. Thus, the potentially broken \Q@this@ object is not visible while the \Q@Hamster@'s position is updated. 
The invariant is checked at the end of the \Q@moveTo(p)@ method, just before \Q@this@ would become visible again.
This technique loosely corresponds to an implicit pack and unpack: we `unpack' \Q@this@ before reading the field, then we work on the field's value while the invariant of \Q@this@ is not known to hold, finally when returning, we `pack' \Q!this! and check its invariant before allowing it to be used again.
\end{comment}

Note: since only \Q@Cage@ has an invariant,
 only the code of \Q@Cage@ needs to be handled carefully; allowing the code for \Q@Point@ and \Q@Hamster@ to be unremarkable.
Thus our verification approach is more self contained and modular.
 This contrasts with Spec\#: all code involved in  verification needs to be designed with verification in mind~\cite{barnett2011specification}.
% The best solution we found was to define our own equality for \Q@Point@ instead of relying on \Q@Object.Equals@,
% thus we could not use \Q@List.Contains@ and \Q@List.IndexOf@.

\subheading{Comparison with Spec\#}
We now show our hamster example in the system most similar to ours, Spec\#:
%or \small or \footnotesize etc.
\begin{lstlisting}[
language={[Sharp]C}, morekeywords={invariant,ensures,requires,expose,exists}]
// Note: assume everything is `public'
class Point { double x; double y; Point(double x, double y) {..}
  [Pure] bool Equal(double x, double y) { return x == this.x && y == this.y; } }
class Hamster{[Peer] Point pos; Hamster([Captured] Point pos){..} }
class Cage {
  [Rep] Hamster h; [Rep, ElementsRep] List<Point> path;
  Cage([Captured] Hamster h, [Captured] List<Point> path)
    requires Owner.Same(Owner.ElementProxy(path), path); {
      this.h = h; this.path = path; base(); }
  invariant exists {int i in (0 : this.path.Count);
    this.path[i].Equal(this.h.pos.x, this.h.pos.y) };
  void Move() {
    int i = 0;
    while(i<path.Count && !path[i].Equal(h.pos.x,h.pos.y)){ i++; }
    expose(this) { this.h.pos = this.path[i%this.path.Count]; }
  }
}
\end{lstlisting}

In both this and our original version, we designed \Q@Point@ and \Q@Hamster@ in a general way, and not solely to be used by classes with an invariant: thus \Q@Point@ is not an immutable class.

The Spec\# approach uses ownership: the \Q@Rep@ attribute on the \Q@h@ and \Q@path@ fields means its value is owned by the enclosing \Q@Cage@, similarly the \Q@ElementsRep@ attribute on the \Q@path@ field means its \emph{elements} are owned by the \Q@Cage@. Conversely, in the \Q@Hamster@ class, the \Q@Peer@ annotation on the \Q@pos@ field means its value is owned by the owner of the enclosing \Q@Hamster@, thus if a \Q@Cage@ owns a \Q@Hamster@, it also owns the \Q@Hamster@'s \Q@pos@. The \Q@Captured@ annotations on the constructor parameters of \Q@Cage@ and \Q@Hamster@ means that the passed in values must be un-owned and the body of the constructor may modify their owners (the owner is automatically updated when the parameter is assigned to a \Q@Rep@ or \Q@Peer@ field
%, which is exactly what the bodies of the constructors do\footnote{The last line of \Q@Cage@s constructor, \Q@base@ simply calls its base class's (\Q@Object@'s) constructor, this is effectively a no-op. Spec\# however, does not know it is a side-effect free operation, so had the call been at the begining (the default if ommited), it would be unable to verify that the constructors pre-condition still holds when the \Q@path@ field is assigned.}
%
).
% The pre-condition, \Q@Owner.Same(Owner.ElementProxy(path), path)@ requires that the elements of the \Q@path@ parameter are owned by the owner of the (un-owned) \Q@path@ itself (i.e. the elements are also un-owned).
%Marco: it was a very involved sentence. To explain it properly it would take much more text. Do we need it?

Though we don't want either \Q@pos@ or \Q@path@ to ever mutate, Spec\# currently has no way of enforcing that an \emph{instance} of a non-immutable class is itself immutable.\footnote{There is a paper~\cite{DBLP:conf/vstte/LeinoMW08} that describes a simple solution to this problem: assign ownership of the object to a special predefined `freezer' object, which never gives up mutation permission. However, this does not appear to have been implemented. This would provide similar flexibility to the RC system we use, which allows an initially mutable object to be promoted to immutable.} In Spec\#, an \Q@invariant()@ can only access fields on owned or immutable objects, thus necessitating our use of the \Q@Peer@ and \Q@Rep@ annotations on the \Q@pos@ and \Q@path@ fields.

Note that this prevents multiple \Q@Cage@s from sharing the same point instance in their \Q@path@.
%, unless they are owned by the same object (such as a \Q@DoubleCage@ containing two \Q@Hamster@s).  
Had we made \Q@Point@ an immutable class, we would get no such restriction. A similar problem applies to our \Q@pos@ field: the \Q@pos@ of \Q@Hamster@s in different  \Q@Cage@s cannot be the same \Q@Point@ instance.
Note how if we consider being in the ROG of an object's capsule fields as being `owned' by the object, our \Q@capsule@ fields behave like \Q@Rep@ fields; similarly, \Q@mut@ fields (that are in the ROG of a \Q@capsule@ field) behave like \Q@Peer@ fields.

The \Q@expose(this)@ block is needed, since in Spec\# in order to modify a field of an object (like \Q@this.h.pos@), we must first ``expose'' its owner (the \Q@Cage@). During an \Q@expose@ block, Spec\# will not assume the invariant of the exposed object, but will ensure it is re-established at the end of the block. This is similar to our concept of capsule mutators (like our \Q@moveTo@ method above), however it is supported by adding an extra syntactic construct (the \Q@expose@ block), which we avoid.

Finally, note the custom \Q@Equal(x,y)@ method on \Q@Point@: this is needed since we can't overload the usual \Q@Object.Equals(other)@ method because it is marked as \Q@Reads(ReadsAttribute.Reads.Nothing)@, which requires the method not read any fields, even those of its receiver.
We resorted to making our own \Q@Equal(x,y)@ method. Since it is called in \Q@Cage@'s invariant, Spec\# requires it to be annotated as \Q@Pure@, this requires that it can only read fields of objects owned by the \emph{receiver} of the method, so a method \Q@[Pure] bool Equal(Point that)@ can read the fields of \Q@this@, but not the fields of \Q@that@. Of course this would make the method unusable in \Q@Cage@ since the \Q@Point@s we are comparing equality against do not own each other. As such, the simplest solution is to just pass the fields of the other point to the method.
Sadly this mean we can no longer use \Q@List@'s \Q@Contains(elem)@ and \Q@IndexOf(elem)@ methods, rather we have to expand out their code manually.

% In addition, for our equality method to be (indirectly) called in an invariant, it must be marked as \Q@Pure@, however this requires it only reads fields owned by the \Q@receiver@, however the \Q@Point@s in \Q@path@ and \Q@Hamster@ do not own eachother. \IOComm{"eachother" is more appropriate here than "each other", as that means something different; I could also use "one another" if you want} Thus we made our \Q@Equal@ method simply take the fields of the other \Q@Point@, these are immutable value types and so can be freely used in a \Q@Pure@ method.
%Since we had to make our own \Q@Equal@ method, we could not use the library defined \Q@Contains@ method, and instead had to implement this operation manually.

Even with all the above annotations, we needed special care in creating \Q@Cage@s:%\vspace{-1.860px}% magic number that prevents the listings background going onto the next page
\begin{lstlisting}[
%basicstyle=\footnotesize,
language={[Sharp]C}, morekeywords={invariant,ensures,requires,expose,exists}]
List<Point> pl = new List<Point>{new Point(0,0),new Point(0,1)};
Owner.AssignSame(pl, Owner.ElementProxy(pl));
Cage c = new Cage(new Hamster(new Point(0, 0)), pl);
\end{lstlisting}

\noindent In Spec\# objects start their life as un-owned, so each \Q@new@ instruction above returns an unowned object. However when the \Q@Point@s are placed inside the \Q@pl@ list, Spec\# loses track of this. Thus the \Q@AssignSame@ call is needed to mark the elements of \Q@pl@ as still being unowned (since \Q@pl@ itself is unowned).
Contrast this with our system which requires no such operation; we can simply write:
\begin{lstlisting}
Cage c=new Cage(new Hamster(new Point(0,0)),List.of(new Point(0,0),new Point(0,1)));
\end{lstlisting}

%3 read 2 capsule 3 mut extra method moveTo
%----
In Spec\#, we had to add $10$ different annotations, of $8$ different kinds, some of which are quite involved.
In comparison, our approach requires only $8$ simple keywords of $3$ different kinds. However, we needed to write 
separate \Q@pos()@ and \Q@moveTo(p)@ methods.
%  Moreover we had been unable to reuse 
% \Q@Object.Equals@, \Q@List.IndexOf@ and % \Q@List.Contains@.
% Note: we had to add a new class \Q@PureObject@, since the \Q@Objec@ constructor is not annotated as \Q@[Pure]@.
%3 pure,
%1 peer
%3 captured
%2 rep
%1 ElementsRep
%1 requires Owner.Same(Owner.ElementProxy(path), path);
%1 invariant
%1 exists
%expose(this)
%re implementation of indexOf
%dumb equals(double,double)
%dumb class PureObject { [Pure] PureObject() { } }
%Owner.AssignSame(pl, Owner.ElementProxy(pl));
% manually handle ownership details while instantiating a \Q@new Cage(..)@.
% Note how the \Q@expose@ block cover plays the same role of our \Q@moveTo@ method.

%We evaluate our contribution by means of case studies;


% This causes the rest of the file to be ignore:
\endinput % rewrote Spec\# analysis and put it in the hamsterExample
\lstset{language={[Sharp]C}, morekeywords={invariant,ensures,requires,expose,exists,capsule}}

\subheading{Is that really the best way to use Spec\#?}  
Here we describe exactly why encoded the
hamster example in the way we did. For brevity, we will assume the default accessibility is \Q@public@, whilst in both Spec\# and C\#, it is actually \Q@private@.

\subheading{The \Q@Point@ Class} 
The typical way of writing a \Q@Point@ class in C\# is as follows:
\begin{lstlisting}
class Point {
	double x, y;
	Point(double x, double y) { this.x = x; this.y = y; }
}
\end{lstlisting}

This works exactly as is in Spec\#, however we have difficulty if we want to define equality of \Q@Point@s (see below).

\subheading{The \Q@Hamster@ Class} 
The \Q@Hamster@ class in C\# would simply be:
\begin{lstlisting}
class Hamster {
	Point pos;
	Hamster(Point pos) { this.pos = pos; }
}
\end{lstlisting}

Though this is legal in Spec\#, it is practically useless. Spec\# has no way of knowing whether \Q@pos@ is \emph{valid} or \emph{consistent}. If \Q@pos@ is not known to be valid, one will be unable to pass it to almost any method, since by default methods implicitly require their receivers and arguments to be valid (compare this with our invariant protocol, which guarantees that any reachable object is valid).
If \Q@pos@ is not known to be consistent, one will be unable to mutate it, by updating one of its fields or by passing it as an argument (or receiver) to a non-\Q@Pure@ method.
Though we don't want \Q@pos@ to ever mutate, Spec\# currently has no way of enforcing that an \emph{instance} of a non-immutable class is itself immutable\footnote{There is a paper~\cite{DBLP:conf/vstte/LeinoMW08} that describes a simple solution to this problem: assign ownership of the object to a special predefined `freezer' object, which never gives up mutation permission, however this does not appear to have been implemented; this would provide similar flexibility to the RC system we use, which allows an initially mutable object to be promoted to immutable.}, as such we will simply refrain from mutating it.

To enable Spec\# to reason about \Q@pos@'s validity, we will require that it be a \emph{peer} of the enclosing \Q@Hamster@; we can do this by annotating \Q@pos@ with \Q@[Peer]@. Peers are objects that have the same owner, implying that  whenever one is valid and/or consistent, the other one also is. This means that if we have a \Q@Hamster@, we can use its \Q@pos@, in the same ways as we could use the \Q@Hamster@.

To simplify instantiation of \Q@Hamster@s, their constructors will take unowned \Q@Point@s; Spec\# will then automatically make such \Q@Point@ a peer. This is achieved by taking a \Q@[Captured]@ \Q@Point@ in the constructor (note how similar this is to taking a \Q@capsule@ \Q@Point@). Note that unlike our system, this prevents multiple \Q@Hamster@s from sharing the same \Q@Point@, unless both \Q@Hamster@s have the same owner, if \Q@Point@ were immutable, there would be no such restriction.

With the aforementioned modifications, the \Q@Hamster@ becomes:
\begin{lstlisting}
class Hamster {
  [Peer] Point pos;
  Hamster([Captured] Point pos) { this.pos = pos; }
}
\end{lstlisting}

%If however, we did want \Q@Point@ to be an immutable/value type, the original unannotated version would not have any problems.

\subheading{The \Q@Cage@ Class} 
The natural way to write this class in C\#, if it had native support for class invariants like Spec\#, would be:
\begin{lstlisting}
class Cage {
  Hamster h;
  List<Point> path;
  Cage(Hamster h, List<Point> path){this.h=h; this.path=path;}
  invariant this.path.Contains(this.h.pos);
  void Move() { 
    int index = this.path.IndexOf(this.h.pos);
    this.h.pos = this.path[index % this.path.Count]; } 
}
\end{lstlisting}

However for the above \Q@invariant@ to be admissible in Spec\#, \Q@this.path@ and \Q@this.h@ must both be owned by \Q@this@. In addition, the \emph{elements} of \Q@this.path@ need to be owned by \Q@this@, since \Q@this.path.Conatains@ will read them. Note that \Q@this.h.pos@ also needs to be owned by \Q@this@, however since \Q@pos@ is declared as \Q@[Peer]@, if \Q@this@ owns \Q@this.h@, it also owns \Q@this.h.pos@. To fix the invariant, we will declare \Q@h@, \Q@path@, and the elements of \Q@path@ as \emph{reps} (i.e. they are owned by the containing object). Finally, since \Q@Move@ modifies \Q@this.h@, \Q@this.h@ needs to be made consistent, which requires that the owner (\Q@this@) be made invalid; this can be achieved by using an \Q@expose(this)@ statement. \Q@expose(this){@\emph{body}\Q@}@ marks \Q@this@ as invalid, executes \emph{body}, checks that the invariant of \Q@this@ holds, and then marks \Q@this@ valid again.
As we did with the \Q@Hamster@, we will simply take unowned \Q@h@ and \Q@path@ values, however we also need the elements of \Q@path@ to be unowned; since Spec\# has no \Q@[ElementsCaptured]@ annotation, we will require \Q@path@ to be unowned, and its elements (denoted by \Q@Owner.ElementProxy(path)@) to be owned by the same owner as \Q@path@ (which is \Q@null@).
\begin{lstlisting}
class Cage {
  [Rep] public Hamster h;
  [Rep, ElementsRep] List<Point> path;
	
  Cage([Captured] Hamster h, [Captured] List<Point> path)
    requires Owner.Same(Owner.ElementProxy(path), path);
  { this.h = h; this.path = path; }
	
  invariant this.path.Contains(this.h.pos);
  void Move() { 
    int index = this.path.IndexOf(this.h.pos);
    expose(this){this.h.pos=this.path[index%this.path.Count]; }} 
}
\end{lstlisting}

The above constructor now fails to verify, since Boogie is unconvinced that its pre-condition actually holds when we initialise \Q@this.path@. This is because the constructor for \Q@Object@ (the default base class if none is provided) is not marked as \Q@[Pure]@; since it is (implicitly) called upon entry to \Q@Cage@'s constructor, Boogie has no idea as to what memory could've mutated, and so it doesn't know whether the pre-condition still holds. The solution is to explicitly call it, but at the end of the constructor: \Q@{this.h = h; this.path = path; base();}@.

The above \Q@Cage@ code however does not work, since \Q@List@ operations, such as \Q@Contains@ and \Q@IndexOf@, will call the virtual \Q@Object.Equals@ method to compute equality of \Q@Point@s. However \Q@Object.Equals@ implements \emph{reference} equality, whereas we want \emph{value} equality.

\subheading{Defining Equality of \Q@Point@s}
The obvious solution in C\# is to just override \Q@Object.Equals@ accordingly, and let dynamic dispatch handle the rest:
\begin{lstlisting}
class Point {
  .. // as before
  override bool Equals(Object? o) {
    Point? that = o as Point;
    return that!=null && this.x == that.x && this.y == that.y;}
}
\end{lstlisting}

However this fails in Spec\# since \Q@Object.Equals@ is annotated with \Q@[Pure]@\\*\Q@[Reads(ReadsAttribute.Reads.Nothing)]@, and of course every overload of it must also satisfy this. The \Q@Reads@ annotations specifies that the method cannot read fields of \emph{any} object, not even the receiver, this makes overloading the method useless.
% Our best guess as to why \Q@Object.Equals@ is annotated like that is that they expect it to be the default reference-equality, annotating it like this could aid static verification as it implies that whether or not two objects are equal cannot change, even if their fields are modified.

We resorted to making our own \Q@Equal@ method. Since it is called in \Q@Cage@'s invariant, Spec\# requires it to be annotated as \Q@[Pure]@, and either annotated with \Q@[Reads(ReadsAttribute.Reads.Nothing)]@ or \\* \Q@[Reads(ReadsAttribute.Reads.Owned)]@ (the default, if the method is \Q@[Pure]@). The latter annotation means it can only read fields of objects owned by the \emph{receiver} of the method, so a \Q@[Pure] bool Equal(Point that)@ method can read the fields of \Q@this@, but not the fields of \Q@that@. Of course this would make the method unusable in \Q@Cage@ since the \Q@Point@s we are comparing equality against do not own each other. As such, the simplest solution is to just pass the fields of the other point to the method:
\begin{lstlisting}
[Pure] bool Equal(double x, double y) {
  return x == this.x && y == this.y;}
\end{lstlisting}

Sadly however this mean we can no longer use \Q@List@'s \Q@Contains@ and \Q@IndexOf@ methods, rather we have to expand out their code manually.
\lstset{language=FortyTwo}
