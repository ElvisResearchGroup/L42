\section{Introduction}
%Points:
%1. L42 et al are 'realistic'
%2. OC's are like Java, and not like haskell...
%3. TM are opt in
%4. Not advocating but merely pointing out that TMs and %OCs can support CI
%5. Modularity of our approach: don't need to look at %other contracts or invariants

%6. PURITY!! (explain why this is important)
%7. Talk and unpack (as a 3rd? invariant protocal)
%\saveSpace\saveSpace
Software verification is the art of reasoning 
on a program: given a precondition, 
we can verify that a specific method or expression
satisfy a postcondition.
Under this lens, class invariants can be seen as implicit pre and post condition.
However, in a real system, it may be impossible to ensure any precondition. For example, 
in the absence of on the fly static verification of dynamically loaded code there is no way to ensure a
subclass respect behavioural subtyping.
In particular, run-time verification of pre conditions can not verify the behaviour of all the methods of the parameters.

We present an approach to enforce class invariants without needing any reasoning.

Classes can have a boolean method \Q@invariant@,
and any call of \Q@invariant@, everywhere in the system is guaranteed to return \Q@true@.
We obtain such feat in the following way:
\begin{itemize}
\item We have a rich type system, supporting type modifiers \Q@read,mut,imm,capsule@. Similar versions of those modifiers are common in literature.
\item The standard library is designed using object capabilities.
\item The invariant method can use this only to access immutable and capsule fields.
\item The constructor uses this only to initialize the fields.
\item encapsulated state can be accesses with certain limitations
\item The invariant method is checked at run time after the constructor, after every field update and
after methods accessing encapsulated state in mutation.
\item Mutation need to be performed so that all the mutation appear to be atomic with 
\end{itemize}
\LINE

Three recent language design leverage on TMs and OCs to provide safe parallelism.
Those are realistic languages [..].
The programming style required by OCs, while different from Java, is much more near to Java then Haskell. TMs on the other side are completely opt-in, and code can [..]
Here we show an approach, allowing those 3 language 
to  enforce class invariants with a combination of run-time checks and static reasoning.
We rely on TMs and OCs in they way they are already supported in M\#,Pony and L42.

Here we do not advocate in favour or against adding
TMs and/or OCs to Java, C\# or any other language, but we point out how languages supporting TMs and OCs can
enforce class invariants in a very broad scope, where
each class can be modularly analysed by assuming only that all the other classes are well typed.
There is no need to verify or even consider possible contracts or invariants present on any other class.

\LINE
More in the detail, our approach 

Run-time invariant checks are added only in 3 points in the code: at the end of constructors, after field updates and after methods accessing encapsulated state.


We implement our approach over L42 and we perform a
case study showing that our approach requires 
exponentially less invariant checks with respect to the conventional one: to check before and after every method call.


Here we show an approach, allowing those 3 language 
to  enforce that every observable object respect its class invariants.
That is, classes can have a boolean method \Q@invariant@,
and any call of \Q@invariant@, everywhere in the system is guaranteed to return \Q@true@.
The invariant can be broken by following a programming pattern where the broken object becomes temporarily inaccessible.\footnote{
This is similar to what happen with pack/unpack in Spec\#.
Eiffel, D, Jose and other just require the invariant to hold at the start and at the end of every public method. This requirement has been considered too weak by many works~\cite{??}
}





L42 have TMs and OCs
CI can be soundly implemented over TMs and OCs
in an efficient and simple way

Since TMs and OCs allows to soundly/simply support CI,
we should port TMs and OCs in many other languages


\LINE


We only require all the code to be well typed and 


 while 
there are approaches handling aliasing and mutation,
this is made particularly difficult by mutation, aliasing, dynamic dispatch, dynamic class loading,
I/O, and exceptions.
We propose here 

In
that the 
is based on the idea that 
given certain precondition


%empty weakest precondPoints:


% 1. L42 et al are 'realistic'
% 2. OC's are like Java, and not like haskell...
% 3. TM are opt in
% 4. Not advocating but merely pointing out that TMs and OCs can support CI
% 5. Modularity of our approach: don't need to look at other contracts or invariants

% 6. PURITY!! (explain why this is important)
% 7. Talk and unpack (as a 3rd? invariant protocal)Points:
% 1. L42 et al are 'realistic'
% 2. OC's are like Java, and not like haskell...
% 3. TM are opt in
% 4. Not advocating but merely pointing out that TMs and OCs can support CI
% 5. Modularity of our approach: don't need to look at other contracts or invariants

% 6. PURITY!! (explain why this is important)
% 7. Talk and unpack (as a 3rd? invariant protocal)ition

%box pattern encodes atomicity of mutation so that
%the invariant is expected to hold after every atomic mutation


%alex: monday 10am
%----------------

\newpage
Reasoning about imperative object oriented (OO) programs is a non trivial task,
made particularly difficult by mutation, aliasing, dynamic dispatch, I/O, and exceptions. \IO{Their are many ways to perform such reasoning, here we use the type-system to restrict, but not prevent, such behaviour of OO in order to be able to soundly enforce class-invariants with runtime-verification (RV) }
% \IO{[dynamic class loading]},

\IO{\emph{Type modifiers} (TMs) are a (typically opt-in) type system feature that allows reasoning about aliasing and mutation; recently, a new design for them} has emerged that radically improves their usability.
Three different research languages are being independently developed relying on this new design: {M\#}\footnote{Microsoft is quite secretive about their M\# project, we believe M\# is the language described in ~\cite{?}.}~\cite{?}, Pony~\cite{?} and L42~\cite{?}.
These project are reasonably large: several millions lines of M\# code exists and it is used by a large private Microsoft project; Pony and L42 have large libraries and are active open source projects. In particular the TMs of these languages are used to provide automatic/correct parallelism~\cite{GordonEtAl12,clebsch2015deny,clebsch2017orca,?}.


% Intro to OCs

\IO{\emph{Object capabilities} (OCs)} are a programming style that need to be respected while implementing the primitives of the standard library, and native calls that could be non deterministic, such as reading from files or generating random-number.

In this style operations like I/O are not provided by static methods, but by instance methods of classes whose instantiation is kept under control.
OCs are typically not part-of the type system nor do they require special runtime support beyond that provided by a memory-safe language. However, languages that allow user code to directly perform native calls, without going through the standard-library, requires some minimal-type system support to enforce the OC pattern over such calls.
OCs are also supported by M\#, Pony and L42.

\IO{In this paper we show how we can add sound enforcement of class-invariants to languages that already have such TMs and OCs.\footnote{\IO{I.e. we are not trying to suggest adding TMs and OCs to conventional languages.}}}
% instead of the 
%expressive but necessarily more involved
%detailed composable annotations that are common in static-verification.


%without 
%\IO{[restricting dynamic class loading] without requiring static-verification}
% or employing a verification language.

We propose a way of soundly verifying class invariants by using a minimalistic code instrumentation strategy, which can be applied to languages like M\#, Pony or  L42  without adding any additional syntax, nor changing their type systems.
 A class encodes its invariants by providing a boolean method called \Q@invariant@; we say that an object whose \Q@invariant@ method would return true is \emph{valid}. Our approach uses the guarantees already provided by TMs and OCs to ensure that \Q@invariant@ is \emph{pure}\footnote{We say a method is \emph{pure} if it is deterministic and does not mutate existing state.} and called before a possibly invalid object becomes reachable. Such calls to \Q@invariant@ are not present in the source code but are instead performed by our instrumentation; in addition, if such a call returns \Q@false@ (indicating the receiver was invalid) an unchecked exception is thrown; this is similar to failed assertions in Java.

Different verification approaches have different invariant protocols~\cite{FlexibleInvariants}, specifying when the invariant is expected to hold and when it is allowed to be violated.
The common invariant protocol~\cite{?}
%JML, OOSC..., D, Eiffel
only enforces the invariant on the receiver before and after every public method call. Sadly, this allows such receivers to be invalidated and passed around to other code.

% TODO: Call the invariant protocl 'validation'
Our invariant protocol follows the mindset of Spec\#~\cite{?}: we allow temporally violating an object's invariant\footnote{If you were to step with a debugger, you may still see invalid objects, and \Q@invariant@ calls failing.} when their reachable object graph (ROG) is being updated; however, we guarantee that such objects are not used during this process.\footnote{Spec\# is more general and, as an extra feature, allows specially declared methods to use invalid objects, while we require such objects to never be used.}
Unlike Spec\#~\cite{?}, our system allows sound catching of invariant failures; this is sound since TMs can guarantee that the invalid object is garbage-collectable.

Our approach has 2 main benefits over the traditional one: it greatly simplifies reasoning since programmers can assume that objects accessible to them are valid\footnote{In particular, all parameters (including \Q@this@) and their ROGs will be valid.}. And it requires that the invariant be checked significantly less often; we fully implemented our approach over L42 and we conducted a case-study which shows that the traditional approach staggeringly performs 1,000,000 times more runtime-checks than ours.
% systems performs 10,000 less  is up to 10,000 times slower.
% \item Debugging is easier since the invariant failure always happens when the method invalidating the object is still active on the stack trace.
%\end{itemize}
Here we present a sound and efficient system, designed to be simple and easy to use for non-verification experts. Our simplicity however comes with a compromise: we can not verify some forms of data-structures, such as collections of mutable elements.\footnote{Our approach does not prevent correctly implement such data-structures, rather we do not support encoding the correctness of such objects as a class-invariant.}

In section \ref{TODO} we compare against a state of the art system, Spec\#, and show how our system can both verify a significant number of the class-invariants they can (including for example mutable collections of immutable objects) whilst having a significantly lower annotation-burden (about 3 times less).
This is because we use TMs for aliasing and mutability control, which are a simple, easy, and coarse grained
technique. On the other hand, the static-analysis employed by Spec\# is more fine grained but requires more frequent and involved annotations.


% Static verification of method contracts can prove the correctness of OO programs, but this is costly and requires defining the desired semantics of the program in a verification language.
% This however requires limiting dynamic class loading, ensuring that only statically verified code can be loaded.
% Without this restriction, even predicting the behaviour of an innocent looking call like 
% \Q@myPoint.getX()@ is impossible: the dynamic type of \Q@myPoint@ can refer to a dynamically loaded class
% whose method \Q@.getX()@ uses I/O to behave non deterministically, or even to format the user’s hard drive.

% Recently, a new design for \emph{type modifiers}(TMs) has emerged, that radically improves their usability.
% Three different research languages are being independently developed relying on this new design: \IO{Gordon}, Pony and 42.
% Those modifiers include \Q@mut@, \Q@imm@ and \Q@capsule@, to represent mutable, immutable and encapsulated data.
% Gordon and Pony need those type informsations to better exploit multi core machine.
% These modifiers provide opt-in restrictions/guarantees: code where everything has a \Q@mut@-type is as flexible as Java.
% However, when \Q@imm@/\Q@capsule@ are used some code may perform better on a multi core machine, either automatically (gordon) or if the program uses actors (pony).
% 42 developers also recognize that those modifiers may provide benefit for parallelism, but they focus on the implications on third party library usage safety.

% Point: new TM is usable, and there are 3 languages using it for parallelism




	% todo let marco right this:
		% breif summary of what they are for
		% citations....
	 	% breif mention of how much their used

% Point:OC and SES are also needed for parallelism and are present in 42, likely present in Gordon and irrelevant in pony

% Gordon briefly discuss that they relies on \emph{Object capabilities} (OCs) to make their approach sound in respect to interaction with I/O. To the best of our knowledge Gordon do not disclose interaction with exceptions, while Pony do not have exceptions in the first place.


% Point: thus, those 3 system are good canditate for our validation

% Their work do not 
% In order to be sound they need OC and exception safety
%  Object capabilities can been used to enforce determinism and the absence of I/O~\cite{finifter2008verifiable}.

%Recent research on \emph{type modifiers} (TMs) and \emph{object capabilities} (OCs) offer us a simpler way to perform high-level reasoning on OO programs, without restricting dynamic class loading or employing a verification language.
%Type modifiers have been used to enforce ownership and automatic/correct parallelism~cite{GordonEtAl12,clebsch2015deny,clebsch2017orca}. Object capabilities can been used to enforce determinism and the absence of I/O~\cite{finifter2008verifiable}.



%Given these benefits, many emerging languages (such as Rust~\cite{matsakis2014rust} and newspeak~\cite{bracha2010modules}) support some form of type modifiers and/or object capabilities.



% As for failed assertions in Java, an unchecked exception is thrown if such a call returns \Q@false@ (indicating that the object is invalid).

% Note that we do not expect invocations of \Q@invariant@ to be present in the code written by the programmer, \Q@invariant@ invocations are injected by our instrumentation.


%That is, we do not ensure the absence of invalid objects, instead we use exceptions to steer execution to a point where the type system ensures the object is not reachable.
% If you were to step with a debugger, you may see invalid objects, and validation checks failing.

% Our approach hijacks the already present TMs and OCs to ensure that a 
% \Q@invariant@ is automatically called at the right times during execution; an unchecked exception is raise
% if such method return false. We use TMs and OCs also to check that the \Q@invariant@ method is `pure', hence they represent logic predicates on the receiver reachable object graph (ROG)%
%We say that an object \Q@o@ is valid when \Q@o.invariant()@ would return \Q@true@, and is invalid otherwise (e.g. if it returns \Q@false@, throws an exception or does not terminate).

%\hrule

%TODO improve?
%simplifies the writing of code that is unrelated to invariant-checking

%------------------------------
%---
% \item \REVComm{It is the responsibility of the class \emph{author} to ensure the invariant is preserved across all public methods, whereas with validation it is the responsibility of the class \emph{user} to not (indirectly) mutate the object in any way that will break it}{2}{What does this mean?}.

%\end{itemize}
% \noindent Our system is more primitive and general than class invariants, and can be used as a building block to encode 
% them.
% and allows 
%global (global is not the right word)
% reasoning through the program.
% \REVComm{Indeed, class invariants can be represented in our system by defining validity as 
%your ``and'' was wrong, is ``or''
% the class invariant holding or a mutator method currently running (this is easily implemented with a boolean % flag).}{3}{Hard to understand setence}
%, and all objects observing modified by the execution
%creating such objects are made be unreachable if the corresponding unchecked exception was to be captured.

%This approach contrasts with an alternative approach, %that of static verification, in which case a program that %could produce invalid objects would fail to compile. Our %approach is 
%more flexible 
%as it allows for a program that may produce invalid objects to still be useful, with the programmer only needing to handle them when they occur.

% \subsection{Random stuff }
% We show that a na\"{i}ve approach that just checks \validate{} at the end of constructors and setters would be insufficient. We take advantage of TM+OC to conservatively identify classes where validation is possible.
%

\begin{comment}
\noindent\textit{Similarity with checked casts:}
%\subsection{Extended type system and similarity with checked casts}
Even if not as good as full static verification, from a formal perspective
% our system clearly aid reasoning.
our system provides sound guarantees that aid reasoning.

Programmers are still responsible for creating and mutating objects in order to preserve their validity;
but once an object has been successfully created/mutated, it is guaranteed to be valid.
Attempting to create a new invalid object, or to mutate an object into an invalid one, causes
a run time error.

This is similar to casts in Java with respect to casts in C:
In both languages the programmer is responsible for casting values correctly;
however, in Java casts are soundly checked: if in a Java program the control flow \REVComm{goes over a cast}{2}{unclear about whether it describes a cast that succeeds or fails/},
we know that in that specific
execution such assumption was correct.
On the other hand incorrect C casts just default to undefined behaviour:
if a C execution continue after a cast, we know absolutely nothing. The same execution
may fail when run on a different machine, 
or even on the same one in another moment.
In the same sense as checked-casts, object construction and ROG mutation are soundly checked by our system,
and success in a mutation means that in that specific execution, validity was preserved.

\end{comment}


%load Item i
%  playCargo(this.cargo(),i)
%
%static playCargo(mut Cargo c, Item i)
%  c.add(i)
%  c.remove(i)
%  
% 
%valid/validation/validity
%
%name of the method: valid/validate
%-----------------------------------
%
%
%%awesome capability objects
%
%%awesome type modifiers
%
%awesome exception safety %%no, we try to sell Exception safety as an application of TM
%
%%with those 3 together we get validation
%
%%is like a extended type system %? yes by comparision with casts?
%
%contrast with class invariants
%
%
%valid only access fields
%
%valid
%
%
%patterns wrapper/state variable allows for flexible invariants
%
%verify the programmer intentions are self-consistent
%
%encoding post conditions and pre conditions as specialized types
%
%code example
%

%%%
%%%May coonect with "OffensiveProgramming/DefensiveProgramming".
%%%


\noindent\textit{Structure of our paper:}
TODO...

\begin{comment}
 We now present how a real language could support
validation. 
We would like to underline that the
features we need have all been presented in the past, but their application to sound validation has never been explored.

%To ease our explanation, we present various categories
%of objects where we can support validation, starting from the simpler one to handle.
In Section \ref{s:background} we will introduce one by one the various
language features cooperating
to support validation. %, and we work under the assumption that there are no static variables.
We will provide a brief introduction to type modifiers, object capabilities, and exception safety.

In Sections \ref{s:validate}, \ref{s:immState}, and \ref{s:encapsulated}, we will show how TM and OC enable validation.
To clearly communicate what kind of checks the language semantics should perform,
 we will write them down as if they were `generated source code'.
However, our proposed approach is independent of the actual technique to insert these checks, and they may instead be inserted directly into bytecode, or they can be part of the underlying semantics of a virtual machine.

In Section \ref{s:meaning} we will provide formal definitions, and in Appendix \ref{s:proof} we provide  a proof that our language soundly enforces validation.
\end{comment}

