\documentclass[english,submission,code=tt]{programming}
\makeatletter
\DeclareOldFontCommand{\rm}{\normalfont\rmfamily}{\mathrm}
\DeclareOldFontCommand{\sf}{\normalfont\sffamily}{\mathsf}
\DeclareOldFontCommand{\tt}{\normalfont\ttfamily}{\mathtt}
\DeclareOldFontCommand{\bf}{\normalfont\bfseries}{\mathbf}
\DeclareOldFontCommand{\it}{\normalfont\itshape}{\mathit}
\DeclareOldFontCommand{\sl}{\normalfont\slshape}{\@nomath\sl}
\DeclareOldFontCommand{\sc}{\normalfont\scshape}{\@nomath\sc}
\makeatother
\usepackage[backend=biber]{biblatex} % Use Biblatex
\addbibresource{main.bib}
\addbibresource{main.bib}
\usepackage{mathpartir}
\usepackage{amsmath}
\usepackage{amsthm}
\theoremstyle{plain}
\newcounter{definition}
\newtheorem{Definition}[definition]{Definition}
\newcounter{assumption}
\newtheorem{Assumption}[assumption]{Assumption}
\newcounter{lemma}
\newtheorem{Lemma}[lemma]{Lemma}
\usepackage{listings}
\usepackage{xcolor}
\usepackage{letltxmacro}
\usepackage{mathtools}
\usepackage{mathpartir}
%\usepackage{stix}

\definecolor{darkRed}{RGB}{100,0,10}
\definecolor{darkBlue}{RGB}{10,0,100}
\newcommand*{\ttfamilywithbold}{\fontfamily{pcr}\selectfont}
%\newcommand*{\ttfamilywithbold}{\ttfamily}

%found on http://tex.stackexchange.com/questions/4198/emphasize-word-beginning-with-uppercase-letters-in-code-with-lstlisting-package
%\lstset{language=FortyTwo,identifierstyle=\idstyle}
%
\makeatletter
\newcommand*\idstyle{%
        \expandafter\id@style\the\lst@token\relax
}
\def\id@style#1#2\relax{%
        \ifcat#1\relax\else
                \ifnum`#1=\uccode`#1%
                        \ttfamilywithbold\bfseries
                \fi
        \fi
}
\makeatother

\lstset{language=Java,
  basicstyle=\upshape\ttfamily\footnotesize,%\small,%\scriptsize,
  keywordstyle=\upshape\bfseries\color{darkRed},
  showstringspaces=false,
  mathescape=true,
  xleftmargin=0pt,
  xrightmargin=0pt,
  breaklines=false,
  breakatwhitespace=false,
  breakautoindent=false,
 identifierstyle=\idstyle,
 morekeywords={method,Use,This,constructor,as,into,rename},
 deletekeywords={double},
 literate=
  {\%}{{\mbox{\textbf{\%}}}}1
  {~} {$\sim$}1
%  {<}{$\langle$}1
%  {>}{$\rangle$}1
}

\newcommand*{\SavedLstInline}{}
\LetLtxMacro\SavedLstInline\lstinline
\DeclareRobustCommand*{\lstinline}{%
	\ifmmode
	\let\SavedBGroup\bgroup
	\def\bgroup{%
		\let\bgroup\SavedBGroup
		\hbox\bgroup
	}%
	\fi
	\SavedLstInline
}

\newcommand\saveSpace{\vspace{-2pt}}

\newcommand\Rotated[1]{\begin{turn}{90}\begin{minipage}{12em}#1\end{minipage}\end{turn}}

\newcommand{\Q}{\lstinline}
\newenvironment{bnf}{$\begin{aligned}}{\end{aligned}$}
\newcommand{\production}[3]{\textit{#1}&\Coloneqq\textit{#2}&\text{#3}}
\newcommand{\prodNextLine}[2]{&\quad\quad\textit{#1}&\text{#2}}
\newenvironment{defye}{\\\indent$\begin{aligned}}{\end{aligned}$\\}
\newcommand{\defy}[2]{\!\!\!\!\!\!&&#1&\coloneqq#2\\}
%\newcommand{\defyc}[1]{&\phantom{\coloneqq}\ \ #1\\}
\newcommand{\defyc}[1]{\!\!\!\!\!\!\rlap{\quad \quad #1}&&\\}
\newcommand{\defya}[2]{#1&\!\!\!\!\!\!&\coloneqq#2\\}

%\newcommand{\prodFull}[3]{#1&::=&\mbox{#2}&\mbox{#3}}
\newcommand{\prodInline}[2]{#1\Coloneqq#2}
\newcommand{\terminal}[1]{\ensuremath{$\texttt{#1}$}}
%\newcommand{\metavariable}[1]{\ensuremath{\mathit{#1}}}

\newcommand{\Rulename}[1]{{\textsc{#1}}}
\newcommand{\ctx}[1]{\ensuremath{\mathcal{E}_#1}\!}
\newcommand{\libi}[2]{\Q@\{@\Q!interface!\ #1\Q{;} #2\Q@\}@}
\newcommand{\lib}[3]{\Q!interface!\ensuremath{?}\ \libc{#1}{#2}{#3}}
\newcommand{\libc}[3]{\,\Q@\{@\!#1\Q{;}\ #2 \Q{;}\ #3\Q@\}@\!\!}

\newcommand{\rp}[1]{\Q{(}\!#1\Q{)}}
\newcommand{\eq}[1]{\,\Q{=}#1}
\newcommand{\red}[3]{#1\,\Q{<}#2\eq#3\,\Q{>}}
\newcommand{\summ}[2]{#1\ \Q{<+}\ #2}
\newcommand{\from}[2]{#1\ensuremath{[}#2\ensuremath{]}}
\newcommand{\mmid}{{\ensuremath{{\mid}}}\!}
\newcommand{\hole}{\ensuremath{\square}}
\newcommand{\s}[1]{\ensuremath{\mathit{#1s}}}
\makeatletter
\newcommand{\This}[1]{\Q!This!#1\nextpath}
\newcommand{\Cs}[1]{#1\nextpath}
\newcommand{\nextpath}{\@ifnextchar\bgroup{\gobblenextpath}{}}
\newcommand{\gobblenextpath}[1]{\Q!.!#1\@ifnextchar\bgroup{\gobblenextpath}{}}
\makeatother



%--------------------------
\newcommand{\mynotes}[3]{{\color{#2} {\sc #1}: #3}}
\newcommand\isaac[1]{\mynotes{Isaac}{blue}{#1}}

\newcommand\IO[1]{\color{blue}{#1}}
\newcommand\marco[1]{\mynotes{Marco}{green}{#1}}


%\lstset{language=FortyTwo, morekeywords={imm,new,class,this,assert}}
\newcommand{\saveSpace}{\vspace{-3px}}
\newcommand{\loseSpace}{\vspace{1ex}}

\newcommand{\LINE}{%
%		{\noindent\rule{\textwidth}{1pt}}
\par%
	\noindent\makebox[\linewidth]{\textcolor{blue}{\rule{\paperwidth}{1pt}}}%
\par%
%
}
\newcommand{\REVComm}[3]{%
	\textcolor{red}{%
		#1\footnote{%
			\textcolor{red}{%
				\textbf{REV#2{:} #3}%
			}%
		}%
	}%
}

\newcommand{\REVIComm}[3]{%
	\textcolor{red}{%
		#1 \textcolor{red}{%
				\textbf{[REV#2{:} #3]}%
		}%
	}%
}

\newcommand{\validate}{\Q@invariant@}

\begin{document}
\paperdetails{
perspective=theoretical,
area={Type systems}
}
%\title{Validation}
\title{Enforcement of Class Invariants using Type Modifiers and Object Capabilities}
%\author{Authors omitted for double-blind review.}
\author{Isaac Oscar Gariano}
\author{Marco Servetto}
%\affiliation{Victoria University of Wellington}
%\email{marco.servetto@ecs.vuw.ac.nz}
\author{Alex Potanin}
\affiliation{Victoria University of Wellington}
%\email{alex@ecs.vuw.ac.nz}

\keywords{type modifiers, object capabilities, runtime verification, class invariants}


\begin{CCSXML}
<ccs2012>
	<concept>
		<concept_id>10003752.10010124.10010138.10010139</concept_id>
		<concept_desc>Theory of computation~Invariants</concept_desc>
		<concept_significance>500</concept_significance>
	</concept>
	<concept>
		<concept_id>10003752.10010124.10010138.10010142</concept_id>
		<concept_desc>Theory of computation~Program verification</concept_desc>
		<concept_significance>500</concept_significance>
	</concept>
	<concept>
		<concept_id>10011007.10011006.10011008.10011009.10011011</concept_id>
		<concept_desc>Software and its engineering~Object oriented languages</concept_desc>
		<concept_significance>500</concept_significance>
	</concept>
	<concept>
		<concept_id>10011007.10010940.10010992.10010998.10011001</concept_id>
		<concept_desc>Software and its engineering~Dynamic analysis</concept_desc>
		<concept_significance>300</concept_significance>
	</concept>
	<concept>
		<concept_id>10011007.10011006.10011008.10011024.10011032</concept_id>
		<concept_desc>Software and its engineering~Constraints</concept_desc>
		<concept_significance>300</concept_significance>
	</concept>
</ccs2012>
\end{CCSXML}

\ccsdesc[500]{Theory of computation~Invariants}
\ccsdesc[500]{Theory of computation~Program verification}
\ccsdesc[500]{Software and its engineering~Object oriented languages}
\ccsdesc[300]{Software and its engineering~Dynamic analysis}
\ccsdesc[300]{Software and its engineering~Constraints}

\maketitle

\begin{abstract}
${}_{}$


\noindent\textit{Context:} % What is the broad context of the work? What is the importance of the general research area?
Object-oriented programming languages through sub-typing and dynamic dispatch provide great flexibility: they allow code to be adapted/specialised to behave differently in different contexts.
However this flexibility hampers code reasoning, which is made even more complex by dynamic class loading (supported by many mainstream OO languages).
Thus object behaviour is nearly completely unrestricted. This is further complicated with the support OO languages typically have for exceptions, imperative code, and I/O.
In the absence of on the fly static verification of dynamically loaded code, it is difficult for programmers to write code that is correct in a library setting.

\loseSpace
\noindent\textit{Inquiry:} %What problem or question does the paper address? How has this problem or question been addressed by others (if at all)?
We wish to guarantee that for all observable objects, their class-invariant holds.
%We wish to guarantee that a class's invariant holds for all its observable instances.
We wish to ensure simpler and stronger reasoning over the usual convention that class invariants 
need only hold on boundaries of public instance methods.

Most prior work on enforcing class invariants assume unverified/unverifiable restrictions on library code, 
dynamic class loading and I/O, or even just trust programmers to use the provided tools correctly.
For example, static verification often restricts dynamic class loading, while 
run-time verification often unsoundly allows non deterministic code in class-invariants.

%We wish to allow for stronger reasoning over this by ensuring that non boke
%We wish to guarantee that a user defined property holds for all observable instances of a class.
%This is a variation of class invariants, where objects in a broken state can never be observed.

\loseSpace
\noindent\textit{Approach:} %What was done that unveiled new knowledge?
We combine previous work on type modifiers and object capabilities to ensure function-purity, aliasing 
and mutability control. Type modifiers and object capabilities are useful in their own right, 
but we use these facilities here to enforce class-invariants by injecting runtime-validation code. 
Our approach is sound since we ensure that class-invariants are pure and can only be violated in 
controlled situations.

\loseSpace
\noindent\textit{Knowledge:} %What new facts were uncovered? If the research was not results oriented, what new capabilities are enabled by the work?
By means of examples, we show how hard it is to reason about code behaviour in
the context of dynamic class loading and I/O.
Building over type modifiers and object capabilities, we lay down the foundation needed to
reason about OO code in this context. Without such reasoning
enforcing class-invariants for all objects involved in execution becomes nearly impossible.
We show, that in comparison to other approaches, type-modifiers and object capabilities allow for such reasoning to be both simpler and sound in a broader scope.

\loseSpace
\noindent\textit{Grounding:} %What argument, feasibility proof, artifacts, or results and evaluation support this work?
We demonstrate, by means of a case study, empirical evidence of the performance and compactness of our approach in comparison to Spec\# and the conventional runtime-verification methodology.

We formally model a class of languages that soundly enforce class-invariants. This model is formally verified by a proof, parametric on the presence of a type system which guarantees certain properties of type modifiers. Type systems with such properties have already been explored, formalized, and proved in prior work.

\loseSpace
\noindent\textit{Importance:} %Why does this work matter?
Class-invariants allow programmers to reason about what states objects can be in. Our approach
makes such reasoning both sound and strong: if a programmer has access to an object, it and 
all objects reachable from it are in a valid state.
This is of great practical importance since it is independent of the presence of buggy code, or even dynamically loaded malicious code.


%TODO
% \REVComm{Our work represents a philosophical shift}{2}{This makes it sound like run-time verification has never been considered before.} similar to the jump from unchecked casts in C to checked casts in Java: it is the programmer's responsibility to create valid objects and to preserve validation
% while mutating objects, however a validation failure is soundly detected by a run-time exception,
% and even after capturing such an exception, validation still holds for all objects involved in the execution.
%Class invariants provide guarantees about the state of objects throughout the execution.
%Runtime verification of class invariants is
%a hard problem due to issues with aliasing, exceptions,
%non-deterministic invariants, I/O, subtyping and untrusted code.
%We challenge this problem in the context of
%a Java-like language where the invariants are expressed in the language itself.
%We formally define \textbf{Sound Invariant Checking}
%and formally prove that a combination of carefully selected type modifiers, object capabilities,
% and strong exception safety is sufficient
%to handle Sound Invariant Checking for the most common categories of objects.

\end{abstract}

\section{Introduction}
\label{s:intro}
%\newpage
%\LINE

Object oriented programming languages through subtyping and dynamic dispatch provide great flexibility: they
allow code to be adapted \IO{and} specialised to behave differently in different contexts.
%%, which is made even more complex by dynamic class loading (supported by many mainstream OO languages).
However this flexibility hampers code \IO{reasoning,} since object behaviour is usually nearly completely
unrestricted. This is further complicated with the support OO languages typically have for exceptions,
memory mutation, and I/O.
% Class invariants are a well known technique to help write correct code, however there are various different interpretations of when they should hold.
% invariant protocols, specifying when the invariant is expected to hold and when is checked. 
%% In the absence of on the fly static verification of dynamically loaded code, it is difficult for programmers to write code that is correct in a library setting.


Class invariants are an important concept when reasoning about software correctness.
They can be presented as documentation, checked as part of static verification, or, as we do in this paper, monitored for violations using runtime verification.
In our system, a class specifies its invariant by defining a boolean method called \Q@invariant@.
We say that an object's invariant holds when its \Q@invariant@ method would return \Q@true@.
%\IODel{We say that an object is \REV{\emph{valid}}{3}{
%If this term really is necessary (I'm not sure it is), then perhaps it should be used more regularly.}
%if calling \Q@invariant@ would return \Q@true@.}

An \emph{invariant protocol}~\cite{FlexibleInvariants} specifies when invariants need to be checked, and when they can be assumed; if such checks guarantee said assumptions, the protocol is sound.
The two main sound invariant protocols present in literature are \emph{visible state semantic} \cite{Meyer:1988:OSC:534929} and the \emph{Boogie/Pack-Unpack methodology}~\cite{DBLP:journals/jot/BarnettDFLS04}.\IO{The} visible state semantics expect \IO{the invariants of receivers} to hold before and after every public method call, and after constructors. Invariants are simply checked at all such points, \IO{thus} this approach is obviously sound; however this can be incredibly inefficient, even in simple cases.
In contrast, the \IO{pack/unpack} methodology marks all objects as either \emph{packed} or \emph{unpacked}, where a packed object is one whose invariant is expected to hold.
In this approach, an object's invariant is checked only by the pack operation.
In order for this to be sound, some form of aliasing and/or mutation control is necessary. For example, Spec\#, which follows the \IO{pack/unpack} methodology, uses a theorem prover, together with source code annotations.
While Spec\# can be used for full static verification, it conveniently allows invariant checks to be performed
at \IO{runtime}, 
\IO{whilst} statically verifying aliasing, purity and other similar standard properties.
This allows us to closely compare our approach with Spec\#.

Instead of using automated theorem proving, 
it is becoming more popular to verify aliasing and immutability using a type system.
For example, there are 3 languages: L42~\cite{ServettoZucca15,ServettoEtAl13a,JOT:issue_2011_01/article1,GianniniEtAl16}, Pony~\cite{clebsch2015deny,clebsch2017orca}, and the language of Gordon et.~al.~\cite{GordonEtAl12} that use Type Modifiers (TMs) and Object Capabilities (OCs) to ensure safe and deterministic parallelism.%
\footnote{TMs are called \emph{reference capabilities} in other works. We use the term TM here
to not confuse them with object capabilities, another technique we also use in this paper.}
While studying those languages, we discovered an elegant way to enforce invariants.


\subheading{Example}
We now show a simple example, where we have a \Q@Cage@ class with a list of \IO{\Q@Point@s, \Q@path@,} and \Q@Hamster@ with a \IO{\Q@Point@,} \Q@pos@\IO{. The} \Q@Cage@ \IO{will} move the \Q@Hamster@ across its path\IO{, it} has an invariant: the \Q@Hamster@'s location is within the \Q@Cage@'s path.

% While Spec\# requires specialised \Q@Point@, \Q@Hamster@, and \Q@Cage@ declarations to be able to enforce the invariant, our version manages to capture the required information in just a few annotations on \Q@Cage@ and leaves \Q@Point@ and \Q@Hamster@ unmodified.

%	if(that==null || !(that instanceof Point)){return false;}
% 	return ((Point)that).x==this.x && ((Point)that).y==this.y; 
%  }
\begin{lstlisting}
class Point { Double x; Double y; Point(Double x, Double y) {..}
  @Override read method Bool equals(read Object that) {
    return that instanceof Point &&
      this.x == ((Point)that).x && this.y == ((Point)that).y; }
}
class Hamster {Point pos; //pos is imm by default
  Hamster(Point pos) {..} 
}
class Cage {
  capsule Hamster h;
  List<Point> path; //path is imm by default
  Cage(capsule Hamster h, List<Point> path) {..}
  read method Bool invariant() {
    return this.path.contains(this.h.pos); }
  mut method Void move() {
    Int index = 1 + this.path.indexOf(this.h.pos));
    this.moveTo(this.path.get(index % this.path.size())); }
  mut method Void moveTo(Point p) { this.h.pos = p; }
}
\end{lstlisting}
Many verification approaches take advantage of the separation between \IO{primitive/value} types and objects, since the former are immutable and do not support reference equality.
However, our approach works in a pure OO setting without such a distinction. Hence we write all type names in \Q@BoldTitleCase@ to underline this. Note: to save space, here and in the rest of the paper we omit the bodies of constructors that simply initialise fields with the values of constructor parameters\IO{, but we show their signature in order to show any annotations.}

We use the \Q@read@ annotation on \Q@equals@ to express that it does not modify either the
receiver or the parameter. In \Q@Cage@ we use 
\REV{\Q@capsule@}{2}{???} to ensure
that the \Q@Hamster@'s \emph{reachable object graph} (ROG) is fully under the control
of the containing \Q@Cage@. 
We annotated the \Q@move@
and \Q@moveTo@ methods with \Q@mut@, since they modify
their receivers ROG. The default annotation is always \Q@imm@, thus \Q@Cage@'s \Q@path@ field is a deeply immutable list of \Q@Point@s.
% Note how we just use \Q@List.contains()@ and \Q@List.indexOf()@
% to check if the hamster position is inside the list.
% The conventional syntax correctly instantiates a \Q@Cage@:
% \Q@new Cage(new Hamster(new Point(..)), List.of(new Point(...))@.
Our system performs \IO{runtime} checks for the invariant
at the end of \Q@Cage@'s constructor, \Q@moveTo@ method, and after any update to one of its fields.
The \Q@moveTo@ method is the only place in the above code where the invariant may be broken.  \IOComm{This is not true} However, there is only a single occurrence of \Q@this@ and it is used to read the \Q@h@ field. We \IO{use the guarantees of} TMs to ensure that no alias to \Q@this@ could be reachable from either \Q@h@ or the immutable \Q@Point@ parameter. Thus, the potentially broken \Q@this@ object is not visible while the \Q@Hamster@'s position is updated. 
The invariant is checked at the end of the \Q@moveTo@ method, just before \Q@this@ would become visible again.
This technique loosely corresponds to an implicit pack and unpack: we use \Q@this@ only to read the field value, then we work on its value while the invariant of \Q@this@ is not known to \IO{hold,} finally we check the invariant before allowing the object to  be used again.

Note: since only \Q@Cage@ has an invariant,
 only \Q@Cage@ has special restrictions, allowing the code for \Q@Point@ and \Q@Hamster@ to be unremarkable.
 This is not the case in Spec\#: all code involved in  verification needs to be designed with verification in mind~\cite{barnett2011specification}.
% The best solution we found was to define our own equality for \Q@Point@ instead of relying on \Q@Object.Equals@,
% thus we could not use \Q@List.Contains@ and \Q@List.IndexOf@.

\subheading{Spec\# example} Here we show the previous example in \IO{Spec\#,} the system most similar to ours (see appendix \ref{s:hamster} for a more detailed discussion about this solution):
%or \small or \footnotesize etc.
\begin{lstlisting}[
language={[Sharp]C}, morekeywords={invariant,ensures,requires,expose,exists}]
// Note: assume everything is `public'
class Point { double x; double y; Point(double x, double y) {..}
  [Pure] bool Equal(double x, double y) {
    return x == this.x && y == this.y; }
}
class Hamster{[Peer]Point pos; 
  Hamster([Captured]Point pos){..}
}
class Cage {
  [Rep] Hamster h; [Rep, ElementsRep] List<Point> path;
  Cage([Captured] Hamster h, [Captured] List<Point> path)
    requires Owner.Same(Owner.ElementProxy(path), path); {
      this.h = h; this.path = path; base(); }
  invariant exists {int i in (0 : this.path.Count);
    this.path[i].Equal(this.h.pos.x, this.h.pos.y) };
  void Move() {
    int i = 0;
    while(i<path.Count && !path[i].Equal(h.pos.x,h.pos.y)){i++;}
    expose(this) {this.h.pos = this.path[i%this.path.Count];}}
}
\end{lstlisting}
In both versions, we designed \Q@Point@ and \Q@Hamster@ in a general way, and not solely to be used by classes with an invariant, in particular \Q@Point@ is not an immutable class. However, doing this in Spec\# proved difficult, in particular we were unable to override \Q@Object.Equals@, or even define a usable \Q@equals@ method that takes a \Q@Point@, as such we could not call either \Q@List<Point>.Contains@ or \Q@List<Point>.IndexOf@.
 
Even with all the above annotations, we still needed special care in creating \Q@Cage@s:
\begin{lstlisting}[
%basicstyle=\footnotesize,
language={[Sharp]C}, morekeywords={invariant,ensures,requires,expose,exists}]
List<Point> pl = new List<Point>{new Point(0,0),new Point(0,1)};
Owner.AssignSame(pl, Owner.ElementProxy(pl));
Cage c = new Cage(new Hamster(new Point(0, 0)), pl);
\end{lstlisting}
\IOComm{Delete grey box above?\\}%
Whereas with our system we can simply write:
\begin{lstlisting}
List<Point> pl = List.of(new Point(0, 0), new Point(0, 1));
Cage c = new Cage(new Hamster(new Point(0, 0)), pl);
\end{lstlisting}

%3 read 2 capsule 3 mut extra method moveTo
%----
In Spec\# we had to add $10$ different annotations, of $8$ different kinds; some of which were quite involved. In comparison, our approach requires only $7$ simple keywords, of $3$ different \IO{kinds;} \IO{however we did need to write our code differently by creating} a separate \Q@moveTo@ method, since we do not want to burden our language with extra constructs such as Spec\#'s \Q@expose@.
%  Moreover we had been unable to reuse 
% \Q@Object.Equals@, \Q@List.IndexOf@ and % \Q@List.Contains@.
% Note: we had to add a new class \Q@PureObject@, since the \Q@Objec@ constructor is not annotated as \Q@[Pure]@.
%3 pure,
%1 peer
%3 captured
%2 rep
%1 ElementsRep
%1 requires Owner.Same(Owner.ElementProxy(path), path);
%1 invariant
%1 exists
%expose(this)
%re implementation of indexOf
%dumb equals(double,double)
%dumb class PureObject { [Pure] PureObject() { } }
%Owner.AssignSame(pl, Owner.ElementProxy(pl));
% manually handle ownership details while instantiating a \Q@new Cage(..)@.
% Note how the \Q@expose@ block cover plays the same role of our \Q@moveTo@ method.
%\begin{lstlisting}
%class SafeMovable implements Widget{
%  @Override read method Widgets children(){return %this.box.cs;}
%  @Override read method Int left(){return this.box.l;}
%  @Override read method Int top(){return this.box.t;}
%  @Override read method Int width(){return this.w;}
%  @Override read method Int height(){return this.h;}
%  capsule Box box;
%  Int w; Int h;
%  read method Bool invariant(){//iterate on box.cs, check:
%    //not overlap with each other, are inside the widget bounds
%  }
%  SafeMovable(Int w,Int h,capsule List<Widget> cs) {
%    this.w=w; this.h=h; this.box=boxWithButton(cs);}
%  static method capsule Box boxWithButton(capsule Widgets cs){
%    mut Box b=new Box(5,5,cs);
%    b.cs.add(new Button(0,0,10,10,new MoveAction(b));
%    return b;//b is declare mut, but it is soundly returned capsule
%  }}
%\end{lstlisting}

%\begin{lstlisting}
%class Box{
%  Int l; Int t; mut List<Widget> cs;
%  Box(Int l, Int t, mut List<Widget> cs){...} }
%
%class MoveAction implements Action{
%  mut Box outer; MoveAction(mut Box outer){this.outer=outer;}
%  mut method Void process(Event event) {this.outer.l+=1;} }
%\end{lstlisting}

%Ideally, we would like the invariant to be dynamically checked 
%at the end of the constructor and at the end of the \Q@move@ method.
%However, this would be unsound without some form of aliasing control over \Q@Hamster@,
%the \Q@List@ and all the \Q@Point@s, as shown in the following example\footnote{
%The visible state semantic prevent \Q@toString()@ from producing a non nonsensical result
%by checking the invariant and throwing an invariant error.
%}
%\begin{lstlisting}
%List<Point>ps=Arrays.asList(new Point(2,3),new Point(4,5));
%Cage c=new Cage(new Hamster(new Point(2,3),ps);
%//invariant holds here
%ps.get(0).x=8;//invariant is broken here, since ps
%//was accessible from outside the cage
%c.toString()//the hamster is in a position that is not on the list
%c.invariant();//return false!!
%\end{lstlisting}

%Moreover, this is unsound also if we can not ensure determinism of the invariant method;
%for example we could have an \Q@EvilList@

%\begin{lstlisting}
%class EvilList<T> extends ArrayList<T>{..
%  @Override boolean contains(T elem){return new Random().bool();}
%}
%..
%List<Point>ps=Arrays.asList(new Point(2,3),new Point(4,5));
%Cage c=new Cage(new Hamster(new Point(2,3),ps);
%//invariant happens to holds at the end of the constructor by chance,
%c.invariant();//here instead it return false!!
%\end{lstlisting}

%Despite the code for \Q@Cage.invariant()@ intuitively looking correct and deterministic, the above calls to it are not. Obviously this breaks any reasoning and should be considered unsound. 
%In particular, note how in the presence of dynamic class loading, 
%we can not make any assumption on the dynamic type of \Q@path@.
%A:TrustInvariant<><{ int num}
%B:Invariant<><{A a}


%\begin{lstlisting}
%class Point {Int x; Int y;
%  /*... constructor, equals and other obvious utility methods*/}
%class Hamster { Point pos; Hamster(Point pos){this.pos=pos;} }
%class Cage {
%  capsule Hamster h;  List<Point> path;
%  Cage(capsule Hamster h, List<Point> path){ this.h=h; this.path=path; }
%  read method Bool invariant(){
%    return this.path.contains(this.h.pos);
%  }
%  mut method Void move() {
%    Int index=1+this.path.indexOf(this.h.pos);
%    if (index>=this.path.size()) {index=0;}
%    this.moveTo(this.path.get(index));
%  }
%  mut method Void moveTo(Point p){  this.h.pos=p }//new method
%  @Override method String toString(){
%    return "hPos:"+this.h.pos+", path:"+this.path;
%  }
%}
%\end{lstlisting}

%\begin{lstlisting}
%Point:Data<><{var Num x, var Num y}
%Points:Collections.vector(ofMut:Point)
%Hamster:Data<><{var Point pos}
%Cage:Data<><{
%  capsule Hamster h
%  Points path
%  read method Bool #invariant()
%    this.path().contains(this.h().pos())
%  mut method Void move() (
%    var Size index=1Size+this.path().indexOfLeft(val:this.h().pos())
%    if index>=this.path().size() (index:=0Size)
%    this.move(newPos:this.path().val(index))
%    )
%  mut method Void move(Point newPos)
%    this.#h().pos(newPos)
%  }
%\end{lstlisting}
%In L42 we only need to add 
%1 \Q@read@, 2 \Q@mut@ and 1 \Q@capsule@ annotations.
%We also need to add a new method \Q@moveTo@. This is equivalent to the explicit \Q@expose(this){..}@
%block required in Spec\#.

%To keep the syntax familiar, we present our code example in a tweaked Java syntax using type modifiers.
%If a method override an interface method, we inherit the modifiers from the interface.
%Any non annotated type is implicitly immutable.
%Note how we just added 1 \Q@read@, 2 \Q@mut@ and 2 \Q@capsule@ annotations; in L42 constructors can be automatically generated, this would remove the need for 1 of the\Q@capsule@ annotations.
%We also need to add a new method \Q@moveTo@. This is equivalent to the explicit \Q@expose(this){..}@
%block required in Spec\#.

%In L42 any non annotated type is implicitly immutable.
%We also need to add a new method \Q@moveTo@. This is equivalent to the explicit \Q@expose(this){..}@
%block required in Spec\#.
%In this paper we will show how those minor code modifications are sufficient
%to statically verify that runtime verification is needed only
%after the constructor and after the \Q@moveTo@ method.
%We will also show how designing the standard library in OCs style
%ensures that any \Q@read@ method with no parameters (as \Q@invariant()@) is
%deterministic.
%We obtain our results thanks to coarse grained type system support and a careful design of the standard library, where all the possible sources of non determinism follow the OCs style.
%---------------------

%We evaluate our contribution by means of case studies;
\subheading{\IO{Summary}}
We have fully implemented our protocol in L42\footnote{An experimental version of L42 supporting the protocol described in this paper, together with the full code of our case studies is available at \url{http://l42.is/ProgrammingArtifact.zip} \IOComm{Change the URL?}.}\footnote{We also believe it would be easy to implement our protocol in Pony and Gordon et. al.'s language.}, we used this implementation to implement and test an interactive GUI involving a class with an invariant. On a test case with $5$ objects with an invariant, 
our protocol \IO{performed} only $77$ invariant checks, whereas visible state semantics performs over $53$ million checks \IOComm{Eiffel results?}. We also compared with Spec\#, whose invariant protocol \IO{performs the same number of checks} as ours, however the annotation burden was almost $4$ times higher \IODel{than our approach}.

In this paper we argue that our protocol is not only more succinct than the \IO{pack/unpack} approach, but is also easier and safer to use.
Moreover, our approach deals with more scenarios than most prior work: we allow sound catching of invariant failures\IODel{,} and also carefully handle non deterministic operations like I/O.
%In our case study we show that
%we can still encode most of the examples explored in ~\cite{???} (including for example mutable collections of immutable objects) whilst having a significantly lower annotation-burden.
Section \ref{s:TMsAndOCs} explains the \emph{type modifier} and \emph{object capability} support we use for this work.
Section \ref{s:protocol} \IO{explains} the details of our invariant protocol\IO{, and section \ref{s:formalism} formalises a language enforcing this protocol}.
Sections \ref{s:immutable} and \ref{s:encapsulated}\IO{, respectively, explain} and \IO{motivate} how our protocol can handle invariants over immutable and encapsulated \IO{data.}
Section \ref{s:case-study} presents our GUI case study and compares it against visible state semantics and Spec\#.
Sections \ref{s:related} and \ref{s:conclusion} provide related work and conclusions.

\IO{Appendix \ref{s:proof} provides a proof that our invariant protocol is sound. Sections \ref{s:hamster} and \ref{s:MoreCaseStudies} provide further detailed comparisons against Spec\#.}

%http://www.cs.cmu.edu/~NatProg/papers/p496-coblenz-Glacier-ICSE-2017.pdf



\section{Background}
\label{s:background}
\noindent\textit{Type Modifiers:}
Type modifiers are a well known language mechanism~\cite{TschantzErnst05,BirkaErnst04,OstlundEtAl08,clebsch2015deny,GianniniEtAl16,GordonEtAl12} allowing static verification of mutability and aliasing properties of objects.
With sightly different names and semantics, the three most common modifiers for object references are:
\begin{itemize}
\item Mutable (\Q@mut@): the referenced object can be mutated, as in most languages without modifiers.
\item Readonly (\Q@read@): the referenced object cannot be mutated by such reference, but in the program there may be mutable references to this same object, so mutation can still be observed. 
\item Immutable (\Q@imm@): the referenced object can never mutate. Like \Q@read@ references, one cannot mutate through an \Q@imm@ reference, however \Q@imm@ references also guarantees that the referenced object will never be mutated, not even through another reference.
\end{itemize}
%In the context of object-oriented programming, type modifiers may also apply to the implicit \Q@this@ parameter in method declarations, restricting the type of references the method can be called on. In addition, due to the deep meanings we type field access on object references to be the most restrictive of the object references modifier and the field’s. As \Q@read@ references impose no assumptions about aliasing, any \Q@imm@ or \Q@mut@ expression can be safely implicitly promoted to \Q@read@, whereas other conversions are not generally safe.
%\loseSpace
TM are different to field or variable modifiers like Java’s \Q@final@: TM applies to references,  \Q@final@ specifies what can be done to the field itself. In comparison to \Q@imm@:

\begin{itemize}
\item A \Q@final@ variable/field cannot be reassigned, it always refers to the same object; however, the referenced object itself may be mutated.
\item A reference of an \Q@imm@ type however refers to an object that will never be mutated, and neither will its ROG. However, a field of type \Q@imm@ may be updated to another \Q@imm@ reference.
\footnote{In C, this is similar to the difference between \Q@const *A@ and \Q@*const A@, where a \Q@final imm@ variable would be like \Q@const *const A@.}
\end{itemize}



\noindent Consider the following  example usage of \Q@mut@, \Q@imm@ and \Q@read@:
\begin{lstlisting}
mut Point mp = new Point(1, 2);
mp.x = 3; // ok
imm Point ip = new Point(1, 2);
$\Comment{}$ip.x = 3; // type error
read Point rp = mp; // ok read is a common supertype of imm/mut
$\Comment{}$rp.x = 3; // type error
mp.x = 5; // ok, and now we can observe rp.x == 5
ip = new Point(3, 5); // ok, ip is not final
\end{lstlisting}
\noindent We cannot use a \Q@read@ reference to cause mutation, but we have no guarantee of the absence of mutation; in our example we can observe a change in \Q@rp@ caused by a mutation inside \Q@mp@.


There are several possible interpretations of the semantics of type modifiers.
Here we assume the full/deep meaning:
\begin{itemize}
  \item all the objects in the ROG of an immutable object are immutable;
  this corresponds to UML DataTypes,
  \item a mutable field accessed from a \Q@read@ reference produce a \Q@read@ reference,
  \item no \emph{down}-casting is allowed between different type modifiers.
\end{itemize}


\noindent There are many different existing techniques and type systems that handle the modifiers above~\cite{ZibinEtAl10,ClarkeWrigstad03,HallerOdersky10,GordonEtAl12,ServettoZucca15}.
The main progress in the last couple of years is with the flexibility of such type systems: where the programmer should use \Q@imm@ to represent objects that would obviously be modelled as UML DataTypes, and \Q@mut@ nearly everywhere else; the system will be able to transparently promote/recover~\cite{GordonEtAl12,clebsch2015deny,ServettoZucca15} the type modifiers, adapting them to their use context.
To see a glimpse of this flexibility, consider the following example:
\saveSpace
\begin{lstlisting}
mut Point mCenter = new Point(1, 2);
mut Circle mc = new Circle(mCenter, /*radius*/7);
mc.radius = 3; // ok
imm Circle ic = new Circle(new Point(0, 0), 7); // ok imm
imm Point ip = ic.center; // ok, ROG immutable
read Circle rc = mc
read Point rp = rc.center; // ok, fields of read Circle are read
$\Comment{}$mut Point mp = rc.center; // type error
\end{lstlisting}
\saveSpace

Here \Q@imm Circle ic@ and \Q@mut Circle mc@ are both initialized with \Q@new Circle(...)@.
This is not a special feature of \Q@new@ expressions: since \Q@new@ returns a \Q@mut@ and any expression typed as \Q@mut@ that only uses immutable variables can safely be promoted to \Q@imm@.
% (since the returned value could not possibly be aliased).
%FALSE: it can be internally aliased!
Such flexibility is also visible where \Q@rc.center@ returns a \Q@read@ but \Q@ic.center@ returns an \Q@imm@: any expression typed as read that only
uses immutable variables can safely be promoted to \Q@imm@.

 %(since \Q@ic@ is \Q@imm@, and \Q@imm@ is a deep modifier).
%true fact but not sufficient?

%With this kind of type system, we can ensure immutable classes by just declaring all their fields as final and immutable.%
%\footnote{
%In Java,  to ensure a class is immutable we need:
%the class must be final, all the fields must be final of immutable
%classes (thus no interface fields, final classes all the way down),
%and the SecurityManager need to properly tame reflection.}

% Not sure about this paragraph:



\loseSpace

TM are very useful in restricting the scope of mutation. 
Any expression that does not use any \Q@mut@ 
variable declared outside of such expression does not modify objects visible outside. This consideration is particularly useful to understand code in the presence of exceptions. Other authors have identified the concept of Strong Exception Safety~\cite{Abrahams2000} as a general issue when reasoning about objects state after catching an exception:
when a \Q@try-catch@ catches an exception, the visible objects must be the same as before the \Q@try@ block started its execution.
This can be obtained leveraging TM in the following way:
\begin{itemize}
\item all thrown exceptions are immutable objects,
\item code inside a \Q@try@ block is typed as if all \Q@mut@ variables declared outside of the block are \Q@read@.
\end{itemize}

\loseSpace
\noindent\textit{Object Capabilities:}
While type modifiers are statically verified properties of references, object capabilities are run time characteristics of specific objects.

Conceptually, an object capability is a communicable, unforgeable token of authority, a key to access special functionality: only certain objects with `special' powers can do `special' actions, and those objects are obtained in a controlled way. We call such objects `capability objects'.
Their main use case is to allow for fine-grained control over what sections of code are allowed to do. For example, in Java \Q@System.in@ is a capability object (it has the capability to read input); however it is globally accessible: thus any code could use it, preventing reasoning about determinism.
In a language enforcing object capabilities the \Q@main@ method could take a \Q@System@ object as a parameter, and using that object is the only way to perform I/O, as in \Q@mySystem.println("hello")@.
Moreover, the \Q@System@ class would have no visible constructor, and all its I/O methods would require a mutable (\Q@mut@) receiver.
Many other operations, like random numbers and file management 
%may just take a \Q@mut System@ object as a parameter.
could also work this way.
\noindent This design has been explored in literature by Joe-E~\cite{finifter2008verifiable}.

Here we use TM to guarantee that any method that is not (indirectly) passed a \Q@mut@ reference to a capability object will not use any capabilities:
\begin{itemize}
\item all capability-methods must require a \Q@mut@ receiver,
\item there are no global variables\footnote{Note: even just allowing \Q@imm@
global variables would prevent reasoning over determinism due to the possibility of global variable updates; however constant/final globals of an \Q@imm@ type would not cause such problems.},
\item user code cannot directly create a capability object: they can only indirectly do so through an existing \Q@mut@ capability object reference

% NOTE: SOMEWHERE MAKE IT CLEAR THAT NON-DETERMINISM CAN ONLY OCCUR THROUGH A CAPABILITY OBJECT
\end{itemize}

%-----------------------------------------------------------


%define simple objects
%show solution  for simple person: requires 3 properties

%show solution is sound --> proof in appendix
%naive is unsound - person 3 bugs


\section{The \Q@.validate()@ Method}
\label{s:validate}
Thanks to TM and OC, we can now express the signature for \Q@.validate()@ method as follows:
\saveSpace
\begin{lstlisting}
read method imm Bool validate();
\end{lstlisting}
\saveSpace
%The method is \Q@read@: thus the method body will see the \Q@this@ object as a \Q@read@ reference; and has no other parameters. 
%By starting from a \Q@read@ reference and nothing else, we are guaranteed that the method is pure:
If the class containing the validate method has a super-class, we would automatically check \Q@super.validate()@ at the beginning of the sub-class’s \Q@.validate()@ method, this is required as otherwise `invalid' objects could easily be created by simply overriding \Q@.validate()@.
As this method is declared as \Q@read@, and only takes the implicit parameter \Q@this@ (as \Q@read@), we can guarantee the method is pure:
\begin{itemize}
\item the ROG from \Q@this@ is only accessed as \Q@read@ (or \Q@imm@), thus it cannot be mutated\footnote{
This can even be safe in a multi-thread environment: TM are often used to ensure correct parallelism; for example threads may be required to not share \Q@mut@ data, thus a \Q@read@ reference could only be mutated by a \Q@mut@ reference under the control of the same thread.
},
\item if a capability object (such as \Q@System@) is in the ROG of \Q@this@, then it can only be accessed as \Q@read@, preventing use of its capability (such as I/O),
\item nothing else is accessible (we do not have global variables).
\end{itemize}

\noindent Also note \Q@.validate()@ is not declared as throwing any exceptions, thus it can only leak unchecked exceptions.


Clearly the \Q@.validate()@ method must be able to take an invalid \Q@this@, since the purpose of such method is to distinguish valid and invalid objects.\footnote{
At a first look this may seam an open contradiction
with the aim of this work, however only calls to validate inserted by the language semantic can take an invalid \Q@this@. As for any other method, when the application code calls \Q@.validate()@,
\Q@this@ is guaranteed to be valid.
} However, if we allow the method to use \Q@this@ directly (e.g. storing it in a local variable or passing it to a method), we would break the guarantees of validation (namely: `invalid objects are not reachable by application code'); as such we enforce the simple restriction that \Q@.validate()@ may only use \Q@this@ to access fields.
As a relaxation, we could allow calling instance methods that in turn only use \Q@this@ to access fields, or call other such instance methods. With this relaxation, the semantics of \Q@.validate()@ need to be understood with the body of those methods inlined; thus the semantic of the inlined code need to be logically reinterpreted in the context of \Q@.validate()@, where \Q@this@ may be invalid.
In some sense, those inlined methods and field accesses can be thought as macro-expanded.


Finally,
the code of \Q@.validate()@ can not access  \Q@mut@ fields, because their content can be changed by unrelated code.
With the modifiers presented so far, \Q@.validate()@ can only access \Q@imm@ fields.
We will later introduce a \Q@capsule@ modifier to allow more flexible validation.


% JUSTIFY that the fields are valid...

%validable objects are not circular (do not belong in their ROG of any of its fields)
%validate as a predicate on object fields never really see this.
%
%clarifications:
%a validate check is never needed/generated/injected when working on a read x
%multi threading is not relevant/supported
%validable objects are not circular (do not belong in their ROG of any of its fields)

\section{Validating immutable state}
\label{s:immState}
In this section we consider validation over fields of \Q@imm@ types.\footnote{
In a real language, for conciseness one could make the \Q@imm@ modifier the default, allowing it to be omitted and our \Q@Person@ example class would only use 3 type modifiers; however we explicitly use it here for clarity.
}
In the next section we expand our technique.

In the following code \Q@Person@ has a single immutable (non final) field \Q@name@:
\begin{lstlisting}
class Person {
  read method imm Bool validate() {return !name.isEmpty();}
  private imm String name;
  read method imm String name() {return this.name;}
  mut method imm Strinig name(imm String name) {this.name = name;}
  Person(imm String name) {this.name = name;} }
\end{lstlisting}
\Q@Person@, only has immutable fields and the constructor only uses \Q@this@ to access/update fields.%; we say such a class is \emph{simple}.
%\Q@Person@, only has immutable fields and the constructor 
%uses the parameters to directly initialize (all) the fields.
% We say such a class is \emph{simple}.%
%\footnote{
%We consider only standard contractors for simplicity of exposition.
%More complex constructors could be supported, provided that \Q@this@ is only used to access fields, we do discuss them for simplicity.}
The difference with respect to UML DataTypes 
%immutable types (like UML DataTypes)
%UML datatypes are aclass property. immutable types are often an instance one (so no final fields) 
 is that fields are not required to be final, thus the object can change state during its lifetime. This means that the ROGs of all the \emph{fields} are immutable, but the object itself may be mutable.
%Of course UML DataTypes
%immutable types
%No, a type is not a class
% are just a special case of simple classes.
Validation for such a class can easily be enforced by generating checks on the result of \Q@.validate()@, immediately after each field update, and at the end of the constructor%
%\footnote{Since the constructor only initialises fields; as with the \Q@.validate()@ method itself, we allow field uses since \Q@this@ is not directly reached.}
%would require the initial/default value of \Q@this@ to be valid.}
:
% If a simple class provides a \Q@.validate()@ method, then validation will be enforced.
% For \Q@Person@, intuitively, the code would behave as follow:

%\Comment{if we made this public, all users who update the field need to call validate}%
%There are many interpretations for your comment
%why you deleted my code comments?
\begin{lstlisting}
class Person {
  read method imm Bool validate() {return !name.isEmpty();}
  private imm String name;
  read method imm String name() {return this.name;}
  mut method imm String name(imm String name) {
    this.name = name; // check after field update
    if (!this.validate()) {throw new Error(...);} 
  }
  Person(imm String name) {
    this.name = name; // check at end of constructor
    if (!this.validate()) {throw new Error(...);}
}}// Generated code, not directly written by the programmer
\end{lstlisting}
%... $\MComment{validation error}$ 

% Many programmers attempted to write similar code in mainstream languages like Java to ensure  that some property always holds. Indeed, at first look, this code seems to correctly enforce validation. Sadly, without relying on TM and OC, the former code would be broken: just making the fields private and checking the \Q@.validate()@ method at the \textbf{end of the constructor} and at the \textbf{end of mutator methods} is not enough to enforce validation.
% The trick is that our intuition relies not on statically verified properties, or on the semantics of the language, but on the expectations about `correct' behaviour of \Q@String@. We need to enforce Validation without assuming the behaviour of other objects.

\noindent If we were to relax (as in Rust) or even eliminate (as in Java) the support for TM or OC, the validation of the \Q@Person@ class would become harder, or even impossible. We now proceed to show examples where
relaxation of TM or OC breaks our validation. 

\loseSpace
\noindent\textit{Unrestricted use of object-capabilities:}
Allowing \Q@.validate()@ to (indirectly) use an object capability could allow for it to be non deterministic, by either:
\begin{itemize}
\item allowing \Q@.validate()@ to (indirectly) access a \Q@mut@ reference to a capability-object,
\item relaxing the rule that capability-methods must have a \Q@mut@ receiver, or
\item allowing capability objects to be created out of thin air.
\end{itemize}

\noindent For example consider this simple and contrived (mis)use of person:
\begin{lstlisting}
class EvilString extends String {
  @Override read method Bool isEmpty() {
    // Creates a new capability object out of thin air
    return new Random().bool();
} }
...
imm method mut Person createPersons(imm String name){
  // we can not be sure wether name is an `EvilString'
  mut Person schrodinger1 = new Person(name); // exception here?
  mut Person schrodinger2 = new Person(name); // what about here?
  ...}
\end{lstlisting}
Despite the code for \Q@Persion.validate()@ intuitively looking correct and deterministic, the above calls to it are not. Obviously this breaks any reasoning and makes our validation unsound. 
In particular, note how in the presence of dynamic class loading, 
we can not make any assumption on the dynamic type of \Q@name@.
%???
%Even if we disallow subtyping the same problem could still occur if we had a strange implementing of \Q@String@, or \Q@Persion.validate()@ itself.

\loseSpace
\noindent\textit{Allowing internal mutations/back-doors:}
% TODO: Come up with better title
Suppose we relax our mutation rules, by allowing interior mutability
as in Rust and Javari: thus allowing mutation of the ROG of an immutable object through back-doors:
\begin{lstlisting}
class AtomicBool {
  imm method imm Void store(imm Bool val){
    ... // Mutate through an imm reciever
  }
}

class NastyString extends String {
  imm AtomicBool evil = new AtomicBool(false);
  imm method imm Void nasty() {
    this.evil.store(true); // this imm method can do mutation
  }

  @Override read method Bool isEmpty() {
    return this.evil.load() ? false : super.isEmpty();
  }
  ...
}
...
imm NastyString name = new NastyString("bob");
imm Person person = new Person(name); // person is valid
name.nasty(); // mutate the ROG of person, without it noticing
// person is now invalid!
\end{lstlisting}

In this example we use \Q@AtomicBool@ as a back-door to remotely break the invariant of \Q@person@ without any interaction with the \Q@person@ object itself.
%mine: yes, too strong: For validation we need the language to guarantee true deep immutability.
%your: just points outside: It would require some powerful static or dynamic analysis to keep track of every case the ROG of \Q@Person@ could be indirectly mutated, and insert validity checks appropriately, however ensuring deep mutability trivialises this for simple classes.


Allowing such back-doors could also be used to break the determinism of the \Q@.validate()@ method: they may store and read information about previous calls.

%In our simple example, \Q@Person@ objects can be mutated using the setter, and exposed using the getter.
%We may consider the getter to be safe since in modern languages we expect strings to be immutable objects.
%\footnote{While we can update the field \Q@name@ to point to another string, we cannot mutate the string object itself.
%To obtain  \Q@"Hello"@ from \Q@"hello"@ we need to create a whole new string object that looks like the old one except for the first character. This would be different in older languages like C, where strings are just mutable arrays of characters.}
%
%Again, the assumption that they are immutable depends on the correctness of the code inside \Q@String@: if there was a bug in the \Q@String@ class, or any \Q@String@ subclass, then executing 
%\Q@println(bob.name())@ may change \Q@bob@ by quietly changing a part of its ROG.
%Again, checking
%what methods mutate states cannot be responsibility of the \Q@Person@ programmer.
%For Validation we need a language supporting aliasing and mutability control.
%\begin{comment}
%\item Sample Bug 1:
%Suppose there was a bug in \Q@String.isEmpty()@, causing the method to non-deterministically return \Q@true@ or \Q@false@.
%What would it mean for Validation?
%Would a \Q@Person@ be at the same time 
%valid and invalid?
%
%Only deterministic methods can be used for validation.
%Ensuring this cannot be responsibility of the \Q@Person@ programmer, since it may depend on third party code, as shown in this example.
%However, statically checking if a method is deterministic is hard/impossible in most imperative object-oriented languages.
%
%While we may not expect the presence of bugs in the standard library class \Q@String@, the same behaviour can be achieved with subtyping:
%\saveSpace
%\begin{lstlisting}
%class EvilStr extends String{
%  method Bool isEmpty(){
%    return new Random().bool();
%  }}
%...
%String name=...$\Comment{can this be an EvilStr?}$
%Person bob=new Person(name);
%\end{lstlisting}
%\saveSpace
%As you can see, it is hard to make sound claims about Validation.
%
%\item Sample Bug 2:
%In our simple example, \Q@Person@ objects can be mutated using the setter, and exposed using the getter.
%We may consider the getter to be safe since in modern languages we expect strings to be immutable objects.
%\footnote{While we can update the field \Q@name@ to point to another string, we cannot mutate the string object itself.
%To obtain  \Q@"Hello"@ from \Q@"hello"@ we need to create a whole new string object that looks like the old one except for the first character. This would be different in older languages like C, where strings are just mutable arrays of characters.}
%
%Again, the assumption that they are immutable depends on the correctness of the code inside \Q@String@: if there was a bug in the \Q@String@ class, or any \Q@String@ subclass, then executing 
%\Q@println(bob.name())@ may change \Q@bob@ by quietly changing a part of its ROG.
%
%Again, checking
%what methods mutate states cannot be responsibility of the \Q@Person@ programmer.
%For Validation we need a language supporting aliasing and mutability control.
%\end{comment}

\loseSpace
\noindent\textit{Strong Exception Safety:}
The ability to catch and recover from validation failures is extremely useful as it allows the program to take corrective action.
This may be implemented with a conventional \Q@try-catch@, since violations are represented by throwing unchecked exceptions. Due to the guarantees of strong exception safety, the only trace that the invalid object existed is the exception thrown; any object that has been mutated/created during the \Q@try@ block is now unreachable (as happens in alias burying~\cite{boyland2001alias}).

However, if instead we chose not to enforce strong exception safety, an invalid object could be easily made reachable:
\saveSpace
\begin{lstlisting}[morekeywords={assert}, escapechar=\%]
mut Person bob = new Person("bob");
// Catch and ignore validation failure:
try {bob.name("");} catch (imm Error t){}
assert bob.name().isEmpty(); // now we have a rechable invalid object!
\end{lstlisting}
\saveSpace
As you can see, recovering from a validation failure in this way is unsound and breaks the guarantees of validation.
Strong exception safety is a useful property to enforce, but for the specific purpose of validation this could be relaxed by restricting only \Q@try-catch@ blocks that could capture unchecked exceptions.
Since calls to \Q@.validate()@ may only throw unchecked-exceptions, violating strong exception safety within a \Q@try-catch@ that cannot catch unchecked-exceptions would not break validation.

%LATER: This means that we could relax our Strong Exception Safety to hold only on unchecked exceptions (by restricting only \Q@try-catch@ blocks that capture unchecked exceptions.



% One of the advantages of checking Validation at run time, is that
% we can allow the program can take corrective actions if a property is violated.
% This may be implemented with a conventional \Q@try-catch@ if violations are represented by throwing errors.
% However, there is an issue with exceptions modelling invalid objects: they can be captured when the invalid object is still in scope. For example:


%As you can see, if we can capture validation failures as normal exceptions %(very desirable feature) then we may end up using invalid objects.
%Moreover,
% as shown before with the example of transferring cargo between two boats,
%after an invariant has been violated, even objects with valid invariant may be in an unexpected state.

% This situation is a general issue about reasoning on the state after recovering from exceptions.
% In particular, as shown in the example this prevent sound validation.

% Note how this produces a different semantics with respect to static verification, where violations
% never happened. However this will not necessarily lead to a broken semantics:
%Thanks to Strong exception safety we have a system where either the application terminate
%when an invalid object is detected, or where any witness of the execution causing the invalid object is erased from history
%those objects and all the witnesses will be garbage collected
% (as happens in alias burying~\cite{boyland2001alias}).
%In our example, this means that to continue execution after a detected bug, 
%we would require to garbage collect the overloaded boat, their cargo and probably most of the commercial port too.








%\subsubsection*{Solving Issue 3: Constructors}
%\saveSpace
%Exposing \Q@this@ during construction is a generally recognized problem~\cite{gil2009we}.
%A simple solution is to require all constructors to 
%simply take a parameter for each field and to just initialize the fields.
%An advantage of such approach is syntactic brevity: constructors are implicitly defined
%by the set of fields and thus there is no need to define them manually.
%\textbf{Expressive initialization operations can still be performed, by following the factory pattern.}
%\saveSpace


%\subsubsection*{Recap}
%By utilising type modifiers (\Q@imm@, \Q@mut@ and \Q@read@), object capabilities and immutable exceptions we obtain sound runtime verification for immutable classes/UML data types.
\saveSpace
\section{Validating encapsulated state}
\label{s:encapsulated}
\saveSpace
%Suppose we wanted to have a type and mutate it's state, but have such mutation's validated:


%It is very common for an object to be logically defined by the composition of sub-objects.
%The head object would then driving mutation of the sub-objects, and public methods
%of the sub-objects may be used in the validation. 
% However, the sub objects may come from a third party library, that is not annotated with contracts, and the %authors may change their behaviour in the future. Worst, they actual dynamic type may be dynamically loaded
%so that there is no way to predict their behaviour.
%That is, we are unable or unwilling to constrain sub-objects to
% cooperate into verification. We aim for verification to be correct independently of
% possibly buggy, possibly even malicious sub-objects.


Consider managing shipping items, where we keep an acceptable ship weight:
\saveSpace
%shipping list /UPS cost too much over 300
\begin{lstlisting}
class ShippingList {
  capsule Items items;
  read method Bool invariant() {
    return this.items.weight() <= 300;}
  ShippingList(capsule Items items) {
    this.items = items;
    if (!this.invariant()) {throw Error(...);}}//injected check
  mut method Void addItem(Item item) {
    this.items.add(item);
    if (!this.invariant()) {throw Error(...);}}}//injected check
\end{lstlisting}
\saveSpace
To handle this class we just inject calls to \Q@invariant@ at the end of the constructor and the \Q@addItem@ method.
This is safe since the \Q@items@ field is declared \Q@capsule@.
Relaxing our system to allows a \Q@mut@ modifier for
the \Q@items@ field and the corresponding constructor parameter 
makes the code broken:
the cargo we received in the constructor may be already compromised:
\saveSpace
\begin{lstlisting}
mut ShippingList l = new ShippingList(evilAlias);
// l is ok now, but we can break it
evilAlias.addItem(new HeavyItem()); // l is now invalid!
\end{lstlisting}
\saveSpace 
As you can see it would be possible for external code with no knowledge of the \Q@ShippingList@ to mutate its items.%
\footnote{
Conventional ownership solves these problems by requiring a deep clone of all the data the constructor takes as input, as well as all exposed data (possibly through getters).
In order to write correct library code in mainstream languages like Java/C++, defensive cloning~\cite{Bloch08} is needed.
%\REVComm{
For performance reasons, this is hardly done in practice and is a continuous source of bugs and unexpected behaviour~\cite{Bloch08}.}
%}{2}{citation to support this?}

Our restrictions on capsule mutators ensure that capsule fields are essentially an exclusive mutable reference.
Removing these restrictions would break our invariant protocol.
If we was to allow \Q@x.items@ to be seen as \Q@mut@ even if
\Q@x@ is not \Q@this@, then even if the \Q@ShippingList@ has full control at initialization time,
such control may be lost later:
\saveSpace
\begin{lstlisting}
mut ShippingList l = new ShippingList(new Items());
// l is ok now
mut Items evilAlias = l.items // here l loses control
// l is still ok
evilAlias.addItem(new HeavyItem()); // we invalidated l now
\end{lstlisting}
\saveSpace
If we allow for a \Q@mut@ return type this would be accepted:
\saveSpace
\begin{lstlisting}
mut method mut Items expose(C c) {return c.foo(this.items);}
\end{lstlisting}
\saveSpace
\noindent Depending on dynamic dispatch on \Q@c@, \Q@c.foo()@ may just be the identity function, thus
we would get in the same situation of the former example.
%Static analysis is usually unable/unwilling to track precise behaviour of dynamic dispatch.


%In addition to the above we put restrictions on any \Q@mut@ and \Q@capsule@ methods using a \Q@capsule@ field (we call such methods `capsule mutators'):
%\begin{itemize}
%\item only a single use of \Q@this@ is allowed (and is the one that uses the field),
%\item no \Q@mut@ or \Q@read@ parameters are allowed (apart from the implicit \Q@this@ parameter)
%\item and the return type cannot be \Q@mut@.
%\end{itemize}
%\noindent  Moreover, if the used capsule field is referenced in \validate, a \Q@this.validate()@ call is injected at the end of the method.


Allowing \Q@this@ to be used more then once can also cause problems:
\saveSpace
\begin{lstlisting}
mut method imm Void multiThis(C c) {
  read Foo f = c.foo(this);
  this.items.add(new HeavyItem());
  f.hi();} // Can `this' be observed here?
\end{lstlisting}
\saveSpace
\noindent If the former code was accepted, depending on dynamic dispatch on \Q@c@,
\Q@this@ may be reachable from \Q@f@, thus \Q@f.hi()@ may observe an invalid object.

In order to ensure that a second reference to \Q@this@ is not reachable through the parameters we only accept \Q@imm@ and \Q@capsule@ parameters.
If we were however to accept a \Q@read@ parameter, as in the example below,
we would be in the same situation as the former example, where \Q@f@ may contain
a reference to \Q@this@:
\saveSpace
\begin{lstlisting}
mut method imm Void addHeavy(read Foo f) {
  this.items.add(new HeavyItem())
  f.hi()} // Can `this' be observed here?
...
mut ShippingList l = new ShippingList();
read Foo f = new Foo(l);
l.addHeavy(f); // We pass another reference to `l' through f
\end{lstlisting}
\saveSpace

%, we would have the same problem with a \Q@read@ paramater. ... justify why we ned capsule
% The boat will sink if the weight of the cargo goes over 300. However, 
% \Q@Item@ and \Q@Items@ come from a third party library,  are not annotated with contracts and the authors may change their behaviour in the future. 
% All the code using \Q@Boat@  (client code) would like to  assume the boat has not sunk yet.
% In turn, that depends on the behaviour of \Q@Items.weight()@, thus the meaning of the \Q@Boat@ invariant is parametric on the everchanging meaning of  \Q@Items.weight()@.
% Can the code in the \Q@Boat@ class somehow enforce that for every possible well typed \Q@Item@ and \Q@Items@, client code will interact only with valid (non sunk)  boats?
% That is, we are unable or unwilling to constrain \Q@Item@ and \Q@Items@ to
% cooperate into making \Q@Boat@s unsinkable; 
% we aim to make so that \Q@Boat@s can be correct independently of
% possibly buggy, possibly even malicious \Q@Item@ and \Q@Items@ implementations.
% Indeed, thanks to the encapsulation, any kind of check in the language,
% as in `\Q@if(cargo.weight()>=300){..}@', would delegate the 
% behaviour to untrusted code in \Q@Items@.

% \textbf{without any knowledge about the behaviour of \Q@add()@ and \Q@weight()@}
% \footnote{A statically verified system with contracts on all methods may have this kind of knowledge.}
% there is no way we can discover the invariant violation without actually adding the objects and checking the 
% weight after the fact; thus in the general case violations can only be detected 
% when a broken object is already present in the system.
% Remember that to keep our approach lightweight,
% we do not rely on pre-post conditions; thus
% the behaviour of \Q@Items.weight()@ and \Q@Items.add(item)@ is uncertain.
% The names may suggest a specific behaviour, but there is no contract annotated on such methods.

% Note also that in the general case there is no way to fix a broken object,
%or to perform a deep clone and to test the operation on the clone first.


%REWRITE THIS BIT
%Here capsule fields 
%as input to our code-generation / \Q@validate()@-injection; that is, \Q@capsule T f@ is expanded by the language into:
%\begin{itemize}
%\item Induce a \Q@capsule@ parameter for the generated %constructor.
%\item Require to be updated with a \Q@capsule@ expression.
%\item Are accessed as a \Q@mut@ field.
%Access is \textbf{not} a destructive read.
% However methods accessing them are kept under
%strict control; either
%\begin{itemize}
%\item they have \Q@read this@: they act like a normal %getter, and can not propagate
%writing permission over the ROG of that field.
%Indeed, with \Q@read this@, any field access \Q@this.f@ will be typed \Q@read@ or \Q@imm@.
%\item they have \Q@mut this@, no parameter is \Q@mut@ or \Q@read@,
%the return type is not \Q@mut@ and \Q@this@ appear exactly one time in
%the method body: we call those methods \textbf{exposers}, and the invariant is going to be checked at the end of
%the exposers.
%\end{itemize}


%\end{itemize}
%Exposers are the key part of our solution.


% Those restrictions also enforce that while executing a capsule-mutator no object outside the ROG of \Q@this@ can be mutated, and thus capability objects cannot be usedI/O can not be performed: the capability objects are externally visible mutable objects and thus the type system will never place them into a \Q@capsule@.
%\subsection{The true expressibility of capsule modifiers}
%A capsule mutator method is a wrapper of a logical operation on a field, which is guaranteed to not see the \Q@this@ object.
%Thus, if \Q@this@ where to become broken during 
%the method's execution, we could not observe it until after. At first glance, it may seems that capsule %mutators allows for limited kinds of mutations.
%This is however not the case, consider the following
%general capsule mutator method that allows to apply any possible transformation over the content of a capsule %field:
%At first glance it mayseem from

\saveSpace
\section{Case studies}
\label{s:patterns}
\saveSpace

%interface Foo{ma mb}
%
%class Raw implements Foo{
%  no validate 
%}
%class ValidFoo1 implemens Foo{
%  private capsule Foo inner;
%  ma(){
%    this.inner.ma();
%  }
%  validate
%}
%
%
%class Raw {
%  mut method mut A stuff(read A a) {
%      if a.bar() {
%         x = ...
%      }
%      return new A(x)
%  }
%
%  read method imm Object stuffPre(read A a) {
%     return a.bar()
%  }
%  read method mut A stuffPost(imm Object o) {
%     return new A(x)
%  }
%  mut method imm Object stuffInner(imm Object) {
%      x = ...
%  }
%}
%class ValidFoo1 {
%	cpasule Raw r;
%  mut method mut A stuff(read A a) {
%     imm Object p = this.r.stuffPre(a);
%     imm Object res = this.transformR(r -> r.stuffInner(p));
%      return this.r.stuffPost(res)
%  }
%}
%
%
%
%class ValidFoo2 implemens Foo{
%  private capsule Foo inner;
%  validate
%}
\subsection{GUI}
Here we show that we are able to verify classes with circular mutable object graphs, that interact with the real world using I/O.
%Here we discuss how to use conventional OO programming patterns to take advantage of our system. Using those
%patterns allows to circumvent some of the apparent limitations of our system.
% At first glance it would seem that by only being able to validate immutable and encapsulated state one could not create validated classes with complicated, mutable interconnected object graphs.
%We show that this is not the case by encoding
%Our invariant is that everything in every movable container (and the top level component) should 
%-be inside the container 
%-not overlap with anyhing else inside
%Our containers represent boxes that completley contain other non-overlapping boxes.
Our case study involves a GUI with containers (\Q@SafeMovable@s) and \Q@Button@s (\MS{see figure}), the \Q@SafeMovable@ class has an invariant to ensure that its children are completely contained within it, and do not overlap. We have a \Q@Widget@ interface which provides methods to get size and position as well as children (a list of \Q@Widget@s). Both \Q@SafeMovable@s and \Q@Button@s implement \Q@Widget@. Crucially, since the children of \Q@SafeMovable@ is a list of \Q@Widget@s, it can contain other \Q@SafeMovable@s, and all queries to their size and position are dynamically dispatched, and are also used in \Q@SafeMovable.invariant@.

% Gui where the graphic representation of the widgets are guaranteed to not overlap.
% \Q@SafeMovable@ is a \Q@Widget@ whose position can be dynamically changed.
% \Q@Widget@s can contains more widgets as children.
% The invariant of a \Q@SafeMovable sm@
% is that all \Q@Widget@s inside \Q@sm.children()@,
% do not overlap with each others, and are
% fully contained in the area of \Q@sm@.

% Crucially, one of the children may be another
% instance of \Q@SafeMovable@.

Here we show a simplified solution, where  \Q@SafeMovable@ has just one \Q@Button@, and certain sizes and positions are fixed.

\begin{lstlisting}
class SafeMovable implements Widget{
  @Override read method Widgets children(){return this.box.c;}
  @Override read method Int left(){return this.box.l;}
  @Override read method Int top(){return this.box.t;}
  @Override read method Int width(){return 300;}
  @Override read method Int height(){return 300;}
  @Override  mut method Void dispatch(Event e){
    for(Widget w:this.box.c){w.dispatch(e);}}
  capsule Box box;
  read method Bool invariant(){..}
  SafeMovable(capsule List<Widget> cs){this.box=mkBox(c);}
  static method capsule Box mkBox(capsule Widgets c){
    mut Box b=new Box(5,5,cs);
    b.c.add(new Button(0,0,10,10,new MoveAction(b));
    return b;}}//mut b is soundly promoted to capsule

class Box{ Int l; Int t; mut List<Widget> c;
  Box(Int l, Int t, mut List<Widget> c){...} }

class MoveAction implements Action{
  mut Box outer; MoveAction(mut Box outer){this.outer=outer;}
  mut method Void process(Event event) {this.outer.l+=1;} }
..
void main(mut System system){
  system.newGui().display(new SafeMovable(..));}
\end{lstlisting}

As you can see, \Q@Box@es encapsulate the state of the \Q@SafeMovable@s that can change over time:
\Q@left@, \Q@top@ and \Q@children@. Also note how the ROG of \Q@Box@ is circular, since
the \Q@MoveAction@s inside a \Q@Button@ needs a reference to the containing \Q@Box@ in order to move it.
Note that even though the children of a \Q@SafeMovable@ are fully encapsulated we can still easily dispatch events to them by implementing \Q@Widget.dispatch@. Once a \Q@Button@ receives an \Q@Event@ with a matching ID it will then call its \Q@Action@'s \Q@process@ method. 

%Our main function uses a capability-object to display the top-level \Q@Widget@ and its \Q@children@, as well as dispatch events to it. 
Our example also shows that the restrictions of TMs and OCs are flexible enough to encode interactive GUI programs, where widgets may circularly reference other widgets.
In order to perform this case study, we had to first implement a simple GUI Library in L42. This library uses object capabilities to draw the widgets on screen, as well as fetch and dispatch the events. Importantly, neither our application, nor the underlying GUI library require back-doors into either our type-modifier or capability system to function; demonstrating the practical usability of our restrictions.



\noindent{\textit{The Invariant:}}
\Q@SafeMovable@ is the only class in our GUI that has an invariant, our system injects tests for it in two places: the end of its constructor and the end of its \Q@dispatch@ method (since it is a capsule-mutator). Their are no other checks inserted since we never do a field update on  \Q@SafeMovable@s. The code for the invariant is just a couple of simple nested loops:
\begin{lstlisting}
read method Bool invariant() {
  for (Widget w1 : this.box.c) {
    if(!this.inside(w1)) { return false; }
    for (Widget w2 : this.box.c) {
      if (w1 != w2 && SafeMovable.overlap(w1, w2)) {
        return false;}}}
  return true;
}
\end{lstlisting}
Here \Q@SafeMovable.overlap@ is a static method that simply checks that the bounds of the widgets don't overlap. \Q@the@ call to \Q@this.inside(w1)@ similarly checks that the widget is not outside the bounds of this; this instance method call is allowed as the function only uses \Q@this@ to access its \Q@width@ and \Q@height@.


\noindent{\textit{Our Experiment:}}
% This code is a simplified version of our first case study. In the full code the \Q@SafeMovable@ constructors take \Q@left@, \Q@top@, \Q@width@ and \Q@height@ parameters and can either take a \Q@box@ directly or will generate one with $4$ \Q@Button@s}
% we we have $4$ buttons, each button moves in one of the $4$ cardinal directions.
Counting both \Q@SafeMovable@s and \Q@Button@s our main method creates $21$ widgets: a top level (green) \Q@SafeMovable@ without buttons, containing $4$ (red, blue, and black) \Q@SafeMovable@s with
$4$ (gray) buttons each. When a button is pressed it moves the containing \Q@SafeMovable@ a small amount in the corresponding direction.
%Each container has 4 gray buttons one for each cardinal direction.
% In our set up
% the top level \Q@SafeMovable@
% contains a big red \Q@SafeMovable@
%  containing $2$ smaller blue \Q@SafeMovable@. One of those contains a tiny black \Q@SafeMovable@.
%Our invariant is that everything in every movable container (and the top level component) should 
%-be inside the container 
%-not overlap with anyhing else inside
%Our containers represent boxes that completley contain other non-overlapping boxes.

This set up is not overly involved, the maximum nesting level of \Q@Widget@s is $5$.
Our main method automatically presses each of the $16$ buttons once. In L42, using the approach of this paper, this resulted in $77$ calls to \Q@SafeMovable@s invariant. 

\noindent{\textit{Comparison with visible state semantics:}}
As an experiment we set our implementation to generate invariant checks following the standard approach for visible state semantics (the one used by Eiffel and D), where the invariant is instead check at the start and end of \emph{every} public method of \Q@SafeMovable@s, including the simple getters implementing the \Q@Widgment@ interface. When we ran our test case the invariant was instead called $52,734,053$ times (it took about $4$ days to run), in comparison to the $77$ we got when using our invariant protocol. The reason the number of checks is so high is due to it being exponetial in the depth of the GUI: the invariant of a \Q@SafeMovable@ will call the \Q@width@, \Q@height@, \Q@left@ and \Q@top@ methods of its children, which may themselves be \Q@SafeMovable@s, and hence such calls will themselves invoke invariant checks. Note that like other systems in the literature, we do not perform invariant checks when calling instance methods (such as \Q@this.inside@) on \Q@this@ within the invariant of \Q@SafeMovable@, had we done that we would have gotten an infinite recursion.

% It can be surprising to see such extreme difference. We ran our example with less widgets, and our results suggest an exponential growth in the cost of the conventional approach. For example by removing 2 containers (and their 8 buttons) we get....
% We now explain how this exponential explosion happens:
% For the outer-most box to check its invariant, it needs to call methods \Q@left,top,height@ and \Q@with@ to all its contained widgets.
% If one of those widget has \Q@invariant@, when such public method is called,
% its \Q@invariant@ is checked (twice). This requires to call \Q@left,top,height@ and \Q@with@ to all its contained widgets, some of those may also have \Q@invariant@s.

% In literature there is attention to prevent method called from an invariant to call public methods on \Q@this@; it would cause the system to go in loop.
% However when calling methods on other objects is allowed, if those objects have invariant, this cause the aforementioned explosion. 

% Our example also shows that the restrictions of TM's and OC's are flexible enough to encode interactive GUI programs, where widgets may circularly reference other widgets.
% In order to perform this case study, we had to first implemented a simple Gui Library in L42. This library uses object capabilities to draw the widgets on screen, as well as fetch and dispatch the events.
% The gui library abstract away all the details of drawing and events; the user code need only to provide concrete classes implementing the \Q@Widget@ interface.

\noindent{\textit{Comparison with Spec\#}}
We encoded our Widget example also on Spec\#.
The result is the same of L42: the invariant is checked $77$ times, and in exactly the same locations of L42.
However, we found great difficulty to encode our example.
We used the Boogie static checker to verify all the aliasing-ownership properties needed to
ensure that the $77$ run-time invariant checks soundly enforce that the invariant holds when is expected.
This of course includes preventing the Gui to ever display two overlapping Widget. 

We believe this comparison is fair since we got both systems to essentially do the same thing:
\begin{itemize}
\item they statically verify ownership/aliasing annotations
\item they check the admissibility/valididty of a the invariant code
\item they perform sufficient runtime-checking of the invariant
\end{itemize}
To encode the GUI example in L42, we only needed to annotate \Q@mut/read@ the mutable/readable methods
and types (NUM annotations), and \Q@capsule@ a single field. 
Spec\# annotations are much more involved that that, and we had to annotate methods
with purity and ownership annotations. The ownership annotations included parameter, method and field attributes as well as requires, ensures and modifies clauses, as well as explicit ownership assignment statements.
Since Spec\# annotation can be involved, as for example EXAMPLE
we are going to count the number of tokens (excluding \Q@.@ and parenthesis).
This gave us a total of NUM tokens, that is more then 3 times the amount needed in L42.
Moreover, in L42 we only use 2 different kinds of annotations (mut/capsule), while in Spec\# we use ... kinds of annotations. This corroborate our statement that our system is easier to use for programmers that are not experts in sw verification.
Counting the number of characters we get .... against ... in L42.  

Our design, using an inner \Q@Box@ object is a common pattern in static verification: to encapsulate all the relevant mutating state into an encapsulated sub-object
that are never exposed to the users.
We also found natural to use the box pattern
also in Spec\# requires.
This dramatically simplifies the handling of the circular object graph, where the button event can refer back, in order to change the widget position.

We encoded our Widget example also on Microsoft Code Contract; their system also ran the invariant checking $77$ times. Their system is easy to use, but it is unsound since it is built over an unsound/incomplete static verifier [?].

\noindent\textit{Family, a worst case scenario for L42}
For our second case study, we wished to make an example where the performance of L42 and the conventional approach was similar. We forged an example when a \Q@Family@ has a list of parents and a list of children;
all the childrens need to be younger then their parents and every \Q@Person@ need to have a non empty name and a positive age.  
We model the pass of time with a \Q@processDay@ method, and we simulate $3$ years of life (that is, $3\times365$ days) of a family of $4$.
The age of a \Q@Person@ grow when its birthday is processed.
Notably, \Q@processDay@ is a \Q@mut@ method that can potentially mutate any person in the system, thus
L42 have to run a lot of invariant checks. The object graph here is very shallow: the \Q@Family@ holds the \Q@Person@s and that is it.
However, even in this case we get about $10$ times less invariant calls: Num in the conventional approach and Num in L42.

\noindent\textit{Expressiveness:}
Finally, in our this third case study we 
shown that even if we do not aim to expressiveness, but to simplicity, soundness and efficiency, we are still able to express a reasonable amount of cases.
We encoded in L42 all the examples present in papers~\cite{??}.
We can express all the examples except ....
Again, we quantify the annotation burden and we discover....


\noindent\textit{The transform pattern:}
A capsule mutator method is a wrapper over a logical operation on a field, which is guaranteed to not see the \Q@this@ object.
Thus, if \Q@this@ is made invalid during 
the method's execution, we could not observe it until after the method completes.
At first glance, it may seem that capsule mutators allow only very limited kinds of mutation.
This is however not the case. 

Consider the following
simple pattern to allow flexible use of capsule fields: define a \Q@transform@ function and a \Q@ItemTransformer@ interface like so:
\saveSpace
\begin{lstlisting}
interface Transformer<T> {method Void apply(mut T items);}
class ShippingList {...
  mut method Void transformItems(Transformer<Items> t) {
   t.apply(this.items);
   if (!this.validate()) {throw Error(...);}}}//injected check
\end{lstlisting}
The \Q@.transformItems()@ method 
offers an expressive power similar to a
\Q@mut@ getter, but ensures that 
the field content cannot leak out.
For example:
\begin{lstlisting}[escapechar=\%]
// Lambda Expression that creates a new Transformer<Items>
this.transformItems(items -> items.add(new Item(..)))
\end{lstlisting}
%//`i' is captured by the closure.
%// `imm' and `capsule' varaibles can be captured.

%    %\Comment{}%this.items.add(i);
%    // Cant instead capture `this': it can't be typed %as `imm'
%    // since `ItemTransformer.transform()' is an %`imm' method
%  })
%}
%  // instead of:
  %\Comment{}%this.exposeItems().add(i)

Note that the code above does not access a capsule field but merely calls a method that does; thus
it is \emph{not} a capsule mutator method, so it is not constrained by the restrictions on them. Code like the above would also allow one to mutate multiple capsule fields in one method.
Our pattern cooperates with the language’s restrictions to ensure each mutation is completed as a separate operation, that is perceived by the rest of the system
as if it was atomic.%
%,  i.e. they can't see or update the other capsule fields.


%\begin{lstlisting}[escapechar=\%]
%class List {
%  mut List prev;
%  mut List next;
%  Object elem;  
%  read method Bool ok() {
%    return this.next.prev==this && this.prev.next==this &&..;}
%  read method Int size(){
%    if(next==this){return 1;} return next.size()+1;}}
%\end{lstlisting}
%Clearly the \Q@mut@ fields of \Q@List@ cannot be marked as \Q@capsule@.
%However, only \Q@capsule@ and \Q@imm@
%fields can be accessed in \validate.
%Thus, \Q@.innerValidate()@ can not be the \validate{} method for \Q@List@.
%The solution is to use a `box' over our \Q@List@, and to validate our `box':
%
%\loseSpace\loseSpace\loseSpace
%\begin{lstlisting}[escapechar=\%]
%class ListBox { 
%  capsule List inner;
%  read method imm Bool invariant() {return this.inner.ok();}
%  read method Int size(){ return this.inner.size();}
%\end{lstlisting}
%\saveSpace
%Encoding this example in Spec\# would be much more verbose (see case study XX) and still require
%a \Q@ListBox@ object,
%while the visible state semantic of Eiffel or D would cause an large amount of \Q@invariant@ checks
%if the list had any recursive method; consider for example the \Q@size@ method:
%if there was an invariant with visible state semantic on \Q@List@, calling \Q@List.size()@ would require 
%calling \Q@List.invariant()@ before and after the method execution. If the list has more then one element, the recursive \Q@size@ call would also call the invariant twice.
%We would also want to create forwarding methods in \Q@ListBox@ for all public methods defined in \Q@List@. This approach allows the validation of many interesting and practically useful data-structures.
%However the limitations of capsule mutator methods mean that any \Q@mut@ methods in \Q@ListBox@ taking \Q@read@ or \Q@mut@ parameters, or returning \Q@mut@, cannot be trivially forwarded.
%% since they necessitate mutating a \Q@capsule@, instead complicated and involved forwarding would be needed, if it is even possible.
%Our example is about a list of immutable objects.
%To instead validate a list of \Q@mut@ objects we would need to use our box pattern not just around the list,
%but around a section of data encapsulating both the list and all the contained elements.
%This is because our simple \Q@capsule@ modifier requires the whole ROG to be encapsulated.
%Conceptually, it would be better for the list (of mutable objects) to be validated by its
%head, since the behaviour of the contained objects is transparent to the validation criteria. 
%Our limitation relates to full encapsulation and contrasts with flexible encapsulation as in 
%ownership~\cite{ClarkeEtAl98}. However, neither traditional flexible encapsulation/ownership, nor our language are capable of verifying that \Q@List.elem@ is not (indirectly) referenced within \Q@ListBox.validate()@.
%

\saveSpace
\section{Formalisation of Validation}
\label{s:meaning}
\saveSpace
In order to model our system, we need to formalise an imperative object-oriented language
with exceptions and object capabilities,  and with a rich type system
supporting \Q@mut, imm, read, capsule@ and strong exception safety.
Formally modelling the semantics of such a language is easy, but 
modelling and proving correctness of such a rich type system would deserve a paper
of its own, and indeed many such papers exist in literature%
~\cite{ServettoEtAl13a,ServettoZucca15,GordonEtAl12,clebsch2015deny,JOT:issue_2011_01/article1}.
Thus, we are going to assume that there is an expressive and sound type system enforcing
those properties, and instead focus on validation.
To provide a good modularisation for our reasoning, 
we will clearly list the properties we need to rely upon, so that \emph{every type
system supporting those properties} supports validation.

To encode object capabilities and I/O, we assume a special location
$c$ of type \Q@Cap@.
This location would refer to an object whose fields model, for example, the content of input and output files.
All its methods must require a \Q@mut this@, and shall mutate the ROG of $c$, and the main expression will start with
such $c$ location in scope. In order to simplify our proof, we assume that $c$ only has \Q@mut@ fields, hence it is always valid (i.e. $c.validate()$ is defined to evaluate to \Q@true@).
We strive to keep our small step semantics as conventional as possible; following \REVRComm{Pierce~\cite{pierce2002types}}{2}{A citation to Featherweight Java (TOPLAS 2001) would be specific than [this]} we assume:
\begin{itemize}
\item An implicit program/class-table.
\item Memory $\sigma:\!:\!= l\mapsto C\{\Many{v}\}$ as a finite map from locations $l$ to annotated tuples $C\{\Many{v}\}$ representing objects,
where $C$ is the class name and $\Many{v}$ contains the values of the fields.
We use the notation $\sigma[l.f=v]$ to update an object field and $\sigma[l.f]$ to access the field.
\item A main expression that is reduced in the context of such a memory and program.
\item A reduction relation $\sigma|\e\rightarrow \sigma'|e'$.
\item A type system $\Sigma;\Gamma\vdash\e:T$, where 
the expression $\e$ can contain locations $l$ and free variables $x$;
the type of locations is encoded in the memory environment $\Sigma:\!:\!= l\mapsto C$
and the type of the free variables is encoded in the variable environment $\Gamma:\!:\!= x\mapsto T$.
\item We use $\Sigma^\sigma$ to trivially extract $\Sigma$ from a memory $\sigma$.
\item The special capability object location $c$ of the \Q@Cap@ class; instances of \Q@Cap@ cannot be created with a \Q@new@ expression.
\item We have a special \emph{monitor} expression \Q@M(@$l$\Q@;@$\e_1$\Q@;@$\e_2$\Q@)@.
Those expressions are not present in the source code but are inserted by the reduction.
\REVRComm{Initially we will have $\e_2= l$\Q@.validate()@;
$\e_1$ will be reduced until it becomes a value, then
$\e_2$ will be reduced to test if $l$ is invalid.}{3}{Hard to understand sentence} We annotate the monitor-expression with $l$ to track
that if the validation check fails, it is precisely $l$ that is invalid.
We use a failed monitor expression (i.e. when $\e_2$ is \Q@false@) to represent an \Q@error@ expression.
\item Before reducing the body of a \Q@try@, we annotate it with a snapshot of 
the state of the memory. This is used to annotate that such state will not be mutated by executing the body of the \Q@try@.
\end{itemize}

To keep our formalization focused on
the challenges of validation, 
there are some
tweaks with respect to our informal description of our approach.
From a formal perspective 
these changes do not change expressiveness:
\begin{itemize}
\item We require that all fields are instance-private, as opposed to only capsule fields. One could always provide getters and setters to simulate public fields.
\item We do not have explicit constructor definitions, rather we assume that all constructors are of the canonical form
\Q@$C$($T_1 x_1$,$\ldots$,$T_n x_n$) {this.$f_1$=$x_1$;$\ldots$;this.$f_n$=$x_n$;}@,
 where $T_1,\ldots\T_n$ are the types (including modifiers) of the fields of $C$.
To provide more flexible initialization one could always make a factory method.
\item We require that \Q@.validate()@ can only use \Q@this@ to access fields,
this can be achieved by inlining method calls.
%or if they are recursive, replacing them with calls to methods that take the fields of \Q@this@ instead of \Q@this@ itself.
\item For simplicity, we do not have actual exception objects,
rather we just have a concept of an \emph{error} with no associated value.
We believe adding traditional exceptions would not cause any interesting variation of our proof.
\end{itemize}


\newcommand{\ctxG}{\myCalBig{G}}
\renewcommand{\vs}{\Many{v}}
\renewcommand{\Opt}[1]{#1?}
\begin{figure}
\!\!\!\!
\begin{grammatica}
\produzione{\e}{\x\mid l\mid\Kw{true}\mid\Kw{false}\mid \e\singleDot\m\oR\es\cR\mid \e\singleDot\f 
\mid\e\singleDot\f\equals\e 
\mid\Kw{new}\ C\oR\es\cR
\mid\Kw{try}\ \oC\e_1\cC\ \Kw{catch}\ \oC\e_2\cC
}{expression}\\
\seguitoProduzione{
\mid \Kw{M}\oR l;\e_1;\e_2\cR\mid\Kw{try}^{\sigma}\oC\e_1\cC\ \Kw{catch}\ \oC\e_2\cC
}{run-time expr.}\\
\produzione{v}{l}{value}\\
\produzione{\ctx_v}{\square
\mid \ctx_v\singleDot m\oR\es\cR
\mid v\singleDot\m\oR\Many{v}_1,\ctx_v,\es_2\cR
%\mid \ctx_v\singleDot\f 
%\mid \ctx_v\singleDot\f\equals\e
\mid v\singleDot\f\equals\ctx_v
}{evaluation ctx}\\
\seguitoProduzione{
\mid \Kw{new}\ C\oR\Many{v}_1,\ctx_v,\es_2\cR
\mid \Kw{M}\oR l;\ctx_v;\e\cR
\mid \Kw{M}\oR l;v;\ctx_v\cR
\mid \Kw{try}^\sigma\oC\ctx_v\cC\ \Kw{catch}\ \oC\e\cC}{}\\

\produzione{\ctx}{\square\mid\ctx\singleDot m\oR\es\cR\mid\e\singleDot\m\oR\es_1,\ctx,\es_2\cR
%\mid \ctx\singleDot\f 
%\mid \ctx\singleDot\f\equals\e
\mid \e\singleDot\f\equals\ctx
\mid \Kw{new}\ C\oR\es_1,\ctx,\es_2\cR
}{full ctx}\\
\seguitoProduzione{
\mid
\Kw{M}\oR l;\ctx;\e\cR\mid
\Kw{M}\oR l;\e;\ctx\cR\mid
\Kw{try}^{\sigma?}\oC\ctx\cC\ \Kw{catch}\ \oC\e\cC\mid
\Kw{try}^{\sigma?}\oC\e\cC\ \Kw{catch}\ \oC\ctx\cC

}{}\\


%\produzione{M_l}{\ctx[M\oR l,\e\cR]}{}\\
%\produzione{\ctxG_l}{
%  M_l\singleDot\m\oR\es_1,\ctx,\es_2\cR
% |\e\singleDot\m\oR\es_1, M_l, \es_2, \ctx, \es_3\cR
% |M_l\singleDot\f\equals\ctx
% |\Kw{new}\ C\oR\es_1,M_l,\es_2,\ctx,\es_3\cR
% |\Kw{try}\oC\ctx\cC\ \Kw{catch}\ \oC\e\cC
% |\ctx[\ctxG_l]}{}\\
\produzione{CD}{\Kw{class}\ C\ \Kw{implements}\ \Many{C}\oC\Many{F}\,\Many{M}\cC\mid 
\Kw{interface}\ C\ \Kw{implements}\ \Many{C}\oC\Many{M}\cC
}{class decl}\\
\produzione{F}{\T\ \f;}{field}\\
\produzione{M}{\mdf\, \Kw{method}\, \T\ \m\oR\T_1\,\x_1,\ldots,\T_n\,\x_n\cR\ \Opt\e}{method}\\
\produzione{\mdf}{\Kw{mut}\mid\Kw{imm}\mid\Kw{capsule}\mid\Kw{read}}{type modifier}\\
\produzione{\T}{\mdf\,C}{type}\\
\produzione{r_l}{
 v\singleDot\m\oR\Many{v}\cR
\mid v\singleDot\f
\mid v_1\singleDot\f\equals v_2
\mid \Kw{new}\,C\oR\Many{v}\cR
\quad\text{with }l\in \{v,v_1,v_2,\Many{v}\}
}{$l$ inside a redex}\\
\produzione{\mathit{error}}{
\ctx_v[\Kw{M}\oR l; v;\Kw{false}\cR]
\quad\text{with }
\ctx_v \text{not of form}\ \ctx_v'[\Kw{try}^{\sigma?}\oC\ctx_v''\cC\ \Kw{catch}\ \oC\_\cC]
}{validation error}
\end{grammatica}
\caption{Grammar}
\end{figure}


\loseSpace
\noindent\textit{Grammar and Well-Formedness Criteria:}
The detailed grammar is exposed in Figure 1.
As explained before, the only non-standard expression is the monitor.
We denote with $r_l$ a redex that contains the location $l$.

\noindent Our well formedness criteria are:
\begin{itemize}
\item All field accesses in method bodies are of the form
\Q@this.@$f$. Thus we require all fields to be instance-private.

\item Field accesses in the main expression, 
must be of the form $l\singleDot\f$.

\item \Q@.validate()@ takes a \Q@read this@, and uses \Q@this@ only to access fields. Even calling methods on \Q@this@
is disallowed.
\item All the fields referred in \Q@.validate()@ are either \Q@imm@ or \Q@capsule@.
\item All the methods that access capsule fields 
either have a \Q@read this@,
or have a \Q@mut/capsule this@, no \Q@mut@ or \Q@read@ parameters, no \Q@mut@ result and 
must use \Q@this@ exactly once in their body.
\item 
During reduction, locations $l$ that are preserved by a \Q@try@ block are
never monitored; formally 
in $\Kw{try}^\sigma\oC\e\cC\_$, $\e$ is not of the form $\ctx[$\Q@M(@$l;\_$\Q@)@$]$ with $l\in\sigma$.
\end{itemize}

We model subtyping with interfaces 
and we do not consider subclassing.
Indeed interfaces do not have an implemented \Q@.validate()@ method, objects implementing those interfaces do.
To enrich our formalism with subclassing, we would need to add the 
well-formed criteria that \Q@validate()@ 
methods start by checking the result of \Q@super@.\Q@validate()@.

\begin{figure}
\!\!
$\!\!\!\!\!\begin{array}{l}
 \inferrule[(update)]{{}_{}}{
\sigma|l.f\equals{}v\rightarrow \sigma[l.f=v]|
\Kw{M}\oR l;l;l\singleDot\Kw{validate}\oR\cR\cR
 }{}
\quad
 \inferrule[(new)]{{}_{}}{
\sigma|\Kw{new}\ C\oR\vs\cR\rightarrow \sigma,l\mapsto C\{\vs\}|
\Kw{M}\oR l;l;l\singleDot\Kw{validate}\oR\cR\cR
 }{}
\\[5ex]
 \inferrule[(mcall)]{{}_{}}{
\sigma|l\singleDot\m\oR v_1,\ldots,v_n\cR\rightarrow \sigma|
\e'[\Kw{this}=l,\x_1=v_1,\ldots,x_n=v_n]
 }{
  \begin{array}{l}
  \sigma(l)=C\{\_\}\\
  C.m=\mdf\,\Kw{method}\,\T\,\m\oR\T_1\,\x_1\ldots\T_n\x_n\cR\e\\

\text{if }\ \exists \f\text{ such that}\ \ C.f=\Kw{capsule}\,\_,
\mdf=\Kw{mut},
\\*\quad\f\, \text{inside}\, C\singleDot\m
\text{ and }
\f\,\text{inside}\, C\singleDot\Kw{validate}

\\*
\text{then }\e'=\Kw{M}\oR l;\e;l\singleDot\Kw{validate}\oR\cR\cR\\*
\text{otherwise }\ \e'= \e
  \end{array}
}
\\[5ex]
 \inferrule[(monitor exit)]{{}_{}}{
\sigma|\Kw{M}\oR l; v;\Kw{true}\cR\rightarrow \sigma|v
 }{}
\quad

 \inferrule[(ctxv)]{\sigma_0|\e_0\rightarrow\sigma_1|\e_1}{
\sigma_0|\ctx_v[\e_0]\rightarrow \sigma_1|\ctx_v[\e_1]
 }{}

\quad
 \inferrule[(try enter)]{{}_{}}{
\sigma|\Kw{try}\ \oC \e_1\cC\ \Kw{catch}\ \oC\e_2\cC\rightarrow 
\sigma|\Kw{try}^\sigma\oC\e_1\cC\ \Kw{catch}\ \oC\e_2\cC
 }{}
\quad


\\[5ex]


 \inferrule[(try ok)]{{}_{}}{
\sigma,\sigma'|\Kw{try}^{\sigma}\oC v\cC\ \Kw{catch}\ \oC\_\cC\rightarrow \sigma,\sigma'|v
 }{}
\quad

 \inferrule[(try error)]{{}_{}}{
\sigma,\_|\Kw{try}^\sigma\oC \mathit{error}\cC\ \Kw{catch}\ \oC\e\cC\rightarrow \sigma|\e
 }
\quad
 \inferrule[(access)]{{}_{}}{
\sigma|l.f\rightarrow \sigma|\sigma[l.f]
 }{}
%\quad
\end{array}$
\caption{Reduction rules}
\end{figure}

\loseSpace
\noindent\REVRComm{\textit{Reduction rules:}}{2}{This discussion is surprisingly short}
Reduction rules are defined in Figure 2.
These rules are pretty standard;
\textsc{mcall}
uses the intuitive auxiliary function \emph{inside}
formally defined as follow:

$%\begin{array}{l}
\f\, \text{inside}\, C\singleDot\m\text{ holds iff }
C\singleDot\m=\_\,\Kw{method}\_\,\ctx[\Kw{this}\singleDot\f]
%\end{array}
$

%\noindent Inserting the monitor expressions during reduction is convenient for the proof,
%but it could instead be done ahead of time.

That is, the monitor is added for all field update and new objects, but;
for method calls the monitor is added only if the method has a \Q@mut@ modifier and its body accesses \Q@capsule@ field.

The interaction with monitors and exceptions is interesting:
a monitor releases the value if the check is \Q@true@, and produces an error if the 
check is \Q@false@.
If either $\e_1$ or $\e_2$ are not values, the execution is propagated inside
by \textsc{ctxv}.
If either $\e_1$ or $\e_2$ evaluate to an error, such error is captured by 
\textsc{try error}.
Thanks to strong exception safety
we do not need to worry
if the (partial) execution of $\e_1$ broke the $l$ object.
If the language were to support checked and unchecked exceptions, but offered 
strong exception safety only for the unchecked ones, then 
the type system should require neither $\e_1$ nor $\e_2$ leak 
checked exceptions.





%WHERE TO PUT THIS?
%Note that for \Q@capsule@ fields, the constructor and the field update
%will require \Q@capsule@ for the correspoi, while the field access will produce a \Q@mut@.



\loseSpace
\noindent\textit{Axiomatic type properties:}
As discussed, instead of providing a concrete set of type rules, we provide a set of properties
that such a type system needs to respect.
To express these properties, we first need some auxiliary definitions:

%\noindent\textbf{Define}
%$\mathit{encapsulatedObj}(C)$:\\*
%${}_{}$\quad\quad \Q@class @$C$\,\Q@implements @$\Many{C}$\Q@{@$\,\Many{F}\,\Many{M}$\Q@}@
% and $\forall \mdf\,C\,\f \in \Many{F},\ \mdf \in \{\Kw{imm},\Kw{capsule}\}$\\*
%\noindent As we discussed, only encapsulated objects can support invariants;
%their class declarations only have immutable or capsule fields. Note how here we see immutable
%and simple objects as special cases of encapsulated ones.

\noindent\textbf{Define} $\mathit{erog}(\sigma,l_0)$:\\*
\indent $l \in \mathit{erog}(\sigma,l_0)
\text{ if } \Sigma^\sigma(l_0).f \in \{\Kw{imm}\,\_,\Kw{capsule}\,\_\}
\text{ and } l \in \mathit{rog}(\sigma,\sigma(l_0).f)
$\\*
\noindent
The encapsulated ROG of $l_0$ is composed by all the objects
in the ROG of its immutable and capsule fields.


\noindent\textbf{Define} $\mathit{mutatable}(l,\sigma,\e)$:\\*
\indent with $T=\Kw{imm}\,\Sigma^\sigma(l)$ and $\e=\ctx[l]$,\\*
\indent $\Sigma^\sigma;\x:T\vdash\ctx[\x]:T'$ does not hold for any $T'$.\\*
\noindent That is, an object is mutatable by a $\sigma,\e$ if there is an occurrence of 
$l$ in $e$ that when seen as immutable makes the expression ill-typed.



\noindent\textbf{Define}$\ \sigma_0|e_0\Rightarrow \sigma_1|e_1$:\\*
\indent iff $\{\sigma_1|\e_1\}=\{\sigma|\e \text{ where } \sigma_0|e_0\rightarrow \sigma|e\}$

%if $\ \sigma_0|e_0\rightarrow \sigma|e$ then $\sigma_1|\e_1=\sigma|\e$
% $\exists! \sigma_1|\e_1$ such that $\sigma_0|\e_0\rightarrow \sigma_1|\e_1$\\*
\noindent We define
a deterministic reduction arrow.
Here we require that there is exactly one reduction possible.


%We can now assume the following properties over the type system:

\begin{Assumption}[Progress]
if $\Sigma^{\sigma_0};\emptyset\vdash e_0: T_0$,
and $e_0$ not a value or $\mathit{error}$, then
$\sigma_0|e_0\rightarrow \sigma_1|e_1$
\end{Assumption}


\begin{Assumption}[SubjectReductionBase]
if $\Sigma^{\sigma_0};\emptyset\vdash e_0: T_0$,
$\sigma_0|e_0\rightarrow \sigma_1|e_1$,
then
$\Sigma^{\sigma_1};\emptyset\vdash e_1: T_1$
\end{Assumption}


\begin{Assumption}[MutField]
\ \\
\indent(1)\ if $\Sigma;\Gamma\vdash\e\singleDot\f:\Kw{mut}\,\_$
then $\Sigma;\Gamma\vdash\e:\Kw{mut}\,\_$
,\ and 
\\*\indent(2)
if $\Sigma;\Gamma\vdash\e_0\singleDot\f\equals\e_1:T$
then $\Sigma;\Gamma\vdash\e_0:\Kw{mut}\,\_$
\end{Assumption}
\noindent If the result of a field access is mutable,
the receiver is mutable too, and the receiver of a field update is always mutable.

\begin{Assumption}[HeadNotCircular]
if
$\Sigma^\sigma;\Gamma\vdash l:T$
then $l\notin\text{erog}(\sigma,l)$
\end{Assumption}
\noindent
\noindent An object is not part of the ROG of its immutable or capsule fields.


\begin{Assumption}[CapsuleTree]
If   $\Sigma^\sigma;\Gamma\vdash \e:\T$,
$l_2\in\text{erog}(\sigma,l_1)$,
$l_1\in\text{erog}(\sigma,l_0)$,\\*
and
$\mathit{mutatable}(l_2,\sigma,\e)$
then 
$l_2\notin\text{erog}(\sigma\setminus l_1,l_0)$
\end{Assumption}
\noindent In a well typed $\sigma,e$, if mutatable $l_2$ is reachable from
$l_1$, and $l_1$ is reachable from $l_0$,
then all the paths connecting $l_0$ and $l_2$ pass trough $l_1$; thus
if we was to remove the node $l_1$ from the object graph, $l_0$ would not reach $l_2$ any more.


CapsuleTree and HeadNotCircular together 
shows that capsule fields section the object graph into a tree of nested `balloons',
where nodes are mutable encapsulated objects and
edges are given by reachability between those objects in the original memory:

$l_2$ is in the encapsulated ROG of $l_1$;
$l_2$ is mutatable and is reached trough $l_1$, thus
it must be reachable by a \Q@capsule@ field.
Thanks to HeadNotCircular and $l_1\in\text{erog}(\sigma,l_0)$ we can derive 
$l_0\notin\text{erog}(\sigma,l_1)$.



\begin{Assumption}[Determinism]
if $\emptyset;\Gamma\vdash \e:\T$, 
$\forall x \Gamma(x)\neq\Kw{mut}\,\_$, and
$\sigma | \e'\rightarrow^+ \sigma' | \e''$
then 
$\sigma | \e'\Rightarrow^+ \sigma,\_ | \e''$,
where $\e'=\e[x_1=l_1,\ldots,x_n=l_n]$ and $\Sigma^\sigma;\emptyset\vdash \e':\T$
\end{Assumption}
\noindent The execution of an expression
with no \Q@mut@ free variables is deterministic and does not
  mutate pre existing memory (and thus does not not perform I/O by mutating pre existing $c$).


\begin{Assumption}[StrongExceptionSafety]
if $\Sigma^{\sigma,\sigma'};\emptyset\vdash \ctx[\Kw{try}^\sigma\oC\e_0\cC\ \Kw{catch}\ \oC\e_1\cC]:\T$
and\\*
$
\sigma,\sigma'|\ctx[\Kw{try}^\sigma\oC\e_0\cC\ \Kw{catch}\ \oC\e_1\cC]\rightarrow 
\sigma''|\ctx[\Kw{try}^\sigma\oC\e'\cC\ \Kw{catch}\ \oC\e_1\cC]
$
then 
$\sigma''=\sigma,\_$
and
$\Sigma^\sigma;\emptyset\vdash \ctx[\e_1]:\T$
\end{Assumption}
\noindent
For each \Q@try-catch@, the execution preserves the memory needed to continue the execution in case of error
(the memory visible outside of the \Q@try@).%

%Thanks to how our reduction rules are designed, especially \textsc{try error},
%@Progress will need to rely on @StrongExceptionSafety internally.

Note that our last well formedness rule requires 
\textsc{update} and \textsc{mcall} to introduce
monitor expressions only over locations
that are not preserved by a \Q@try@ block.
This can be achieved since monitors are introduced
around $\mathit{mutating}$ operations
(and \textsc{new}),
and StrongExceptionSafety ensures no mutation happens on the preserved memory.

To the best of our knowledge, only the type system of 42~\cite{ServettoEtAl13a,ServettoZucca15}
 supports all these assumptions out of the box,
while both Gordon~\cite{GordonEtAl12} and Pony~\cite{clebsch2015deny,clebsch2017orca} supports all except StrongExceptionSafety,
however it should be trivial to modify them to support it:
the \Q@try-catch@ rule could be modified to
$\emptyset;\Gamma\vdash\Kw{try}\ \oC\e_0\cC\ \Kw{catch}\ \oC\e_1\cC:\T$
if\\* $\emptyset;
\Gamma,\{x:\Kw{read}\,C | x:\Kw{mut}\,C\,\in\Gamma\}
\vdash\e_0:\T$ and $\emptyset;\Gamma\vdash\e_1:\T$,
i.e. $e_0$ can be typed when seeing all externally defined mutable references as \Q@read@.

\loseSpace
\noindent\textit{Statement of Validation:}
%We first need to define what it means for an object to be valid:
An object is \emph{valid} iff calling its \Q@.validate()@ method would
deterministically produce \Q@true@ in a finite number of steps, i.e. it does not evaluate to \Q@false@, fail to terminate, or produce an error.
We also require that evaluating \Q@.validate()@ preserve existing memory ($\sigma$), but new objects ($\sigma'$) can be created and freely mutated.

\noindent\textbf{Define} $valid(\sigma,l)$:\\*
\indent $\sigma | l.validate()\Rightarrow^+ \sigma,\sigma’ | \text{\Q@true@}$

\noindent In order for validation to be meaningful it needs to be possible for \Q@.validate()@ to potentially observe an invalid object. However, invalid objects should not be observed outside of \Q@.validate()@.
For this purpose we define the set of trusted steps, 
as the call to \Q@.validate()@ and the field accesses inside a monitor.
Note that just the single small-step reduction
of calling \Q@.validate()@ is trusted, not the whole evaluation of the \Q@.validate()@ expression.


\noindent\textbf{Define} $\mathit{trusted}(\ctx_v,r_l)$:\\*
\indent either
$r_l=l$\Q@.validate()@ and
 $\ctx_v=\ctx_v'[$\Q@M(@$l$\Q@;@$v$\Q@;@$\square$\Q@)@$]$\\*
\indent or
$r_l=l$\Q@.f@ and
 $\ctx_v=\ctx_v'[$\Q@M(@$l$\Q@;@$v$\Q@;@$\ctx_v''$\Q@)@$]$

\noindent Finally, we can now define what it means for a language to soundly enforce validation: every object involved in any untrusted redex is valid.

\begin{theorem}[Sound Validation]
if $c:\Kw{Cap};\emptyset\vdash \e: \T$ and
$c\mapsto\Kw{Cap}\{\_\}|\e\rightarrow^+ \sigma|\ctx_v[r_l]$, then
either $valid(\sigma,l)$ or $\mathit{trusted}(\ctx_v,r_l)$.
\end{theorem}

We believe this property captures very precisely our statement in Section~\ref{s:validation}.
The proof is in Appendix~\ref{s:proof}. 
%The structure of the proof is interesting:
%It is hard to prove Sound Validation directly,
%so we first define a stronger property,
%called Stronger Sound Validation and
%we show that it is preserved during reduction by mean of conventional Progress and Subject Reduction.
%That is,
%Progress+Subject Reduction $\Rightarrow$ Stronger Sound Validation
%and Stronger Sound Validation $\Rightarrow$ Sound Validation.
\section[Related Work]{Related Work}
\label{s:related}
\subheading{Reference Capabilities}
We rely on a combination of RCs supported by at least 3 languages/lines of research:
L42~\cite{ServettoZucca15,ServettoEtAl13a,JOT:issue_2011_01/article1,GianniniEtAl16},
Pony~\cite{clebsch2015deny,clebsch2017orca}, and Gordon \etal~\cite{GordonEtAl12}.
%each of these works is accompanied by proofs about the properties of those modifiers.
%\IOComm{The Pony ones have no proofs!}
%Since such proofs have already been done,
%in this work we just assume the required properties.
They all support full/deep interpretation (see page 5), without back doors.
Former work~\cite{Boyland10,boyland2003checking,Hogg91,Smith:2000:AT:645394.651903,DBLP:conf/pldi/AikenFKT03} (which eventually enabled the work of Gordon \etal)  does not consider promotion and 
infers uniqueness/isolation/immutability only when starting from references that have been tracked with restrictive annotations along their whole lifetime.
Other approaches like Javari~\cite{TschantzErnst05,Boyland06}
and Rust~\cite{matsakis2014rust}
provide back doors, which are not easily verifiable as being used properly.
%Many approaches try to preserve purity ~\cite{pearce2011jpure}, but here we also need aliasing control.

Ownership~\cite{ClarkeEtAl13,ZibinEtAl10,DietlEtAl07} is a popular form of aliasing control often used as a building block for static verification~\cite{%
muller2002modular,%
barnett2011specification%
}.  However, ownership does not require the whole ROG of an object to be `owned'. This complicates restricting the data accessible by invariants.

\subheading{Object Capabilities}
In the literature, OCs are used to provide a wide range of guarantees, and many variations are present.
Object capabilities~\cite{RobustComposition}, in conjunction with reference capabilities, are able to
 enforce purity of code in a modular way, without requiring the use of monads.
L42 and Gordon use OCs simply to reason about I/O and non-determinism. This approach is best exemplified by Joe-E~\cite{finifter2008verifiable}, which is a self-contained and minimalistic language using OCs over a subset of Java in order to reason about determinism.
However, in order for Joe-E to be a subset of Java, they leverage a simplified model of immutability:
immutable classes must be final and have only final fields that refer to immutable classes.
%Instances of immutable classes are immutable objects.
In Joe-E, every method that only takes instances of immutable classes is pure.
Thus their model would not allow the verification of purity for invariant methods of mutable objects.
In contrast our model has a more fine grained representation of mutability: it is \emph{reference-based} instead of \emph{class-based}.
Thanks to this crucial difference, in our work every method taking only \Q@read@ or \Q@imm@ \emph{references} is pure, regardless of their class type; in particular, we allow the parameter of such a method to be mutated later on by other code.
%;both in the sense that no object visible outside of the method is mutated, but also that it is deterministic.

\subheading{Invariant protocols}
Invariants are a fundamental part of the design by contract methodology. 
Invariant protocols differ wildly and can be unsound or complicated, particularly due to re-entrancy and aliasing~\cite{leino2004object,drossopoulou2008unified,meyer2016class}. 

While invariant protocols all check and assume the invariant of an object after its construction, they handle invariants differently across object lifetimes; popular approaches include:%
%literature on class invariant accepts that sometime the object invariant may not hold,
%and that is exacerbated because of 
%Leino, K. R. M. and Müller, P.: Object Invariants in Dynamic Contexts (ECOOP), 2004.
%S. Drossopoulou and A. Francalanza and  P. Müller and A. J. Summers: A Unified Framework for Verification Techniques for Object Invariants ECOOP 2008. 
%There are different options as to what object-invariants are known to hold:
\begin{itemize}
\item The invariants of objects in a \textit{steady} state are known to hold: that is when execution is not inside any of the objects' public methods~\cite{Gopinathan:2008:RMO:1483018.1483028}. Invariants need to be constantly maintained between calls to public methods.%~\cite{WikiInvariant}.
\item 
%\LINE
The invariant of the receiver before a public method call and at the end of every public method body needs to be ensured. The invariant of the receiver at the beginning of a public method body and after a public method call can be assumed~\cite{Burdy2005,drossopoulou2008unified}.  
Some approaches ensure the invariant of the receiver of the \emph{calling} method, rather than the \emph{called} method~\cite{DBLP:journals/scp/MullerPL06}.
JML~\cite{JML} relaxes these requirements for helper methods, whose semantics are the same as if they were inlined.
%\LINE
%The invariant of the receiver (some approaches require the invariant of 'this' instead~\cite{?}) before a public (or non-helper~\cite{JML}) method call, and at the end of every method body needs to be ensured. The invariant of the receiver at the beginning of a public method body, and after a public method call can be assumed~\cite{Burdy2005,drossopoulou2008unified}.  


\item The same as above, but only for the bodies of `selectively exported' (i.e. not instance-private) methods, and only for `qualified' (i.e. not \Q@this@) calls~\cite{meyer2016class}.
\item The invariant of an object is assumed only when a contract requires the object be `packed'. It is checked after an explicit `pack' operation, and objects can later be `unpacked'~\cite{DBLP:journals/jot/BarnettDFLS04}.
%\url{https://en.wikipedia.org/wiki/Class_invariant}}; %\item
%constantly maintained when the object is \textit{closed};
%invariant can be manually opened and closed by using special operations; % Add cite here!
\end{itemize}\SS\LS[0.5]
\noindent These different protocols can be deceivingly similar.
Note that all those approaches fail our strict requirements and allow for broken objects to be observed.
Some approaches like JML suggest verifying a simpler approach (that method calls preserve the invariant of the \emph{receiver}) but assume a stronger one (the invariant of \emph{every} object, except \Q@this@, holds).


% use the unsound option of assuming one protocol, but actually checking another.

%DONE IN INTRO breaking class invariants = bug in class code
%braking validation= DEPEND.

%To encode this range of invariant semantics
%in our approach we can add a boolean \Q@isOpen@ field and add \Q@this.isOpen || ..@
%in front of the validity condition.
%Validation can be used to manually encode complex scenarios,
%for example if a method called on an object needs to break the invariant of another object,
%it can do so by manually setting the \Q@isOpen@ flag on the other object.


%On ownership verification
%Peter Mueller and Arnd Peotzsch Heffter,  eg Müller, P.: Modular Specification and Verification of Object-Oriented Programs, 2002.
%M. Barnett and M. Fähndrich and K. R. M. Leino and P. Müller and W. Schulte and H. Venter: Specification and Verification: The Spec# Experience. Communications of the ACM, 2011.
\newcommand\sepItems{\saveSpace\saveSpace\saveSpace\\*${}_{}$\\*${}_{}\,\bullet\,$}
%
%\LINE
\subheading{Security and Scalability}
Our approach allows verifying an object's invariant independently of the execution context.
This is in contrast to the main strategy of static verification: to verify a method, the system assumes the contracts of other methods, and the content of those contracts is the starting point for their proof.
Thus, static verification proceeds like a mathematical proof: a program is valid if it is all correct, but a single error invalidates all claims. This makes it hard to perform verification on large programs, or when independently maintained third party libraries are involved.
%This is less problematic with a type system, since its properties are more coarse grained, simpler and easier to check.
 Static verification has more flexible and fine-grained annotations and often relies on a fragile theorem prover as a backend.

%\REVComm{
%To solve this issue, static verification systems are %starting to
%}{2}{[is this correct?] verification of reference %monitors, gradual typing, and contracts have been %explored for longer}
To soundly verify code embedded in an untrusted environment, as in gradual typing~\cite{DBLP:conf/oopsla/TakikawaSDTF12,DBLP:conf/popl/WrigstadNLOV10}, it is possible to 
consider a verified core and a runtime verified boundary.
One can see our approach as an extremely modularized	version of such a system: every class is its
own verified core, and the rest of the code could have Byzantine behaviour. Our formal proofs show that every class that compiles/type checks is 
soundly handled by our protocol, independently of the behaviour of code that uses such class or any other surrounding code.

Our approach works both  in a library setting and with the open world assumption.
Consider for example the work of Parkinson~\cite{parkinson2007class}: he verified a property of the \Q@Subject/Observer@ pattern. However, the proof relies on (any override of) the \Q@Subject.register(Observer)@ method respecting its contract. Such assumption is unrealistic in a real-world system with dynamic class loading, and could trivially be broken by a user-defined \Q@EvilSubject@: checking contracts at load time is impractical and is not done by any verification systems we know of.

\subheading{Static Verification}
AutoProof~\cite{DBLP:conf/fm/PolikarpovaTFM14} is a static verifier for Eiffel that also follows the Boogie methodology, but extends it with \emph{semantic collaboration} where objects keep track of their invariants' dependencies using ghost state.

Dafny~\cite{DBLP:conf/sigada/Leino12} is a new language where all code is statically verified. It supports invariants with its \Q!{:autocontracts}! annotation, which treats a class's \Q!Valid()! function as the invariant and injects pre and post-conditions following visible state semantics;
however it requires objects to be newly allocated (or cloned) before another object's invariant may depend on it.
Dafny is also generally highly restrictive with its rules for mutation and object construction, it also does not provide any means of performing 
non-deterministic I/O.

Spec\#~\cite{Barnett:2004:SPS:2131546.2131549} is a language built on top of C\#. It adds various annotations such as method contracts and class invariants. 
It primarily follows the Boogie methodology~\cite{DBLP:journals/tcs/NaumannB06} where (implicit) annotations are used to specify and modify the owner of objects and whether their invariants are required to hold. Invariants can be \emph{ownership} based~\cite{DBLP:journals/jot/BarnettDFLS04}, where an invariant only depends on objects it owns; or \emph{visibility} based~\cite{DBLP:conf/mpc/BarnettN04,DBLP:conf/ecoop/LeinoM04}, where an invariant may depend on objects it doesn't own, provided that the class of such objects know about this dependence. Unlike our approach, Spec\# does not restrict the aliases that may exist for an object, rather it restricts object mutation: an object cannot be modified if the invariant of its owner is required to hold. This 
allows invariants to query owned mutable objects whose ROG is not fully encapsulated. However as we showed in Section \ref{s:case-study}, it can become much more difficult to work with and requires significant annotation, since merely having an alias to an object
is insufficient to modify it or call its methods.
%tells you nothing about it, hindering
%modification and method calls.
Spec\# also works with existing .NET libraries by annotating them with contracts, however such annotations are not verified. Spec\#, like us, does perform runtime checks for invariants and throws unchecked exceptions on failure.  However Spec\# does not allow soundly recovering from an invariant failure, since catching unchecked exceptions in Spec\# is intentionally unsound.~\cite{Leino2004ExceptionSF}




%Spec\# is statically verified wheras we rely on a type system: we have 4 type-modifiers that can be applied anywhere a type can be used (like a variable declaration) and the type-system uses a small set of fixed deterministic rules.
%Wheras the static-verification aproach has much more flexible and fine-grained annotations (with meth pre/post conditions) and uses a theoreom prover as a back-end, this can make it harder for users to program as it is not obvious whether the theorom prover will accept a program or not.
% In addition, the approach of a static-verifier can also be non-modular: changes to the body of one method could affect whether another is verified.

%many works on static verification, such as thoser Spec\#~\cite{??}


\subheading{Specification languages}
Using a specification language based on the mathematical metalanguage and different from the programming language's semantics may seem attractive, since it can express uncomputable concepts, has no mutation or non-determinism, and is often easier to formally reason about.
However, a study~\cite{chalin2007logical} discovered that developers expect specification languages to follow the semantics of the underling language, including short-circuit semantics and arithmetic exceptions; thus for example \Q@1/0@\,\Q@||@\,\Q@2>1@ should not hold, while \Q@2>1@\,\Q@||@\,\Q@1/0@ should, thanks to short circuiting.
This study was influential enough to convince JML to change its interpretation of logical expressions
accordingly~\cite{chalin2008jml}.
Dafny~\cite{DBLP:conf/sigada/Leino12} uses a hybrid approach: it has mostly the same language for both specification and execution. Specification (`ghost') contexts can use uncomputable constructs such as universal quantification over infinite sets, whereas runtime contexts allow mutation, object allocation and print statements. The semantics of shared constructs (such as short circuiting logic operators) is the same in both contexts.
Most runtime verification systems, such as ours, use a metacircular approach: specifications are simply code in the underlying language. Since specifications are checked at runtime, they are unable to verify uncomputable contracts.

Ensuring determinism in a non-functional language is challenging. Spec\# recognizes the need for purity/determinism when method calls are allowed in contracts~\cite{barnett200499} `\emph{There are three main current approaches: a) forbid the use of functions in specifications, b) allow only provably pure functions, or c) allow programmers free use
	of functions. The first approach is not scalable, the second overly restrictive and
	the third unsound}'.
	They recognize that many tools unsoundly use option (c), such as AsmL~\cite{barnett2003runtime}.
Spec\# aims to follow (b) but only considers non-determinism caused by memory mutation, and allows other non deterministic operations, such as I/O and random number generation. In Spec\# the following verifies:\lstset{morekeywords={assert, bool}}
%\saveSpace\saveSpace\begin{lstlisting}
\Q![Pure] bool uncertain() {return new Random().Next() %$$ 2 == 0;}!\\*
%\end{lstlisting}
And so \Q@assert uncertain() == uncertain();@ also verifies, but randomly fails with an exception at runtime.
As you can see, failing to handle non-determinism jeopardises reasoning.

A simpler and more restrictive solution to these problems is to restrict `pure' functions so that they can only read final fields and call other pure functions. This is the approach used by~\cite{Flanagan06hybridtypes}. One advantage of their approach is that invariants (which must be `pure') can read from a chain of final fields, even when they are contained in otherwise mutable objects. However their approach completely prevents invariants from mutating newly allocated objects, thus greatly restricting how computations can be performed.

Our approach also have connection with runtime verification. See Appendix \ref{s:runtime-verification} for related work on runtime verification.

%allowing mutation can cause specifications to affect the operational behaviour of code, which is against their purpose.

% They propose a concept of observational purity, that if completely fleshed out could possibly be a great addition to our proposed system. We speculate that some  primitive language support may be needed, for example implementing the Flyweight pattern as part of the language semantics.


% The most friendly aattractive to specify contracts (such as class invariants) in the same source language as the code. This common approach is used by most runtime verification systems, like our work, Eiffel and D. However their



% What's wrong with using the same language for both specification and implementation?
% What are peoples approaches



%${}_{}$\sepItems









%%%%%%%%%%%%%%%%%%%%%%%%%%%%%%%%%%%%%%%%%%%%%%%%%%%%%

%1 aliasing control
%  example hamster can be broken with those 2 lines
%2 I/O /determinism
%  hamster EvilPoint with random equal is accepted
%3 exceptions
%  spec sharp is happy to be unsound with capturing %unchecked exceptions
%---
%*TMs, OCs
%
%*expand on invariant protocol
%
%*RV tool
%------------
%*spec# unsond, parkinson critique, and static %verification is like math proof
%
%*Soundness or not
%
%*C#purity, dedicate spec language



%\noindent\textit{Theorem provers and SAT solvers}
%Rather than providing a simple set of rules as to what a \Q@validate@ method can contain,
%and where to insert calls to it, we could instead rely on implementation-specific static analysis:
% in which a \Q@validate@ method is valid iff the compiler can prove that it is deterministic
% and that it’s generated \Q@validate()@ calls are sufficient to enforce validation.
%Though approaches like this are frequently used such as with unifying Java’s generic-wildcards [], Rust’s ‘borrow checker’, …; we believe that would not produce a good result for our purposes: 
%\begin{itemize}
%	\item it would mean that a programmer would have no way of telling whether their code would compile, in particular code compiling would depend on the specific compiler (version) used.
%	\item the runtime cost of validation would be completely unpridictibable; since it is deterministic there is nothing stopping the compiler from calling \Q@validate@ any number of times, and at any point in time.
%	\item When a validation error could be throw would likewise be unpredictable, though it should happen after an object is made invalid\footnote{technically our definition of validation technically allows the error to happen sooner, as long as it’s not too late; however pre-emptive errors like this would be extremely hard to debug}, it could happen any time before it’s use. Making matters worse, if multiple object’s would be invalidated before either is used, which one’s error would be thrown is unconstrained
%	\item This approach will not work well in the pressence of dynamic code loading, in particular it woud likley significantly slow down such loading or spurioslly fail depending on what other code has been loaded
%\end{itemize}

\section{Conclusions and Future Work}
Our approach follows the principles of \emph{Offensive programming}
~\cite{stephens2015beginning}, where 
no attempt to fix or recover the invalid object is performed and
%	\begin{itemize}
%\item
 the failure (an unchecked exception)
		is raised close to the defect: the method directly calling the operation breaking the invariant is still on the stack trace.
%}{3}{[meaning] is not clear} (the operation creating an invalid object), i.e. we ``fail-fast''.    
%		\item
%	\end{itemize}


%The aim of our work is only to enforce object invariants, so we do not present complexities unnecessary for this purpose.
Our work builds on TMs and OCs.
Their popularity is growing, and we expect future languages to support some variation of these.
Crucially, any language already designed with TMs and OCs
can also support our invariant protocol with minimal added complexity.


We demonstrate the applicability and simplicity of our approach by a GUI example.
Our invariant protocol perform several orders of magnitude less checks that the visible state semantic,
and requires much less annotations 
then Spec\#, (the most comparable system in this context).
In the appendix\ref{??}
we formalize our invariant protocol and we prove it sound. We do not formalise any specific type system, to stay parametric over the various existing type systems which provably enforce the properties we require for our proof (and much more).


One interesting avenue for future work would be
using invariants to encode pre and post conditions
as done by~\cite{??}:%dependent/refinement types
a method could be declared taking in input a record of arguments with invariant, and can return a wrapper over the arguments and the result.
Such approach may be quite verbose, but would ensure that the precondition on the argument holds for the whole execution of the method, instead of just holding at the beginning.

%It could be worthwhile formalising the minimal type system required by validation.



%However the restrictions we make to ensures that \Q@validate@ is deterministic, namely those the type-system enforces due to its signature, seem quite flexible and reasonable;

%%%%%examples of things that future work may investigate allowing are deterministic I/O and multi-threading. 


The language we presented here restricts the form of \validate{} and capsule mutator methods; in particular
our strong restrictions of capsule mutator methods
allows injection of \validate{} calls merely at the end of such methods.
While these restrictions do not interfere with simple
invariants, to verify complex mutable data-structures the box pattern is required.
However, we believe this pattern, although verbose, is simple and understandable. While it may be possible for a more
complex and fragile type system to reduce the need for the box pattern
 while still ensuring our desired semantics, we prioritize simplicity and generality.

%, however such a language is unlikely to be easily understood by programmers;
%being able to predict whether code would be well typed allows programmers
%to better take advantage of the language.

Directions that could be investigated to improve our work include the addition of syntax-sugar to ease the burden of the box and the transform patterns; type modifier inference, and support for flexible ownership types.

%Our work, in comparison to previous RV techniques,
%aims to be efficient by limiting the number of validation calls, however we have \REVComm{no empirical evaluation of our approach's performance}{3}{\label{CONTRA1}contradictory to [see footnote \ref{CONTRA2}]}.
%To improve efficiency it could be worth investigating elision of unnecessary validation calls
%or even only validating parts of objects (by running the part of \Q@validate@ that could fail).
%Conclusions? future work?
%@StrongExceptionSafety is 
%a very strong property,
%and some languages may be unwilling to commit to always preserve it.
%In particular, depending on the details of a specific language
% releasing resources as in \Q@finally@ blocks may require
%some relaxation of @StrongExceptionSafety. Sound releasing of resources could be interesting
%future work.

\section{Conclusions and Future Work}
\label{s:conclusion}
Our approach follows the principles of \emph{offensive programming}
~\cite{stephens2015beginning}\IODel{,} wher\IO{e:}
no attempt to fix or recover an invalid object is performe\IO{d, a}nd
%	\begin{itemize}
%\item
 failures (unchecked exceptions)
		are raised close to their cause: at the end of constructors creating invalid objects and immediately after field updates and instance methods that invalidate their receivers.

%}{3}{[meaning] is not clear} (the operation creating an invalid object), i.e. we ``fail-fast''.    
%		\item
%	\end{itemize}


%The aim of our work is only to enforce object invariants, so we do not present complexities unnecessary for this purpose.
Our work builds on a specific form of TMs and OCs, whose
popularity is growing, and we expect future languages to support some variation of these.
Crucially, any language already designed with such TMs and OCs
can also support our invariant protocol with minimal added complexity.


We demonstrated the applicability and simplicity of our approach with a GUI example.
Our invariant protocol performs several orders of magnitude less checks than visible state semantics, and requires much less annotation 
than Spec\#, (the system with the most comparable goals). In Section~\ref{s:formalism} we formalised our invariant protocol and in Appendix~\ref{s:proof} we prove it sound.
%In appendix~\ref{s:formalism} we formalise our invariant protocol and prove it sound. 
To stay parametric over the various existing type systems which provably enforce the properties we require for our proof (and much more), we do not formalise any specific type system.


% a method could be declared as taking a class whose invariant corresponds to the method's pre-condition,  and returning a class whose invariant corresponds to the pos-condition.


% Such approach may be quite verbose, but would ensure that the precondition on the argument holds for the whole execution of the method, instead of just holding at the beginning.

%It could be worthwhile formalising the minimal type system required by validation.



%However the restrictions we make to ensures that \Q@validate@ is deterministic, namely those the type-system enforces due to its signature, seem quite flexible and reasonable;

%%%%%examples of things that future work may investigate allowing are deterministic I/O and multi-threading. 

The language we presented here restricts the forms of \Q@invariant@ and capsule mutator methods;
such strong restrictions allow for sound and efficient injection of invariant checks. 
\IOBlock{Merge this and the next paragraph \& compact them!}{These restrictions do not get in the way of writing invariants over immutable data, but the box pattern is required for verifying complex mutable data structures. We believe this pattern, although verbose, is simple and understandable. While it may be possible for a more complex and fragile type system to reduce the need for the pattern whilst still ensuring our desired semantics, we prioritize simplicity and generality. }

\IOBlock{Merge into the previous paragraph \& compact}{In order to obtain safety, simplicity, and efficiency we traded some expressive power:
the \Q@invariant@ method can only refer to immutable and encapsulated state.
This means that while we can easily verify that a doubly linked list of immutable elements
is correctly linked up,
we can not do the same for a doubly linked lists of mutable elements. Our approach does not prevent correctly implementing such data structures, but the \Q@invariant@ method would be unable to access the list's nodes, since they would contain \Q@mut@ references to shared objects.
In order to verify such data structures we could add a special kind of field which cannot be (transitively) accessed by invariants; such fields could freely refer to any object. We are however unsure if such complexity would be justified.} \IOComm{Mention flexible ownership types as a potential solution? (Assuming it is)}

% To verify those data-structures, in future work
% we may investigate a special kind of field that
% could be accessed only using a \Q@mut@ receiver.
% Such fields would be allowed to refer to not encapsulated state, 
% and they would be unreachable from the invariant code,that starts from a \Q@read this@.

% \LINE
% The language we presented here restricts the form of \Q@invariant@ and capsule mutator methods. 
% We have shown that such restrictions, albeit strong, allow sound and efficient injection of invariant checks. 
% While our restrictions do not hamper writing invariants over immutable data, invariants over complex mutable % data require the box pattern.  We believe the box pattern, although verbose, is simple and understandable, but % could be improved with syntax sugar.

% Our goals of simplicity and efficiency come at the cost of expressivity: we are unable to express invariants % over non-encapsulated mutable structures, 
% though a more complex and fragile type system may reduce such limitations. However we believe we have % demonstrated that our limitations are not too severe and that we have achieved our goals.
% \LINE

%, however such a language is unlikely to be easily understood by programmers;
%being able to predict whether code would be well typed allows programmers
%to better take advantage of the language.

For an implementation of our work to be sound, catching exceptions like stack overflows or out of memory
cannot be allowed in \Q@invariant@ methods, since they are not deterministically thrown.
%For an implementation of our work to be sound, non-deterministic exceptions like stack overflows or out of memory
%errors cannot be caught in invariants.
%this
%use exception catching as a non deterministic conditional choice, 
%allowing non deterministic behaviour.
Currently L42 never allows catching them, however we could also write a (native) capability method (which can't be used inside an invariant) that enables catching them. Another option worth exploring would be to make such exceptions deterministic, perhaps by giving invariants fixed stack and heap sizes.

Other directions that could be investigated to improve our work include the addition of syntax sugar to ease the burden of the box and the transform patterns; type modifier inference\IODel{, and support for flexible ownership types}.

%Our work, in comparison to previous RV techniques,
%aims to be efficient by limiting the number of validation calls, however we have \REVComm{no empirical evaluation of our approach's performance}{3}{\label{CONTRA1}contradictory to [see footnote \ref{CONTRA2}]}.
%To improve efficiency it could be worth investigating elision of unnecessary validation calls
%or even only validating parts of objects (by running the part of \Q@validate@ that could fail).

\begin{comment}
In literature, both static and runtime verification discuss
the correctness of common programming patterns in conventional languages.
Their struggle is proof of how hard it is to deal with the expressive power of unrestricted imperative object-oriented programming.
 Here instead we build on languages using TMs and OCs to tame the use of imperative features. In this way
we have a fresh start where static variables are disallowed, unchecked exceptions require care to be captured, and I/O is allowed only when an opportune capability object is reachable.
Following those restrictions allow simpler reasoning.
The philosophy of our approach is to be like an extended type system: 
It is the programmer's decision
to annotate a field with a certain type,
or the class with a certain \validate.
If the program is well-typed, they are not questioned in their intent.
During execution the system is solely responsible for soundly enforcing the invariant protocol.
This is in sharp contrast with most work in RV, that is often conceived more as a tool to ease debugging:
both deciding properties and enforcing them is controlled by the programmers.
This is also different from static verification,
%where the properties are ensured instead of enforced.
where the properties are ensured ahead of time instead of being enforced during execution.
%Static verification is very heavy weight, and often impractical/restrictive.

%Both static and runtime verification
%aim to monitor a wide range of properties; to this aim they accept a 
%great deal of complexity, and require the programmer to develop a deep understanding
%over the behaviour and the structure of code.
%For example, the specification of method’s pre and post-conditions
%encode a generalization of the program behaviour in the dedicated specification language.
%This means that, even in the best case scenario, 
%using pre/post-conditions the user is required to specify the program semantics twice:
%first in the specification language and then in the underlying programming language.
%In comparison, our approach aims to only verify conditions on immutable or well encapsulated state.
%This makes our approach \emph{ultra-lightweight}:
%the programmer specifies only the desired \Q@invariant@ method.

Moreover, our approach does not aim to replace static or run-time verification,
but is a building block they can rely upon.
\end{comment}

%\noindent\textit{Our approach:}


\printbibliography
\appendix
\section{Proof} 
\label{s:proof}

\begin{theorem}[Sound Validation]
	if $c:\Kw{Cap};\emptyset\vdash \e: \T$ and
	$c\mapsto\Kw{Cap}\{\_\}|\e\rightarrow^+ \sigma|\ctx_v[r_l]$, then
	either $valid(\sigma,l)$ or $\mathit{trusted}(\ctx_v,r_l)$.
\end{theorem}

We believe this property captures very precisely our statement in Section~\ref{s:validation}.

It is hard to prove Sound Validation directly,
so we first define a stronger property,
called \emph{Stronger Sound Validation} and
show that it is preserved during reduction by means of conventional 
Progress and Subject Reduction (Progress is one of our assumption,
while Subject Reduction relies heavily on SubjectReductionBase).
That is,
Progress+Subject Reduction $\Rightarrow$ Stronger Sound Validation,
\\*and Stronger Sound Validation $\Rightarrow$ Sound Validation.

\subsection{Stronger Sound Validation $\Rightarrow$ Sound Validation}

Stronger Sound Validation depends on 
$\mathit{wellEncapsulated}$, $\mathit{monitored}$
and $OK$:

\noindent\textbf{Define} $\mathit{wellEncapsulated}(\sigma,\e,l_0)$:\\*
\indent$\forall l \in \mathit{erog}(\sigma,l_0), \text{not}\ \mathit{mutatable}(l,\sigma,\e)$

\noindent The main idea is that an object is well encapsulated if its encapsulated state is safe from
modification. 

\noindent\textbf{Define} $\mathit{monitored}(\e,l)$:\\*
\indent$\e=\ctx_v[M(l;\e_1;\e_2)]$ and either $\e_1=l$ or $l$ is not inside $\e_1$.

\noindent An object is monitored if the execution
is currently inside of a monitor for that object, and
the monitored expression $\e_1$ does not
contains $l$ as a \emph{proper} subexpression.

A monitored object is associated with an expression that can not observe it, but may 
reference its internal representation directly.
In this way, we can safely modify its representation before checking for the invariant.

The idea is that at the start the object will be valid and $\e_1$ will contain $l$;
but during reduction, the $l$ reference will be used in order to
give access to the internal state of $l$; only after that moment, the object may become invalid.


\noindent\textbf{Define} $OK(\sigma,e)$:\\
\indent $\forall l\in\dom(\sigma)$
  either\\
\indent\indent 1. $\mathit{garbage}(l,\sigma,\e)$\\
\indent\indent 2. $\mathit{valid}(\sigma,l)$ and $\mathit{wellEncapsulated}(\sigma,\e,l)$\\
\indent\indent 3. $\mathit{monitored}(\e,l)$

Finally, the system is in a valid state with respect to validation
if for all the objects in the memory, one of these 3 cases apply:
%the class of the object has no invariant method;
the object is not (transitively) reachable from the expression (thus can be garbage collected);
the object is valid, and the object is encapsulated;
or the object is currently monitored.

\begin{theorem}[Stronger Sound Validation]
if $c:\Kw{Cap};\emptyset\vdash \e_0: \T_0$ and
$c\mapsto\Kw{Cap}\{\_\}|\e_0\rightarrow^+ \sigma|\e$, then
$OK(\sigma,\e)$
\end{theorem}
\noindent Starting from only the capability object,
any well typed expression $\e_0$ can be reduced for an arbitrary amount of steps,
and $OK$ will always hold.
\\
\begin{theorem} Stronger Sound Validation $\Rightarrow$ Sound Validation
\end{theorem}
\begin{proof}
\noindent By Stronger Sound Validation, each $l$ in the current redex must be $OK$:
\begin{enumerate}
	\item If $l$ is garbage, it cannot be in the current redex, a contradiction.
	\item If $\mathit{valid}(\sigma,l)$, then $l$ is valid, so thanks to Determinism
	no invalid object could be observed.
	\item Otherwise, if $\mathit{monitored}(\e,l)$ then either:
	\begin{itemize}
	 \item we are executing inside of $\e_1$ thus the current redex is inside of a sub-expression of the monitor that does not contain $l$, a contradiction.
	 \item or we are executing inside $\e_2$:
	 by our reduction rules, all monitor expressions start with 
	 $\e_2=l$\Q@.validate()@, thus the first execution step
	 of $\e_2$ is trusted. Following execution steps are also trusted, since by well formedness the body of invariant methods only use \Q@this@ (now translated to $l$) to access fields.
	\end{itemize}
\end{enumerate}
In any of the possible cases above, Sound Validation holds for $l$, and so it holds for all redexes.
\end{proof}

\subsection{Subject Reduction}

\noindent\textbf{Define} $\text{fieldGuarded}(\sigma,\e)$:\\*
\indent$\forall \ctx$ such that $\e=\ctx[l\singleDot\f] $
and $\Sigma^\sigma(l).f=\Kw{capsule}\,\_$, and $\f\mathrel{\mathit{inside}} \Sigma^\sigma(l).\mathit{validate}$\\*
\indent\indent either 
$\forall T, \forall C, \Sigma^\sigma;\x:\Kw{mut}\,C\,\not\vdash\ctx[\x]:T$, or\\*
\indent\indent $\ctx=\ctx'[$\Q@M(@$l$\Q@;@$\ctx''$\Q@;@$\e$\Q@)@$]$ and $l$ is contained exactly once in $\ctx''$

That is, all \emph{mutating} capsule field accesses are individually guarded by monitors.
Note how we use $C$ in $\x:\Kw{mut}\,C$ to guess the type of the accessed field,
and that we use the full context $\ctx$ instead of the evaluation context $\ctx_v$
to refer to field accesses everywhere in the expression $\e$.


\begin{theorem}[Subject Reduction]
if $\Sigma^{\sigma_0};\emptyset\vdash e_0: T_0$,
$\sigma_0|e_0\rightarrow \sigma_1|e_1$,
$OK(\sigma_0,\e_0)$
and
$\mathit{fieldGuarded}(\sigma_0,\e_0)$
then
$\Sigma^{\sigma_1};\emptyset\vdash e_1: T_1$,
$OK(\sigma_1,e_1)$ and
$\mathit{fieldGuarded}(\sigma_1,\e_1)$
\end{theorem}

\begin{theorem}
	Progress + Subject Reduction $\Rightarrow$ Stronger Sound Validation
\end{theorem}
\begin{proof}
This proof proceeds by induction in the usual manner.

\emph{Base Case}: At the start of the execution, the memory is going to only contain $c$: since $c$ is defined to be initially $\mathit{valid}$, and has only \Q@mut@ fields, and so it is trivially $\mathit{wellEncapsulated}$, thus $OK(c\mapsto\Kw{Cap},e)$.

\emph{Induction}: By Progress we always have another evaluation step to take, by Subject Reduction such a step will preserve $\mathit{OK}$, and so by induction $\mathit{OK}$ holds after any number of steps.

Note how for the proof garbage collection is important: 
when the \Q@validate()@ method evaluates to \Q@false@, 
execution can continue only if the offending object is classified as garbage.
\end{proof}

\subsection{Expose Instrumentation}
We first introduce a lemma derived from well formedness and the type system:
\begin{Lemma}[ExposerInstrumentation]
If $\sigma_0 | \e_0\rightarrow \sigma_1 |\e_1$ and
$\text{fieldGuarded}(\sigma_0,\e_0)$
\\*
then $\text{fieldGuarded}(\sigma_1,\e_1)$
\end{Lemma}
\begin{proof}
The only rule that can 
introduce a new field access is \textsc{mcall}.
In that case, ExposerInstrumentation holds
by well formedness (all field accesses in methods are of the form \Q@this.f@) 
and since \textsc{mcall} inserts a monitor while invoking capsule mutator methods, and not field accesses themselves. If however the method is not a \Q@mut@ method but still accesses a capsule field, by MutField such a field access expression cannot be typed as \Q@mut@ and so no monitor is needed.

Note that \textsc{monitor exit} is fine because monitors are removed only when
 $e_1$ is a value.
\end{proof}

\subsection{Proof of Subject Reduction}
Any reduction step can be obtained
by exactly one application of rule \textsc{ctx} and then one other rule. Thus the proof can simply proceed by cases on such other applied rule.

By SubjectReductionBase and ExposerInstrumentation, 
$\Sigma^{\sigma_1};\emptyset\vdash e_1: T_1$ and  $\mathit{fieldGuarded}(\sigma_1,\e_1)$. So we just need to proceed by cases on the reduction rule applied to verify that $OK(\sigma_1,\e_1)$ holds:


\begin{enumerate}
\item (\textsc{update}) $\sigma|l\singleDot f\equals v\rightarrow \sigma'|\e'$:
	\begin{itemize}
	  \item By \textsc{update} $\e'=\Kw{M}\oR l;l;l\singleDot\text{validate}\oR\cR\cR$, thus $\mathit{monitored}(\e,l)$.
	  \item Every $l_1$ such that $l\in \mathit{rog}(\sigma,l_1)$ will verify the same case as the former step:
	  \begin{itemize}
	  	\item If it was $\mathit{garbage}$, clearly it still is.
	  	\item If it was $\mathit{monitored}$, it also still is.
	    \item Otherwise it was $\mathit{valid}$ and $\mathit{wellEncapsulated}$:
			\begin{itemize}
				\item If $l\in \mathit{erog}(\sigma,l_1)$ we have a contradiction since $mutatable(l, \sigma, e)$, (by MutField)
		    	\item Otherwise, by our well-formedess criteria that \Q@.validate()@ only accesses \Q@imm@ and \Q@capsule@ fields, and by Determinism it is clearly the case that $\mathit{valid}$ still holds;
				By HeadNotCircular it cannot be the case that $l\in \mathit{erog}(\sigma',l_1)$ and so $l_1$ is still $\mathit{wellEncapsulated}$.
		  	\end{itemize}
	  \end{itemize}
	  \item Every other $l_0$ is not in the reachable object graph of $l$
	  thus it being $\mathit{OK}$ could not have been effected by this reduction step.
	\end{itemize}

\item (\textsc{access}) $\sigma|l\singleDot f \rightarrow \sigma|v$:
	\begin{itemize} 
		\item If $l$ was $valid$ and $wellEncapsulated$:
		\begin{itemize}
			\item If we have now broken $wellEncapsulated$ we must have made something in its $erog$  $mutatable$. As we can only type \Q@capsule@ fields as \Q@mut@ and not \Q@imm@ fields, by FieldMut we must have that $f$ is \Q@capsule@ and $l\singleDot f$ is being typed as \Q@mut@. By $\mathit{fieldGuarded}(\sigma_0,\e_0)$ the former step must have been inside a monitor \Q@M(@$l$\Q@;@$\ctx_v[l$\Q@.f@$]$\Q@;@$\e$\Q@)@
		    and the $l$ under reduction was the only occurrence of $l$.
		    Since $f$ is a capsule, we know that $l\notin \text{erog}(\sigma,l)$
		    by HeadNotCircular. Thus in our new step $l$ is not $inside$ $\ctx_v[v]$. Thus $l$ must be $monitored$ and hence it is $OK$.
		    
		    \item Otherwise, $l$ is still $OK$
    	\end{itemize}

		\item Nothing that was $\mathit{garbage}$ could have been made reachable by this expression, since the only value we produced was $v$ and it was reachable through $l$ (and so could not have been garbage), thus $garbage$ is still $OK$.
		
		\item As we don’t change any monitors here, nothing that was $monitored$ could have been made un-$monitored$, and so it is still $OK$.
		
		\item Suppose some $l_0$ was $wellEncapsulated$ and $valid$:
		\begin{itemize}
			\item If $l$ was in the $rog$ of $l_0$, by CapsulaeTree, if $l$ was in the $erog$ of $l$, then $v$ can only be reached from $l_0$ by passing through $l$, and so we could not have made $l_0$ non-$wellEncapsulated$. In addition, since only things in the $erog$ can be referenced by $\singleDot\Kw{validate}\oR\cR$, $l_0$’s validity can not depend on $l$, and by Determinism it is still the case that $l_0$ is $valid$. And so we can’t have effected $l_0$ being $OK$.
			\item Otherwise this reduction step could not have affected $l_0$ so $l_0$ is still $OK$.
		\end{itemize}
\end{itemize}

\item (\textsc{mcall}, \textsc{try enter} and \textsc{try ok}):

	These reduction steps do not modify memory, nor do they modify the memory-locations reachable inside of main-expression, nor do they modify any monitor expressions. Therefore it cannot have any effect on the $garbage$, $wellEncapsulated$, $valid$ (due to Determinism) or $monitored$ properties of any memory locations, thus $\mathit{OK}$ still holds.

\item (\textsc{new}) $\sigma|\Kw{new}\ C\oR\vs\cR\rightarrow \sigma,l\mapsto C\{\vs\}| \Kw{M}\oR l;l;l\singleDot\text{validate}\oR\cR\cR$:

	Clearly the newly created object ($l$) is monitored. As for \textsc{mcall}, other objects and properties are not disturbed, and so $\mathit{OK}$ still holds.


\item (\textsc{monitor exit}) $\sigma|\Kw{M}\oR l; v;\Kw{true}\cR\rightarrow \sigma|v$:
\begin{itemize}
	\item As monitor expressions are not present in the original source code, it must have been introduced by \textsc{update}, \textsc{mcall}, or \textsc{new}. In each case the 3\textsuperscript{rd} expression started of as $l\singleDot\Kw{validate}\oR\cR)$, and it has now (eventually) been reduced to $\Kw{true}$, thus by Determinism $l$ is $valid$.

	\item  If the monitor was introduced by \textsc{update}, then $v = l$. We must have had that $l$ was well encapsulated before \textsc{update} was executed (since it can’t have been garbage and $monitored$), as \textsc{update} itself preserves this property and we haven’t modified memory in anyway, we must still have that $l$ is $wellEncapsulated$. As $l$ is $valid$ and $wellEncapsulated$ it is $OK$.

	\item If the monitor was introduced by \textsc{mcall}. Then it was due to calling a capsule-mutator method that mutated a field $f$.
	\begin{itemize}
		\item A location that was $garbage$ obviously still is, and so is also $OK$.
		\item No location that was $valid$ could have been made non-valid since this reduction rule performs no mutation of memory. If a location was $wellEncapsulated$ before the only way it could be non-$wellEncapsulated$ is if we somehow leaked a \Q@mut@ reference to something, but by our well-formedness rules $v$ cannot be typed as \Q@mut@ and so we can’t have affected $wellEncapsulated$, hence such thing is still $OK$.
		\item The only location that could have been made un-$monitored$ is $l$ itself. By our well-formedness criteria $l$ was only used to modify $l.f$, and we have no parameters by which we could have made $l.f$ non-$wellEncapsulated$, since that would violate CapsuleTree. As nothing else in $l$ was modified, and it must have been $wellEncapsulated$ before the \textsc{mcall}, it still is, and since  $l$ is valid, it is $OK$.
	\end{itemize}
	\item Otherwise the monitor was introduced by \textsc{new}. Since we require that \Q@capsule@ fields and \Q@imm@ fields are only initialised to \Q@capsule@ and \Q@imm@ expressions, by CapsuleTree the resulting value, $l$, must be $wellEncapsulated$, since $l$ is also $valid$ we have that $l$ is $OK$.

\end{itemize}

\item (\textsc{try error}) $\sigma,\sigma_0|\Kw{try}^\sigma\oC \mathit{error}\cC\ \Kw{catch}\ \oC\e\cC\rightarrow \sigma|\e$:

	By StrongExceptionSafety we know that $\sigma_0$ is $\mathit{garbage}$ with respect to $\ctx_v[\e]$. By our well-formedness criteria no location inside $\sigma$ could have been $monitored$.

	Since we don’t modify memory, everything in $\sigma_0$ is $\mathit{garbage}$ and nothing inside $\sigma$ was previously monitored, it is still clearly the case that everything in $\sigma$ is $\mathit{OK}$
\end{enumerate}
\end{document}