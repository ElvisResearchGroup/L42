In environments offering a powerful enough aspect-oriented support,
it could be possible to detect any field update in the whole ROG of
any object, following the strategy of 
Gopinathan et al.~\cite{Gopinathan:2008:RMO:1483018.1483028}.
Using such power would even be possible to roll-back the very field update that caused 
the invariant to fail, making the object coherent again.
However, we think this would be a \textbf{terrible idea causing unexpected strange behaviour}: for example
assume that we are moving object between two boats:
the overflowing object may be removed from the \Q@cargo@ of the second boat, but it would not
be placed back in the first boat. It would look like the object has disappeared.
The important point here is that the program would be in an unexpected state
even if no object invariants are violated, and this would happen \textbf{because} of the 
invariant checking/fixing behaviour, not because of code written by the programmer.

We believe that the only viable option is to detect violations after the fact.



%
%\saveSpace
%\section{Performance considerations and Practical experience}
%\saveSpace
%We believe our sound approach can monitor arbitrary programs
%for a fraction of the cost of many other competitors.
%Many other works%
%~\cite{feldman2006jose,fahndrich2010embedded,abercrombie2002jcontractor,tran2003design}
% check (that is, run)
%the invariant code at the start and at the end of every public
%method.
%In  our approach,
%we only call the invariant a single time for each mutator method.
%We expect this to result in at least 10 times reduction over
%the number of required invariant checks:
%for mutator methods, we call \Q@invariant()@ only one time instead of two,
%and we
%do not call it at all for non-mutator methods, that
% are usually more numerous and invoked more often.
%
%${}_{}$
%
%\noindent We are in the process of creating a practical language (\url{http://L42.is})
% where sound invariant checking can be enforced by decorating a class.
%In 42 the following code expresses the \Q@Boat@ example.
%\begin{lstlisting}
%Boat:Data<><{
%  S name,$\Comment{the default modifier is imm}$
%  capsule Items cargo,
%  Void load(Item item)
%    this.#cargo().add(item),
%  Item unload(String itemName)
%    this.#cargo().extract(itemName),
%  read method Bool #invariant()
%    !this.name().isEmpty() & this.cargo().weight()<=300Num,
%  }
%\end{lstlisting}
%\Q@Data@ takes that code literal and creates a \Q@Boat@ class with invariant.
%Moreover, \Q@equals@, \Q@toString@ and many other utility methods are generated/added.
%
%The 42 project is starting to reach maturity, and is showing that sound and practical
% run time verification of arbitrary invariants written in the language itself is possible. 
%In 42 we have only final classes and interfaces (code reuse is obtained by metaprogramming)
%This means that every variable of a class type hold an instance of that exact type, and 
%thus, thanks to SIC, exactly that class invariant will hold.
%
%

\saveSpace
\section{Related work}
\label{s:related}
\saveSpace
\noindent\textit{Type Modifiers:}
We show how a combination of
\Q@mut@, \Q@imm@, \Q@read@ and \Q@capsule@ provides the tools to ensure enough encapsulation in a sound way. 
It is responsibility of the language, and not of the programmer,
to verify that encapsulated objects actually have control of the mutation of their whole ROG.
This requires to use modifiers with a deep/strong interpretation, without back-doors.
This means that the approaches like ownership~\cite{ClarkeEtAl13,ZibinEtAl10,DietlEtAl07}
are not suitable since they aim to enforce the property only for
the annotated object and/or its owned objects.
Approaches like Javari~\cite{TschantzErnst05,Boyland06} and Rust~\cite{matsakis2014rust}
are also unsuitable since they introduce back-doors, and expect the programmer to use them properly.
Many approaches tries to just preserve purity (as for example~\cite{pearce2011jpure}), but here we need also aliasing control.


\noindent
\textit{Object Capabilities:}
Object capabilities, in conjunction with type modifiers seems to be able to
 enforce purity of sections of code in a friendly and modular way.
Here we presented this idea very informally.
The language Joe-E~\cite{finifter2008verifiable}
explores how to use object capabilities to ensure
pure behaviour for methods.
In order to express Joe-E as a subset of Java,
they leverage on a simplified model of immutability:
Immutable classes are final classes where all the fields are final and refers to other immutable classes.
Instances of immutable classes are immutable objects.
\textbf{In Joe-E every method taking in input only immutable objects is pure.}
Their model would not allow to verify purity for invariant methods of mutable objects
 (as Simple/Encapsulated objects).
In our model we have a more fine grained representation of immutable/read:
 \emph{reference based} instead of \emph{class based}.
\textbf{
In our model, every method taking in input only \Q@read@ or \Q@imm@ references is pure.
}
% We are surprised that we cannot find many formal works on the promising relation between monads and object capabilities.


\noindent\textit{Strong Exception Safety:}
Exception safety seems at first glance a smaller issue with respect 
to the other two, but is the final piece that lets the whole system work in a real world setting.
Note that state of the art type systems to enforce exception safety
 do not restrict code that do not capture errors, and
only the point of error capturing is constrained.


\noindent\textit{Dedicated specification language or underling language:}
Using a specification languages near to the logic and disjointed from a specific language
semantic may seems attractive, however
a study~\cite{chalin2007logical} discovered that developers expect
the specification language to use the semantic of the underling language, including
short circuit semantic and arithmetic exceptions; thus for example
\Q@1/0 || 2>1@
should not hold, while 
\Q@2>1 ||1/0@ should hold thanks to short circuit semantic.
This study was influential enough to convince JML to change its interpretation of logical expressions
accordingly~\cite{chalin2008jml}.
We believe this is evidence that using a method in the underlying language to encode the invariant is
a developers-friendly solution.


%With our great surprise we had difficulties finding much related work in the area 
%of soundly and automatically injecting run time invariant tests in class based, nominally typed
%object oriented languages.
%For example  

\noindent\textit{Ensuring class invariants:}
JML offers a tool (jmlrac~\cite{Burdy2005}) that is intentionally unsound,
while InvTS~\cite{gorbovitski08efficient} lets you write Python conditions
 that are verified on a set of Python objects, but the programmer needs to be able
 to predict which objects are in need of being checked and to use a simpler domain
 specific language to target them. That is, the programmer may make a mistake
while using this domain specific language, and the invariant checking
will not be triggered.

By looking to a survey by Voigt et al.~\cite{Voigt2013} and the extensive MOP project~\cite{meredith2012overview},
it seems that most RV works have a philosophy that is radically different from ours:
indeed RV systems and frameworks are often tools empowering the user
into implementing what kind of monitoring they see fit for their specific problem at hand.
This means that they are responsible for deciding, designing and encoding both the 
logic properties and the instrumentation criteria that will perform operations
in certain moments in time~\cite{meredith2012overview}.
In practice, this means that the logic and the instrumentation and the implementation end up connected:
a specific instrumentation strategy is good only to test certain logic properties in certain applications.
No guarantee is given that the implemented instrumentation strategy is able to support
the required logic in the monitored application.

In this work we define a language where a minimal, standardized,
efficient and completely general purpose instrumentation strategy can soundly verify any invariant
expressible as a\\* \Q@read method imm Bool invariant()@, for any well-typed program; with open world assumption
and possible Byzantine behaviour of any object in the system.

By seeing class invariant as a part of the type of the object,
the `RV tool' philosophy is akin to letting the programmer customize the behaviour of the
type system: the programmer implementation may be unsound; while our philosophy is
to give the user a way to represent complex and expressive types (in the form of arbitrary code in 
the \Q@invariant()@ method), but 
the type system implementation is fixed in stone by the language designer.

\newcommand\sepItems{\saveSpace\saveSpace\saveSpace\\*${}_{}$\\*$\bullet\,$}

\noindent\textit{Sound monitors:}
\noindent Many works attempts to move out of the `RV tool' philosophy to ensure RV Monitors work as expected. Here
we mention some of them:
\sepItems
In avionics, where memory allocation is disallowed, making reasoning about aliasing much simpler~\cite{laurent2015assuring}:
``\emph{Runtime Verification (RV) can act as the last line of defense to
protect the public safety, but only if the RV system itself is trusted.}''.
\sepItems
In domain specific languages~\cite{ferrari2002guardians}:
``\emph{Proof techniques for establishing security properties}''.
\sepItems
On assertions over restrictive domain specific languages, to tame some of the C/C++
undefined behaviour issues~\cite{agten2015sound}:
``\emph{no verified assertion in the verified
module will ever fail at runtime, even if the module runs as part of
a vulnerable application that is subject to code injection attacks}''.
\sepItems
Works on class based OO languages, but focusing only on pre and post conditions~\cite{findler2001contract}:
``\emph{we  study  the  problem  of  contract  enforcement in
an object-oriented world from  a foundational perspective.   More
specifically, we study contracts as refinements of types}''.
\sepItems
Jose~\cite{feldman2006jose} seems keen on sound theoretical principles:
``\emph{There are two issues involved
in the implementation of such a tool: the correct enforcement of the theoretical principles, and the instrumentation
of the code. Most previous tools tackle both issues, but have
subtle failures in one or the other.}'',\\*
but when it comes to class invariants just perform checks around public methods:
``\emph{Invariants are checked at the end of the lowest-level constructor,
 and at the beginning and end of every non-private non-static method}''.\\*
The same behaviour happens in `Code Contracts'~\cite{fahndrich2010embedded},
jContractor~\cite{abercrombie2002jcontractor}
and~\cite{tran2003design}:``\emph{Invariants may be checked both before and after
method  execution  for  calls  crossing  object  boundaries.
Assertion  failures  result  in  appropriate  exceptions.}''.\\*
In this way methods still have to assume any received object may be broken; in such case calling any
public getter would trigger an error, but until the object is just passed around
(and for example stored in collections), the broken state will not be detected.
The already discussed~\cite{Gopinathan:2008:RMO:1483018.1483028}
explains the danger of those approaches very clearly:
``\emph{there are many instances where o's invariant is violated by the programmer inadvertently changing the state
of p when o is in a steady state. Typically, o and p
are objects exposed by the API, and the programmer (who is the user of the API), unaware of
the dependency between o and p, calls a method of p in such a way that
o's invariant is violated. The fact that the violation occurred is detected
much later, when a method of o is called again, and it is difficult to 
determine exactly where such violations occur.}''\\*
Critically, even~\cite{Gopinathan:2008:RMO:1483018.1483028} is still unsound since they do not address neither
the risk of non-determinism in the \Q@invariant()@ method, 
nor the presence of exceptions.
\sepItems
Works over C\# recognize the need
of purity/determinism when methods calls are allowed in contracts~\cite{barnett200499}
``\emph{There are three main current approaches: a) forbid the use of functions in specifications, b) allow only provably pure functions, or c) allow programmers free use
of functions. The first approach is not scalable, the second overly restrictive and
the third unsound.}''\\*
They recognize that many tools unsoundly uses (c), as AsmL~\cite{barnett2003runtime}.
They propose a concept of observational purity, than when completely fleshed out
could possibly be a great addition to our proposed type system. We speculate that some 
primitive language support may be needed, for example implementing the Flyweight pattern 
as part of the language semantic.

\noindent\textit{Class Invariant Interpretations:}
As we discussed, in an OO language we cannot prevent the existence of broken objects,
but only limit their impact on the program execution.
Of course we are not the first ones to consider this issue, and in the literature we have many possible definitions
of \textbf{when} the invariant is expected to hold:
\begin{itemize}
\item  when the object is in a \textit{steady} state:
 the execution is not inside any of its methods~\cite{Gopinathan:2008:RMO:1483018.1483028};
\item
at the start and at the end of every public method
(though it is unclear how to deal with recursive methods)~\cite{Burdy2005};
\item
constantly maintained between calls to public methods~\footnote{\url{https://en.wikipedia.org/wiki/Class_invariant}};
\item
constantly maintained when the object is \textit{closed};
invariant can be manually opened and closed by using special operations; % Add cite here!
\item
before every field getter --- this way an invalid value may never be observed
even if present in memory for a long time.
\end{itemize}
\noindent We believe our requirement is the strongest amongst the ones found in literature,
thus it offers the best support to reasoning.
Moreover, since the invariant is defined in the language itself,
the user can encode open/close semantics if is needed,
by adding a mutable boolean \Q@isOpen@ field and adding \Q@this.isOpen || ..@
in front of the invariant condition.

\noindent However, there could be yet a stronger interpretation:
to start checking \Q@invariant()@ directly
after any field update in the whole ROG.
With aspect orientation it is possible to do just that~\cite{Gopinathan:2008:RMO:1483018.1483028},
however even if we could accept the intrinsic performance cost of such solution,
many argue for the importance of
doing \textbf{multiple tightly connected field updates} before checking the invariant again.
We understand the need for this
expressive power; the Decorator Pattern
is sufficient to supports such expressive power while 
preserving SIC and the reasoning benefits coming from 
class invariants:
define a `state class' (e.g. \Q@BoatState@) containing the state but no invariant;
the `target class' (e.g. \Q@Boat@) will then have a single 
\Q@capsule@ field of type `state class' and the \Q@invariant()@ method.
In this way, any method taking a `target class' parameter can
rely on the invariant, while any method taking a `state class' parameter will not.
As for the Decorator, a common interface can be introduced to support reasoning on both kinds of objects.

Conclusions?
@StrongExceptionSafety is 
a very strong property,
and some languages may be unwilling to commit to always preserve it.
In particular, depending on the details of a specific language
 releasing resources as in \Q@finally@ blocks may require
some relaxation of @StrongExceptionSafety. Sound releasing of resources could be interesting
future work.

%Overall, we are very confused on finding so many approaches asking for the programmer to
%actually understand what their code is doing in the details.
%It is undesirable to reach such level of understanding while relying only on the specification
%of the used (third party) libraries.

%Our interpretation of invariant is sightly different from the usual open/close
%semantics.
%Consider a class containing two equal numbers:
%\begin{lstlisting}
%class EqualPair {
%  imm Int a, imm Int b
%  read method imm Bool invariant() {return a==b;}
%  mut method imm Void update(imm Int x) {
%    this.setA(x);this.setB(x);
%  }
%}
%\end{lstlisting}
%In our system the \Q@update@ method clearly breaks the invariant, and running it will cause an invariant error/exception.
%
%In the conventional open/close semantics, the programmer could write something like:
%\begin{lstlisting}
%  mut method imm Void update(imm Int x){
%    this.openInvariant();
%    this.setA(x);this.setB(x);
%    this.closeInvariant();
%  }
%\end{lstlisting}
%To allow the two mutations to happen while the invariant is open.
%Our way to handle encapsulated objects exacerbates this difference; consider the following class
%containing non empty lists:
%\begin{lstlisting}
%class NonEmptyList{
%  capsule ListS list, imm String label
%  read method imm Bool invariant(){
%    return !list.isEmpty();}
%  mut method imm Void update(imm String x){
%    this.#list().clear();$\Comment{list is now empty}$
%    this.#list().add(x);$\Comment{what if we swap}$
%    this.setLabel(x);$\Comment{ those two lines?}$
%  }
%}
%\end{lstlisting}
%The code as presented is fine, but if we just swap the two last lines, then we would set the label while the object is debated.
%This would still cause a re-checking of the invariant, that will fail. 
%
%We are unsure if these two examples show a limitation of our approach with respect to explicit open/close semantics or if we capture more closely the expectations of a programmer.
%If it is the second case, then is our last code example conceptually wrong independently of the ordering of the lines?
%
%Note how our approach is clearly simpler/more primitive than the explicit open/close semantics.
%We could easily extend our work by making the compiler add a boolean field to all classes,
%and modify the invariant checking in the methods using exposers as follows:
%\begin{lstlisting}
%public mut method imm Void addAllergy(imm Str x){
%  imm Bool wasOpen=this.open;
%  this.open=true;
%  this.#allergies().add(x);
%  this.open=wasOpen;$\Comment{ thanks of strong exception}$
%  $\Comment{ safety there is no need of 'finally' here}$
%  if(!this.open && !this.invariant()){
%    throw new Error(...);}}
%\end{lstlisting}


