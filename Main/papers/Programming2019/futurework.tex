\section{Conclusions and Future Work}
The aim of our work is only to enforce validation, we do not present complexities unnecessary to enforce validation and we do not formalize any specific type system, to stay parametric over 
the various existing type systems which provably enforce the properties we require for our proof (and much more).
In essence we present what we \emph{believe} to be the simplest sound system.
It could be worthwhile formalising the minimal type system required by validation.



%However the restrictions we make to ensures that \Q@validate@ is deterministic, namely those the type-system enforces due to its signature, seem quite flexible and reasonable;

%%%%%examples of things that future work may investigate allowing are deterministic I/O and multi-threading. 


The language we have presented here restricts the form of \Q@.validate()@
and capsule mutator methods; in particular
our strong restrictions of capsule mutator methods
allows to inject \Q@.validate()@ calls just at the end of such methods.
While those restrictions do not interferer with simple
forms of validation, to verify complex mutable data-structures we necessitate verbose patterns (such as our `box').

We believe it is possible to relax our restrictions whilst
still ensuring our desired semantics, however such a language is unlikely to be easily understood by programmers;
being able to predict whether code would be well typed allows programmers
to better take advantage of the language.
Directions that could be investigated to improve our work include the addition of syntax-sugar to ease the burden of our suggested patterns; type-modifier inference, and support for flexible ownership types.

Our work, in comparison to previous RV techniques, aims to be efficient by limiting the number of validation calls, however we have no empirical evaluation of our approach performance.
To improve efficiency it could be worth investigating elision of unnecessary validation calls
or even only validating parts of objects (by running the part of \Q@validate@ that could fail).