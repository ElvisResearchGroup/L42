\appendix
\section{Proofs} 

\begin{Theorem}[Sound Validation]
if $c:\Kw{Cap}\vdash \e: \T$ and
$c:\Kw{Cap}\{\_;\_\}|\e\rightarrow^+ \sigma|\ctx_v[r_l]$, then
either $H(\sigma,l)$ or $\mathit{trusted}(\ctx_v,r_l)$.
\end{Theorem}

We believe this property captures very precisely our statement in Section~\ref{s:def}.

It is hard to prove Sound Validation directly,
so we first define a stronger property,
called Stronger Sound Validation and
we show that it is preserved during reduction by mean of conventional 
Progress and Subject Reduction (Progress is one of our assumption,
while Subject Reduction relies heavily on SubjectReductionBase).
That is,
Progress+Subject Reduction $\Rightarrow$ Stronger Sound Validation
\\*and Stronger Sound Validation $\Rightarrow$ Sound Validation.

\subsection{Stronger Sound Validation $\Rightarrow$ Sound Validation}

Stronger Sound Validation depends on 
$\mathit{wellEncapsulated}$, $\mathit{monitored}$
and $IOK$ (Invariant OK):

\noindent\textbf{Define} $\mathit{wellEncapsulated}(\sigma,\e,l_0)$:\\*
${}_{}$\quad\quad$\forall l \in \mathit{erog}(\sigma,l_0), \text{not}\ \mathit{mutated}(l,\sigma,\e)$

\noindent The main idea is that an object is encapsulated if its internal representation is safe from
modification. 

\noindent\textbf{Define} $\mathit{monitored}(\e,l)$:\\*
${}_{}$\quad\quad  $\e=\ctx_v[M(l,\e_1;\e_2)]$ and either $l=\e_1$ or $l$ not inside $\e_1$.

\noindent An object is monitored if the execution
is currently inside of a monitor for that object, and
the monitored expression $\e_1$ \textbf{does not}
contains $l$ as a proper subexpression.

That is, a monitored object is associated with an expression that can not observe him, but may 
reference its internal representation directly.
In this way, we can safely modify its representation before checking for the invariant.

The idea is that at the start the object invariant will hold and $\e_1$ will contains $l$;
but during reduction, the $l$ reference will be used in order to
give access to the internal state of $l$. Only after that moment, the invariant may be broken.


\noindent\textbf{Define} $IOK(\sigma,e)$:\\
${}_{}$\quad\quad  $\forall l\in\dom(\sigma)$
  either
\begin{enumerate}
%   \item ${}_{}$ $l$ has not invariant
   \item $\mathit{garbage}(l,\sigma,\e)$
   \item $H(\sigma,l)$ and $\mathit{wellEncapsulated}(\sigma,\e,l)$
   \item $\mathit{monitored}(\e,l)$
\end{enumerate}

\noindent
Finally, the system is in a coherent state with respect to invariant checking 
if for all the objects in the memory, one of those 3 cases apply:
%the class of the object has no invariant method;
the object is not (transitively) reachable from the expression, thus can be garbage collected;
the object invariant holds, and the object is encapsulated;
or the object is currently monitored.

Thus, on one side, we accept invalid objects in the garbage and during the monitoring process.
on the other side, in addition of requiring validation to hold $H$, we
also require those objects to be $\mathit{wellEncapsulated}$;
it is needed later to prove that the future steps will be $IOK$ too.


\begin{Theorem}[Stronger Sound Validation]
if $c:\Kw{Cap}\vdash \e_0: \T_0$ and
$c:\Kw{Cap}\{\_;\_\}|\e_0\rightarrow^+ \sigma|\e$, then
$IOK(\sigma,\e)$
\end{Theorem}
\noindent Starting from only the capability object,
any well typed expression $\e_0$ can be reduced for an arbitrary amount of steps,
and $IOK$ will always hold.
\\
\textbf{Proof }Stronger Sound Validation $\Rightarrow$ Sound Validation:\\*
\noindent To prove that Stronger Sound Validation implies Sound Validation
we consider for each $l$ the 3 cases of $IOK$: the $l$ involved in the current redex
will be one of those.
Indeed, this proof shows invalid objects are never observed:
If $l$ is garbage, is not in the current redex.
If $H(\sigma,l)$, the invariant holds, so thanks to Determinism
no invalid object is observed.
Finally, if $\mathit{monitored}(\e,l)$
then, either we are executing inside of $\e_1$, thus
the current redex is inside of a subexpression of the monitor that does not contains $l$, thus
$l$ is not in the current redex.
Or we are executing inside $\e_2$:
by our reduction rules, all monitor expressions starts with 
$\e_2=l$\Q@.validate()@, thus the first execution step
of $\e_2$ is trusted.
Following execution steps are also trusted since thanks
to well formedness the body of invariant methods only
uses \Q@this@ to access fields.


\subsection{Subject Reduction}

\noindent\textbf{Define} $\text{fieldGuarded}(\sigma,\e)$:\\*
${}_{}$\quad\quad$\forall \ctx$ such that $\e=\ctx[l\singleDot\f], $
and $\Sigma^\sigma(l).f=\Kw{capsule}\,\_$\\*
${}_{}$\quad\quad\quad either $\forall\, T,C,\quad\ \Sigma^\sigma;\x:\Kw{imm}\,C\,\vdash\ctx[\x]:T$ does not hold\\*
${}_{}$\quad\quad\quad or $\ctx=\ctx'[$\Q@M(@$l$\Q@;@$\ctx''$\Q@;@$\_$\Q@)@$]$ and $l$ is contained exactly once in $\ctx''$ \\*
\noindent We check that all the mutating capsule field accesses
are individually guarded by monitors.
Note how we use $C$ in $\x:\Kw{imm}\,C$ to guess the type of the accessed field,
and that we use the full context $\ctx$ instead of the evaluation context $\ctx_v$
to refer to field accesses everywhere in the expression $\e$.



\begin{Theorem}[Subject Reduction]
if $\Sigma^{\sigma_0}\vdash e_0: T_0$,
$\sigma_0|e_0\rightarrow \sigma_1|e_1$,
$IOK(\sigma_0,\e_0)$
and
$\mathit{fieldGuarded}(\sigma_0,\e_0)$
then
$\Sigma^{\sigma_1}\vdash e_1: T_1$,
$IOK(\sigma_1,e_1)$ and
$\mathit{fieldGuarded}(\sigma_1,\e_1)$
\end{Theorem}

\noindent\textbf{Proof }Progress + Subject Reduction $\Rightarrow$ Stronger Sound Validation:\\*
This proof proceed by induction as the usual Progress + Subject Reduction $\Rightarrow$ Soundness
proof.
Note that at the start of the execution, the memory is going to be empty, thus
$IOK(\emptyset,e)$  and $\mathit{fieldGuarded}(\emptyset,\e)$
will hold.
Then step by step, objects are created and their status is kept under control.
Note how here garbage collection is important: 
when the \Q@validate()@ method terminate with \Q@false@, 
execution can continue only if the offending object is classified as garbage.

\subsection{Proof of Subject Reduction}
We first introduce some lemmas
derived from the well formedness criteria and the type guarantees
\begin{Lemma}[ConstructorInstrumentation]
  In our reduction, executing rule \textsc{new} creating an object with invariant $l$, creates a monitor expression for $l$.
(Holds by rule \textsc{new})
\end{Lemma}
\begin{Lemma}[SetterInstrumentation]
  In our reduction, executing rule \textsc{f-update} on an object with invariant $l$, creates a monitor expression for $l$.
(Holds by rule \textsc{f-update})
\end{Lemma}
\begin{Lemma}[ExposerInstrumentation]
if $\sigma_0 | \e_0\rightarrow \sigma_1 |\e_1$ and
$\text{fieldGuarded}(\sigma_0,\e_0)$
\\*
then $\text{fieldGuarded}(\sigma_1,\e_1)$

  In our reduction, all occurrences of capsule field access $l\singleDot\f$ used for mutation
   are surrounded by a monitor expression for $l$.
  This means that reduction preserve $\text{fieldGuarded}(\sigma,\e)$.
\end{Lemma}
\noindent\textbf{Proof }of ExposerInstrumentation:\\*
The only rule that can 
introduce a new field access is \textsc{m-call}.
In that case, ExposerInstrumentation holds
by well formedness (all field access in methods are of form \Q@this.f@) 
and by rule \textsc{m-call} inserting a monitor while invoking exposer methods.
Note that rule \textsc{monitor exit} is fine because monitors are removed only when
 $e_1$ is a value.

\noindent\textbf{Proof }of Subject Reduction:\\*
Any reduction step can be obtained
by exactly one application of rule \textsc{ctx} and then one other rule.
Thus the proof can simply proceed by cases on such other applied rule.

As a further simplification, for each case 
$\Sigma^{\sigma_1}\vdash e_1: T_1$ holds by SubjectReductionBase
and 
$\mathit{fieldGuarded}(\sigma_1,\e_1)$ 
holds thanks to ExposerInstrumentation.
Thus, for each case of reduction, we just verify that $IOK(\sigma_1,\e_1)$ holds.
That is, we need to check that for all locations in $\sigma_1$, one of the cases (1-3) hold.


\noindent\textbf{case field update} $l.f=v\rightarrow \e$:
  \begin{itemize}
  \item by SetterInstrumentation 
  \e=\Q@M(@$l;l;l$\Q@.invariant())@;
  thus (3)$\mathit{monitored}(\e,l)$.
  \item Every $l_1$ so that $l\in \text{rog}(\sigma,l_1)$ will verify the same case
  of the former step:
  if it was (1)[garbage],clearly it still holds.
  If it was (3)[monitored] then it still holds too.
  It could not have been (2): $l_1$ is not $\mathit{wellEncapsulated}$ since 
$\mathit{mutated}(l,	\sigma,\e)$ (by @MutField).
  \item Every other $l_0$ do not reach $l$, thus its invariant status is not
 influenced by this reductions step.
\end{itemize}

\noindent\textbf{case field access} $l.f\rightarrow v$:

    If for $l$ $IOK$ holds by (2),  
    it is possible that the next step is not encapsulated.
    This would mean that the field $f$ is a capsule and that we are required
to type it as \Q@mut@ to type the expression for the next step.
By $\mathit{fieldGuarded}(\sigma_0,\e_0)$
    the former step was inside of a monitor \Q@M(@$l$\Q@;@$\ctx_v[l$\Q@.f@$]$\Q@;@$\e$\Q@)@
    and the $l$ under reduction was the only occurrence of $l$.
    since $f$ is a capsule, we know that $l\notin \text{erog}(\sigma,l)$
    by HeadNonCircular.
    Thus in the new step not $l\, \text{inside}\ \ctx_v[v]$.
    Thus for l (3)[monitored] holds.
    
We still need to show that properties $\mathit{monitored}$ and $\mathit{wellEncapsulated}$
 for other objects are
not disturbed. This is the point where our aliasing and mutability control are most crucial:
We know that mutable $v$ is (directly) reachable from
$l$ that have invariant.
Thanks to CapsuleTree we know that for all $l_0$ reaching $l$,
$v$ can be reached by $l_0$ only passing trough $l$.
Thus, we can conclude  $l_0$ is not encapsulated in the former step (containing mutable $l$).
Thus, $l_0$ is either without invariant, garbage or monitored.
None of those 3 cases can be disturbed by a field access.


\noindent\textbf{case meth call}:\\*
  This reduction step does not influence any object in the memory and does not
disturb the properties $\mathit{monitored}$ and $\mathit{wellEncapsulated}$.

\noindent\textbf{case new}:\\*
  If $C$ has invariant, then by @ConstructionInstrumentation the new object is monitored.
As for the method call, other objects and properties are not disturbed.


\noindent\textbf{case monitor exit} \Q@M(@$l;v;$\Q@true)@$\rightarrow v$ :
  \begin{itemize}
\item
    If it was a setter $v=l$, and 
    thanks to Determinism the execution of invariant is deterministic;
    thus for $l$ in the former step both case (2) and (3) holds.
    In the next step (2) will hold for $l$.
\item
    If it was a capsule mutator method, thanks to Determinism the execution
 of \Q@.validate()@ is deterministic;
    thus for $l$ in the former step both $H$ and case (3) holds.
    Thanks to ExposerInstrumentation $v$ is offered without mutation permissions, so
    In the next step $l$ is encapsulated and (2) will hold.
\item
    If it is was a constructor, 
    then $v$ is encapsulated and thanks to Determinism
    the execution of invariant is deterministic, thus in the next step (2) will hold.
\end{itemize}

\noindent\textbf{case try enter and try ok}
This case do not influence any object in the memory and does not
disturb the properties $\mathit{monitored}$ and $\mathit{wellEncapsulated}$.

\noindent\textbf{case try catch} $\sigma,\sigma_0|\Kw{try}^\sigma \oC\mathit{error}\cC\Kw{catch}\, \e\rightarrow \sigma|\e$:\\*
From the premise we know 
$IOK(\sigma,\sigma_0;\ctx_v[\Kw{try}^\sigma \oC\mathit{error}\cC\Kw{catch}\, \e])$;
thus we need to show
$IOK(\sigma;\ctx_v[\e])$.
By StrongExceptionSafety we know that $\sigma_0$ is garbage with respect to $\ctx_v[\e]$.

There could be many $l$ inside $\sigma,\sigma_0$ that are $\mathit{monitored}$
in the former step thanks to monitor expressions inside $\mathit{error}$.
However, all such $l$ are defined inside $\sigma_0$,
for the last well formedness condition.