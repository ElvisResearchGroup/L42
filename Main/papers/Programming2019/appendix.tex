\appendix
\section{Proof} 
\label{s:proof}

\begin{theorem}[Sound Validation]
	if $c:\Kw{Cap};\emptyset\vdash \e: \T$ and
	$c\mapsto\Kw{Cap}\{\_\}|\e\rightarrow^+ \sigma|\ctx_v[r_l]$, then
	either $valid(\sigma,l)$ or $\mathit{trusted}(\ctx_v,r_l)$.
\end{theorem}

We believe this property captures very precisely our statement in Section~\ref{s:validation}.

It is hard to prove Sound Validation directly,
so we first define a stronger property,
called \emph{Stronger Sound Validation} and
show that it is preserved during reduction by means of conventional 
Progress and Subject Reduction (Progress is one of our assumption,
while Subject Reduction relies heavily on SubjectReductionBase).
That is,
Progress+Subject Reduction $\Rightarrow$ Stronger Sound Validation,
\\*and Stronger Sound Validation $\Rightarrow$ Sound Validation.

\subsection{Stronger Sound Validation $\Rightarrow$ Sound Validation}

Stronger Sound Validation depends on 
$\mathit{wellEncapsulated}$, $\mathit{monitored}$
and $OK$:

\noindent\textbf{Define} $\mathit{wellEncapsulated}(\sigma,\e,l_0)$:\\*
\indent$\forall l \in \mathit{erog}(\sigma,l_0), \text{not}\ \mathit{mutatable}(l,\sigma,\e)$

\noindent The main idea is that an object is well encapsulated if its encapsulated state is safe from
modification. 

\noindent\textbf{Define} $\mathit{monitored}(\e,l)$:\\*
\indent$\e=\ctx_v[M(l;\e_1;\e_2)]$ and either $\e_1=l$ or $l$ is not inside $\e_1$.

\noindent An object is monitored if the execution
is currently inside of a monitor for that object, and
the monitored expression $\e_1$ does not
contains $l$ as a \emph{proper} subexpression.

A monitored object is associated with an expression that can not observe it, but may 
reference its internal representation directly.
In this way, we can safely modify its representation before checking for the invariant.

The idea is that at the start the object will be valid and $\e_1$ will contain $l$;
but during reduction, the $l$ reference will be used in order to
give access to the internal state of $l$; only after that moment, the object may become invalid.


\noindent\textbf{Define} $OK(\sigma,e)$:\\
\indent $\forall l\in\dom(\sigma)$
  either\\
\indent\indent 1. $\mathit{garbage}(l,\sigma,\e)$\\
\indent\indent 2. $\mathit{valid}(\sigma,l)$ and $\mathit{wellEncapsulated}(\sigma,\e,l)$\\
\indent\indent 3. $\mathit{monitored}(\e,l)$

Finally, the system is in a valid state with respect to validation
if for all the objects in the memory, one of these 3 cases apply:
%the class of the object has no invariant method;
the object is not (transitively) reachable from the expression (thus can be garbage collected);
the object is valid, and the object is encapsulated;
or the object is currently monitored.

\begin{theorem}[Stronger Sound Validation]
if $c:\Kw{Cap};\emptyset\vdash \e_0: \T_0$ and
$c\mapsto\Kw{Cap}\{\_\}|\e_0\rightarrow^+ \sigma|\e$, then
$OK(\sigma,\e)$
\end{theorem}
\noindent Starting from only the capability object,
any well typed expression $\e_0$ can be reduced for an arbitrary amount of steps,
and $OK$ will always hold.
\\
\begin{theorem} Stronger Sound Validation $\Rightarrow$ Sound Validation
\end{theorem}
\begin{proof}
\noindent By Stronger Sound Validation, each $l$ in the current redex must be $OK$:
\begin{enumerate}
	\item If $l$ is garbage, it cannot be in the current redex, a contradiction.
	\item If $\mathit{valid}(\sigma,l)$, then $l$ is valid, so thanks to Determinism
	no invalid object could be observed.
	\item Otherwise, if $\mathit{monitored}(\e,l)$ then either:
	\begin{itemize}
	 \item we are executing inside of $\e_1$ thus the current redex is inside of a sub-expression of the monitor that does not contain $l$, a contradiction.
	 \item or we are executing inside $\e_2$:
	 by our reduction rules, all monitor expressions start with 
	 $\e_2=l$\Q@.validate()@, thus the first execution step
	 of $\e_2$ is trusted. Following execution steps are also trusted, since by well formedness the body of invariant methods only use \Q@this@ (now translated to $l$) to access fields.
	\end{itemize}
\end{enumerate}
In any of the possible cases above, Sound Validation holds for $l$, and so it holds for all redexes.
\end{proof}

\subsection{Subject Reduction}

\noindent\textbf{Define} $\text{fieldGuarded}(\sigma,\e)$:\\*
\indent$\forall \ctx$ such that $\e=\ctx[l\singleDot\f] $
and $\Sigma^\sigma(l).f=\Kw{capsule}\,\_$, and $\f\mathrel{\mathit{inside}} \Sigma^\sigma(l).\mathit{validate}$\\*
\indent\indent either 
$\forall T, \forall C, \Sigma^\sigma;\x:\Kw{mut}\,C\,\not\vdash\ctx[\x]:T$, or\\*
\indent\indent $\ctx=\ctx'[$\Q@M(@$l$\Q@;@$\ctx''$\Q@;@$\e$\Q@)@$]$ and $l$ is contained exactly once in $\ctx''$

That is, all \emph{mutating} capsule field accesses are individually guarded by monitors.
Note how we use $C$ in $\x:\Kw{mut}\,C$ to guess the type of the accessed field,
and that we use the full context $\ctx$ instead of the evaluation context $\ctx_v$
to refer to field accesses everywhere in the expression $\e$.


\begin{theorem}[Subject Reduction]
if $\Sigma^{\sigma_0};\emptyset\vdash e_0: T_0$,
$\sigma_0|e_0\rightarrow \sigma_1|e_1$,
$OK(\sigma_0,\e_0)$
and
$\mathit{fieldGuarded}(\sigma_0,\e_0)$
then
$\Sigma^{\sigma_1};\emptyset\vdash e_1: T_1$,
$OK(\sigma_1,e_1)$ and
$\mathit{fieldGuarded}(\sigma_1,\e_1)$
\end{theorem}

\begin{theorem}
	Progress + Subject Reduction $\Rightarrow$ Stronger Sound Validation
\end{theorem}
\begin{proof}
This proof proceeds by induction in the usual manner.

\emph{Base Case}: At the start of the execution, the memory is going to only contain $c$: since $c$ is defined to be initially $\mathit{valid}$, and has only \Q@mut@ fields, and so it is trivially $\mathit{wellEncapsulated}$, thus $OK(c\mapsto\Kw{Cap},e)$.

\emph{Induction}: By Progress we always have another evaluation step to take, by Subject Reduction such a step will preserve $\mathit{OK}$, and so by induction $\mathit{OK}$ holds after any number of steps.

Note how for the proof garbage collection is important: 
when the \Q@validate()@ method evaluates to \Q@false@, 
execution can continue only if the offending object is classified as garbage.
\end{proof}

\subsection{Expose Instrumentation}
We first introduce a lemma derived from well formedness and the type system:
\begin{Lemma}[ExposerInstrumentation]
If $\sigma_0 | \e_0\rightarrow \sigma_1 |\e_1$ and
$\text{fieldGuarded}(\sigma_0,\e_0)$
\\*
then $\text{fieldGuarded}(\sigma_1,\e_1)$
\end{Lemma}
\begin{proof}
The only rule that can 
introduce a new field access is \textsc{mcall}.
In that case, ExposerInstrumentation holds
by well formedness (all field accesses in methods are of the form \Q@this.f@) 
and since \textsc{mcall} inserts a monitor while invoking capsule mutator methods, and not field accesses themselves. If however the method is not a \Q@mut@ method but still accesses a capsule field, by MutField such a field access expression cannot be typed as \Q@mut@ and so no monitor is needed.

Note that \textsc{monitor exit} is fine because monitors are removed only when
 $e_1$ is a value.
\end{proof}

\subsection{Proof of Subject Reduction}
Any reduction step can be obtained
by exactly one application of rule \textsc{ctx} and then one other rule.



Thus the proof can simply proceed by cases on such other applied rule.

By SubjectReductionBase and ExposerInstrumentation, 
$\Sigma^{\sigma_1};\emptyset\vdash e_1: T_1$ and  $\mathit{fieldGuarded}(\sigma_1,\e_1)$. So we just need to proceed by cases on the reduction rule applied to verify that $OK(\sigma_1,\e_1)$ holds:


\begin{enumerate}
\item (\textsc{update}) $\sigma|l\singleDot f\equals v\rightarrow \sigma'|\e'$:
\begin{itemize}
  \item By \textsc{update} $\e'=\Kw{M}\oR l;l;l\singleDot\text{validate}\oR\cR\cR$, thus $\mathit{monitored}(\e,l)$.
  \item Every $l_1$ such that $l\in \mathit{rog}(\sigma,l_1)$ will verify the same case
  as the former step:
  \begin{itemize}
  	\item If it was $\mathit{garbage}$, clearly it still is.
  	\item If it was $\mathit{monitored}$, it also still is.
    \item Otherwise it was $\mathit{valid}$ and $\mathit{wellEncapsulated}$:
		\begin{itemize}
			\item If $l\in \mathit{erog}(\sigma,l_1)$ we have a contradiction since $mutatable(l, \sigma, e)$, (by MutField)
	    \item Otherwise, by our well-formedess criteria that \Q@.validate()@ only accesses \Q@imm@ and \Q@capsule@ fields, and by Determinism it is clearly the case that $\mathit{valid}$ still holds;
By HeadNotCircular it cannot be the case that $l\in \mathit{erog}(\sigma',l_1)$ and so $l_1$ is still $\mathit{wellEncapsulated}$.
	  \end{itemize}
  \end{itemize}
  \item Every other $l_0$ is not in the reachable object graph of $l$
  thus it being $\mathit{OK}$ could not have been effected by this reduction step.
\end{itemize}

\item (\textsc{access}) $\sigma|l\singleDot f \rightarrow \sigma|v$:


\begin{itemize} 
	\item If $l$ was $valid$ and $wellEncapsulated$:
	
	\begin{itemize}
	\item If we have now broken $wellEncapsulated$ we must have made something in it’s $erog$  $mutatable$. As we can only type \Q@capsule@ fields as \Q@mut@ and not \Q@imm@ fields, by FieldMut we must have that $f$ is \Q@capsule@ and $l\singleDot f$ is being typed as \Q@mut@. By $\mathit{fieldGuarded}(\sigma_0,\e_0)$ the former step must have been inside a monitor \Q@M(@$l$\Q@;@$\ctx_v[l$\Q@.f@$]$\Q@;@$\e$\Q@)@
    and the $l$ under reduction was the only occurrence of $l$.
    Since $f$ is a capsule, we know that $l\notin \text{erog}(\sigma,l)$
    by HeadNotCircular. Thus in our new step $l$ is not $inside$ $\ctx_v[v]$. Thus $l$ must be $monitored$ and hence it is $OK$.
    
    \item Otherwise, $l$ is still $OK$

	\end{itemize}
\item Suppose some other $l_0$ was $wellEncapsulated$ and $valid$:
\begin{itemize}
	\item Suppose this reduction step has effected this, than it must have been that case that $l$ is in the $rog$ of $l_0$. By CapsuleTree, $v$ can only be reached from $l_0$ by passing through $l$, and so $l_0$ cannot have been in the $erog$ of $l$, and so we could not have made $l_0$ non-$wellEncapsulated$. In addition, since only things in the $erog$ can be referenced by $\singleDot\Kw{validate}\oR\cR$, $l_0$’s validity can not depend on $l$, and by Determinism it is still the case that $l_0$ is $valid$. And so we can’t have, as we just assumed, effected $l_0$ being $OK$.
	\item Otherwise $l_0$ is still $OK$.
\end{itemize}

\item Nothing that was $\mathit{garbage}$ could have been made reachable by this expression, since the only value we produced was $v$ and it was reachable through $l$ (and so could not have been garbage), thus $garbage$ is still $OK$.

\item As we don’t change any monitors here, nothing that was $monitored$ could have been made un-$monitored$, and so it is still $OK$.
\end{itemize}
\item (\textsc{mcall}, \textsc{try enter} and \textsc{try ok}):

These reduction steps do not modify memory, nor do they modify the memory-locations reachable inside of main-expression, nor do they modify any monitor expressions. Therefore it cannot have any effect on the $garbage$, $wellEncapsulated$, $valid$ (due to Determinism) or $monitored$ properties of any memory locations, thus $\mathit{OK}$ still holds.

\item (\textsc{new}) $\sigma|\Kw{new}\ C\oR\vs\cR\rightarrow \sigma,l\mapsto C\{\vs\}| \Kw{M}\oR l;l;l\singleDot\text{validate}\oR\cR\cR$:


Clearly the newly created object ($l$) is monitored. As for \textsc{mcall}, other objects and properties are not disturbed, and so $\mathit{OK}$ still holds.


\item (\textsc{monitor exit}) $\sigma|\Kw{M}\oR l; v;\Kw{true}\cR\rightarrow \sigma|v$:
\begin{itemize}
	\item As monitor expressions are not present in the original source code, it must have been introduced by \textsc{update}, \textsc{mcall}, or \textsc{new}. In each case the 3\textsuperscript{rd} expression started of as $l\singleDot\Kw{validate}\oR\cR)$, and it has now (eventually) been reduced to $\Kw{true}$, thus by Determinism $l$ is $valid$.

	\item  If the monitor was introduced by \textsc{update}, then $v = l$. As $l$ is $valid$ and $wellEncapsulated$ it is $OK$. \textbf{TODO: why is it well encapsulated?}

\item If the monitor was introduced by \textsc{mcall}.

%    If it was a capsule mutator method, thanks to Determinism the execution
% of \Q@.validate()@ is deterministic;
%    thus for $l$ in the former step both $H$ and case (3) holds.
%    Thanks to ExposerInstrumentation $v$ is offered without mutation permissions, so
%    In the next step $l$ is encapsulated and (2) will hold.
\item Otherwise the monitor was introduced by \textsc{new}.

%    If it is was a constructor, 
%    then $v$ is encapsulated and thanks to Determinism
%    the execution of invariant is deterministic, thus in the next step (2) will hold.
\end{itemize}


\item (\textsc{try error}) $\sigma,\sigma_0|\Kw{try}^\sigma\oC \mathit{error}\cC\ \Kw{catch}\ \oC\e\cC\rightarrow \sigma|\e$:

By StrongExceptionSafety we know that $\sigma_0$ is $\mathit{garbage}$ with respect to $\ctx_v[\e]$. By our well-formedness criteria no location inside $\sigma$ could have been $monitored$.

Since we don’t modify memory, everything in $\sigma_0$ is $\mathit{garbage}$ and nothing inside $\sigma$ was previously monitored, it is still clearly the case that everything in $\sigma$ is $\mathit{OK}$
\end{enumerate}