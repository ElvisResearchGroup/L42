%\documentclass[a4paper,UKenglish]{lipics-v2018}
\documentclass[acmsmall,review=false,anonymous]{acmart}%
\settopmatter{printfolios=true,printccs=false,printacmref=false}
\usepackage{hyperref}
%% For double-blind review submission, w/ CCS and ACM Reference
%\documentclass[acmsmall,review,anonymous]{acmart}\settopmatter{printfolios=true}
%% For single-blind review submission, w/o CCS and ACM Reference (max submission space)
%\documentclass[acmsmall,review]{acmart}\settopmatter{printfolios=true,printccs=false,printacmref=false}
%% For single-blind review submission, w/ CCS and ACM Reference
%\documentclass[acmsmall,review]{acmart}\settopmatter{printfolios=true}
%% For final camera-ready submission, w/ required CCS and ACM Reference
%\documentclass[acmsmall]{acmart}\settopmatter{}


%% Journal information
%% Supplied to authors by publisher for camera-ready submission;
%% use defaults for review submission.
\acmJournal{PACMPL}
\acmVolume{1}
\acmNumber{CONF} % CONF = POPL or ICFP or OOPSLA
\acmArticle{1}
\acmYear{2018}
\acmMonth{1}
\acmDOI{} % \acmDOI{10.1145/nnnnnnn.nnnnnnn}
\startPage{1}

%% Copyright information
%% Supplied to authors (based on authors' rights management selection;
%% see authors.acm.org) by publisher for camera-ready submission;
%% use 'none' for review submission.
\setcopyright{none}
%\setcopyright{acmcopyright}
%\setcopyright{acmlicensed}
%\setcopyright{rightsretained}
%\copyrightyear{2018}           %% If different from \acmYear

%% Bibliography style
\bibliographystyle{ACM-Reference-Format}
%% Citation style
%% Note: author/year citations are required for papers published as an
%% issue of PACMPL.
\citestyle{acmauthoryear}   %% For author/year citations


%%%%%%%%%%%%%%%%%%%%%%%%%%%%%%%%%%%%%%%%%%%%%%%%%%%%%%%%%%%%%%%%%%%%%%
%% Note: Authors migrating a paper from PACMPL format to traditional
%% SIGPLAN proceedings format must update the '\documentclass' and
%% topmatter commands above; see 'acmart-sigplanproc-template.tex'.
%%%%%%%%%%%%%%%%%%%%%%%%%%%%%%%%%%%%%%%%%%%%%%%%%%%%%%%%%%%%%%%%%%%%%%


%% Some recommended packages.
\usepackage{booktabs}   %% For formal tables:
%% http://ctan.org/pkg/booktabs
\usepackage{subcaption} %% For complex figures with subfigures/subcaptions
%% http://ctan.org/pkg/subcaption

%% _---------------------------------------------------------------------------
\usepackage{verbatim}
\usepackage{listings}
\usepackage{xcolor}
\usepackage{letltxmacro}
\usepackage{mathtools}
\usepackage{mathpartir}
%\usepackage{stix}

\definecolor{darkRed}{RGB}{100,0,10}
\definecolor{darkBlue}{RGB}{10,0,100}
\newcommand*{\ttfamilywithbold}{\fontfamily{pcr}\selectfont}
%\newcommand*{\ttfamilywithbold}{\ttfamily}

%found on http://tex.stackexchange.com/questions/4198/emphasize-word-beginning-with-uppercase-letters-in-code-with-lstlisting-package
%\lstset{language=FortyTwo,identifierstyle=\idstyle}
%
\makeatletter
\newcommand*\idstyle{%
        \expandafter\id@style\the\lst@token\relax
}
\def\id@style#1#2\relax{%
        \ifcat#1\relax\else
                \ifnum`#1=\uccode`#1%
                        \ttfamilywithbold\bfseries
                \fi
        \fi
}
\makeatother

\lstset{language=Java,
  basicstyle=\upshape\ttfamily\footnotesize,%\small,%\scriptsize,
  keywordstyle=\upshape\bfseries\color{darkRed},
  showstringspaces=false,
  mathescape=true,
  xleftmargin=0pt,
  xrightmargin=0pt,
  breaklines=false,
  breakatwhitespace=false,
  breakautoindent=false,
 identifierstyle=\idstyle,
 morekeywords={method,Use,This,constructor,as,into,rename},
 deletekeywords={double},
 literate=
  {\%}{{\mbox{\textbf{\%}}}}1
  {~} {$\sim$}1
%  {<}{$\langle$}1
%  {>}{$\rangle$}1
}

\newcommand*{\SavedLstInline}{}
\LetLtxMacro\SavedLstInline\lstinline
\DeclareRobustCommand*{\lstinline}{%
	\ifmmode
	\let\SavedBGroup\bgroup
	\def\bgroup{%
		\let\bgroup\SavedBGroup
		\hbox\bgroup
	}%
	\fi
	\SavedLstInline
}

\newcommand\saveSpace{\vspace{-2pt}}

\newcommand\Rotated[1]{\begin{turn}{90}\begin{minipage}{12em}#1\end{minipage}\end{turn}}

\newcommand{\Q}{\lstinline}
\newenvironment{bnf}{$\begin{aligned}}{\end{aligned}$}
\newcommand{\production}[3]{\textit{#1}&\Coloneqq\textit{#2}&\text{#3}}
\newcommand{\prodNextLine}[2]{&\quad\quad\textit{#1}&\text{#2}}
\newenvironment{defye}{\\\indent$\begin{aligned}}{\end{aligned}$\\}
\newcommand{\defy}[2]{\!\!\!\!\!\!&&#1&\coloneqq#2\\}
%\newcommand{\defyc}[1]{&\phantom{\coloneqq}\ \ #1\\}
\newcommand{\defyc}[1]{\!\!\!\!\!\!\rlap{\quad \quad #1}&&\\}
\newcommand{\defya}[2]{#1&\!\!\!\!\!\!&\coloneqq#2\\}

%\newcommand{\prodFull}[3]{#1&::=&\mbox{#2}&\mbox{#3}}
\newcommand{\prodInline}[2]{#1\Coloneqq#2}
\newcommand{\terminal}[1]{\ensuremath{$\texttt{#1}$}}
%\newcommand{\metavariable}[1]{\ensuremath{\mathit{#1}}}

\newcommand{\Rulename}[1]{{\textsc{#1}}}
\newcommand{\ctx}[1]{\ensuremath{\mathcal{E}_#1}\!}
\newcommand{\libi}[2]{\Q@\{@\Q!interface!\ #1\Q{;} #2\Q@\}@}
\newcommand{\lib}[3]{\Q!interface!\ensuremath{?}\ \libc{#1}{#2}{#3}}
\newcommand{\libc}[3]{\,\Q@\{@\!#1\Q{;}\ #2 \Q{;}\ #3\Q@\}@\!\!}

\newcommand{\rp}[1]{\Q{(}\!#1\Q{)}}
\newcommand{\eq}[1]{\,\Q{=}#1}
\newcommand{\red}[3]{#1\,\Q{<}#2\eq#3\,\Q{>}}
\newcommand{\summ}[2]{#1\ \Q{<+}\ #2}
\newcommand{\from}[2]{#1\ensuremath{[}#2\ensuremath{]}}
\newcommand{\mmid}{{\ensuremath{{\mid}}}\!}
\newcommand{\hole}{\ensuremath{\square}}
\newcommand{\s}[1]{\ensuremath{\mathit{#1s}}}
\makeatletter
\newcommand{\This}[1]{\Q!This!#1\nextpath}
\newcommand{\Cs}[1]{#1\nextpath}
\newcommand{\nextpath}{\@ifnextchar\bgroup{\gobblenextpath}{}}
\newcommand{\gobblenextpath}[1]{\Q!.!#1\@ifnextchar\bgroup{\gobblenextpath}{}}
\makeatother



%--------------------------
\newcommand{\mynotes}[3]{{\color{#2} {\sc #1}: #3}}
\newcommand\isaac[1]{\mynotes{Isaac}{blue}{#1}}

\newcommand\IO[1]{\color{blue}{#1}}
\newcommand\marco[1]{\mynotes{Marco}{green}{#1}}


\makeatletter
\renewcommand*\idstyle{%
        \expandafter\id@style\the\lst@token\relax
}
\def\id@style#1#2\relax{%
        \ifcat#1\relax\else
                \ifnum`#1=\uccode`#1%
                        \ttfamilywithbold\bfseries
                \fi
        \fi
}
\makeatother

\lstset{language=Java,
  basicstyle=\upshape\ttfamily\footnotesize,%\small,%\scriptsize,
  keywordstyle=\upshape\bfseries\color{darkRed},
  showstringspaces=false,
  mathescape=true,
  xleftmargin=0pt,
  xrightmargin=0pt,
  breaklines=false,
  breakatwhitespace=false,
  breakautoindent=false,
 identifierstyle=\idstyle,
 % Colour all these symbols brown! 
 otherkeywords = {<,+,>,\{,\},(,),.,=,;,\,}, % WHY is the , being coloured red and not brown?
 morekeywords = [2]{<,+,>,\{,\},(,),.,=,;,\,},
 keywordstyle = [2]{\upshape\bfseries\color{brown}},
 morekeywords={then,This,This0,This1,This2,This3,This4,This5},
 %deletekeywords={double},
 literate=
% {,}{\color{brown}\texttt{,}}1
  {\%}{{\mbox{\textbf{\%}}}}1
  {~} {$\sim$}1
%  {<}{$\langle$}1
%  {>}{$\rangle$}1
}


% Magic to make lstinline work in mathmode....
\newcommand*{\SavedLstInline}{}
\LetLtxMacro\SavedLstInline\lstinline
\DeclareRobustCommand*{\lstinline}{%
	\ifmmode
	\let\SavedBGroup\bgroup
	\def\bgroup{%
		\let\bgroup\SavedBGroup
		\hbox\bgroup
	}%
	\fi
	\SavedLstInline
}


% Based on the source for latex's @ifnextchar and @ifnch @xifnch
% \@grabnextchar{cmd}
% chews up the next (non-space) character, let's \nextchar to that character, and then executes cmd
\long\def\@grabnextchar#1{
	\def\@@cmd{#1}%
	\futurelet\@let@token\@grabnch}

\def\@grabnch{%
	\ifx\@let@token\@sptoken
	\let\@@do\@xgrabnch
	\else
	\let\nextchar=\@let@token
	\let\@@do\@@cmd
	\fi
	\@@do}
{\def\:{\@xgrabnch} \expandafter\gdef\: {\futurelet\@let@token\@grabnch}}

\makeatother
\usepackage{xspace}
\usepackage{tabularx}
\usepackage{letltxmacro}
\usepackage{mathtools}
\usepackage{mathpartir}

\makeatletter
\newcommand{\@eq}{\,\,\Q{=}\!\,\,}
% A nice questionmark
\newcommand{\@code}[1]{\ensuremath{#1}}
\newcommand{\q}{\ensuremath{?}}
\newcommand{\Empty}{\ensuremath{\epsilon}}
\newcommand{\rp}[1]{\Q{(}\!#1\Q{)}}
\newcommand{\cp}[1]{\,\Q{\{}\!\,#1\,\Q{\}}\!}
\newcommand{\eq}[1]{\,\Q{=}#1}
\newcommand{\ddd}{\ensuremath{{}\mathrel{\ldots}{}}}

%%%% Grammar forms

	\newcommand{\@meth}[5]{\@code{#1#2\ #3\rp{#4}#5}}

% Methods: any, static, instance, abstract, abstract instance, abstract static
\newcommand{\meth}  [4]{\@meth{\Q{static}\q\,}{#1}{#2}{#3}{\ #4}}
\newcommand{\smeth} [4]{\@meth{\Q{static}\,}  {#1}{#2}{#3}{\ #4}}
\newcommand{\imeth} [4]{\@meth{}              {#1}{#2}{#3}{\ #4}}
\newcommand{\ameth} [3]{\@meth{\Q{static}\q\,}{#1}{#2}{#3}{}}
\newcommand{\asmeth}[3]{\@meth{\Q{static}\,}  {#1}{#2}{#3}{}}
\newcommand{\aimeth}[3]{\@meth{}              {#1}{#2}{#3}{}}

% Nested class
\newcommand{\ncd}[2]{\cd{\Q{private}\q\,#1}{#2}}
\newcommand{\pcd}[2]{\cd{\Q{private}\,#1}{#2}}
\newcommand{\cd}[2]{\@code{#1\@eq#2}}
% Class literals
\newcommand{\cl}[3]{\@code{\cp{#1\Q{;}\ #2\Q{;}\ #3}}}
	
	%\@code{\ensuremath{\Q@\{@#1\Q{;}\ #2\Q{;}\ #3\,\Q@\}@\!}}}

% what was once a program... but is now a \Sigma
\newcommand{\State}[2]{#1;#2}
\newcommand{\Exists}[2]{\ensuremath{\exists #1\text{ such that } #2}}
\newcommand{\ForAll}[2]{\ensuremath{\forall #1,\quad #2}}

%New
\renewcommand{\This}{\Q{This}\xspace}
%\renewcommand{\T}[1]{#1\@nextpath}
%	\newcommand{\@nextpath}{\@ifnextchar\bgroup{\@gobblenextpath}{\xspace}}
%	\newcommand{\@gobblenextpath}[1]{\Q!.!\@code{#1}\@ifnextchar\bgroup{\@gobblenextpath}{\xspace}}
\renewcommand{\.}{\Q!.!\xspace}
%-------------------

% The sum operator
\newcommand{\summ}[2]{\@code{#1\ \Q{<+}\ #2}}

% Redirect operator, and redirect map
\newcommand{\red}[1]{\@code{#1}\,\Q!<!\!\,\@nextCsP{\,\Q!>!}}
\newcommand{\redm}{\@nextCsP{}}
	\newcommand{\@nextCsP}[1]{\@ifnextchar\bgroup{\@gobblenextCs{}{#1}}{#1}}
	\newcommand{\@gobblenextCs}[3]{#1\@code{#3}\@ifnextchar\bgroup{\@eq \@gobblenextP{#2}}{#2\xspace}}
	\newcommand{\@gobblenextP}[2]{\@code{#2}\@ifnextchar\bgroup{\@gobblenextCs{\Q{,}\,}{#1}}{#1\xspace}}

% new expression, field access expressions, method call expression\
\newcommand{\en}[2]{\@code{\Q{new}\ #1\rp{#2}}}
\newcommand{\ef}[2]{\@code{#1\Q{.}#2}}
\newcommand{\ec}[3]{\ef{#1}{#2\rp{#3}}}

% Because mathmode puts too much space
\newcommand{\DVz}{\ensuremath{\mathit{DVz}}}
\newcommand{\Dz}{\ensuremath{\mathit{Dz}}}
\newcommand{\DVs}{\ensuremath{\mathit{DVs}}}
\newcommand{\Ds}{\ensuremath{\mathit{Ds}}}
\newcommand{\Id}{\ensuremath{\mathit{Id}}}
\newcommand{\Mz}{\ensuremath{\mathit{Mz}}}
\newcommand{\MVz}{\ensuremath{\mathit{MVz}}}
\newcommand{\Tx}{\ensuremath{\mathit{Tx}}}
\newcommand{\Tz}{\ensuremath{\mathit{Tz}}}
\newcommand{\Txs}{\ensuremath{\mathit{Txs}}}
\newcommand{\Csz}{\ensuremath{\mathit{Csz}}}

\newcommand{\Ok}{\ensuremath{\textbf{Ok}}}
% Meta-notation
\newcommand{\op}[2]{{\ensuremath{{}_{.\textbf{#1}({#2})}}}}
\newcommand{\opn}[1]{{\ensuremath{\textbf{#1}}}}
\newcommand{\fop}[2]{{\ensuremath{\footnotesize{\textbf{#1}}({#2})}}}

% operators

\renewcommand{\members}[1]{\fop{members}{#1}}
\newcommand{\idd}[1]{\fop{id}{#1}}
\renewcommand{\dom}[1]{\fop{dom}{#1}}
\newcommand{\ran}[1]{\fop{range}{#1}}
\newcommand{\mdom}[1]{\fop{mdom}{#1}}
\newcommand{\mini}[2]{#1\op{min}{#2}}
\newcommand{\minid}[3]{#1\op{min}{#2, #3}}
\newcommand{\from}[3]{#1\op{from}{#2,#3}}

% Just to typset some words nicecly
\newcommand{\where}{,\quad\text{where }}
\newcommand{\wherec}{,\quad\text{where:}}
\newcommand{\wwhere}{\text{where }}
\newcommand{\otherwise}{,\quad\text{otherwise}}
\newcommand{\imp}{\ensuremath{\Rightarrow}}
\newcommand{\reda}[3]{\ensuremath{#1_{#2}[#3]}}

% Environemnts

\newenvironment{defs}{\setlength{\tabcolsep}{0pt}\noindent\tabularx{\textwidth}{l>{\hfill}X}\midrule}{\endtabularx\\% This is stupid (if I put a \midrule before the \endtabularx it tries to duplicate it and I get errors...)
	\begin{tabularx}{\textwidth}{X}\midrule\end{tabularx}}
% a := b
\newcommand{\defi}[3]{\ensuremath{#1\coloneqq#2}&\llap{\text{#3}}\\}
% a iff b 
\newcommand{\defip}[3]{\ensuremath{#1\text{ iff }#2}&\llap{\text{#3}}\\}
% a % (for things that are always true)
\newcommand{\defit}[2]{\ensuremath{#1}&\llap{\text{#2}}\\}
%   b % indentended, for continuiation of another defi form
\newcommand{\defic}[2]{\ensuremath{\quad\quad#1}&\llap{\text{#2}}\\}


\newenvironment{grammar}{%
	\setlength{\tabcolsep}{0pt}%
	\noindent\tabularx{\textwidth}{rl>{\hfill}X}\midrule}{\endtabularx\\% This is stupid (if I put a \midrule before the \endtabularx it tries to duplicate it and I get errors...)
	\begin{tabularx}{\textwidth}{X}\midrule\end{tabularx}%
}
\newcommand{\produ}[3]{\ensuremath{#1}&\ensuremath{{}\Coloneqq#2}&\llap{\text{#3}}\\}
\newcommand{\prodc}[2]{&\ensuremath{\quad\quad#1}&\llap{\text{#2}}\\}
% --------------
\newcommand{\defis}{\midrule}


\newenvironment{irules}{\noindent\[\begin{array}{l}\midrule}{\unskip\\\midrule\end{array}\]}
\newcommand{\irule}[3]{\inferrule[(#1)]{#2}{#3}\qquad}
\newcommand{\iruleSep}{\unskip\\[5ex]}

%\newcommand{\irule}[3]{%
%		\everymath={\displaystyle}%
%		\ensuremath{
%			\Rulename{#1}\ \frac{\hspace{-0.5em}\begin{array}{c}#2\end{array}\hspace{-0.5em}}{#3}%
%	}
%}
	
% this version formats things almost exactly the same as \inferrule, whereas mine (above) looks better!
%\newcommand{\irule}[3]{%
%	\everymath={\displaystyle}%
%	\ensuremath{
%		\begin{array}{l}%
%		\begin{array}{l}\Rulename{#1}\end{array}\\%
%		\frac{\hspace{-1em}
%				\begin{array}{c}#2\end{array}%
%			\hspace{-1em}}{#3}%
%		\end{array}%
%	}\hfill
%}
% ----------------------------------------------------------------------------
%\newcommand{\qquad}{\quad\quad}

\newcommand{\Rulename}[1]{{\textsc{(#1)}}}
\renewcommand{\ctx}[1]{\ensuremath{\mathcal{E}_#1}\!}

\makeatother

%\title{Using nested classes as associated types.}
%\titlerunning{Dummy short title}%optional, please use if title is longer than one line
%\author{Authors omitted for double-bind review.}{Unspecified Institution.}{}{}{}
%\authorrunning{Authors omitted for double-bind review.} %mandatory. First: Use abbreviated first/middle names. Second (only in severe cases): Use first author plus 'et al.'
%\Copyright{Authors omitted for double-bind review.} %mandatory, please use full first names. LIPIcs license is "CC-BY";  http://creativecommons.org/licenses/by/3.0/

%%%%%%%%%%%%%%%%%%%%%%%%%%%%%%%%%%%%%%%%%%%%%%%%%%%%%%

\begin{document}


%% Title information
\title[Redirect!]{Retroactive Generics and Inferred Associated Types using Trait Adaptation}  
%% [Short Title] is optional;
%% when present, will be used in
%% header instead of Full Title.
%\titlenote{with title note}             %% \titlenote is optional;
%% can be repeated if necessary;
%% contents suppressed with 'anonymous'
%\subtitle{Subtitle}                     %% \subtitle is optional
%\subtitlenote{with subtitle note}       %% \subtitlenote is optional;
%% can be repeated if necessary;
%% contents suppressed with 'anonymous'


%% Author information
%% Contents and number of authors suppressed with 'anonymous'.
%% Each author should be introduced by \author, followed by
%% \authornote (optional), \orcid (optional), \affiliation, and
%% \email.
%% An author may have multiple affiliations and/or emails; repeat the
%% appropriate command.
%% Many elements are not rendered, but should be provided for metadata
%% extraction tools.

%% Author with single affiliation.
\author{First1 Last1}
\authornote{with author1 note}          %% \authornote is optional;
%% can be repeated if necessary
\orcid{nnnn-nnnn-nnnn-nnnn}             %% \orcid is optional
\affiliation{
	\position{Position1}
	\department{Department1}              %% \department is recommended
	\institution{Institution1}            %% \institution is required
	\streetaddress{Street1 Address1}
	\city{City1}
	\state{State1}
	\postcode{Post-Code1}
	\country{Country1}                    %% \country is recommended
}
\email{first1.last1@inst1.edu}          %% \email is recommended

%% Author with two affiliations and emails.
\author{First2 Last2}
\authornote{with author2 note}          %% \authornote is optional;
%% can be repeated if necessary
\orcid{nnnn-nnnn-nnnn-nnnn}             %% \orcid is optional
\affiliation{
	\position{Position2a}
	\department{Department2a}             %% \department is recommended
	\institution{Institution2a}           %% \institution is required
	\streetaddress{Street2a Address2a}
	\city{City2a}
	\state{State2a}
	\postcode{Post-Code2a}
	\country{Country2a}                   %% \country is recommended
}
\email{first2.last2@inst2a.com}         %% \email is recommended
\affiliation{
	\position{Position2b}
	\department{Department2b}             %% \department is recommended
	\institution{Institution2b}           %% \institution is required
	\streetaddress{Street3b Address2b}
	\city{City2b}
	\state{State2b}
	\postcode{Post-Code2b}
	\country{Country2b}                   %% \country is recommended
}
\email{first2.last2@inst2b.org}         %% \email is recommended


%% Abstract
%% Note: \begin{abstract}...\end{abstract} environment must come
%% before \maketitle command
\begin{abstract}
	Text of abstract \ldots.
\end{abstract}


%% 2012 ACM Computing Classification System (CSS) concepts
%% Generate at 'http://dl.acm.org/ccs/ccs.cfm'.
\begin{CCSXML}
	<ccs2012>
	<concept>
	<concept_id>10011007.10011006.10011008</concept_id>
	<concept_desc>Software and its engineering~General programming languages</concept_desc>
	<concept_significance>500</concept_significance>
	</concept>
	<concept>
	<concept_id>10003456.10003457.10003521.10003525</concept_id>
	<concept_desc>Social and professional topics~History of programming languages</concept_desc>
	<concept_significance>300</concept_significance>
	</concept>
	</ccs2012>
\end{CCSXML}

\ccsdesc[500]{Software and its engineering~General programming languages}
\ccsdesc[300]{Social and professional topics~History of programming languages}
%% End of generated code


%% Keywords
%% comma separated list
\keywords{keyword1, keyword2, keyword3}  %% \keywords are mandatory in final camera-ready submission


%% \maketitle
%% Note: \maketitle command must come after title commands, author
%% commands, abstract environment, Computing Classification System
%% environment and commfalseands, and keywords command.
\maketitle

%\section{Introduction}\label{sec:intro}
In mainstream OO languages like Java, C++ or C\# subclassing 
implies subtyping. For example, a Java subclass definition, such as 
\Q@class A extends B {}@
\noindent does two things at the same time:
it {\bf inherits} code from \lstinline{B}; and it {\bf creates
a subtype} of \lstinline{B}. Therefore in a language like Java, 
a subclass is \emph{always} a subtype of the extended class.

Historically, there has been a lot of focus on
separating subtyping from subclassing~\cite{cook}.  This is claimed to be
good for code-reuse, design and reasoning. There are at
least two distinct situations where the separation of subtyping and 
subclassing is helpful.

\begin{itemize}

\item {\bf Allowing inheritance/reuse even when subtyping is impossible:} 
In some situations a subclass contains methods whose signatures 
are incompatible with the superclass, yet inheritance is still
possible. A typical example, which was illustrated by Cook et al.~\cite{cook}, are 
classes with \emph{binary methods}~\cite{bruce96binary}.

\item {\bf Preventing unintended subtyping:} For certain classes we
  would like to inherit code without creating a subtype even if, from
  the typing point of view, subtyping is still possible. A typical
  example~\cite{LaLonde:1991:SSS:110673.110679} of this are methods for collection classes such as \emph{Sets} and
  \emph{Bags}. Bag implementations can often inherit 
  from Set implementations, and the interfaces of the two collection types are
  similar and type compatible. 
  However, from the logical point-of-view a Bag is \emph{not a
    subtype} of a Set. 

\end{itemize}

Type systems based on structural typing~\cite{cook} can deal with the first
situation well, but not the second. Since structural subtyping
accounts for the types of the methods only, a Bag would be a subtype
of a Set if the two interfaces are type compatible. For dealing with
the second situation nominal subtyping is preferable. With nominal
subtyping an explicit subtyping relation must be signalled by the
programmer. Thus if subtyping is not desired, the idea is that 
programmer can simply {\bf not} declare a subtyping relationship.

While there is no problem in subtyping without subclassing, in the design
of most nominal OO languages subclassing implies subtyping in a
fundamental way. This is because of what we call the
\emph{this-leaking problem}, illustrated by the following
(Java) code:

\begin{lstlisting}[language=Java]
  class A{ int ma(){return Utils.m(this);} }
  class Utils{static int m(A a){..}}
\end{lstlisting}

Method \lstinline{A.ma} passes \lstinline{this} as \lstinline{A} to \Q@Util.m@.
This code is correct, and there is no subtyping/subclassing.  Now, lets add a class \Q@B@

\begin{lstlisting}[language=Java]
  class B extends A{ int mb(){return this.ma();} }  
\end{lstlisting}

%%Class \lstinline{B} does two things at the same time: 
%%1) it {\bf inherits} the method \lstinline{ma} from
%%\lstinline{A}; and 2) it creates a {\bf subtype} of \lstinline{A}.
\noindent We can see an invocation of \lstinline{A.ma} inside
\lstinline{B.mb}, where the self-reference \lstinline{this} is of type \lstinline{B}. 
The execution will eventually call \lstinline{Utils.m} with an
instance of \lstinline{B}. However, \emph{this can be correct only if \lstinline{B} is a subtype of
\lstinline{A}}. 

%If Java was to support a mechanism to allow reuse/inheritance 
%without introducing subtyping, such as:
%
%\begin{lstlisting}[language=Java]
%  class B inherits A{ int mb(){return this.ma();} }
%\end{lstlisting}
%
%\noindent Then an invocation of 
%\lstinline{mb} would be type-unsafe (i.e. it would 
%result in a run-time type error). 
%Note that here the intention of using the imaginary keyword {\bf
%  inherits} is to allow the code from \lstinline{A} to be inherited 
%without \lstinline{B} becoming a subtype of \lstinline{A}. 
%However this breaks type-safety. The problem is that the
%self-reference \lstinline{this} in class \lstinline{B} has 
%type \lstinline{B}. Thus, when \lstinline{this} is passed as an argument to 
%the method \lstinline{Utils.m} as a result of the invocation of
%\lstinline{mb}, it will have a type that is incompatible with the
%expected argument of type \lstinline{A}.  


As a thought experiment, imagine that Java code-reuse (the {\bf extends} keyword) was not introducing subtyping: then an invocation of 
\lstinline{B.mb} would result in a run-time type error.
The problem is that the
self-reference \lstinline{this} in class \lstinline{B} has 
type \lstinline{B}. Thus, when \lstinline{this} is passed as an argument to 
the method \lstinline{Utils.m} as a result of the invocation of
\lstinline{mb}, it will have a type that is incompatible with the
expected argument of type \lstinline{A}.  
Therefore, every OO language with the minimal features exposed in the example (using \lstinline{this},
{\bf extends} and method calls) is forced to accept that subclassing implies
subtyping\footnote{C++ allows to "extends privately", but it is a limitation over
  subtyping visibility, not over subtyping itself.  The
  former example would be accepted even if \lstinline{B} was to
  "privately extends" \lstinline{A}}.
  

What the \emph{this-leaking problem} shows is that adopting a more flexible
nominally typed OO model where subclassing does not imply subtyping is
not trivial: a more substantial change in the language design is
necessary.  In essence we believe that, in languages like Java, classes do too many
things at once. In particular they act both as units of \emph{use} and
\emph{reuse}: classes can be \emph{use}d as types and can be instantiated;
classes can also be subclassed to provide \emph{reuse} of code.
We are aware of at least 3 independently designed research
languages that address the \emph{this-leaking problem}:
\begin{itemize}
\item In {\bf TraitRecordJ (TR)}~\cite{Bettini:2010:ISP:1774088.1774530,BETTINI2013521,Bettini2015282}
each construct has a single responsibility: classes instantiate objects,
interfaces induce types, records express state and traits are reuse units.
\item {\bf Package Templates (PT)}~\cite{KrogdahlMS09,DBLP:journals/taosd/AxelsenSKM12,DBLP:conf/gpce/AxelsenK12}
are an extension of (full) Java where new packages can be ``synthesized'' by mixing
and integrating code templates. As an extension of Java, PT allows but does not require
separation of inheritance and subtyping.
\item {\bf
    DeepFJig}~\cite{deep,servetto2014meta,fjig} is
a module composition language where the main idea is that
nested classes with the same name are recursively composed.
\end{itemize}
This paper aims at showing a simple language design, called \name,
which addresses the \emph{this-leaking problem} and decouples subtyping from inheritance.
Leveraging on traits, in this work we aim to synthesize the best ideas
of those 3 very different designs, while at the same time coming up with a simpler and
improved design for separating subclassing from subtyping.
The keys ideas in \name are to divide between code designed for
\textbf{use} and code designed for \textbf{reuse}, and a novel
approach to state in traits that avoids the complexities introduced by
fields and their initialisation in previous approaches.
In \name there are two separate concepts: classes
and traits~\cite{Traits:ECOOP2003}. Classes are meant for code use, and cannot be used
for reuse. In some sense classes in \name are like final classes in
Java. Traits are meant for code reuse and multiple traits can be
composed to form a class which can then be instantiated. Traits 
cannot be instantiated (or used) directly. Such design allows
subtyping and code reuse to be treated separately, which in turn
brings several benefits in terms of flexibility and code reuse.

We first focus on an example-driven presentation to illustrate how to
improve use and reuse. 
in Section~\ref{sec:formal} we then provide a compact 1 page formalization.
Most of the hard technical aspects of the
semantics have been studied in previous 
work~\cite{Bettini:2010:ISP:1774088.1774530,BETTINI2013521,Bettini2015282,KrogdahlMS09,DBLP:journals/taosd/AxelsenSKM12,DBLP:conf/gpce/AxelsenK12,deep,servetto2014meta,fjig},
and the design of \name synthesises some of these previously studied
concepts.
The language design ideas have been implemented in the {\bf 42 language}, which supports all
the examples we show in the paper, and is available at:\url{http://l42.is} (this site can be accessed without breaking double blind review).

In summary, our contributions are:

%\marco{We need to talk of unanticipated extensions?}
%\bruno{Talk more about typing aspects here. Summarise the important 
%aspects of the design of \name.}

%\bruno{Need to say something about state?}

%The language design of \name is adopted by the 42 programming 
%language, All the examples shown in this paper can be run in 42.
%The compiler for 42 is available at: 

%\url{http://l42.is}



\begin{itemize}
\item We identify the {\bf this-leaking problem} and argue why it
  makes the separation of subclassing and subtyping difficult.
\item We synthesize the key ideas of previous designs that solve the
  this-leaking problem into {\bf a novel and
  simple language design}. This language is the logic core of the language 42, and 
  all the examples in the paper can be encoded as valid 42 programs. 

\item We illustrate how the new design {\bf improves both code use and code
  reuse}.
\item We propose {\bf a clean and elegant approach to handle of state} in a trait based language.

\item We show how to extend our system with nested classes, and how this make us challenge the expression problem.
\item we show the simplicity of our approach by providing a compact 1 page formalization
\item we perform 3 case studies, comparing our work with other approaches, showing that we can express the same examples in a cleaner and more modular manner.
\end{itemize}
\saveSpace
\saveSpace


\section{Language grammar and well formedness}
We apply our ideas on a simplified object oriented language with nominal typing and (nested)
interfaces  and final classes.
Code reuse is obtained by trait composition, thus the source code would be
a sequence of top level declarations $D$ followed by a main expression;
a lower-case identifier $t$ is a trait name, while an upper case
identifier $C$ is a class name.
To simplify our terminology, instead of distinguishing between 
nested classes and nested interfaces, we will call \emph{nested class} any member of a code literal 
named by a class identifier $C$. Thus, the term \emph{class} may denote either an \emph{interface class} (interface for short) or a \emph{final class}.

\noindent
\begin{minipage}{0.65\textwidth}
\begin{bnf}
\hline
\production{%
e}      {x \mmid{} e\Q{.}m\rp{es} \mmid{}T\Q{.}m\rp{es}
\mmid{} e\Q{.}x \mmid{} \Q{new} T\rp{es}
}{expression}\\\production{interface literal}
%\\\prodNextLine{%
\mmid{} \libc{Tz}{Mz}{K}
}        {code literal}\\\production{%
M}      {\Q{static}$?$ T m\rp{Txs} e$?$ \mmid{} \Q{private}$?$ C\eq{}E }                                                    {member}\\\production{%
K}      {\rp{Txz}$?$}                                          {state}\\\production{%
%CD}     {C\eq{}E}                                                          {class declaration}\\\production{%
%CV}     {C\eq{}LV}                                                         {evaluated class declaration}\\\production{%
%source}      {Ds e}                                                             {source code}\\\production{%
E}      {L \mmid{} t \mmid{} \summ{$E_1$}{$E_2$} \mmid{} E\Q@<@R\Q@>@}           {Code Expr.}%
%amt}    {T m\rp{Txs}}                                                      {abstract method}\\\production{%
%mt}     {\Q{static}$?$ T m\rp{Txs} e$?$}                                                 {method}\\\production{%
\\\production{%
R}      {$Cs_1 \eq{T_1}\ldots Cs_n\eq{T_n}$}                    {redirect map}\\\hline
\end{bnf}
\end{minipage}
\begin{minipage}[t]{0.5\textwidth}
\begin{bnf}
\hline
\production{%
T}      {\Q{This}n\Q{.}Cs}                                                 {types}\\\production{%
Tx}     {T x}                                                              {parameter}\\\production{%
D}      {id\eq{}E}                                                         {declaration}\\\production{%
id}     {C \mmid{} t}                                                      {class/trait id}\\\production{%
v}      {\Q{new} T\rp{vs}}                                                 {value}\\\production{%
LV}      {\ldots}{}                                                
\\\hline
\end{bnf}
\end{minipage}


In the context of nested classes, types are paths. Syntactically,
we represent them as relative paths of form 
$\This{n}{Cs}$, where the number $n$ identify the root of our path:
\Q@This@/\This0\ is the current class, \This1\ is the enclosing class, \This2\ is the enclosing enclosing class and so on. $\This{n}{Cs}$
refers to the class obtained by navigating throughout  $Cs$ starting from $\This{n}$.
By using a larger then needed $n$, there could be multiple different types referring to the same class.
We require all types to be in the form where the smallest possible $n$ is used.

Code literals $L$
serve the role of class/trait bodies; they contain the set of implemented interfaces
$Tz$, the set of members $Mz$ and their (optional) state.
In the concrete syntax we will use \Q@implements@ in front of a non empty list of implemented interfaces
and we will omit parenthesis around a non empty set of fields.
To simplify our formalism, we delegate some sanity checks to well formedness: all the fields in the state $K$ have different names;
no two methods or nested classes with the same name ($m$ or $C$) are declared in a code literal,
and no nested class is named $\This{n}$ for any number $n$;
in any method headers, all parameters have different names, and no parameter is named \Q@this@.

A class member $M$ can be a (private) nested class or a (static) method.
Abstract methods are just methods without a body. 
Well formed interface methods can only be abstract and non-static.
To facilitate code reuse, classes can have (static) abstract methods; code composition is expected to 
provide an implementation for those or, as we will see, redirect away the whole class.
We could easily support private methods too, but to simplify our formalism we consider private only for nested classes. In a well formed code literal, in all types of form $\This{n}{Cs}{C}{Cs'}$,
if $C$ denotes a private nested class, then $Cs$ is empty.

Expressions are used as body of (static) methods and for the main expression.
They are variables $x$ (including \Q@this@)
and conventional (static) method calls.
Field access and \Q@new@ expressions are included but with restricted usage:
well formed field accesses are of form \Q@this.@$x$ in method bodies and
$v$\Q@.@$x$  in the main expression, while 
well formed \Q@new@ expressions have to be of form \Q@new This0(@$xs$\Q@)@ in method bodies
and of form $v$ in the main expression.
Those restrictions greatly simply reasoning about code reuse, since they require different classes to
only communicate by calling (static) methods. Supporting unrestricted fields and constructors would make the formalism much more involved without adding much of a conceptual underpinning.
Values are of form \textit{\Q{new} T\rp{vs}}.

For brevity, in the concrete syntax we assume a syntactic sugar declaring
a static \Q@of@ method (that serve as a factory) and all fields getters; 
thus the order of the fields would induce the order of the factory arguments.
In the core calculus we just assume such methods to be explicitly declared.

Finally, we examine the shape of a nested class: \textit{\Q{private}$?$ C\eq{}E}.
The right hand side is not just a code literal but a code composition expression $E$.
In trait composition, the code expression will be reduced/flattened to a code literal $L$
during compilation.
Code expressions denote an algebra of code composition, starting from code literal $L$
and trait names $t$, referring to a literal declared before by \textit{t\eq{}E}.
We consider two operators: conventional preferential sum 
 \textit{\summ{$E_1$}{$E_2$}} and our novel redirect\textit{\red{E}{Cs}{T}}.

\subsection{Compilation process/flattening}
The compilation process consists in flattening all the $E$ into $L$,
starting from the innermost leftmost $E$. This means that sum and redirect work on $LV$s:
a kind of $L$, where all the nested classes are of form \textit{\Q@private@? C\eq{}LV}.
The execution happens after compilation and consist in the conventional execution of the main expression $e$
in the context of the fully reduced declarations, where all trait composition has
been flatted away.
Thus, execution is very simple and standard and behaves like a variation of FJ[] with interfaces
instead of inheritance, and where nested classes are just a way to hierarchically organize code names.
On the other side, code composition in this setting is very interesting and powerful, where
nested classes are much more than name organization: they support in a simple and intuitive way
expressive code reuse patterns.
To flatten an $E$ we need to understand the behaviour of the two operators, and how to 
load the code of a trait: since it was written in another place, the syntactic representation of
the types need to be updated.
For each of those points we will first provide some informal explanation and then we will proceed formalizing
the precise behaviour.

\subsubsection{Redirect}

Redirect %\textit{\red{LV}{$Cs_1:T_1\ldots Cs_n:T_n$}{}}
takes a library literal and produce a modified version of it where some nested classes has been removed
and all the types referencing such nested classes are now referring to an external type. It is easy to use this feature to encode a generic list:

%final class LinkedList<T> {
%  boolean isEmpty(){ return impl == Impl.empty; }
%  T head() { return impl.asCons().elem; }
%  LinkedList tail(){ return impl.asCons().tail; }
%  LinkedList cons(T e){return new LinkedList<T>(){{
%    impl=new Cons(){{elem=e; tail=impl}};}};}
%  private final Supplier<Cons> impl = empty;
%  static final Supplier<Cons> empty = ()->{throw ..;};
%  private static class Cons implements Supplier<Cons>{
%    T elem; Impl tail; Cons get(){return this;}}
%  } 
\begin{lstlisting}
list={
  Elem={}
  static This0 empty()= new This0(Empty.of())
  boolean isEmpty()= this.impl().isEmpty()
  Elem head()= this.impl.asCons().tail()
  This0 tail()=this.impl.asCons().tail()
  This0 cons(Elem e)=new This0(Cons.of(e, this.impl)
  private Impl={interface   Bool isEmpty()  Cons asCons()}
  private Empty={implements This1
    Bool isEmpty()=true  Cons asCons()=../*error*/
    ()}//() means no fields
  private Cons={implements This1
    Bool isEmpty()=false  Cons asCons()=this
    Elem elem Impl tail }
  Impl impl
  }
IntList=list<Elem=Int>
...
IntList.Empty.of().push(3).top()==4 //example usage
\end{lstlisting}
This would flatten into
\begin{lstlisting}
list={/*as before*/
//IntList=list<Elem=Int>
IntList={
  //Elem={} no more nested class Elem
  static This0 empty()= new This0(Empty.of())
  boolean isEmpty()= this.impl().isEmpty()
  Int head()= this.impl.asCons().tail()
  This0 tail()=this.impl.asCons().tail()
  This0 cons(Int e)=new This0(Cons.of(e, this.impl)
  private Impl={interface   Bool isEmpty()  Cons asCons()}
  private Empty={/*as before*/}
  private Cons={implements This1
    Bool isEmpty()=false  Cons asCons()=this
    Int elem Impl tail }
  Impl impl
  }//everywhere there was "Elem", now there is "Int"
\end{lstlisting}

Redirect can be propagated in the same way generics parameters are propagate:
For example, in Java one could write code as below,
\begin{lstlisting}
class ShapeGroup<T extends Shape>{
  List<T> shapes;
  ..}
//alternative implementation
class ShapeGroup<T extends Shape,L extends List<T>>{
  L shapes;
  ..}
\end{lstlisting}
to denote a class containing a list of a certain kind of \Q@Shape@s.
In our approach, one could write the equivalent
\begin{lstlisting}
shapeGroup={
  MyShape={implements Shape}
  List=list<Elem=MyShape>
  List shapes
  ..}
\end{lstlisting}
With redirect, \Q@shapeGroup@ follow both roles of the two Java examples;
indeed there are two reasonable ways to reuse this code

\Q@Triangolation=shapeGroup<MyShape=Triangle>@,
if we have a \Q@Triangle@ class and we would like the concrete list type
used inside to be local to the \Q@Triangolation@,\\*
or 
\Q@Triangolation=shapeGroup<List=Triangles>@,
if we have a preferred implementation for the list of triangles that is going to be used by our
\Q@Triangolation@.
Those two versions would flatten as follow:
\begin{lstlisting}
//Triangolation=shapeGroup<MyShape=Triangle>
Triangolation={
  List=/*list with Triangle instead of Elem*/
  List shapes
  ..}

//Triangolation=shapeGroup<List=Triangles>
//exapands to shapeGroup<List=Triangles,MyShape=Triangle>
Triangolation={
  Triangles shapes
  ..}
\end{lstlisting}
As you can see, with redirect we do not decide a priori what is generic and what is not.

Redirect can not always succeed. For example, if we was to attempt
\Q@shapeGroup<List=Int>@ the flattening process would fail with an error similar to a 
invalid generic instantiation.

\subsubsection{Preferential sum; sum and redirect working together}
The sum of two traits is conceptually a trait with the sum of the traits members, and the union of the implemented interfaces.
If the two traits both define a method with the same name, some resolution strategy is applied.
In the symmetric sum[] the two methods need to have the same signature and at least one of them need to be abstract.
With preferential sum (sometimes called override), if they are both implemented, the right implementation is chosen and the left one is discarded.
Since in our model we have nested classes, nested classes with the same name will be recursively composed.

We chose preferential sum since is simpler to use in short code examples.~\footnote{symmetric sum is often presented in conjunction with a restrict operator that makes some methods abstract.}
Since the focus of the paper is the novel redirect operator, instead of the well known sum, we will handle summing state and interfaces in the simplest possible way:
a class with state can only be summed with a class without state, and an interface can only be summed with another interface with identical methods signatures.

In literature it has been shown how trait composition with (recursively composed) nested classes can 
elegantly handle the expression problem and a range of similar design challenges.
Here we will show some examples where sum and redirect cooperate to produce interesting code reuse patterns:

\begin{lstlisting}
listComp=list<+{
  Elem:{ Int geq(This e)}//-1/0/1 for smaller, equals, greater
  static Elem max2(Elem e1, Elem e2)=if e1.geq(e2)>0 then e1, else e2
  Elem max(Elem candidate)=
    if This.isEmpty() then candidate
    else this.tail().max(This.max2(this.head(),candidate))
  Elem min(Elem candidate)=...
  This0 sort()=...
  }
\end{lstlisting}
As you can see, we can \emph{extends} our generic type while refining our generic argument:
\Q@Elem@ of \Q@listComp@ now needs a \Q@geq@ method.

While this is also possible with conventional inheritance and F-Bound polymorphism, we think this solution is logically simpler then the equivalent Java
\begin{lstlisting}
class ListComp<Elem extends Comparable<Elem>> extends LinkedList<Elem>{
  ../*body as before*/
  }
\end{lstlisting}

Another interesting way to use sum is to modularize behaviour delegation: consider the following
(not efficient for the sake of compactness) implementation of \Q@set@, where the way to compare elements is not fixed:
\begin{lstlisting}
set:{
  Elem:{}
  List=list<Elem=Elem>
  static This0 empty()= new This0(List.empty())
  Bool contains(Elem e)=../*uses eq and hash*/
  Int size()=..
  This add(Elem e)=...
  This remove(Elem e)=...
  Bool eq(Elem e1,Elem e2)//abstract
  Int hash(Elem e)//abstract
  List asList //to allow iteration
  }
eqElem={
  Elem={ Bool equals(Elem e)/*abstract*/}
  Bool eq(Elem e1,Elem e2)=e1.equals(e2)
  }
hashElem={
  Elem={ Int hash(Elem e)/*abstract*/}
  Int hash(Elem e)=e.hash()
  }
Strings=(set<+eqElem<+eqHash)<Elem=String>
LongStrings=(set<+eqElem)<Elem=String> <+{
  Int hash(String e)=e.size()
  }//for very long strings, size is a faster hash
\end{lstlisting}
Note how 
\Q@(set<+eqElem<+eqHash)<Elem=String>@
is equivalent to\\*
 \Q@set<Elem=String> <+eqElem<Elem=String> <+eqHash<Elem=String>@.

Consider the signature \Q@Bool equals(Elem e)@.
This is different from the common signature \Q@Bool equals(Object e)@. What is the best
signature for \Q@equals@ is an open research question, where most approaches advise either the
first or the second one. Our \Q@eqElem@, as written, can support both:
\Q@Strings@ would be correctly define both if \Q@String.equals@ signature
has a \Q@String@ or an \Q@Object@ parameter.EXPAND on method subtyping.


\subsection{Moving traits around in the program}
It is not trivial to formalize the way types like \Q@This1.A.B@ %$\This{3}{\Q@A@}{\Q@B@}$
have to be adapted so that when code is moved around in different depths of nesting the 
refereed classes stay the same.
This is needed during flattening, when a trait $t$ is reused, but also during reduction, when a method body is inlined in the main expression, and during typing, where a method body is typed depending on the signature of other methods in the system.

To this aim we define a concept of program 
$\textit{p}\Coloneqq\textit{Ds\Q{;} DVz}$
where 
$\textit{DV}\Coloneqq\textit{id\eq{LV}}$; as a representation of 
the code as seen from a certain point inside of the source code. It is the most interesting form of the grammar,
used for virtually all reduction and typing rules. On the left of the `$;$' is a stack representing which (nested) declaration is currently being processed,
 the bottom of the stack (rightmost) \textit{D} represents the top level declaration of the source-program that is currently being processed, while the other elements of the stack are nested classes nested inside of each other.
  The right of the `$;$' represents the top-level declarations that have already been compiled, this is necessary to look up top-level classes and traits.
Summarizing, each of the $\textit{D}_0\ldots\textit{D}_n$
represents the outer nested level $0..n$, while
the \textit{DVs} component represent the already flattened portion of the program top level, that is 
the outer nested level $n+1$
%\begin{bnf}
%\production{%
%p}      {Ds\Q{;} DVs}                                                     {program}\\\production{%
%DL}     {id\eq{}L}                                                         {partially-evaluated-declaration}\\\production{%
%DV}     {id\eq{}LV}                                                       {evaluated-declaration}%\\\production{%
%Mid}    {C \mmid{} m}                                                      {member-id}%\\\production{%
%\end{bnf}
Thus, for example in the program
\begin{lstlisting}
A={()}
t={ B={()}   This1.A m(This0.B b)}
C={D={E=t}}
H=t<B=A>
\end{lstlisting}
the flattened body of \Q@C.D.E@ will be 
\Q@{ B={()}   This3.A m(This0.B b)}@, where the path
\Q@This1.A@ is now \Q@This3.A@ while the path \Q@This0.B@ stays the same: types defined internally will
stay untouched.
The program $p$ in the observation point \Q@E=t@ is
\begin{lstlisting}
A={()}
t={ B={()}   This1.A m(This0.B b)}
C={D={E=t}};
C={D={E=t}},//this means, we entered in C
D={E=t}//this means, we entered in D
\end{lstlisting}

%We will use $p$ as a function, so we will write
%.....p(T) and p(T.m)
%we define now From,
%....

In order to fetch the code literals corresponding to $t$,
we define notation $p[t]$\\*
(=\Q@{ B={()}   This3.A m(This0.B b)}@).
Such notation transforms the types so that they keep referring to the same nested classes.
We also rely on the notation $p[T]$, to extract just methods and the list of implemented interfaces, in a form were they are useful for direct comparison with 
$T$.
for example, if the program contains \Q@{B={}  This0 m(This0.B x)}@
in position 
\Q@This2.A@, $p[$\Q@This2.A@$]$ would be 
\Q@{This2.A m(This2.A.B x)}@.
%We also use notation $L[Cs=E]$ to update the code expression in $Cs$ to $E$,
%and $p\op{min}{T}=T'$ to minimize types to the required form when the $n$
%is as small as possible.



We now present formal definition for those operations.
We will use members $Mz$ as a function containing both method names $m$
and class names $C$ in its domain; thus we will assume
notation $\dom{Mz}$, $Mz(m)$, $Mz(C)$ with the usual meaning.
Under here, we define useful auxiliary notations to
access literals $L$ with functional notation with the intent of accessing their members. We define notations $L[Cs=E]=L'$ and $Mz[C=E]=Mz'$ serving the role of function update.
We use those notations to define $p(T)=LV$ accessing a program $p$ as function. We also define operations on programs: $p\op{push}{D}=p'$, allowing to work with programs as if they was stacks, and
$p\op{min}{T}=T'$, denoting the shortest type $T'$ referring to the same 
nested class of $T$.
We define $\from{T}{T',j}$ and $\from{L}{T,j}$; we omit all the trivial propagation cases of form $\from{M}{T,j}$, $\from{K}{T,j}$ and $\from{e}{T,j}$.
%Finally, we we combine those to notation for the
%most common task of getting the value of a literal, in a way that can be understand from the current location: $p[t]$ and $p[T]$:


\noindent
\begin{minipage}{0.48\textwidth}
\noindent\!\!\!$\begin{array}{l}
\hline
(\DLs\Q@;@ \DVs)\op{push}{id\eq{L}} =
\id\eq{L},\DLs\Q@;@ \DVs\\
\hline
(; \_, C\eq{L}, \_)(\This{0}{C}{Cs})=\mathit{L(Cs)}\\
p\op{push}{\_\eq{L}}(\This{0}{Cs})=L(\mathit{Cs})\\
p\op{push}{\_}(\This{n+1}{Cs})=p(\This{n}{Cs})\\
\hline
\fop{members}{\lib{\_}{Mz}{\_}}=Mz\\\hline
L(m)=\fop{members}{L}(m)\\\hline
L(C)=\fop{members}{L}(C)\\\hline
\dom{L}=\dom{\fop{members}{L}}\\\hline
\mdom{L}=\{m \in \dom{L}\}\\
\end{array}$
\end{minipage}%
\begin{minipage}{0.5\textwidth}
$\begin{array}{l}
\hline
(Mz,\Q@private@? C\eq{\_})[C=E]=Mz,\Q@private@? C\eq{E}\\\hline
LV({\emptyset})=LV\\
L(C\Q@.@Cs)=L(C)(Cs)\\
\hline
L[\Empty=E] = E\\
\lib{Tz}{Mz}{K?}[C\Q@.@Cs=E]=\\\quad
\lib{Tz}{Mz[C=Mz(C)[Cs=E]]}{K?}
\\\hline
p\op{min}{\This{n+1}{\id_n}{Cs}} = p\op{min}{\This{n}{Cs}}\\\quad
  \text{where }p = \id_0 \eq{L_0}\ldots\id_n\eq{L_n} \_\Q@;@ Ds\\
\text{otherwise } p\op{min}{T} = T\\
\end{array}$
\end{minipage}

\noindent\!\!\!\!
$\begin{array}{l}
\hline
\from{\This{n}{Cs}}{T,j}
=
\This{n}{Cs} \quad \textit{with }n<j
\\
\from{\This{n+j}{Cs}}{\This{m}{C_1\ldots C_k},j}
=
\This{m+j}{C_1\ldots C_{k-n}} \quad \textit{with }n\leq k
\\
\from{\This{n+j}{Cs}}{\This{m}{C_1\ldots C_k},j}
=
\This{m+j+n-k}{C_1\ldots C_{k-n}{Cs}} \quad \textit{with }n> k
\\
\from{
\libc{\Q@interface@? Tz}{Mz}{K}
}{T,j-1}
=
\libc{\Q@interface@? \from{Tz}{T,j}}{\from{Mz}{T,j}}{\from{K}{T,j}}
\\\hline
(DL_1\ldots DL_n; \_, t\eq{LV})[t]
=p\op{min}{\from{LV}{\This{n},0}}\\\hline
p[T]=p\op{min}{
\lib{\from{Tz}{T,0}}{\from{Mz}{T,0}}{}
}
\quad\text{where }p(T)=\lib{Tz}{Mz}{K?}
\\\hline
\end{array}$


%For space reasons, those notations are defined in the appendix.
The type system and the reduction of the main program are in appendix. They are very straight forward: thanks to flattening, they are a simple nominal type system and reduction over a FJ-like language, with no generics or special method dispatch rules.


%A declaration $D$ is just an $id = E$, representing that $id$ is declared to be the value of $E$, we also have $CD, CV, DL$, and $DV$ that constrain the forms of the LHS and RHS of the declaration.

%A literal $L$ has 4 components, an optional interface keyword, a list of implemented interfaces, a list of members, and an optional constructor. For simplicity, interfaces can only contain abstract-methods ($amt$) as members, and cannot have  constructors. A member $M$, is either an (potentially abstract) methood $mt$ or a nested class declaration $(CD)$. A member value $MV$, is a member that has been fully compiled. An $mid$ is an identifier, identifying a member.
%Constructors, $K$, contain a $Txs$ indicating the type and names of fields. An $e$ is normal fetherweight-java style expression, it has variables $x$, method calls $e.m(es)$, field accesses $e.x$ and object creation $new es$.

%$CtxV$ is the evaluation context for class-expressions $E$, and $ctxv$ is the usuall one for $e$’s.
\begin{comment}
Define operations on p
--------------------------------------
p.evilPush(L) = (C = L, p)
	for fresh C

p.push(id) = (id = L, p)
    p = (id' = {_;_, id = L, _ ;_}, _; Ds)

(id = L, p).pop() = p
(id = L, p).top() = L

Define equivy ops...
------------------------------
empty =p empty
P, Ps =p P', Ps' iff:
	p.minimize(P) = p.minimize(P')
	Ps =p Ps'

Pz subseteq_p Pz' iff:
	p.minimize(Pz) subseteq p.minimize(Pz')

p.minimize(empty) = empty
p.minimize(P, Pz) = p.minimize(P), p.minimize(Pz)

p.minimize(Thisn+1.idn.Cs) = p.minimize(Thisn.Cs):
  p = id0 = L0, ..., idn = Ln, _; Ds
  p(Thisn.Cs) = L
  // TODO: Check that Ln is an LV instead?

otherwise p.minimize(P) = P

define dom(Mz) = Midz
===========================================
dom(empty) = empty
dom(C = E, Mz) = C, dom(Mz)
dom(T m(Txs), Mz) = m, dom(Mz)
\end{comment}

\section{Flattening}

%Aside from the redirect operation itself, compilation/flattening is the most interesting part.
Flattening is defined by reduction arrow $\Ds \Rightarrow \Ds'$, where eventually $\Ds'$ is going to reach form $\DVs$  and $p; \id \vdash E \Rightarrow E’$, where eventually $E'$ is going to reach form $LV$. The $\id$ represents the identifier of the type/trait that we are currently compiling, it is needed since it will be the name of \This0, and we use to the fact that refers to the same nested class as $\This1{\id}$.
Rule \textsc{(Top)}  selects the leftmost $\id\eq{E}$
where $E$ is not of form $LV$ and $\DVz$: a 
well typed subset of the preceeding declarations. 
$E$ is flattened in the contex of such $\DVz$, thus
by rule \textsc{(Trait)} $\DVz$ must contain all the trait names used in $E$.
In the judgement $p; \id \vdash E \Rightarrow E’$
$\id$ is only used in order to grow the program $p$ in rule 
\textsc{(L-enter)}, and $p$ itself is only needed for 
\textsc{(redirect)}.
The \textsc{(CtxV)} rule is the standard context, the \textsc{(L-enter)} rule propegates compilation inside of nested-classes, \textsc{(trait)} merely evaluates a trait reference to it’s defined body,
finally \textsc{(sum)} and \textsc{(redirect)} perform our two meta-operations by propagating to 
corresponding auxiliary definitions. We will present those two rules in the two sections below.
Note how we require their input to be already in the \emph{minimized}
form, that is, all the $T$ uses the shortest way to refer to their corresponding nested class. 
This prevents the programmer from expressing some difficult cases. Consider for example using two different ways to refer to $A$, redirect $A$ and then adding it back:
\begin{lstlisting}
B=...
X={ A:{}    Void m(This1.X.A p1, This0.A p2)} <A=B> <+ {A:{}}
//should flattening redirect only p2 or also p1
X={ A:{}    Void m(??? p1, This1.B p2)}
\end{lstlisting}
The complete L42 language solves those issues, but here we present a simplified version.
%For simplicity rule \textsc{(sum)} is given in a highly non computational form,
%where non deterministically we select the result $LV_3$ and we use it in $p'$ and we also
%require it to be the result of 
%$LV_1 \Q@<+@_{p'} LV_2 = LV_3$.
%This rule uses $p'$ only to check for errors.
\subsection{Sum}
Rule \textsc{(sum)} just delegate the work on the auxiliary notation defined below:

\noindent$\begin{array}{l}
\hline
L_1 \Q@<+@ L_2 = \lib{Tz_1 \cup Tz_2}{Mz \Q@<+@ Mz',Mz_1,Mz_2}{K?} \\
\quad  L_1=\lib{Tz_1}{Mz,Mz_1}{K?_1},\quad
\quad  L_2=\lib{Tz_2}{Mz',Mz_2}{K?_2}\\
\quad  \{\Empty, K?_1, K?_2\} = \{\Empty, K?\}\\
\quad  \text{if}\ \Q@interface@? = \Q@interface@\  \text{then}\ \mdom{L_1}= \mdom{L_2}\\
\hline
T m\rp{Txs} e? \Q@<+@ T m\rp{Txs} e = T m\rp{Txs} e\\
T m\rp{Txs} e? \Q@<+@ T m\rp{Txs} = T m\rp{Txs} e?\\
(C \Q@=@ L) \Q@<+@
 (C \Q@=@ L') = C \Q@=@\ L \Q@<+@ L,% \text{with } p'=p\op{push}{C\Q@=@p(\This{0}{C}}
\\\hline
\end{array}$

%On its right, we define the used auxiliary notation,
%showing how to sum literals and members.
As usual in definitions of sum operators,
the implemented interfaces is the union of the interfaces of $L_1$ and $L_2$, the members with the same domain are recursivelly composed while the members with disjoint domains are directly included.
Since method and nested class identifiers must be unique in a well formed $L$ and $M_1 \Q@<+@ M_2$  being defined only if the identifier is the same,
our definition forces $\dom{Mz}=\dom{Mz'}$ and
$\dom{Mz_1}$ disjoint $\dom{Mz_2}$.
For simplicity here 
 we require at most one class to have a state; if both have no state, the result will have no state, otherwise the result will have the only present state (the set $\{empty,K?\}$ mathematically express this requirement in a compact way);
we also allow summing
only interfaces with interfaces and final classes with final classes. When two interfaces are composed both sides must define the same methods.
This is because other nested classes inside $L_1$ may be implementing such interface, and adding methods to such interface would require those classes to somehow add an implementation for those methods too.
In literature there are expressive ways to soundly handle merging different state, composing interfaces with final classes and
adding methods to interfaces, but they are out of scope in this work.

Member composition $M_1 \Q@<+@ M_2$ uses
the implementation from the right hand side, if available,
otherwise if the right hand side is abstract, the body is took from the left side.
Composing nested classes, note how they can not be \Q@private@; it is possible to sum two literals only if their private nested classes have different private names. This constraint can always be obtained by alpha-renaming them:
we assume a form of alpha-reaming for private nested classes, that will consistently rename all the 
paths of form $\This{n}{C}{Cs'}$, where 
$\This{n}{C}$ refer to such private nested class. The trivial definition of such alpha renaming is given in the appendix.


\subsection{Redirect}
Rule \textsc{(redirect)} is the centre of our interest for this work. As for sum we check that the $LV$ is in minimized form.
Moreover, to have a single data structure $p'$ where all the types correctly points to the corresponding nested classes, we add the $L$ to the top of our current program. 
Notation $R/\id$ is defined as\\*
$\begin{array}{l}
\hline
Cs_0\eq{\This{n}{C}{Cs}}=Cs_0\eq{\This{n+1}{C}{Cs}},
\text {where either } C\neq\id \text { or } n>0
\\\hline
\end{array}$

In addition of adding $1$ to all the types provided in the redirect map, since they was relative to $p$ and not $p'$, it also 
checks that $R$ actually refers to types external of $LV$, by preventing types of form
$\This{0}{\id}{\_}$.

Notation $p\op{redirectSet}{R}$
computes the set of nested classes that need to be redirected if $R$ is redirected. This is information depend just from $LV$ (the top of the program) and the domain of $R$. RedirectSet is easly computable.

\noindent $\begin{array}{l}
\hline
\dom{R} \subseteq p\op{redirectSet}{R}\\
\fop{internals}{\fop{exposedTypes}{p[\This{0}{Cs}]}} \subseteq p\op{redirectSet}{R}
\quad \text{with } Cs \in p\op{redirectSet}{R}\\
\hline
\fop{exposedTypes}{\lib{Tz}{Mz}{K?}} = Tz, \fop{exposedTypes}{Mz}\\
\fop{exposedTypes}{\Q@static@? T_0 m\rp{T_1 x_1\ldots T_n x_n} e?}= T_0\ldots T_n\\
\hline
\fop{internals}{Tz}=\{Cs\mid \This{0}{Cs} \in Tz\}
\\\hline
\end{array}$

The intuition behind \opName{redirectSet} is that if the signature of a nested class mention another nested class, they must be redirected together.
Consider the following simple example:
\begin{lstlisting}
t={A={B size()} B={} ...}
Res=t<A=String>
\end{lstlisting}
If we were to redirect \Q@A@, we would need to redirect also \Q@B@:
the type \Q@B@ is nested inside \Q@t@, thus
\Q@String@ would not be able to reach it.
The only reasonable solution is to redirect \Q@A@ and \Q@B@ together.

For our redirection (and $p'\op{bestRedirection}{}$) to be well defined, we need to check that $p\op{redirectable}{Csz}$
This is again a check local to the $LV$ (the top of the program) and is also easily computable.

\noindent $\begin{array}{l}
\hline
\fop{redirectable}{p,Csz} \text{iff}\\\quad
    \Empty \notin Csz \\\quad
    \text{if } Cs\in Csz \text{ then } \This{0}{Cs} \in \dom{p}\\\quad
    \text{if } Cs\in Csz \text{ and }C \in\dom{p(\This{0}{Cs})}
    \text{ then } Cs\Q@.@C \in Csz\\\quad
    \text{if } Cs\Q@.@C\Q@.@\_\in Csz
    \text{ then } p(\This{0}{Cs}) =\lib{\_}{C \eq{L}\_}{\_}    
\\\hline
\end{array}$

That is, the empty path is not redirectable, every nested class of a redirect path must be redirected away,
and all paths must traverse only non-private $C$.

Finally,  $p\op{bestRedirection}{R}$, given
a $p$ and an $R$ (that are valid input for redirection as defined above)
can denote the best complete map, mapping any element of $Csz$ into a suitable type in $p$.
This is the centerpiece of our formal framework and his definition will be the main topic of the next section.

Given the complete mapping $R'$, to produce the flattened result we first
remove all the elements of $Csz$ from $LV$, and then we
apply $R'$ as a rename, renaming all internal paths $Cs \in Csz$ to the corresponding external type $R'(Cs)$.
Those two notations are formally defined as following:

\noindent$\begin{array}{l}
\hline
LV\op{remove}{Cs_1\ldots Cs_n}=LV\op{remove}{Cs_1}\ldots\op{remove}{Cs_n}\\
LV[Cs\Q@.@C=\_]\op{remove}{Cs\Q@.@C}=LV \text{where } Cs\Q@.@C \notin\dom{LV}\\
\hline
R(L)=R_\Empty(L)\\
R_{Cs}(\lib{Tz}{Mz}{K?})=\lib{R_{Cs}(Tz)}{R_{Cs}(Mz)}{R_{Cs}(K?)}\\
R_{Cs}(C\eq{L})=C\eq{R_{Cs\Q@.@C}(L)}\\
R_{Cs}(M),R_{Cs}(e),R_{Cs}(K)\quad \text{ simply propagate on the structure until $T$ is reached}\\
R_{C_1\ldots C_n}(T)=\This{n+k+1}{Cs'}\quad\text{where }T\op{from}{\This0{C_1\ldots C_n}}=\This0{Cs},\ 
R(Cs)=\This{k}{Cs'}\\
\text{otherwise } R_{Cs}(T)=T
\\\hline
\end{array}$

%The second clause of \opName{remove} requires the $Cs$ to be ordered in such a way where the inner-most nested classes are removed first.
Rename must keep track of the explored $Cs$ in order to distinguish
internal paths that need to be renamed, and the mapped type need to look out of the whole explored $Cs$ and the top level code literal (thus $n+k+1$).



\begin{figure}
  \caption{Flattening}
\noindent$\begin{array}{l}
\hline
Ds\Rightarrow Ds' \text{ and } p;id\vdash E\Rightarrow E',  \text{where   
\begin{bnf}
\production{%
\ctx{V}}{\hole \mmid{}  \summ{\ctx{V}}{E} %
                \mmid{}  \summ{LV}{\ctx{V}} \mmid{} \red{\ctx{V}}{Cs}{T}}  {}
\end{bnf}}\\\hline
\\
%\inferrule[(top)]{
%	a \xrightarrow[b]{} c\quad
%	\forall i<3 a\vdash b:\text{OK}\\\\
%	\forall i<3 a\vdash b:\text{OK}
%}{
%	1+2
%	\rightarrow
%	3
%}\begin{array}{l}
%a\\b\\c
%\end{array}
%\\
\inferrule[(Top)]{
\DVz \subseteq \DVs\\\\
\DVz \vdash \textbf{Ok}\\\\
\Empty; \DVz; id \vdash E \Rightarrow E'
}{
\DVs\ \id \Q@=@ E \Ds \Rightarrow \DVs\ \id \Q@=@ E' \Ds
}\quad

\quad\quad
\inferrule[(L-enter)]{
p\op{push}{id \Q{=} L[C = E]}; C \vdash E \Rightarrow E'
}{
p; \id \vdash L[C = E] \Rightarrow L[C = E']
}

\quad\quad
\inferrule[(trait)]{
}{
p; \id \vdash t \Rightarrow p[t]
}

\\[5ex]
\inferrule[(sum)]{
LV_i=p\op{min}{\id\eq{LV_i}}\\\\
LV_1 \Q@<+@ LV_2 = LV%\\\\
%C' \textit{fresh} \\\\
%p'=p\op{push}{C'\Q@=@ LV_3}
}{
p; \id \vdash LV_1 \Q@<+@ LV_2 \Rightarrow LV
}
\quad\quad
\inferrule[(redirect)]{
  LV=p\op{min}{\id\eq{LV}}\\\\
  p' = p\op{push}{\id\eq{LV}}\\\\
  Csz = p'\op{redirectSet}{R/\id}\\\\
  p'\op{redirectable}{Csz}\\\\
  R' =p'\op{bestRedirection}{R/\id}
}{
p; \id \vdash LV \Q@<@R\Q@>@ \Rightarrow   R'(LV\op{remove}{Csz}) 
}\\[5ex]\hline
\end{array}$
\end{figure}


%We have two-top level reduction rules defining our language, of the form $Ds e ––> Ds’ e$ which simply reduces the source-code.
%The first rule $(compile)$ ‘compiles’ each top-level declaration (using a well-typed subset of allready compiled top-level declarations), this reduces the defining expresion.
%The second rule, $(main)$ is executed once all the top-level declarations have compiled (i.e. are now fully evaluated class literals), it typechecks the top-level declarations and the main expression, and then procedes to reduce it.
%In principle only one-typechecking is needed, but we repeat it to avoid declaring more rules.
%\begin{verbatim}
%DVs |- Ok
%DVs |- e : T
%DVs |- e --> e'
%(main)---------------------------------- for some type T
%DVs e --> DVs e'
%\end{verbatim}

%\begin{defye}%
%	\defy{L[C\eq{E'}]}{\lib{Tz}{\s{MV}\ C\eq{E'}\ \s{M}}{K?}}
%	\defyc{\text{where } L = \lib{Tz}{\s{MV}\ C\eq{\_}\ \s{M}}{K?}}
%	\defy{\s{T} \in p}{\forall T \in \s{T} \bullet p(T) \text{ is defined}}% WHERE WE USE IT?
%\end{defye}


%We will also use $p[T.m]$ to extra

\section{BestRedirect}
Best redirection balance three aspects:
\begin{itemize}
\item Validity: if the mapping is applied to well typed code (as in the rule \textsc{(redirect)}) then the result is still well typed.
\item Stability: changing little details on the code base (as for example adding a new unrelated nested class) do not change the selected map.
This applies to both $LV$ itself (internal stability)
and the rest of the program (external stability).
\item Specificity: when multiple options are available, the most specific is chosen.
\end{itemize}

To better divide the various aspect, we will use
functions of form $(p,R)\rightarrow Rz$, producing valid mappings
for any program $p$ and starting map $R$.
All of those functions will respect 
\opName{possibleRedirections}.
Rule \textsc{redirect} ensures 
\opName{possibleRedirections} for the input mapping,
here we check that is also verified for the complete mapping.

\noindent$\begin{array}{l}
\hline
R' \in \fop{possibleRedirections}{p, R} \text{ if }\\\quad
R \subseteq R'\\\quad
\dom{R'} = \fop{redirectSet}{p, R}\\\quad
(p, R') \in \opName{validProblems}\\
\hline
(p, Cs_1\eq{T_1}\ldots Cs_n\eq{T_n}) \in \opName{validProblems} \text{ iff }
\forall i \in 1..n:\\\quad
    p\op{minimize}{T_i}=T_i\\\quad
    T_i \text{not of form } \This0{\_}\\\quad
    p \vdash p[T] : \textbf{OK}\\\quad % // Guaranteed by our reduction rules and type-system? // TODO is this the right from?
    \fop{redirectable}{p, \fop{redirectSet}{p, R}}
%  // All of these, save for the well-typedness, are guaranteed by Redirect
\\\hline
\end{array}$

We now define $\opName{validRedirections}$ as one of such functions.
This is the most complete function achieving both validity and
internal stability. It is based on the judgement
$p \vdash T \subseteq L$
to be read as: under the program $p$,
 T is structurally a subtype of the literal $L$.
Some more auxiliary notation is used: the obvious \opName{isInterface}
and the more interesting 
\opName{superClasses} and method subtyping
$p\vdash M \leq M'$.
In \opName{superClasses} we add $T$ so that F-Bound polymorphism may work as expected, so that is possible to redirect \Q@{implements Foo}@
not only to any class implementing \Q@Foo@ but also to
\Q@Foo@ itself.
Method subtyping is given in the expressive form where the return type can be more specific, and the parameter types can be more general.

\noindent$\begin{array}{l}
\hline
R' \in \fop{validRedirections}{p,R}\ \text{iff}\\\quad
  R' \in \fop{possibleRedirections}{p, R}\\\quad
  \forall Cs \in \dom{R'}\ \, p \vdash  p[R'(Cs)] : R'(Cs) \subseteq R'(p[Cs]):Cs\\
%\end{array}$
%
%//SuperClasses is Pz',P,Any. In this way F-bound polimoprhism works as usual: {implements Foo} can be redirected to Foo
%\noindent$\begin{array}{l}
\hline
p \vdash P \subseteq \lib{Tz}{Mz}{\_}\text{ iff } \\\quad
% p|- P; {interface?' implements Pz' mwtz', ncz'} <= Cs; {interface? implements Pz mwtz, ncz}
 Tz \subseteq \fop{superClasses}{p, P}\\\quad 
 \forall m \in \dom{Mz}: \ \,
%// This implicity checks sdom(mwtz) subseteq sdom(mwtz') 
    p \vdash p[P](m) \leq Mz(m)\\\quad
 \text{if }\Q@interface@?=\Q@interface@ \text{ then  } 
%// If the LHS is not an interface
%// One can only call class methods on a non-interface, so if the RHS has them, than the LHS can't be an interface
%  // If the LHS is an interface, we need to ensure that any valid implementation of the LHS
%  // Is a valid implemention of the RHS, which requires that the RHS have the exact same method signatures as the LHS
    \forall m \in \dom{p[P]}\ \,  %// This implicity checks the sdom
      p \vdash Mz(m) \leq p[P](m)\\\quad

  \text{if } \fop{interface}{p[P]} \text{ then  }
    \Q@static@ T m\rp{Txs}\_ \notin Mz
\text{ else } \Q@interface@?=\Empty
\\
\hline
\fop{isInterface}{L} \text{iff } L=\libi\_\_\\
\hline
\fop{superClasses}{p,T}=\{T\}\cup
\fop{superClasses}{T_1}\cup\ldots\cup\fop{superClasses}{T_n}
\\\quad\text{ with }\ p[T]=\lib{T_1\ldots T_n}\_\_\\
\hline
p\vdash 
\Q@static@?\, T'_0\, m\rp{T_1 x_1\ldots T_n x_n}\_\leq
\Q@static@?\, T_0\, m\rp{T'_1 x'_1\ldots T'_n x'_n}\_
\\\quad\text{ with }
T_0\in \fop{superClasses}{p,T'_0}\ldots
T_n\in \fop{superClasses}{p,T'_n}
\\\hline
\end{array}$

Note how $\opName{validRedirections}$, while mathematically sound,
is incredibly hard to compute:
while it is easy to check if a certain 
$R' \in \fop{validRedirections}{p,R}$, finding naively all such $R'$
would require examining every possible permutation.
In particular, subtyping allows for redirections to be conceptually took out of thin-air.
Consider the following example:
\begin{lstlisting}
I=interface {..}
A= {method A m(I x)}
C={implements I ..}
t={B: {} T: {method T m(B x)}}
Res=t<T=A>
\end{lstlisting}
Clearly, selecting \Q@C@ as a candidate to complete the map is a valid choice but is also an arbitrary choice that should not be made while automatically completing the mapping. What if type \Q@D={implements I ..}@
was introduced while maintaining the program? the completed redirect map may change unpredictably.
As you can see from the former example, stability is an important requirement to allow for code maintainability.
To model stability, we define the concept of 
similar programs:

\noindent$\begin{array}{l}
\hline
DLs; DVz\, DVz'\in \fop{similarPrograms}{DLs; DVz}
\\\hline
\end{array}$

Note how we just add new declarations at the outermost level.
We will later prove that this is sufficient to ensure that 
adding/removing unrelated classes anywhere in the program would still not change the selected completed mapping.
Finally, we have all formal tools to define 
\opName{bestRedirection}, representing the high level specification of what correctly completing a mapping means.

\noindent$\begin{array}{l}
\hline
\fop{bestRedirection}{p, R} = \fop{stableMostSpecific}{p, R, \opName{validRedirections}}\\
%\hline
%\fop{stableMostSpecific}{p, R, f} = R'
% \text{ iff }\forall p' \in \fop{similarPrograms}{p}\\\quad \fop{mostSpecificRedirection}{p', f(p, R)} = R'\\
\hline
\fop{stableMostSpecific}{p, R, f} = R_0
 \text{iff } \forall p' \in \fop{similarPrograms}{p}\\\quad
      R_0 \in f(p',R)\text{ and }
\forall R_1 \in f(p',R) \ \, \fop{moreSpecific}{p,R_0,R_1}
\\
\hline
\fop{moreSpecific}{p,
  Cs_1\eq{T_1}\ldots Cs_n\eq{T_n},
  Cs_1\eq{T'_1}\ldots Cs_n\eq{T'_n}
}\\\quad
  T'_1 \in \fop{superClasses}{p,T_1}\ldots   T'_n \in \fop{superClasses}{p,T_n}
\\\hline
\end{array}$

The best redirection is a \opName{validRedirection}
that is the most specific across all similar programs.
While \opName{bestRedirection} in the current form is not
practically computable, it is clear from the formulation
a good stepping stone to obtain a computable algorithm 
would be to
replace \opName{validRedirections}
with an computable algorithm producing a subset of \opName{validRedirections} and behaving identically for
all the \opName{similarPrograms}.

If multiple solutions are available, providing one of those non deterministically would clearly break stability.
In this case, instead of just refusing to complete the mapping, we attempt to find the most specific solution: a solution where every individual mapping maps to the most specific type with respect to all the other available mappings.
Choosing the most specific solution is the desired solution in many practical cases; for example consider this variation of the former example, where
\Q@B@ is the return type instead of an argument type:

\begin{lstlisting}
I=interface { ..}
C={implements I ..}
A={C m()=..}
t={B={ } T={method B m()} ..}
Res=t<T=A>
\end{lstlisting}
\opName{bestRedirection} complete this mapping as \Q@<T=A, B=C>@ thanks to
choosing the most specific, since also \Q@B=I@ is a valid option.
In a language with a global supertype like \Q@Any@/\Q@Object@, that would
be yet another option.
Indeed, an alternative version selecting the  least specific
option may complete the mapping selecting \Q@Any@/\Q@Object@
every time a nested was declared with empty body. That in turn is very common since it is the Java equivalent of not requiring any
\Q@extends T@ constraints on a generic type.


\section{Properties of \opName{bestRedirection}}
\subsection{Internal/external stability}
\subsection{Meta-Level soundness}

\section{A computable \opName{bestRedirection}: \opName{choseRedirection}}

\section{Redirect applications}

\subsection{Graph example}
We now consider an example where Redirect simplifies the code quite a lot:
We have a \Q@Node@ and \Q@Edge@ concepts for a graph.
The \Q@Node@ have a list of \Q@Edge@s.
A \Q@isConnected@ function takes a list of \Q@Node@s.
A \Q@getConnected@ function takes \Q@Node@ and return a set of \Q@Node@s.
\begin{lstlisting}
graphUtils={
  Edges:list<+{Node start() Node end()}
  Node:{Edges connections()}
  Nodes:set<Elem=Node>//note that we do not specify equals/hash
  static Bool isConnected(Nodes nodes)=
    if(nodes.size()=0) then true
    else getConnected(nodes.asList().head()).size()==nodes.size()
  static Nodes getConnected(Node node)=getConnected(node,Nodes.empty())
  static Nodes getConnected(Node node,Nodes collected)=
    if(collected.contains(node)) then collected
    else connectEdges(node.connections(),collected.add(node))
  static Nodes connectEdges(Edges e,Nodes collected)=
    if( e.isEmpty()) then collected
    else connectEdges(e.tail(),collected.add(e.head().end()))
  }
\end{lstlisting}

We have shown the full code instead of omitting implementations to show that
the code inside of an highly general code like the former is pretty conventional.
Just declare nested classes as if they was the concrete desired types. Note how we can easly create a new \@Nodes@ by doing \Q@Nodes.empty()@.

Here we show how to instantiate \Q@graphUtils@ to a graph representing cities connected by streets,
where the streets are annotated with their length, and \Q@Edges@ is a priority queue, to optimize
finding the shortest path between cities.

\begin{lstlisting}
Map:{
  Street:{City start,City end, Int size}
  City:{}
  Streets:priorityQueue<Elem=Street><+{    
    Int geq(Street e1,Street e2)=e1.size()-e2.size()}
  }<+{
  Streets:{}
  City:{Streets connections, Int index}//index identify the node
  Cities:set<Elem=City><+{
    Bool eq(City e1,City e2) e1.index==e2.index
    Int hash(City e) e.index
    }
  Cities cities
  //more methods
  }
MapUtils=graphUtils<Nodes=Map.Cities>
//infers Nodes.List, Node, Edges, Edge
\end{lstlisting}

In Appending 2 we will show our best attempt to encode this graph example in Java, Rust and Scala.
In short, we discovered...

\subsection{Loading libraries}

Most languages have a standard library.
The standard library have two goals:
\begin{itemize}
\item Be a set of useful features for programmers to use.
\item Be a starting point for third party libraries.
\end{itemize}
While the first point is quite obvious, the second one is a little surprising: third party libraries will communicate with
the user code mostly by using standard library types:
\Q@String@s, \Q@Collection@s and if we are in a pure OO language,
also \Q@Boolean@s, \Q@Integer@s, \Q@Double@s and so on.
The number of types involved in just take input and produce output is much larger then one could expect, since all kind of errors need
to be considered too, and if the language support reflection, all those classes and their errors may end up being transitively required.
The goal of the standard library is.
For example, assume a simple library taking in input a \Q@String@
and producing a \Q@String@: What are its dependencies?

\Q@String@ has an \Q@isEmpty():Boolean@ method, 
the \Q@Boolean@ has a \Q@toString():String@ method, a circular dependency.
\Q@String@ has a \Q@size():Integer@ method, and
a \Q@getChar(Integer pos):Char@ method.
another typical method of strings is @split(String regex):ListString@,
returning a list, that extends some general collection, and so on.
If reflection is not implemented with Mirrors[],
any of those object would have a \Q@class():Class@ method,
and \Q@Class@ would have methods to connect with most other reflection classes.

Those dependencies are usually not a problem because we assume the standard library to be fixed and always available.
Or, if you prefer, we are forced to design programming languages together with their standard library because those dependencies are too hard to manage directly.

With Redirect we can get free from this chain, and every third party library can just declare a the set of dependencies that are really needed.
A single redirect application can then ``load'' the library in the current scope, where a variation of the standard library is available, but not necessarily exactly the library used to develop such third party library.

For example, a library may only need pass indexes around, without directly
doing arithmetic, and may never use ..

\begin{lstlisting}
library={//this code is fully self contained
  N={}
  C={ Bool equal(C x)}
  S={N size(), C getChar(N index)}
  S myLibFunction(S x)=....
  Map=interface{
    S string()
    C char()
    N integer()
    }
  }
...
Int={..}
Char={..}
String={..}
Map=interface{String string() ... }
LoadedLib=library<Map=Map>
\end{lstlisting}
Since the code of $t$ is self contained (do not refer to any class in the outer program, it is possible to just ship it independently of the standard library of the target.
The code of $library$ can be typechecked once, and then
any other program may load it as shown.
Any program defining a \Q@Map@ interface with some types that we expect libraries to rely upon can be used in conjunction with $library$.
In this example, if \Q@Char@ is not a valid structural subtype of \Q@C@,
the redirect would fail with a meaningful error message.

By the stability theorem, we get a good formal characterization of what are the acceptable shapes of the program so that the redirect would succeed.
However, even if the program do not match the expectations of the library,
it could be possible to tweak the code to make it work.
...

\section{Appendix?}

PUT LATER?However, he type system of the language is more restrictive when 
it comes to refine an interface method, allowing only return type refinement. This is not just to align our calculus with existing languages like Java/C\# and C++, but is required to make reasoning about parameter types influential while expanding redirect mappings.END PUT LATER

\begin{bnf}
\production{%
\ctx{V}}{\hole \mmid{}  \summ{\ctx{V}}{E} %
                \mmid{}  \summ{LV}{\ctx{V}} \mmid{} \red{\ctx{V}}{Cs}{T}}  {context of library-evaluation}\\\production{%
%LV}     {\libi{Tz}{amtz}{}\ \ \mmid{}\ \ \libc{Tz}{MVs}{K$?$}}       {literal value}\\\production{%
%MV}     {C\eq{}LV \mmid{} mt}                                                    {}\\\production{%
\ctx{v}}{\hole \mmid{}  \ctx{v}\Q{.}m\rp{es} \mmid{}  v\Q{.}m\rp{vs \ctx{v} es} %
	\mmid{} T\Q{.}m\rp{vs \ctx{v} es}  }           {}%\\\production{%
%DL}     {id\eq{}L}                                                         {partially-evaluated-declaration}\\\production{%
%DV}     {id\eq{}LV}                                                       {evaluated-declaration}\\\production{%
%Mid}    {C \mmid{} m}                                                      {member-id}\\\production{%
%p}      {DLs\Q{;} DVs}                                                     {program}
\end{bnf}



\section{Type System}

The type system is split into two parts: type checking programs and class literals, and the typechecking of expressions. The latter part is mostly convential, it involves typing judgments of the form $p; Txs \vdash e : T$, with the usual program $p$ and variable environement $Txs$ (often called $\Gamma$ in the literature). rule ($Ds ok$) type checks a sequence of top-level declarations by simply push each declaration onto a program and typecheck the resulting program.
Rule $p ok$ typechecks a program by check the topmost class literal: we type check each of it’s members (including all nested classes), check that it properly implements each interface it claims to, does something weird, and finanly check check that it’s constructor only referenced existing types,

\begin{verbatim}


Define p |- Ok
===========================================================

D1; Ds |- Ok ... Dn; Ds|- Ok
(Ds ok) ------------------------------ Ds = D1 ... Dn
Ds |- Ok

p |- M1 : Ok .... p |- Mn : Ok
p |- P1 : Implemented .... p |- Pn : Implemented
p |- implements(Pz; Ms) /*WTF?*/                   if K? = K: p.exists(K.Txs.Ts)
(p ok) ------------------------------------------- p.top() = interface? {P1...Pn; M1, ..., Mn; K?}
p |- Ok

p.minimize(Pz) subseteq p.minimize(p.top().Pz)
amt1 _ in p.top().Ms ... amtn _ in p.top().Ms
(P implemented) ----------------------------------------------- p[P] = interface {Pz; amt1 ... amtn;}
p |- P : Implemented

(amt-ok) ------------------- p.exists(T, Txs.Ts)
p |- T m(Tcs) : Ok

p; This0 this, Txs |- e : T
(mt-ok) ------------------------------ p.exists(T, Txs.Ts)
p |- T m(Tcs) e : Ok

C = L, p |- Ok
(cd-Ok) -------------------
p |- C = L : OK

\end{verbatim}

Rule $(P implemented)$ checks that an interface is properly implemented by the program-top, we simply check that it declares that it implements every one of the interfaces super-interfaces and methods.
Rules $(amt-ok)$ and $(mt-ok)$ are straightforward, they both check that types mensioned in the method signature exist, and ofcourse for the latter case, that the body respects this signature.

To typecheck a nested class declaration, we simply push it onto the program and typecheck the top-of the program as before.


The expression typesystem is mostly straightforward and similar to feartherwieght Java, notable we we use $p[T]$ to look up information about types, as it properly ‘from’s paths, and use a classes constructor definitions to determine the types of fields.

\begin{verbatim}
Define p; Txs |- e : T
=====================================
(var)
----------------------- T x in Txs
p;  Txs |- x : T

(call)
p; Txs |- e0 : T0
...
p; Txs |- en : Tn
-----------------------------------  T' m(T1 x1 ... Tn xn) _ in p[T0].Ms
p; Txs |- e0.m(e1 ... en) : T'

(field)
p; Txs |- e : T
---------------------------------------  p[T].K = constructor(_ T' x _)
p; Txs |- e.x : T'


(new)
p; Txs |- e1 : T1 ... p; Txs |- en : Tn
------------------------------------------- p[T].K = constructor(T1 x1 ... Tn xn)
p; Txs |- new T(e1 ... en)


(sub)
p; Txs |- e : T
-----------------------------------  T' in p[T].Pz
p; Txs |- e : T'


(equiv)
p; Txs |- e : T
-----------------------------------  T =p T'
p; Txs |- e : T'
\end{verbatim}





FROM and minimize that will go in the appendix:

To fetch a trait form a program, we will use notation $p(t)=LV$, to 
fetch a class we will use $p(T)$.

To look up the definition of a class in the program we will use the notation
%$p(t)=LV$ and
$p(T)=\textit{LV}$, which is defined by the following:% but only if the RHS denotes an $LV$:

We will use members $Mz$ as a function containing both method names $m$
and class names $C$ in its domain; thus we will assume
notation $\dom{Mz}$, $Mz(m)$, $Mz(C)$ with the usual meaning.
Under here, we define useful auxiliary notations to
access literals $L$ with functional notation with the intent of accessing their members. We define notations $L[Cs=E]=L'$ and $Mz[C=E]=Mz'$ serving the role of function update.
We use those notations to define $p(T)=LV$ accessing a program $p$ as function. We also define operations on programs: $p\op{push}{D}=p'$, allowing to work with programs as if they was stacks, and
$p\op{min}{T}=T'$, denoting the shortest type $T'$ referring to the same 
nested class of $T$.
We define $\from{T}{T',j}$ and $\from{L}{T,j}$; we omit all the trivial propagation cases of form $\from{M}{T,j}$, $\from{K}{T,j}$ and $\from{e}{T,j}$.
Finally, we we combine those to notation for the
most common task of getting the value of a literal, in a way that can be understand from the current location: $p[t]$ and $p[T]$:


\noindent
\begin{minipage}{0.48\textwidth}
\noindent\!\!\!$\begin{array}{l}
\hline
(\DLs\Q@;@ \DVs)\op{push}{id\eq{L}} =
\id\eq{L},\DLs\Q@;@ \DVs\\
\hline
(; \_, C\eq{L}, \_)(\This{0}{C}{Cs})=\mathit{L(Cs)}\\
p\op{push}{\_\eq{L}}(\This{0}{Cs})=L(\mathit{Cs})\\
p\op{push}{\_}(\This{n+1}{Cs})=p(\This{n}{Cs})\\
\hline
\fop{members}{\lib{\_}{Mz}{\_}}=Mz\\\hline
L(m)=\fop{members}{L}(m)\\\hline
L(C)=\fop{members}{L}(C)\\\hline
\dom{L}=\dom{\fop{members}{L}}\\\hline
\mdom{L}=\{m \in \dom{L}\}\\
\end{array}$
\end{minipage}%
\begin{minipage}{0.5\textwidth}
$\begin{array}{l}
\hline
(Mz,\Q@private@? C\eq{\_})[C=E]=Mz,\Q@private@? C\eq{E}\\\hline
LV({\emptyset})=LV\\
L(C\Q@.@Cs)=L(C)(Cs)\\
\hline
L[\Empty=E] = E\\
\lib{Tz}{Mz}{K?}[C\Q@.@Cs=E]=\\\quad
\lib{Tz}{Mz[C=Mz(C)[Cs=E]]}{K?}
\\\hline
p\op{min}{\This{n+1}{\id_n}{Cs}} = p\op{min}{\This{n}{Cs}}\\\quad
  \text{where }p = \id_0 \eq{L_0}\ldots\id_n\eq{L_n} \_\Q@;@ Ds\\
\text{otherwise } p\op{min}{T} = T\\
\end{array}$
\end{minipage}

\noindent\!\!\!\!
$\begin{array}{l}
\hline
\from{\This{n}{Cs}}{T,j}
=
\This{n}{Cs} \quad \textit{with }n<j
\\
\from{\This{n+j}{Cs}}{\This{m}{C_1\ldots C_k},j}
=
\This{m+j}{C_1\ldots C_{k-n}} \quad \textit{with }n\leq k
\\
\from{\This{n+j}{Cs}}{\This{m}{C_1\ldots C_k},j}
=
\This{m+j+n-k}{C_1\ldots C_{k-n}{Cs}} \quad \textit{with }n> k
\\
\from{
\libc{\Q@interface@? Tz}{Mz}{K}
}{T,j-1}
=
\libc{\Q@interface@? \from{Tz}{T,j}}{\from{Mz}{T,j}}{\from{K}{T,j}}
\\\hline
(DL_1\ldots DL_n; \_, t\eq{LV})[t]
=p\op{min}{\from{LV}{\This{n},0}}\\\hline
p[T]=p\op{min}{
\lib{\from{Tz}{T,0}}{\from{Mz}{T,0}}{}
}
\quad\text{where }p(T)=\lib{Tz}{Mz}{K?}
\\\hline
\end{array}$


sdgsd

\noindent\begin{defye}%
%\defy{(\_; \_, t\eq{}LV, \_)(t)}{\mathit{LV}}%
\defy{(\DLs\Q{;} \DVs)\op{push}{id\eq{L}}}{
id\eq{L},\DLs\Q{;} \DVs}%
\defy{(; \_, C\eq{}L, \_)(\This{0}{C}{Cs})}{\mathit{L(Cs)}}%
%\defy{(id\eq{}L, p)(\This{0}{Cs})}{L(\mathit{Cs})}%
\defy{p\op{push}{\_\eq{L}}(\This{0}{Cs})}{L(\mathit{Cs})}%
%p' id=L,A;B = id=L,p
%\defy{(id\eq{}L, p)(\This{n+1}{Cs})}{p(\This{n}{Cs})}%shorter version, not wrong but confusing
\defy{p\op{push}{\_}(\This{n+1}{Cs})}{p(\This{n}{Cs})}%
\defy{LV({\emptyset})}{LV}%
\defy{\lib{\_}{\_, \Q@private@?\,C\eq{L_0}, \_}{\_}(\Cs{C}{\s{C}})}{L_0(Cs) }%\text{ where } L = \lib{\_}{\_, C\eq{L_0}, \_}{\_}}%
	\defyc{\text{where } L = a}
\end{defye}


This notation just fetch the referred $LV$ without any modification.
To adapt the paths we define $\from{T_0}{T_1,j}$, $\from{L}{T,j}$ and $p\op{minimize}{T}$ as following:
\begin{defye}%
  \defy{(DL_1\ldots DL_n; \_, t\eq{LV},\_)[t]}{\from{LV}{\This{n}}}
	\defy{p[T]}{p\op{minimize}{\from{p(T)}{T}}}%
\end{defye}
\\${}_{}$\\

%\saveSpace\saveSpace\section{Formalization}\label{sec:formal}
\saveSpace\saveSpace

Here we show a simple formalization for the language we presented so far.
We also model nested classes, but in order to avoid uninteresting complexities, we assume that
all the type names are fully qualified from the top level, so the examples shown before should be
written like \Q@This.Exp@, \Q@This.Sum@ and so on.
In a real language, a simple pre-processor may take care of this step.

In most languages, when implementing an interface, the programmer may avoid repeating abstract methods
they do not wish to implement.
In that same spirit, in our simplified formalization we consider source code containing
all the methods imported from interfaces. In a real language, a normalization process
may hide this abstraction\footnote{
In the full 42 language scoping is indeed supported by an initial de-sugaring, and a normalization phase takes care of importing methods from interfaces.
}.
We also consider a binary operator sum \Q@+@ instead of the variable arity operator \Q@use@.

Figure 1 contains the complete formalization for the 
compilation process and the type system for \name.
It starts with the syntax, then
we show the compilation process and the typing rules.




\begin{figure}
%NEW FORMALISATION below
% Syntax
% D::=TD|CD
% TE::=t:E Trait Decl Expr
% CE::=C:E Class Decl
% TD::=t:L
% CD::=C:L
% E::= L| t| E+E | E[rename T.m1->m2]|E[rename T1->T2]|E[redirect T1->T2]
% L::= {interface? implements Ts Ms}//all L are like LC in 42
% T::=C|C.T // .T is a shortcut for This.T
% M::= static? method T m(T1 x1..Tn xn) e? | CD
% e::= x| e.m(es) | T.m(es)

\begin{bnf}
\prodFull\mID{\mt\mid\mC}{class or trait name}\\
\prodFull\mDE{\mID\terminalCode{=}\mE}{Meta-declaration}\\
\prodFull\mD{\mID\terminalCode{=}\mL}{Declaration}\\
\prodFull\mE{\mL \mid \mt \mid \mE\,\terminalCode{+}\mE
\mid \ldots
%\mid \mE\terminalCode{[rename}\ \mT\terminalCode{.}\mm_1\ \terminalCode{to}\ \mm_2\terminalCode{]}
}{Code Expression}\\
%\prodNextLine{
% \mid
%\mE\terminalCode{[rename}\ \mT_1\ \terminalCode{in}\ \mT_2\terminalCode{]} \mid
%\mE\terminalCode{[redirect}\ \mT_1\ \terminalCode{to}\ \mT_2\terminalCode{]}}{Code Expression}\\
\prodFull\mL{
\oC \Opt{\terminalCode{interface}}\ \terminalCode{implements} \overline\mT\ \overline\mM\ \cC}{Code Literal}\\
\prodFull\mT{\mC \mid \mC\terminalCode{.}\mT}{Type}\\
\prodFull\mM{\Opt{\terminalCode{static}}\ \terminalCode{method}\ \mT\ \mm\oR\overline{\mT\,\mx}\cR \Opt\me \mid \mC\terminalCode{=}\mL}{Member}\\

\prodFull\me{\mx \mid \me\terminalCode{.}\mm\oR\overline\me\cR \mid \mT\terminalCode{.}\mm\oR\overline\me\cR}{Expression}\\

\prodFull{v_{{\smallDs}}}{\mT\terminalCode{.}\mm\oR\overline{v_{{\smallDs}}}\cR
\text{  where }\mm \text{ is abstract in }\overline\mD(\mT)
}{value}\\
\prodFull{\ctx_{\smallDs}}{[]\mid
\ctx\terminalCode{.}\mm\oR\overline\me\cR
\mid
v_{{\smallDs}}\terminalCode{.}\mm\oR\overline{v_{{\smallDs}}},\ctx,\overline\me\cR
\mid
\mT\terminalCode{.}\mm\oR\overline{v_{{\smallDs}}},\ctx,\overline\me\cR
}{evaluation ctx}\\
\prodFull{\ctx_c}{[]\mid\ctx\,\terminalCode+\mE | \mL\,\terminalCode+\ctx |\ldots}{compilation ctx}\\
\prodFull{\ctx}{[]\mid\ctx\,\terminalCode+\mE | \mE\,\terminalCode+\ctx |\ldots}{ctx}\\
\end{bnf}\\
\\
\newcommand{\pushLeft}{\!\!\!\!\!\!\!\!\!\!\!\!}
$\begin{array}{l}

%       D.E -->^+_CDs L  CDs|-CD1:OK .. CDs|-CDn:OK       CDs=CD1..CDn
% (top)---------------------------------------------------------------    D.E not of form L
%      CD1..CDn CDs' D Ds -> CDs CDs' D[with E=L] Ds

\pushLeft\inferrule[(top)]{
  \mE_0 \xrightarrow[\smallDs]{} \mE_1
  \\
  \forall \mD\in\overline\mD,
  \overline\mD\vdash\mD:\text{OK}
  }{ 
    \overline\mD \ \overline{\mD'}\ \mID\terminalCode{=}\mE_0 \ \overline{\mDE}
    \rightarrow 
    \overline\mD\ \overline{\mD'}\ \mID\terminalCode{=}\mE_1\ \overline{\mDE}
  } %{\overline\mD=\mD_1..\mD_n }
\quad\quad
%
%     ------------------------
%      t -->_CDs CDs(t)

 \inferrule[(look-up)]{
    \ 
  }{ 
    \mt \xrightarrow[\smallDs]{}\ \overline\mD(\mt)
  }
\quad

\inferrule[(ctx-c)]{
    \mE_0 \xrightarrow[\smallDs]{}\ \mE_1
  }{ 
     {\ctx}_c[\mE_0] \xrightarrow[\smallDs]{}\ {\ctx}_c[\mE_1]
  }
\quad
%
%      --------------------------      L = L1+L2
%      L1+L2  -->_CDs L

\inferrule[(sum)]{
    \
  }{ 
     \mL_1\,\terminalCode{+}\mL_2 \xrightarrow[\smallDs]{}\ \mL
  }\mL = \mL_1+\mL_2
\\[5ex]
%  C;CDs,C=L |- L[This=C] :OK
% ----------------------------------------- coherent(L)
%  CDs|-C=L : OK

\pushLeft\inferrule[(CD-OK)]{
    \mC;\overline\mD,\mC\terminalCode{=}\mL_1\vdash \mL_1\ :\text{OK}
  }{ 
     \overline\mD \vdash \mC\terminalCode{=}\mL_0\ :\text{OK}
  }
\begin{array}{l}
\mL_1=\mL_0[\terminalCode{This}=\mC]\\
\text{coherent}(\mC,\mL_1)
\end{array}
\quad\quad 

%    This;CDs,This=L |- L :OK
%----------------------------------------
%    CDs|-t=L : OK

\inferrule[(TD-OK)]{
    \terminalCode{This};\overline\mD,\terminalCode{This=}\mL\vdash \mL\ :\text{OK}
  }{ 
     \overline\mD \vdash \mt\terminalCode{=}\mL\ :\text{OK}
  }
\\[5ex] 

%  forall i in 1..k T;CDs|-Mi:Ok
%--------------------------------------------------  L={interface? implements T1..Tn M1..Mk} 
%  T;CDs|-L:Ok                                         forall i in 1..n 	CDs(Ti).interface?=interface
%                                                             forall i in 1..n and m in 	dom(CDs(Ti)), m in dom(L)

\pushLeft\inferrule[(L-OK)]{
    \forall\mM\in\overline\mM,
  \mT;\overline\mD\vdash\mM:\text{OK}
  }{ 
     \mT;\overline\mD \vdash  \oC \Opt{\terminalCode{interface}}\ \terminalCode{implements} \overline\mT \ \overline\mM \cC \ :\text{OK}\\
  } 
%\begin{array}{l} 
%  \mL=\oC \Opt{\terminalCode{interface}}\ \terminalCode{implements} \overline\mT \ \overline\mM \cC \\
%  \forall \mT\in\overline\mT \text{and } m \in \dom(\mD(\mT)), \mm \in \dom(\mL)
%   \end{array}
\quad
\inferrule[(Nested-OK)]{
    \mT\terminalCode{.}\mC;\overline\mD\vdash \mL\ :\text{OK}
  }{ 
     \mT;\overline\mD \vdash \mC\terminalCode{=}\mL\ :\text{OK}
  }

\\[5ex] 

%  if e?=e then CDs; G|-e:T                         
%----------------------------------------------------------   forall T in CDs(C).Ts, if m in dom(CDs(Ti)) then
%   T;CDs|-static? T0 m(T1 x1..Tn xn) e?              static? T0 m(T1 x1..Tn xn) in CDs(Ti)
%                                                                        if static?=static then G=x1:T1 .. xn:Tn
%                                                                        else G=this:T,x1:T1 .. xn:Tn

\pushLeft\inferrule[(Method-OK)]{
    \text{if}\ \Opt\me=\me\ \text{then}\ \overline\mD; \mG\vdash\me:\mT
  }{ 
     \mT;\overline\mD \vdash \Opt{\terminalCode{static}}\ \terminalCode{method}\ \mT_0\ \mm\oR\mT_1\,\mx_1\ldots\mT_n\,\mx_n\cR \Opt\me
  } \begin{array}{l} 
  \text{if}\ \Opt{\terminalCode{static}}=\terminalCode{static}\\
  \quad \text{then}\ \mG=\mx_1:\mT_1\ .. \ \mx_n:\mT_n\ \\
  \quad\text{else}\ \mG=\terminalCode{this}:\mT,\mx_1:\mT_1\ ..\ \mx_n:\mT_n
  \\
%removed, now is well formedness
%  \forall \mT \in \text{implementsOf}(\overline\mD(\mC)),\ \text{if}\ \mm \in \dom(\overline\mD(\mT))\ \text{then} \\
%  \quad\Opt{\terminalCode{static}}\ \terminalCode{method}\ \mT_0\ \mm\oR\overline{\mT\,\mx}\cR \in \overline\mD(\mT) \\
   \end{array}
\\[5ex] 



\pushLeft\inferrule[(subsumption)]{
%  \begin{array}{l}
    \overline\mD; \mG\vdash\me: \mT_1  \\\\
    \overline\mD\vdash\mT_1 \leq \mT_2
%  \end{array}
  }{ 
     \overline\mD; \mG\vdash\me: \mT_2
  }
\quad \inferrule[(static-method-call)]{
    \overline\mD;\mG\vdash\me_1:\mT_1\ \ldots \ \overline\mD;\mG\vdash\me_n:\mT_n
  }{ 
    \overline\mD;\mG\vdash \mT_0.\mm\oR\me_1\ \ldots \ \me_n\cR:\mT
  } \terminalCode{static method}\ \mT\ \mm\oR\mT_1\,\mx_1\ldots\mT_n\,\mx_n\cR \text{\_} \in \overline\mD\oR\mT_0 \cR
\\[5ex] 

%    CDs;G|-e0:T0 .. CDs;G|-en:Tn
%---------------------------------------------    static T m(T1 x1..Tn xn) _ in CDs(T0)
%  CDs;G|-e0.m(e1..en):T

\pushLeft\inferrule[(x)]{
    \
  }{ 
    \overline\mD; \mG\vdash\mx: \mG\oR\mx\cR
  }
\quad
\inferrule[(method-call)]{
    \overline\mD;\mG\vdash\me_0:\mT_0\ \ldots \ \overline\mD;\mG\vdash\me_n:\mT_n
  }{ 
    \overline\mD;\mG\vdash \me_0.\mm\oR\me_1\ \ldots \ \me_n\cR:\mT
  } \terminalCode{method}\ \mT\ \mm\oR\mT_1\,\mx_1\ldots\mT_n\,\mx_n\cR \text{\_} \in \overline\mD\oR\mT_0 \cR

\\[5ex] 
\pushLeft\inferrule[(ctxv)]{\me_0\xrightarrow[\smallDs]{}\me_1}{
 \ctx_{\smallDs}[\me_0]\xrightarrow[\smallDs]{} \ctx_{\smallDs}[\me_1]
 }

\quad
\inferrule[(s-m)]{{}_{}}{
 \mT\terminalCode{.}\mm\oR\overline\vds\cR\xrightarrow[\smallDs]{}
 \text{meth}(\overline\mD(\mT,\mm),\overline\vds)
}
\quad
\inferrule[(m)]{{}_{}}{
 \vds\terminalCode{.}\mm\oR\overline\vds\cR\xrightarrow[\smallDs]{}
 \text{meth}(\overline\mD(\mT,\mm),\vds\,\overline\vds)
}\vds=\mT\terminalCode{.}\mm'\oR\_\cR\\
\end{array}
$\\
\caption{Formalization}
\end{figure}

\subsection{Syntax}

%In the following section, we present a simplified grammar of \name. 
We use $\mt$ and $\mC$ to represent lower case trait and upper case class identifiers respectively.
To declare a trait \mTD\ or a class \mCD, we can use either a code literal \mL\ or a trait
expression $\mE$. Note how in $\mE$\ you can refer to a trait by name.
In full 42, we support various operators including the ones presented before,
 but here we only show the single sum operator \Q@+@.
This operation is a generalization to the case of nested classes of the simplest and most elegant
trait composition operator~\cite{ducasse2006traits}.
Code literals \mL\ can be marked as interfaces. We use `?' to represent optional terms.
Note that the interface keyword is inside the curly brackets,
so an upper case name associated with an interface literal is a class-interface, while a lowercase one is a trait-interface.
Then we have a set of implemented interfaces and a set of member
declarations, they can be methods or nested classes.
The members of a code literal are a set, thus their order is immaterial.
If a code literal implements zero interfaces, the concrete syntax omits the \Q@implements@ keyword.

Methods \mMD~can be instance methods or \Q@static@ methods. 
A static method in \name is similar to a \Q@static@ method in Java but can be abstract.
This is very useful in the context of code composition.
To denote a method as abstract, instead of an optional keyword we just omit the implementation \me.

Finally, expressions $\me$ are just variables, method calls or static method calls.
The ugliness of having two different kinds of method calls is an artefact of our simplifications.
In the full 42 language, type names are a kind of expression whose type helps to model metaclasses.
Our concept of abstract state implies we have no \Q@new@ expressions, and
the values are just calls to abstract static methods.
Thus values are parametric on the shape of the specific programs $\overline\mD$.

We then show the evaluation context, the compilation context and full
context.

\subsection{Well-formedness}

The whole program, that is $\overline\mDE$, is well formed if
all the traits and classes at top level have unique names. The special class name
\Q@This@ is not one of those,
and the subtype relations are consistent:
this means that the implementation of interfaces is not circular,
and that $\forall\ \_\terminalCode{=}\ctx[\mL]\in\overline\mDE, \mathit{consistentSubtype}(\overline\mDE,\terminalCode{This=}\mL;\mL)$

\noindent That is, every literal declares
all the methods that are declared in its super interfaces, and declare them with the same exact type.\footnote{The full 42 language allows for covariant return types, like in Java.}


\noindent\textbf{Define }$\mathit{consistentSubtype}(\overline\mDE;\mL)$\\
$\begin{array}{l}
\!\!\!\bullet\ \mathit{consistentSubtype}(
  \overline\mDE,
  \oC
  \Opt{\terminalCode{interface}}
  \terminalCode{implements}\overline\mT\ 
  \overline\mM
  \cC
  )\quad\text{where}\\

\quad\quad
\forall \mT\in\overline\mT,\overline\mDE(\mT)=\oC\terminalCode{interface}\,\_\cC
 \text{,\footnotemark}
\\
\quad\quad \forall\ \_\terminalCode{=}\mL\in  \overline\mM, 
\mathit{consistentSubtype}(\overline\mDE;\mL) 

\text{ and }
\\
\quad\quad 
\forall \mm, \mT\in\overline\mT,
\text{if}\,\overline\mDE(\mT,\mm)=\terminalCode{method}\ \mT_0 \mm\oR
\overline{\mT\,\mx}
%\mT'_1\,\mx'_1\ldots\mT'_k\,\mx'_k
\cR
\,\text{then}\,
\terminalCode{method}\ \mT_0 \mm\oR
%\mT_1\,\mx_1\ldots\mT_n\,\mx_n
\overline{\mT\,\mx}
\cR\Opt\me
\in\overline\mM

%\mT_0=\mT'_0, \overline{\mT\,\mx}=\overline{\mT\,\mx}'
%\mT_0\ldots\mT_n=\mT'_0\ldots\mT'_k


\\
\end{array}$
\footnotetext{That is, in this simplified version 
in order to implements an interface nested in a different top level name, such interface can not be generated using a trait expression. This limitation is lifted in the full language.}
${}_{}$\\*
${}_{}$\\*
\noindent A code literal \mL\ is well formed if:
\begin{itemize}
\item all method parameters have unique names and the special parameter name \Q@this@ is not declared
 in the parameter list,
\item all methods in a code literal have unique names,
\item all nested classes have unique names, and no nested class is called \Q@This@,
\item all used variables are in scope, and
\item all methods in an interface are abstract, and there are no interface static methods.
\end{itemize}

\saveSpace
\subsection{Compilation process}
\saveSpace
Usually the compilation process is not modelled, but here it is the \textbf{most interesting part}.
Here we explain in the detail the flattening process and how and when compilation errors may arise.
It is composed by rules \Rulename{top},\ \Rulename{look-up},\ \Rulename{ctx-c} and \Rulename{sum}.
If we were to formally model more composition operators, each operator would have its own  rule.

Rule \Rulename{top}
compiles the leftmost top level (trait or class) declaration that needs to be compiled.
In order to do so,
it identifies the subset of the program $\overline\mD$ that can already be typed (second premise).
Then the expression is executed under the control of such compiled program (first premise).
According to rule \Rulename{look-up}, all the traits inside the expression need to
be compiled, that is $\forall\mt. \mE=\ctx[\mt], \mt\in\dom(\overline\mD)$.
If a large enough $\overline\mD$ cannot be typed, this would cause a compilation error
at this stage.
Rule \Rulename{look-up}
replaces a trait name $\mt$ with the corresponding literal $\mL$.
Thanks to the fact that $\overline\mD$ is all well typed, we know that $\mL$ is well typed too.
Rule \Rulename{ctx-c}
uses the compilation context to decide what step to apply next.
It enforces a deterministic left-right call by value\footnote{
In the flattening process, values are code literals $\mL$.} reduction;
thus the leftmost invalid sum that is performed will be the one providing the compilation error.

Keeping in mind the order of members in a literal is immaterial, rule \Rulename{sum}
applies the operator:

\noindent\textbf{Define }$\mL_1+\mL_2, \ \overline{\mM}+\overline{\mM},\ \mM+\mM$\\
$\begin{array}{l}
\!\!\!\bullet\ \mL_1+\mL_2 =\mL_3\quad\text{where}\\
\quad\quad \mL_1= \oC \Opt{\terminalCode{interface}}\ \terminalCode{implements} \overline\mT_1\ \overline\mM_1\ \overline\mM_0\cC\\
\quad\quad \mL_2= \oC \Opt{\terminalCode{interface}}\ \terminalCode{implements} \overline\mT_2\ \overline\mM_2\ \overline\mM_0\cC\\
\quad\quad \mL_3= \oC \Opt{\terminalCode{interface}}\ \terminalCode{implements} \overline\mT_1,\overline\mT_2\ \overline\mM_1,\overline\mM_2\ (\overline\mM_0+\overline\mM_0')\cC\\
\quad\quad \dom(\overline\mM_1)
%\pitchfork
\,\text{disjoint}\,
 \dom(\overline\mM_2) \text{ and } \dom(\overline\mM_0)\ =\ \dom(\overline\mM_0')\\

\!\!\!\bullet\ \mM_1..\mM_n+\mM'_1+\mM'_n\ = \ \mM_1+\mM'_1..\mM_n+\mM'_n\\

\!\!\!\bullet\ \mC\terminalCode{=}\mL_1+\mC\terminalCode{=}\mL_2\ = \ \mC\terminalCode{=}\mL_3\quad if \mL_1+\mL_2\\

\!\!\!\bullet\ \mM_1+\mM_2=\mM_2+\mM_1\\

\!\!\!\bullet\ \Opt{\terminalCode{static}}\ \terminalCode{method}\ \mT_0\ \mm\oR\overline{\mT\,\mx}\cR \ + \ \Opt{\terminalCode{static}}\ \terminalCode{method}\ \mT_0\ \mm\oR\overline{\mT\,\mx}\cR \Opt\me = \Opt{\terminalCode{static}}\ \terminalCode{method}\ \mT_0\ \mm\oR\overline{\mT\,\mx}\cR \Opt\me\\
\end{array}$

Sum composes the content of the arguments
by taking the union of their members and the union of their \Q@implements@.
Members with the same name are recursively composed.
There are three cases where the composition is impossible.
\begin{itemize}
\item MethodClash: two methods with the same name are composed,
but either their headers have different types or they are both implemented.
\item ClassClash: a class is composed with an interface.%
\footnote{
The full language offers some relaxation here, so that for example an empty class can be seen as an empty interface during composition.
}
\item ImplementsClash:
The result code would not be well formed.
For example
\begin{lstlisting}
t1={
  A:{interface method Void a()}
  B:{}
   }
t2={
  A:{interface}
  B:{implements A}
  }
\end{lstlisting}
Naively, \Q@t1+t2@ should result in a class \Q@B@ implementing \Q@A@ with method \Q@a()@,
but \Q@B@ would not offer such method \Q@a()@.%
\footnote{While in \name it could be possible to try to patch class \Q@B@, for example adding a
abstract method \Q@a()@, we chose to give an error in this case, since in the full 42 language
such patch would 
be able to turn private nested classes
into abstract (private) ones.}

ImplementsClash can happen only when composing nested interfaces. Note that while the first two kind of errors are obtained directly by the definition of 
$\mL_1+\mL_2$, ImplementsClash is obtained since injecting the resulting 
$\mL$ in the program would make it ill-formed by 
$\mathit{consistentSubtype}(\overline\mDE,\mL)$.
\end{itemize}

\subsection{Typing}
Typing is composed by rules \Rulename{cd-ok}, \Rulename{td-ok},
\Rulename{l-ok},
\Rulename{nested-ok} and \Rulename{method-ok},
followed by expression typing rules
\Rulename{subsumption}, \Rulename{method-call}, \Rulename{x} and \Rulename{static-method-call}.

Rules \Rulename{cd-ok} and \Rulename{td-ok}
are interesting: a top level class is typed by replacing all occurrences of the name `\Q@This@' with the class name $C$,
and is required to be coherent.
On the other side, a top level trait is typed by temporary adding to the typed program a mapping for
\Q@This@.

\noindent\textbf{Define }$\text{coherent}(\mT,\mL)$\\
$\begin{array}{l}
\!\!\!\bullet\ \text{coherent}(\mT,
\oC \Opt{\terminalCode{interface}}\ \terminalCode{implements} \overline\mT\ \overline\mM\cC
)\quad\text{where}\\

\quad\quad \forall \mC\terminalCode{=}\mL'\in\overline\mM \text{coherent}(\mT\terminalCode{.}\mC,\mL')\\
\quad\quad \text{either }
\Opt{\terminalCode{interface}}=\terminalCode{interface}\\
\quad\quad\quad \text{or } 
\forall\ 
\terminalCode{method}\ \mT'\ \mm\oR\overline{\mT\,\mx}\cR \in\overline\mM,\ 
\text{state}(\text{factory}(\mT,\overline\mM),\terminalCode{method}\ \mT'\ \mm\oR\overline{\mT\,\mx}\cR)
\end{array}$

\noindent A Library is \emph{coherent} if 
all the nested classes are coherent,
and either the Library is an interface,
there are no static methods, or all the static methods
are a valid \emph{state} method of the candidate \emph{factory}.
Note, by asking for
$\terminalCode{method}\ \mT'\ \mm\oR\overline{\mT\,\mx}\cR \in\overline\mM$
we select only the abstract methods.

\noindent\textbf{Define }$\text{factory}(\mT,\overline\mM)$\\
$\begin{array}{l}

\!\!\!\bullet\ \text{factory}(\mT,\mM_1\ldots\mM_n)=\mM_i=\terminalCode{static method}\ \mT\, \mm
\oR
\_
\cR

\quad\text{where}\\
\quad\quad \forall j\neq i.\ \mM_j=
\text{not of the form}\ \terminalCode{static method}\ \_\, \_
\oR
\_
\cR
\end{array}$

\noindent The factory is the only static abstract  method, and
its return type is the nominal type of our class.

\noindent\textbf{Define }$\text{state}(\mM,\mM')$\\
$\begin{array}{l}


\!\!\!\bullet\ \text{state}(
\terminalCode{static}\ \terminalCode{method}\ \mT\ \mm\oR\mT_1\,\mx_1\ldots\mT_n\,\mx_n\cR,
\terminalCode{method}\ \mT_i\ \mx_i\oR\cR
)\\

%\!\!\!\bullet\ \text{state}(
%\terminalCode{static}\ \terminalCode{method}\ \mT\ \mm\oR\mT_1\,\mx_1\ldots\mT_n\,\mx_n\cR,
%\terminalCode{method}\ \terminalCode{Void} \mx_i\oR\mT_i\,\terminalCode{that}\cR
%)\\

\!\!\!\bullet\ \text{state}(
\terminalCode{static}\ \terminalCode{method}\ \mT\ \mm\oR\mT_1\,\mx_1\ldots\mT_n\,\mx_n\cR,
\terminalCode{method}\ \mT\ \terminalCode{with}\mx_i\oR\mT_i\,\terminalCode{that}\cR
)\\

\end{array}$

\noindent A non static method is part of the \emph{abstract state} if 
it is a valid getter or wither. This simple formalism without imperative features do not offer setters.


Rule \Rulename{Nested-OK} helps to accumulate the type of \Q@this@ so that rule \Rulename{Method-OK}
can use it.
Rule \Rulename{L-OK} is so simple since all the checks
related to correctly implementing interfaces are delegated to the well formedness criteria.
The other rules are straightforward and standard.

\subsection{Formal properties}
In addition to conventional soundness of the expression reduction,
\name ensures soundness of the compilation process itself.
A similar property was called meta-level-soundness in~\cite{servetto2014meta}; here we can obtain the same result in
a much simpler setting.
We denote $\mathit{wrong}(\overline\mD,\mE)$ to be the count of $\mL$ such that
$\mE=\ctx[\mL]\ \text{and not}\ \overline\mD\vdash\mL:\text{OK}$.

\begin{Theorem}[Compilation Soundness]

if $\mE_0 \xrightarrow[\smallDs]{} \mE_1$
then $\mathit{wrong}(\overline\mD,\mE_0)\geq\mathit{wrong}(\overline\mD,\mE_1)$.
\end{Theorem}
This can be proved by cases on the applied reduction rule:
\begin{itemize}
\item
\Rulename{look-up} preserve the number of wrong literals,
since $t \in \overline\mD$ and $\overline\mD$ are well typed by \Rulename{top} second preconditions.
\item \Rulename{sum} either preserves or reduces the number of
wrong literals:
the core of the proof is to show the sum of two well typed literals produces a well typed one.
A code literal is well typed (\Rulename{l-ok}) if all the method bodies are correct.
This holds since those same method bodies
are well typed in a strictly poorer environment with respect to the one used to type the result.
This is because every member in the result
is structurally a subtype of
the corresponding member in the argument.
Note that by well formedness, if \Rulename{sum}
is applied, the result still respects 
$\mathit{consistentSubtype}$.
\end{itemize}
\noindent 
Compilation Soundness has two important corollaries:
\begin{itemize}
\item If a class is declared without literals,
and the flattening is successful then \mC\ is well-typed: there is no need of further checking.
\item On the other side, if a class is declared by using literals $\mL_1\ldots\mL_n$, and after successful flattening $\mC = \mL$ can not be type-checked,
then the issue was originally present in one of $\mL_1\ldots\mL_n$.
This also means that as an optimization strategy
 we may remember what method bodies come from traits and what method bodies come from code literals, in order to type-check only the latter.

If the result can not be type-checked, does not means
that is intrinsically ill-typed: it may happen that a 
referred type is declared \emph{after} the current class. 
As we see in the next section, we leverage on this 
to allow recursive types.
 \end{itemize}






\subsection{Advantages of our compilation process}


Our typing discipline is very simple from a formal perspective,  
and is what distinguishes our approach from a simple minded code composition macro~\cite{bawden1999quasiquotation}
or a rigid module composition~\cite{ancona2002calculus}. 
It is built on two core ideas:

\paragraph{1: traits are \textbf{well-typed} before being reused.}
 For example in

\saveSpace\begin{lstlisting}
t={method int m() 2 
   method int n() this.m()+1}
\end{lstlisting}\saveSpace

\noindent \Q@t@ is well typed since \Q@m()@ is declared inside of \Q@t@, while

\saveSpace\begin{lstlisting}
t1={method int n() this.m()+1} 
\end{lstlisting}\saveSpace
\noindent would be ill-typed.

\paragraph{2: code literals are \textbf{not required to be well-typed} before flattening.}${}_{}$\\*
A code literal $\mL$ in a declaration $\mD$
is must be well formed and respect
$\mathit{consistentSubtype}$. However 
it is not type-checked before flattening,
and only the result is expected to be well-typed.
This example using the trait \Q@t@ is correct:

\saveSpace\begin{lstlisting}
C= Use t, {method int k() this.n()+this.m()}
\end{lstlisting}\saveSpace

\noindent The code literal
\Q@{method int k() this.n()+this.m()}@, 
is not well typed,
since \Q@n@, \Q@m@ are not locally defined. However in 
\name the result of the flattening is well-typed.
This is not the case in many similar works in literature~\cite{deep,Bettini2015282,Bergel2007} where the
literals have to be \emph{self contained}. In this case we would have been forced to
declare abstract methods \Q@n@ and \Q@m@, even if \Q@t@ already 
provides such methods.

This relaxation allows multiple declarations to be flattened one at the time, without typing them individually, and then to type them all together.
In this way, we support recursive types%
\footnote{
OO languages leverage on recursive types most of the times:
for example \Q@String@ may offer a \Q@Int size()@
method, and \Q@Int@ may offer a \Q@String toString()@ method.
This means that typing classes 
\Q@String@ and \Q@Int@ in isolation one at a time is not possible.}
between multiple $\mC$\Q@=@$\mE$ \textbf{without
the need of predicting the resulting shape}%
\footnote{This is fundamental in the full language where arbitrary code can be run at compile time, making impossible to predict the resulting shape.}.

As seen in \Rulename{top}, our compilation process
proceeds in a top-down fashion, flattening one declaration at a time,
and declarations need type-checking
only where their type is first needed,
that is, when they are required to type a trait $\mt$ used in an expression $\mE$.
That is, in \name typing and flattening are interleaved. We assume our compilation process to stop as soon as 
an error arises. 
For example
\saveSpace\begin{lstlisting}
ta:{method int ma() 2}
tc:{method int mc(A a,B b) b.mb(a)}
A: Use ta
B:{method int mb(A a) a.ma()+1}
C: Use tc, {method int hello() 1}
\end{lstlisting}\saveSpace
In this scenario, since we go top down, we first need to generate \Q@A@.
To generate \Q@A@, we need to use \Q@ta@ (but we do not need
\Q@tc@, in rule \Rulename{top} $\overline\mD=$\Q@ta@ and $\overline\mD'=$\Q@tc@).
At this moment, \Q@tc@ cannot be compiled/checked alone:
information about \Q@A@ and \Q@B@ is needed.
In order to modularly ensure well-typedness,
we require only \Q@ta@ to be well typed at this stage. If \Q@ta@ was not well-typed
a type error could be generated at this stage.

Now, we need to generate \Q@C@, and we need to ensure well-typedness of \Q@tc@.
Now \Q@A@ is already well typed (since generated by \use\ \Q@ta@, with no \mL),
and \Q@B@ can be typed. Finally \Q@tc@ can be typed and used.
If \Rulename{sum} could not be performed (for example it \Q@tc@ had a method \Q@hello@ too)
a composition error could be generated at this stage.
On the opposite side, if \Q@B@ and \Q@C@ were swapped, as in
\saveSpace\begin{lstlisting}
C: Use tc, {method int hello() 1}
B:{method int mb(A a) a.ma()+1}
\end{lstlisting}\saveSpace
\noindent
we would be unable to type \Q@tc@, since we need to know the type of \Q@A@ and \Q@B@.
A type error would be generated, on the lines of ``flattening of \Q@C@
requires \Q@tc@, \Q@tc@ requires \Q@B@ that is defined later''.

%In this example, a more expressive compilation/precompilation process 
%could compute a dependency graph and, if possible, reorganize the list,

\paragraph{The cost: what expressive power we lose}${}_{}$\\*
For simplicity, that declarations are always provided in the right
order, if such order exists.
An example of a ``morally correct'' program where no right order exists is the following:
\saveSpace\begin{lstlisting}
t={ int mt(A a) a.ma()}
A=Use t {int ma() 1}
\end{lstlisting}\saveSpace

We just wrote an involved program where the correctness of trait \Q@t@ depends of 
\Q@A@, that is in turn generated using trait \Q@t@.
We believe any typing allowing those programs would be fragile with respect to code evolution,
and could make human understanding the code-reuse process much harder/involved.

%Rewriting our example in Java may help to show how involved it is.
%\saveSpace\begin{lstlisting}
%class T{ int mt(A a){return a.ma();}
%class A extends T {int ma() {return 1;}}
%\end{lstlisting}\saveSpace

In sharp contrast with
many other approaches (TR, PT, DeepFJig and in some sense even Java, C\# and Scala)
we chose to not support this kind of involved programs.
In a system without inference for method types,
if the result of composition operators depends only on the
structural shape of their input (as for \use)
it is indeed possible to optimistically compute the resulting structural shape of the classes
and use it to type involved examples like the former.

TR, PT, DeepFJig, Java, C\# and Scala
accept a great complexity in order to \textbf{predict the structural shape} of the resulting code before doing the actual code reuse/adaptation.
Those approaches logically divide the program in groups of mutually dependent classes, where each group may depend on a number of other groups.
This form a direct acyclic graph of groups.
To type a group, all depended groups are typed, then
the signature/structural shape of all
the classes of the group is extracted.
Finally, with the information of the depended groups and the one extracted
from the current group, it is possible to type-check the implementation of each class in the group.
%Following this model, it is reasonable to assume that flattening happens group by group, before extracting the class signatures.



%\paragraph{In \name, typechecking before compiling would be redundant}${}_{}$\\*
%In the world of strongly typed languages we are tempted to
%first check that all can go well, and then perform the flattening.
%This would however be overcomplicated without any observable difference:
%Indeed, in the \Q@A,B,C@ example above there is no difference
%between
%\begin{itemize}
%\item  (1) First check \Q@B@ and produce \Q@B@ code (that also contains \Q@B@ structural shape),
%  (2) then use \Q@B@ shape to check \Q@C@ and produce \Q@C@ code;\ 
%or a more involved
%\item  (1) First check \Q@B@ and discover just \Q@B@ structural shape as result of the checking,
%  (2) then use \Q@B@ shape to check \Q@C@.
%  (3) Finally produce both \Q@B@ and \Q@C@ code.
%\end{itemize}
%
%
%This may seems a dangerous relaxation at first, but also Java has the same behaviour:
%\saveSpace\begin{lstlisting}[language=Java]
%  class A{ int ma() {return 2;}  int n(){return this.ma()+1;} }
%  class B extends A{ int mb(){return this.ma();} }
%\end{lstlisting}\saveSpace
%\noindent in \Q@B@ we can call \lstinline{this.ma()} even if in the curly braces there is no declaration for \Q@ma()@.
%
%



\noindent 
In the world of strongly typed languages we are tempted to
first check that all can go well, and then perform the flattening. However, that we can reuse code only by naming traits; but our point of relaxation is {\bf only} the code literal: in no way an error can ``move around'' and be duplicated during the compilation process.
Our approach allows for safe libraries of traits and classes to be typechecked once and then deployed and reused by multiple clients: no type error will emerge from library code.
%However, we do not force the programmer to write self-contained code where all the abstract method definition are explicitly declared.


\saveSpace
\subsection{Expression reduction}
\saveSpace
Reduction rules are incredibly simple and standard.
A great advantage of our compilation model is that expressions are executed on
a simple fully flattened program, 
where all the composition operators have been removed.
From the point of view of expression reduction, \name is a simple language of 
interfaces and final classes, where nested classes gives structure to the code but have no special semantics.
The reduction of expressions is composed by rules
\Rulename{ctx-v},\Rulename{s-m} and \Rulename{m}.
The only interesting point is the auxiliary function meth:


\noindent\textbf{Define }$\text{meth}(\mM,\overline\vds)$

$\begin{array}{l}

\!\!\!\bullet\text{meth}(\terminalCode{static method}\ \mT\ \mm\oR\mT_1\, \mx_1\ldots\mT_n\,\mx_n\cR\me,\vds_1\ldots\vds_n)=\me[\mx_1=\vds_1\ldots\me_n=\vds_n]
\\

\!\!\!\bullet\text{meth}(\terminalCode{method}\ \mT\ \mm\oR\mT_1\, \mx_1\ldots\mT_n\,\mx_n\cR\me,\vds_0\ldots\vds_n)=\me[\terminalCode{this}=\vds_0,\mx_1=\vds_1\ldots\me_n=\vds_n]
\\

\!\!\!\bullet\text{meth}(\terminalCode{method}\ \mT_i\ \mx_i\oR\cR,\mT\terminalCode{.}\mm\oR\vds_1\ldots\vds_n\cR)=\vds_i\\
\quad \quad\text{where}\ \ \overline\mD(\mT,\mm) =
\terminalCode{static method}
\ \mT\,\mm\oR\mT_1\,\mx_1\ldots\mT_n\,\mx_n\cR
\\

\!\!\!\bullet\text{meth}(\terminalCode{method}\ \mT\ \terminalCode{with}\mx_i\oR\mT_i\,\terminalCode{that}\cR,\mT\terminalCode{.}\mm\oR\vds_1\ldots\vds_i\ldots\vds_n\cR,
\vds
)=
\mT\terminalCode{.}\mm\oR\vds_1\ldots\vds\ldots\vds_n\cR
\\
\quad \quad\text{where}\ \ \overline\mD(\mT,\mm) =
\terminalCode{static method}
\ \mT\,\mm\oR\mT_1\,\mx_1\ldots\mT_n\,\mx_n\cR
\end{array}$

\noindent 
Here we take care of reading bodies and preparing for
execution.
The first case is about static methods,
the second is about instance methods.
The third and fourth cases are more interesting, since they take care of
the abstract state:
the third case takes care of getters and the fourth takes care of withers.
In our formalization we are not modelling state mutation, so there is 
no case for setters.

For space reasons, we omit the proof of conventional soundness for the
reduction. It is unsurprising, since the flattened calculus is like a
simplified version of Featherweight Java~\cite{igarashi2001featherweight}.
%\section{Conclusions, extensions and practical applications}

In this paper we explained a simple model to 
radically decouple inheritance/code reuse and subtyping.
One important point is that our decoupling does not
makes the language more complex:
% since
%interfaces (subtyping without subclassing)
%exists in both Java and C\#.
we \textbf{replace the concept} of abstract classes with
the concept of traits, while keeping the concepts of
interfaces and final classes.
Concrete non final classes are simply not needed in our model.

The model presented here is easy to extend.
More composition operators can be added in addition to \use.
In particular variants of the sophisticated operators of DJ are
included in the full 42 language.
 Indeed we can add any operator respecting following criteria:

\begin{itemize}
\item The operator does not need to be total, but if it fails it needs to provide an error that will be reported to the programmer.
\item When the operator takes in input only traits (they are going to be well typed code), if a result is produced,
 such result is also well typed.
\item When the operator takes in input also code literals, if a non-well typed result is produced,
the type error must be traced back to code in one of those not-yet typed code literals.
 \end{itemize}
 

 
 Our simplified model represents the conceptual core of  42: a novel full blown programming language,
using the ideas presented in this paper to obtain reliable and understandable metaprogramming.
Formalization (in progress) for full 42 can be found at
\url{http://}\footnote{Omitted for anonymous review}. 
%\verb@urlOmittedForDoubleBlindReview@.
%\verb@github.com/ElvisResearchGroup/L42/tree/master/Main/formal@.
42 extends our model allowing
flattening to execute arbitrary computations.
In such model we do not need an explicit notion of traits: they are encoded as methods returning a code literal.
42 also has features less related to code composition, like
  a strong type system supporting aliasing mutability and circularity control,
   checked exceptions, and errors (unchecked exceptions) with strong-exception-safety.

\begin{comment}
42 do not have a finite set of composition operators; they can be
added using the built in support for native method calls. They can
be dynamically checked to verify that they are well behaved
according to our predicate, or they can be trusted to achieve
efficiency.
\end{comment}



\appendix

\bibliography{main}


\end{document}
