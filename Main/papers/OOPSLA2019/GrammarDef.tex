\usepackage{xspace}
\usepackage{xparse}
\usepackage{tabularx}
\usepackage{letltxmacro}
\usepackage{mathtools}
\usepackage{mathpartir}

\makeatletter
\newcommand{\@eq}{\,\,\Q{=}\!\,\,}
% A nice questionmark
\newcommand{\@code}[1]{\ensuremath{#1}}
\newcommand{\q}{\ensuremath{?}}
\newcommand{\Empty}{\ensuremath{\epsilon}\xspace}
\newcommand{\EmptyZ}{\ensuremath{\emptyset}}
\newcommand{\rp}[1]{\Q{(}\!#1\Q{)}}
\newcommand{\cp}[1]{\,\Q{\{}\!\,#1\,\Q{\}}\!}
\newcommand{\eq}[1]{\,\Q{=}#1}
\newcommand{\cdc}{\ensuremath{, \ldots, }}
\newcommand{\ddd}{\ensuremath{\ldots}}
%%%% Grammar forms

	\newcommand{\@meth}[5]{\@code{#1#2\ #3\rp{#4}#5}}

% Methods: any, static, instance, abstract, abstract instance, abstract static
\newcommand{\meth}  [4]{\@meth{\static\q\,}{#1}{#2}{#3}{\ #4}}
\newcommand{\smeth} [4]{\@meth{\static\,}  {#1}{#2}{#3}{\ #4}}
\newcommand{\imeth} [4]{\@meth{}              {#1}{#2}{#3}{\ #4}}
\newcommand{\ameth} [3]{\@meth{\static\q\,}{#1}{#2}{#3}{}}
\newcommand{\asmeth}[3]{\@meth{\static\,}  {#1}{#2}{#3}{}}
\newcommand{\aimeth}[3]{\@meth{}              {#1}{#2}{#3}{}}

% Nested class
\newcommand{\ncdd}[1]{\private\q\,#1}
\newcommand{\ncd}[2]{\cd{\private\q\,#1}{#2}}
\newcommand{\cd}[2]{\@code{#1\@eq#2}}
% code literals
\newcommand{\cl}[3]{\@code{\cp{#1\Q{;}\ #2\Q{;}\ #3}}}
	
	%\@code{\ensuremath{\Q@\{@#1\Q{;}\ #2\Q{;}\ #3\,\Q@\}@\!}}}

% what was once a program... but is now a \Sigma
\newcommand{\Exists}[2]{\ensuremath{\exists #1\text{ such that } #2}}
\newcommand{\ForAll}[2]{\ensuremath{\forall #1,\ #2}}

%New
\renewcommand{\This}{\Q{This}\xspace}
\newcommand{\private}{\Q{private}\xspace}
\renewcommand{\abstract}{\Q{abstract}\xspace}
\renewcommand{\interface}{\Q{interface}\xspace}
\newcommand{\static}{\Q{static}\xspace}
\newcommand{\new}{\Q{new}\xspace}
\renewcommand{\this}{\Q{this}\xspace}

\renewcommand{\T}[1][\p]{\ensuremath{\mathbf{T}^{#1}}}

\NewDocumentCommand{\Tn}{ O{\T} O{n}}{\ensuremath{{#1}^{#2}}}
%\newcommand{\State}[2]{#1;#2}
\NewDocumentCommand{\DS}{O{T}O{\Dz}}{\ensuremath{#2; #1}}
\NewDocumentCommand{\pS}{O{\Ds}O{\DFz}}{\ensuremath{#2; #1}}
\newcommand{\pSS}[1]{\pS[#1]}

%\renewcommand{\T}[1]{#1\@nextpath}
%	\newcommand{\@nextpath}{\@ifnextchar\bgroup{\@gobblenextpath}{\xspace}}
%	\newcommand{\@gobblenextpath}[1]{\Q!.!\@code{#1}\@ifnextchar\bgroup{\@gobblenextpath}{\xspace}}
\renewcommand{\.}{\Q!.!\xspace}
%-------------------

% The sum operator
\newcommand{\summ}[2]{\@code{#1\ \Q{<+}\ #2}}

% Redirect operator, and redirect map
\newcommand{\red}[1]{\@code{#1}\,\Q!<!\!\,\@nextCsP{\,\Q!>!}}
\newcommand{\redm}{\@nextCsP{}}
	\newcommand{\@nextCsP}[1]{\@ifnextchar\bgroup{\@gobblenextCs{}{#1}}{#1}}
	\newcommand{\@gobblenextCs}[3]{#1\@code{#3}\@ifnextchar\bgroup{\@eq \@gobblenextP{#2}}{#2\xspace}}
	\newcommand{\@gobblenextP}[2]{\@code{#2}\@ifnextchar\bgroup{\@gobblenextCs{\Q{,}\,}{#1}}{#1\xspace}}

% new expression, field access expressions, method call expression\
\newcommand{\en}[2]{\@code{\new\ #1\rp{#2}}}
\newcommand{\ef}[2]{\@code{#1\Q{.}#2}}
\newcommand{\ec}[3]{\ef{#1}{#2\rp{#3}}}

% Because mathmode puts too much space
\newcommand{\Cs}{\ensuremath{\mathit{Cs}}}
\newcommand{\Csz}{\ensuremath{\mathit{Csz}}}
\newcommand{\DF}{\ensuremath{\mathit{DF}}}
\newcommand{\DFs}{\ensuremath{\mathit{DFs}}}
\newcommand{\DFz}{\ensuremath{\mathit{DFz}}}
\newcommand{\Ds}{\ensuremath{\mathit{Ds}}}
\renewcommand{\es}{\ensuremath{\mathit{es}}}
\newcommand{\id}{\ensuremath{\mathit{id}}}
\newcommand{\LF}{\ensuremath{\mathit{LF}}}
\newcommand{\MF}{\ensuremath{\mathit{MF}}}
\newcommand{\MFz}{\ensuremath{\mathit{MFz}}}
\newcommand{\Mz}{\ensuremath{\mathit{Mz}}}
\newcommand{\Tx}{\ensuremath{\mathit{Tx}}}
\newcommand{\Txs}{\ensuremath{\mathit{Txs}}}
\newcommand{\Tz}{\ensuremath{\mathit{Tz}}}

\newcommand{\Ok}{\ensuremath{\textbf{Ok}}}
% Meta-notation
\newcommand{\mop}[3]{{\ensuremath{{#1}_{.\textbf{#2}({#3})}}}}
\newcommand{\opn}[1]{{\ensuremath{\textbf{#1}}}}
\newcommand{\fop}[2]{{\ensuremath{\footnotesize{\textbf{#1}}({#2})}}}

% operators

\renewcommand{\members}[1]{\fop{members}{#1}}
\newcommand{\idd}[1]{\fop{id}{#1}}
\renewcommand{\dom}[1]{\fop{dom}{#1}}
\newcommand{\ran}[1]{\fop{range}{#1}}
\newcommand{\mdom}[1]{\fop{mdom}{#1}}
\newcommand{\mini}[2]{#1\op{min}{#2}}
\newcommand{\minid}[3]{#1\op{min}{#2, #3}}
\newcommand{\from}[3]{#1\op{from}{#2,#3}}

% Just to typset some words nicecly
\newcommand{\where}{,\quad\text{where }}
\newcommand{\wherec}{,\quad\text{where:}}
\newcommand{\wwhere}{\text{where }}
\newcommand{\otherwise}{,\quad\text{otherwise}}
\newcommand{\imp}{\ensuremath{\Rightarrow}}
\newcommand{\reda}[3]{\ensuremath{{#1[#2]}_{#3}}}

% Environemnts

\newenvironment{defs}{\setlength{\tabcolsep}{0pt}\noindent\tabularx{\textwidth}{l>{\hfill}X}\midrule}{\endtabularx\\% This is stupid (if I put a \midrule before the \endtabularx it tries to duplicate it and I get errors...)
	\begin{tabularx}{\textwidth}{X}\midrule\end{tabularx}}
% a := b
\newcommand{\defi}[3]{\ensuremath{#1\coloneqq#2}&\llap{\text{#3}}\\}
% a if b 
\newcommand{\defip}[3]{\ensuremath{#1\text{ if }#2}&\llap{\text{#3}}\\}
% a % (for things that are always true)
\newcommand{\defit}[2]{\ensuremath{#1}&\llap{\text{#2}}\\}
\newcommand{\defii}[1]{\ensuremath{
		#1\coloneqq\text{\ldots\ simply propagate to the other rules\ \ldots}}\\}
%   b % indentended, for continuiation of another defi form
\newcommand{\defic}[2]{\ensuremath{\quad\quad#1}&\llap{\text{#2}}\\}
\newcommand{\defin}[1]{#1\\}

\newenvironment{grammar}{%
	\setlength{\tabcolsep}{0pt}%
	\noindent\tabularx{\textwidth}{rl>{\hfill}X}\midrule}{\endtabularx\\% This is stupid (if I put a \midrule before the \endtabularx it tries to duplicate it and I get errors...)
	\begin{tabularx}{\textwidth}{X}\midrule\end{tabularx}%
}
\newcommand{\produ}[3]{\ensuremath{#1}&\ensuremath{{}\Coloneqq#2}&\llap{\text{#3}}\\}
\newcommand{\produh}[4]{\ensuremath{#1{#2}}&\ensuremath{#1{{}\Coloneqq#3}}&\llap{#1{\text{#4}}}\\}
\newcommand{\prodc}[2]{&\ensuremath{\quad\quad#1}&\llap{\text{#2}}\\}
% --------------
\newcommand{\defis}{\midrule}


\newenvironment{irules}{\noindent\[\begin{array}{c}\midrule}{\unskip\\\midrule\end{array}\]}
\NewDocumentCommand{\irule}{O{} m O{} m}{%
	\ifstrempty{#1}{%
		\inferrule{#2}{#4}
	}{%
		\inferrule[(#1)]{#2}{#4}
	}%
	\ifstrempty{#3}{}{\!\!\ensuremath{\begin{array}{l}#3\end{array}}}%
	\qquad}
\newcommand{\iruleSep}{\unskip\\[5ex]}

%\newcommand{\irule}[3]{%
%		\everymath={\displaystyle}%
%		\ensuremath{
%			\Rulename{#1}\ \frac{\hspace{-0.5em}\begin{array}{c}#2\end{array}\hspace{-0.5em}}{#3}%
%	}
%}
	
% this version formats things almost exactly the same as \inferrule, whereas mine (above) looks better!
%\newcommand{\irule}[3]{%
%	\everymath={\displaystyle}%
%	\ensuremath{
%		\begin{array}{l}%
%		\begin{array}{l}\Rulename{#1}\end{array}\\%
%		\frac{\hspace{-1em}
%				\begin{array}{c}#2\end{array}%
%			\hspace{-1em}}{#3}%
%		\end{array}%
%	}\hfill
%}
% ----------------------------------------------------------------------------
%\newcommand{\qquad}{\quad\quad}

\newcommand{\Rulename}[1]{{\textsc{(#1)}}}
\renewcommand{\ctx}[1]{\ensuremath{\mathcal{E}_#1}\!}
\renewcommand{\G}{\ensuremath{\Gamma}\xspace}
\newcommand{\vvdots}{\ensuremath{\smash{\lower 0.5ex \vdots}}\xspace} %just to make them centered and not take up too much space!
\makeatother