\section{Invariants Over Immutable State}
\label{s:immutable}
In this section we consider validation over fields of \Q@imm@ types.
%\footnote{
%In a real language, for conciseness one could make the \Q@imm@ modifier the default, allowing it to be omitted and our \Q@Person@ example class would only use 3 type modifiers; however we explicitly use it here for clarity.
%}
In the next section we detail our technique for \Q@capsule@ fields.

In the following code \Q@Person@ has a single immutable (non final) field \Q@name@:
\begin{lstlisting}
class Person {
  read method Bool invariant() { return !name.isEmpty(); }
  private String name;//the default modifier imm is applied here
  read method String name() { return this.name; }
  mut method String name(String name) { this.name = name; }
  Person(String name) { this.name = name; }
}
\end{lstlisting}
\Q@Person@ only has immutable fields and its constructor only uses \Q@this@ to initialise them.
%; we say such a class is \emph{simple}.
%\Q@Person@, only has immutable fields and the constructor 
%uses the parameters to directly initialize (all) the fields.
% We say such a class is \emph{simple}.%
%\footnote{
%We consider only standard contractors for simplicity of exposition.
%More complex constructors could be supported, provided that \Q@this@ is only used to access fields, we do discuss them for simplicity.}
% The difference with respect to UML DataTypes 
%immutable types (like UML DataTypes)
%UML datatypes are aclass property. immutable types are often an instance one (so no final fields) 
Note that the \Q@name@ field is not final, thus \Q@Person@ objects can change state during their lifetime. This means that the ROGs of all \Q@Person@s fields are immutable, but \Q@Person@s themselves may be mutable.
%Of course UML DataTypes
%immutable types
%No, a type is not a class
% are just a special case of simple classes.
We can easily enforce \Q@Person@'s invariant by generating checks on the result of \Q@this.invariant()@: immediately after each field update, and at the end of the constructor.%
%\footnote{Since the constructor only initialises fields; as with the \Q@invariant@ method itself, we allow field uses since \Q@this@ is not directly reached.}
%would require the initial/default value of \Q@this@ to be valid.}

% If a simple class provides a \Q@invariant@ method, then validation will be enforced.
% For \Q@Person@, intuitively, the code would behave as follow:

%\Comment{if we made this public, all users who update the field need to call validate}%
%There are many interpretations for your comment
%why you deleted my code comments?
\begin{lstlisting}
class Person { .. // Same as before
  mut method String name(String name) {
    this.name = name; // check after field update
    if (!this.invariant()) { throw new Error(...); }}
  Person(String name) {
    this.name = name; // check at end of constructor
    if (!this.invariant()) { throw new Error(...); }}
}
\end{lstlisting}
%... $\MComment{validation error}$ 
%
% Many programmers attempted to write similar code in mainstream languages like Java to ensure  that some property always holds. Indeed, at first look, this code seems to correctly enforce validation. Sadly, without relying on TM and OC, the former code would be broken: just making the fields private and checking the \Q@invariant@ method at the \textbf{end of the constructor} and at the \textbf{end of mutator methods} is not enough to enforce validation.
% The trick is that our intuition relies not on statically verified properties, or on the semantics of the language, but on the expectations about `correct' behaviour of \Q@String@. We need to enforce Validation without assuming the behaviour of other objects.

Such checks will be generated/injected, and not directly written by the programmer. If we were to relax (as in Rust), or even eliminate (as in Java), the support for TMs or OCs, the enforcement of our invariant protocol for the \Q@Person@ class would become harder, or even impossible. 

\subheading{Unrestricted use of non determinism} Allowing the \Q@invariant@ method to (indirectly) perform a non deterministic operation, such as by creating new capability objects, could break our guarantee that (manually) calling it always returns \Q!true!.
%\Q@invariant@ to be non-deterministic.}%
% 
For example consider this simple and contrived (mis)use of person:
\begin{lstlisting}[morekeywords={assert}]
class EvilString extends String {
  @Override read method Bool isEmpty() {
    // Create a new capability object out of thin air
    return new Random().bool(); }
} ..
method mut Person createPersons(String name) {
  // we can not be sure that name is not an EvilString
  mut Person schrodinger = new Person(name); // exception here?
  assert schrodinger.invariant(); // will this fail?
  ..}
\end{lstlisting}
%//  mut Person schrodinger2 = new Person(name); // what about here?

Despite the code for \Q@Person.invariant@ intuitively looking correct and deterministic, the above call to it is not. Obviously this breaks any reasoning and would make our protocol unsound. 
In particular, note how in the presence of dynamic class loading, we have no way of knowing what the type of \Q@name@ could be. Since our system allows non determinism only through capability objects, and 
restricts their creation, the above example would be prevented.
 %Languages like Java, Rust and Pony, which do not require the user of object-capabilities to perform non-deterministic operations, suffer from . 
%???
%Even if we disallow subtyping the same problem could still occur if we had a strange implementing of \Q@String@, or \Q@Person.validate()@ itself.

\subheading{Allowing Internal Mutation Through Back Doors}
Suppose we relax our rules by allowing interior mutability
as in Rust and Javari, where sneaky mutation
of the ROG of an `immutable' object is allowed.
Those back doors are usually motivated by performance reasons, however in~\cite{GordonEtAl12} they
briefly discuss how a few trusted language primitives can be used to perform caching and other needed optimisations,
without the need for back doors.

Our example shows that such back doors can be used to break determinism of \Q@invariant@ methods, by allowing the invariant to store and read information about previous calls. In the following example we use \Q@MagicCounter@ as a back door to remotely break the invariant of \Q@person@ without any interaction with the \Q@person@ object itself:
\begin{lstlisting}
class MagicCounter {
  method Int increment(){
    //Magic mutation through an imm receiver, equivalent to i++
}}
class NastyS extends String {..
  MagicCounter evil = new MagicCounter(0);
  @Override read method Bool isEmpty() {
    return this.evil.increment() != 2; }
} ..
NastyS name = new NastyS("bob"); //TMs believe name's ROG is imm
Person person = new Person(name); // person is valid, counter=1
name.increment(); // counter == 2, person is now broken
person.invariant(); // returns false!, counter == 3
person.invariant(); // returns true, counter == 4
\end{lstlisting}

%mine: yes, too strong: For validation we need the language to guarantee true deep immutability.
%your: just points outside: It would require some powerful static or dynamic analysis to keep track of every case the ROG of \Q@Person@ could be indirectly mutated, and insert validity checks appropriately, however ensuring deep mutability trivialises this for simple classes.
% Allowing such back-doors could also be used to break the determinism of the \Q@invariant@ method:
% information can be stored about previous calls.
% In this example you can see how the invariant get to be \Q@false@ and then \Q@true@ again.
%In our simple example, \Q@Person@ objects can be mutated using the setter, and exposed using the getter.
%We may consider the getter to be safe since in modern languages we expect strings to be immutable objects.
%\footnote{While we can update the field \Q@name@ to point to another string, we cannot mutate the string object itself.
%To obtain  \Q@"Hello"@ from \Q@"hello"@ we need to create a whole new string object that looks like the old one except for the first character. This would be different in older languages like C, where strings are just mutable arrays of characters.}
%
%Again, the assumption that they are immutable depends on the correctness of the code inside \Q@String@: if there was a bug in the \Q@String@ class, or any \Q@String@ subclass, then executing 
%\Q@println(bob.name())@ may change \Q@bob@ by quietly changing a part of its ROG.
%Again, checking
%what methods mutate states cannot be responsibility of the \Q@Person@ programmer.
%For Validation we need a language supporting aliasing and mutability control.
%\begin{comment}
%\item Sample Bug 1:
%Suppose there was a bug in \Q@String.isEmpty()@, causing the method to non-deterministically return \Q@true@ or \Q@false@.
%What would it mean for Validation?
%Would a \Q@Person@ be at the same time 
%valid and invalid?
%
%Only deterministic methods can be used for validation.
%Ensuring this cannot be responsibility of the \Q@Person@ programmer, since it may depend on third party code, as shown in this example.
%However, statically checking if a method is deterministic is hard/impossible in most imperative object-oriented languages.
%
%While we may not expect the presence of bugs in the standard library class \Q@String@, the same behaviour can be achieved with subtyping:
%\saveSpace
%\begin{lstlisting}
%class EvilStr extends String{
%  method Bool isEmpty(){
%    return new Random().bool();
%  }}
%...
%String name=...$\Comment{can this be an EvilStr?}$
%Person bob=new Person(name);
%\end{lstlisting}
%\saveSpace
%As you can see, it is hard to make sound claims about Validation.
%
%\item Sample Bug 2:
%In our simple example, \Q@Person@ objects can be mutated using the setter, and exposed using the getter.
%We may consider the getter to be safe since in modern languages we expect strings to be immutable objects.
%\footnote{While we can update the field \Q@name@ to point to another string, we cannot mutate the string object itself.
%To obtain  \Q@"Hello"@ from \Q@"hello"@ we need to create a whole new string object that looks like the old one except for the first character. This would be different in older languages like C, where strings are just mutable arrays of characters.}
%
%Again, the assumption that they are immutable depends on the correctness of the code inside \Q@String@: if there was a bug in the \Q@String@ class, or any \Q@String@ subclass, then executing 
%\Q@println(bob.name())@ may change \Q@bob@ by quietly changing a part of its ROG.
%
%Again, checking
%what methods mutate states cannot be responsibility of the \Q@Person@ programmer.
%For Validation we need a language supporting aliasing and mutability control.
%\end{comment}

\subheading{Strong Exception Safety}
The ability to catch and recover from invariant failures is extremely useful as it allows programs to take corrective action.
Since we represent invariant failures by throwing unchecked exceptions, programs can recover from them with a conventional \Q@try@--\Q@catch@.
%\REVComm{
%	Due to the guarantees of strong exception safety, the only trace that an invalid object existed is the exception thrown; any object that has been mutated/created during the \Q@try@ block is now unreachable (as happens in alias burying~\cite{boyland2001alias}).
	Due to the guarantees of strong exception safety, any object that has been mutated during a \Q@try@ block is now unreachable (as happens in alias burying~\cite{boyland2001alias}). In addition, since unchecked exceptions are immutable, they can not contain a \Q@read@ reference to any object (such as the \Q@this@ reference seen by \Q@invariant@ methods). These two properties ensure that an object whose invariant fails will be unreachable after the invariant failure has been captured. %in a \Q@catch@.	
%}{3}{\label{SES2} [see footnote \ref{SES1}]}.
If instead we were to not enforce strong exception safety, an invalid object could be made reachable:
%\saveSpace
\begin{lstlisting}[morekeywords={assert}, escapechar=\%]
mut Person bob = new Person("bob");
// Catch and ignore invariant failure:
try { bob.name(""); } catch (Error t) { } // ill typed in L42
assert bob.invariant(); // bob is invalid!
\end{lstlisting}

As you can see, recovering from an invariant failure in this way is unsound and would break our protocol.
%Strong exception safety is a useful property to enforce, but for the specific purpose of validation this could be relaxed by restricting only \Q@try-catch@ blocks that could capture unchecked exceptions.
%Since calls to \Q@invariant@ may only throw unchecked-exceptions, violating strong exception safety within a \Q@try-catch@ that cannot catch unchecked-exceptions would not break our protocol.


%LATER: This means that we could relax our Strong Exception Safety to hold only on unchecked exceptions (by restricting only \Q@try-catch@ blocks that capture unchecked exceptions.



% One of the advantages of checking Validation at run time, is that
% we can allow the program can take corrective actions if a property is violated.
% This may be implemented with a conventional \Q@try-catch@ if violations are represented by throwing errors.
% However, there is an issue with exceptions modelling invalid objects: they can be captured when the invalid object is still in scope. For example:


%As you can see, if we can capture validation failures as normal exceptions %(very desirable feature) then we may end up using invalid objects.
%Moreover,
% as shown before with the example of transferring cargo between two boats,
%after an invariant has been violated, even objects with valid invariant may be in an unexpected state.

% This situation is a general issue about reasoning on the state after recovering from exceptions.
% In particular, as shown in the example this prevent sound validation.

% Note how this produces a different semantics with respect to static verification, where violations
% never happened. However this will not necessarily lead to a broken semantics:
%Thanks to Strong exception safety we have a system where either the application terminate
%when an invalid object is detected, or where any witness of the execution causing the invalid object is erased from history
%those objects and all the witnesses will be garbage collected
% (as happens in alias burying~\cite{boyland2001alias}).
%In our example, this means that to continue execution after a detected bug, 
%we would require to garbage collect the overloaded boat, their cargo and probably most of the commercial port too.








%\subsubheading*{Solving Issue 3: Constructors}
%\saveSpace
%Exposing \Q@this@ during construction is a generally recognized problem~\cite{gil2009we}.
%A simple solution is to require all constructors to 
%simply take a parameter for each field and to just initialize the fields.
%An advantage of such approach is syntactic brevity: constructors are implicitly defined
%by the set of fields and thus there is no need to define them manually.
%\textbf{Expressive initialization operations can still be performed, by following the factory pattern.}
%\saveSpace


%\subsubheading*{Recap}
%By utilising type modifiers (\Q@imm@, \Q@mut@ and \Q@read@), object capabilities and immutable exceptions we obtain sound runtime verification for immutable classes/UML data types.
